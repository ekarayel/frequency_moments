\documentclass[11pt,a4paper]{article}
\usepackage[T1]{fontenc}
\usepackage{isabelle,isabellesym}
\usepackage{amsmath}
\usepackage{amsthm}
\newcommand{\size}[1]{\lvert#1\rvert}
\newcommand{\var}{\mathrm{Var}}
\newcommand{\expectation}{\mathrm{E}}

% further packages required for unusual symbols (see also
% isabellesym.sty), use only when needed

%\usepackage{amssymb}
  %for \<leadsto>, \<box>, \<diamond>, \<sqsupset>, \<mho>, \<Join>,
  %\<lhd>, \<lesssim>, \<greatersim>, \<lessapprox>, \<greaterapprox>,
  %\<triangleq>, \<yen>, \<lozenge>

%\usepackage{eurosym}
  %for \<euro>

%\usepackage[only,bigsqcap,fatsemi,interleave,sslash]{stmaryrd}
  %for \<Sqinter>, \<Zsemi>

%\usepackage{eufrak}
  %for \<AA> ... \<ZZ>, \<aa> ... \<zz> (also included in amssymb)

%\usepackage{textcomp}
  %for \<onequarter>, \<onehalf>, \<threequarters>, \<degree>, \<cent>,
  %\<currency>

% this should be the last package used
\usepackage{pdfsetup}

% urls in roman style, theory text in math-similar italics
\urlstyle{rm}
\isabellestyle{it}

% for uniform font size
%\renewcommand{\isastyle}{\isastyleminor}


\begin{document}

\title{Formalization of Randomized Approximation Algorithms for Frequency Moments}
\author{Emin Karayel}
\maketitle

\tableofcontents

% sane default for proof documents
\parindent 0pt\parskip 0.5ex

% generated text of all theories
%
\begin{isabellebody}%
\setisabellecontext{Encoding}%
%
\isadelimdocument
%
\endisadelimdocument
%
\isatagdocument
%
\isamarkupsection{Encoding%
}
\isamarkuptrue%
%
\endisatagdocument
{\isafolddocument}%
%
\isadelimdocument
%
\endisadelimdocument
%
\isadelimtheory
%
\endisadelimtheory
%
\isatagtheory
\isacommand{theory}\isamarkupfalse%
\ Encoding\isanewline
\ \ \isakeyword{imports}\ Main\ {\isachardoublequoteopen}HOL{\isacharminus}{\kern0pt}Library{\isachardot}{\kern0pt}Sublist{\isachardoublequoteclose}\ {\isachardoublequoteopen}HOL{\isacharminus}{\kern0pt}Library{\isachardot}{\kern0pt}Extended{\isacharunderscore}{\kern0pt}Real{\isachardoublequoteclose}\ {\isachardoublequoteopen}HOL{\isacharminus}{\kern0pt}Library{\isachardot}{\kern0pt}FuncSet{\isachardoublequoteclose}\ \isanewline
\ \ HOL{\isachardot}{\kern0pt}Transcendental\isanewline
\isakeyword{begin}%
\endisatagtheory
{\isafoldtheory}%
%
\isadelimtheory
%
\endisadelimtheory
%
\begin{isamarkuptext}%
This section contains a flexible library for encoding high level data structures into bit
strings. The library defines encoding functions for primitive types, as well as combinators
to build encodings for more complex types. It is used to measure the size of the data structures.%
\end{isamarkuptext}\isamarkuptrue%
\isacommand{fun}\isamarkupfalse%
\ is{\isacharunderscore}{\kern0pt}prefix\ \isakeyword{where}\ \isanewline
\ \ {\isachardoublequoteopen}is{\isacharunderscore}{\kern0pt}prefix\ {\isacharparenleft}{\kern0pt}Some\ x{\isacharparenright}{\kern0pt}\ {\isacharparenleft}{\kern0pt}Some\ y{\isacharparenright}{\kern0pt}\ {\isacharequal}{\kern0pt}\ prefix\ x\ y{\isachardoublequoteclose}\ {\isacharbar}{\kern0pt}\isanewline
\ \ {\isachardoublequoteopen}is{\isacharunderscore}{\kern0pt}prefix\ {\isacharunderscore}{\kern0pt}\ {\isacharunderscore}{\kern0pt}\ {\isacharequal}{\kern0pt}\ False{\isachardoublequoteclose}\isanewline
\isanewline
\isacommand{type{\isacharunderscore}{\kern0pt}synonym}\isamarkupfalse%
\ {\isacharprime}{\kern0pt}a\ encoding\ {\isacharequal}{\kern0pt}\ {\isachardoublequoteopen}{\isacharprime}{\kern0pt}a\ {\isasymrightharpoonup}\ bool\ list{\isachardoublequoteclose}\isanewline
\isanewline
\isacommand{definition}\isamarkupfalse%
\ is{\isacharunderscore}{\kern0pt}encoding\ {\isacharcolon}{\kern0pt}{\isacharcolon}{\kern0pt}\ {\isachardoublequoteopen}{\isacharprime}{\kern0pt}a\ encoding\ {\isasymRightarrow}\ bool{\isachardoublequoteclose}\isanewline
\ \ \isakeyword{where}\ {\isachardoublequoteopen}is{\isacharunderscore}{\kern0pt}encoding\ f\ {\isacharequal}{\kern0pt}\ {\isacharparenleft}{\kern0pt}{\isasymforall}x\ y{\isachardot}{\kern0pt}\ is{\isacharunderscore}{\kern0pt}prefix\ {\isacharparenleft}{\kern0pt}f\ x{\isacharparenright}{\kern0pt}\ {\isacharparenleft}{\kern0pt}f\ y{\isacharparenright}{\kern0pt}\ {\isasymlongrightarrow}\ x\ {\isacharequal}{\kern0pt}\ y{\isacharparenright}{\kern0pt}{\isachardoublequoteclose}\isanewline
\isanewline
\isacommand{lemma}\isamarkupfalse%
\ encoding{\isacharunderscore}{\kern0pt}imp{\isacharunderscore}{\kern0pt}inj{\isacharcolon}{\kern0pt}\isanewline
\ \ \isakeyword{assumes}\ {\isachardoublequoteopen}is{\isacharunderscore}{\kern0pt}encoding\ f{\isachardoublequoteclose}\isanewline
\ \ \isakeyword{shows}\ {\isachardoublequoteopen}inj{\isacharunderscore}{\kern0pt}on\ f\ {\isacharparenleft}{\kern0pt}dom\ f{\isacharparenright}{\kern0pt}{\isachardoublequoteclose}\isanewline
%
\isadelimproof
\ \ %
\endisadelimproof
%
\isatagproof
\isacommand{apply}\isamarkupfalse%
\ {\isacharparenleft}{\kern0pt}rule\ inj{\isacharunderscore}{\kern0pt}onI{\isacharparenright}{\kern0pt}\isanewline
\ \ \isacommand{using}\isamarkupfalse%
\ assms\ \isacommand{by}\isamarkupfalse%
\ {\isacharparenleft}{\kern0pt}simp\ add{\isacharcolon}{\kern0pt}is{\isacharunderscore}{\kern0pt}encoding{\isacharunderscore}{\kern0pt}def{\isacharcomma}{\kern0pt}\ force{\isacharparenright}{\kern0pt}%
\endisatagproof
{\isafoldproof}%
%
\isadelimproof
\isanewline
%
\endisadelimproof
\isanewline
\isacommand{definition}\isamarkupfalse%
\ decode\ \isakeyword{where}\ \isanewline
\ \ {\isachardoublequoteopen}decode\ f\ t\ {\isacharequal}{\kern0pt}\ {\isacharparenleft}{\kern0pt}\isanewline
\ \ \ \ if\ {\isacharparenleft}{\kern0pt}{\isasymexists}{\isacharbang}{\kern0pt}z{\isachardot}{\kern0pt}\ is{\isacharunderscore}{\kern0pt}prefix\ {\isacharparenleft}{\kern0pt}f\ z{\isacharparenright}{\kern0pt}\ {\isacharparenleft}{\kern0pt}Some\ t{\isacharparenright}{\kern0pt}{\isacharparenright}{\kern0pt}\ then\ \isanewline
\ \ \ \ \ \ {\isacharparenleft}{\kern0pt}let\ z\ {\isacharequal}{\kern0pt}\ {\isacharparenleft}{\kern0pt}THE\ z{\isachardot}{\kern0pt}\ is{\isacharunderscore}{\kern0pt}prefix\ {\isacharparenleft}{\kern0pt}f\ z{\isacharparenright}{\kern0pt}\ {\isacharparenleft}{\kern0pt}Some\ t{\isacharparenright}{\kern0pt}{\isacharparenright}{\kern0pt}\ in\ {\isacharparenleft}{\kern0pt}z{\isacharcomma}{\kern0pt}\ drop\ {\isacharparenleft}{\kern0pt}length\ {\isacharparenleft}{\kern0pt}the\ {\isacharparenleft}{\kern0pt}f\ z{\isacharparenright}{\kern0pt}{\isacharparenright}{\kern0pt}{\isacharparenright}{\kern0pt}\ t{\isacharparenright}{\kern0pt}{\isacharparenright}{\kern0pt}\isanewline
\ \ \ \ else\ \isanewline
\ \ \ \ \ \ {\isacharparenleft}{\kern0pt}undefined{\isacharcomma}{\kern0pt}\ t{\isacharparenright}{\kern0pt}\isanewline
\ \ \ \ {\isacharparenright}{\kern0pt}{\isachardoublequoteclose}\isanewline
\isanewline
\isacommand{lemma}\isamarkupfalse%
\ decode{\isacharunderscore}{\kern0pt}elim{\isacharcolon}{\kern0pt}\isanewline
\ \ \isakeyword{assumes}\ {\isachardoublequoteopen}is{\isacharunderscore}{\kern0pt}encoding\ f{\isachardoublequoteclose}\isanewline
\ \ \isakeyword{assumes}\ {\isachardoublequoteopen}f\ x\ {\isacharequal}{\kern0pt}\ Some\ r{\isachardoublequoteclose}\isanewline
\ \ \isakeyword{shows}\ {\isachardoublequoteopen}decode\ f\ {\isacharparenleft}{\kern0pt}r{\isacharat}{\kern0pt}r{\isadigit{1}}{\isacharparenright}{\kern0pt}\ {\isacharequal}{\kern0pt}\ {\isacharparenleft}{\kern0pt}x{\isacharcomma}{\kern0pt}r{\isadigit{1}}{\isacharparenright}{\kern0pt}{\isachardoublequoteclose}\isanewline
%
\isadelimproof
%
\endisadelimproof
%
\isatagproof
\isacommand{proof}\isamarkupfalse%
\ {\isacharminus}{\kern0pt}\isanewline
\ \ \isacommand{have}\isamarkupfalse%
\ a{\isacharcolon}{\kern0pt}{\isachardoublequoteopen}{\isasymAnd}y{\isachardot}{\kern0pt}\ is{\isacharunderscore}{\kern0pt}prefix\ {\isacharparenleft}{\kern0pt}f\ y{\isacharparenright}{\kern0pt}\ {\isacharparenleft}{\kern0pt}Some\ {\isacharparenleft}{\kern0pt}r{\isacharat}{\kern0pt}r{\isadigit{1}}{\isacharparenright}{\kern0pt}{\isacharparenright}{\kern0pt}\ {\isasymLongrightarrow}\ y\ {\isacharequal}{\kern0pt}\ x{\isachardoublequoteclose}\isanewline
\ \ \isacommand{proof}\isamarkupfalse%
\ {\isacharminus}{\kern0pt}\isanewline
\ \ \ \ \isacommand{fix}\isamarkupfalse%
\ y\isanewline
\ \ \ \ \isacommand{assume}\isamarkupfalse%
\ {\isachardoublequoteopen}is{\isacharunderscore}{\kern0pt}prefix\ {\isacharparenleft}{\kern0pt}f\ y{\isacharparenright}{\kern0pt}\ {\isacharparenleft}{\kern0pt}Some\ {\isacharparenleft}{\kern0pt}r{\isacharat}{\kern0pt}r{\isadigit{1}}{\isacharparenright}{\kern0pt}{\isacharparenright}{\kern0pt}{\isachardoublequoteclose}\isanewline
\ \ \ \ \isacommand{then}\isamarkupfalse%
\ \isacommand{obtain}\isamarkupfalse%
\ u\ \isakeyword{where}\ u{\isacharunderscore}{\kern0pt}{\isadigit{1}}{\isacharcolon}{\kern0pt}\ {\isachardoublequoteopen}f\ y\ {\isacharequal}{\kern0pt}\ Some\ u{\isachardoublequoteclose}\ {\isachardoublequoteopen}prefix\ u\ {\isacharparenleft}{\kern0pt}r{\isacharat}{\kern0pt}r{\isadigit{1}}{\isacharparenright}{\kern0pt}{\isachardoublequoteclose}\isanewline
\ \ \ \ \ \ \isacommand{by}\isamarkupfalse%
\ {\isacharparenleft}{\kern0pt}metis\ is{\isacharunderscore}{\kern0pt}prefix{\isachardot}{\kern0pt}elims{\isacharparenleft}{\kern0pt}{\isadigit{1}}{\isacharparenright}{\kern0pt}\ option{\isachardot}{\kern0pt}sel{\isacharparenright}{\kern0pt}\isanewline
\ \ \ \ \isacommand{hence}\isamarkupfalse%
\ {\isachardoublequoteopen}prefix\ u\ r\ {\isasymor}\ prefix\ r\ u{\isachardoublequoteclose}\ \isanewline
\ \ \ \ \ \ \isacommand{using}\isamarkupfalse%
\ prefix{\isacharunderscore}{\kern0pt}def\ prefix{\isacharunderscore}{\kern0pt}same{\isacharunderscore}{\kern0pt}cases\ \isacommand{by}\isamarkupfalse%
\ blast\isanewline
\ \ \ \ \isacommand{hence}\isamarkupfalse%
\ {\isachardoublequoteopen}is{\isacharunderscore}{\kern0pt}prefix\ {\isacharparenleft}{\kern0pt}f\ y{\isacharparenright}{\kern0pt}\ {\isacharparenleft}{\kern0pt}f\ x{\isacharparenright}{\kern0pt}\ {\isasymor}\ is{\isacharunderscore}{\kern0pt}prefix\ {\isacharparenleft}{\kern0pt}f\ x{\isacharparenright}{\kern0pt}\ {\isacharparenleft}{\kern0pt}f\ y{\isacharparenright}{\kern0pt}{\isachardoublequoteclose}\isanewline
\ \ \ \ \ \ \isacommand{using}\isamarkupfalse%
\ u{\isacharunderscore}{\kern0pt}{\isadigit{1}}\ assms{\isacharparenleft}{\kern0pt}{\isadigit{2}}{\isacharparenright}{\kern0pt}\ \isacommand{by}\isamarkupfalse%
\ simp\isanewline
\ \ \ \ \isacommand{thus}\isamarkupfalse%
\ {\isachardoublequoteopen}y\ {\isacharequal}{\kern0pt}\ x{\isachardoublequoteclose}\isanewline
\ \ \ \ \ \ \isacommand{using}\isamarkupfalse%
\ assms{\isacharparenleft}{\kern0pt}{\isadigit{1}}{\isacharparenright}{\kern0pt}\ \isacommand{apply}\isamarkupfalse%
\ {\isacharparenleft}{\kern0pt}simp\ add{\isacharcolon}{\kern0pt}is{\isacharunderscore}{\kern0pt}encoding{\isacharunderscore}{\kern0pt}def{\isacharparenright}{\kern0pt}\ \isacommand{by}\isamarkupfalse%
\ blast\isanewline
\ \ \isacommand{qed}\isamarkupfalse%
\isanewline
\ \ \isacommand{have}\isamarkupfalse%
\ b{\isacharcolon}{\kern0pt}{\isachardoublequoteopen}is{\isacharunderscore}{\kern0pt}prefix\ {\isacharparenleft}{\kern0pt}f\ x{\isacharparenright}{\kern0pt}\ {\isacharparenleft}{\kern0pt}Some\ {\isacharparenleft}{\kern0pt}r{\isacharat}{\kern0pt}r{\isadigit{1}}{\isacharparenright}{\kern0pt}{\isacharparenright}{\kern0pt}{\isachardoublequoteclose}\ \isanewline
\ \ \ \ \isacommand{using}\isamarkupfalse%
\ assms{\isacharparenleft}{\kern0pt}{\isadigit{2}}{\isacharparenright}{\kern0pt}\ \isacommand{by}\isamarkupfalse%
\ simp\isanewline
\ \ \isacommand{have}\isamarkupfalse%
\ c{\isacharcolon}{\kern0pt}{\isachardoublequoteopen}{\isasymexists}{\isacharbang}{\kern0pt}z{\isachardot}{\kern0pt}\ is{\isacharunderscore}{\kern0pt}prefix\ {\isacharparenleft}{\kern0pt}f\ z{\isacharparenright}{\kern0pt}\ {\isacharparenleft}{\kern0pt}Some\ {\isacharparenleft}{\kern0pt}r{\isacharat}{\kern0pt}r{\isadigit{1}}{\isacharparenright}{\kern0pt}{\isacharparenright}{\kern0pt}{\isachardoublequoteclose}\ \isanewline
\ \ \ \ \isacommand{using}\isamarkupfalse%
\ a\ b\ \isacommand{by}\isamarkupfalse%
\ auto\isanewline
\ \ \isacommand{have}\isamarkupfalse%
\ d{\isacharcolon}{\kern0pt}{\isachardoublequoteopen}{\isacharparenleft}{\kern0pt}THE\ z{\isachardot}{\kern0pt}\ is{\isacharunderscore}{\kern0pt}prefix\ {\isacharparenleft}{\kern0pt}f\ z{\isacharparenright}{\kern0pt}\ {\isacharparenleft}{\kern0pt}Some\ {\isacharparenleft}{\kern0pt}r{\isacharat}{\kern0pt}r{\isadigit{1}}{\isacharparenright}{\kern0pt}{\isacharparenright}{\kern0pt}{\isacharparenright}{\kern0pt}\ {\isacharequal}{\kern0pt}\ x{\isachardoublequoteclose}\ \isanewline
\ \ \ \ \isacommand{using}\isamarkupfalse%
\ a\ b\ \isacommand{by}\isamarkupfalse%
\ blast\isanewline
\ \ \isacommand{show}\isamarkupfalse%
\ {\isachardoublequoteopen}decode\ f\ {\isacharparenleft}{\kern0pt}r{\isacharat}{\kern0pt}r{\isadigit{1}}{\isacharparenright}{\kern0pt}\ {\isacharequal}{\kern0pt}\ {\isacharparenleft}{\kern0pt}x{\isacharcomma}{\kern0pt}r{\isadigit{1}}{\isacharparenright}{\kern0pt}{\isachardoublequoteclose}\isanewline
\ \ \ \ \isacommand{using}\isamarkupfalse%
\ c\ d\ assms{\isacharparenleft}{\kern0pt}{\isadigit{2}}{\isacharparenright}{\kern0pt}\ \isacommand{by}\isamarkupfalse%
\ \ {\isacharparenleft}{\kern0pt}simp\ add{\isacharcolon}{\kern0pt}\ decode{\isacharunderscore}{\kern0pt}def{\isacharparenright}{\kern0pt}\isanewline
\isacommand{qed}\isamarkupfalse%
%
\endisatagproof
{\isafoldproof}%
%
\isadelimproof
\isanewline
%
\endisadelimproof
\isanewline
\isacommand{lemma}\isamarkupfalse%
\ decode{\isacharunderscore}{\kern0pt}elim{\isacharunderscore}{\kern0pt}{\isadigit{2}}{\isacharcolon}{\kern0pt}\isanewline
\ \ \isakeyword{assumes}\ {\isachardoublequoteopen}is{\isacharunderscore}{\kern0pt}encoding\ f{\isachardoublequoteclose}\isanewline
\ \ \isakeyword{assumes}\ {\isachardoublequoteopen}x\ {\isasymin}\ dom\ f{\isachardoublequoteclose}\isanewline
\ \ \isakeyword{shows}\ {\isachardoublequoteopen}decode\ f\ {\isacharparenleft}{\kern0pt}the\ {\isacharparenleft}{\kern0pt}f\ x{\isacharparenright}{\kern0pt}{\isacharat}{\kern0pt}r{\isadigit{1}}{\isacharparenright}{\kern0pt}\ {\isacharequal}{\kern0pt}\ {\isacharparenleft}{\kern0pt}x{\isacharcomma}{\kern0pt}r{\isadigit{1}}{\isacharparenright}{\kern0pt}{\isachardoublequoteclose}\isanewline
%
\isadelimproof
\ \ %
\endisadelimproof
%
\isatagproof
\isacommand{using}\isamarkupfalse%
\ assms\ decode{\isacharunderscore}{\kern0pt}elim\ \isacommand{by}\isamarkupfalse%
\ fastforce%
\endisatagproof
{\isafoldproof}%
%
\isadelimproof
\isanewline
%
\endisadelimproof
\isanewline
\isacommand{lemma}\isamarkupfalse%
\ snd{\isacharunderscore}{\kern0pt}decode{\isacharunderscore}{\kern0pt}suffix{\isacharcolon}{\kern0pt}\isanewline
\ \ {\isachardoublequoteopen}suffix\ {\isacharparenleft}{\kern0pt}snd\ {\isacharparenleft}{\kern0pt}decode\ f\ t{\isacharparenright}{\kern0pt}{\isacharparenright}{\kern0pt}\ t{\isachardoublequoteclose}\isanewline
%
\isadelimproof
%
\endisadelimproof
%
\isatagproof
\isacommand{proof}\isamarkupfalse%
\ {\isacharparenleft}{\kern0pt}cases\ {\isachardoublequoteopen}{\isasymexists}{\isacharbang}{\kern0pt}z{\isachardot}{\kern0pt}\ is{\isacharunderscore}{\kern0pt}prefix\ {\isacharparenleft}{\kern0pt}f\ z{\isacharparenright}{\kern0pt}\ {\isacharparenleft}{\kern0pt}Some\ t{\isacharparenright}{\kern0pt}{\isachardoublequoteclose}{\isacharparenright}{\kern0pt}\isanewline
\ \ \isacommand{case}\isamarkupfalse%
\ True\isanewline
\ \ \isacommand{then}\isamarkupfalse%
\ \isacommand{obtain}\isamarkupfalse%
\ z\ \isakeyword{where}\ \ {\isachardoublequoteopen}is{\isacharunderscore}{\kern0pt}prefix\ {\isacharparenleft}{\kern0pt}f\ z{\isacharparenright}{\kern0pt}\ {\isacharparenleft}{\kern0pt}Some\ t{\isacharparenright}{\kern0pt}{\isachardoublequoteclose}\ \isacommand{by}\isamarkupfalse%
\ blast\isanewline
\ \ \isacommand{hence}\isamarkupfalse%
\ {\isachardoublequoteopen}{\isacharparenleft}{\kern0pt}THE\ z{\isachardot}{\kern0pt}\ is{\isacharunderscore}{\kern0pt}prefix\ {\isacharparenleft}{\kern0pt}f\ z{\isacharparenright}{\kern0pt}\ {\isacharparenleft}{\kern0pt}Some\ t{\isacharparenright}{\kern0pt}{\isacharparenright}{\kern0pt}\ {\isacharequal}{\kern0pt}\ z{\isachardoublequoteclose}\ \isacommand{using}\isamarkupfalse%
\ True\ \isacommand{by}\isamarkupfalse%
\ blast\isanewline
\ \ \isacommand{thus}\isamarkupfalse%
\ {\isacharquery}{\kern0pt}thesis\ \isacommand{using}\isamarkupfalse%
\ True\ \isacommand{by}\isamarkupfalse%
\ {\isacharparenleft}{\kern0pt}simp\ add{\isacharcolon}{\kern0pt}\ decode{\isacharunderscore}{\kern0pt}def\ suffix{\isacharunderscore}{\kern0pt}drop{\isacharparenright}{\kern0pt}\isanewline
\isacommand{next}\isamarkupfalse%
\isanewline
\ \ \isacommand{case}\isamarkupfalse%
\ False\isanewline
\ \ \isacommand{then}\isamarkupfalse%
\ \isacommand{show}\isamarkupfalse%
\ {\isacharquery}{\kern0pt}thesis\ \isacommand{by}\isamarkupfalse%
\ {\isacharparenleft}{\kern0pt}simp\ add{\isacharcolon}{\kern0pt}decode{\isacharunderscore}{\kern0pt}def{\isacharparenright}{\kern0pt}\isanewline
\isacommand{qed}\isamarkupfalse%
%
\endisatagproof
{\isafoldproof}%
%
\isadelimproof
\isanewline
%
\endisadelimproof
\isanewline
\isacommand{lemma}\isamarkupfalse%
\ snd{\isacharunderscore}{\kern0pt}decode{\isacharunderscore}{\kern0pt}len{\isacharcolon}{\kern0pt}\isanewline
\ \ \isakeyword{assumes}\ {\isachardoublequoteopen}decode\ f\ t\ {\isacharequal}{\kern0pt}\ {\isacharparenleft}{\kern0pt}u{\isacharcomma}{\kern0pt}v{\isacharparenright}{\kern0pt}{\isachardoublequoteclose}\isanewline
\ \ \isakeyword{shows}\ {\isachardoublequoteopen}length\ v\ {\isasymle}\ length\ t{\isachardoublequoteclose}\isanewline
%
\isadelimproof
\ \ %
\endisadelimproof
%
\isatagproof
\isacommand{using}\isamarkupfalse%
\ snd{\isacharunderscore}{\kern0pt}decode{\isacharunderscore}{\kern0pt}suffix\ assms\ suffix{\isacharunderscore}{\kern0pt}length{\isacharunderscore}{\kern0pt}le\ \isanewline
\ \ \isacommand{by}\isamarkupfalse%
\ {\isacharparenleft}{\kern0pt}metis\ snd{\isacharunderscore}{\kern0pt}conv{\isacharparenright}{\kern0pt}%
\endisatagproof
{\isafoldproof}%
%
\isadelimproof
\isanewline
%
\endisadelimproof
\isanewline
\isacommand{lemma}\isamarkupfalse%
\ encoding{\isacharunderscore}{\kern0pt}by{\isacharunderscore}{\kern0pt}witness{\isacharcolon}{\kern0pt}\isanewline
\ \ \isakeyword{assumes}\ {\isachardoublequoteopen}{\isasymAnd}x\ y{\isachardot}{\kern0pt}\ x\ {\isasymin}\ dom\ f\ {\isasymLongrightarrow}\ g\ {\isacharparenleft}{\kern0pt}the\ {\isacharparenleft}{\kern0pt}f\ x{\isacharparenright}{\kern0pt}{\isacharat}{\kern0pt}y{\isacharparenright}{\kern0pt}\ {\isacharequal}{\kern0pt}\ {\isacharparenleft}{\kern0pt}x{\isacharcomma}{\kern0pt}y{\isacharparenright}{\kern0pt}{\isachardoublequoteclose}\isanewline
\ \ \isakeyword{shows}\ {\isachardoublequoteopen}is{\isacharunderscore}{\kern0pt}encoding\ f{\isachardoublequoteclose}\isanewline
%
\isadelimproof
%
\endisadelimproof
%
\isatagproof
\isacommand{proof}\isamarkupfalse%
\ {\isacharminus}{\kern0pt}\isanewline
\ \ \isacommand{have}\isamarkupfalse%
\ {\isachardoublequoteopen}{\isasymAnd}x\ y{\isachardot}{\kern0pt}\ is{\isacharunderscore}{\kern0pt}prefix\ {\isacharparenleft}{\kern0pt}f\ x{\isacharparenright}{\kern0pt}\ {\isacharparenleft}{\kern0pt}f\ y{\isacharparenright}{\kern0pt}\ {\isasymLongrightarrow}\ x\ {\isacharequal}{\kern0pt}\ y{\isachardoublequoteclose}\isanewline
\ \ \isacommand{proof}\isamarkupfalse%
\ {\isacharminus}{\kern0pt}\isanewline
\ \ \ \ \isacommand{fix}\isamarkupfalse%
\ x\ y\isanewline
\ \ \ \ \isacommand{assume}\isamarkupfalse%
\ a{\isacharcolon}{\kern0pt}{\isachardoublequoteopen}is{\isacharunderscore}{\kern0pt}prefix\ {\isacharparenleft}{\kern0pt}f\ x{\isacharparenright}{\kern0pt}\ {\isacharparenleft}{\kern0pt}f\ y{\isacharparenright}{\kern0pt}{\isachardoublequoteclose}\isanewline
\ \ \ \ \isacommand{then}\isamarkupfalse%
\ \isacommand{obtain}\isamarkupfalse%
\ d\ \isakeyword{where}\ d{\isacharunderscore}{\kern0pt}def{\isacharcolon}{\kern0pt}{\isachardoublequoteopen}the\ {\isacharparenleft}{\kern0pt}f\ x{\isacharparenright}{\kern0pt}{\isacharat}{\kern0pt}d\ {\isacharequal}{\kern0pt}\ the\ {\isacharparenleft}{\kern0pt}f\ y{\isacharparenright}{\kern0pt}{\isachardoublequoteclose}\isanewline
\ \ \ \ \ \ \isacommand{apply}\isamarkupfalse%
\ {\isacharparenleft}{\kern0pt}case{\isacharunderscore}{\kern0pt}tac\ {\isacharbrackleft}{\kern0pt}{\isacharbang}{\kern0pt}{\isacharbrackright}{\kern0pt}\ {\isachardoublequoteopen}f\ x{\isachardoublequoteclose}{\isacharcomma}{\kern0pt}\ case{\isacharunderscore}{\kern0pt}tac\ {\isacharbrackleft}{\kern0pt}{\isacharbang}{\kern0pt}{\isacharbrackright}{\kern0pt}\ {\isachardoublequoteopen}f\ y{\isachardoublequoteclose}{\isacharcomma}{\kern0pt}\ simp{\isacharcomma}{\kern0pt}\ simp{\isacharcomma}{\kern0pt}\ simp{\isacharcomma}{\kern0pt}\ simp{\isacharparenright}{\kern0pt}\isanewline
\ \ \ \ \ \ \isacommand{by}\isamarkupfalse%
\ {\isacharparenleft}{\kern0pt}metis\ prefixE{\isacharparenright}{\kern0pt}\isanewline
\ \ \ \ \isacommand{have}\isamarkupfalse%
\ {\isachardoublequoteopen}x\ {\isasymin}\ dom\ f{\isachardoublequoteclose}\ \isacommand{using}\isamarkupfalse%
\ a\ \isacommand{apply}\isamarkupfalse%
\ {\isacharparenleft}{\kern0pt}simp\ add{\isacharcolon}{\kern0pt}dom{\isacharunderscore}{\kern0pt}def\ del{\isacharcolon}{\kern0pt}not{\isacharunderscore}{\kern0pt}None{\isacharunderscore}{\kern0pt}eq{\isacharparenright}{\kern0pt}\isanewline
\ \ \ \ \ \ \isacommand{by}\isamarkupfalse%
\ {\isacharparenleft}{\kern0pt}metis\ is{\isacharunderscore}{\kern0pt}prefix{\isachardot}{\kern0pt}simps{\isacharparenleft}{\kern0pt}{\isadigit{2}}{\isacharparenright}{\kern0pt}\ a{\isacharparenright}{\kern0pt}\isanewline
\ \ \ \ \isacommand{hence}\isamarkupfalse%
\ {\isachardoublequoteopen}g\ {\isacharparenleft}{\kern0pt}the\ {\isacharparenleft}{\kern0pt}f\ y{\isacharparenright}{\kern0pt}{\isacharparenright}{\kern0pt}\ {\isacharequal}{\kern0pt}\ {\isacharparenleft}{\kern0pt}x{\isacharcomma}{\kern0pt}d{\isacharparenright}{\kern0pt}{\isachardoublequoteclose}\ \isacommand{using}\isamarkupfalse%
\ assms\ \isacommand{by}\isamarkupfalse%
\ {\isacharparenleft}{\kern0pt}simp\ add{\isacharcolon}{\kern0pt}d{\isacharunderscore}{\kern0pt}def{\isacharbrackleft}{\kern0pt}symmetric{\isacharbrackright}{\kern0pt}{\isacharparenright}{\kern0pt}\isanewline
\ \ \ \ \isacommand{moreover}\isamarkupfalse%
\ \isacommand{have}\isamarkupfalse%
\ {\isachardoublequoteopen}y\ {\isasymin}\ dom\ f{\isachardoublequoteclose}\ \isacommand{using}\isamarkupfalse%
\ a\ \isacommand{apply}\isamarkupfalse%
\ {\isacharparenleft}{\kern0pt}simp\ add{\isacharcolon}{\kern0pt}dom{\isacharunderscore}{\kern0pt}def\ del{\isacharcolon}{\kern0pt}not{\isacharunderscore}{\kern0pt}None{\isacharunderscore}{\kern0pt}eq{\isacharparenright}{\kern0pt}\isanewline
\ \ \ \ \ \ \isacommand{by}\isamarkupfalse%
\ {\isacharparenleft}{\kern0pt}metis\ is{\isacharunderscore}{\kern0pt}prefix{\isachardot}{\kern0pt}simps{\isacharparenleft}{\kern0pt}{\isadigit{3}}{\isacharparenright}{\kern0pt}\ a{\isacharparenright}{\kern0pt}\isanewline
\ \ \ \ \isacommand{hence}\isamarkupfalse%
\ {\isachardoublequoteopen}g\ {\isacharparenleft}{\kern0pt}the\ {\isacharparenleft}{\kern0pt}f\ y{\isacharparenright}{\kern0pt}{\isacharparenright}{\kern0pt}\ {\isacharequal}{\kern0pt}\ {\isacharparenleft}{\kern0pt}y{\isacharcomma}{\kern0pt}{\isacharbrackleft}{\kern0pt}{\isacharbrackright}{\kern0pt}{\isacharparenright}{\kern0pt}{\isachardoublequoteclose}\ \isacommand{using}\isamarkupfalse%
\ assms{\isacharbrackleft}{\kern0pt}\isakeyword{where}\ y{\isacharequal}{\kern0pt}{\isachardoublequoteopen}{\isacharbrackleft}{\kern0pt}{\isacharbrackright}{\kern0pt}{\isachardoublequoteclose}{\isacharbrackright}{\kern0pt}\ \isacommand{by}\isamarkupfalse%
\ simp\isanewline
\ \ \ \ \isacommand{ultimately}\isamarkupfalse%
\ \isacommand{show}\isamarkupfalse%
\ {\isachardoublequoteopen}x\ {\isacharequal}{\kern0pt}\ y{\isachardoublequoteclose}\ \isacommand{by}\isamarkupfalse%
\ simp\isanewline
\ \ \isacommand{qed}\isamarkupfalse%
\isanewline
\ \ \isacommand{thus}\isamarkupfalse%
\ {\isacharquery}{\kern0pt}thesis\ \isacommand{by}\isamarkupfalse%
\ {\isacharparenleft}{\kern0pt}simp\ add{\isacharcolon}{\kern0pt}is{\isacharunderscore}{\kern0pt}encoding{\isacharunderscore}{\kern0pt}def{\isacharparenright}{\kern0pt}\isanewline
\isacommand{qed}\isamarkupfalse%
%
\endisatagproof
{\isafoldproof}%
%
\isadelimproof
\isanewline
%
\endisadelimproof
\isanewline
\isacommand{fun}\isamarkupfalse%
\ bit{\isacharunderscore}{\kern0pt}count\ {\isacharcolon}{\kern0pt}{\isacharcolon}{\kern0pt}\ {\isachardoublequoteopen}bool\ list\ option\ {\isasymRightarrow}\ ereal{\isachardoublequoteclose}\ \isakeyword{where}\isanewline
\ \ {\isachardoublequoteopen}bit{\isacharunderscore}{\kern0pt}count\ None\ {\isacharequal}{\kern0pt}\ {\isasyminfinity}{\isachardoublequoteclose}\ {\isacharbar}{\kern0pt}\isanewline
\ \ {\isachardoublequoteopen}bit{\isacharunderscore}{\kern0pt}count\ {\isacharparenleft}{\kern0pt}Some\ x{\isacharparenright}{\kern0pt}\ {\isacharequal}{\kern0pt}\ ereal\ {\isacharparenleft}{\kern0pt}length\ x{\isacharparenright}{\kern0pt}{\isachardoublequoteclose}\isanewline
\isanewline
\isacommand{fun}\isamarkupfalse%
\ append{\isacharunderscore}{\kern0pt}encoding\ {\isacharcolon}{\kern0pt}{\isacharcolon}{\kern0pt}\ {\isachardoublequoteopen}bool\ list\ option\ {\isasymRightarrow}\ bool\ list\ option\ {\isasymRightarrow}\ bool\ list\ option{\isachardoublequoteclose}\ {\isacharparenleft}{\kern0pt}\isakeyword{infixr}\ {\isachardoublequoteopen}{\isacharat}{\kern0pt}\isactrlsub S{\isachardoublequoteclose}\ {\isadigit{6}}{\isadigit{5}}{\isacharparenright}{\kern0pt}\isanewline
\ \ \isakeyword{where}\isanewline
\ \ \ \ {\isachardoublequoteopen}append{\isacharunderscore}{\kern0pt}encoding\ {\isacharparenleft}{\kern0pt}Some\ x{\isacharparenright}{\kern0pt}\ {\isacharparenleft}{\kern0pt}Some\ y{\isacharparenright}{\kern0pt}\ {\isacharequal}{\kern0pt}\ Some\ {\isacharparenleft}{\kern0pt}x{\isacharat}{\kern0pt}y{\isacharparenright}{\kern0pt}{\isachardoublequoteclose}\ {\isacharbar}{\kern0pt}\isanewline
\ \ \ \ {\isachardoublequoteopen}append{\isacharunderscore}{\kern0pt}encoding\ {\isacharunderscore}{\kern0pt}\ {\isacharunderscore}{\kern0pt}\ {\isacharequal}{\kern0pt}\ None{\isachardoublequoteclose}\isanewline
\isanewline
\isacommand{lemma}\isamarkupfalse%
\ bit{\isacharunderscore}{\kern0pt}count{\isacharunderscore}{\kern0pt}append{\isacharcolon}{\kern0pt}\ {\isachardoublequoteopen}bit{\isacharunderscore}{\kern0pt}count\ {\isacharparenleft}{\kern0pt}x{\isadigit{1}}{\isacharat}{\kern0pt}\isactrlsub Sx{\isadigit{2}}{\isacharparenright}{\kern0pt}\ {\isacharequal}{\kern0pt}\ bit{\isacharunderscore}{\kern0pt}count\ x{\isadigit{1}}\ {\isacharplus}{\kern0pt}\ bit{\isacharunderscore}{\kern0pt}count\ x{\isadigit{2}}{\isachardoublequoteclose}\isanewline
%
\isadelimproof
\ \ %
\endisadelimproof
%
\isatagproof
\isacommand{by}\isamarkupfalse%
\ {\isacharparenleft}{\kern0pt}cases\ x{\isadigit{1}}{\isacharcomma}{\kern0pt}\ simp{\isacharcomma}{\kern0pt}\ cases\ x{\isadigit{2}}{\isacharcomma}{\kern0pt}\ simp{\isacharcomma}{\kern0pt}\ simp{\isacharparenright}{\kern0pt}%
\endisatagproof
{\isafoldproof}%
%
\isadelimproof
%
\endisadelimproof
%
\begin{isamarkuptext}%
Encodings for lists%
\end{isamarkuptext}\isamarkuptrue%
\isacommand{fun}\isamarkupfalse%
\ list\isactrlsub S\ \isakeyword{where}\isanewline
\ \ {\isachardoublequoteopen}list\isactrlsub S\ f\ {\isacharbrackleft}{\kern0pt}{\isacharbrackright}{\kern0pt}\ {\isacharequal}{\kern0pt}\ Some\ {\isacharbrackleft}{\kern0pt}False{\isacharbrackright}{\kern0pt}{\isachardoublequoteclose}\ {\isacharbar}{\kern0pt}\isanewline
\ \ {\isachardoublequoteopen}list\isactrlsub S\ f\ {\isacharparenleft}{\kern0pt}x{\isacharhash}{\kern0pt}xs{\isacharparenright}{\kern0pt}\ {\isacharequal}{\kern0pt}\ Some\ {\isacharbrackleft}{\kern0pt}True{\isacharbrackright}{\kern0pt}{\isacharat}{\kern0pt}\isactrlsub Sf\ x{\isacharat}{\kern0pt}\isactrlsub Slist\isactrlsub S\ f\ xs{\isachardoublequoteclose}\isanewline
\isanewline
\isacommand{function}\isamarkupfalse%
\ decode{\isacharunderscore}{\kern0pt}list\ {\isacharcolon}{\kern0pt}{\isacharcolon}{\kern0pt}\ {\isachardoublequoteopen}{\isacharparenleft}{\kern0pt}{\isacharprime}{\kern0pt}a\ {\isasymRightarrow}\ bool\ list\ option{\isacharparenright}{\kern0pt}\ {\isasymRightarrow}\ bool\ list\ \isanewline
\ \ {\isasymRightarrow}\ {\isacharprime}{\kern0pt}a\ list\ {\isasymtimes}\ bool\ list{\isachardoublequoteclose}\ \isanewline
\ \ \isakeyword{where}\ \isanewline
\ \ \ \ {\isachardoublequoteopen}decode{\isacharunderscore}{\kern0pt}list\ e\ {\isacharparenleft}{\kern0pt}True{\isacharhash}{\kern0pt}x{\isadigit{0}}{\isacharparenright}{\kern0pt}\ {\isacharequal}{\kern0pt}\ {\isacharparenleft}{\kern0pt}\isanewline
\ \ \ \ \ \ let\ {\isacharparenleft}{\kern0pt}r{\isadigit{1}}{\isacharcomma}{\kern0pt}x{\isadigit{1}}{\isacharparenright}{\kern0pt}\ {\isacharequal}{\kern0pt}\ decode\ e\ x{\isadigit{0}}\ in\ {\isacharparenleft}{\kern0pt}\isanewline
\ \ \ \ \ \ \ \ let\ {\isacharparenleft}{\kern0pt}r{\isadigit{2}}{\isacharcomma}{\kern0pt}x{\isadigit{2}}{\isacharparenright}{\kern0pt}\ {\isacharequal}{\kern0pt}\ decode{\isacharunderscore}{\kern0pt}list\ e\ x{\isadigit{1}}\ in\ {\isacharparenleft}{\kern0pt}r{\isadigit{1}}{\isacharhash}{\kern0pt}r{\isadigit{2}}{\isacharcomma}{\kern0pt}x{\isadigit{2}}{\isacharparenright}{\kern0pt}{\isacharparenright}{\kern0pt}{\isacharparenright}{\kern0pt}{\isachardoublequoteclose}\ {\isacharbar}{\kern0pt}\isanewline
\ \ \ \ {\isachardoublequoteopen}decode{\isacharunderscore}{\kern0pt}list\ e\ {\isacharparenleft}{\kern0pt}False{\isacharhash}{\kern0pt}x{\isadigit{0}}{\isacharparenright}{\kern0pt}\ {\isacharequal}{\kern0pt}\ {\isacharparenleft}{\kern0pt}{\isacharbrackleft}{\kern0pt}{\isacharbrackright}{\kern0pt}{\isacharcomma}{\kern0pt}\ x{\isadigit{0}}{\isacharparenright}{\kern0pt}{\isachardoublequoteclose}\ {\isacharbar}{\kern0pt}\isanewline
\ \ \ \ {\isachardoublequoteopen}decode{\isacharunderscore}{\kern0pt}list\ e\ {\isacharbrackleft}{\kern0pt}{\isacharbrackright}{\kern0pt}\ {\isacharequal}{\kern0pt}\ undefined{\isachardoublequoteclose}\isanewline
%
\isadelimproof
\ \ %
\endisadelimproof
%
\isatagproof
\isacommand{by}\isamarkupfalse%
\ pat{\isacharunderscore}{\kern0pt}completeness\ auto%
\endisatagproof
{\isafoldproof}%
%
\isadelimproof
\isanewline
%
\endisadelimproof
\isacommand{termination}\isamarkupfalse%
\isanewline
%
\isadelimproof
\ \ %
\endisadelimproof
%
\isatagproof
\isacommand{apply}\isamarkupfalse%
\ {\isacharparenleft}{\kern0pt}relation\ {\isachardoublequoteopen}measure\ {\isacharparenleft}{\kern0pt}{\isasymlambda}{\isacharparenleft}{\kern0pt}{\isacharunderscore}{\kern0pt}{\isacharcomma}{\kern0pt}x{\isacharparenright}{\kern0pt}{\isachardot}{\kern0pt}\ length\ x{\isacharparenright}{\kern0pt}{\isachardoublequoteclose}{\isacharparenright}{\kern0pt}\isanewline
\ \ \isacommand{by}\isamarkupfalse%
\ {\isacharparenleft}{\kern0pt}simp{\isacharplus}{\kern0pt}{\isacharcomma}{\kern0pt}\ metis\ le{\isacharunderscore}{\kern0pt}imp{\isacharunderscore}{\kern0pt}less{\isacharunderscore}{\kern0pt}Suc\ snd{\isacharunderscore}{\kern0pt}decode{\isacharunderscore}{\kern0pt}len{\isacharparenright}{\kern0pt}%
\endisatagproof
{\isafoldproof}%
%
\isadelimproof
\isanewline
%
\endisadelimproof
\isanewline
\isacommand{lemma}\isamarkupfalse%
\ list{\isacharunderscore}{\kern0pt}encoding{\isacharunderscore}{\kern0pt}dom{\isacharcolon}{\kern0pt}\isanewline
\ \ \isakeyword{assumes}\ {\isachardoublequoteopen}set\ l\ {\isasymsubseteq}\ dom\ f{\isachardoublequoteclose}\isanewline
\ \ \isakeyword{shows}\ {\isachardoublequoteopen}l\ {\isasymin}\ dom\ {\isacharparenleft}{\kern0pt}list\isactrlsub S\ f{\isacharparenright}{\kern0pt}{\isachardoublequoteclose}\isanewline
%
\isadelimproof
\ \ %
\endisadelimproof
%
\isatagproof
\isacommand{using}\isamarkupfalse%
\ assms\ \isacommand{apply}\isamarkupfalse%
\ {\isacharparenleft}{\kern0pt}induction\ l{\isacharcomma}{\kern0pt}\ simp\ add{\isacharcolon}{\kern0pt}dom{\isacharunderscore}{\kern0pt}def{\isacharcomma}{\kern0pt}\ simp{\isacharparenright}{\kern0pt}\ \isacommand{by}\isamarkupfalse%
\ fastforce%
\endisatagproof
{\isafoldproof}%
%
\isadelimproof
\isanewline
%
\endisadelimproof
\isanewline
\isacommand{lemma}\isamarkupfalse%
\ list{\isacharunderscore}{\kern0pt}bit{\isacharunderscore}{\kern0pt}count{\isacharcolon}{\kern0pt}\isanewline
\ \ {\isachardoublequoteopen}bit{\isacharunderscore}{\kern0pt}count\ {\isacharparenleft}{\kern0pt}list\isactrlsub S\ f\ xs{\isacharparenright}{\kern0pt}\ {\isacharequal}{\kern0pt}\ {\isacharparenleft}{\kern0pt}{\isasymSum}x\ {\isasymleftarrow}\ xs{\isachardot}{\kern0pt}\ bit{\isacharunderscore}{\kern0pt}count\ {\isacharparenleft}{\kern0pt}f\ x{\isacharparenright}{\kern0pt}\ {\isacharplus}{\kern0pt}\ {\isadigit{1}}{\isacharparenright}{\kern0pt}\ {\isacharplus}{\kern0pt}\ {\isadigit{1}}{\isachardoublequoteclose}\isanewline
%
\isadelimproof
\ \ %
\endisadelimproof
%
\isatagproof
\isacommand{apply}\isamarkupfalse%
\ {\isacharparenleft}{\kern0pt}induction\ xs{\isacharcomma}{\kern0pt}\ simp{\isacharcomma}{\kern0pt}\ simp\ add{\isacharcolon}{\kern0pt}bit{\isacharunderscore}{\kern0pt}count{\isacharunderscore}{\kern0pt}append{\isacharparenright}{\kern0pt}\ \isanewline
\ \ \isacommand{by}\isamarkupfalse%
\ {\isacharparenleft}{\kern0pt}metis\ add{\isachardot}{\kern0pt}commute\ add{\isachardot}{\kern0pt}left{\isacharunderscore}{\kern0pt}commute\ one{\isacharunderscore}{\kern0pt}ereal{\isacharunderscore}{\kern0pt}def{\isacharparenright}{\kern0pt}%
\endisatagproof
{\isafoldproof}%
%
\isadelimproof
\isanewline
%
\endisadelimproof
\isanewline
\isacommand{lemma}\isamarkupfalse%
\ list{\isacharunderscore}{\kern0pt}bit{\isacharunderscore}{\kern0pt}count{\isacharunderscore}{\kern0pt}est{\isacharcolon}{\kern0pt}\isanewline
\ \ \isakeyword{assumes}\ {\isachardoublequoteopen}{\isasymAnd}x{\isachardot}{\kern0pt}\ x\ {\isasymin}\ set\ xs\ {\isasymLongrightarrow}\ bit{\isacharunderscore}{\kern0pt}count\ {\isacharparenleft}{\kern0pt}f\ x{\isacharparenright}{\kern0pt}\ {\isasymle}\ a{\isachardoublequoteclose}\isanewline
\ \ \isakeyword{shows}\ {\isachardoublequoteopen}bit{\isacharunderscore}{\kern0pt}count\ {\isacharparenleft}{\kern0pt}list\isactrlsub S\ f\ xs{\isacharparenright}{\kern0pt}\ {\isasymle}\ ereal\ {\isacharparenleft}{\kern0pt}length\ xs{\isacharparenright}{\kern0pt}\ {\isacharasterisk}{\kern0pt}\ {\isacharparenleft}{\kern0pt}a{\isacharplus}{\kern0pt}{\isadigit{1}}{\isacharparenright}{\kern0pt}\ {\isacharplus}{\kern0pt}\ {\isadigit{1}}{\isachardoublequoteclose}\isanewline
%
\isadelimproof
%
\endisadelimproof
%
\isatagproof
\isacommand{proof}\isamarkupfalse%
\ {\isacharminus}{\kern0pt}\isanewline
\ \ \isacommand{have}\isamarkupfalse%
\ a{\isacharcolon}{\kern0pt}{\isachardoublequoteopen}sum{\isacharunderscore}{\kern0pt}list\ {\isacharparenleft}{\kern0pt}map\ {\isacharparenleft}{\kern0pt}{\isasymlambda}{\isacharunderscore}{\kern0pt}{\isachardot}{\kern0pt}\ {\isacharparenleft}{\kern0pt}a{\isacharplus}{\kern0pt}{\isadigit{1}}{\isacharparenright}{\kern0pt}{\isacharparenright}{\kern0pt}\ xs{\isacharparenright}{\kern0pt}\ {\isacharequal}{\kern0pt}\ length\ xs\ {\isacharasterisk}{\kern0pt}\ {\isacharparenleft}{\kern0pt}a{\isacharplus}{\kern0pt}{\isadigit{1}}{\isacharparenright}{\kern0pt}{\isachardoublequoteclose}\isanewline
\ \ \ \ \isacommand{apply}\isamarkupfalse%
\ {\isacharparenleft}{\kern0pt}induction\ xs{\isacharcomma}{\kern0pt}\ simp{\isacharparenright}{\kern0pt}\isanewline
\ \ \ \ \isacommand{by}\isamarkupfalse%
\ {\isacharparenleft}{\kern0pt}simp{\isacharcomma}{\kern0pt}\ subst\ plus{\isacharunderscore}{\kern0pt}ereal{\isachardot}{\kern0pt}simps{\isacharparenleft}{\kern0pt}{\isadigit{1}}{\isacharparenright}{\kern0pt}{\isacharbrackleft}{\kern0pt}symmetric{\isacharbrackright}{\kern0pt}{\isacharcomma}{\kern0pt}\ subst\ ereal{\isacharunderscore}{\kern0pt}left{\isacharunderscore}{\kern0pt}distrib{\isacharcomma}{\kern0pt}\ simp{\isacharplus}{\kern0pt}{\isacharparenright}{\kern0pt}\isanewline
\isanewline
\ \ \isacommand{have}\isamarkupfalse%
\ b{\isacharcolon}{\kern0pt}\ {\isachardoublequoteopen}{\isasymAnd}x{\isachardot}{\kern0pt}\ x\ {\isasymin}\ set\ xs\ {\isasymLongrightarrow}\ bit{\isacharunderscore}{\kern0pt}count\ {\isacharparenleft}{\kern0pt}f\ x{\isacharparenright}{\kern0pt}\ {\isacharplus}{\kern0pt}{\isadigit{1}}\ {\isasymle}\ a{\isacharplus}{\kern0pt}{\isadigit{1}}{\isachardoublequoteclose}\isanewline
\ \ \ \ \isacommand{using}\isamarkupfalse%
\ assms\ add{\isacharunderscore}{\kern0pt}right{\isacharunderscore}{\kern0pt}mono\ \isacommand{by}\isamarkupfalse%
\ blast\isanewline
\isanewline
\ \ \isacommand{show}\isamarkupfalse%
\ {\isacharquery}{\kern0pt}thesis\ \ \isanewline
\ \ \ \ \isacommand{using}\isamarkupfalse%
\ assms\ a\ \ b\ sum{\isacharunderscore}{\kern0pt}list{\isacharunderscore}{\kern0pt}mono{\isacharbrackleft}{\kern0pt}\isakeyword{where}\ g{\isacharequal}{\kern0pt}{\isachardoublequoteopen}{\isasymlambda}{\isacharunderscore}{\kern0pt}{\isachardot}{\kern0pt}\ a{\isacharplus}{\kern0pt}{\isadigit{1}}{\isachardoublequoteclose}\ \isakeyword{and}\ f{\isacharequal}{\kern0pt}{\isachardoublequoteopen}{\isasymlambda}x{\isachardot}{\kern0pt}\ bit{\isacharunderscore}{\kern0pt}count\ {\isacharparenleft}{\kern0pt}f\ x{\isacharparenright}{\kern0pt}{\isacharplus}{\kern0pt}{\isadigit{1}}{\isachardoublequoteclose}\ \isakeyword{and}\ xs{\isacharequal}{\kern0pt}{\isachardoublequoteopen}xs{\isachardoublequoteclose}{\isacharbrackright}{\kern0pt}\isanewline
\ \ \ \ \isacommand{by}\isamarkupfalse%
\ {\isacharparenleft}{\kern0pt}simp\ add{\isacharcolon}{\kern0pt}list{\isacharunderscore}{\kern0pt}bit{\isacharunderscore}{\kern0pt}count\ ereal{\isacharunderscore}{\kern0pt}add{\isacharunderscore}{\kern0pt}le{\isacharunderscore}{\kern0pt}add{\isacharunderscore}{\kern0pt}iff{\isadigit{2}}{\isacharparenright}{\kern0pt}\isanewline
\isacommand{qed}\isamarkupfalse%
%
\endisatagproof
{\isafoldproof}%
%
\isadelimproof
\isanewline
%
\endisadelimproof
\isanewline
\isacommand{lemma}\isamarkupfalse%
\ list{\isacharunderscore}{\kern0pt}bit{\isacharunderscore}{\kern0pt}count{\isacharunderscore}{\kern0pt}estI{\isacharcolon}{\kern0pt}\isanewline
\ \ \isakeyword{assumes}\ {\isachardoublequoteopen}{\isasymAnd}x{\isachardot}{\kern0pt}\ x\ {\isasymin}\ set\ xs\ {\isasymLongrightarrow}\ bit{\isacharunderscore}{\kern0pt}count\ {\isacharparenleft}{\kern0pt}f\ x{\isacharparenright}{\kern0pt}\ {\isasymle}\ a{\isachardoublequoteclose}\isanewline
\ \ \isakeyword{assumes}\ {\isachardoublequoteopen}ereal\ {\isacharparenleft}{\kern0pt}real\ {\isacharparenleft}{\kern0pt}length\ xs{\isacharparenright}{\kern0pt}{\isacharparenright}{\kern0pt}\ {\isacharasterisk}{\kern0pt}\ {\isacharparenleft}{\kern0pt}a{\isacharplus}{\kern0pt}{\isadigit{1}}{\isacharparenright}{\kern0pt}\ {\isacharplus}{\kern0pt}\ {\isadigit{1}}\ {\isasymle}\ h{\isachardoublequoteclose}\isanewline
\ \ \isakeyword{shows}\ {\isachardoublequoteopen}bit{\isacharunderscore}{\kern0pt}count\ {\isacharparenleft}{\kern0pt}list\isactrlsub S\ f\ xs{\isacharparenright}{\kern0pt}\ {\isasymle}\ h{\isachardoublequoteclose}\isanewline
%
\isadelimproof
\ \ %
\endisadelimproof
%
\isatagproof
\isacommand{using}\isamarkupfalse%
\ list{\isacharunderscore}{\kern0pt}bit{\isacharunderscore}{\kern0pt}count{\isacharunderscore}{\kern0pt}est{\isacharbrackleft}{\kern0pt}OF\ assms{\isacharparenleft}{\kern0pt}{\isadigit{1}}{\isacharparenright}{\kern0pt}{\isacharbrackright}{\kern0pt}\ assms{\isacharparenleft}{\kern0pt}{\isadigit{2}}{\isacharparenright}{\kern0pt}\ order{\isacharunderscore}{\kern0pt}trans\ \isacommand{by}\isamarkupfalse%
\ fastforce%
\endisatagproof
{\isafoldproof}%
%
\isadelimproof
\ \isanewline
%
\endisadelimproof
\isanewline
\isacommand{lemma}\isamarkupfalse%
\ list{\isacharunderscore}{\kern0pt}encoding{\isacharunderscore}{\kern0pt}aux{\isacharcolon}{\kern0pt}\isanewline
\ \ \isakeyword{assumes}\ {\isachardoublequoteopen}is{\isacharunderscore}{\kern0pt}encoding\ f{\isachardoublequoteclose}\isanewline
\ \ \isakeyword{shows}\ {\isachardoublequoteopen}x\ {\isasymin}\ dom\ {\isacharparenleft}{\kern0pt}list\isactrlsub S\ f{\isacharparenright}{\kern0pt}\ {\isasymLongrightarrow}\ decode{\isacharunderscore}{\kern0pt}list\ f\ {\isacharparenleft}{\kern0pt}the\ {\isacharparenleft}{\kern0pt}list\isactrlsub S\ f\ x{\isacharparenright}{\kern0pt}\ {\isacharat}{\kern0pt}\ y{\isacharparenright}{\kern0pt}\ {\isacharequal}{\kern0pt}\ {\isacharparenleft}{\kern0pt}x{\isacharcomma}{\kern0pt}\ y{\isacharparenright}{\kern0pt}{\isachardoublequoteclose}\isanewline
%
\isadelimproof
%
\endisadelimproof
%
\isatagproof
\isacommand{proof}\isamarkupfalse%
\ \ {\isacharparenleft}{\kern0pt}induction\ x{\isacharparenright}{\kern0pt}\isanewline
\ \ \isacommand{case}\isamarkupfalse%
\ Nil\isanewline
\ \ \isacommand{then}\isamarkupfalse%
\ \isacommand{show}\isamarkupfalse%
\ {\isacharquery}{\kern0pt}case\ \isacommand{by}\isamarkupfalse%
\ simp\isanewline
\isacommand{next}\isamarkupfalse%
\isanewline
\ \ \isacommand{case}\isamarkupfalse%
\ {\isacharparenleft}{\kern0pt}Cons\ a\ x{\isacharparenright}{\kern0pt}\isanewline
\ \ \isacommand{then}\isamarkupfalse%
\ \isacommand{show}\isamarkupfalse%
\ {\isacharquery}{\kern0pt}case\isanewline
\ \ \ \ \isacommand{apply}\isamarkupfalse%
\ {\isacharparenleft}{\kern0pt}cases\ {\isachardoublequoteopen}f\ a{\isachardoublequoteclose}{\isacharcomma}{\kern0pt}\ simp\ add{\isacharcolon}{\kern0pt}dom{\isacharunderscore}{\kern0pt}def{\isacharparenright}{\kern0pt}\isanewline
\ \ \ \ \isacommand{apply}\isamarkupfalse%
\ {\isacharparenleft}{\kern0pt}cases\ {\isachardoublequoteopen}list\isactrlsub S\ f\ x{\isachardoublequoteclose}{\isacharcomma}{\kern0pt}\ simp\ add{\isacharcolon}{\kern0pt}dom{\isacharunderscore}{\kern0pt}def{\isacharparenright}{\kern0pt}\isanewline
\ \ \ \ \isacommand{using}\isamarkupfalse%
\ assms\ \isacommand{by}\isamarkupfalse%
\ {\isacharparenleft}{\kern0pt}simp\ add{\isacharcolon}{\kern0pt}\ dom{\isacharunderscore}{\kern0pt}def\ decode{\isacharunderscore}{\kern0pt}elim{\isacharparenright}{\kern0pt}\isanewline
\isacommand{qed}\isamarkupfalse%
%
\endisatagproof
{\isafoldproof}%
%
\isadelimproof
\isanewline
%
\endisadelimproof
\isanewline
\isacommand{lemma}\isamarkupfalse%
\ list{\isacharunderscore}{\kern0pt}encoding{\isacharcolon}{\kern0pt}\isanewline
\ \ \isakeyword{assumes}\ {\isachardoublequoteopen}is{\isacharunderscore}{\kern0pt}encoding\ f{\isachardoublequoteclose}\isanewline
\ \ \isakeyword{shows}\ {\isachardoublequoteopen}is{\isacharunderscore}{\kern0pt}encoding\ {\isacharparenleft}{\kern0pt}list\isactrlsub S\ f{\isacharparenright}{\kern0pt}{\isachardoublequoteclose}\isanewline
%
\isadelimproof
\ \ %
\endisadelimproof
%
\isatagproof
\isacommand{by}\isamarkupfalse%
\ {\isacharparenleft}{\kern0pt}metis\ encoding{\isacharunderscore}{\kern0pt}by{\isacharunderscore}{\kern0pt}witness{\isacharbrackleft}{\kern0pt}\isakeyword{where}\ g{\isacharequal}{\kern0pt}{\isachardoublequoteopen}decode{\isacharunderscore}{\kern0pt}list\ f{\isachardoublequoteclose}{\isacharbrackright}{\kern0pt}\ list{\isacharunderscore}{\kern0pt}encoding{\isacharunderscore}{\kern0pt}aux\ assms{\isacharparenright}{\kern0pt}%
\endisatagproof
{\isafoldproof}%
%
\isadelimproof
%
\endisadelimproof
%
\begin{isamarkuptext}%
Encoding for natural numbers%
\end{isamarkuptext}\isamarkuptrue%
\isacommand{fun}\isamarkupfalse%
\ nat{\isacharunderscore}{\kern0pt}encoding{\isacharunderscore}{\kern0pt}aux\ {\isacharcolon}{\kern0pt}{\isacharcolon}{\kern0pt}\ {\isachardoublequoteopen}nat\ {\isasymRightarrow}\ bool\ list{\isachardoublequoteclose}\ \isanewline
\ \ \isakeyword{where}\isanewline
\ \ \ \ {\isachardoublequoteopen}nat{\isacharunderscore}{\kern0pt}encoding{\isacharunderscore}{\kern0pt}aux\ {\isadigit{0}}\ {\isacharequal}{\kern0pt}\ {\isacharbrackleft}{\kern0pt}False{\isacharbrackright}{\kern0pt}{\isachardoublequoteclose}\ {\isacharbar}{\kern0pt}\isanewline
\ \ \ \ {\isachardoublequoteopen}nat{\isacharunderscore}{\kern0pt}encoding{\isacharunderscore}{\kern0pt}aux\ {\isacharparenleft}{\kern0pt}Suc\ n{\isacharparenright}{\kern0pt}\ {\isacharequal}{\kern0pt}\ True{\isacharhash}{\kern0pt}{\isacharparenleft}{\kern0pt}odd\ n{\isacharparenright}{\kern0pt}{\isacharhash}{\kern0pt}nat{\isacharunderscore}{\kern0pt}encoding{\isacharunderscore}{\kern0pt}aux\ {\isacharparenleft}{\kern0pt}n\ div\ {\isadigit{2}}{\isacharparenright}{\kern0pt}{\isachardoublequoteclose}\isanewline
\isanewline
\isacommand{fun}\isamarkupfalse%
\ N\isactrlsub S\ \isakeyword{where}\ {\isachardoublequoteopen}N\isactrlsub S\ n\ {\isacharequal}{\kern0pt}\ Some\ {\isacharparenleft}{\kern0pt}nat{\isacharunderscore}{\kern0pt}encoding{\isacharunderscore}{\kern0pt}aux\ n{\isacharparenright}{\kern0pt}{\isachardoublequoteclose}\isanewline
\isanewline
\isacommand{fun}\isamarkupfalse%
\ decode{\isacharunderscore}{\kern0pt}nat\ {\isacharcolon}{\kern0pt}{\isacharcolon}{\kern0pt}\ {\isachardoublequoteopen}bool\ list\ {\isasymRightarrow}\ nat\ {\isasymtimes}\ bool\ list{\isachardoublequoteclose}\ \isanewline
\ \ \isakeyword{where}\isanewline
\ \ \ \ {\isachardoublequoteopen}decode{\isacharunderscore}{\kern0pt}nat\ {\isacharparenleft}{\kern0pt}False{\isacharhash}{\kern0pt}y{\isacharparenright}{\kern0pt}\ {\isacharequal}{\kern0pt}\ {\isacharparenleft}{\kern0pt}{\isadigit{0}}{\isacharcomma}{\kern0pt}y{\isacharparenright}{\kern0pt}{\isachardoublequoteclose}\ {\isacharbar}{\kern0pt}\isanewline
\ \ \ \ {\isachardoublequoteopen}decode{\isacharunderscore}{\kern0pt}nat\ {\isacharparenleft}{\kern0pt}True{\isacharhash}{\kern0pt}x{\isacharhash}{\kern0pt}xs{\isacharparenright}{\kern0pt}\ {\isacharequal}{\kern0pt}\ \isanewline
\ \ \ \ \ \ {\isacharparenleft}{\kern0pt}let\ {\isacharparenleft}{\kern0pt}n{\isacharcomma}{\kern0pt}\ rs{\isacharparenright}{\kern0pt}\ {\isacharequal}{\kern0pt}\ decode{\isacharunderscore}{\kern0pt}nat\ xs\ in\ {\isacharparenleft}{\kern0pt}n\ {\isacharasterisk}{\kern0pt}\ {\isadigit{2}}\ {\isacharplus}{\kern0pt}\ {\isadigit{1}}\ {\isacharplus}{\kern0pt}\ {\isacharparenleft}{\kern0pt}if\ x\ then\ {\isadigit{1}}\ else\ {\isadigit{0}}{\isacharparenright}{\kern0pt}{\isacharcomma}{\kern0pt}\ rs{\isacharparenright}{\kern0pt}{\isacharparenright}{\kern0pt}{\isachardoublequoteclose}\ {\isacharbar}{\kern0pt}\isanewline
\ \ \ \ {\isachardoublequoteopen}decode{\isacharunderscore}{\kern0pt}nat\ {\isacharunderscore}{\kern0pt}\ {\isacharequal}{\kern0pt}\ undefined{\isachardoublequoteclose}\isanewline
\isanewline
\isacommand{lemma}\isamarkupfalse%
\ nat{\isacharunderscore}{\kern0pt}encoding{\isacharunderscore}{\kern0pt}aux{\isacharcolon}{\kern0pt}\isanewline
\ \ {\isachardoublequoteopen}decode{\isacharunderscore}{\kern0pt}nat\ {\isacharparenleft}{\kern0pt}nat{\isacharunderscore}{\kern0pt}encoding{\isacharunderscore}{\kern0pt}aux\ x\ {\isacharat}{\kern0pt}\ y{\isacharparenright}{\kern0pt}\ {\isacharequal}{\kern0pt}\ {\isacharparenleft}{\kern0pt}x{\isacharcomma}{\kern0pt}\ y{\isacharparenright}{\kern0pt}{\isachardoublequoteclose}\isanewline
%
\isadelimproof
\ \ %
\endisadelimproof
%
\isatagproof
\isacommand{by}\isamarkupfalse%
\ {\isacharparenleft}{\kern0pt}induction\ x\ rule{\isacharcolon}{\kern0pt}nat{\isacharunderscore}{\kern0pt}encoding{\isacharunderscore}{\kern0pt}aux{\isachardot}{\kern0pt}induct{\isacharcomma}{\kern0pt}\ simp{\isacharcomma}{\kern0pt}\ simp\ add{\isacharcolon}{\kern0pt}mult{\isachardot}{\kern0pt}commute{\isacharparenright}{\kern0pt}%
\endisatagproof
{\isafoldproof}%
%
\isadelimproof
\isanewline
%
\endisadelimproof
\isanewline
\isacommand{lemma}\isamarkupfalse%
\ nat{\isacharunderscore}{\kern0pt}encoding{\isacharcolon}{\kern0pt}\isanewline
\ \ \isakeyword{shows}\ {\isachardoublequoteopen}is{\isacharunderscore}{\kern0pt}encoding\ N\isactrlsub S{\isachardoublequoteclose}\isanewline
%
\isadelimproof
\ \ %
\endisadelimproof
%
\isatagproof
\isacommand{by}\isamarkupfalse%
\ {\isacharparenleft}{\kern0pt}rule\ encoding{\isacharunderscore}{\kern0pt}by{\isacharunderscore}{\kern0pt}witness{\isacharbrackleft}{\kern0pt}\isakeyword{where}\ g{\isacharequal}{\kern0pt}{\isachardoublequoteopen}decode{\isacharunderscore}{\kern0pt}nat{\isachardoublequoteclose}{\isacharbrackright}{\kern0pt}{\isacharcomma}{\kern0pt}\ simp\ add{\isacharcolon}{\kern0pt}nat{\isacharunderscore}{\kern0pt}encoding{\isacharunderscore}{\kern0pt}aux{\isacharparenright}{\kern0pt}%
\endisatagproof
{\isafoldproof}%
%
\isadelimproof
\isanewline
%
\endisadelimproof
\isanewline
\isacommand{lemma}\isamarkupfalse%
\ nat{\isacharunderscore}{\kern0pt}bit{\isacharunderscore}{\kern0pt}count{\isacharcolon}{\kern0pt}\isanewline
\ \ {\isachardoublequoteopen}bit{\isacharunderscore}{\kern0pt}count\ {\isacharparenleft}{\kern0pt}N\isactrlsub S\ n{\isacharparenright}{\kern0pt}\ {\isasymle}\ {\isadigit{2}}\ {\isacharasterisk}{\kern0pt}\ log\ {\isadigit{2}}\ {\isacharparenleft}{\kern0pt}real\ n{\isacharplus}{\kern0pt}{\isadigit{1}}{\isacharparenright}{\kern0pt}\ {\isacharplus}{\kern0pt}\ {\isadigit{1}}{\isachardoublequoteclose}\isanewline
%
\isadelimproof
%
\endisadelimproof
%
\isatagproof
\isacommand{proof}\isamarkupfalse%
\ {\isacharparenleft}{\kern0pt}induction\ n\ rule{\isacharcolon}{\kern0pt}nat{\isacharunderscore}{\kern0pt}encoding{\isacharunderscore}{\kern0pt}aux{\isachardot}{\kern0pt}induct{\isacharparenright}{\kern0pt}\isanewline
\ \ \isacommand{case}\isamarkupfalse%
\ {\isadigit{1}}\isanewline
\ \ \isacommand{then}\isamarkupfalse%
\ \isacommand{show}\isamarkupfalse%
\ {\isacharquery}{\kern0pt}case\ \isacommand{by}\isamarkupfalse%
\ simp\isanewline
\isacommand{next}\isamarkupfalse%
\isanewline
\ \ \isacommand{case}\isamarkupfalse%
\ {\isacharparenleft}{\kern0pt}{\isadigit{2}}\ n{\isacharparenright}{\kern0pt}\isanewline
\ \ \isacommand{have}\isamarkupfalse%
\ {\isachardoublequoteopen}log\ {\isadigit{2}}\ {\isadigit{2}}\ {\isacharplus}{\kern0pt}\ log\ {\isadigit{2}}\ {\isacharparenleft}{\kern0pt}{\isadigit{1}}\ {\isacharplus}{\kern0pt}\ real\ {\isacharparenleft}{\kern0pt}n\ div\ {\isadigit{2}}{\isacharparenright}{\kern0pt}{\isacharparenright}{\kern0pt}\ {\isasymle}\ log\ {\isadigit{2}}\ {\isacharparenleft}{\kern0pt}{\isadigit{2}}\ {\isacharplus}{\kern0pt}\ real\ n{\isacharparenright}{\kern0pt}{\isachardoublequoteclose}\isanewline
\ \ \ \ \isacommand{apply}\isamarkupfalse%
\ {\isacharparenleft}{\kern0pt}subst\ log{\isacharunderscore}{\kern0pt}mult{\isacharbrackleft}{\kern0pt}symmetric{\isacharbrackright}{\kern0pt}{\isacharcomma}{\kern0pt}\ simp{\isacharcomma}{\kern0pt}\ simp{\isacharcomma}{\kern0pt}\ simp{\isacharparenright}{\kern0pt}\isanewline
\ \ \ \ \isacommand{by}\isamarkupfalse%
\ {\isacharparenleft}{\kern0pt}subst\ log{\isacharunderscore}{\kern0pt}le{\isacharunderscore}{\kern0pt}cancel{\isacharunderscore}{\kern0pt}iff{\isacharcomma}{\kern0pt}\ simp{\isacharplus}{\kern0pt}{\isacharparenright}{\kern0pt}\isanewline
\ \ \isacommand{hence}\isamarkupfalse%
\ {\isachardoublequoteopen}{\isadigit{1}}\ {\isacharplus}{\kern0pt}\ {\isadigit{2}}\ {\isacharasterisk}{\kern0pt}\ log\ {\isadigit{2}}\ {\isacharparenleft}{\kern0pt}{\isadigit{1}}\ {\isacharplus}{\kern0pt}\ real\ {\isacharparenleft}{\kern0pt}n\ div\ {\isadigit{2}}{\isacharparenright}{\kern0pt}{\isacharparenright}{\kern0pt}\ {\isacharplus}{\kern0pt}\ {\isadigit{1}}\ {\isasymle}\ {\isadigit{2}}\ {\isacharasterisk}{\kern0pt}\ log\ {\isadigit{2}}\ {\isacharparenleft}{\kern0pt}{\isadigit{2}}\ {\isacharplus}{\kern0pt}\ real\ n{\isacharparenright}{\kern0pt}{\isachardoublequoteclose}\isanewline
\ \ \ \ \isacommand{by}\isamarkupfalse%
\ simp\isanewline
\ \ \isacommand{thus}\isamarkupfalse%
\ {\isacharquery}{\kern0pt}case\ \isacommand{using}\isamarkupfalse%
\ {\isadigit{2}}\ \isacommand{by}\isamarkupfalse%
\ {\isacharparenleft}{\kern0pt}simp\ add{\isacharcolon}{\kern0pt}add{\isachardot}{\kern0pt}commute{\isacharparenright}{\kern0pt}\ \isanewline
\isacommand{qed}\isamarkupfalse%
%
\endisatagproof
{\isafoldproof}%
%
\isadelimproof
\isanewline
%
\endisadelimproof
\isanewline
\isacommand{lemma}\isamarkupfalse%
\ nat{\isacharunderscore}{\kern0pt}bit{\isacharunderscore}{\kern0pt}count{\isacharunderscore}{\kern0pt}est{\isacharcolon}{\kern0pt}\isanewline
\ \ \isakeyword{assumes}\ {\isachardoublequoteopen}n\ {\isasymle}\ m{\isachardoublequoteclose}\isanewline
\ \ \isakeyword{shows}\ {\isachardoublequoteopen}bit{\isacharunderscore}{\kern0pt}count\ {\isacharparenleft}{\kern0pt}N\isactrlsub S\ n{\isacharparenright}{\kern0pt}\ {\isasymle}\ {\isadigit{2}}\ {\isacharasterisk}{\kern0pt}\ log\ {\isadigit{2}}\ {\isacharparenleft}{\kern0pt}{\isadigit{1}}{\isacharplus}{\kern0pt}real\ m{\isacharparenright}{\kern0pt}\ {\isacharplus}{\kern0pt}\ {\isadigit{1}}{\isachardoublequoteclose}\isanewline
%
\isadelimproof
%
\endisadelimproof
%
\isatagproof
\isacommand{proof}\isamarkupfalse%
\ {\isacharminus}{\kern0pt}\isanewline
\ \ \isacommand{have}\isamarkupfalse%
\ {\isachardoublequoteopen}{\isadigit{2}}\ {\isacharasterisk}{\kern0pt}\ log\ {\isadigit{2}}\ {\isacharparenleft}{\kern0pt}{\isadigit{1}}\ {\isacharplus}{\kern0pt}\ real\ n{\isacharparenright}{\kern0pt}\ {\isacharplus}{\kern0pt}\ {\isadigit{1}}\ {\isasymle}\ {\isadigit{2}}\ {\isacharasterisk}{\kern0pt}\ log\ {\isadigit{2}}\ {\isacharparenleft}{\kern0pt}{\isadigit{1}}{\isacharplus}{\kern0pt}real\ m{\isacharparenright}{\kern0pt}\ {\isacharplus}{\kern0pt}\ {\isadigit{1}}{\isachardoublequoteclose}\ \isanewline
\ \ \ \ \isacommand{using}\isamarkupfalse%
\ assms\ \isacommand{by}\isamarkupfalse%
\ simp\isanewline
\ \ \isacommand{thus}\isamarkupfalse%
\ {\isacharquery}{\kern0pt}thesis\isanewline
\ \ \ \ \isacommand{by}\isamarkupfalse%
\ {\isacharparenleft}{\kern0pt}metis\ nat{\isacharunderscore}{\kern0pt}bit{\isacharunderscore}{\kern0pt}count\ le{\isacharunderscore}{\kern0pt}ereal{\isacharunderscore}{\kern0pt}le\ \ add{\isachardot}{\kern0pt}commute{\isacharparenright}{\kern0pt}\isanewline
\isacommand{qed}\isamarkupfalse%
%
\endisatagproof
{\isafoldproof}%
%
\isadelimproof
%
\endisadelimproof
%
\begin{isamarkuptext}%
Encoding for integers%
\end{isamarkuptext}\isamarkuptrue%
\isacommand{fun}\isamarkupfalse%
\ I\isactrlsub S\ {\isacharcolon}{\kern0pt}{\isacharcolon}{\kern0pt}\ {\isachardoublequoteopen}int\ {\isasymRightarrow}\ bool\ list\ option{\isachardoublequoteclose}\isanewline
\ \ \isakeyword{where}\ \isanewline
\ \ \ \ {\isachardoublequoteopen}I\isactrlsub S\ n\ {\isacharequal}{\kern0pt}\ {\isacharparenleft}{\kern0pt}if\ n\ {\isasymge}\ {\isadigit{0}}\ then\ Some\ {\isacharbrackleft}{\kern0pt}True{\isacharbrackright}{\kern0pt}{\isacharat}{\kern0pt}\isactrlsub SN\isactrlsub S\ {\isacharparenleft}{\kern0pt}nat\ n{\isacharparenright}{\kern0pt}\ else\ Some\ {\isacharbrackleft}{\kern0pt}False{\isacharbrackright}{\kern0pt}{\isacharat}{\kern0pt}\isactrlsub S\ {\isacharparenleft}{\kern0pt}N\isactrlsub S\ {\isacharparenleft}{\kern0pt}nat\ {\isacharparenleft}{\kern0pt}{\isacharminus}{\kern0pt}n{\isacharminus}{\kern0pt}{\isadigit{1}}{\isacharparenright}{\kern0pt}{\isacharparenright}{\kern0pt}{\isacharparenright}{\kern0pt}{\isacharparenright}{\kern0pt}{\isachardoublequoteclose}\ \isanewline
\isanewline
\isacommand{fun}\isamarkupfalse%
\ decode{\isacharunderscore}{\kern0pt}int\ {\isacharcolon}{\kern0pt}{\isacharcolon}{\kern0pt}\ {\isachardoublequoteopen}bool\ list\ {\isasymRightarrow}\ {\isacharparenleft}{\kern0pt}int\ {\isasymtimes}\ bool\ list{\isacharparenright}{\kern0pt}{\isachardoublequoteclose}\isanewline
\ \ \isakeyword{where}\ \isanewline
\ \ \ \ {\isachardoublequoteopen}decode{\isacharunderscore}{\kern0pt}int\ {\isacharparenleft}{\kern0pt}True{\isacharhash}{\kern0pt}xs{\isacharparenright}{\kern0pt}\ {\isacharequal}{\kern0pt}\ {\isacharparenleft}{\kern0pt}{\isasymlambda}{\isacharparenleft}{\kern0pt}x{\isacharcolon}{\kern0pt}{\isacharcolon}{\kern0pt}nat{\isacharcomma}{\kern0pt}y{\isacharparenright}{\kern0pt}{\isachardot}{\kern0pt}\ {\isacharparenleft}{\kern0pt}int\ x{\isacharcomma}{\kern0pt}\ y{\isacharparenright}{\kern0pt}{\isacharparenright}{\kern0pt}\ {\isacharparenleft}{\kern0pt}decode{\isacharunderscore}{\kern0pt}nat\ xs{\isacharparenright}{\kern0pt}{\isachardoublequoteclose}\ {\isacharbar}{\kern0pt}\ \isanewline
\ \ \ \ {\isachardoublequoteopen}decode{\isacharunderscore}{\kern0pt}int\ {\isacharparenleft}{\kern0pt}False{\isacharhash}{\kern0pt}xs{\isacharparenright}{\kern0pt}\ {\isacharequal}{\kern0pt}\ {\isacharparenleft}{\kern0pt}{\isasymlambda}{\isacharparenleft}{\kern0pt}x{\isacharcolon}{\kern0pt}{\isacharcolon}{\kern0pt}nat{\isacharcomma}{\kern0pt}y{\isacharparenright}{\kern0pt}{\isachardot}{\kern0pt}\ {\isacharparenleft}{\kern0pt}{\isacharminus}{\kern0pt}{\isacharparenleft}{\kern0pt}int\ x{\isacharparenright}{\kern0pt}{\isacharminus}{\kern0pt}{\isadigit{1}}{\isacharcomma}{\kern0pt}\ y{\isacharparenright}{\kern0pt}{\isacharparenright}{\kern0pt}\ {\isacharparenleft}{\kern0pt}decode{\isacharunderscore}{\kern0pt}nat\ xs{\isacharparenright}{\kern0pt}{\isachardoublequoteclose}\ {\isacharbar}{\kern0pt}\isanewline
\ \ \ \ {\isachardoublequoteopen}decode{\isacharunderscore}{\kern0pt}int\ {\isacharbrackleft}{\kern0pt}{\isacharbrackright}{\kern0pt}\ {\isacharequal}{\kern0pt}\ undefined{\isachardoublequoteclose}\isanewline
\isanewline
\isacommand{lemma}\isamarkupfalse%
\ int{\isacharunderscore}{\kern0pt}encoding{\isacharcolon}{\kern0pt}\ {\isachardoublequoteopen}is{\isacharunderscore}{\kern0pt}encoding\ I\isactrlsub S{\isachardoublequoteclose}\isanewline
%
\isadelimproof
\ \ %
\endisadelimproof
%
\isatagproof
\isacommand{apply}\isamarkupfalse%
\ {\isacharparenleft}{\kern0pt}rule\ encoding{\isacharunderscore}{\kern0pt}by{\isacharunderscore}{\kern0pt}witness{\isacharbrackleft}{\kern0pt}\isakeyword{where}\ g{\isacharequal}{\kern0pt}{\isachardoublequoteopen}decode{\isacharunderscore}{\kern0pt}int{\isachardoublequoteclose}{\isacharbrackright}{\kern0pt}{\isacharparenright}{\kern0pt}\isanewline
\ \ \isacommand{by}\isamarkupfalse%
\ {\isacharparenleft}{\kern0pt}simp\ add{\isacharcolon}{\kern0pt}nat{\isacharunderscore}{\kern0pt}encoding{\isacharunderscore}{\kern0pt}aux{\isacharparenright}{\kern0pt}%
\endisatagproof
{\isafoldproof}%
%
\isadelimproof
\isanewline
%
\endisadelimproof
\isanewline
\isacommand{lemma}\isamarkupfalse%
\ int{\isacharunderscore}{\kern0pt}bit{\isacharunderscore}{\kern0pt}count{\isacharcolon}{\kern0pt}\isanewline
\ \ {\isachardoublequoteopen}bit{\isacharunderscore}{\kern0pt}count\ {\isacharparenleft}{\kern0pt}I\isactrlsub S\ x{\isacharparenright}{\kern0pt}\ {\isasymle}\ {\isadigit{2}}\ {\isacharasterisk}{\kern0pt}\ log\ {\isadigit{2}}\ {\isacharparenleft}{\kern0pt}{\isasymbar}x{\isasymbar}{\isacharplus}{\kern0pt}{\isadigit{1}}{\isacharparenright}{\kern0pt}\ {\isacharplus}{\kern0pt}\ {\isadigit{2}}{\isachardoublequoteclose}\isanewline
%
\isadelimproof
%
\endisadelimproof
%
\isatagproof
\isacommand{proof}\isamarkupfalse%
\ {\isacharminus}{\kern0pt}\isanewline
\ \ \isacommand{have}\isamarkupfalse%
\ a{\isacharcolon}{\kern0pt}{\isachardoublequoteopen}{\isasymnot}{\isacharparenleft}{\kern0pt}{\isadigit{0}}\ {\isasymle}\ x{\isacharparenright}{\kern0pt}\ {\isasymLongrightarrow}\ {\isadigit{1}}\ {\isacharplus}{\kern0pt}\ {\isadigit{2}}\ {\isacharasterisk}{\kern0pt}\ log\ {\isadigit{2}}\ {\isacharparenleft}{\kern0pt}{\isacharminus}{\kern0pt}\ real{\isacharunderscore}{\kern0pt}of{\isacharunderscore}{\kern0pt}int\ x{\isacharparenright}{\kern0pt}\ {\isasymle}\ {\isadigit{1}}\ {\isacharplus}{\kern0pt}\ {\isadigit{2}}\ {\isacharasterisk}{\kern0pt}\ log\ {\isadigit{2}}\ {\isacharparenleft}{\kern0pt}{\isadigit{1}}\ {\isacharminus}{\kern0pt}\ real{\isacharunderscore}{\kern0pt}of{\isacharunderscore}{\kern0pt}int\ x{\isacharparenright}{\kern0pt}{\isachardoublequoteclose}\isanewline
\ \ \ \ \isacommand{by}\isamarkupfalse%
\ simp\isanewline
\ \ \isacommand{show}\isamarkupfalse%
\ {\isacharquery}{\kern0pt}thesis\isanewline
\ \ \ \ \isacommand{apply}\isamarkupfalse%
\ {\isacharparenleft}{\kern0pt}cases\ {\isachardoublequoteopen}x\ {\isasymge}\ {\isadigit{0}}{\isachardoublequoteclose}{\isacharparenright}{\kern0pt}\isanewline
\ \ \ \ \ \isacommand{using}\isamarkupfalse%
\ nat{\isacharunderscore}{\kern0pt}bit{\isacharunderscore}{\kern0pt}count{\isacharbrackleft}{\kern0pt}\isakeyword{where}\ n{\isacharequal}{\kern0pt}{\isachardoublequoteopen}nat\ x{\isachardoublequoteclose}{\isacharbrackright}{\kern0pt}\ \isacommand{apply}\isamarkupfalse%
\ {\isacharparenleft}{\kern0pt}simp\ add{\isacharcolon}{\kern0pt}\ bit{\isacharunderscore}{\kern0pt}count{\isacharunderscore}{\kern0pt}append\ add{\isachardot}{\kern0pt}commute{\isacharparenright}{\kern0pt}\isanewline
\ \ \ \ \ \isacommand{using}\isamarkupfalse%
\ nat{\isacharunderscore}{\kern0pt}bit{\isacharunderscore}{\kern0pt}count{\isacharbrackleft}{\kern0pt}\isakeyword{where}\ n{\isacharequal}{\kern0pt}{\isachardoublequoteopen}nat\ {\isacharparenleft}{\kern0pt}{\isacharminus}{\kern0pt}x{\isacharminus}{\kern0pt}{\isadigit{1}}{\isacharparenright}{\kern0pt}{\isachardoublequoteclose}{\isacharbrackright}{\kern0pt}\ \isacommand{apply}\isamarkupfalse%
\ {\isacharparenleft}{\kern0pt}simp\ add{\isacharcolon}{\kern0pt}\ bit{\isacharunderscore}{\kern0pt}count{\isacharunderscore}{\kern0pt}append\ add{\isachardot}{\kern0pt}commute{\isacharparenright}{\kern0pt}\isanewline
\ \ \ \ \ \isacommand{using}\isamarkupfalse%
\ a\ order{\isacharunderscore}{\kern0pt}trans\ \isacommand{by}\isamarkupfalse%
\ blast\isanewline
\isacommand{qed}\isamarkupfalse%
%
\endisatagproof
{\isafoldproof}%
%
\isadelimproof
\isanewline
%
\endisadelimproof
\isanewline
\isacommand{lemma}\isamarkupfalse%
\ int{\isacharunderscore}{\kern0pt}bit{\isacharunderscore}{\kern0pt}count{\isacharunderscore}{\kern0pt}est{\isacharcolon}{\kern0pt}\isanewline
\ \ \isakeyword{assumes}\ {\isachardoublequoteopen}abs\ n\ {\isasymle}\ m{\isachardoublequoteclose}\isanewline
\ \ \isakeyword{shows}\ {\isachardoublequoteopen}bit{\isacharunderscore}{\kern0pt}count\ {\isacharparenleft}{\kern0pt}I\isactrlsub S\ n{\isacharparenright}{\kern0pt}\ {\isasymle}\ {\isadigit{2}}\ {\isacharasterisk}{\kern0pt}\ log\ {\isadigit{2}}\ {\isacharparenleft}{\kern0pt}m{\isacharplus}{\kern0pt}{\isadigit{1}}{\isacharparenright}{\kern0pt}\ {\isacharplus}{\kern0pt}\ {\isadigit{2}}{\isachardoublequoteclose}\isanewline
%
\isadelimproof
%
\endisadelimproof
%
\isatagproof
\isacommand{proof}\isamarkupfalse%
\ {\isacharminus}{\kern0pt}\isanewline
\ \ \isacommand{have}\isamarkupfalse%
\ {\isachardoublequoteopen}{\isadigit{2}}\ {\isacharasterisk}{\kern0pt}\ log\ {\isadigit{2}}\ {\isacharparenleft}{\kern0pt}abs\ n{\isacharplus}{\kern0pt}{\isadigit{1}}{\isacharparenright}{\kern0pt}\ {\isacharplus}{\kern0pt}\ {\isadigit{2}}\ {\isasymle}\ {\isadigit{2}}\ {\isacharasterisk}{\kern0pt}\ log\ {\isadigit{2}}\ {\isacharparenleft}{\kern0pt}m{\isacharplus}{\kern0pt}{\isadigit{1}}{\isacharparenright}{\kern0pt}\ {\isacharplus}{\kern0pt}\ {\isadigit{2}}{\isachardoublequoteclose}\ \isacommand{using}\isamarkupfalse%
\ assms\ \isacommand{by}\isamarkupfalse%
\ simp\isanewline
\ \ \isacommand{thus}\isamarkupfalse%
\ {\isacharquery}{\kern0pt}thesis\ \isacommand{using}\isamarkupfalse%
\ assms\ le{\isacharunderscore}{\kern0pt}ereal{\isacharunderscore}{\kern0pt}le\ int{\isacharunderscore}{\kern0pt}bit{\isacharunderscore}{\kern0pt}count\ \isacommand{by}\isamarkupfalse%
\ blast\isanewline
\isacommand{qed}\isamarkupfalse%
%
\endisatagproof
{\isafoldproof}%
%
\isadelimproof
%
\endisadelimproof
%
\begin{isamarkuptext}%
Encoding for Cartesian products%
\end{isamarkuptext}\isamarkuptrue%
\isacommand{fun}\isamarkupfalse%
\ encode{\isacharunderscore}{\kern0pt}prod\ {\isacharcolon}{\kern0pt}{\isacharcolon}{\kern0pt}\ {\isachardoublequoteopen}{\isacharprime}{\kern0pt}a\ encoding\ {\isasymRightarrow}\ {\isacharprime}{\kern0pt}b\ encoding\ {\isasymRightarrow}\ {\isacharparenleft}{\kern0pt}{\isacharprime}{\kern0pt}a\ {\isasymtimes}\ {\isacharprime}{\kern0pt}b{\isacharparenright}{\kern0pt}\ encoding{\isachardoublequoteclose}\ {\isacharparenleft}{\kern0pt}\isakeyword{infixr}\ {\isachardoublequoteopen}{\isasymtimes}\isactrlsub S{\isachardoublequoteclose}\ {\isadigit{6}}{\isadigit{5}}{\isacharparenright}{\kern0pt}\isanewline
\ \ \isakeyword{where}\ \isanewline
\ \ \ \ {\isachardoublequoteopen}encode{\isacharunderscore}{\kern0pt}prod\ e{\isadigit{1}}\ e{\isadigit{2}}\ x\ {\isacharequal}{\kern0pt}\ e{\isadigit{1}}\ {\isacharparenleft}{\kern0pt}fst\ x{\isacharparenright}{\kern0pt}{\isacharat}{\kern0pt}\isactrlsub S\ e{\isadigit{2}}\ {\isacharparenleft}{\kern0pt}snd\ x{\isacharparenright}{\kern0pt}{\isachardoublequoteclose}\isanewline
\isanewline
\isacommand{fun}\isamarkupfalse%
\ decode{\isacharunderscore}{\kern0pt}prod\ {\isacharcolon}{\kern0pt}{\isacharcolon}{\kern0pt}\ {\isachardoublequoteopen}{\isacharprime}{\kern0pt}a\ encoding\ {\isasymRightarrow}\ {\isacharprime}{\kern0pt}b\ encoding\ {\isasymRightarrow}\ bool\ list\ {\isasymRightarrow}\ {\isacharparenleft}{\kern0pt}{\isacharprime}{\kern0pt}a\ {\isasymtimes}\ {\isacharprime}{\kern0pt}b{\isacharparenright}{\kern0pt}\ {\isasymtimes}\ bool\ list{\isachardoublequoteclose}\isanewline
\ \ \isakeyword{where}\isanewline
\ \ \ \ {\isachardoublequoteopen}decode{\isacharunderscore}{\kern0pt}prod\ e{\isadigit{1}}\ e{\isadigit{2}}\ x{\isadigit{0}}\ {\isacharequal}{\kern0pt}\ {\isacharparenleft}{\kern0pt}\isanewline
\ \ \ \ \ \ let\ {\isacharparenleft}{\kern0pt}r{\isadigit{1}}{\isacharcomma}{\kern0pt}x{\isadigit{1}}{\isacharparenright}{\kern0pt}\ {\isacharequal}{\kern0pt}\ decode\ e{\isadigit{1}}\ x{\isadigit{0}}\ in\ {\isacharparenleft}{\kern0pt}\isanewline
\ \ \ \ \ \ \ \ let\ {\isacharparenleft}{\kern0pt}r{\isadigit{2}}{\isacharcomma}{\kern0pt}x{\isadigit{2}}{\isacharparenright}{\kern0pt}\ {\isacharequal}{\kern0pt}\ decode\ e{\isadigit{2}}\ x{\isadigit{1}}\ in\ {\isacharparenleft}{\kern0pt}{\isacharparenleft}{\kern0pt}r{\isadigit{1}}{\isacharcomma}{\kern0pt}r{\isadigit{2}}{\isacharparenright}{\kern0pt}{\isacharcomma}{\kern0pt}x{\isadigit{2}}{\isacharparenright}{\kern0pt}{\isacharparenright}{\kern0pt}{\isacharparenright}{\kern0pt}{\isachardoublequoteclose}\isanewline
\isanewline
\isacommand{lemma}\isamarkupfalse%
\ prod{\isacharunderscore}{\kern0pt}encoding{\isacharunderscore}{\kern0pt}dom{\isacharcolon}{\kern0pt}\isanewline
\ \ {\isachardoublequoteopen}x\ {\isasymin}\ dom\ {\isacharparenleft}{\kern0pt}e{\isadigit{1}}\ {\isasymtimes}\isactrlsub S\ e{\isadigit{2}}{\isacharparenright}{\kern0pt}\ {\isacharequal}{\kern0pt}\ {\isacharparenleft}{\kern0pt}fst\ x\ {\isasymin}\ dom\ e{\isadigit{1}}\ {\isasymand}\ snd\ x\ {\isasymin}\ dom\ e{\isadigit{2}}{\isacharparenright}{\kern0pt}{\isachardoublequoteclose}\isanewline
%
\isadelimproof
\ \ %
\endisadelimproof
%
\isatagproof
\isacommand{apply}\isamarkupfalse%
\ {\isacharparenleft}{\kern0pt}case{\isacharunderscore}{\kern0pt}tac\ {\isacharbrackleft}{\kern0pt}{\isacharbang}{\kern0pt}{\isacharbrackright}{\kern0pt}\ {\isachardoublequoteopen}e{\isadigit{1}}\ {\isacharparenleft}{\kern0pt}fst\ x{\isacharparenright}{\kern0pt}{\isachardoublequoteclose}{\isacharparenright}{\kern0pt}\isanewline
\ \ \ \isacommand{apply}\isamarkupfalse%
\ {\isacharparenleft}{\kern0pt}case{\isacharunderscore}{\kern0pt}tac\ {\isacharbrackleft}{\kern0pt}{\isacharbang}{\kern0pt}{\isacharbrackright}{\kern0pt}\ {\isachardoublequoteopen}e{\isadigit{2}}\ {\isacharparenleft}{\kern0pt}snd\ x{\isacharparenright}{\kern0pt}{\isachardoublequoteclose}{\isacharparenright}{\kern0pt}\isanewline
\ \ \isacommand{by}\isamarkupfalse%
\ {\isacharparenleft}{\kern0pt}simp\ add{\isacharcolon}{\kern0pt}dom{\isacharunderscore}{\kern0pt}def\ del{\isacharcolon}{\kern0pt}not{\isacharunderscore}{\kern0pt}None{\isacharunderscore}{\kern0pt}eq{\isacharparenright}{\kern0pt}{\isacharplus}{\kern0pt}%
\endisatagproof
{\isafoldproof}%
%
\isadelimproof
\isanewline
%
\endisadelimproof
\isanewline
\isacommand{lemma}\isamarkupfalse%
\ prod{\isacharunderscore}{\kern0pt}encoding{\isacharcolon}{\kern0pt}\isanewline
\ \ \isakeyword{assumes}\ {\isachardoublequoteopen}is{\isacharunderscore}{\kern0pt}encoding\ e{\isadigit{1}}{\isachardoublequoteclose}\isanewline
\ \ \isakeyword{assumes}\ {\isachardoublequoteopen}is{\isacharunderscore}{\kern0pt}encoding\ e{\isadigit{2}}{\isachardoublequoteclose}\isanewline
\ \ \isakeyword{shows}\ {\isachardoublequoteopen}is{\isacharunderscore}{\kern0pt}encoding\ {\isacharparenleft}{\kern0pt}encode{\isacharunderscore}{\kern0pt}prod\ e{\isadigit{1}}\ e{\isadigit{2}}{\isacharparenright}{\kern0pt}{\isachardoublequoteclose}\isanewline
%
\isadelimproof
%
\endisadelimproof
%
\isatagproof
\isacommand{proof}\isamarkupfalse%
\ \ {\isacharparenleft}{\kern0pt}rule\ encoding{\isacharunderscore}{\kern0pt}by{\isacharunderscore}{\kern0pt}witness{\isacharbrackleft}{\kern0pt}\isakeyword{where}\ g{\isacharequal}{\kern0pt}{\isachardoublequoteopen}decode{\isacharunderscore}{\kern0pt}prod\ e{\isadigit{1}}\ e{\isadigit{2}}{\isachardoublequoteclose}{\isacharbrackright}{\kern0pt}{\isacharparenright}{\kern0pt}\isanewline
\ \ \isacommand{fix}\isamarkupfalse%
\ x\ y\isanewline
\ \ \isacommand{assume}\isamarkupfalse%
\ a{\isacharcolon}{\kern0pt}{\isachardoublequoteopen}x\ {\isasymin}\ dom\ {\isacharparenleft}{\kern0pt}e{\isadigit{1}}\ {\isasymtimes}\isactrlsub S\ e{\isadigit{2}}{\isacharparenright}{\kern0pt}{\isachardoublequoteclose}\isanewline
\isanewline
\ \ \isacommand{have}\isamarkupfalse%
\ b{\isacharcolon}{\kern0pt}{\isachardoublequoteopen}e{\isadigit{1}}\ {\isacharparenleft}{\kern0pt}fst\ x{\isacharparenright}{\kern0pt}\ {\isacharequal}{\kern0pt}\ Some\ {\isacharparenleft}{\kern0pt}the\ {\isacharparenleft}{\kern0pt}e{\isadigit{1}}\ {\isacharparenleft}{\kern0pt}fst\ x{\isacharparenright}{\kern0pt}{\isacharparenright}{\kern0pt}{\isacharparenright}{\kern0pt}{\isachardoublequoteclose}\isanewline
\ \ \ \ \isacommand{by}\isamarkupfalse%
\ {\isacharparenleft}{\kern0pt}metis\ a\ prod{\isacharunderscore}{\kern0pt}encoding{\isacharunderscore}{\kern0pt}dom\ domIff\ option{\isachardot}{\kern0pt}exhaust{\isacharunderscore}{\kern0pt}sel{\isacharparenright}{\kern0pt}\isanewline
\ \ \isacommand{have}\isamarkupfalse%
\ c{\isacharcolon}{\kern0pt}{\isachardoublequoteopen}e{\isadigit{2}}\ {\isacharparenleft}{\kern0pt}snd\ x{\isacharparenright}{\kern0pt}\ {\isacharequal}{\kern0pt}\ Some\ {\isacharparenleft}{\kern0pt}the\ {\isacharparenleft}{\kern0pt}e{\isadigit{2}}\ {\isacharparenleft}{\kern0pt}snd\ x{\isacharparenright}{\kern0pt}{\isacharparenright}{\kern0pt}{\isacharparenright}{\kern0pt}{\isachardoublequoteclose}\ \isanewline
\ \ \ \ \isacommand{by}\isamarkupfalse%
\ {\isacharparenleft}{\kern0pt}metis\ a\ prod{\isacharunderscore}{\kern0pt}encoding{\isacharunderscore}{\kern0pt}dom\ domIff\ option{\isachardot}{\kern0pt}exhaust{\isacharunderscore}{\kern0pt}sel{\isacharparenright}{\kern0pt}\isanewline
\isanewline
\ \ \isacommand{show}\isamarkupfalse%
\ {\isachardoublequoteopen}decode{\isacharunderscore}{\kern0pt}prod\ e{\isadigit{1}}\ e{\isadigit{2}}\ {\isacharparenleft}{\kern0pt}the\ {\isacharparenleft}{\kern0pt}{\isacharparenleft}{\kern0pt}e{\isadigit{1}}\ {\isasymtimes}\isactrlsub S\ e{\isadigit{2}}{\isacharparenright}{\kern0pt}\ x{\isacharparenright}{\kern0pt}\ {\isacharat}{\kern0pt}\ y{\isacharparenright}{\kern0pt}\ {\isacharequal}{\kern0pt}\ {\isacharparenleft}{\kern0pt}x{\isacharcomma}{\kern0pt}\ y{\isacharparenright}{\kern0pt}{\isachardoublequoteclose}\isanewline
\ \ \ \ \isacommand{apply}\isamarkupfalse%
\ {\isacharparenleft}{\kern0pt}simp{\isacharcomma}{\kern0pt}\ subst\ b{\isacharcomma}{\kern0pt}\ subst\ c{\isacharparenright}{\kern0pt}\isanewline
\ \ \ \ \isacommand{apply}\isamarkupfalse%
\ simp\isanewline
\ \ \ \ \isacommand{using}\isamarkupfalse%
\ assms\ b\ c\ \isacommand{by}\isamarkupfalse%
\ {\isacharparenleft}{\kern0pt}simp\ add{\isacharcolon}{\kern0pt}decode{\isacharunderscore}{\kern0pt}elim{\isacharparenright}{\kern0pt}\isanewline
\isacommand{qed}\isamarkupfalse%
%
\endisatagproof
{\isafoldproof}%
%
\isadelimproof
\isanewline
%
\endisadelimproof
\isanewline
\isacommand{lemma}\isamarkupfalse%
\ prod{\isacharunderscore}{\kern0pt}bit{\isacharunderscore}{\kern0pt}count{\isacharcolon}{\kern0pt}\isanewline
\ \ {\isachardoublequoteopen}bit{\isacharunderscore}{\kern0pt}count\ {\isacharparenleft}{\kern0pt}{\isacharparenleft}{\kern0pt}e\isactrlsub {\isadigit{1}}\ {\isasymtimes}\isactrlsub S\ e\isactrlsub {\isadigit{2}}{\isacharparenright}{\kern0pt}\ {\isacharparenleft}{\kern0pt}x\isactrlsub {\isadigit{1}}{\isacharcomma}{\kern0pt}x\isactrlsub {\isadigit{2}}{\isacharparenright}{\kern0pt}{\isacharparenright}{\kern0pt}\ {\isacharequal}{\kern0pt}\ bit{\isacharunderscore}{\kern0pt}count\ {\isacharparenleft}{\kern0pt}e\isactrlsub {\isadigit{1}}\ x\isactrlsub {\isadigit{1}}{\isacharparenright}{\kern0pt}\ {\isacharplus}{\kern0pt}\ bit{\isacharunderscore}{\kern0pt}count\ {\isacharparenleft}{\kern0pt}e\isactrlsub {\isadigit{2}}\ x\isactrlsub {\isadigit{2}}{\isacharparenright}{\kern0pt}{\isachardoublequoteclose}\isanewline
%
\isadelimproof
\ \ %
\endisadelimproof
%
\isatagproof
\isacommand{by}\isamarkupfalse%
\ {\isacharparenleft}{\kern0pt}simp\ add{\isacharcolon}{\kern0pt}bit{\isacharunderscore}{\kern0pt}count{\isacharunderscore}{\kern0pt}append{\isacharparenright}{\kern0pt}%
\endisatagproof
{\isafoldproof}%
%
\isadelimproof
\isanewline
%
\endisadelimproof
\isanewline
\isacommand{lemma}\isamarkupfalse%
\ prod{\isacharunderscore}{\kern0pt}bit{\isacharunderscore}{\kern0pt}count{\isacharunderscore}{\kern0pt}{\isadigit{2}}{\isacharcolon}{\kern0pt}\isanewline
\ \ {\isachardoublequoteopen}bit{\isacharunderscore}{\kern0pt}count\ {\isacharparenleft}{\kern0pt}{\isacharparenleft}{\kern0pt}e{\isadigit{1}}\ {\isasymtimes}\isactrlsub S\ e{\isadigit{2}}{\isacharparenright}{\kern0pt}\ x{\isacharparenright}{\kern0pt}\ {\isacharequal}{\kern0pt}\ bit{\isacharunderscore}{\kern0pt}count\ {\isacharparenleft}{\kern0pt}e{\isadigit{1}}\ {\isacharparenleft}{\kern0pt}fst\ x{\isacharparenright}{\kern0pt}{\isacharparenright}{\kern0pt}\ {\isacharplus}{\kern0pt}\ bit{\isacharunderscore}{\kern0pt}count\ {\isacharparenleft}{\kern0pt}e{\isadigit{2}}\ {\isacharparenleft}{\kern0pt}snd\ x{\isacharparenright}{\kern0pt}{\isacharparenright}{\kern0pt}{\isachardoublequoteclose}\isanewline
%
\isadelimproof
\ \ %
\endisadelimproof
%
\isatagproof
\isacommand{by}\isamarkupfalse%
\ {\isacharparenleft}{\kern0pt}simp\ add{\isacharcolon}{\kern0pt}bit{\isacharunderscore}{\kern0pt}count{\isacharunderscore}{\kern0pt}append{\isacharparenright}{\kern0pt}%
\endisatagproof
{\isafoldproof}%
%
\isadelimproof
%
\endisadelimproof
%
\begin{isamarkuptext}%
Encoding for dependent sums%
\end{isamarkuptext}\isamarkuptrue%
\isacommand{fun}\isamarkupfalse%
\ encode{\isacharunderscore}{\kern0pt}dependent{\isacharunderscore}{\kern0pt}sum\ {\isacharcolon}{\kern0pt}{\isacharcolon}{\kern0pt}\ {\isachardoublequoteopen}{\isacharprime}{\kern0pt}a\ encoding\ {\isasymRightarrow}\ {\isacharparenleft}{\kern0pt}{\isacharprime}{\kern0pt}a\ {\isasymRightarrow}\ {\isacharprime}{\kern0pt}b\ encoding{\isacharparenright}{\kern0pt}\ {\isasymRightarrow}\ {\isacharparenleft}{\kern0pt}{\isacharprime}{\kern0pt}a\ {\isasymtimes}\ {\isacharprime}{\kern0pt}b{\isacharparenright}{\kern0pt}\ encoding{\isachardoublequoteclose}\ {\isacharparenleft}{\kern0pt}\isakeyword{infixr}\ {\isachardoublequoteopen}{\isasymtimes}\isactrlsub D{\isachardoublequoteclose}\ {\isadigit{6}}{\isadigit{5}}{\isacharparenright}{\kern0pt}\isanewline
\ \ \isakeyword{where}\ \isanewline
\ \ \ \ {\isachardoublequoteopen}encode{\isacharunderscore}{\kern0pt}dependent{\isacharunderscore}{\kern0pt}sum\ e{\isadigit{1}}\ e{\isadigit{2}}\ x\ {\isacharequal}{\kern0pt}\ e{\isadigit{1}}\ {\isacharparenleft}{\kern0pt}fst\ x{\isacharparenright}{\kern0pt}{\isacharat}{\kern0pt}\isactrlsub S\ e{\isadigit{2}}\ {\isacharparenleft}{\kern0pt}fst\ x{\isacharparenright}{\kern0pt}\ {\isacharparenleft}{\kern0pt}snd\ x{\isacharparenright}{\kern0pt}{\isachardoublequoteclose}\isanewline
\isanewline
\isacommand{lemma}\isamarkupfalse%
\ dependent{\isacharunderscore}{\kern0pt}encoding{\isacharcolon}{\kern0pt}\isanewline
\ \ \isakeyword{assumes}\ {\isachardoublequoteopen}is{\isacharunderscore}{\kern0pt}encoding\ e{\isadigit{1}}{\isachardoublequoteclose}\isanewline
\ \ \isakeyword{assumes}\ {\isachardoublequoteopen}{\isasymAnd}x{\isachardot}{\kern0pt}\ is{\isacharunderscore}{\kern0pt}encoding\ {\isacharparenleft}{\kern0pt}e{\isadigit{2}}\ x{\isacharparenright}{\kern0pt}{\isachardoublequoteclose}\isanewline
\ \ \isakeyword{shows}\ {\isachardoublequoteopen}is{\isacharunderscore}{\kern0pt}encoding\ {\isacharparenleft}{\kern0pt}encode{\isacharunderscore}{\kern0pt}dependent{\isacharunderscore}{\kern0pt}sum\ e{\isadigit{1}}\ e{\isadigit{2}}{\isacharparenright}{\kern0pt}{\isachardoublequoteclose}\isanewline
%
\isadelimproof
%
\endisadelimproof
%
\isatagproof
\isacommand{proof}\isamarkupfalse%
\ {\isacharminus}{\kern0pt}\isanewline
\ \ \isacommand{define}\isamarkupfalse%
\ d\ \isakeyword{where}\ {\isachardoublequoteopen}d\ {\isacharequal}{\kern0pt}\ {\isacharparenleft}{\kern0pt}{\isasymlambda}x{\isadigit{0}}{\isachardot}{\kern0pt}\ \isanewline
\ \ \ \ let\ {\isacharparenleft}{\kern0pt}r{\isadigit{1}}{\isacharcomma}{\kern0pt}\ x{\isadigit{1}}{\isacharparenright}{\kern0pt}\ {\isacharequal}{\kern0pt}\ decode\ e{\isadigit{1}}\ x{\isadigit{0}}\ in\ \isanewline
\ \ \ \ let\ {\isacharparenleft}{\kern0pt}r{\isadigit{2}}{\isacharcomma}{\kern0pt}\ x{\isadigit{2}}{\isacharparenright}{\kern0pt}\ {\isacharequal}{\kern0pt}\ decode\ {\isacharparenleft}{\kern0pt}e{\isadigit{2}}\ r{\isadigit{1}}{\isacharparenright}{\kern0pt}\ x{\isadigit{1}}\ in\ {\isacharparenleft}{\kern0pt}{\isacharparenleft}{\kern0pt}r{\isadigit{1}}{\isacharcomma}{\kern0pt}r{\isadigit{2}}{\isacharparenright}{\kern0pt}{\isacharcomma}{\kern0pt}\ x{\isadigit{2}}{\isacharparenright}{\kern0pt}{\isacharparenright}{\kern0pt}{\isachardoublequoteclose}\isanewline
\isanewline
\ \ \isacommand{have}\isamarkupfalse%
\ a{\isacharcolon}{\kern0pt}\ {\isachardoublequoteopen}{\isasymAnd}x{\isachardot}{\kern0pt}\ x\ {\isasymin}\ dom\ {\isacharparenleft}{\kern0pt}e{\isadigit{1}}\ {\isasymtimes}\isactrlsub D\ e{\isadigit{2}}{\isacharparenright}{\kern0pt}\ {\isasymLongrightarrow}\ fst\ x\ {\isasymin}\ dom\ e{\isadigit{1}}{\isachardoublequoteclose}\isanewline
\ \ \ \ \isacommand{apply}\isamarkupfalse%
\ {\isacharparenleft}{\kern0pt}simp\ add{\isacharcolon}{\kern0pt}dom{\isacharunderscore}{\kern0pt}def\ del{\isacharcolon}{\kern0pt}not{\isacharunderscore}{\kern0pt}None{\isacharunderscore}{\kern0pt}eq{\isacharparenright}{\kern0pt}\ \isanewline
\ \ \ \ \isacommand{using}\isamarkupfalse%
\ append{\isacharunderscore}{\kern0pt}encoding{\isachardot}{\kern0pt}simps\ \isacommand{by}\isamarkupfalse%
\ metis\isanewline
\ \ \isacommand{have}\isamarkupfalse%
\ b{\isacharcolon}{\kern0pt}\ {\isachardoublequoteopen}{\isasymAnd}x{\isachardot}{\kern0pt}\ x\ {\isasymin}\ dom\ {\isacharparenleft}{\kern0pt}e{\isadigit{1}}\ {\isasymtimes}\isactrlsub D\ e{\isadigit{2}}{\isacharparenright}{\kern0pt}\ {\isasymLongrightarrow}\ snd\ x\ {\isasymin}\ dom\ {\isacharparenleft}{\kern0pt}e{\isadigit{2}}\ {\isacharparenleft}{\kern0pt}fst\ x{\isacharparenright}{\kern0pt}{\isacharparenright}{\kern0pt}{\isachardoublequoteclose}\isanewline
\ \ \ \ \isacommand{apply}\isamarkupfalse%
\ {\isacharparenleft}{\kern0pt}simp\ add{\isacharcolon}{\kern0pt}dom{\isacharunderscore}{\kern0pt}def\ del{\isacharcolon}{\kern0pt}not{\isacharunderscore}{\kern0pt}None{\isacharunderscore}{\kern0pt}eq{\isacharparenright}{\kern0pt}\ \isanewline
\ \ \ \ \isacommand{using}\isamarkupfalse%
\ append{\isacharunderscore}{\kern0pt}encoding{\isachardot}{\kern0pt}simps\ \isacommand{by}\isamarkupfalse%
\ metis\isanewline
\ \ \isacommand{have}\isamarkupfalse%
\ c{\isacharcolon}{\kern0pt}\ {\isachardoublequoteopen}{\isasymAnd}x{\isachardot}{\kern0pt}\ x\ {\isasymin}\ dom\ {\isacharparenleft}{\kern0pt}e{\isadigit{1}}\ {\isasymtimes}\isactrlsub D\ e{\isadigit{2}}{\isacharparenright}{\kern0pt}\ {\isasymLongrightarrow}\ e{\isadigit{1}}\ {\isacharparenleft}{\kern0pt}fst\ x{\isacharparenright}{\kern0pt}\ {\isacharequal}{\kern0pt}\ Some\ {\isacharparenleft}{\kern0pt}the\ {\isacharparenleft}{\kern0pt}e{\isadigit{1}}\ {\isacharparenleft}{\kern0pt}fst\ x{\isacharparenright}{\kern0pt}{\isacharparenright}{\kern0pt}{\isacharparenright}{\kern0pt}{\isachardoublequoteclose}\isanewline
\ \ \ \ \isacommand{using}\isamarkupfalse%
\ a\ \isacommand{by}\isamarkupfalse%
\ {\isacharparenleft}{\kern0pt}simp\ add{\isacharcolon}{\kern0pt}\ domIff{\isacharparenright}{\kern0pt}\isanewline
\ \ \isacommand{have}\isamarkupfalse%
\ d{\isacharcolon}{\kern0pt}\ {\isachardoublequoteopen}{\isasymAnd}x{\isachardot}{\kern0pt}\ x\ {\isasymin}\ dom\ {\isacharparenleft}{\kern0pt}e{\isadigit{1}}\ {\isasymtimes}\isactrlsub D\ e{\isadigit{2}}{\isacharparenright}{\kern0pt}\ {\isasymLongrightarrow}\ e{\isadigit{2}}\ {\isacharparenleft}{\kern0pt}fst\ x{\isacharparenright}{\kern0pt}\ {\isacharparenleft}{\kern0pt}snd\ x{\isacharparenright}{\kern0pt}\ {\isacharequal}{\kern0pt}\ Some\ {\isacharparenleft}{\kern0pt}the\ {\isacharparenleft}{\kern0pt}e{\isadigit{2}}\ {\isacharparenleft}{\kern0pt}fst\ x{\isacharparenright}{\kern0pt}\ {\isacharparenleft}{\kern0pt}snd\ x{\isacharparenright}{\kern0pt}{\isacharparenright}{\kern0pt}{\isacharparenright}{\kern0pt}{\isachardoublequoteclose}\isanewline
\ \ \ \ \isacommand{using}\isamarkupfalse%
\ b\ \isacommand{by}\isamarkupfalse%
\ {\isacharparenleft}{\kern0pt}simp\ add{\isacharcolon}{\kern0pt}\ domIff{\isacharparenright}{\kern0pt}\isanewline
\ \ \isacommand{show}\isamarkupfalse%
\ {\isacharquery}{\kern0pt}thesis\isanewline
\ \ \ \ \isacommand{apply}\isamarkupfalse%
\ {\isacharparenleft}{\kern0pt}rule\ encoding{\isacharunderscore}{\kern0pt}by{\isacharunderscore}{\kern0pt}witness{\isacharbrackleft}{\kern0pt}\isakeyword{where}\ g{\isacharequal}{\kern0pt}{\isachardoublequoteopen}d{\isachardoublequoteclose}{\isacharbrackright}{\kern0pt}{\isacharparenright}{\kern0pt}\isanewline
\ \ \ \ \isacommand{apply}\isamarkupfalse%
\ {\isacharparenleft}{\kern0pt}simp\ add{\isacharcolon}{\kern0pt}d{\isacharunderscore}{\kern0pt}def{\isacharcomma}{\kern0pt}\ subst\ c{\isacharcomma}{\kern0pt}\ simp{\isacharcomma}{\kern0pt}\ subst\ d{\isacharcomma}{\kern0pt}\ simp{\isacharparenright}{\kern0pt}\isanewline
\ \ \ \ \isacommand{using}\isamarkupfalse%
\ assms\ a\ b\ \isacommand{by}\isamarkupfalse%
\ {\isacharparenleft}{\kern0pt}simp\ add{\isacharcolon}{\kern0pt}d{\isacharunderscore}{\kern0pt}def\ decode{\isacharunderscore}{\kern0pt}elim{\isacharunderscore}{\kern0pt}{\isadigit{2}}{\isacharparenright}{\kern0pt}\isanewline
\isacommand{qed}\isamarkupfalse%
%
\endisatagproof
{\isafoldproof}%
%
\isadelimproof
\isanewline
%
\endisadelimproof
\isanewline
\isacommand{lemma}\isamarkupfalse%
\ dependent{\isacharunderscore}{\kern0pt}bit{\isacharunderscore}{\kern0pt}count{\isacharcolon}{\kern0pt}\isanewline
\ \ {\isachardoublequoteopen}bit{\isacharunderscore}{\kern0pt}count\ {\isacharparenleft}{\kern0pt}{\isacharparenleft}{\kern0pt}e\isactrlsub {\isadigit{1}}\ {\isasymtimes}\isactrlsub D\ e\isactrlsub {\isadigit{2}}{\isacharparenright}{\kern0pt}\ {\isacharparenleft}{\kern0pt}x\isactrlsub {\isadigit{1}}{\isacharcomma}{\kern0pt}x\isactrlsub {\isadigit{2}}{\isacharparenright}{\kern0pt}{\isacharparenright}{\kern0pt}\ {\isacharequal}{\kern0pt}\ bit{\isacharunderscore}{\kern0pt}count\ {\isacharparenleft}{\kern0pt}e\isactrlsub {\isadigit{1}}\ x\isactrlsub {\isadigit{1}}{\isacharparenright}{\kern0pt}\ {\isacharplus}{\kern0pt}\ bit{\isacharunderscore}{\kern0pt}count\ {\isacharparenleft}{\kern0pt}e\isactrlsub {\isadigit{2}}\ x\isactrlsub {\isadigit{1}}\ x\isactrlsub {\isadigit{2}}{\isacharparenright}{\kern0pt}{\isachardoublequoteclose}\isanewline
%
\isadelimproof
\ \ %
\endisadelimproof
%
\isatagproof
\isacommand{by}\isamarkupfalse%
\ {\isacharparenleft}{\kern0pt}simp\ add{\isacharcolon}{\kern0pt}bit{\isacharunderscore}{\kern0pt}count{\isacharunderscore}{\kern0pt}append{\isacharparenright}{\kern0pt}%
\endisatagproof
{\isafoldproof}%
%
\isadelimproof
%
\endisadelimproof
%
\begin{isamarkuptext}%
This lemma helps derive an encoding on the domain of an injective function using an 
existing encoding on its image.%
\end{isamarkuptext}\isamarkuptrue%
\isacommand{lemma}\isamarkupfalse%
\ encoding{\isacharunderscore}{\kern0pt}compose{\isacharcolon}{\kern0pt}\isanewline
\ \ \isakeyword{assumes}\ {\isachardoublequoteopen}is{\isacharunderscore}{\kern0pt}encoding\ f{\isachardoublequoteclose}\isanewline
\ \ \isakeyword{assumes}\ {\isachardoublequoteopen}inj{\isacharunderscore}{\kern0pt}on\ g\ {\isacharbraceleft}{\kern0pt}x{\isachardot}{\kern0pt}\ P\ x{\isacharbraceright}{\kern0pt}{\isachardoublequoteclose}\isanewline
\ \ \isakeyword{shows}\ {\isachardoublequoteopen}is{\isacharunderscore}{\kern0pt}encoding\ {\isacharparenleft}{\kern0pt}{\isasymlambda}x{\isachardot}{\kern0pt}\ if\ P\ x\ then\ f\ {\isacharparenleft}{\kern0pt}g\ x{\isacharparenright}{\kern0pt}\ else\ None{\isacharparenright}{\kern0pt}{\isachardoublequoteclose}\isanewline
%
\isadelimproof
\ \ %
\endisadelimproof
%
\isatagproof
\isacommand{using}\isamarkupfalse%
\ assms\ \isacommand{by}\isamarkupfalse%
\ {\isacharparenleft}{\kern0pt}simp\ add{\isacharcolon}{\kern0pt}\ inj{\isacharunderscore}{\kern0pt}onD\ is{\isacharunderscore}{\kern0pt}encoding{\isacharunderscore}{\kern0pt}def{\isacharparenright}{\kern0pt}%
\endisatagproof
{\isafoldproof}%
%
\isadelimproof
%
\endisadelimproof
%
\begin{isamarkuptext}%
Encoding for extensional maps defined on an enumerable set.%
\end{isamarkuptext}\isamarkuptrue%
\isacommand{definition}\isamarkupfalse%
\ fun\isactrlsub S\ {\isacharcolon}{\kern0pt}{\isacharcolon}{\kern0pt}\ {\isachardoublequoteopen}{\isacharprime}{\kern0pt}a\ list\ {\isasymRightarrow}\ {\isacharprime}{\kern0pt}b\ encoding\ {\isasymRightarrow}\ {\isacharparenleft}{\kern0pt}{\isacharprime}{\kern0pt}a\ {\isasymRightarrow}\ {\isacharprime}{\kern0pt}b{\isacharparenright}{\kern0pt}\ encoding{\isachardoublequoteclose}\ \ {\isacharparenleft}{\kern0pt}\isakeyword{infixr}\ {\isachardoublequoteopen}{\isasymrightarrow}\isactrlsub S{\isachardoublequoteclose}\ {\isadigit{6}}{\isadigit{5}}{\isacharparenright}{\kern0pt}\ \ \isakeyword{where}\isanewline
\ \ {\isachardoublequoteopen}fun\isactrlsub S\ xs\ e\ f\ {\isacharequal}{\kern0pt}\ {\isacharparenleft}{\kern0pt}\isanewline
\ \ \ \ if\ f\ {\isasymin}\ extensional\ {\isacharparenleft}{\kern0pt}set\ xs{\isacharparenright}{\kern0pt}\ then\ \isanewline
\ \ \ \ \ \ list\isactrlsub S\ e\ {\isacharparenleft}{\kern0pt}map\ f\ xs{\isacharparenright}{\kern0pt}\isanewline
\ \ \ \ else\isanewline
\ \ \ \ \ \ None{\isacharparenright}{\kern0pt}{\isachardoublequoteclose}\isanewline
\isanewline
\isacommand{lemma}\isamarkupfalse%
\ encode{\isacharunderscore}{\kern0pt}extensional{\isacharcolon}{\kern0pt}\isanewline
\ \ \isakeyword{assumes}\ {\isachardoublequoteopen}is{\isacharunderscore}{\kern0pt}encoding\ e{\isachardoublequoteclose}\isanewline
\ \ \isakeyword{shows}\ {\isachardoublequoteopen}is{\isacharunderscore}{\kern0pt}encoding\ {\isacharparenleft}{\kern0pt}{\isasymlambda}x{\isachardot}{\kern0pt}\ {\isacharparenleft}{\kern0pt}xs\ {\isasymrightarrow}\isactrlsub S\ e{\isacharparenright}{\kern0pt}\ x{\isacharparenright}{\kern0pt}{\isachardoublequoteclose}\isanewline
%
\isadelimproof
\ \ %
\endisadelimproof
%
\isatagproof
\isacommand{apply}\isamarkupfalse%
\ {\isacharparenleft}{\kern0pt}simp\ add{\isacharcolon}{\kern0pt}fun\isactrlsub S{\isacharunderscore}{\kern0pt}def{\isacharparenright}{\kern0pt}\isanewline
\ \ \isacommand{apply}\isamarkupfalse%
\ {\isacharparenleft}{\kern0pt}rule\ encoding{\isacharunderscore}{\kern0pt}compose{\isacharbrackleft}{\kern0pt}\isakeyword{where}\ f{\isacharequal}{\kern0pt}{\isachardoublequoteopen}list\isactrlsub S\ e{\isachardoublequoteclose}{\isacharbrackright}{\kern0pt}{\isacharparenright}{\kern0pt}\isanewline
\ \ \ \isacommand{apply}\isamarkupfalse%
\ {\isacharparenleft}{\kern0pt}metis\ list{\isacharunderscore}{\kern0pt}encoding\ assms{\isacharparenright}{\kern0pt}\isanewline
\ \ \isacommand{apply}\isamarkupfalse%
\ {\isacharparenleft}{\kern0pt}rule\ inj{\isacharunderscore}{\kern0pt}onI{\isacharcomma}{\kern0pt}\ simp{\isacharparenright}{\kern0pt}\isanewline
\ \ \isacommand{using}\isamarkupfalse%
\ extensionalityI\ \isacommand{by}\isamarkupfalse%
\ fastforce%
\endisatagproof
{\isafoldproof}%
%
\isadelimproof
\isanewline
%
\endisadelimproof
\isanewline
\isacommand{lemma}\isamarkupfalse%
\ extensional{\isacharunderscore}{\kern0pt}bit{\isacharunderscore}{\kern0pt}count{\isacharcolon}{\kern0pt}\isanewline
\ \ \isakeyword{assumes}\ {\isachardoublequoteopen}f\ {\isasymin}\ extensional\ {\isacharparenleft}{\kern0pt}set\ xs{\isacharparenright}{\kern0pt}{\isachardoublequoteclose}\isanewline
\ \ \isakeyword{shows}\ {\isachardoublequoteopen}bit{\isacharunderscore}{\kern0pt}count\ {\isacharparenleft}{\kern0pt}{\isacharparenleft}{\kern0pt}xs\ {\isasymrightarrow}\isactrlsub S\ e{\isacharparenright}{\kern0pt}\ f{\isacharparenright}{\kern0pt}\ {\isacharequal}{\kern0pt}\ {\isacharparenleft}{\kern0pt}{\isasymSum}x\ {\isasymleftarrow}\ xs{\isachardot}{\kern0pt}\ bit{\isacharunderscore}{\kern0pt}count\ {\isacharparenleft}{\kern0pt}e\ {\isacharparenleft}{\kern0pt}f\ x{\isacharparenright}{\kern0pt}{\isacharparenright}{\kern0pt}\ {\isacharplus}{\kern0pt}\ {\isadigit{1}}{\isacharparenright}{\kern0pt}\ {\isacharplus}{\kern0pt}\ {\isadigit{1}}{\isachardoublequoteclose}\isanewline
%
\isadelimproof
\ \ %
\endisadelimproof
%
\isatagproof
\isacommand{using}\isamarkupfalse%
\ assms\ \isanewline
\ \ \isacommand{by}\isamarkupfalse%
\ {\isacharparenleft}{\kern0pt}simp\ add{\isacharcolon}{\kern0pt}fun\isactrlsub S{\isacharunderscore}{\kern0pt}def\ list{\isacharunderscore}{\kern0pt}bit{\isacharunderscore}{\kern0pt}count\ comp{\isacharunderscore}{\kern0pt}def{\isacharparenright}{\kern0pt}%
\endisatagproof
{\isafoldproof}%
%
\isadelimproof
%
\endisadelimproof
%
\begin{isamarkuptext}%
Encoding for ordered sets.%
\end{isamarkuptext}\isamarkuptrue%
\isacommand{fun}\isamarkupfalse%
\ set\isactrlsub S\ \isakeyword{where}\ {\isachardoublequoteopen}set\isactrlsub S\ e\ S\ {\isacharequal}{\kern0pt}\ {\isacharparenleft}{\kern0pt}if\ finite\ S\ then\ list\isactrlsub S\ e\ {\isacharparenleft}{\kern0pt}sorted{\isacharunderscore}{\kern0pt}list{\isacharunderscore}{\kern0pt}of{\isacharunderscore}{\kern0pt}set\ S{\isacharparenright}{\kern0pt}\ else\ None{\isacharparenright}{\kern0pt}{\isachardoublequoteclose}\isanewline
\isanewline
\isacommand{lemma}\isamarkupfalse%
\ encode{\isacharunderscore}{\kern0pt}set{\isacharcolon}{\kern0pt}\isanewline
\ \ \isakeyword{assumes}\ {\isachardoublequoteopen}is{\isacharunderscore}{\kern0pt}encoding\ e{\isachardoublequoteclose}\isanewline
\ \ \isakeyword{shows}\ {\isachardoublequoteopen}is{\isacharunderscore}{\kern0pt}encoding\ {\isacharparenleft}{\kern0pt}{\isasymlambda}S{\isachardot}{\kern0pt}\ set\isactrlsub S\ e\ S{\isacharparenright}{\kern0pt}{\isachardoublequoteclose}\isanewline
%
\isadelimproof
\ \ %
\endisadelimproof
%
\isatagproof
\isacommand{apply}\isamarkupfalse%
\ simp\isanewline
\ \ \isacommand{apply}\isamarkupfalse%
\ {\isacharparenleft}{\kern0pt}rule\ encoding{\isacharunderscore}{\kern0pt}compose{\isacharbrackleft}{\kern0pt}\isakeyword{where}\ f{\isacharequal}{\kern0pt}{\isachardoublequoteopen}list\isactrlsub S\ e{\isachardoublequoteclose}{\isacharbrackright}{\kern0pt}{\isacharparenright}{\kern0pt}\isanewline
\ \ \ \isacommand{apply}\isamarkupfalse%
\ {\isacharparenleft}{\kern0pt}metis\ assms\ list{\isacharunderscore}{\kern0pt}encoding{\isacharparenright}{\kern0pt}\isanewline
\ \ \isacommand{apply}\isamarkupfalse%
\ {\isacharparenleft}{\kern0pt}rule\ inj{\isacharunderscore}{\kern0pt}onI{\isacharcomma}{\kern0pt}\ simp{\isacharparenright}{\kern0pt}\isanewline
\ \ \isacommand{by}\isamarkupfalse%
\ {\isacharparenleft}{\kern0pt}metis\ sorted{\isacharunderscore}{\kern0pt}list{\isacharunderscore}{\kern0pt}of{\isacharunderscore}{\kern0pt}set{\isachardot}{\kern0pt}set{\isacharunderscore}{\kern0pt}sorted{\isacharunderscore}{\kern0pt}key{\isacharunderscore}{\kern0pt}list{\isacharunderscore}{\kern0pt}of{\isacharunderscore}{\kern0pt}set{\isacharparenright}{\kern0pt}%
\endisatagproof
{\isafoldproof}%
%
\isadelimproof
\isanewline
%
\endisadelimproof
\isanewline
\isacommand{lemma}\isamarkupfalse%
\ set{\isacharunderscore}{\kern0pt}bit{\isacharunderscore}{\kern0pt}count{\isacharcolon}{\kern0pt}\isanewline
\ \ \isakeyword{assumes}\ {\isachardoublequoteopen}finite\ S{\isachardoublequoteclose}\isanewline
\ \ \isakeyword{shows}\ {\isachardoublequoteopen}bit{\isacharunderscore}{\kern0pt}count\ {\isacharparenleft}{\kern0pt}set\isactrlsub S\ e\ S{\isacharparenright}{\kern0pt}\ {\isacharequal}{\kern0pt}\ {\isacharparenleft}{\kern0pt}{\isasymSum}x\ {\isasymin}\ S{\isachardot}{\kern0pt}\ bit{\isacharunderscore}{\kern0pt}count\ {\isacharparenleft}{\kern0pt}e\ x{\isacharparenright}{\kern0pt}{\isacharplus}{\kern0pt}{\isadigit{1}}{\isacharparenright}{\kern0pt}{\isacharplus}{\kern0pt}{\isadigit{1}}{\isachardoublequoteclose}\isanewline
%
\isadelimproof
\ \ %
\endisadelimproof
%
\isatagproof
\isacommand{using}\isamarkupfalse%
\ assms\ sorted{\isacharunderscore}{\kern0pt}list{\isacharunderscore}{\kern0pt}of{\isacharunderscore}{\kern0pt}set\isanewline
\ \ \isacommand{by}\isamarkupfalse%
\ {\isacharparenleft}{\kern0pt}simp\ add{\isacharcolon}{\kern0pt}list{\isacharunderscore}{\kern0pt}bit{\isacharunderscore}{\kern0pt}count\ sum{\isacharunderscore}{\kern0pt}list{\isacharunderscore}{\kern0pt}distinct{\isacharunderscore}{\kern0pt}conv{\isacharunderscore}{\kern0pt}sum{\isacharunderscore}{\kern0pt}set{\isacharparenright}{\kern0pt}%
\endisatagproof
{\isafoldproof}%
%
\isadelimproof
\isanewline
%
\endisadelimproof
\isanewline
\isacommand{lemma}\isamarkupfalse%
\ set{\isacharunderscore}{\kern0pt}bit{\isacharunderscore}{\kern0pt}count{\isacharunderscore}{\kern0pt}est{\isacharcolon}{\kern0pt}\isanewline
\ \ \isakeyword{assumes}\ {\isachardoublequoteopen}finite\ S{\isachardoublequoteclose}\isanewline
\ \ \isakeyword{assumes}\ {\isachardoublequoteopen}card\ S\ {\isasymle}\ m{\isachardoublequoteclose}\isanewline
\ \ \isakeyword{assumes}\ {\isachardoublequoteopen}{\isadigit{0}}\ {\isasymle}\ a{\isachardoublequoteclose}\isanewline
\ \ \isakeyword{assumes}\ {\isachardoublequoteopen}{\isasymAnd}x{\isachardot}{\kern0pt}\ x\ {\isasymin}\ S\ {\isasymLongrightarrow}\ bit{\isacharunderscore}{\kern0pt}count\ {\isacharparenleft}{\kern0pt}f\ x{\isacharparenright}{\kern0pt}\ {\isasymle}\ a{\isachardoublequoteclose}\isanewline
\ \ \isakeyword{shows}\ {\isachardoublequoteopen}bit{\isacharunderscore}{\kern0pt}count\ {\isacharparenleft}{\kern0pt}set\isactrlsub S\ f\ S{\isacharparenright}{\kern0pt}\ {\isasymle}\ ereal\ {\isacharparenleft}{\kern0pt}real\ m{\isacharparenright}{\kern0pt}\ {\isacharasterisk}{\kern0pt}\ {\isacharparenleft}{\kern0pt}a{\isacharplus}{\kern0pt}{\isadigit{1}}{\isacharparenright}{\kern0pt}\ {\isacharplus}{\kern0pt}\ {\isadigit{1}}{\isachardoublequoteclose}\isanewline
%
\isadelimproof
%
\endisadelimproof
%
\isatagproof
\isacommand{proof}\isamarkupfalse%
\ {\isacharminus}{\kern0pt}\isanewline
\ \ \isacommand{have}\isamarkupfalse%
\ {\isachardoublequoteopen}bit{\isacharunderscore}{\kern0pt}count\ {\isacharparenleft}{\kern0pt}set\isactrlsub S\ f\ S{\isacharparenright}{\kern0pt}\ {\isasymle}\ ereal\ {\isacharparenleft}{\kern0pt}length\ {\isacharparenleft}{\kern0pt}sorted{\isacharunderscore}{\kern0pt}list{\isacharunderscore}{\kern0pt}of{\isacharunderscore}{\kern0pt}set\ S{\isacharparenright}{\kern0pt}{\isacharparenright}{\kern0pt}\ {\isacharasterisk}{\kern0pt}\ {\isacharparenleft}{\kern0pt}a{\isacharplus}{\kern0pt}{\isadigit{1}}{\isacharparenright}{\kern0pt}\ {\isacharplus}{\kern0pt}\ {\isadigit{1}}{\isachardoublequoteclose}\isanewline
\ \ \ \ \isacommand{using}\isamarkupfalse%
\ assms{\isacharparenleft}{\kern0pt}{\isadigit{4}}{\isacharparenright}{\kern0pt}\ assms{\isacharparenleft}{\kern0pt}{\isadigit{1}}{\isacharparenright}{\kern0pt}\ list{\isacharunderscore}{\kern0pt}bit{\isacharunderscore}{\kern0pt}count{\isacharunderscore}{\kern0pt}est{\isacharbrackleft}{\kern0pt}\isakeyword{where}\ xs{\isacharequal}{\kern0pt}{\isachardoublequoteopen}sorted{\isacharunderscore}{\kern0pt}list{\isacharunderscore}{\kern0pt}of{\isacharunderscore}{\kern0pt}set\ S{\isachardoublequoteclose}{\isacharbrackright}{\kern0pt}\ \isacommand{by}\isamarkupfalse%
\ simp\isanewline
\ \ \isacommand{also}\isamarkupfalse%
\ \isacommand{have}\isamarkupfalse%
\ {\isachardoublequoteopen}{\isachardot}{\kern0pt}{\isachardot}{\kern0pt}{\isachardot}{\kern0pt}\ {\isasymle}\ ereal\ {\isacharparenleft}{\kern0pt}real\ m{\isacharparenright}{\kern0pt}\ {\isacharasterisk}{\kern0pt}\ {\isacharparenleft}{\kern0pt}a{\isacharplus}{\kern0pt}{\isadigit{1}}{\isacharparenright}{\kern0pt}\ {\isacharplus}{\kern0pt}\ {\isadigit{1}}{\isachardoublequoteclose}\isanewline
\ \ \ \ \isacommand{apply}\isamarkupfalse%
\ {\isacharparenleft}{\kern0pt}rule\ add{\isacharunderscore}{\kern0pt}mono{\isacharparenright}{\kern0pt}\isanewline
\ \ \ \ \isacommand{apply}\isamarkupfalse%
\ {\isacharparenleft}{\kern0pt}rule\ ereal{\isacharunderscore}{\kern0pt}mult{\isacharunderscore}{\kern0pt}right{\isacharunderscore}{\kern0pt}mono{\isacharparenright}{\kern0pt}\isanewline
\ \ \ \ \isacommand{using}\isamarkupfalse%
\ assms\ \isacommand{by}\isamarkupfalse%
\ simp{\isacharplus}{\kern0pt}\isanewline
\ \ \isacommand{finally}\isamarkupfalse%
\ \isacommand{show}\isamarkupfalse%
\ {\isacharquery}{\kern0pt}thesis\ \isacommand{by}\isamarkupfalse%
\ simp\isanewline
\isacommand{qed}\isamarkupfalse%
%
\endisatagproof
{\isafoldproof}%
%
\isadelimproof
\isanewline
%
\endisadelimproof
%
\isadelimtheory
\isanewline
%
\endisadelimtheory
%
\isatagtheory
\isacommand{end}\isamarkupfalse%
%
\endisatagtheory
{\isafoldtheory}%
%
\isadelimtheory
%
\endisadelimtheory
%
\end{isabellebody}%
\endinput
%:%file=Encoding.tex%:%
%:%11=1%:%
%:%27=3%:%
%:%28=3%:%
%:%29=4%:%
%:%30=5%:%
%:%31=6%:%
%:%40=8%:%
%:%41=9%:%
%:%42=10%:%
%:%44=12%:%
%:%45=12%:%
%:%46=13%:%
%:%47=14%:%
%:%48=15%:%
%:%49=16%:%
%:%50=16%:%
%:%51=17%:%
%:%52=18%:%
%:%53=18%:%
%:%54=19%:%
%:%55=20%:%
%:%56=21%:%
%:%57=21%:%
%:%58=22%:%
%:%59=23%:%
%:%62=24%:%
%:%66=24%:%
%:%67=24%:%
%:%68=25%:%
%:%69=25%:%
%:%70=25%:%
%:%75=25%:%
%:%78=26%:%
%:%79=27%:%
%:%80=27%:%
%:%81=28%:%
%:%86=33%:%
%:%87=34%:%
%:%88=35%:%
%:%89=35%:%
%:%90=36%:%
%:%91=37%:%
%:%92=38%:%
%:%99=39%:%
%:%100=39%:%
%:%101=40%:%
%:%102=40%:%
%:%103=41%:%
%:%104=41%:%
%:%105=42%:%
%:%106=42%:%
%:%107=43%:%
%:%108=43%:%
%:%109=44%:%
%:%110=44%:%
%:%111=44%:%
%:%112=45%:%
%:%113=45%:%
%:%114=46%:%
%:%115=46%:%
%:%116=47%:%
%:%117=47%:%
%:%118=47%:%
%:%119=48%:%
%:%120=48%:%
%:%121=49%:%
%:%122=49%:%
%:%123=49%:%
%:%124=50%:%
%:%125=50%:%
%:%126=51%:%
%:%127=51%:%
%:%128=51%:%
%:%129=51%:%
%:%130=52%:%
%:%131=52%:%
%:%132=53%:%
%:%133=53%:%
%:%134=54%:%
%:%135=54%:%
%:%136=54%:%
%:%137=55%:%
%:%138=55%:%
%:%139=56%:%
%:%140=56%:%
%:%141=56%:%
%:%142=57%:%
%:%143=57%:%
%:%144=58%:%
%:%145=58%:%
%:%146=58%:%
%:%147=59%:%
%:%148=59%:%
%:%149=60%:%
%:%150=60%:%
%:%151=60%:%
%:%152=61%:%
%:%158=61%:%
%:%161=62%:%
%:%162=63%:%
%:%163=63%:%
%:%164=64%:%
%:%165=65%:%
%:%166=66%:%
%:%169=67%:%
%:%173=67%:%
%:%174=67%:%
%:%175=67%:%
%:%180=67%:%
%:%183=68%:%
%:%184=69%:%
%:%185=69%:%
%:%186=70%:%
%:%193=71%:%
%:%194=71%:%
%:%195=72%:%
%:%196=72%:%
%:%197=73%:%
%:%198=73%:%
%:%199=73%:%
%:%200=73%:%
%:%201=74%:%
%:%202=74%:%
%:%203=74%:%
%:%204=74%:%
%:%205=75%:%
%:%206=75%:%
%:%207=75%:%
%:%208=75%:%
%:%209=76%:%
%:%210=76%:%
%:%211=77%:%
%:%212=77%:%
%:%213=78%:%
%:%214=78%:%
%:%215=78%:%
%:%216=78%:%
%:%217=79%:%
%:%223=79%:%
%:%226=80%:%
%:%227=81%:%
%:%228=81%:%
%:%229=82%:%
%:%230=83%:%
%:%233=84%:%
%:%237=84%:%
%:%238=84%:%
%:%239=85%:%
%:%240=85%:%
%:%245=85%:%
%:%248=86%:%
%:%249=87%:%
%:%250=87%:%
%:%251=88%:%
%:%252=89%:%
%:%259=90%:%
%:%260=90%:%
%:%261=91%:%
%:%262=91%:%
%:%263=92%:%
%:%264=92%:%
%:%265=93%:%
%:%266=93%:%
%:%267=94%:%
%:%268=94%:%
%:%269=95%:%
%:%270=95%:%
%:%271=95%:%
%:%272=96%:%
%:%273=96%:%
%:%274=97%:%
%:%275=97%:%
%:%276=98%:%
%:%277=98%:%
%:%278=98%:%
%:%279=98%:%
%:%280=99%:%
%:%281=99%:%
%:%282=100%:%
%:%283=100%:%
%:%284=100%:%
%:%285=100%:%
%:%286=101%:%
%:%287=101%:%
%:%288=101%:%
%:%289=101%:%
%:%290=101%:%
%:%291=102%:%
%:%292=102%:%
%:%293=103%:%
%:%294=103%:%
%:%295=103%:%
%:%296=103%:%
%:%297=104%:%
%:%298=104%:%
%:%299=104%:%
%:%300=104%:%
%:%301=105%:%
%:%302=105%:%
%:%303=106%:%
%:%304=106%:%
%:%305=106%:%
%:%306=107%:%
%:%312=107%:%
%:%315=108%:%
%:%316=109%:%
%:%317=109%:%
%:%318=110%:%
%:%319=111%:%
%:%320=112%:%
%:%321=113%:%
%:%322=113%:%
%:%323=114%:%
%:%324=115%:%
%:%325=116%:%
%:%326=117%:%
%:%327=118%:%
%:%328=118%:%
%:%331=119%:%
%:%335=119%:%
%:%336=119%:%
%:%345=121%:%
%:%347=123%:%
%:%348=123%:%
%:%349=124%:%
%:%350=125%:%
%:%351=126%:%
%:%352=127%:%
%:%353=127%:%
%:%354=128%:%
%:%355=129%:%
%:%356=130%:%
%:%358=132%:%
%:%359=133%:%
%:%360=134%:%
%:%363=135%:%
%:%367=135%:%
%:%368=135%:%
%:%373=135%:%
%:%376=136%:%
%:%377=136%:%
%:%380=137%:%
%:%384=137%:%
%:%385=137%:%
%:%386=138%:%
%:%387=138%:%
%:%392=138%:%
%:%395=139%:%
%:%396=140%:%
%:%397=140%:%
%:%398=141%:%
%:%399=142%:%
%:%402=143%:%
%:%406=143%:%
%:%407=143%:%
%:%408=143%:%
%:%409=143%:%
%:%414=143%:%
%:%417=144%:%
%:%418=145%:%
%:%419=145%:%
%:%420=146%:%
%:%423=147%:%
%:%427=147%:%
%:%428=147%:%
%:%429=148%:%
%:%430=148%:%
%:%435=148%:%
%:%438=149%:%
%:%439=150%:%
%:%440=150%:%
%:%441=151%:%
%:%442=152%:%
%:%449=153%:%
%:%450=153%:%
%:%451=154%:%
%:%452=154%:%
%:%453=155%:%
%:%454=155%:%
%:%455=156%:%
%:%456=156%:%
%:%457=157%:%
%:%458=158%:%
%:%459=158%:%
%:%460=159%:%
%:%461=159%:%
%:%462=159%:%
%:%463=160%:%
%:%464=161%:%
%:%465=161%:%
%:%466=162%:%
%:%467=162%:%
%:%468=163%:%
%:%469=163%:%
%:%470=164%:%
%:%476=164%:%
%:%479=165%:%
%:%480=166%:%
%:%481=166%:%
%:%482=167%:%
%:%483=168%:%
%:%484=169%:%
%:%487=170%:%
%:%491=170%:%
%:%492=170%:%
%:%493=170%:%
%:%498=170%:%
%:%501=171%:%
%:%502=172%:%
%:%503=172%:%
%:%504=173%:%
%:%505=174%:%
%:%512=175%:%
%:%513=175%:%
%:%514=176%:%
%:%515=176%:%
%:%516=177%:%
%:%517=177%:%
%:%518=177%:%
%:%519=177%:%
%:%520=178%:%
%:%521=178%:%
%:%522=179%:%
%:%523=179%:%
%:%524=180%:%
%:%525=180%:%
%:%526=180%:%
%:%527=181%:%
%:%528=181%:%
%:%529=182%:%
%:%530=182%:%
%:%531=183%:%
%:%532=183%:%
%:%533=183%:%
%:%534=184%:%
%:%540=184%:%
%:%543=185%:%
%:%544=186%:%
%:%545=186%:%
%:%546=187%:%
%:%547=188%:%
%:%550=189%:%
%:%554=189%:%
%:%555=189%:%
%:%564=191%:%
%:%566=193%:%
%:%567=193%:%
%:%568=194%:%
%:%569=195%:%
%:%570=196%:%
%:%571=197%:%
%:%572=198%:%
%:%573=198%:%
%:%574=199%:%
%:%575=200%:%
%:%576=200%:%
%:%577=201%:%
%:%578=202%:%
%:%579=203%:%
%:%580=204%:%
%:%581=205%:%
%:%582=206%:%
%:%583=207%:%
%:%584=207%:%
%:%585=208%:%
%:%588=209%:%
%:%592=209%:%
%:%593=209%:%
%:%598=209%:%
%:%601=210%:%
%:%602=211%:%
%:%603=211%:%
%:%604=212%:%
%:%607=213%:%
%:%611=213%:%
%:%612=213%:%
%:%617=213%:%
%:%620=214%:%
%:%621=215%:%
%:%622=215%:%
%:%623=216%:%
%:%630=217%:%
%:%631=217%:%
%:%632=218%:%
%:%633=218%:%
%:%634=219%:%
%:%635=219%:%
%:%636=219%:%
%:%637=219%:%
%:%638=220%:%
%:%639=220%:%
%:%640=221%:%
%:%641=221%:%
%:%642=222%:%
%:%643=222%:%
%:%644=223%:%
%:%645=223%:%
%:%646=224%:%
%:%647=224%:%
%:%648=225%:%
%:%649=225%:%
%:%650=226%:%
%:%651=226%:%
%:%652=227%:%
%:%653=227%:%
%:%654=227%:%
%:%655=227%:%
%:%656=228%:%
%:%662=228%:%
%:%665=229%:%
%:%666=230%:%
%:%667=230%:%
%:%668=231%:%
%:%669=232%:%
%:%676=233%:%
%:%677=233%:%
%:%678=234%:%
%:%679=234%:%
%:%680=235%:%
%:%681=235%:%
%:%682=235%:%
%:%683=236%:%
%:%684=236%:%
%:%685=237%:%
%:%686=237%:%
%:%687=238%:%
%:%697=240%:%
%:%699=242%:%
%:%700=242%:%
%:%701=243%:%
%:%702=244%:%
%:%703=245%:%
%:%704=246%:%
%:%705=246%:%
%:%706=247%:%
%:%707=248%:%
%:%708=249%:%
%:%709=250%:%
%:%710=251%:%
%:%711=252%:%
%:%712=252%:%
%:%715=253%:%
%:%719=253%:%
%:%720=253%:%
%:%721=254%:%
%:%722=254%:%
%:%727=254%:%
%:%730=255%:%
%:%731=256%:%
%:%732=256%:%
%:%733=257%:%
%:%740=258%:%
%:%741=258%:%
%:%742=259%:%
%:%743=259%:%
%:%744=260%:%
%:%745=260%:%
%:%746=261%:%
%:%747=261%:%
%:%748=262%:%
%:%749=262%:%
%:%750=263%:%
%:%751=263%:%
%:%752=263%:%
%:%753=264%:%
%:%754=264%:%
%:%755=264%:%
%:%756=265%:%
%:%757=265%:%
%:%758=265%:%
%:%759=266%:%
%:%765=266%:%
%:%768=267%:%
%:%769=268%:%
%:%770=268%:%
%:%771=269%:%
%:%772=270%:%
%:%779=271%:%
%:%780=271%:%
%:%781=272%:%
%:%782=272%:%
%:%783=272%:%
%:%784=272%:%
%:%785=273%:%
%:%786=273%:%
%:%787=273%:%
%:%788=273%:%
%:%789=274%:%
%:%799=276%:%
%:%801=278%:%
%:%802=278%:%
%:%803=279%:%
%:%804=280%:%
%:%805=281%:%
%:%806=282%:%
%:%807=282%:%
%:%808=283%:%
%:%809=284%:%
%:%811=286%:%
%:%812=287%:%
%:%813=288%:%
%:%814=288%:%
%:%815=289%:%
%:%818=290%:%
%:%822=290%:%
%:%823=290%:%
%:%824=291%:%
%:%825=291%:%
%:%826=292%:%
%:%827=292%:%
%:%832=292%:%
%:%835=293%:%
%:%836=294%:%
%:%837=294%:%
%:%838=295%:%
%:%839=296%:%
%:%840=297%:%
%:%847=298%:%
%:%848=298%:%
%:%849=299%:%
%:%850=299%:%
%:%851=300%:%
%:%852=300%:%
%:%853=301%:%
%:%854=302%:%
%:%855=302%:%
%:%856=303%:%
%:%857=303%:%
%:%858=304%:%
%:%859=304%:%
%:%860=305%:%
%:%861=305%:%
%:%862=306%:%
%:%863=307%:%
%:%864=307%:%
%:%865=308%:%
%:%866=308%:%
%:%867=309%:%
%:%868=309%:%
%:%869=310%:%
%:%870=310%:%
%:%871=310%:%
%:%872=311%:%
%:%878=311%:%
%:%881=312%:%
%:%882=313%:%
%:%883=313%:%
%:%884=314%:%
%:%887=315%:%
%:%891=315%:%
%:%892=315%:%
%:%897=315%:%
%:%900=316%:%
%:%901=317%:%
%:%902=317%:%
%:%903=318%:%
%:%906=319%:%
%:%910=319%:%
%:%911=319%:%
%:%920=321%:%
%:%922=323%:%
%:%923=323%:%
%:%924=324%:%
%:%925=325%:%
%:%926=326%:%
%:%927=327%:%
%:%928=327%:%
%:%929=328%:%
%:%930=329%:%
%:%931=330%:%
%:%938=331%:%
%:%939=331%:%
%:%940=332%:%
%:%941=332%:%
%:%943=334%:%
%:%944=335%:%
%:%945=336%:%
%:%946=336%:%
%:%947=337%:%
%:%948=337%:%
%:%949=338%:%
%:%950=338%:%
%:%951=338%:%
%:%952=339%:%
%:%953=339%:%
%:%954=340%:%
%:%955=340%:%
%:%956=341%:%
%:%957=341%:%
%:%958=341%:%
%:%959=342%:%
%:%960=342%:%
%:%961=343%:%
%:%962=343%:%
%:%963=343%:%
%:%964=344%:%
%:%965=344%:%
%:%966=345%:%
%:%967=345%:%
%:%968=345%:%
%:%969=346%:%
%:%970=346%:%
%:%971=347%:%
%:%972=347%:%
%:%973=348%:%
%:%974=348%:%
%:%975=349%:%
%:%976=349%:%
%:%977=349%:%
%:%978=350%:%
%:%984=350%:%
%:%987=351%:%
%:%988=352%:%
%:%989=352%:%
%:%990=353%:%
%:%993=354%:%
%:%997=354%:%
%:%998=354%:%
%:%1007=356%:%
%:%1008=357%:%
%:%1010=359%:%
%:%1011=359%:%
%:%1012=360%:%
%:%1013=361%:%
%:%1014=362%:%
%:%1017=363%:%
%:%1021=363%:%
%:%1022=363%:%
%:%1023=363%:%
%:%1032=365%:%
%:%1034=367%:%
%:%1035=367%:%
%:%1036=368%:%
%:%1040=372%:%
%:%1041=373%:%
%:%1042=374%:%
%:%1043=374%:%
%:%1044=375%:%
%:%1045=376%:%
%:%1048=377%:%
%:%1052=377%:%
%:%1053=377%:%
%:%1054=378%:%
%:%1055=378%:%
%:%1056=379%:%
%:%1057=379%:%
%:%1058=380%:%
%:%1059=380%:%
%:%1060=381%:%
%:%1061=381%:%
%:%1062=381%:%
%:%1067=381%:%
%:%1070=382%:%
%:%1071=383%:%
%:%1072=383%:%
%:%1073=384%:%
%:%1074=385%:%
%:%1077=386%:%
%:%1081=386%:%
%:%1082=386%:%
%:%1083=387%:%
%:%1084=387%:%
%:%1093=389%:%
%:%1095=391%:%
%:%1096=391%:%
%:%1097=392%:%
%:%1098=393%:%
%:%1099=393%:%
%:%1100=394%:%
%:%1101=395%:%
%:%1104=396%:%
%:%1108=396%:%
%:%1109=396%:%
%:%1110=397%:%
%:%1111=397%:%
%:%1112=398%:%
%:%1113=398%:%
%:%1114=399%:%
%:%1115=399%:%
%:%1116=400%:%
%:%1117=400%:%
%:%1122=400%:%
%:%1125=401%:%
%:%1126=402%:%
%:%1127=402%:%
%:%1128=403%:%
%:%1129=404%:%
%:%1132=405%:%
%:%1136=405%:%
%:%1137=405%:%
%:%1138=406%:%
%:%1139=406%:%
%:%1144=406%:%
%:%1147=407%:%
%:%1148=408%:%
%:%1149=408%:%
%:%1150=409%:%
%:%1151=410%:%
%:%1152=411%:%
%:%1153=412%:%
%:%1154=413%:%
%:%1161=414%:%
%:%1162=414%:%
%:%1163=415%:%
%:%1164=415%:%
%:%1165=416%:%
%:%1166=416%:%
%:%1167=416%:%
%:%1168=417%:%
%:%1169=417%:%
%:%1170=417%:%
%:%1171=418%:%
%:%1172=418%:%
%:%1173=419%:%
%:%1174=419%:%
%:%1175=420%:%
%:%1176=420%:%
%:%1177=420%:%
%:%1178=421%:%
%:%1179=421%:%
%:%1180=421%:%
%:%1181=421%:%
%:%1182=422%:%
%:%1188=422%:%
%:%1193=423%:%
%:%1198=424%:%

%
\begin{isabellebody}%
\setisabellecontext{Field}%
%
\isadelimdocument
%
\endisadelimdocument
%
\isatagdocument
%
\isamarkupsection{Field%
}
\isamarkuptrue%
%
\endisatagdocument
{\isafolddocument}%
%
\isadelimdocument
%
\endisadelimdocument
%
\isadelimtheory
%
\endisadelimtheory
%
\isatagtheory
\isacommand{theory}\isamarkupfalse%
\ Field\isanewline
\ \ \isakeyword{imports}\ Main\ {\isachardoublequoteopen}HOL{\isacharminus}{\kern0pt}Algebra{\isachardot}{\kern0pt}Ring{\isacharunderscore}{\kern0pt}Divisibility{\isachardoublequoteclose}\ {\isachardoublequoteopen}HOL{\isacharminus}{\kern0pt}Algebra{\isachardot}{\kern0pt}IntRing{\isachardoublequoteclose}\isanewline
\isakeyword{begin}%
\endisatagtheory
{\isafoldtheory}%
%
\isadelimtheory
%
\endisadelimtheory
%
\begin{isamarkuptext}%
This section contains a proof that the factor ring \isa{ZFact\ p} for
\isa{prime\ p} is a field. Note that the bulk of the work has already been done in
HOL-Algebra, in particular it is established that \isa{ZFact\ p} is a domain.

However, any domain with a finite carrier is already a field. This can be seen by establishing that
multiplication by a non-zero element is an injective map between the elements of the carrier of the
domain. But an injective map between sets of the same non-finite cardinality is also surjective.
Hence we can find the unit element in the image of such a map.

Additionally the canonical bijection between \isa{ZFact\ p} and \isa{{\isacharbraceleft}{\kern0pt}{\isadigit{0}}{\isachardot}{\kern0pt}{\isachardot}{\kern0pt}{\isacharless}{\kern0pt}p{\isacharbraceright}{\kern0pt}}
is introduced, which is useful for hashing natural numbers.%
\end{isamarkuptext}\isamarkuptrue%
\isacommand{definition}\isamarkupfalse%
\ zfact{\isacharunderscore}{\kern0pt}embed\ {\isacharcolon}{\kern0pt}{\isacharcolon}{\kern0pt}\ {\isachardoublequoteopen}nat\ {\isasymRightarrow}\ nat\ {\isasymRightarrow}\ int\ set{\isachardoublequoteclose}\ \isakeyword{where}\isanewline
\ \ {\isachardoublequoteopen}zfact{\isacharunderscore}{\kern0pt}embed\ p\ k\ {\isacharequal}{\kern0pt}\ Idl\isactrlbsub {\isasymZ}\isactrlesub \ {\isacharbraceleft}{\kern0pt}int\ p{\isacharbraceright}{\kern0pt}\ {\isacharplus}{\kern0pt}{\isachargreater}{\kern0pt}\isactrlbsub {\isasymZ}\isactrlesub \ {\isacharparenleft}{\kern0pt}int\ k{\isacharparenright}{\kern0pt}{\isachardoublequoteclose}\isanewline
\isanewline
\isacommand{lemma}\isamarkupfalse%
\ zfact{\isacharunderscore}{\kern0pt}embed{\isacharunderscore}{\kern0pt}ran{\isacharcolon}{\kern0pt}\isanewline
\ \ \isakeyword{assumes}\ {\isachardoublequoteopen}p\ {\isachargreater}{\kern0pt}\ {\isadigit{0}}{\isachardoublequoteclose}\isanewline
\ \ \isakeyword{shows}\ {\isachardoublequoteopen}zfact{\isacharunderscore}{\kern0pt}embed\ p\ {\isacharbackquote}{\kern0pt}\ {\isacharbraceleft}{\kern0pt}{\isadigit{0}}{\isachardot}{\kern0pt}{\isachardot}{\kern0pt}{\isacharless}{\kern0pt}p{\isacharbraceright}{\kern0pt}\ {\isacharequal}{\kern0pt}\ carrier\ {\isacharparenleft}{\kern0pt}ZFact\ p{\isacharparenright}{\kern0pt}{\isachardoublequoteclose}\isanewline
%
\isadelimproof
%
\endisadelimproof
%
\isatagproof
\isacommand{proof}\isamarkupfalse%
\ {\isacharminus}{\kern0pt}\isanewline
\ \ \isacommand{have}\isamarkupfalse%
\ {\isachardoublequoteopen}zfact{\isacharunderscore}{\kern0pt}embed\ p\ {\isacharbackquote}{\kern0pt}\ {\isacharbraceleft}{\kern0pt}{\isadigit{0}}{\isachardot}{\kern0pt}{\isachardot}{\kern0pt}{\isacharless}{\kern0pt}p{\isacharbraceright}{\kern0pt}\ {\isasymsubseteq}\ carrier\ {\isacharparenleft}{\kern0pt}ZFact\ p{\isacharparenright}{\kern0pt}{\isachardoublequoteclose}\isanewline
\ \ \isacommand{proof}\isamarkupfalse%
\ {\isacharparenleft}{\kern0pt}rule\ subsetI{\isacharparenright}{\kern0pt}\isanewline
\ \ \ \ \isacommand{fix}\isamarkupfalse%
\ x\isanewline
\ \ \ \ \isacommand{assume}\isamarkupfalse%
\ {\isachardoublequoteopen}x\ {\isasymin}\ zfact{\isacharunderscore}{\kern0pt}embed\ p\ {\isacharbackquote}{\kern0pt}\ {\isacharbraceleft}{\kern0pt}{\isadigit{0}}{\isachardot}{\kern0pt}{\isachardot}{\kern0pt}{\isacharless}{\kern0pt}p{\isacharbraceright}{\kern0pt}{\isachardoublequoteclose}\isanewline
\ \ \ \ \isacommand{then}\isamarkupfalse%
\ \isacommand{obtain}\isamarkupfalse%
\ m\ \isakeyword{where}\ m{\isacharunderscore}{\kern0pt}def{\isacharcolon}{\kern0pt}\ {\isachardoublequoteopen}zfact{\isacharunderscore}{\kern0pt}embed\ p\ m\ {\isacharequal}{\kern0pt}\ x{\isachardoublequoteclose}\ \isacommand{by}\isamarkupfalse%
\ blast\isanewline
\ \ \ \ \isacommand{have}\isamarkupfalse%
\ {\isachardoublequoteopen}zfact{\isacharunderscore}{\kern0pt}embed\ p\ m\ {\isasymin}\ carrier\ {\isacharparenleft}{\kern0pt}ZFact\ p{\isacharparenright}{\kern0pt}{\isachardoublequoteclose}\ \isanewline
\ \ \ \ \ \ \isacommand{by}\isamarkupfalse%
\ {\isacharparenleft}{\kern0pt}simp\ add{\isacharcolon}{\kern0pt}\ ZFact{\isacharunderscore}{\kern0pt}def\ ZFact{\isacharunderscore}{\kern0pt}defs{\isacharparenleft}{\kern0pt}{\isadigit{2}}{\isacharparenright}{\kern0pt}\ int{\isachardot}{\kern0pt}a{\isacharunderscore}{\kern0pt}rcosetsI\ zfact{\isacharunderscore}{\kern0pt}embed{\isacharunderscore}{\kern0pt}def{\isacharparenright}{\kern0pt}\isanewline
\ \ \ \ \isacommand{thus}\isamarkupfalse%
\ {\isachardoublequoteopen}x\ {\isasymin}\ carrier\ {\isacharparenleft}{\kern0pt}ZFact\ p{\isacharparenright}{\kern0pt}{\isachardoublequoteclose}\ \isacommand{using}\isamarkupfalse%
\ m{\isacharunderscore}{\kern0pt}def\ \isacommand{by}\isamarkupfalse%
\ auto\isanewline
\ \ \isacommand{qed}\isamarkupfalse%
\isanewline
\ \ \isacommand{moreover}\isamarkupfalse%
\ \isacommand{have}\isamarkupfalse%
\ {\isachardoublequoteopen}carrier\ {\isacharparenleft}{\kern0pt}ZFact\ p{\isacharparenright}{\kern0pt}\ {\isasymsubseteq}\ zfact{\isacharunderscore}{\kern0pt}embed\ p\ {\isacharbackquote}{\kern0pt}\ {\isacharbraceleft}{\kern0pt}{\isadigit{0}}{\isachardot}{\kern0pt}{\isachardot}{\kern0pt}{\isacharless}{\kern0pt}p{\isacharbraceright}{\kern0pt}{\isachardoublequoteclose}\isanewline
\ \ \isacommand{proof}\isamarkupfalse%
\ {\isacharparenleft}{\kern0pt}rule\ subsetI{\isacharparenright}{\kern0pt}\isanewline
\ \ \ \ \isacommand{define}\isamarkupfalse%
\ I\ \isakeyword{where}\ {\isachardoublequoteopen}I\ {\isacharequal}{\kern0pt}\ Idl\isactrlbsub {\isasymZ}\isactrlesub \ {\isacharbraceleft}{\kern0pt}int\ p{\isacharbraceright}{\kern0pt}{\isachardoublequoteclose}\isanewline
\ \ \ \ \isacommand{fix}\isamarkupfalse%
\ x\isanewline
\ \ \ \ \isacommand{have}\isamarkupfalse%
\ coset{\isacharunderscore}{\kern0pt}elim{\isacharcolon}{\kern0pt}\ {\isachardoublequoteopen}{\isasymAnd}\ x\ R\ I{\isachardot}{\kern0pt}\ x\ {\isasymin}\ a{\isacharunderscore}{\kern0pt}rcosets\isactrlbsub R\isactrlesub \ I\ {\isasymLongrightarrow}\ {\isacharparenleft}{\kern0pt}{\isasymexists}y{\isachardot}{\kern0pt}\ x\ {\isacharequal}{\kern0pt}\ I\ {\isacharplus}{\kern0pt}{\isachargreater}{\kern0pt}\isactrlbsub R\isactrlesub \ y{\isacharparenright}{\kern0pt}{\isachardoublequoteclose}\isanewline
\ \ \ \ \ \ \isacommand{using}\isamarkupfalse%
\ assms\ \isacommand{apply}\isamarkupfalse%
\ {\isacharparenleft}{\kern0pt}simp\ add{\isacharcolon}{\kern0pt}FactRing{\isacharunderscore}{\kern0pt}simps{\isacharparenright}{\kern0pt}\ \isacommand{by}\isamarkupfalse%
\ blast\isanewline
\ \ \ \ \isacommand{assume}\isamarkupfalse%
\ a{\isacharcolon}{\kern0pt}{\isachardoublequoteopen}x\ {\isasymin}\ carrier\ {\isacharparenleft}{\kern0pt}ZFact\ {\isacharparenleft}{\kern0pt}int\ p{\isacharparenright}{\kern0pt}{\isacharparenright}{\kern0pt}{\isachardoublequoteclose}\isanewline
\ \ \ \ \isacommand{obtain}\isamarkupfalse%
\ y{\isacharprime}{\kern0pt}\ \isakeyword{where}\ y{\isacharunderscore}{\kern0pt}{\isadigit{0}}{\isacharcolon}{\kern0pt}\ {\isachardoublequoteopen}x\ {\isacharequal}{\kern0pt}\ I\ {\isacharplus}{\kern0pt}{\isachargreater}{\kern0pt}\isactrlbsub {\isasymZ}\isactrlesub \ y{\isacharprime}{\kern0pt}{\isachardoublequoteclose}\ \isanewline
\ \ \ \ \ \ \isacommand{apply}\isamarkupfalse%
\ {\isacharparenleft}{\kern0pt}simp\ add{\isacharcolon}{\kern0pt}I{\isacharunderscore}{\kern0pt}def\ carrier{\isacharunderscore}{\kern0pt}def\ ZFact{\isacharunderscore}{\kern0pt}def\ FactRing{\isacharunderscore}{\kern0pt}simps{\isacharparenright}{\kern0pt}\isanewline
\ \ \ \ \ \ \isacommand{by}\isamarkupfalse%
\ {\isacharparenleft}{\kern0pt}metis\ coset{\isacharunderscore}{\kern0pt}elim\ FactRing{\isacharunderscore}{\kern0pt}def\ ZFact{\isacharunderscore}{\kern0pt}def\ a\ partial{\isacharunderscore}{\kern0pt}object{\isachardot}{\kern0pt}select{\isacharunderscore}{\kern0pt}convs{\isacharparenleft}{\kern0pt}{\isadigit{1}}{\isacharparenright}{\kern0pt}{\isacharparenright}{\kern0pt}\isanewline
\ \ \ \ \isacommand{define}\isamarkupfalse%
\ y\ \isakeyword{where}\ {\isachardoublequoteopen}y\ {\isacharequal}{\kern0pt}\ y{\isacharprime}{\kern0pt}\ mod\ p\ {\isacharminus}{\kern0pt}y{\isacharprime}{\kern0pt}{\isachardoublequoteclose}\isanewline
\ \ \ \ \isacommand{hence}\isamarkupfalse%
\ {\isachardoublequoteopen}y\ mod\ p\ {\isacharequal}{\kern0pt}\ {\isadigit{0}}{\isachardoublequoteclose}\ \isacommand{by}\isamarkupfalse%
\ {\isacharparenleft}{\kern0pt}simp\ add{\isacharcolon}{\kern0pt}\ mod{\isacharunderscore}{\kern0pt}diff{\isacharunderscore}{\kern0pt}left{\isacharunderscore}{\kern0pt}eq{\isacharparenright}{\kern0pt}\isanewline
\ \ \ \ \isacommand{hence}\isamarkupfalse%
\ y{\isacharunderscore}{\kern0pt}{\isadigit{1}}{\isacharcolon}{\kern0pt}{\isachardoublequoteopen}y\ {\isasymin}\ I{\isachardoublequoteclose}\ \isacommand{using}\isamarkupfalse%
\ I{\isacharunderscore}{\kern0pt}def\ \isanewline
\ \ \ \ \ \ \isacommand{by}\isamarkupfalse%
\ {\isacharparenleft}{\kern0pt}metis\ Idl{\isacharunderscore}{\kern0pt}subset{\isacharunderscore}{\kern0pt}eq{\isacharunderscore}{\kern0pt}dvd\ int{\isacharunderscore}{\kern0pt}Idl{\isacharunderscore}{\kern0pt}subset{\isacharunderscore}{\kern0pt}ideal\ mod{\isacharunderscore}{\kern0pt}{\isadigit{0}}{\isacharunderscore}{\kern0pt}imp{\isacharunderscore}{\kern0pt}dvd{\isacharparenright}{\kern0pt}\isanewline
\ \ \ \ \isacommand{have}\isamarkupfalse%
\ y{\isacharunderscore}{\kern0pt}{\isadigit{3}}{\isacharcolon}{\kern0pt}{\isachardoublequoteopen}y\ {\isacharplus}{\kern0pt}\ y{\isacharprime}{\kern0pt}\ {\isacharless}{\kern0pt}\ p\ {\isasymand}\ y\ {\isacharplus}{\kern0pt}\ y{\isacharprime}{\kern0pt}\ {\isasymge}\ {\isadigit{0}}{\isachardoublequoteclose}\ \isanewline
\ \ \ \ \ \ \isacommand{using}\isamarkupfalse%
\ y{\isacharunderscore}{\kern0pt}def\ assms{\isacharparenleft}{\kern0pt}{\isadigit{1}}{\isacharparenright}{\kern0pt}\ \isacommand{by}\isamarkupfalse%
\ auto\isanewline
\ \ \ \ \isacommand{hence}\isamarkupfalse%
\ y{\isacharunderscore}{\kern0pt}{\isadigit{2}}{\isacharcolon}{\kern0pt}{\isachardoublequoteopen}y\ {\isasymoplus}\isactrlbsub {\isasymZ}\isactrlesub \ y{\isacharprime}{\kern0pt}\ {\isacharless}{\kern0pt}\ p\ {\isasymand}\ y\ {\isasymoplus}\isactrlbsub {\isasymZ}\isactrlesub \ y{\isacharprime}{\kern0pt}\ {\isasymge}\ {\isadigit{0}}{\isachardoublequoteclose}\ \isacommand{using}\isamarkupfalse%
\ int{\isacharunderscore}{\kern0pt}add{\isacharunderscore}{\kern0pt}eq\ \isacommand{by}\isamarkupfalse%
\ presburger\isanewline
\ \ \ \ \isacommand{then}\isamarkupfalse%
\ \isacommand{have}\isamarkupfalse%
\ a{\isadigit{3}}{\isacharcolon}{\kern0pt}\ {\isachardoublequoteopen}I\ {\isacharplus}{\kern0pt}{\isachargreater}{\kern0pt}\isactrlbsub {\isasymZ}\isactrlesub \ y{\isacharprime}{\kern0pt}\ {\isacharequal}{\kern0pt}\ I\ {\isacharplus}{\kern0pt}{\isachargreater}{\kern0pt}\isactrlbsub {\isasymZ}\isactrlesub \ {\isacharparenleft}{\kern0pt}y\ {\isasymoplus}\isactrlbsub {\isasymZ}\isactrlesub \ y{\isacharprime}{\kern0pt}{\isacharparenright}{\kern0pt}{\isachardoublequoteclose}\ \isacommand{using}\isamarkupfalse%
\ I{\isacharunderscore}{\kern0pt}def\ \isanewline
\ \ \ \ \ \ \isacommand{by}\isamarkupfalse%
\ {\isacharparenleft}{\kern0pt}metis\ {\isacharparenleft}{\kern0pt}no{\isacharunderscore}{\kern0pt}types{\isacharcomma}{\kern0pt}\ lifting{\isacharparenright}{\kern0pt}\ y{\isacharunderscore}{\kern0pt}{\isadigit{1}}\ UNIV{\isacharunderscore}{\kern0pt}I\ abelian{\isacharunderscore}{\kern0pt}group{\isachardot}{\kern0pt}a{\isacharunderscore}{\kern0pt}coset{\isacharunderscore}{\kern0pt}add{\isacharunderscore}{\kern0pt}assoc\ \isanewline
\ \ \ \ \ \ \ \ \ \ int{\isachardot}{\kern0pt}Idl{\isacharunderscore}{\kern0pt}subset{\isacharunderscore}{\kern0pt}ideal{\isacharprime}{\kern0pt}\ int{\isachardot}{\kern0pt}a{\isacharunderscore}{\kern0pt}rcos{\isacharunderscore}{\kern0pt}zero\ int{\isachardot}{\kern0pt}abelian{\isacharunderscore}{\kern0pt}group{\isacharunderscore}{\kern0pt}axioms\isanewline
\ \ \ \ \ \ \ \ \ \ int{\isachardot}{\kern0pt}cgenideal{\isacharunderscore}{\kern0pt}eq{\isacharunderscore}{\kern0pt}genideal\ int{\isachardot}{\kern0pt}cgenideal{\isacharunderscore}{\kern0pt}ideal\ int{\isachardot}{\kern0pt}genideal{\isacharunderscore}{\kern0pt}one\ int{\isacharunderscore}{\kern0pt}carrier{\isacharunderscore}{\kern0pt}eq{\isacharparenright}{\kern0pt}\isanewline
\ \ \ \ \isacommand{obtain}\isamarkupfalse%
\ w{\isacharcolon}{\kern0pt}{\isacharcolon}{\kern0pt}nat\ \ \isakeyword{where}\ y{\isacharunderscore}{\kern0pt}{\isadigit{4}}{\isacharcolon}{\kern0pt}\ {\isachardoublequoteopen}int\ w\ {\isacharequal}{\kern0pt}\ y\ {\isasymoplus}\isactrlbsub {\isasymZ}\isactrlesub \ y{\isacharprime}{\kern0pt}{\isachardoublequoteclose}\ \isanewline
\ \ \ \ \ \ \isacommand{using}\isamarkupfalse%
\ y{\isacharunderscore}{\kern0pt}{\isadigit{2}}\ nonneg{\isacharunderscore}{\kern0pt}int{\isacharunderscore}{\kern0pt}cases\ \isacommand{by}\isamarkupfalse%
\ metis\isanewline
\ \ \ \ \isacommand{have}\isamarkupfalse%
\ {\isachardoublequoteopen}x\ {\isacharequal}{\kern0pt}\ I\ {\isacharplus}{\kern0pt}{\isachargreater}{\kern0pt}\isactrlbsub {\isasymZ}\isactrlesub \ {\isacharparenleft}{\kern0pt}int\ w{\isacharparenright}{\kern0pt}{\isachardoublequoteclose}\ \isakeyword{and}\ {\isachardoublequoteopen}w\ {\isacharless}{\kern0pt}\ p{\isachardoublequoteclose}\ \isacommand{using}\isamarkupfalse%
\ y{\isacharunderscore}{\kern0pt}{\isadigit{2}}\ a{\isadigit{3}}\ y{\isacharunderscore}{\kern0pt}{\isadigit{0}}\ y{\isacharunderscore}{\kern0pt}{\isadigit{4}}\ \isacommand{by}\isamarkupfalse%
\ presburger{\isacharplus}{\kern0pt}\ \ \isanewline
\ \ \ \ \isacommand{thus}\isamarkupfalse%
\ {\isachardoublequoteopen}x\ {\isasymin}\ zfact{\isacharunderscore}{\kern0pt}embed\ p\ {\isacharbackquote}{\kern0pt}\ {\isacharbraceleft}{\kern0pt}{\isadigit{0}}{\isachardot}{\kern0pt}{\isachardot}{\kern0pt}{\isacharless}{\kern0pt}p{\isacharbraceright}{\kern0pt}{\isachardoublequoteclose}\ \isacommand{by}\isamarkupfalse%
\ {\isacharparenleft}{\kern0pt}simp\ add{\isacharcolon}{\kern0pt}zfact{\isacharunderscore}{\kern0pt}embed{\isacharunderscore}{\kern0pt}def\ I{\isacharunderscore}{\kern0pt}def{\isacharparenright}{\kern0pt}\isanewline
\ \ \isacommand{qed}\isamarkupfalse%
\isanewline
\ \ \isacommand{ultimately}\isamarkupfalse%
\ \isacommand{show}\isamarkupfalse%
\ {\isacharquery}{\kern0pt}thesis\ \isacommand{using}\isamarkupfalse%
\ order{\isacharunderscore}{\kern0pt}antisym\ \isacommand{by}\isamarkupfalse%
\ auto\isanewline
\isacommand{qed}\isamarkupfalse%
%
\endisatagproof
{\isafoldproof}%
%
\isadelimproof
\isanewline
%
\endisadelimproof
\isanewline
\isacommand{lemma}\isamarkupfalse%
\ zfact{\isacharunderscore}{\kern0pt}embed{\isacharunderscore}{\kern0pt}inj{\isacharcolon}{\kern0pt}\isanewline
\ \ \isakeyword{assumes}\ {\isachardoublequoteopen}p\ {\isachargreater}{\kern0pt}\ {\isadigit{0}}{\isachardoublequoteclose}\isanewline
\ \ \isakeyword{shows}\ {\isachardoublequoteopen}inj{\isacharunderscore}{\kern0pt}on\ {\isacharparenleft}{\kern0pt}zfact{\isacharunderscore}{\kern0pt}embed\ p{\isacharparenright}{\kern0pt}\ {\isacharbraceleft}{\kern0pt}{\isadigit{0}}{\isachardot}{\kern0pt}{\isachardot}{\kern0pt}{\isacharless}{\kern0pt}p{\isacharbraceright}{\kern0pt}{\isachardoublequoteclose}\isanewline
%
\isadelimproof
%
\endisadelimproof
%
\isatagproof
\isacommand{proof}\isamarkupfalse%
\isanewline
\ \ \isacommand{fix}\isamarkupfalse%
\ x\isanewline
\ \ \isacommand{fix}\isamarkupfalse%
\ y\isanewline
\ \ \isacommand{assume}\isamarkupfalse%
\ a{\isadigit{1}}{\isacharcolon}{\kern0pt}\ {\isachardoublequoteopen}x\ {\isasymin}\ {\isacharbraceleft}{\kern0pt}{\isadigit{0}}{\isachardot}{\kern0pt}{\isachardot}{\kern0pt}{\isacharless}{\kern0pt}p{\isacharbraceright}{\kern0pt}{\isachardoublequoteclose}\isanewline
\ \ \isacommand{assume}\isamarkupfalse%
\ a{\isadigit{2}}{\isacharcolon}{\kern0pt}\ {\isachardoublequoteopen}y\ {\isasymin}\ {\isacharbraceleft}{\kern0pt}{\isadigit{0}}{\isachardot}{\kern0pt}{\isachardot}{\kern0pt}{\isacharless}{\kern0pt}p{\isacharbraceright}{\kern0pt}{\isachardoublequoteclose}\isanewline
\ \ \isacommand{assume}\isamarkupfalse%
\ {\isachardoublequoteopen}zfact{\isacharunderscore}{\kern0pt}embed\ p\ x\ {\isacharequal}{\kern0pt}\ zfact{\isacharunderscore}{\kern0pt}embed\ p\ y{\isachardoublequoteclose}\isanewline
\ \ \isacommand{hence}\isamarkupfalse%
\ {\isachardoublequoteopen}Idl\isactrlbsub {\isasymZ}\isactrlesub \ {\isacharbraceleft}{\kern0pt}int\ p{\isacharbraceright}{\kern0pt}\ {\isacharplus}{\kern0pt}{\isachargreater}{\kern0pt}\isactrlbsub {\isasymZ}\isactrlesub \ int\ x\ {\isacharequal}{\kern0pt}\ Idl\isactrlbsub {\isasymZ}\isactrlesub \ {\isacharbraceleft}{\kern0pt}int\ p{\isacharbraceright}{\kern0pt}\ {\isacharplus}{\kern0pt}{\isachargreater}{\kern0pt}\isactrlbsub {\isasymZ}\isactrlesub \ int\ y{\isachardoublequoteclose}\isanewline
\ \ \ \ \isacommand{by}\isamarkupfalse%
\ {\isacharparenleft}{\kern0pt}simp\ add{\isacharcolon}{\kern0pt}zfact{\isacharunderscore}{\kern0pt}embed{\isacharunderscore}{\kern0pt}def{\isacharparenright}{\kern0pt}\isanewline
\ \ \isacommand{hence}\isamarkupfalse%
\ {\isachardoublequoteopen}int\ x\ {\isasymominus}\isactrlbsub {\isasymZ}\isactrlesub \ int\ y\ {\isasymin}\ Idl\isactrlbsub {\isasymZ}\isactrlesub \ {\isacharbraceleft}{\kern0pt}int\ p{\isacharbraceright}{\kern0pt}{\isachardoublequoteclose}\isanewline
\ \ \ \ \isacommand{using}\isamarkupfalse%
\ ring{\isachardot}{\kern0pt}quotient{\isacharunderscore}{\kern0pt}eq{\isacharunderscore}{\kern0pt}iff{\isacharunderscore}{\kern0pt}same{\isacharunderscore}{\kern0pt}a{\isacharunderscore}{\kern0pt}r{\isacharunderscore}{\kern0pt}cos\ \isanewline
\ \ \ \ \isacommand{by}\isamarkupfalse%
\ {\isacharparenleft}{\kern0pt}metis\ UNIV{\isacharunderscore}{\kern0pt}I\ int{\isachardot}{\kern0pt}cgenideal{\isacharunderscore}{\kern0pt}eq{\isacharunderscore}{\kern0pt}genideal\ int{\isachardot}{\kern0pt}cgenideal{\isacharunderscore}{\kern0pt}ideal\ int{\isachardot}{\kern0pt}ring{\isacharunderscore}{\kern0pt}axioms\ int{\isacharunderscore}{\kern0pt}carrier{\isacharunderscore}{\kern0pt}eq{\isacharparenright}{\kern0pt}\isanewline
\ \ \isacommand{hence}\isamarkupfalse%
\ {\isachardoublequoteopen}p\ dvd\ {\isacharparenleft}{\kern0pt}int\ x\ {\isacharminus}{\kern0pt}\ int\ y{\isacharparenright}{\kern0pt}{\isachardoublequoteclose}\ \isacommand{apply}\isamarkupfalse%
\ {\isacharparenleft}{\kern0pt}simp\ add{\isacharcolon}{\kern0pt}int{\isacharunderscore}{\kern0pt}Idl{\isacharparenright}{\kern0pt}\ \isanewline
\ \ \ \ \isacommand{using}\isamarkupfalse%
\ int{\isacharunderscore}{\kern0pt}a{\isacharunderscore}{\kern0pt}minus{\isacharunderscore}{\kern0pt}eq\ \isacommand{by}\isamarkupfalse%
\ force\isanewline
\ \ \isacommand{thus}\isamarkupfalse%
\ {\isachardoublequoteopen}x\ {\isacharequal}{\kern0pt}\ y{\isachardoublequoteclose}\ \isacommand{using}\isamarkupfalse%
\ a{\isadigit{1}}\ a{\isadigit{2}}\isanewline
\ \ \ \ \isacommand{apply}\isamarkupfalse%
\ {\isacharparenleft}{\kern0pt}simp{\isacharparenright}{\kern0pt}\ \isanewline
\ \ \ \ \isacommand{by}\isamarkupfalse%
\ {\isacharparenleft}{\kern0pt}metis\ {\isacharparenleft}{\kern0pt}full{\isacharunderscore}{\kern0pt}types{\isacharparenright}{\kern0pt}\ cancel{\isacharunderscore}{\kern0pt}comm{\isacharunderscore}{\kern0pt}monoid{\isacharunderscore}{\kern0pt}add{\isacharunderscore}{\kern0pt}class{\isachardot}{\kern0pt}diff{\isacharunderscore}{\kern0pt}cancel\ diff{\isacharunderscore}{\kern0pt}less{\isacharunderscore}{\kern0pt}mono{\isadigit{2}}\ dvd{\isacharunderscore}{\kern0pt}{\isadigit{0}}{\isacharunderscore}{\kern0pt}right\ dvd{\isacharunderscore}{\kern0pt}diff{\isacharunderscore}{\kern0pt}commute\ less{\isacharunderscore}{\kern0pt}imp{\isacharunderscore}{\kern0pt}diff{\isacharunderscore}{\kern0pt}less\ less{\isacharunderscore}{\kern0pt}imp{\isacharunderscore}{\kern0pt}of{\isacharunderscore}{\kern0pt}nat{\isacharunderscore}{\kern0pt}less\ linorder{\isacharunderscore}{\kern0pt}neqE{\isacharunderscore}{\kern0pt}nat\ of{\isacharunderscore}{\kern0pt}nat{\isacharunderscore}{\kern0pt}{\isadigit{0}}{\isacharunderscore}{\kern0pt}less{\isacharunderscore}{\kern0pt}iff\ zdiff{\isacharunderscore}{\kern0pt}int{\isacharunderscore}{\kern0pt}split\ zdvd{\isacharunderscore}{\kern0pt}not{\isacharunderscore}{\kern0pt}zless{\isacharparenright}{\kern0pt}\isanewline
\isacommand{qed}\isamarkupfalse%
%
\endisatagproof
{\isafoldproof}%
%
\isadelimproof
\isanewline
%
\endisadelimproof
\isanewline
\isacommand{lemma}\isamarkupfalse%
\ zfact{\isacharunderscore}{\kern0pt}embed{\isacharunderscore}{\kern0pt}bij{\isacharcolon}{\kern0pt}\isanewline
\ \ \isakeyword{assumes}\ {\isachardoublequoteopen}p\ {\isachargreater}{\kern0pt}\ {\isadigit{0}}{\isachardoublequoteclose}\isanewline
\ \ \isakeyword{shows}\ {\isachardoublequoteopen}bij{\isacharunderscore}{\kern0pt}betw\ {\isacharparenleft}{\kern0pt}zfact{\isacharunderscore}{\kern0pt}embed\ p{\isacharparenright}{\kern0pt}\ {\isacharbraceleft}{\kern0pt}{\isadigit{0}}{\isachardot}{\kern0pt}{\isachardot}{\kern0pt}{\isacharless}{\kern0pt}p{\isacharbraceright}{\kern0pt}\ {\isacharparenleft}{\kern0pt}carrier\ {\isacharparenleft}{\kern0pt}ZFact\ p{\isacharparenright}{\kern0pt}{\isacharparenright}{\kern0pt}{\isachardoublequoteclose}\isanewline
%
\isadelimproof
\ \ %
\endisadelimproof
%
\isatagproof
\isacommand{apply}\isamarkupfalse%
\ {\isacharparenleft}{\kern0pt}rule\ bij{\isacharunderscore}{\kern0pt}betw{\isacharunderscore}{\kern0pt}imageI{\isacharparenright}{\kern0pt}\isanewline
\ \ \isacommand{using}\isamarkupfalse%
\ zfact{\isacharunderscore}{\kern0pt}embed{\isacharunderscore}{\kern0pt}inj\ zfact{\isacharunderscore}{\kern0pt}embed{\isacharunderscore}{\kern0pt}ran\ assms\ \isacommand{by}\isamarkupfalse%
\ auto%
\endisatagproof
{\isafoldproof}%
%
\isadelimproof
\ \isanewline
%
\endisadelimproof
\isanewline
\isacommand{lemma}\isamarkupfalse%
\ zfact{\isacharunderscore}{\kern0pt}card{\isacharcolon}{\kern0pt}\isanewline
\ \ \isakeyword{assumes}\ {\isachardoublequoteopen}{\isacharparenleft}{\kern0pt}p\ {\isacharcolon}{\kern0pt}{\isacharcolon}{\kern0pt}\ nat{\isacharparenright}{\kern0pt}\ {\isachargreater}{\kern0pt}\ {\isadigit{0}}{\isachardoublequoteclose}\isanewline
\ \ \isakeyword{shows}\ {\isachardoublequoteopen}card\ {\isacharparenleft}{\kern0pt}carrier\ {\isacharparenleft}{\kern0pt}ZFact\ {\isacharparenleft}{\kern0pt}int\ p{\isacharparenright}{\kern0pt}{\isacharparenright}{\kern0pt}{\isacharparenright}{\kern0pt}\ {\isacharequal}{\kern0pt}\ p{\isachardoublequoteclose}\isanewline
%
\isadelimproof
\ \ %
\endisadelimproof
%
\isatagproof
\isacommand{apply}\isamarkupfalse%
\ {\isacharparenleft}{\kern0pt}subst\ zfact{\isacharunderscore}{\kern0pt}embed{\isacharunderscore}{\kern0pt}ran{\isacharbrackleft}{\kern0pt}OF\ assms{\isacharcomma}{\kern0pt}\ symmetric{\isacharbrackright}{\kern0pt}{\isacharparenright}{\kern0pt}\isanewline
\ \ \isacommand{by}\isamarkupfalse%
\ {\isacharparenleft}{\kern0pt}metis\ card{\isacharunderscore}{\kern0pt}atLeastLessThan\ card{\isacharunderscore}{\kern0pt}image\ diff{\isacharunderscore}{\kern0pt}zero\ zfact{\isacharunderscore}{\kern0pt}embed{\isacharunderscore}{\kern0pt}inj{\isacharbrackleft}{\kern0pt}OF\ assms{\isacharbrackright}{\kern0pt}{\isacharparenright}{\kern0pt}%
\endisatagproof
{\isafoldproof}%
%
\isadelimproof
\isanewline
%
\endisadelimproof
\isanewline
\isacommand{lemma}\isamarkupfalse%
\ zfact{\isacharunderscore}{\kern0pt}finite{\isacharcolon}{\kern0pt}\isanewline
\ \ \isakeyword{assumes}\ {\isachardoublequoteopen}{\isacharparenleft}{\kern0pt}p\ {\isacharcolon}{\kern0pt}{\isacharcolon}{\kern0pt}\ nat{\isacharparenright}{\kern0pt}\ {\isachargreater}{\kern0pt}\ {\isadigit{0}}{\isachardoublequoteclose}\isanewline
\ \ \isakeyword{shows}\ {\isachardoublequoteopen}finite\ {\isacharparenleft}{\kern0pt}carrier\ {\isacharparenleft}{\kern0pt}ZFact\ {\isacharparenleft}{\kern0pt}int\ p{\isacharparenright}{\kern0pt}{\isacharparenright}{\kern0pt}{\isacharparenright}{\kern0pt}{\isachardoublequoteclose}\isanewline
%
\isadelimproof
\ \ %
\endisadelimproof
%
\isatagproof
\isacommand{using}\isamarkupfalse%
\ zfact{\isacharunderscore}{\kern0pt}card\ \isanewline
\ \ \isacommand{by}\isamarkupfalse%
\ {\isacharparenleft}{\kern0pt}metis\ assms\ card{\isacharunderscore}{\kern0pt}ge{\isacharunderscore}{\kern0pt}{\isadigit{0}}{\isacharunderscore}{\kern0pt}finite{\isacharparenright}{\kern0pt}%
\endisatagproof
{\isafoldproof}%
%
\isadelimproof
\isanewline
%
\endisadelimproof
\isanewline
\isacommand{lemma}\isamarkupfalse%
\ finite{\isacharunderscore}{\kern0pt}domains{\isacharunderscore}{\kern0pt}are{\isacharunderscore}{\kern0pt}fields{\isacharcolon}{\kern0pt}\isanewline
\ \ \isakeyword{assumes}\ {\isachardoublequoteopen}domain\ R{\isachardoublequoteclose}\isanewline
\ \ \isakeyword{assumes}\ {\isachardoublequoteopen}finite\ {\isacharparenleft}{\kern0pt}carrier\ R{\isacharparenright}{\kern0pt}{\isachardoublequoteclose}\isanewline
\ \ \isakeyword{shows}\ {\isachardoublequoteopen}field\ R{\isachardoublequoteclose}\isanewline
%
\isadelimproof
%
\endisadelimproof
%
\isatagproof
\isacommand{proof}\isamarkupfalse%
\ {\isacharminus}{\kern0pt}\isanewline
\ \ \isacommand{interpret}\isamarkupfalse%
\ domain\ R\ \isacommand{using}\isamarkupfalse%
\ assms\ \isacommand{by}\isamarkupfalse%
\ auto\isanewline
\ \ \isacommand{have}\isamarkupfalse%
\ {\isachardoublequoteopen}Units\ R\ {\isacharequal}{\kern0pt}\ carrier\ R\ {\isacharminus}{\kern0pt}\ {\isacharbraceleft}{\kern0pt}{\isasymzero}\isactrlbsub R\isactrlesub {\isacharbraceright}{\kern0pt}{\isachardoublequoteclose}\isanewline
\ \ \isacommand{proof}\isamarkupfalse%
\ \isanewline
\ \ \ \ \isacommand{have}\isamarkupfalse%
\ {\isachardoublequoteopen}Units\ R\ {\isasymsubseteq}\ carrier\ R{\isachardoublequoteclose}\ \isacommand{by}\isamarkupfalse%
\ {\isacharparenleft}{\kern0pt}simp\ add{\isacharcolon}{\kern0pt}Units{\isacharunderscore}{\kern0pt}def{\isacharparenright}{\kern0pt}\ \isanewline
\ \ \ \ \isacommand{moreover}\isamarkupfalse%
\ \isacommand{have}\isamarkupfalse%
\ {\isachardoublequoteopen}{\isasymzero}\isactrlbsub R\isactrlesub \ {\isasymnotin}\ Units\ R{\isachardoublequoteclose}\isanewline
\ \ \ \ \ \ \isacommand{by}\isamarkupfalse%
\ {\isacharparenleft}{\kern0pt}meson\ assms{\isacharparenleft}{\kern0pt}{\isadigit{1}}{\isacharparenright}{\kern0pt}\ domain{\isachardot}{\kern0pt}zero{\isacharunderscore}{\kern0pt}is{\isacharunderscore}{\kern0pt}prime{\isacharparenleft}{\kern0pt}{\isadigit{1}}{\isacharparenright}{\kern0pt}\ primeE{\isacharparenright}{\kern0pt}\isanewline
\ \ \ \ \isacommand{ultimately}\isamarkupfalse%
\ \isacommand{show}\isamarkupfalse%
\ {\isachardoublequoteopen}Units\ R\ {\isasymsubseteq}\ carrier\ R\ {\isacharminus}{\kern0pt}\ {\isacharbraceleft}{\kern0pt}{\isasymzero}\isactrlbsub R\isactrlesub {\isacharbraceright}{\kern0pt}{\isachardoublequoteclose}\ \isacommand{by}\isamarkupfalse%
\ blast\isanewline
\ \ \isacommand{next}\isamarkupfalse%
\isanewline
\ \ \ \ \isacommand{show}\isamarkupfalse%
\ {\isachardoublequoteopen}carrier\ R\ {\isacharminus}{\kern0pt}\ {\isacharbraceleft}{\kern0pt}{\isasymzero}\isactrlbsub R\isactrlesub {\isacharbraceright}{\kern0pt}\ {\isasymsubseteq}\ Units\ R{\isachardoublequoteclose}\isanewline
\ \ \ \ \isacommand{proof}\isamarkupfalse%
\isanewline
\ \ \ \ \ \ \isacommand{fix}\isamarkupfalse%
\ x\isanewline
\ \ \ \ \ \ \isacommand{assume}\isamarkupfalse%
\ a{\isacharcolon}{\kern0pt}{\isachardoublequoteopen}x\ {\isasymin}\ carrier\ R\ {\isacharminus}{\kern0pt}\ {\isacharbraceleft}{\kern0pt}{\isasymzero}\isactrlbsub R\isactrlesub {\isacharbraceright}{\kern0pt}{\isachardoublequoteclose}\isanewline
\ \ \ \ \ \ \isacommand{define}\isamarkupfalse%
\ f\ \isakeyword{where}\ {\isachardoublequoteopen}f\ {\isacharequal}{\kern0pt}\ {\isacharparenleft}{\kern0pt}{\isasymlambda}y{\isachardot}{\kern0pt}\ y\ {\isasymotimes}\isactrlbsub R\isactrlesub \ x{\isacharparenright}{\kern0pt}{\isachardoublequoteclose}\isanewline
\ \ \ \ \ \ \isacommand{have}\isamarkupfalse%
\ {\isachardoublequoteopen}inj{\isacharunderscore}{\kern0pt}on\ f\ {\isacharparenleft}{\kern0pt}carrier\ R{\isacharparenright}{\kern0pt}{\isachardoublequoteclose}\ \isacommand{apply}\isamarkupfalse%
\ {\isacharparenleft}{\kern0pt}simp\ add{\isacharcolon}{\kern0pt}inj{\isacharunderscore}{\kern0pt}on{\isacharunderscore}{\kern0pt}def\ f{\isacharunderscore}{\kern0pt}def{\isacharparenright}{\kern0pt}\isanewline
\ \ \ \ \ \ \ \ \isacommand{by}\isamarkupfalse%
\ {\isacharparenleft}{\kern0pt}metis\ DiffD{\isadigit{1}}\ DiffD{\isadigit{2}}\ a\ assms{\isacharparenleft}{\kern0pt}{\isadigit{1}}{\isacharparenright}{\kern0pt}\ domain{\isachardot}{\kern0pt}m{\isacharunderscore}{\kern0pt}rcancel\ insertI{\isadigit{1}}{\isacharparenright}{\kern0pt}\isanewline
\ \ \ \ \ \ \isacommand{hence}\isamarkupfalse%
\ {\isachardoublequoteopen}card\ {\isacharparenleft}{\kern0pt}carrier\ R{\isacharparenright}{\kern0pt}\ {\isacharequal}{\kern0pt}\ card\ {\isacharparenleft}{\kern0pt}f\ {\isacharbackquote}{\kern0pt}\ carrier\ R{\isacharparenright}{\kern0pt}{\isachardoublequoteclose}\isanewline
\ \ \ \ \ \ \ \ \isacommand{by}\isamarkupfalse%
\ {\isacharparenleft}{\kern0pt}metis\ card{\isacharunderscore}{\kern0pt}image{\isacharparenright}{\kern0pt}\isanewline
\ \ \ \ \ \ \isacommand{moreover}\isamarkupfalse%
\ \isacommand{have}\isamarkupfalse%
\ {\isachardoublequoteopen}f\ {\isacharbackquote}{\kern0pt}\ carrier\ R\ {\isasymsubseteq}\ carrier\ R{\isachardoublequoteclose}\isanewline
\ \ \ \ \ \ \ \ \isacommand{apply}\isamarkupfalse%
\ {\isacharparenleft}{\kern0pt}rule\ image{\isacharunderscore}{\kern0pt}subsetI{\isacharparenright}{\kern0pt}\ \isacommand{apply}\isamarkupfalse%
\ {\isacharparenleft}{\kern0pt}simp\ add{\isacharcolon}{\kern0pt}f{\isacharunderscore}{\kern0pt}def{\isacharparenright}{\kern0pt}\ \isacommand{using}\isamarkupfalse%
\ a\isanewline
\ \ \ \ \ \ \ \ \isacommand{by}\isamarkupfalse%
\ {\isacharparenleft}{\kern0pt}simp\ add{\isacharcolon}{\kern0pt}\ ring{\isachardot}{\kern0pt}ring{\isacharunderscore}{\kern0pt}simprules{\isacharparenleft}{\kern0pt}{\isadigit{5}}{\isacharparenright}{\kern0pt}{\isacharparenright}{\kern0pt}\isanewline
\ \ \ \ \ \ \isacommand{ultimately}\isamarkupfalse%
\ \isacommand{have}\isamarkupfalse%
\ {\isachardoublequoteopen}f\ {\isacharbackquote}{\kern0pt}\ carrier\ R\ {\isacharequal}{\kern0pt}\ carrier\ R{\isachardoublequoteclose}\ \isacommand{using}\isamarkupfalse%
\ card{\isacharunderscore}{\kern0pt}subset{\isacharunderscore}{\kern0pt}eq\ assms{\isacharparenleft}{\kern0pt}{\isadigit{2}}{\isacharparenright}{\kern0pt}\ \isacommand{by}\isamarkupfalse%
\ metis\isanewline
\ \ \ \ \ \ \isacommand{moreover}\isamarkupfalse%
\ \isacommand{have}\isamarkupfalse%
\ {\isachardoublequoteopen}{\isasymone}\isactrlbsub R\isactrlesub \ {\isasymin}\ carrier\ R{\isachardoublequoteclose}\ \isacommand{by}\isamarkupfalse%
\ simp\isanewline
\ \ \ \ \ \ \isacommand{ultimately}\isamarkupfalse%
\ \isacommand{have}\isamarkupfalse%
\ {\isachardoublequoteopen}{\isasymexists}y\ {\isasymin}\ carrier\ R{\isachardot}{\kern0pt}\ f\ y\ {\isacharequal}{\kern0pt}\ {\isasymone}\isactrlbsub R\isactrlesub {\isachardoublequoteclose}\ \isanewline
\ \ \ \ \ \ \ \ \isacommand{by}\isamarkupfalse%
\ {\isacharparenleft}{\kern0pt}metis\ image{\isacharunderscore}{\kern0pt}iff{\isacharparenright}{\kern0pt}\isanewline
\ \ \ \ \ \ \isacommand{then}\isamarkupfalse%
\ \isacommand{obtain}\isamarkupfalse%
\ y\ \isakeyword{where}\ y{\isacharunderscore}{\kern0pt}carrier{\isacharcolon}{\kern0pt}\ {\isachardoublequoteopen}y\ {\isasymin}\ carrier\ R{\isachardoublequoteclose}\ \isakeyword{and}\ y{\isacharunderscore}{\kern0pt}left{\isacharunderscore}{\kern0pt}inv{\isacharcolon}{\kern0pt}\ {\isachardoublequoteopen}y\ {\isasymotimes}\isactrlbsub R\isactrlesub \ x\ {\isacharequal}{\kern0pt}\ {\isasymone}\isactrlbsub R\isactrlesub {\isachardoublequoteclose}\ \isanewline
\ \ \ \ \ \ \ \ \isacommand{using}\isamarkupfalse%
\ f{\isacharunderscore}{\kern0pt}def\ \isacommand{by}\isamarkupfalse%
\ blast\isanewline
\ \ \ \ \ \ \isacommand{hence}\isamarkupfalse%
\ \ y{\isacharunderscore}{\kern0pt}right{\isacharunderscore}{\kern0pt}inv{\isacharcolon}{\kern0pt}\ {\isachardoublequoteopen}x\ {\isasymotimes}\isactrlbsub R\isactrlesub \ y\ {\isacharequal}{\kern0pt}\ {\isasymone}\isactrlbsub R\isactrlesub {\isachardoublequoteclose}\ \isacommand{using}\isamarkupfalse%
\ assms{\isacharparenleft}{\kern0pt}{\isadigit{1}}{\isacharparenright}{\kern0pt}\ a\ \isanewline
\ \ \ \ \ \ \ \ \isacommand{by}\isamarkupfalse%
\ {\isacharparenleft}{\kern0pt}metis\ DiffD{\isadigit{1}}\ a\ cring{\isachardot}{\kern0pt}cring{\isacharunderscore}{\kern0pt}simprules{\isacharparenleft}{\kern0pt}{\isadigit{1}}{\isadigit{4}}{\isacharparenright}{\kern0pt}\ domain{\isachardot}{\kern0pt}axioms{\isacharparenleft}{\kern0pt}{\isadigit{1}}{\isacharparenright}{\kern0pt}{\isacharparenright}{\kern0pt}\isanewline
\ \ \ \ \ \ \isacommand{show}\isamarkupfalse%
\ {\isachardoublequoteopen}x\ {\isasymin}\ Units\ R{\isachardoublequoteclose}\ \isacommand{using}\isamarkupfalse%
\ y{\isacharunderscore}{\kern0pt}carrier\ y{\isacharunderscore}{\kern0pt}left{\isacharunderscore}{\kern0pt}inv\ y{\isacharunderscore}{\kern0pt}right{\isacharunderscore}{\kern0pt}inv\isanewline
\ \ \ \ \ \ \ \ \isacommand{by}\isamarkupfalse%
\ {\isacharparenleft}{\kern0pt}metis\ DiffD{\isadigit{1}}\ a\ assms{\isacharparenleft}{\kern0pt}{\isadigit{1}}{\isacharparenright}{\kern0pt}\ cring{\isachardot}{\kern0pt}divides{\isacharunderscore}{\kern0pt}one\ domain{\isachardot}{\kern0pt}axioms{\isacharparenleft}{\kern0pt}{\isadigit{1}}{\isacharparenright}{\kern0pt}\ factor{\isacharunderscore}{\kern0pt}def{\isacharparenright}{\kern0pt}\isanewline
\ \ \ \ \isacommand{qed}\isamarkupfalse%
\isanewline
\ \ \isacommand{qed}\isamarkupfalse%
\isanewline
\ \ \isacommand{then}\isamarkupfalse%
\ \isacommand{show}\isamarkupfalse%
\ {\isachardoublequoteopen}field\ R{\isachardoublequoteclose}\ \isacommand{by}\isamarkupfalse%
\ {\isacharparenleft}{\kern0pt}simp\ add{\isacharcolon}{\kern0pt}\ assms{\isacharparenleft}{\kern0pt}{\isadigit{1}}{\isacharparenright}{\kern0pt}\ field{\isachardot}{\kern0pt}intro\ field{\isacharunderscore}{\kern0pt}axioms{\isachardot}{\kern0pt}intro{\isacharparenright}{\kern0pt}\isanewline
\isacommand{qed}\isamarkupfalse%
%
\endisatagproof
{\isafoldproof}%
%
\isadelimproof
\isanewline
%
\endisadelimproof
\isanewline
\isacommand{lemma}\isamarkupfalse%
\ zfact{\isacharunderscore}{\kern0pt}prime{\isacharunderscore}{\kern0pt}is{\isacharunderscore}{\kern0pt}field{\isacharcolon}{\kern0pt}\isanewline
\ \ \isakeyword{assumes}\ {\isachardoublequoteopen}prime\ {\isacharparenleft}{\kern0pt}p\ {\isacharcolon}{\kern0pt}{\isacharcolon}{\kern0pt}\ nat{\isacharparenright}{\kern0pt}{\isachardoublequoteclose}\ \isanewline
\ \ \isakeyword{shows}\ {\isachardoublequoteopen}field\ {\isacharparenleft}{\kern0pt}ZFact\ {\isacharparenleft}{\kern0pt}int\ p{\isacharparenright}{\kern0pt}{\isacharparenright}{\kern0pt}{\isachardoublequoteclose}\isanewline
%
\isadelimproof
%
\endisadelimproof
%
\isatagproof
\isacommand{proof}\isamarkupfalse%
\ {\isacharminus}{\kern0pt}\isanewline
\ \ \isacommand{define}\isamarkupfalse%
\ q\ \isakeyword{where}\ {\isachardoublequoteopen}q\ {\isacharequal}{\kern0pt}\ int\ p{\isachardoublequoteclose}\isanewline
\ \ \isacommand{have}\isamarkupfalse%
\ {\isachardoublequoteopen}finite\ {\isacharparenleft}{\kern0pt}carrier\ {\isacharparenleft}{\kern0pt}ZFact\ q{\isacharparenright}{\kern0pt}{\isacharparenright}{\kern0pt}{\isachardoublequoteclose}\ \isacommand{using}\isamarkupfalse%
\ zfact{\isacharunderscore}{\kern0pt}finite\ assms\ q{\isacharunderscore}{\kern0pt}def\ prime{\isacharunderscore}{\kern0pt}gt{\isacharunderscore}{\kern0pt}{\isadigit{0}}{\isacharunderscore}{\kern0pt}nat\ \isacommand{by}\isamarkupfalse%
\ blast\isanewline
\ \ \isacommand{moreover}\isamarkupfalse%
\ \isacommand{have}\isamarkupfalse%
\ {\isachardoublequoteopen}domain\ {\isacharparenleft}{\kern0pt}ZFact\ q{\isacharparenright}{\kern0pt}{\isachardoublequoteclose}\ \isacommand{using}\isamarkupfalse%
\ ZFact{\isacharunderscore}{\kern0pt}prime{\isacharunderscore}{\kern0pt}is{\isacharunderscore}{\kern0pt}domain\ assms\ q{\isacharunderscore}{\kern0pt}def\ \isacommand{by}\isamarkupfalse%
\ auto\isanewline
\ \ \isacommand{ultimately}\isamarkupfalse%
\ \isacommand{show}\isamarkupfalse%
\ {\isacharquery}{\kern0pt}thesis\ \isacommand{using}\isamarkupfalse%
\ finite{\isacharunderscore}{\kern0pt}domains{\isacharunderscore}{\kern0pt}are{\isacharunderscore}{\kern0pt}fields\ q{\isacharunderscore}{\kern0pt}def\ \isacommand{by}\isamarkupfalse%
\ blast\isanewline
\isacommand{qed}\isamarkupfalse%
%
\endisatagproof
{\isafoldproof}%
%
\isadelimproof
\isanewline
%
\endisadelimproof
%
\isadelimtheory
\isanewline
%
\endisadelimtheory
%
\isatagtheory
\isacommand{end}\isamarkupfalse%
%
\endisatagtheory
{\isafoldtheory}%
%
\isadelimtheory
%
\endisadelimtheory
%
\end{isabellebody}%
\endinput
%:%file=Field.tex%:%
%:%11=1%:%
%:%27=3%:%
%:%28=3%:%
%:%29=4%:%
%:%30=5%:%
%:%39=7%:%
%:%40=8%:%
%:%41=9%:%
%:%42=10%:%
%:%43=11%:%
%:%44=12%:%
%:%45=13%:%
%:%46=14%:%
%:%47=15%:%
%:%48=16%:%
%:%49=17%:%
%:%51=19%:%
%:%52=19%:%
%:%53=20%:%
%:%54=21%:%
%:%55=22%:%
%:%56=22%:%
%:%57=23%:%
%:%58=24%:%
%:%65=25%:%
%:%66=25%:%
%:%67=26%:%
%:%68=26%:%
%:%69=27%:%
%:%70=27%:%
%:%71=28%:%
%:%72=28%:%
%:%73=29%:%
%:%74=29%:%
%:%75=30%:%
%:%76=30%:%
%:%77=30%:%
%:%78=30%:%
%:%79=31%:%
%:%80=31%:%
%:%81=32%:%
%:%82=32%:%
%:%83=33%:%
%:%84=33%:%
%:%85=33%:%
%:%86=33%:%
%:%87=34%:%
%:%88=34%:%
%:%89=35%:%
%:%90=35%:%
%:%91=35%:%
%:%92=36%:%
%:%93=36%:%
%:%94=37%:%
%:%95=37%:%
%:%96=38%:%
%:%97=38%:%
%:%98=39%:%
%:%99=39%:%
%:%100=40%:%
%:%101=40%:%
%:%102=40%:%
%:%103=40%:%
%:%104=41%:%
%:%105=41%:%
%:%106=42%:%
%:%107=42%:%
%:%108=43%:%
%:%109=43%:%
%:%110=44%:%
%:%111=44%:%
%:%112=45%:%
%:%113=45%:%
%:%114=46%:%
%:%115=46%:%
%:%116=46%:%
%:%117=47%:%
%:%118=47%:%
%:%119=47%:%
%:%120=48%:%
%:%121=48%:%
%:%122=49%:%
%:%123=49%:%
%:%124=50%:%
%:%125=50%:%
%:%126=50%:%
%:%127=51%:%
%:%128=51%:%
%:%129=51%:%
%:%130=51%:%
%:%131=52%:%
%:%132=52%:%
%:%133=52%:%
%:%134=52%:%
%:%135=53%:%
%:%136=53%:%
%:%137=54%:%
%:%138=55%:%
%:%139=56%:%
%:%140=56%:%
%:%141=57%:%
%:%142=57%:%
%:%143=57%:%
%:%144=58%:%
%:%145=58%:%
%:%146=58%:%
%:%147=58%:%
%:%148=59%:%
%:%149=59%:%
%:%150=59%:%
%:%151=60%:%
%:%152=60%:%
%:%153=61%:%
%:%154=61%:%
%:%155=61%:%
%:%156=61%:%
%:%157=61%:%
%:%158=62%:%
%:%164=62%:%
%:%167=63%:%
%:%168=64%:%
%:%169=64%:%
%:%170=65%:%
%:%171=66%:%
%:%178=67%:%
%:%179=67%:%
%:%180=68%:%
%:%181=68%:%
%:%182=69%:%
%:%183=69%:%
%:%184=70%:%
%:%185=70%:%
%:%186=71%:%
%:%187=71%:%
%:%188=72%:%
%:%189=72%:%
%:%190=73%:%
%:%191=73%:%
%:%192=74%:%
%:%193=74%:%
%:%194=75%:%
%:%195=75%:%
%:%196=76%:%
%:%197=76%:%
%:%198=77%:%
%:%199=77%:%
%:%200=78%:%
%:%201=78%:%
%:%202=78%:%
%:%203=79%:%
%:%204=79%:%
%:%205=79%:%
%:%206=80%:%
%:%207=80%:%
%:%208=80%:%
%:%209=81%:%
%:%210=81%:%
%:%211=82%:%
%:%212=82%:%
%:%213=83%:%
%:%219=83%:%
%:%222=84%:%
%:%223=85%:%
%:%224=85%:%
%:%225=86%:%
%:%226=87%:%
%:%229=88%:%
%:%233=88%:%
%:%234=88%:%
%:%235=89%:%
%:%236=89%:%
%:%237=89%:%
%:%242=89%:%
%:%245=90%:%
%:%246=91%:%
%:%247=91%:%
%:%248=92%:%
%:%249=93%:%
%:%252=94%:%
%:%256=94%:%
%:%257=94%:%
%:%258=95%:%
%:%259=95%:%
%:%264=95%:%
%:%267=96%:%
%:%268=97%:%
%:%269=97%:%
%:%270=98%:%
%:%271=99%:%
%:%274=100%:%
%:%278=100%:%
%:%279=100%:%
%:%280=101%:%
%:%281=101%:%
%:%286=101%:%
%:%289=102%:%
%:%290=103%:%
%:%291=103%:%
%:%292=104%:%
%:%293=105%:%
%:%294=106%:%
%:%301=107%:%
%:%302=107%:%
%:%303=108%:%
%:%304=108%:%
%:%305=108%:%
%:%306=108%:%
%:%307=109%:%
%:%308=109%:%
%:%309=110%:%
%:%310=110%:%
%:%311=111%:%
%:%312=111%:%
%:%313=111%:%
%:%314=112%:%
%:%315=112%:%
%:%316=112%:%
%:%317=113%:%
%:%318=113%:%
%:%319=114%:%
%:%320=114%:%
%:%321=114%:%
%:%322=114%:%
%:%323=115%:%
%:%324=115%:%
%:%325=116%:%
%:%326=116%:%
%:%327=117%:%
%:%328=117%:%
%:%329=118%:%
%:%330=118%:%
%:%331=119%:%
%:%332=119%:%
%:%333=120%:%
%:%334=120%:%
%:%335=121%:%
%:%336=121%:%
%:%337=121%:%
%:%338=122%:%
%:%339=122%:%
%:%340=123%:%
%:%341=123%:%
%:%342=124%:%
%:%343=124%:%
%:%344=125%:%
%:%345=125%:%
%:%346=125%:%
%:%347=126%:%
%:%348=126%:%
%:%349=126%:%
%:%350=126%:%
%:%351=127%:%
%:%352=127%:%
%:%353=128%:%
%:%354=128%:%
%:%355=128%:%
%:%356=128%:%
%:%357=128%:%
%:%358=129%:%
%:%359=129%:%
%:%360=129%:%
%:%361=129%:%
%:%362=130%:%
%:%363=130%:%
%:%364=130%:%
%:%365=131%:%
%:%366=131%:%
%:%367=132%:%
%:%368=132%:%
%:%369=132%:%
%:%370=133%:%
%:%371=133%:%
%:%372=133%:%
%:%373=134%:%
%:%374=134%:%
%:%375=134%:%
%:%376=135%:%
%:%377=135%:%
%:%378=136%:%
%:%379=136%:%
%:%380=136%:%
%:%381=137%:%
%:%382=137%:%
%:%383=138%:%
%:%384=138%:%
%:%385=139%:%
%:%386=139%:%
%:%387=140%:%
%:%388=140%:%
%:%389=140%:%
%:%390=140%:%
%:%391=141%:%
%:%397=141%:%
%:%400=142%:%
%:%401=143%:%
%:%402=143%:%
%:%403=144%:%
%:%404=145%:%
%:%411=146%:%
%:%412=146%:%
%:%413=147%:%
%:%414=147%:%
%:%415=148%:%
%:%416=148%:%
%:%417=148%:%
%:%418=148%:%
%:%419=149%:%
%:%420=149%:%
%:%421=149%:%
%:%422=149%:%
%:%423=149%:%
%:%424=150%:%
%:%425=150%:%
%:%426=150%:%
%:%427=150%:%
%:%428=150%:%
%:%429=151%:%
%:%435=151%:%
%:%440=152%:%
%:%445=153%:%

%
\begin{isabellebody}%
\setisabellecontext{Float{\isacharunderscore}{\kern0pt}Ext}%
%
\isadelimdocument
%
\endisadelimdocument
%
\isatagdocument
%
\isamarkupsection{Float%
}
\isamarkuptrue%
%
\endisatagdocument
{\isafolddocument}%
%
\isadelimdocument
%
\endisadelimdocument
%
\begin{isamarkuptext}%
This section contains results about floating point numbers in addition to "HOL-Library.Float"%
\end{isamarkuptext}\isamarkuptrue%
%
\isadelimtheory
%
\endisadelimtheory
%
\isatagtheory
\isacommand{theory}\isamarkupfalse%
\ Float{\isacharunderscore}{\kern0pt}Ext\isanewline
\ \ \isakeyword{imports}\ {\isachardoublequoteopen}HOL{\isacharminus}{\kern0pt}Library{\isachardot}{\kern0pt}Float{\isachardoublequoteclose}\ Encoding\isanewline
\isakeyword{begin}%
\endisatagtheory
{\isafoldtheory}%
%
\isadelimtheory
%
\endisadelimtheory
\isanewline
\isanewline
\isacommand{lemma}\isamarkupfalse%
\ round{\isacharunderscore}{\kern0pt}down{\isacharunderscore}{\kern0pt}ge{\isacharcolon}{\kern0pt}\isanewline
\ \ {\isachardoublequoteopen}x\ {\isasymle}\ round{\isacharunderscore}{\kern0pt}down\ prec\ x\ {\isacharplus}{\kern0pt}\ {\isadigit{2}}\ powr\ {\isacharparenleft}{\kern0pt}{\isacharminus}{\kern0pt}prec{\isacharparenright}{\kern0pt}{\isachardoublequoteclose}\isanewline
%
\isadelimproof
\ \ %
\endisadelimproof
%
\isatagproof
\isacommand{using}\isamarkupfalse%
\ round{\isacharunderscore}{\kern0pt}down{\isacharunderscore}{\kern0pt}correct\ \isacommand{by}\isamarkupfalse%
\ {\isacharparenleft}{\kern0pt}simp{\isacharcomma}{\kern0pt}\ meson\ diff{\isacharunderscore}{\kern0pt}diff{\isacharunderscore}{\kern0pt}eq\ diff{\isacharunderscore}{\kern0pt}eq{\isacharunderscore}{\kern0pt}diff{\isacharunderscore}{\kern0pt}less{\isacharunderscore}{\kern0pt}eq{\isacharparenright}{\kern0pt}%
\endisatagproof
{\isafoldproof}%
%
\isadelimproof
\isanewline
%
\endisadelimproof
\isanewline
\isacommand{lemma}\isamarkupfalse%
\ truncate{\isacharunderscore}{\kern0pt}down{\isacharunderscore}{\kern0pt}ge{\isacharcolon}{\kern0pt}\isanewline
\ \ {\isachardoublequoteopen}x\ {\isasymle}\ truncate{\isacharunderscore}{\kern0pt}down\ prec\ x\ {\isacharplus}{\kern0pt}\ abs\ x\ {\isacharasterisk}{\kern0pt}\ {\isadigit{2}}\ powr\ {\isacharparenleft}{\kern0pt}{\isacharminus}{\kern0pt}prec{\isacharparenright}{\kern0pt}{\isachardoublequoteclose}\isanewline
%
\isadelimproof
%
\endisadelimproof
%
\isatagproof
\isacommand{proof}\isamarkupfalse%
\ {\isacharparenleft}{\kern0pt}cases\ {\isachardoublequoteopen}abs\ x\ {\isachargreater}{\kern0pt}\ {\isadigit{0}}{\isachardoublequoteclose}{\isacharparenright}{\kern0pt}\isanewline
\ \ \isacommand{case}\isamarkupfalse%
\ True\isanewline
\ \ \isacommand{have}\isamarkupfalse%
\ {\isachardoublequoteopen}x\ {\isasymle}\ round{\isacharunderscore}{\kern0pt}down\ {\isacharparenleft}{\kern0pt}int\ prec\ {\isacharminus}{\kern0pt}\ {\isasymlfloor}log\ {\isadigit{2}}\ {\isasymbar}x{\isasymbar}{\isasymrfloor}{\isacharparenright}{\kern0pt}\ x\ {\isacharplus}{\kern0pt}\ {\isadigit{2}}\ powr\ {\isacharparenleft}{\kern0pt}{\isacharminus}{\kern0pt}real{\isacharunderscore}{\kern0pt}of{\isacharunderscore}{\kern0pt}int{\isacharparenleft}{\kern0pt}int\ prec\ {\isacharminus}{\kern0pt}\ {\isasymlfloor}log\ {\isadigit{2}}\ {\isasymbar}x{\isasymbar}{\isasymrfloor}{\isacharparenright}{\kern0pt}{\isacharparenright}{\kern0pt}{\isachardoublequoteclose}\isanewline
\ \ \ \ \isacommand{by}\isamarkupfalse%
\ {\isacharparenleft}{\kern0pt}rule\ round{\isacharunderscore}{\kern0pt}down{\isacharunderscore}{\kern0pt}ge{\isacharparenright}{\kern0pt}\isanewline
\ \ \isacommand{also}\isamarkupfalse%
\ \isacommand{have}\isamarkupfalse%
\ {\isachardoublequoteopen}{\isachardot}{\kern0pt}{\isachardot}{\kern0pt}{\isachardot}{\kern0pt}\ {\isasymle}\ truncate{\isacharunderscore}{\kern0pt}down\ prec\ x\ {\isacharplus}{\kern0pt}\ abs\ x\ {\isacharasterisk}{\kern0pt}\ {\isadigit{2}}\ powr\ {\isacharparenleft}{\kern0pt}{\isacharminus}{\kern0pt}prec{\isacharparenright}{\kern0pt}{\isachardoublequoteclose}\isanewline
\ \ \ \ \isacommand{apply}\isamarkupfalse%
\ {\isacharparenleft}{\kern0pt}rule\ add{\isacharunderscore}{\kern0pt}mono{\isacharparenright}{\kern0pt}\isanewline
\ \ \ \ \ \isacommand{apply}\isamarkupfalse%
\ {\isacharparenleft}{\kern0pt}simp\ add{\isacharcolon}{\kern0pt}truncate{\isacharunderscore}{\kern0pt}down{\isacharunderscore}{\kern0pt}def{\isacharparenright}{\kern0pt}\isanewline
\ \ \ \ \isacommand{apply}\isamarkupfalse%
\ {\isacharparenleft}{\kern0pt}subst\ of{\isacharunderscore}{\kern0pt}int{\isacharunderscore}{\kern0pt}diff{\isacharcomma}{\kern0pt}\ simp{\isacharparenright}{\kern0pt}\isanewline
\ \ \ \ \isacommand{apply}\isamarkupfalse%
\ {\isacharparenleft}{\kern0pt}subst\ powr{\isacharunderscore}{\kern0pt}diff{\isacharparenright}{\kern0pt}\isanewline
\ \ \ \ \isacommand{apply}\isamarkupfalse%
\ {\isacharparenleft}{\kern0pt}subst\ pos{\isacharunderscore}{\kern0pt}divide{\isacharunderscore}{\kern0pt}le{\isacharunderscore}{\kern0pt}eq{\isacharcomma}{\kern0pt}\ simp{\isacharparenright}{\kern0pt}\isanewline
\ \ \ \ \isacommand{apply}\isamarkupfalse%
\ {\isacharparenleft}{\kern0pt}subst\ mult{\isachardot}{\kern0pt}assoc{\isacharparenright}{\kern0pt}\isanewline
\ \ \ \ \isacommand{apply}\isamarkupfalse%
\ {\isacharparenleft}{\kern0pt}subst\ powr{\isacharunderscore}{\kern0pt}add{\isacharbrackleft}{\kern0pt}symmetric{\isacharbrackright}{\kern0pt}{\isacharcomma}{\kern0pt}\ simp{\isacharparenright}{\kern0pt}\isanewline
\ \ \ \ \isacommand{apply}\isamarkupfalse%
\ {\isacharparenleft}{\kern0pt}subst\ le{\isacharunderscore}{\kern0pt}log{\isacharunderscore}{\kern0pt}iff{\isacharbrackleft}{\kern0pt}symmetric{\isacharbrackright}{\kern0pt}{\isacharcomma}{\kern0pt}\ simp{\isacharcomma}{\kern0pt}\ metis\ True{\isacharparenright}{\kern0pt}\isanewline
\ \ \ \ \isacommand{by}\isamarkupfalse%
\ auto\isanewline
\ \ \isacommand{finally}\isamarkupfalse%
\ \isacommand{show}\isamarkupfalse%
\ {\isacharquery}{\kern0pt}thesis\ \isacommand{by}\isamarkupfalse%
\ simp\isanewline
\isacommand{next}\isamarkupfalse%
\isanewline
\ \ \isacommand{case}\isamarkupfalse%
\ False\isanewline
\ \ \isacommand{then}\isamarkupfalse%
\ \isacommand{show}\isamarkupfalse%
\ {\isacharquery}{\kern0pt}thesis\ \isacommand{by}\isamarkupfalse%
\ simp\isanewline
\isacommand{qed}\isamarkupfalse%
%
\endisatagproof
{\isafoldproof}%
%
\isadelimproof
\isanewline
%
\endisadelimproof
\isanewline
\isacommand{lemma}\isamarkupfalse%
\ truncate{\isacharunderscore}{\kern0pt}down{\isacharunderscore}{\kern0pt}pos{\isacharcolon}{\kern0pt}\isanewline
\ \ \isakeyword{assumes}\ {\isachardoublequoteopen}x\ {\isasymge}\ {\isadigit{0}}{\isachardoublequoteclose}\isanewline
\ \ \isakeyword{shows}\ {\isachardoublequoteopen}x\ {\isacharasterisk}{\kern0pt}\ {\isacharparenleft}{\kern0pt}{\isadigit{1}}\ {\isacharminus}{\kern0pt}\ {\isadigit{2}}\ powr\ {\isacharparenleft}{\kern0pt}{\isacharminus}{\kern0pt}prec{\isacharparenright}{\kern0pt}{\isacharparenright}{\kern0pt}\ {\isasymle}\ truncate{\isacharunderscore}{\kern0pt}down\ prec\ x{\isachardoublequoteclose}\isanewline
%
\isadelimproof
\ \ %
\endisadelimproof
%
\isatagproof
\isacommand{apply}\isamarkupfalse%
\ {\isacharparenleft}{\kern0pt}simp\ add{\isacharcolon}{\kern0pt}right{\isacharunderscore}{\kern0pt}diff{\isacharunderscore}{\kern0pt}distrib\ diff{\isacharunderscore}{\kern0pt}le{\isacharunderscore}{\kern0pt}eq{\isacharparenright}{\kern0pt}\isanewline
\ \ \isacommand{by}\isamarkupfalse%
\ {\isacharparenleft}{\kern0pt}metis\ truncate{\isacharunderscore}{\kern0pt}down{\isacharunderscore}{\kern0pt}ge\ assms\ \ abs{\isacharunderscore}{\kern0pt}of{\isacharunderscore}{\kern0pt}nonneg{\isacharparenright}{\kern0pt}%
\endisatagproof
{\isafoldproof}%
%
\isadelimproof
\isanewline
%
\endisadelimproof
\isanewline
\isacommand{lemma}\isamarkupfalse%
\ truncate{\isacharunderscore}{\kern0pt}down{\isacharunderscore}{\kern0pt}eq{\isacharcolon}{\kern0pt}\isanewline
\ \ \isakeyword{assumes}\ {\isachardoublequoteopen}truncate{\isacharunderscore}{\kern0pt}down\ r\ x\ {\isacharequal}{\kern0pt}\ truncate{\isacharunderscore}{\kern0pt}down\ r\ y{\isachardoublequoteclose}\isanewline
\ \ \isakeyword{shows}\ {\isachardoublequoteopen}abs\ {\isacharparenleft}{\kern0pt}x{\isacharminus}{\kern0pt}y{\isacharparenright}{\kern0pt}\ {\isasymle}\ max\ {\isacharparenleft}{\kern0pt}abs\ x{\isacharparenright}{\kern0pt}\ {\isacharparenleft}{\kern0pt}abs\ y{\isacharparenright}{\kern0pt}\ {\isacharasterisk}{\kern0pt}\ {\isadigit{2}}\ powr\ {\isacharparenleft}{\kern0pt}{\isacharminus}{\kern0pt}real\ r{\isacharparenright}{\kern0pt}{\isachardoublequoteclose}\isanewline
%
\isadelimproof
%
\endisadelimproof
%
\isatagproof
\isacommand{proof}\isamarkupfalse%
\ {\isacharminus}{\kern0pt}\ \isanewline
\ \ \isacommand{have}\isamarkupfalse%
\ {\isachardoublequoteopen}x\ {\isacharminus}{\kern0pt}\ y\ {\isasymle}\ truncate{\isacharunderscore}{\kern0pt}down\ r\ x\ {\isacharplus}{\kern0pt}\ abs\ x\ {\isacharasterisk}{\kern0pt}\ {\isadigit{2}}\ powr\ {\isacharparenleft}{\kern0pt}{\isacharminus}{\kern0pt}real\ r{\isacharparenright}{\kern0pt}\ {\isacharminus}{\kern0pt}\ y{\isachardoublequoteclose}\isanewline
\ \ \ \ \isacommand{by}\isamarkupfalse%
\ {\isacharparenleft}{\kern0pt}rule\ diff{\isacharunderscore}{\kern0pt}right{\isacharunderscore}{\kern0pt}mono{\isacharcomma}{\kern0pt}\ rule\ truncate{\isacharunderscore}{\kern0pt}down{\isacharunderscore}{\kern0pt}ge{\isacharparenright}{\kern0pt}\isanewline
\ \ \isacommand{also}\isamarkupfalse%
\ \isacommand{have}\isamarkupfalse%
\ {\isachardoublequoteopen}{\isachardot}{\kern0pt}{\isachardot}{\kern0pt}{\isachardot}{\kern0pt}\ {\isasymle}\ y\ {\isacharplus}{\kern0pt}\ abs\ x\ {\isacharasterisk}{\kern0pt}\ {\isadigit{2}}\ powr\ {\isacharparenleft}{\kern0pt}{\isacharminus}{\kern0pt}real\ r{\isacharparenright}{\kern0pt}\ {\isacharminus}{\kern0pt}\ y{\isachardoublequoteclose}\isanewline
\ \ \ \ \isacommand{apply}\isamarkupfalse%
\ {\isacharparenleft}{\kern0pt}rule\ diff{\isacharunderscore}{\kern0pt}right{\isacharunderscore}{\kern0pt}mono{\isacharcomma}{\kern0pt}\ rule\ add{\isacharunderscore}{\kern0pt}mono{\isacharparenright}{\kern0pt}\isanewline
\ \ \ \ \ \isacommand{apply}\isamarkupfalse%
\ {\isacharparenleft}{\kern0pt}subst\ assms{\isacharparenleft}{\kern0pt}{\isadigit{1}}{\isacharparenright}{\kern0pt}{\isacharcomma}{\kern0pt}\ rule\ truncate{\isacharunderscore}{\kern0pt}down{\isacharunderscore}{\kern0pt}le{\isacharcomma}{\kern0pt}\ simp{\isacharparenright}{\kern0pt}\isanewline
\ \ \ \ \isacommand{by}\isamarkupfalse%
\ simp\isanewline
\ \ \isacommand{also}\isamarkupfalse%
\ \isacommand{have}\isamarkupfalse%
\ {\isachardoublequoteopen}{\isachardot}{\kern0pt}{\isachardot}{\kern0pt}{\isachardot}{\kern0pt}\ {\isasymle}\ abs\ x\ {\isacharasterisk}{\kern0pt}\ {\isadigit{2}}\ powr\ {\isacharparenleft}{\kern0pt}{\isacharminus}{\kern0pt}real\ r{\isacharparenright}{\kern0pt}{\isachardoublequoteclose}\ \isacommand{by}\isamarkupfalse%
\ simp\isanewline
\ \ \isacommand{also}\isamarkupfalse%
\ \isacommand{have}\isamarkupfalse%
\ {\isachardoublequoteopen}{\isachardot}{\kern0pt}{\isachardot}{\kern0pt}{\isachardot}{\kern0pt}\ {\isasymle}\ max\ {\isacharparenleft}{\kern0pt}abs\ x{\isacharparenright}{\kern0pt}\ {\isacharparenleft}{\kern0pt}abs\ y{\isacharparenright}{\kern0pt}\ {\isacharasterisk}{\kern0pt}\ {\isadigit{2}}\ powr\ {\isacharparenleft}{\kern0pt}{\isacharminus}{\kern0pt}real\ r{\isacharparenright}{\kern0pt}{\isachardoublequoteclose}\ \isacommand{by}\isamarkupfalse%
\ simp\isanewline
\ \ \isacommand{finally}\isamarkupfalse%
\ \isacommand{have}\isamarkupfalse%
\ a{\isacharcolon}{\kern0pt}{\isachardoublequoteopen}x\ {\isacharminus}{\kern0pt}\ y\ {\isasymle}\ max\ {\isacharparenleft}{\kern0pt}abs\ x{\isacharparenright}{\kern0pt}\ {\isacharparenleft}{\kern0pt}abs\ y{\isacharparenright}{\kern0pt}\ {\isacharasterisk}{\kern0pt}\ {\isadigit{2}}\ powr\ {\isacharparenleft}{\kern0pt}{\isacharminus}{\kern0pt}real\ r{\isacharparenright}{\kern0pt}{\isachardoublequoteclose}\ \isacommand{by}\isamarkupfalse%
\ simp\isanewline
\isanewline
\ \ \isacommand{have}\isamarkupfalse%
\ {\isachardoublequoteopen}y\ {\isacharminus}{\kern0pt}\ x\ {\isasymle}\ truncate{\isacharunderscore}{\kern0pt}down\ r\ y\ {\isacharplus}{\kern0pt}\ abs\ y\ {\isacharasterisk}{\kern0pt}\ {\isadigit{2}}\ powr\ {\isacharparenleft}{\kern0pt}{\isacharminus}{\kern0pt}real\ r{\isacharparenright}{\kern0pt}\ {\isacharminus}{\kern0pt}\ x{\isachardoublequoteclose}\isanewline
\ \ \ \ \isacommand{by}\isamarkupfalse%
\ {\isacharparenleft}{\kern0pt}rule\ diff{\isacharunderscore}{\kern0pt}right{\isacharunderscore}{\kern0pt}mono{\isacharcomma}{\kern0pt}\ rule\ truncate{\isacharunderscore}{\kern0pt}down{\isacharunderscore}{\kern0pt}ge{\isacharparenright}{\kern0pt}\isanewline
\ \ \isacommand{also}\isamarkupfalse%
\ \isacommand{have}\isamarkupfalse%
\ {\isachardoublequoteopen}{\isachardot}{\kern0pt}{\isachardot}{\kern0pt}{\isachardot}{\kern0pt}\ {\isasymle}\ x\ {\isacharplus}{\kern0pt}\ abs\ y\ {\isacharasterisk}{\kern0pt}\ {\isadigit{2}}\ powr\ {\isacharparenleft}{\kern0pt}{\isacharminus}{\kern0pt}real\ r{\isacharparenright}{\kern0pt}\ {\isacharminus}{\kern0pt}\ x{\isachardoublequoteclose}\isanewline
\ \ \ \ \isacommand{apply}\isamarkupfalse%
\ {\isacharparenleft}{\kern0pt}rule\ diff{\isacharunderscore}{\kern0pt}right{\isacharunderscore}{\kern0pt}mono{\isacharcomma}{\kern0pt}\ rule\ add{\isacharunderscore}{\kern0pt}mono{\isacharparenright}{\kern0pt}\isanewline
\ \ \ \ \ \isacommand{apply}\isamarkupfalse%
\ {\isacharparenleft}{\kern0pt}subst\ assms{\isacharparenleft}{\kern0pt}{\isadigit{1}}{\isacharparenright}{\kern0pt}{\isacharbrackleft}{\kern0pt}symmetric{\isacharbrackright}{\kern0pt}{\isacharcomma}{\kern0pt}\ rule\ truncate{\isacharunderscore}{\kern0pt}down{\isacharunderscore}{\kern0pt}le{\isacharcomma}{\kern0pt}\ simp{\isacharparenright}{\kern0pt}\isanewline
\ \ \ \ \isacommand{by}\isamarkupfalse%
\ simp\isanewline
\ \ \isacommand{also}\isamarkupfalse%
\ \isacommand{have}\isamarkupfalse%
\ {\isachardoublequoteopen}{\isachardot}{\kern0pt}{\isachardot}{\kern0pt}{\isachardot}{\kern0pt}\ {\isasymle}\ abs\ y\ {\isacharasterisk}{\kern0pt}\ {\isadigit{2}}\ powr\ {\isacharparenleft}{\kern0pt}{\isacharminus}{\kern0pt}real\ r{\isacharparenright}{\kern0pt}{\isachardoublequoteclose}\ \isacommand{by}\isamarkupfalse%
\ simp\isanewline
\ \ \isacommand{also}\isamarkupfalse%
\ \isacommand{have}\isamarkupfalse%
\ {\isachardoublequoteopen}{\isachardot}{\kern0pt}{\isachardot}{\kern0pt}{\isachardot}{\kern0pt}\ {\isasymle}\ max\ {\isacharparenleft}{\kern0pt}abs\ x{\isacharparenright}{\kern0pt}\ {\isacharparenleft}{\kern0pt}abs\ y{\isacharparenright}{\kern0pt}\ {\isacharasterisk}{\kern0pt}\ {\isadigit{2}}\ powr\ {\isacharparenleft}{\kern0pt}{\isacharminus}{\kern0pt}real\ r{\isacharparenright}{\kern0pt}{\isachardoublequoteclose}\ \isacommand{by}\isamarkupfalse%
\ simp\isanewline
\ \ \isacommand{finally}\isamarkupfalse%
\ \isacommand{have}\isamarkupfalse%
\ b{\isacharcolon}{\kern0pt}{\isachardoublequoteopen}y\ {\isacharminus}{\kern0pt}\ x\ {\isasymle}\ max\ {\isacharparenleft}{\kern0pt}abs\ x{\isacharparenright}{\kern0pt}\ {\isacharparenleft}{\kern0pt}abs\ y{\isacharparenright}{\kern0pt}\ {\isacharasterisk}{\kern0pt}\ {\isadigit{2}}\ powr\ {\isacharparenleft}{\kern0pt}{\isacharminus}{\kern0pt}real\ r{\isacharparenright}{\kern0pt}{\isachardoublequoteclose}\ \isacommand{by}\isamarkupfalse%
\ simp\isanewline
\isanewline
\ \ \isacommand{show}\isamarkupfalse%
\ {\isacharquery}{\kern0pt}thesis\isanewline
\ \ \ \ \isacommand{using}\isamarkupfalse%
\ abs{\isacharunderscore}{\kern0pt}le{\isacharunderscore}{\kern0pt}iff\ a\ b\ \isacommand{by}\isamarkupfalse%
\ linarith\isanewline
\isacommand{qed}\isamarkupfalse%
%
\endisatagproof
{\isafoldproof}%
%
\isadelimproof
\isanewline
%
\endisadelimproof
\isanewline
\isacommand{definition}\isamarkupfalse%
\ rat{\isacharunderscore}{\kern0pt}of{\isacharunderscore}{\kern0pt}float\ {\isacharcolon}{\kern0pt}{\isacharcolon}{\kern0pt}\ {\isachardoublequoteopen}float\ {\isasymRightarrow}\ rat{\isachardoublequoteclose}\ \isakeyword{where}\ \isanewline
\ \ {\isachardoublequoteopen}rat{\isacharunderscore}{\kern0pt}of{\isacharunderscore}{\kern0pt}float\ f\ {\isacharequal}{\kern0pt}\ of{\isacharunderscore}{\kern0pt}int\ {\isacharparenleft}{\kern0pt}mantissa\ f{\isacharparenright}{\kern0pt}\ {\isacharasterisk}{\kern0pt}\ \isanewline
\ \ \ \ {\isacharparenleft}{\kern0pt}if\ exponent\ f\ {\isasymge}\ {\isadigit{0}}\ then\ {\isadigit{2}}\ {\isacharcircum}{\kern0pt}\ {\isacharparenleft}{\kern0pt}nat\ {\isacharparenleft}{\kern0pt}exponent\ f{\isacharparenright}{\kern0pt}{\isacharparenright}{\kern0pt}\ else\ {\isadigit{1}}\ {\isacharslash}{\kern0pt}\ {\isadigit{2}}\ {\isacharcircum}{\kern0pt}\ {\isacharparenleft}{\kern0pt}nat\ {\isacharparenleft}{\kern0pt}{\isacharminus}{\kern0pt}exponent\ f{\isacharparenright}{\kern0pt}{\isacharparenright}{\kern0pt}{\isacharparenright}{\kern0pt}{\isachardoublequoteclose}\ \isanewline
\isanewline
\isacommand{lemma}\isamarkupfalse%
\ real{\isacharunderscore}{\kern0pt}of{\isacharunderscore}{\kern0pt}rat{\isacharunderscore}{\kern0pt}of{\isacharunderscore}{\kern0pt}float{\isacharcolon}{\kern0pt}\ {\isachardoublequoteopen}real{\isacharunderscore}{\kern0pt}of{\isacharunderscore}{\kern0pt}rat\ {\isacharparenleft}{\kern0pt}rat{\isacharunderscore}{\kern0pt}of{\isacharunderscore}{\kern0pt}float\ x{\isacharparenright}{\kern0pt}\ {\isacharequal}{\kern0pt}\ real{\isacharunderscore}{\kern0pt}of{\isacharunderscore}{\kern0pt}float\ x{\isachardoublequoteclose}\isanewline
%
\isadelimproof
\ \ %
\endisadelimproof
%
\isatagproof
\isacommand{apply}\isamarkupfalse%
\ {\isacharparenleft}{\kern0pt}cases\ x{\isacharparenright}{\kern0pt}\isanewline
\ \ \isacommand{apply}\isamarkupfalse%
\ {\isacharparenleft}{\kern0pt}simp\ add{\isacharcolon}{\kern0pt}rat{\isacharunderscore}{\kern0pt}of{\isacharunderscore}{\kern0pt}float{\isacharunderscore}{\kern0pt}def{\isacharparenright}{\kern0pt}\isanewline
\ \ \isacommand{apply}\isamarkupfalse%
\ {\isacharparenleft}{\kern0pt}rule\ conjI{\isacharparenright}{\kern0pt}\isanewline
\ \ \ \isacommand{apply}\isamarkupfalse%
\ {\isacharparenleft}{\kern0pt}metis\ {\isacharparenleft}{\kern0pt}mono{\isacharunderscore}{\kern0pt}tags{\isacharcomma}{\kern0pt}\ opaque{\isacharunderscore}{\kern0pt}lifting{\isacharparenright}{\kern0pt}\ Float{\isachardot}{\kern0pt}rep{\isacharunderscore}{\kern0pt}eq\ compute{\isacharunderscore}{\kern0pt}real{\isacharunderscore}{\kern0pt}of{\isacharunderscore}{\kern0pt}float\ mantissa{\isacharunderscore}{\kern0pt}exponent\ of{\isacharunderscore}{\kern0pt}int{\isacharunderscore}{\kern0pt}mult\ of{\isacharunderscore}{\kern0pt}int{\isacharunderscore}{\kern0pt}numeral\ of{\isacharunderscore}{\kern0pt}int{\isacharunderscore}{\kern0pt}power\ of{\isacharunderscore}{\kern0pt}rat{\isacharunderscore}{\kern0pt}of{\isacharunderscore}{\kern0pt}int{\isacharunderscore}{\kern0pt}eq{\isacharparenright}{\kern0pt}\isanewline
\ \ \isacommand{by}\isamarkupfalse%
\ {\isacharparenleft}{\kern0pt}metis\ Float{\isachardot}{\kern0pt}rep{\isacharunderscore}{\kern0pt}eq\ Float{\isacharunderscore}{\kern0pt}mantissa{\isacharunderscore}{\kern0pt}exponent\ compute{\isacharunderscore}{\kern0pt}real{\isacharunderscore}{\kern0pt}of{\isacharunderscore}{\kern0pt}float\ of{\isacharunderscore}{\kern0pt}int{\isacharunderscore}{\kern0pt}numeral\ of{\isacharunderscore}{\kern0pt}int{\isacharunderscore}{\kern0pt}power\ of{\isacharunderscore}{\kern0pt}rat{\isacharunderscore}{\kern0pt}divide\ of{\isacharunderscore}{\kern0pt}rat{\isacharunderscore}{\kern0pt}of{\isacharunderscore}{\kern0pt}int{\isacharunderscore}{\kern0pt}eq{\isacharparenright}{\kern0pt}%
\endisatagproof
{\isafoldproof}%
%
\isadelimproof
%
\endisadelimproof
%
\begin{isamarkuptext}%
Definition of an encoding for floating point numbers.%
\end{isamarkuptext}\isamarkuptrue%
\isacommand{definition}\isamarkupfalse%
\ F\isactrlsub S\ \isakeyword{where}\ {\isachardoublequoteopen}F\isactrlsub S\ f\ {\isacharequal}{\kern0pt}\ {\isacharparenleft}{\kern0pt}I\isactrlsub S\ {\isasymtimes}\isactrlsub S\ I\isactrlsub S{\isacharparenright}{\kern0pt}\ {\isacharparenleft}{\kern0pt}mantissa\ f{\isacharcomma}{\kern0pt}exponent\ f{\isacharparenright}{\kern0pt}{\isachardoublequoteclose}\isanewline
\isanewline
\isacommand{lemma}\isamarkupfalse%
\ encode{\isacharunderscore}{\kern0pt}float{\isacharcolon}{\kern0pt}\isanewline
\ \ {\isachardoublequoteopen}is{\isacharunderscore}{\kern0pt}encoding\ F\isactrlsub S{\isachardoublequoteclose}\isanewline
%
\isadelimproof
%
\endisadelimproof
%
\isatagproof
\isacommand{proof}\isamarkupfalse%
\ {\isacharminus}{\kern0pt}\isanewline
\ \ \isacommand{have}\isamarkupfalse%
\ a\ {\isacharcolon}{\kern0pt}\ {\isachardoublequoteopen}inj\ {\isacharparenleft}{\kern0pt}{\isasymlambda}x{\isachardot}{\kern0pt}\ {\isacharparenleft}{\kern0pt}mantissa\ x{\isacharcomma}{\kern0pt}\ exponent\ x{\isacharparenright}{\kern0pt}{\isacharparenright}{\kern0pt}{\isachardoublequoteclose}\isanewline
\ \ \isacommand{proof}\isamarkupfalse%
\ {\isacharparenleft}{\kern0pt}rule\ injI{\isacharparenright}{\kern0pt}\isanewline
\ \ \ \ \isacommand{fix}\isamarkupfalse%
\ x\ y\isanewline
\ \ \ \ \isacommand{assume}\isamarkupfalse%
\ {\isachardoublequoteopen}{\isacharparenleft}{\kern0pt}mantissa\ x{\isacharcomma}{\kern0pt}\ exponent\ x{\isacharparenright}{\kern0pt}\ {\isacharequal}{\kern0pt}\ {\isacharparenleft}{\kern0pt}mantissa\ y{\isacharcomma}{\kern0pt}\ exponent\ y{\isacharparenright}{\kern0pt}{\isachardoublequoteclose}\isanewline
\ \ \ \ \isacommand{hence}\isamarkupfalse%
\ {\isachardoublequoteopen}real{\isacharunderscore}{\kern0pt}of{\isacharunderscore}{\kern0pt}float\ x\ {\isacharequal}{\kern0pt}\ real{\isacharunderscore}{\kern0pt}of{\isacharunderscore}{\kern0pt}float\ y{\isachardoublequoteclose}\isanewline
\ \ \ \ \ \ \isacommand{by}\isamarkupfalse%
\ {\isacharparenleft}{\kern0pt}simp\ add{\isacharcolon}{\kern0pt}mantissa{\isacharunderscore}{\kern0pt}exponent{\isacharparenright}{\kern0pt}\isanewline
\ \ \ \ \isacommand{thus}\isamarkupfalse%
\ {\isachardoublequoteopen}x\ {\isacharequal}{\kern0pt}\ y{\isachardoublequoteclose}\isanewline
\ \ \ \ \ \ \isacommand{by}\isamarkupfalse%
\ {\isacharparenleft}{\kern0pt}metis\ real{\isacharunderscore}{\kern0pt}of{\isacharunderscore}{\kern0pt}float{\isacharunderscore}{\kern0pt}inverse{\isacharparenright}{\kern0pt}\isanewline
\ \ \isacommand{qed}\isamarkupfalse%
\isanewline
\ \ \isacommand{have}\isamarkupfalse%
\ {\isachardoublequoteopen}is{\isacharunderscore}{\kern0pt}encoding\ {\isacharparenleft}{\kern0pt}{\isasymlambda}f{\isachardot}{\kern0pt}\ if\ True\ then\ {\isacharparenleft}{\kern0pt}{\isacharparenleft}{\kern0pt}I\isactrlsub S\ {\isasymtimes}\isactrlsub S\ I\isactrlsub S{\isacharparenright}{\kern0pt}\ {\isacharparenleft}{\kern0pt}mantissa\ f{\isacharcomma}{\kern0pt}exponent\ f{\isacharparenright}{\kern0pt}{\isacharparenright}{\kern0pt}\ else\ None{\isacharparenright}{\kern0pt}{\isachardoublequoteclose}\isanewline
\ \ \ \ \isacommand{apply}\isamarkupfalse%
\ {\isacharparenleft}{\kern0pt}rule\ encoding{\isacharunderscore}{\kern0pt}compose{\isacharbrackleft}{\kern0pt}\isakeyword{where}\ f{\isacharequal}{\kern0pt}{\isachardoublequoteopen}{\isacharparenleft}{\kern0pt}I\isactrlsub S\ {\isasymtimes}\isactrlsub S\ I\isactrlsub S{\isacharparenright}{\kern0pt}{\isachardoublequoteclose}{\isacharbrackright}{\kern0pt}{\isacharparenright}{\kern0pt}\isanewline
\ \ \ \ \ \isacommand{apply}\isamarkupfalse%
\ {\isacharparenleft}{\kern0pt}metis\ prod{\isacharunderscore}{\kern0pt}encoding\ int{\isacharunderscore}{\kern0pt}encoding{\isacharcomma}{\kern0pt}\ simp{\isacharparenright}{\kern0pt}\isanewline
\ \ \ \ \isacommand{by}\isamarkupfalse%
\ {\isacharparenleft}{\kern0pt}metis\ a{\isacharparenright}{\kern0pt}\isanewline
\ \ \isacommand{moreover}\isamarkupfalse%
\ \isacommand{have}\isamarkupfalse%
\ {\isachardoublequoteopen}F\isactrlsub S\ {\isacharequal}{\kern0pt}\ {\isacharparenleft}{\kern0pt}{\isasymlambda}f{\isachardot}{\kern0pt}\ if\ f\ {\isasymin}\ UNIV\ then\ {\isacharparenleft}{\kern0pt}{\isacharparenleft}{\kern0pt}I\isactrlsub S\ {\isasymtimes}\isactrlsub S\ I\isactrlsub S{\isacharparenright}{\kern0pt}\ {\isacharparenleft}{\kern0pt}mantissa\ f{\isacharcomma}{\kern0pt}exponent\ f{\isacharparenright}{\kern0pt}{\isacharparenright}{\kern0pt}\ else\ None{\isacharparenright}{\kern0pt}{\isachardoublequoteclose}\isanewline
\ \ \ \ \isacommand{by}\isamarkupfalse%
\ {\isacharparenleft}{\kern0pt}rule\ ext{\isacharcomma}{\kern0pt}\ simp\ add{\isacharcolon}{\kern0pt}F\isactrlsub S{\isacharunderscore}{\kern0pt}def{\isacharparenright}{\kern0pt}\isanewline
\ \ \isacommand{ultimately}\isamarkupfalse%
\ \isacommand{show}\isamarkupfalse%
\ {\isachardoublequoteopen}is{\isacharunderscore}{\kern0pt}encoding\ F\isactrlsub S{\isachardoublequoteclose}\isanewline
\ \ \ \ \isacommand{by}\isamarkupfalse%
\ simp\isanewline
\isacommand{qed}\isamarkupfalse%
%
\endisatagproof
{\isafoldproof}%
%
\isadelimproof
\isanewline
%
\endisadelimproof
\isanewline
\isacommand{lemma}\isamarkupfalse%
\ truncate{\isacharunderscore}{\kern0pt}mantissa{\isacharunderscore}{\kern0pt}bound{\isacharcolon}{\kern0pt}\isanewline
\ \ {\isachardoublequoteopen}abs\ {\isacharparenleft}{\kern0pt}{\isasymlfloor}x\ {\isacharasterisk}{\kern0pt}\ {\isadigit{2}}\ powr\ {\isacharparenleft}{\kern0pt}real\ r\ {\isacharminus}{\kern0pt}\ real{\isacharunderscore}{\kern0pt}of{\isacharunderscore}{\kern0pt}int\ {\isasymlfloor}log\ {\isadigit{2}}\ {\isasymbar}x{\isasymbar}{\isasymrfloor}{\isacharparenright}{\kern0pt}{\isasymrfloor}{\isacharparenright}{\kern0pt}\ {\isasymle}\ {\isadigit{2}}\ {\isacharcircum}{\kern0pt}\ {\isacharparenleft}{\kern0pt}r{\isacharplus}{\kern0pt}{\isadigit{1}}{\isacharparenright}{\kern0pt}{\isachardoublequoteclose}\ {\isacharparenleft}{\kern0pt}\isakeyword{is}\ {\isachardoublequoteopen}{\isacharquery}{\kern0pt}lhs\ {\isasymle}\ {\isacharunderscore}{\kern0pt}{\isachardoublequoteclose}{\isacharparenright}{\kern0pt}\isanewline
%
\isadelimproof
%
\endisadelimproof
%
\isatagproof
\isacommand{proof}\isamarkupfalse%
\ {\isacharminus}{\kern0pt}\isanewline
\ \ \isacommand{define}\isamarkupfalse%
\ q\ \isakeyword{where}\ {\isachardoublequoteopen}q\ {\isacharequal}{\kern0pt}\ {\isasymlfloor}x\ {\isacharasterisk}{\kern0pt}\ {\isadigit{2}}\ powr\ {\isacharparenleft}{\kern0pt}real\ r\ {\isacharminus}{\kern0pt}\ real{\isacharunderscore}{\kern0pt}of{\isacharunderscore}{\kern0pt}int\ {\isacharparenleft}{\kern0pt}{\isasymlfloor}log\ {\isadigit{2}}\ {\isasymbar}x{\isasymbar}{\isasymrfloor}{\isacharparenright}{\kern0pt}{\isacharparenright}{\kern0pt}{\isasymrfloor}{\isachardoublequoteclose}\isanewline
\isanewline
\ \ \isacommand{have}\isamarkupfalse%
\ {\isachardoublequoteopen}x\ {\isachargreater}{\kern0pt}\ {\isadigit{0}}\ {\isasymLongrightarrow}\ abs\ q\ {\isasymle}\ {\isadigit{2}}\ {\isacharcircum}{\kern0pt}\ {\isacharparenleft}{\kern0pt}r\ {\isacharplus}{\kern0pt}\ {\isadigit{1}}{\isacharparenright}{\kern0pt}{\isachardoublequoteclose}\isanewline
\ \ \isacommand{proof}\isamarkupfalse%
\ {\isacharminus}{\kern0pt}\isanewline
\ \ \ \ \isacommand{assume}\isamarkupfalse%
\ a{\isacharcolon}{\kern0pt}{\isachardoublequoteopen}x\ {\isachargreater}{\kern0pt}\ {\isadigit{0}}{\isachardoublequoteclose}\isanewline
\isanewline
\ \ \ \ \isacommand{have}\isamarkupfalse%
\ {\isachardoublequoteopen}abs\ q\ {\isacharequal}{\kern0pt}\ q{\isachardoublequoteclose}\isanewline
\ \ \ \ \ \ \isacommand{apply}\isamarkupfalse%
\ {\isacharparenleft}{\kern0pt}rule\ abs{\isacharunderscore}{\kern0pt}of{\isacharunderscore}{\kern0pt}nonneg{\isacharparenright}{\kern0pt}\isanewline
\ \ \ \ \ \ \isacommand{apply}\isamarkupfalse%
\ {\isacharparenleft}{\kern0pt}simp\ add{\isacharcolon}{\kern0pt}q{\isacharunderscore}{\kern0pt}def{\isacharparenright}{\kern0pt}\isanewline
\ \ \ \ \ \ \isacommand{using}\isamarkupfalse%
\ a\ \isacommand{by}\isamarkupfalse%
\ simp\isanewline
\ \ \ \ \isacommand{also}\isamarkupfalse%
\ \isacommand{have}\isamarkupfalse%
\ {\isachardoublequoteopen}{\isachardot}{\kern0pt}{\isachardot}{\kern0pt}{\isachardot}{\kern0pt}\ {\isasymle}\ x\ {\isacharasterisk}{\kern0pt}\ {\isadigit{2}}\ powr\ {\isacharparenleft}{\kern0pt}real\ r\ {\isacharminus}{\kern0pt}\ real{\isacharunderscore}{\kern0pt}of{\isacharunderscore}{\kern0pt}int\ {\isasymlfloor}log\ {\isadigit{2}}\ {\isasymbar}x{\isasymbar}{\isasymrfloor}{\isacharparenright}{\kern0pt}{\isachardoublequoteclose}\isanewline
\ \ \ \ \ \ \isacommand{apply}\isamarkupfalse%
\ {\isacharparenleft}{\kern0pt}subst\ q{\isacharunderscore}{\kern0pt}def{\isacharparenright}{\kern0pt}\isanewline
\ \ \ \ \ \ \isacommand{using}\isamarkupfalse%
\ of{\isacharunderscore}{\kern0pt}int{\isacharunderscore}{\kern0pt}floor{\isacharunderscore}{\kern0pt}le\ \isacommand{by}\isamarkupfalse%
\ blast\isanewline
\ \ \ \ \isacommand{also}\isamarkupfalse%
\ \isacommand{have}\isamarkupfalse%
\ {\isachardoublequoteopen}{\isachardot}{\kern0pt}{\isachardot}{\kern0pt}{\isachardot}{\kern0pt}\ {\isacharequal}{\kern0pt}\ x\ {\isacharasterisk}{\kern0pt}\ {\isadigit{2}}\ powr\ real{\isacharunderscore}{\kern0pt}of{\isacharunderscore}{\kern0pt}int\ {\isacharparenleft}{\kern0pt}int\ r\ {\isacharminus}{\kern0pt}\ {\isasymlfloor}log\ {\isadigit{2}}\ {\isasymbar}x{\isasymbar}{\isasymrfloor}{\isacharparenright}{\kern0pt}{\isachardoublequoteclose}\isanewline
\ \ \ \ \ \ \isacommand{by}\isamarkupfalse%
\ auto\isanewline
\ \ \ \ \isacommand{also}\isamarkupfalse%
\ \isacommand{have}\isamarkupfalse%
\ {\isachardoublequoteopen}{\isachardot}{\kern0pt}{\isachardot}{\kern0pt}{\isachardot}{\kern0pt}\ {\isacharequal}{\kern0pt}\ {\isadigit{2}}\ powr\ {\isacharparenleft}{\kern0pt}log\ {\isadigit{2}}\ x\ {\isacharplus}{\kern0pt}\ real{\isacharunderscore}{\kern0pt}of{\isacharunderscore}{\kern0pt}int\ {\isacharparenleft}{\kern0pt}int\ r\ {\isacharminus}{\kern0pt}\ {\isasymlfloor}log\ {\isadigit{2}}\ {\isasymbar}x{\isasymbar}{\isasymrfloor}{\isacharparenright}{\kern0pt}{\isacharparenright}{\kern0pt}{\isachardoublequoteclose}\isanewline
\ \ \ \ \ \ \isacommand{apply}\isamarkupfalse%
\ {\isacharparenleft}{\kern0pt}simp\ add{\isacharcolon}{\kern0pt}powr{\isacharunderscore}{\kern0pt}add{\isacharparenright}{\kern0pt}\isanewline
\ \ \ \ \ \ \isacommand{by}\isamarkupfalse%
\ {\isacharparenleft}{\kern0pt}subst\ powr{\isacharunderscore}{\kern0pt}log{\isacharunderscore}{\kern0pt}cancel{\isacharcomma}{\kern0pt}\ simp{\isacharcomma}{\kern0pt}\ simp{\isacharcomma}{\kern0pt}\ simp\ add{\isacharcolon}{\kern0pt}a{\isacharcomma}{\kern0pt}\ simp{\isacharparenright}{\kern0pt}\isanewline
\ \ \ \ \isacommand{also}\isamarkupfalse%
\ \isacommand{have}\isamarkupfalse%
\ {\isachardoublequoteopen}{\isachardot}{\kern0pt}{\isachardot}{\kern0pt}{\isachardot}{\kern0pt}\ {\isasymle}\ {\isadigit{2}}\ powr\ {\isacharparenleft}{\kern0pt}real\ r\ {\isacharplus}{\kern0pt}\ {\isadigit{1}}{\isacharparenright}{\kern0pt}{\isachardoublequoteclose}\isanewline
\ \ \ \ \ \ \isacommand{apply}\isamarkupfalse%
\ {\isacharparenleft}{\kern0pt}rule\ powr{\isacharunderscore}{\kern0pt}mono{\isacharparenright}{\kern0pt}\isanewline
\ \ \ \ \ \ \isacommand{apply}\isamarkupfalse%
\ simp\ \isanewline
\ \ \ \ \ \ \isacommand{using}\isamarkupfalse%
\ a\ \isacommand{apply}\isamarkupfalse%
\ linarith\isanewline
\ \ \ \ \ \ \isacommand{by}\isamarkupfalse%
\ simp\isanewline
\ \ \ \ \isacommand{also}\isamarkupfalse%
\ \isacommand{have}\isamarkupfalse%
\ {\isachardoublequoteopen}{\isachardot}{\kern0pt}{\isachardot}{\kern0pt}{\isachardot}{\kern0pt}\ {\isacharequal}{\kern0pt}\ {\isadigit{2}}\ {\isacharcircum}{\kern0pt}\ {\isacharparenleft}{\kern0pt}r{\isacharplus}{\kern0pt}{\isadigit{1}}{\isacharparenright}{\kern0pt}{\isachardoublequoteclose}\isanewline
\ \ \ \ \ \ \isacommand{by}\isamarkupfalse%
\ {\isacharparenleft}{\kern0pt}subst\ powr{\isacharunderscore}{\kern0pt}realpow{\isacharbrackleft}{\kern0pt}symmetric{\isacharbrackright}{\kern0pt}{\isacharcomma}{\kern0pt}\ simp{\isacharcomma}{\kern0pt}\ simp\ add{\isacharcolon}{\kern0pt}add{\isachardot}{\kern0pt}commute{\isacharparenright}{\kern0pt}\isanewline
\ \ \ \ \isacommand{finally}\isamarkupfalse%
\ \isacommand{show}\isamarkupfalse%
\ {\isachardoublequoteopen}abs\ q\ {\isasymle}\ {\isadigit{2}}\ {\isacharcircum}{\kern0pt}\ {\isacharparenleft}{\kern0pt}r{\isacharplus}{\kern0pt}{\isadigit{1}}{\isacharparenright}{\kern0pt}{\isachardoublequoteclose}\ \isanewline
\ \ \ \ \ \ \isacommand{by}\isamarkupfalse%
\ {\isacharparenleft}{\kern0pt}metis\ of{\isacharunderscore}{\kern0pt}int{\isacharunderscore}{\kern0pt}le{\isacharunderscore}{\kern0pt}iff\ of{\isacharunderscore}{\kern0pt}int{\isacharunderscore}{\kern0pt}numeral\ of{\isacharunderscore}{\kern0pt}int{\isacharunderscore}{\kern0pt}power{\isacharparenright}{\kern0pt}\isanewline
\ \ \isacommand{qed}\isamarkupfalse%
\isanewline
\ \ \ \ \isanewline
\ \ \isacommand{moreover}\isamarkupfalse%
\ \isacommand{have}\isamarkupfalse%
\ {\isachardoublequoteopen}x\ {\isacharless}{\kern0pt}\ {\isadigit{0}}\ {\isasymLongrightarrow}\ abs\ q\ {\isasymle}\ {\isacharparenleft}{\kern0pt}{\isadigit{2}}\ {\isacharcircum}{\kern0pt}\ {\isacharparenleft}{\kern0pt}r\ {\isacharplus}{\kern0pt}\ {\isadigit{1}}{\isacharparenright}{\kern0pt}{\isacharparenright}{\kern0pt}{\isachardoublequoteclose}\isanewline
\ \ \isacommand{proof}\isamarkupfalse%
\ {\isacharminus}{\kern0pt}\isanewline
\ \ \ \ \isacommand{assume}\isamarkupfalse%
\ a{\isacharcolon}{\kern0pt}{\isachardoublequoteopen}x\ {\isacharless}{\kern0pt}\ {\isadigit{0}}{\isachardoublequoteclose}\isanewline
\ \ \ \ \isacommand{have}\isamarkupfalse%
\ {\isachardoublequoteopen}{\isacharminus}{\kern0pt}{\isacharparenleft}{\kern0pt}{\isadigit{2}}\ {\isacharcircum}{\kern0pt}\ {\isacharparenleft}{\kern0pt}r{\isacharplus}{\kern0pt}{\isadigit{1}}{\isacharparenright}{\kern0pt}\ {\isacharplus}{\kern0pt}\ {\isadigit{1}}{\isacharparenright}{\kern0pt}\ {\isacharequal}{\kern0pt}\ {\isacharminus}{\kern0pt}{\isacharparenleft}{\kern0pt}{\isadigit{2}}\ powr\ {\isacharparenleft}{\kern0pt}real\ r\ {\isacharplus}{\kern0pt}\ {\isadigit{1}}{\isacharparenright}{\kern0pt}{\isacharplus}{\kern0pt}{\isadigit{1}}{\isacharparenright}{\kern0pt}{\isachardoublequoteclose}\isanewline
\ \ \ \ \ \ \isacommand{apply}\isamarkupfalse%
\ {\isacharparenleft}{\kern0pt}subst\ powr{\isacharunderscore}{\kern0pt}realpow{\isacharbrackleft}{\kern0pt}symmetric{\isacharbrackright}{\kern0pt}{\isacharcomma}{\kern0pt}\ simp{\isacharparenright}{\kern0pt}\isanewline
\ \ \ \ \ \ \isacommand{by}\isamarkupfalse%
\ {\isacharparenleft}{\kern0pt}simp\ add{\isacharcolon}{\kern0pt}add{\isachardot}{\kern0pt}commute{\isacharparenright}{\kern0pt}\isanewline
\ \ \ \ \isacommand{also}\isamarkupfalse%
\ \isacommand{have}\isamarkupfalse%
\ \ {\isachardoublequoteopen}{\isachardot}{\kern0pt}{\isachardot}{\kern0pt}{\isachardot}{\kern0pt}\ {\isacharless}{\kern0pt}\ {\isacharminus}{\kern0pt}{\isacharparenleft}{\kern0pt}{\isadigit{2}}\ powr\ {\isacharparenleft}{\kern0pt}log\ {\isadigit{2}}\ {\isacharparenleft}{\kern0pt}{\isacharminus}{\kern0pt}\ x{\isacharparenright}{\kern0pt}\ {\isacharplus}{\kern0pt}\ {\isacharparenleft}{\kern0pt}r\ {\isacharminus}{\kern0pt}\ {\isasymlfloor}log\ {\isadigit{2}}\ {\isasymbar}x{\isasymbar}{\isasymrfloor}{\isacharparenright}{\kern0pt}{\isacharparenright}{\kern0pt}\ {\isacharplus}{\kern0pt}\ {\isadigit{1}}{\isacharparenright}{\kern0pt}{\isachardoublequoteclose}\isanewline
\ \ \ \ \ \ \isacommand{apply}\isamarkupfalse%
\ {\isacharparenleft}{\kern0pt}subst\ neg{\isacharunderscore}{\kern0pt}less{\isacharunderscore}{\kern0pt}iff{\isacharunderscore}{\kern0pt}less{\isacharparenright}{\kern0pt}\isanewline
\ \ \ \ \ \ \isacommand{apply}\isamarkupfalse%
\ {\isacharparenleft}{\kern0pt}rule\ add{\isacharunderscore}{\kern0pt}strict{\isacharunderscore}{\kern0pt}right{\isacharunderscore}{\kern0pt}mono{\isacharparenright}{\kern0pt}\isanewline
\ \ \ \ \ \ \isacommand{apply}\isamarkupfalse%
\ {\isacharparenleft}{\kern0pt}rule\ powr{\isacharunderscore}{\kern0pt}less{\isacharunderscore}{\kern0pt}mono{\isacharparenright}{\kern0pt}\isanewline
\ \ \ \ \ \ \ \isacommand{apply}\isamarkupfalse%
\ {\isacharparenleft}{\kern0pt}simp{\isacharparenright}{\kern0pt}\isanewline
\ \ \ \ \ \ \ \isacommand{using}\isamarkupfalse%
\ a\ \isacommand{apply}\isamarkupfalse%
\ linarith\isanewline
\ \ \ \ \ \ \ \isacommand{by}\isamarkupfalse%
\ simp{\isacharplus}{\kern0pt}\isanewline
\ \ \ \ \isacommand{also}\isamarkupfalse%
\ \isacommand{have}\isamarkupfalse%
\ {\isachardoublequoteopen}{\isachardot}{\kern0pt}{\isachardot}{\kern0pt}{\isachardot}{\kern0pt}\ {\isacharequal}{\kern0pt}\ x\ {\isacharasterisk}{\kern0pt}\ {\isadigit{2}}\ powr\ {\isacharparenleft}{\kern0pt}r\ {\isacharminus}{\kern0pt}\ {\isasymlfloor}log\ {\isadigit{2}}\ {\isasymbar}x{\isasymbar}{\isasymrfloor}{\isacharparenright}{\kern0pt}\ {\isacharminus}{\kern0pt}\ {\isadigit{1}}{\isachardoublequoteclose}\isanewline
\ \ \ \ \ \ \isacommand{apply}\isamarkupfalse%
\ {\isacharparenleft}{\kern0pt}simp\ add{\isacharcolon}{\kern0pt}powr{\isacharunderscore}{\kern0pt}add{\isacharparenright}{\kern0pt}\isanewline
\ \ \ \ \ \ \isacommand{apply}\isamarkupfalse%
\ {\isacharparenleft}{\kern0pt}subst\ powr{\isacharunderscore}{\kern0pt}log{\isacharunderscore}{\kern0pt}cancel{\isacharcomma}{\kern0pt}\ simp{\isacharcomma}{\kern0pt}\ simp{\isacharcomma}{\kern0pt}\ simp\ add{\isacharcolon}{\kern0pt}a{\isacharparenright}{\kern0pt}\isanewline
\ \ \ \ \ \ \isacommand{by}\isamarkupfalse%
\ simp\isanewline
\ \ \ \ \isacommand{also}\isamarkupfalse%
\ \isacommand{have}\isamarkupfalse%
\ {\isachardoublequoteopen}{\isachardot}{\kern0pt}{\isachardot}{\kern0pt}{\isachardot}{\kern0pt}\ {\isasymle}\ q{\isachardoublequoteclose}\isanewline
\ \ \ \ \ \ \isacommand{by}\isamarkupfalse%
\ {\isacharparenleft}{\kern0pt}simp\ add{\isacharcolon}{\kern0pt}q{\isacharunderscore}{\kern0pt}def{\isacharparenright}{\kern0pt}\isanewline
\ \ \ \ \isacommand{also}\isamarkupfalse%
\ \isacommand{have}\isamarkupfalse%
\ {\isachardoublequoteopen}{\isachardot}{\kern0pt}{\isachardot}{\kern0pt}{\isachardot}{\kern0pt}\ {\isacharequal}{\kern0pt}\ {\isacharminus}{\kern0pt}\ abs\ q{\isachardoublequoteclose}\isanewline
\ \ \ \ \ \ \isacommand{apply}\isamarkupfalse%
\ {\isacharparenleft}{\kern0pt}subst\ abs{\isacharunderscore}{\kern0pt}of{\isacharunderscore}{\kern0pt}neg{\isacharparenright}{\kern0pt}\isanewline
\ \ \ \ \ \ \isacommand{using}\isamarkupfalse%
\ a\ \isanewline
\ \ \ \ \ \ \ \isacommand{apply}\isamarkupfalse%
\ {\isacharparenleft}{\kern0pt}simp\ add{\isacharcolon}{\kern0pt}\ mult{\isacharunderscore}{\kern0pt}pos{\isacharunderscore}{\kern0pt}neg{\isadigit{2}}\ q{\isacharunderscore}{\kern0pt}def{\isacharparenright}{\kern0pt}\isanewline
\ \ \ \ \ \ \isacommand{by}\isamarkupfalse%
\ simp\isanewline
\ \ \ \ \isacommand{finally}\isamarkupfalse%
\ \isacommand{have}\isamarkupfalse%
\ {\isachardoublequoteopen}{\isacharminus}{\kern0pt}{\isacharparenleft}{\kern0pt}{\isadigit{2}}\ {\isacharcircum}{\kern0pt}\ {\isacharparenleft}{\kern0pt}r{\isacharplus}{\kern0pt}{\isadigit{1}}{\isacharparenright}{\kern0pt}{\isacharplus}{\kern0pt}{\isadigit{1}}{\isacharparenright}{\kern0pt}\ {\isacharless}{\kern0pt}\ {\isacharminus}{\kern0pt}\ abs\ q{\isachardoublequoteclose}\ \isacommand{using}\isamarkupfalse%
\ of{\isacharunderscore}{\kern0pt}int{\isacharunderscore}{\kern0pt}less{\isacharunderscore}{\kern0pt}iff\ \isacommand{by}\isamarkupfalse%
\ fastforce\isanewline
\ \ \ \ \isacommand{hence}\isamarkupfalse%
\ {\isachardoublequoteopen}{\isacharminus}{\kern0pt}{\isacharparenleft}{\kern0pt}{\isadigit{2}}\ {\isacharcircum}{\kern0pt}\ {\isacharparenleft}{\kern0pt}r{\isacharplus}{\kern0pt}{\isadigit{1}}{\isacharparenright}{\kern0pt}{\isacharparenright}{\kern0pt}\ {\isasymle}\ {\isacharminus}{\kern0pt}\ abs\ q{\isachardoublequoteclose}\ \isacommand{by}\isamarkupfalse%
\ linarith\isanewline
\ \ \ \ \isacommand{thus}\isamarkupfalse%
\ {\isachardoublequoteopen}abs\ q\ {\isasymle}\ {\isadigit{2}}{\isacharcircum}{\kern0pt}{\isacharparenleft}{\kern0pt}r{\isacharplus}{\kern0pt}{\isadigit{1}}{\isacharparenright}{\kern0pt}{\isachardoublequoteclose}\ \isacommand{by}\isamarkupfalse%
\ linarith\isanewline
\ \ \isacommand{qed}\isamarkupfalse%
\isanewline
\isanewline
\ \ \isacommand{moreover}\isamarkupfalse%
\ \isacommand{have}\isamarkupfalse%
\ {\isachardoublequoteopen}x\ {\isacharequal}{\kern0pt}\ {\isadigit{0}}\ {\isasymLongrightarrow}\ abs\ q\ {\isasymle}\ {\isadigit{2}}{\isacharcircum}{\kern0pt}{\isacharparenleft}{\kern0pt}r{\isacharplus}{\kern0pt}{\isadigit{1}}{\isacharparenright}{\kern0pt}{\isachardoublequoteclose}\isanewline
\ \ \ \ \isacommand{by}\isamarkupfalse%
\ {\isacharparenleft}{\kern0pt}simp\ add{\isacharcolon}{\kern0pt}q{\isacharunderscore}{\kern0pt}def{\isacharparenright}{\kern0pt}\isanewline
\ \ \isacommand{ultimately}\isamarkupfalse%
\ \isacommand{have}\isamarkupfalse%
\ {\isachardoublequoteopen}abs\ q\ {\isasymle}\ {\isadigit{2}}{\isacharcircum}{\kern0pt}{\isacharparenleft}{\kern0pt}r{\isacharplus}{\kern0pt}{\isadigit{1}}{\isacharparenright}{\kern0pt}{\isachardoublequoteclose}\isanewline
\ \ \ \ \isacommand{by}\isamarkupfalse%
\ fastforce\isanewline
\ \ \isacommand{thus}\isamarkupfalse%
\ {\isacharquery}{\kern0pt}thesis\ \isacommand{using}\isamarkupfalse%
\ q{\isacharunderscore}{\kern0pt}def\ \isacommand{by}\isamarkupfalse%
\ blast\isanewline
\isacommand{qed}\isamarkupfalse%
%
\endisatagproof
{\isafoldproof}%
%
\isadelimproof
\isanewline
%
\endisadelimproof
\isanewline
\isacommand{lemma}\isamarkupfalse%
\ suc{\isacharunderscore}{\kern0pt}n{\isacharunderscore}{\kern0pt}le{\isacharunderscore}{\kern0pt}{\isadigit{2}}{\isacharunderscore}{\kern0pt}pow{\isacharunderscore}{\kern0pt}n{\isacharcolon}{\kern0pt}\isanewline
\ \ \isakeyword{fixes}\ n\ {\isacharcolon}{\kern0pt}{\isacharcolon}{\kern0pt}\ nat\isanewline
\ \ \isakeyword{shows}\ {\isachardoublequoteopen}n\ {\isacharplus}{\kern0pt}\ {\isadigit{1}}\ {\isasymle}\ {\isadigit{2}}\ {\isacharcircum}{\kern0pt}\ n{\isachardoublequoteclose}\isanewline
%
\isadelimproof
\ \ %
\endisadelimproof
%
\isatagproof
\isacommand{by}\isamarkupfalse%
\ {\isacharparenleft}{\kern0pt}induction\ n{\isacharcomma}{\kern0pt}\ simp{\isacharcomma}{\kern0pt}\ simp{\isacharparenright}{\kern0pt}%
\endisatagproof
{\isafoldproof}%
%
\isadelimproof
\isanewline
%
\endisadelimproof
\isanewline
\isacommand{lemma}\isamarkupfalse%
\ float{\isacharunderscore}{\kern0pt}bit{\isacharunderscore}{\kern0pt}count{\isacharcolon}{\kern0pt}\isanewline
\ \ \isakeyword{fixes}\ m\ {\isacharcolon}{\kern0pt}{\isacharcolon}{\kern0pt}\ int\isanewline
\ \ \isakeyword{fixes}\ e\ {\isacharcolon}{\kern0pt}{\isacharcolon}{\kern0pt}\ int\isanewline
\ \ \isakeyword{defines}\ {\isachardoublequoteopen}f\ {\isasymequiv}\ float{\isacharunderscore}{\kern0pt}of\ {\isacharparenleft}{\kern0pt}m\ {\isacharasterisk}{\kern0pt}\ {\isadigit{2}}\ powr\ e{\isacharparenright}{\kern0pt}{\isachardoublequoteclose}\isanewline
\ \ \isakeyword{shows}\ {\isachardoublequoteopen}bit{\isacharunderscore}{\kern0pt}count\ {\isacharparenleft}{\kern0pt}F\isactrlsub S\ f{\isacharparenright}{\kern0pt}\ {\isasymle}\ {\isadigit{4}}\ {\isacharplus}{\kern0pt}\ {\isadigit{2}}\ {\isacharasterisk}{\kern0pt}\ {\isacharparenleft}{\kern0pt}log\ {\isadigit{2}}\ {\isacharparenleft}{\kern0pt}{\isasymbar}m{\isasymbar}\ {\isacharplus}{\kern0pt}\ {\isadigit{2}}{\isacharparenright}{\kern0pt}\ {\isacharplus}{\kern0pt}\ log\ {\isadigit{2}}\ {\isacharparenleft}{\kern0pt}{\isasymbar}e{\isasymbar}\ {\isacharplus}{\kern0pt}\ {\isadigit{1}}{\isacharparenright}{\kern0pt}{\isacharparenright}{\kern0pt}{\isachardoublequoteclose}\isanewline
%
\isadelimproof
%
\endisadelimproof
%
\isatagproof
\isacommand{proof}\isamarkupfalse%
\ {\isacharparenleft}{\kern0pt}cases\ {\isachardoublequoteopen}m\ {\isasymnoteq}\ {\isadigit{0}}{\isachardoublequoteclose}{\isacharparenright}{\kern0pt}\isanewline
\ \ \isacommand{case}\isamarkupfalse%
\ True\isanewline
\ \ \isacommand{have}\isamarkupfalse%
\ {\isachardoublequoteopen}f\ {\isacharequal}{\kern0pt}\ Float\ m\ e{\isachardoublequoteclose}\ \isanewline
\ \ \ \ \isacommand{by}\isamarkupfalse%
\ {\isacharparenleft}{\kern0pt}simp\ add{\isacharcolon}{\kern0pt}\ f{\isacharunderscore}{\kern0pt}def\ Float{\isachardot}{\kern0pt}abs{\isacharunderscore}{\kern0pt}eq{\isacharparenright}{\kern0pt}\isanewline
\ \ \isacommand{moreover}\isamarkupfalse%
\ \isacommand{have}\isamarkupfalse%
\ f{\isacharunderscore}{\kern0pt}ne{\isacharunderscore}{\kern0pt}{\isadigit{0}}{\isacharcolon}{\kern0pt}\ {\isachardoublequoteopen}f\ {\isasymnoteq}\ {\isadigit{0}}{\isachardoublequoteclose}\ \isacommand{using}\isamarkupfalse%
\ True\ \isacommand{apply}\isamarkupfalse%
\ {\isacharparenleft}{\kern0pt}simp\ add{\isacharcolon}{\kern0pt}f{\isacharunderscore}{\kern0pt}def{\isacharparenright}{\kern0pt}\ \isanewline
\ \ \ \ \isacommand{by}\isamarkupfalse%
\ {\isacharparenleft}{\kern0pt}metis\ Float{\isachardot}{\kern0pt}compute{\isacharunderscore}{\kern0pt}is{\isacharunderscore}{\kern0pt}float{\isacharunderscore}{\kern0pt}zero\ Float{\isachardot}{\kern0pt}rep{\isacharunderscore}{\kern0pt}eq\ is{\isacharunderscore}{\kern0pt}float{\isacharunderscore}{\kern0pt}zero{\isachardot}{\kern0pt}rep{\isacharunderscore}{\kern0pt}eq\ real{\isacharunderscore}{\kern0pt}of{\isacharunderscore}{\kern0pt}float{\isacharunderscore}{\kern0pt}inverse\ zero{\isacharunderscore}{\kern0pt}float{\isachardot}{\kern0pt}rep{\isacharunderscore}{\kern0pt}eq{\isacharparenright}{\kern0pt}\isanewline
\ \ \isacommand{ultimately}\isamarkupfalse%
\ \isacommand{obtain}\isamarkupfalse%
\ i\ {\isacharcolon}{\kern0pt}{\isacharcolon}{\kern0pt}\ nat\ \isakeyword{where}\ m{\isacharunderscore}{\kern0pt}def{\isacharcolon}{\kern0pt}\ {\isachardoublequoteopen}m\ {\isacharequal}{\kern0pt}\ mantissa\ f\ {\isacharasterisk}{\kern0pt}\ {\isadigit{2}}\ {\isacharcircum}{\kern0pt}\ i{\isachardoublequoteclose}\ \isakeyword{and}\ e{\isacharunderscore}{\kern0pt}def{\isacharcolon}{\kern0pt}\ {\isachardoublequoteopen}e\ {\isacharequal}{\kern0pt}\ exponent\ f\ {\isacharminus}{\kern0pt}\ i{\isachardoublequoteclose}\isanewline
\ \ \ \ \isacommand{using}\isamarkupfalse%
\ \ denormalize{\isacharunderscore}{\kern0pt}shift\ \isacommand{by}\isamarkupfalse%
\ blast\isanewline
\isanewline
\ \ \isacommand{have}\isamarkupfalse%
\ b{\isacharcolon}{\kern0pt}{\isachardoublequoteopen}abs\ {\isacharparenleft}{\kern0pt}real{\isacharunderscore}{\kern0pt}of{\isacharunderscore}{\kern0pt}int\ {\isacharparenleft}{\kern0pt}mantissa\ f{\isacharparenright}{\kern0pt}{\isacharparenright}{\kern0pt}\ {\isasymge}\ {\isadigit{1}}{\isachardoublequoteclose}\ \isanewline
\ \ \ \ \isacommand{by}\isamarkupfalse%
\ {\isacharparenleft}{\kern0pt}meson\ dual{\isacharunderscore}{\kern0pt}order{\isachardot}{\kern0pt}refl\ f{\isacharunderscore}{\kern0pt}ne{\isacharunderscore}{\kern0pt}{\isadigit{0}}\ mantissa{\isacharunderscore}{\kern0pt}noteq{\isacharunderscore}{\kern0pt}{\isadigit{0}}\ of{\isacharunderscore}{\kern0pt}int{\isacharunderscore}{\kern0pt}leD{\isacharparenright}{\kern0pt}\isanewline
\isanewline
\ \ \isacommand{have}\isamarkupfalse%
\ c{\isacharcolon}{\kern0pt}\ {\isachardoublequoteopen}{\isadigit{2}}{\isacharasterisk}{\kern0pt}i\ {\isasymle}\ {\isadigit{2}}{\isacharcircum}{\kern0pt}i{\isachardoublequoteclose}\isanewline
\ \ \ \ \isacommand{apply}\isamarkupfalse%
\ {\isacharparenleft}{\kern0pt}cases\ {\isachardoublequoteopen}i\ {\isachargreater}{\kern0pt}\ {\isadigit{0}}{\isachardoublequoteclose}{\isacharparenright}{\kern0pt}\isanewline
\ \ \ \ \ \ \isacommand{using}\isamarkupfalse%
\ suc{\isacharunderscore}{\kern0pt}n{\isacharunderscore}{\kern0pt}le{\isacharunderscore}{\kern0pt}{\isadigit{2}}{\isacharunderscore}{\kern0pt}pow{\isacharunderscore}{\kern0pt}n{\isacharbrackleft}{\kern0pt}\isakeyword{where}\ n{\isacharequal}{\kern0pt}{\isachardoublequoteopen}i{\isacharminus}{\kern0pt}{\isadigit{1}}{\isachardoublequoteclose}{\isacharbrackright}{\kern0pt}\ \isacommand{apply}\isamarkupfalse%
\ simp\isanewline
\ \ \ \ \ \isacommand{apply}\isamarkupfalse%
\ {\isacharparenleft}{\kern0pt}metis\ One{\isacharunderscore}{\kern0pt}nat{\isacharunderscore}{\kern0pt}def\ nat{\isacharunderscore}{\kern0pt}mult{\isacharunderscore}{\kern0pt}le{\isacharunderscore}{\kern0pt}cancel{\isacharunderscore}{\kern0pt}disj\ power{\isacharunderscore}{\kern0pt}commutes\ power{\isacharunderscore}{\kern0pt}minus{\isacharunderscore}{\kern0pt}mult{\isacharparenright}{\kern0pt}\isanewline
\ \ \ \ \isacommand{by}\isamarkupfalse%
\ simp\isanewline
\isanewline
\ \ \isacommand{have}\isamarkupfalse%
\ a{\isacharcolon}{\kern0pt}{\isachardoublequoteopen}{\isasymbar}real{\isacharunderscore}{\kern0pt}of{\isacharunderscore}{\kern0pt}int\ {\isacharparenleft}{\kern0pt}mantissa\ f{\isacharparenright}{\kern0pt}{\isasymbar}\ {\isacharasterisk}{\kern0pt}\ {\isacharparenleft}{\kern0pt}real\ i\ {\isacharplus}{\kern0pt}\ {\isadigit{1}}{\isacharparenright}{\kern0pt}\ {\isacharplus}{\kern0pt}\ real\ i\ {\isasymle}\ {\isasymbar}real{\isacharunderscore}{\kern0pt}of{\isacharunderscore}{\kern0pt}int\ {\isacharparenleft}{\kern0pt}mantissa\ f{\isacharparenright}{\kern0pt}{\isasymbar}\ {\isacharasterisk}{\kern0pt}\ {\isadigit{2}}\ {\isacharcircum}{\kern0pt}\ i\ {\isacharplus}{\kern0pt}\ {\isadigit{1}}{\isachardoublequoteclose}\ \isanewline
\ \ \isacommand{proof}\isamarkupfalse%
\ {\isacharparenleft}{\kern0pt}cases\ {\isachardoublequoteopen}i\ {\isasymge}\ {\isadigit{1}}{\isachardoublequoteclose}{\isacharparenright}{\kern0pt}\isanewline
\ \ \ \ \isacommand{case}\isamarkupfalse%
\ True\isanewline
\ \ \ \ \isacommand{have}\isamarkupfalse%
\ {\isachardoublequoteopen}{\isasymbar}real{\isacharunderscore}{\kern0pt}of{\isacharunderscore}{\kern0pt}int\ {\isacharparenleft}{\kern0pt}mantissa\ f{\isacharparenright}{\kern0pt}{\isasymbar}\ {\isacharasterisk}{\kern0pt}\ {\isacharparenleft}{\kern0pt}real\ i\ {\isacharplus}{\kern0pt}\ {\isadigit{1}}{\isacharparenright}{\kern0pt}\ {\isacharplus}{\kern0pt}\ real\ i\ {\isacharequal}{\kern0pt}\ {\isasymbar}real{\isacharunderscore}{\kern0pt}of{\isacharunderscore}{\kern0pt}int\ {\isacharparenleft}{\kern0pt}mantissa\ f{\isacharparenright}{\kern0pt}{\isasymbar}\ {\isacharasterisk}{\kern0pt}\ {\isacharparenleft}{\kern0pt}real\ i\ {\isacharplus}{\kern0pt}\ {\isadigit{1}}{\isacharparenright}{\kern0pt}\ {\isacharplus}{\kern0pt}\ {\isacharparenleft}{\kern0pt}real\ i\ {\isacharminus}{\kern0pt}\ {\isadigit{1}}{\isacharparenright}{\kern0pt}\ {\isacharplus}{\kern0pt}\ {\isadigit{1}}{\isachardoublequoteclose}\isanewline
\ \ \ \ \ \ \isacommand{by}\isamarkupfalse%
\ simp\isanewline
\ \ \ \ \isacommand{also}\isamarkupfalse%
\ \isacommand{have}\isamarkupfalse%
\ {\isachardoublequoteopen}{\isachardot}{\kern0pt}{\isachardot}{\kern0pt}{\isachardot}{\kern0pt}\ \ {\isasymle}\ {\isasymbar}real{\isacharunderscore}{\kern0pt}of{\isacharunderscore}{\kern0pt}int\ {\isacharparenleft}{\kern0pt}mantissa\ f{\isacharparenright}{\kern0pt}{\isasymbar}\ {\isacharasterisk}{\kern0pt}\ {\isacharparenleft}{\kern0pt}{\isacharparenleft}{\kern0pt}real\ i\ {\isacharplus}{\kern0pt}\ {\isadigit{1}}{\isacharparenright}{\kern0pt}\ {\isacharplus}{\kern0pt}\ {\isacharparenleft}{\kern0pt}real\ i\ {\isacharminus}{\kern0pt}\ {\isadigit{1}}{\isacharparenright}{\kern0pt}{\isacharparenright}{\kern0pt}\ {\isacharplus}{\kern0pt}\ {\isadigit{1}}{\isachardoublequoteclose}\isanewline
\ \ \ \ \ \ \isacommand{apply}\isamarkupfalse%
\ {\isacharparenleft}{\kern0pt}subst\ {\isacharparenleft}{\kern0pt}{\isadigit{2}}{\isacharparenright}{\kern0pt}\ distrib{\isacharunderscore}{\kern0pt}left{\isacharparenright}{\kern0pt}\isanewline
\ \ \ \ \ \ \isacommand{apply}\isamarkupfalse%
\ {\isacharparenleft}{\kern0pt}rule\ add{\isacharunderscore}{\kern0pt}mono{\isacharparenright}{\kern0pt}\isanewline
\ \ \ \ \ \ \ \isacommand{apply}\isamarkupfalse%
\ {\isacharparenleft}{\kern0pt}rule\ add{\isacharunderscore}{\kern0pt}mono{\isacharcomma}{\kern0pt}\ simp{\isacharparenright}{\kern0pt}\isanewline
\ \ \ \ \ \ \ \isacommand{apply}\isamarkupfalse%
\ {\isacharparenleft}{\kern0pt}rule\ order{\isacharunderscore}{\kern0pt}trans{\isacharbrackleft}{\kern0pt}\isakeyword{where}\ y{\isacharequal}{\kern0pt}{\isachardoublequoteopen}{\isadigit{1}}{\isacharasterisk}{\kern0pt}\ {\isacharparenleft}{\kern0pt}real\ i\ {\isacharminus}{\kern0pt}\ {\isadigit{1}}{\isacharparenright}{\kern0pt}{\isachardoublequoteclose}{\isacharbrackright}{\kern0pt}{\isacharcomma}{\kern0pt}\ simp{\isacharparenright}{\kern0pt}\isanewline
\ \ \ \ \ \ \ \isacommand{apply}\isamarkupfalse%
\ {\isacharparenleft}{\kern0pt}rule\ mult{\isacharunderscore}{\kern0pt}right{\isacharunderscore}{\kern0pt}mono{\isacharcomma}{\kern0pt}\ metis\ b{\isacharparenright}{\kern0pt}\isanewline
\ \ \ \ \ \ \ \isacommand{using}\isamarkupfalse%
\ True\ \isacommand{apply}\isamarkupfalse%
\ simp\isanewline
\ \ \ \ \ \ \isacommand{by}\isamarkupfalse%
\ simp\isanewline
\ \ \ \ \isacommand{also}\isamarkupfalse%
\ \isacommand{have}\isamarkupfalse%
\ {\isachardoublequoteopen}{\isachardot}{\kern0pt}{\isachardot}{\kern0pt}{\isachardot}{\kern0pt}\ {\isacharequal}{\kern0pt}\ {\isasymbar}real{\isacharunderscore}{\kern0pt}of{\isacharunderscore}{\kern0pt}int\ {\isacharparenleft}{\kern0pt}mantissa\ f{\isacharparenright}{\kern0pt}{\isasymbar}\ {\isacharasterisk}{\kern0pt}\ {\isacharparenleft}{\kern0pt}{\isadigit{2}}\ {\isacharasterisk}{\kern0pt}\ real\ i{\isacharparenright}{\kern0pt}\ {\isacharplus}{\kern0pt}\ {\isadigit{1}}{\isachardoublequoteclose}\isanewline
\ \ \ \ \ \ \isacommand{by}\isamarkupfalse%
\ simp\isanewline
\ \ \ \ \isacommand{also}\isamarkupfalse%
\ \isacommand{have}\isamarkupfalse%
\ {\isachardoublequoteopen}{\isachardot}{\kern0pt}{\isachardot}{\kern0pt}{\isachardot}{\kern0pt}\ {\isasymle}\ {\isasymbar}real{\isacharunderscore}{\kern0pt}of{\isacharunderscore}{\kern0pt}int\ {\isacharparenleft}{\kern0pt}mantissa\ f{\isacharparenright}{\kern0pt}{\isasymbar}\ {\isacharasterisk}{\kern0pt}\ {\isadigit{2}}\ {\isacharcircum}{\kern0pt}\ i\ {\isacharplus}{\kern0pt}\ {\isadigit{1}}{\isachardoublequoteclose}\isanewline
\ \ \ \ \ \ \isacommand{apply}\isamarkupfalse%
\ {\isacharparenleft}{\kern0pt}rule\ add{\isacharunderscore}{\kern0pt}mono{\isacharparenright}{\kern0pt}\isanewline
\ \ \ \ \ \ \ \isacommand{apply}\isamarkupfalse%
\ {\isacharparenleft}{\kern0pt}rule\ mult{\isacharunderscore}{\kern0pt}left{\isacharunderscore}{\kern0pt}mono{\isacharparenright}{\kern0pt}\ \isanewline
\ \ \ \ \ \ \ \isacommand{using}\isamarkupfalse%
\ c\ of{\isacharunderscore}{\kern0pt}nat{\isacharunderscore}{\kern0pt}mono\ \isacommand{apply}\isamarkupfalse%
\ fastforce\isanewline
\ \ \ \ \ \ \isacommand{by}\isamarkupfalse%
\ simp{\isacharplus}{\kern0pt}\isanewline
\ \ \ \ \isacommand{finally}\isamarkupfalse%
\ \isacommand{show}\isamarkupfalse%
\ {\isacharquery}{\kern0pt}thesis\ \isacommand{by}\isamarkupfalse%
\ simp\isanewline
\ \ \isacommand{next}\isamarkupfalse%
\isanewline
\ \ \ \ \isacommand{case}\isamarkupfalse%
\ False\isanewline
\ \ \ \ \isacommand{hence}\isamarkupfalse%
\ {\isachardoublequoteopen}i\ {\isacharequal}{\kern0pt}\ {\isadigit{0}}{\isachardoublequoteclose}\ \isacommand{by}\isamarkupfalse%
\ simp\isanewline
\ \ \ \ \isacommand{then}\isamarkupfalse%
\ \isacommand{show}\isamarkupfalse%
\ {\isacharquery}{\kern0pt}thesis\ \isacommand{by}\isamarkupfalse%
\ simp\isanewline
\ \ \isacommand{qed}\isamarkupfalse%
\ \isanewline
\isanewline
\ \ \isacommand{have}\isamarkupfalse%
\ {\isachardoublequoteopen}bit{\isacharunderscore}{\kern0pt}count\ {\isacharparenleft}{\kern0pt}F\isactrlsub S\ f{\isacharparenright}{\kern0pt}\ {\isacharequal}{\kern0pt}\ bit{\isacharunderscore}{\kern0pt}count\ {\isacharparenleft}{\kern0pt}I\isactrlsub S\ {\isacharparenleft}{\kern0pt}mantissa\ f{\isacharparenright}{\kern0pt}{\isacharparenright}{\kern0pt}\ {\isacharplus}{\kern0pt}\ bit{\isacharunderscore}{\kern0pt}count\ {\isacharparenleft}{\kern0pt}I\isactrlsub S\ {\isacharparenleft}{\kern0pt}exponent\ f{\isacharparenright}{\kern0pt}{\isacharparenright}{\kern0pt}{\isachardoublequoteclose}\isanewline
\ \ \ \ \isacommand{by}\isamarkupfalse%
\ {\isacharparenleft}{\kern0pt}simp\ add{\isacharcolon}{\kern0pt}f{\isacharunderscore}{\kern0pt}def\ F\isactrlsub S{\isacharunderscore}{\kern0pt}def{\isacharparenright}{\kern0pt}\isanewline
\ \ \isacommand{also}\isamarkupfalse%
\ \isacommand{have}\isamarkupfalse%
\ {\isachardoublequoteopen}{\isachardot}{\kern0pt}{\isachardot}{\kern0pt}{\isachardot}{\kern0pt}\ {\isasymle}\ \isanewline
\ \ \ \ \ \ ereal\ {\isacharparenleft}{\kern0pt}{\isadigit{2}}\ {\isacharasterisk}{\kern0pt}\ {\isacharparenleft}{\kern0pt}log\ {\isadigit{2}}\ {\isacharparenleft}{\kern0pt}real{\isacharunderscore}{\kern0pt}of{\isacharunderscore}{\kern0pt}int\ {\isacharparenleft}{\kern0pt}abs\ {\isacharparenleft}{\kern0pt}mantissa\ f{\isacharparenright}{\kern0pt}\ {\isacharplus}{\kern0pt}\ {\isadigit{1}}{\isacharparenright}{\kern0pt}{\isacharparenright}{\kern0pt}{\isacharparenright}{\kern0pt}{\isacharplus}{\kern0pt}\ {\isadigit{2}}{\isacharparenright}{\kern0pt}\ {\isacharplus}{\kern0pt}\ \isanewline
\ \ \ \ \ \ ereal\ {\isacharparenleft}{\kern0pt}{\isadigit{2}}\ {\isacharasterisk}{\kern0pt}\ {\isacharparenleft}{\kern0pt}log\ {\isadigit{2}}\ {\isacharparenleft}{\kern0pt}real{\isacharunderscore}{\kern0pt}of{\isacharunderscore}{\kern0pt}int\ {\isacharparenleft}{\kern0pt}abs\ {\isacharparenleft}{\kern0pt}exponent\ f{\isacharparenright}{\kern0pt}\ {\isacharplus}{\kern0pt}\ {\isadigit{1}}{\isacharparenright}{\kern0pt}{\isacharparenright}{\kern0pt}{\isacharparenright}{\kern0pt}{\isacharplus}{\kern0pt}\ {\isadigit{2}}{\isacharparenright}{\kern0pt}{\isachardoublequoteclose}\isanewline
\ \ \ \ \isacommand{by}\isamarkupfalse%
\ {\isacharparenleft}{\kern0pt}rule\ add{\isacharunderscore}{\kern0pt}mono{\isacharcomma}{\kern0pt}\ rule\ int{\isacharunderscore}{\kern0pt}bit{\isacharunderscore}{\kern0pt}count{\isacharcomma}{\kern0pt}\ rule\ int{\isacharunderscore}{\kern0pt}bit{\isacharunderscore}{\kern0pt}count{\isacharparenright}{\kern0pt}\isanewline
\ \ \isacommand{also}\isamarkupfalse%
\ \isacommand{have}\isamarkupfalse%
\ {\isachardoublequoteopen}{\isachardot}{\kern0pt}{\isachardot}{\kern0pt}{\isachardot}{\kern0pt}\ {\isacharequal}{\kern0pt}\ ereal\ {\isacharparenleft}{\kern0pt}{\isadigit{4}}\ {\isacharplus}{\kern0pt}\ {\isadigit{2}}\ {\isacharasterisk}{\kern0pt}\ {\isacharparenleft}{\kern0pt}log\ {\isadigit{2}}\ {\isacharparenleft}{\kern0pt}real{\isacharunderscore}{\kern0pt}of{\isacharunderscore}{\kern0pt}int\ {\isacharparenleft}{\kern0pt}abs\ {\isacharparenleft}{\kern0pt}mantissa\ f{\isacharparenright}{\kern0pt}{\isacharparenright}{\kern0pt}\ {\isacharplus}{\kern0pt}\ {\isadigit{1}}{\isacharparenright}{\kern0pt}\ {\isacharplus}{\kern0pt}\ \isanewline
\ \ \ \ \ \ \ \ \ \ \ \ \ \ \ \ \ \ \ \ \ \ \ \ \ \ \ \ \ \ \ \ \ \ \ log\ {\isadigit{2}}\ {\isacharparenleft}{\kern0pt}real{\isacharunderscore}{\kern0pt}of{\isacharunderscore}{\kern0pt}int\ {\isacharparenleft}{\kern0pt}abs\ {\isacharparenleft}{\kern0pt}e\ {\isacharplus}{\kern0pt}\ i{\isacharparenright}{\kern0pt}{\isacharparenright}{\kern0pt}\ {\isacharplus}{\kern0pt}\ {\isadigit{1}}{\isacharparenright}{\kern0pt}{\isacharparenright}{\kern0pt}{\isacharparenright}{\kern0pt}{\isachardoublequoteclose}\isanewline
\ \ \ \ \isacommand{by}\isamarkupfalse%
\ {\isacharparenleft}{\kern0pt}simp\ add{\isacharcolon}{\kern0pt}algebra{\isacharunderscore}{\kern0pt}simps\ e{\isacharunderscore}{\kern0pt}def{\isacharparenright}{\kern0pt}\isanewline
\ \ \isacommand{also}\isamarkupfalse%
\ \isacommand{have}\isamarkupfalse%
\ {\isachardoublequoteopen}{\isachardot}{\kern0pt}{\isachardot}{\kern0pt}{\isachardot}{\kern0pt}\ {\isasymle}\ ereal\ {\isacharparenleft}{\kern0pt}{\isadigit{4}}\ {\isacharplus}{\kern0pt}\ {\isadigit{2}}\ {\isacharasterisk}{\kern0pt}\ {\isacharparenleft}{\kern0pt}log\ {\isadigit{2}}\ {\isacharparenleft}{\kern0pt}real{\isacharunderscore}{\kern0pt}of{\isacharunderscore}{\kern0pt}int\ {\isacharparenleft}{\kern0pt}abs\ {\isacharparenleft}{\kern0pt}mantissa\ f{\isacharparenright}{\kern0pt}{\isacharparenright}{\kern0pt}\ {\isacharplus}{\kern0pt}\ {\isadigit{1}}{\isacharparenright}{\kern0pt}\ {\isacharplus}{\kern0pt}\isanewline
\ \ \ \ \ \ \ \ \ \ \ \ \ \ \ \ \ \ \ \ \ \ \ \ \ \ \ \ \ \ \ \ \ \ \ \ log\ {\isadigit{2}}\ {\isacharparenleft}{\kern0pt}real\ i{\isacharplus}{\kern0pt}{\isadigit{1}}{\isacharparenright}{\kern0pt}\ {\isacharplus}{\kern0pt}\isanewline
\ \ \ \ \ \ \ \ \ \ \ \ \ \ \ \ \ \ \ \ \ \ \ \ \ \ \ \ \ \ \ \ \ \ \ \ log\ {\isadigit{2}}\ {\isacharparenleft}{\kern0pt}abs\ e\ {\isacharplus}{\kern0pt}\ {\isadigit{1}}{\isacharparenright}{\kern0pt}{\isacharparenright}{\kern0pt}{\isacharparenright}{\kern0pt}{\isachardoublequoteclose}\isanewline
\ \ \ \ \isacommand{apply}\isamarkupfalse%
\ {\isacharparenleft}{\kern0pt}simp{\isacharparenright}{\kern0pt}\isanewline
\ \ \ \ \isacommand{apply}\isamarkupfalse%
\ {\isacharparenleft}{\kern0pt}subst\ distrib{\isacharunderscore}{\kern0pt}left{\isacharbrackleft}{\kern0pt}symmetric{\isacharbrackright}{\kern0pt}{\isacharparenright}{\kern0pt}\isanewline
\ \ \ \ \isacommand{apply}\isamarkupfalse%
\ {\isacharparenleft}{\kern0pt}rule\ mult{\isacharunderscore}{\kern0pt}left{\isacharunderscore}{\kern0pt}mono{\isacharparenright}{\kern0pt}\isanewline
\ \ \ \ \ \isacommand{apply}\isamarkupfalse%
\ {\isacharparenleft}{\kern0pt}subst\ log{\isacharunderscore}{\kern0pt}mult{\isacharbrackleft}{\kern0pt}symmetric{\isacharbrackright}{\kern0pt}{\isacharcomma}{\kern0pt}\ simp{\isacharcomma}{\kern0pt}\ simp{\isacharcomma}{\kern0pt}\ simp{\isacharcomma}{\kern0pt}\ simp{\isacharparenright}{\kern0pt}\isanewline
\ \ \ \ \ \isacommand{apply}\isamarkupfalse%
\ {\isacharparenleft}{\kern0pt}subst\ log{\isacharunderscore}{\kern0pt}le{\isacharunderscore}{\kern0pt}cancel{\isacharunderscore}{\kern0pt}iff{\isacharcomma}{\kern0pt}\ simp{\isacharcomma}{\kern0pt}\ simp{\isacharcomma}{\kern0pt}\ simp{\isacharparenright}{\kern0pt}\isanewline
\ \ \ \ \isacommand{apply}\isamarkupfalse%
\ {\isacharparenleft}{\kern0pt}rule\ order{\isacharunderscore}{\kern0pt}trans{\isacharbrackleft}{\kern0pt}\isakeyword{where}\ y{\isacharequal}{\kern0pt}{\isachardoublequoteopen}\ abs\ e\ {\isacharplus}{\kern0pt}\ real\ i\ {\isacharplus}{\kern0pt}\ {\isadigit{1}}{\isachardoublequoteclose}{\isacharbrackright}{\kern0pt}{\isacharcomma}{\kern0pt}\ simp{\isacharparenright}{\kern0pt}\isanewline
\ \ \ \ \isacommand{by}\isamarkupfalse%
\ {\isacharparenleft}{\kern0pt}simp\ add{\isacharcolon}{\kern0pt}algebra{\isacharunderscore}{\kern0pt}simps{\isacharcomma}{\kern0pt}\ simp{\isacharparenright}{\kern0pt}\isanewline
\ \ \isacommand{also}\isamarkupfalse%
\ \isacommand{have}\isamarkupfalse%
\ {\isachardoublequoteopen}{\isachardot}{\kern0pt}{\isachardot}{\kern0pt}{\isachardot}{\kern0pt}\ {\isasymle}\ ereal\ {\isacharparenleft}{\kern0pt}{\isadigit{4}}\ {\isacharplus}{\kern0pt}\ {\isadigit{2}}\ {\isacharasterisk}{\kern0pt}\ {\isacharparenleft}{\kern0pt}log\ {\isadigit{2}}\ {\isacharparenleft}{\kern0pt}real{\isacharunderscore}{\kern0pt}of{\isacharunderscore}{\kern0pt}int\ {\isacharparenleft}{\kern0pt}abs\ {\isacharparenleft}{\kern0pt}mantissa\ f\ {\isacharasterisk}{\kern0pt}\ {\isadigit{2}}\ {\isacharcircum}{\kern0pt}\ i{\isacharparenright}{\kern0pt}{\isacharparenright}{\kern0pt}\ {\isacharplus}{\kern0pt}\ {\isadigit{2}}{\isacharparenright}{\kern0pt}\ {\isacharplus}{\kern0pt}\isanewline
\ \ \ \ log\ {\isadigit{2}}\ {\isacharparenleft}{\kern0pt}abs\ e\ {\isacharplus}{\kern0pt}\ {\isadigit{1}}{\isacharparenright}{\kern0pt}{\isacharparenright}{\kern0pt}{\isacharparenright}{\kern0pt}{\isachardoublequoteclose}\isanewline
\ \ \ \ \isacommand{apply}\isamarkupfalse%
\ {\isacharparenleft}{\kern0pt}simp{\isacharparenright}{\kern0pt}\isanewline
\ \ \ \ \isacommand{apply}\isamarkupfalse%
\ {\isacharparenleft}{\kern0pt}subst\ distrib{\isacharunderscore}{\kern0pt}left{\isacharbrackleft}{\kern0pt}symmetric{\isacharbrackright}{\kern0pt}{\isacharparenright}{\kern0pt}\isanewline
\ \ \ \ \isacommand{apply}\isamarkupfalse%
\ {\isacharparenleft}{\kern0pt}rule\ mult{\isacharunderscore}{\kern0pt}left{\isacharunderscore}{\kern0pt}mono{\isacharparenright}{\kern0pt}\isanewline
\ \ \ \ \ \isacommand{apply}\isamarkupfalse%
\ {\isacharparenleft}{\kern0pt}subst\ log{\isacharunderscore}{\kern0pt}mult{\isacharbrackleft}{\kern0pt}symmetric{\isacharbrackright}{\kern0pt}{\isacharcomma}{\kern0pt}\ simp{\isacharcomma}{\kern0pt}\ simp{\isacharcomma}{\kern0pt}\ simp{\isacharcomma}{\kern0pt}\ simp{\isacharparenright}{\kern0pt}\isanewline
\ \ \ \ \ \isacommand{apply}\isamarkupfalse%
\ {\isacharparenleft}{\kern0pt}subst\ log{\isacharunderscore}{\kern0pt}le{\isacharunderscore}{\kern0pt}cancel{\isacharunderscore}{\kern0pt}iff{\isacharcomma}{\kern0pt}\ simp{\isacharcomma}{\kern0pt}\ simp{\isacharcomma}{\kern0pt}\ simp{\isacharparenright}{\kern0pt}\isanewline
\ \ \ \ \ \isacommand{apply}\isamarkupfalse%
\ {\isacharparenleft}{\kern0pt}subst\ abs{\isacharunderscore}{\kern0pt}mult{\isacharparenright}{\kern0pt}\isanewline
\ \ \ \ \ \isacommand{using}\isamarkupfalse%
\ a\ \isacommand{apply}\isamarkupfalse%
\ {\isacharparenleft}{\kern0pt}simp\ add{\isacharcolon}{\kern0pt}\ distrib{\isacharunderscore}{\kern0pt}right{\isacharparenright}{\kern0pt}\isanewline
\ \ \ \ \isacommand{by}\isamarkupfalse%
\ simp\isanewline
\ \ \isacommand{also}\isamarkupfalse%
\ \isacommand{have}\isamarkupfalse%
\ {\isachardoublequoteopen}{\isachardot}{\kern0pt}{\isachardot}{\kern0pt}{\isachardot}{\kern0pt}\ {\isacharequal}{\kern0pt}\ ereal\ {\isacharparenleft}{\kern0pt}{\isadigit{4}}\ {\isacharplus}{\kern0pt}\ {\isadigit{2}}\ {\isacharasterisk}{\kern0pt}\ {\isacharparenleft}{\kern0pt}log\ {\isadigit{2}}\ {\isacharparenleft}{\kern0pt}real{\isacharunderscore}{\kern0pt}of{\isacharunderscore}{\kern0pt}int\ {\isacharparenleft}{\kern0pt}abs\ m{\isacharparenright}{\kern0pt}\ {\isacharplus}{\kern0pt}\ {\isadigit{2}}{\isacharparenright}{\kern0pt}\ {\isacharplus}{\kern0pt}\ log\ {\isadigit{2}}\ {\isacharparenleft}{\kern0pt}abs\ e\ {\isacharplus}{\kern0pt}\ {\isadigit{1}}{\isacharparenright}{\kern0pt}{\isacharparenright}{\kern0pt}{\isacharparenright}{\kern0pt}{\isachardoublequoteclose}\isanewline
\ \ \ \ \isacommand{by}\isamarkupfalse%
\ {\isacharparenleft}{\kern0pt}simp\ add{\isacharcolon}{\kern0pt}m{\isacharunderscore}{\kern0pt}def{\isacharparenright}{\kern0pt}\isanewline
\ \ \isacommand{finally}\isamarkupfalse%
\ \isacommand{show}\isamarkupfalse%
\ {\isacharquery}{\kern0pt}thesis\ \isacommand{by}\isamarkupfalse%
\ {\isacharparenleft}{\kern0pt}simp\ add{\isacharcolon}{\kern0pt}f{\isacharunderscore}{\kern0pt}def{\isacharbrackleft}{\kern0pt}symmetric{\isacharbrackright}{\kern0pt}\ bit{\isacharunderscore}{\kern0pt}count{\isacharunderscore}{\kern0pt}append\ del{\isacharcolon}{\kern0pt}N\isactrlsub S{\isachardot}{\kern0pt}simps\ I\isactrlsub S{\isachardot}{\kern0pt}simps{\isacharparenright}{\kern0pt}\isanewline
\isacommand{next}\isamarkupfalse%
\isanewline
\ \ \isacommand{case}\isamarkupfalse%
\ False\isanewline
\ \ \isacommand{hence}\isamarkupfalse%
\ {\isachardoublequoteopen}float{\isacharunderscore}{\kern0pt}of\ {\isacharparenleft}{\kern0pt}m\ {\isacharasterisk}{\kern0pt}\ {\isadigit{2}}\ powr\ e{\isacharparenright}{\kern0pt}\ {\isacharequal}{\kern0pt}\ Float\ {\isadigit{0}}\ {\isadigit{0}}{\isachardoublequoteclose}\isanewline
\ \ \ \ \isacommand{apply}\isamarkupfalse%
\ simp\ \isanewline
\ \ \ \ \isacommand{using}\isamarkupfalse%
\ zero{\isacharunderscore}{\kern0pt}float{\isachardot}{\kern0pt}abs{\isacharunderscore}{\kern0pt}eq\ \isacommand{by}\isamarkupfalse%
\ linarith\isanewline
\ \ \isacommand{then}\isamarkupfalse%
\ \isacommand{show}\isamarkupfalse%
\ {\isacharquery}{\kern0pt}thesis\ \isacommand{by}\isamarkupfalse%
\ {\isacharparenleft}{\kern0pt}simp\ add{\isacharcolon}{\kern0pt}f{\isacharunderscore}{\kern0pt}def\ F\isactrlsub S{\isacharunderscore}{\kern0pt}def{\isacharparenright}{\kern0pt}\isanewline
\isacommand{qed}\isamarkupfalse%
%
\endisatagproof
{\isafoldproof}%
%
\isadelimproof
\isanewline
%
\endisadelimproof
\isanewline
\isacommand{lemma}\isamarkupfalse%
\ float{\isacharunderscore}{\kern0pt}bit{\isacharunderscore}{\kern0pt}count{\isacharunderscore}{\kern0pt}zero{\isacharcolon}{\kern0pt}\isanewline
\ \ {\isachardoublequoteopen}bit{\isacharunderscore}{\kern0pt}count\ {\isacharparenleft}{\kern0pt}F\isactrlsub S\ {\isacharparenleft}{\kern0pt}float{\isacharunderscore}{\kern0pt}of\ {\isadigit{0}}{\isacharparenright}{\kern0pt}{\isacharparenright}{\kern0pt}\ {\isacharequal}{\kern0pt}\ {\isadigit{4}}{\isachardoublequoteclose}\isanewline
%
\isadelimproof
\ \ %
\endisadelimproof
%
\isatagproof
\isacommand{apply}\isamarkupfalse%
\ {\isacharparenleft}{\kern0pt}subst\ zero{\isacharunderscore}{\kern0pt}float{\isachardot}{\kern0pt}abs{\isacharunderscore}{\kern0pt}eq{\isacharbrackleft}{\kern0pt}symmetric{\isacharbrackright}{\kern0pt}{\isacharparenright}{\kern0pt}\isanewline
\ \ \isacommand{by}\isamarkupfalse%
\ {\isacharparenleft}{\kern0pt}simp\ add{\isacharcolon}{\kern0pt}F\isactrlsub S{\isacharunderscore}{\kern0pt}def{\isacharparenright}{\kern0pt}%
\endisatagproof
{\isafoldproof}%
%
\isadelimproof
\isanewline
%
\endisadelimproof
\isanewline
\isacommand{lemma}\isamarkupfalse%
\ log{\isacharunderscore}{\kern0pt}est{\isacharcolon}{\kern0pt}\ {\isachardoublequoteopen}log\ {\isadigit{2}}\ {\isacharparenleft}{\kern0pt}real\ n\ {\isacharplus}{\kern0pt}\ {\isadigit{1}}{\isacharparenright}{\kern0pt}\ {\isasymle}\ n{\isachardoublequoteclose}\isanewline
%
\isadelimproof
%
\endisadelimproof
%
\isatagproof
\isacommand{proof}\isamarkupfalse%
\ {\isacharminus}{\kern0pt}\isanewline
\ \ \isacommand{have}\isamarkupfalse%
\ {\isachardoublequoteopen}{\isadigit{1}}\ {\isacharplus}{\kern0pt}\ real\ n\ {\isasymle}\ {\isadigit{2}}\ powr\ {\isacharparenleft}{\kern0pt}real\ n{\isacharparenright}{\kern0pt}{\isachardoublequoteclose}\isanewline
\ \ \ \ \isacommand{using}\isamarkupfalse%
\ suc{\isacharunderscore}{\kern0pt}n{\isacharunderscore}{\kern0pt}le{\isacharunderscore}{\kern0pt}{\isadigit{2}}{\isacharunderscore}{\kern0pt}pow{\isacharunderscore}{\kern0pt}n\ \isacommand{apply}\isamarkupfalse%
\ {\isacharparenleft}{\kern0pt}simp\ add{\isacharcolon}{\kern0pt}\ powr{\isacharunderscore}{\kern0pt}realpow{\isacharparenright}{\kern0pt}\isanewline
\ \ \ \ \isacommand{by}\isamarkupfalse%
\ {\isacharparenleft}{\kern0pt}metis\ numeral{\isacharunderscore}{\kern0pt}power{\isacharunderscore}{\kern0pt}eq{\isacharunderscore}{\kern0pt}of{\isacharunderscore}{\kern0pt}nat{\isacharunderscore}{\kern0pt}cancel{\isacharunderscore}{\kern0pt}iff\ of{\isacharunderscore}{\kern0pt}nat{\isacharunderscore}{\kern0pt}Suc\ of{\isacharunderscore}{\kern0pt}nat{\isacharunderscore}{\kern0pt}mono{\isacharparenright}{\kern0pt}\isanewline
\ \ \isacommand{thus}\isamarkupfalse%
\ {\isacharquery}{\kern0pt}thesis\ \isanewline
\ \ \ \ \isacommand{by}\isamarkupfalse%
\ {\isacharparenleft}{\kern0pt}simp\ add{\isacharcolon}{\kern0pt}\ Transcendental{\isachardot}{\kern0pt}log{\isacharunderscore}{\kern0pt}le{\isacharunderscore}{\kern0pt}iff{\isacharparenright}{\kern0pt}\isanewline
\isacommand{qed}\isamarkupfalse%
%
\endisatagproof
{\isafoldproof}%
%
\isadelimproof
\isanewline
%
\endisadelimproof
\isanewline
\isacommand{lemma}\isamarkupfalse%
\ truncate{\isacharunderscore}{\kern0pt}float{\isacharunderscore}{\kern0pt}bit{\isacharunderscore}{\kern0pt}count{\isacharcolon}{\kern0pt}\isanewline
\ \ {\isachardoublequoteopen}bit{\isacharunderscore}{\kern0pt}count\ {\isacharparenleft}{\kern0pt}F\isactrlsub S\ {\isacharparenleft}{\kern0pt}float{\isacharunderscore}{\kern0pt}of\ {\isacharparenleft}{\kern0pt}truncate{\isacharunderscore}{\kern0pt}down\ r\ x{\isacharparenright}{\kern0pt}{\isacharparenright}{\kern0pt}{\isacharparenright}{\kern0pt}\ {\isasymle}\ {\isadigit{8}}\ {\isacharplus}{\kern0pt}\ {\isadigit{4}}\ {\isacharasterisk}{\kern0pt}\ real\ r\ {\isacharplus}{\kern0pt}\ {\isadigit{2}}{\isacharasterisk}{\kern0pt}log\ {\isadigit{2}}\ {\isacharparenleft}{\kern0pt}{\isadigit{2}}\ {\isacharplus}{\kern0pt}\ abs\ {\isacharparenleft}{\kern0pt}log\ {\isadigit{2}}\ {\isacharparenleft}{\kern0pt}abs\ x{\isacharparenright}{\kern0pt}{\isacharparenright}{\kern0pt}{\isacharparenright}{\kern0pt}{\isachardoublequoteclose}\ \isanewline
\ \ {\isacharparenleft}{\kern0pt}\isakeyword{is}\ {\isachardoublequoteopen}{\isacharquery}{\kern0pt}lhs\ {\isasymle}\ {\isacharquery}{\kern0pt}rhs{\isachardoublequoteclose}{\isacharparenright}{\kern0pt}\isanewline
%
\isadelimproof
%
\endisadelimproof
%
\isatagproof
\isacommand{proof}\isamarkupfalse%
\ {\isacharminus}{\kern0pt}\isanewline
\ \ \isacommand{define}\isamarkupfalse%
\ m\ \isakeyword{where}\ {\isachardoublequoteopen}m\ {\isacharequal}{\kern0pt}\ {\isasymlfloor}x\ {\isacharasterisk}{\kern0pt}\ {\isadigit{2}}\ powr\ {\isacharparenleft}{\kern0pt}real\ r\ {\isacharminus}{\kern0pt}\ real{\isacharunderscore}{\kern0pt}of{\isacharunderscore}{\kern0pt}int\ {\isasymlfloor}log\ {\isadigit{2}}\ {\isasymbar}x{\isasymbar}{\isasymrfloor}{\isacharparenright}{\kern0pt}{\isasymrfloor}{\isachardoublequoteclose}\isanewline
\ \ \isacommand{define}\isamarkupfalse%
\ e\ \isakeyword{where}\ {\isachardoublequoteopen}e\ {\isacharequal}{\kern0pt}\ {\isasymlfloor}log\ {\isadigit{2}}\ {\isasymbar}x{\isasymbar}{\isasymrfloor}\ {\isacharminus}{\kern0pt}\ int\ r{\isachardoublequoteclose}\isanewline
\isanewline
\ \ \isacommand{have}\isamarkupfalse%
\ a{\isacharcolon}{\kern0pt}\ {\isachardoublequoteopen}real\ r\ {\isacharequal}{\kern0pt}\ real{\isacharunderscore}{\kern0pt}of{\isacharunderscore}{\kern0pt}int\ {\isacharparenleft}{\kern0pt}int\ r{\isacharparenright}{\kern0pt}{\isachardoublequoteclose}\ \isacommand{by}\isamarkupfalse%
\ simp\isanewline
\ \ \isacommand{have}\isamarkupfalse%
\ {\isachardoublequoteopen}abs\ m\ {\isacharplus}{\kern0pt}\ {\isadigit{2}}\ {\isasymle}\ {\isadigit{2}}\ {\isacharcircum}{\kern0pt}\ {\isacharparenleft}{\kern0pt}r\ {\isacharplus}{\kern0pt}\ {\isadigit{1}}{\isacharparenright}{\kern0pt}\ {\isacharplus}{\kern0pt}\ {\isadigit{2}}{\isacharcircum}{\kern0pt}{\isadigit{1}}{\isachardoublequoteclose}\isanewline
\ \ \ \ \isacommand{apply}\isamarkupfalse%
\ {\isacharparenleft}{\kern0pt}rule\ add{\isacharunderscore}{\kern0pt}mono{\isacharparenright}{\kern0pt}\isanewline
\ \ \ \ \ \isacommand{using}\isamarkupfalse%
\ truncate{\isacharunderscore}{\kern0pt}mantissa{\isacharunderscore}{\kern0pt}bound\ \isacommand{apply}\isamarkupfalse%
\ {\isacharparenleft}{\kern0pt}simp\ add{\isacharcolon}{\kern0pt}m{\isacharunderscore}{\kern0pt}def{\isacharparenright}{\kern0pt}\isanewline
\ \ \ \ \isacommand{by}\isamarkupfalse%
\ simp\isanewline
\ \ \isacommand{also}\isamarkupfalse%
\ \isacommand{have}\isamarkupfalse%
\ {\isachardoublequoteopen}{\isachardot}{\kern0pt}{\isachardot}{\kern0pt}{\isachardot}{\kern0pt}\ {\isasymle}\ {\isadigit{2}}\ {\isacharcircum}{\kern0pt}\ {\isacharparenleft}{\kern0pt}r{\isacharplus}{\kern0pt}{\isadigit{2}}{\isacharparenright}{\kern0pt}{\isachardoublequoteclose}\isanewline
\ \ \ \ \isacommand{by}\isamarkupfalse%
\ simp\isanewline
\ \ \isacommand{finally}\isamarkupfalse%
\ \isacommand{have}\isamarkupfalse%
\ b{\isacharcolon}{\kern0pt}{\isachardoublequoteopen}abs\ m\ {\isacharplus}{\kern0pt}\ {\isadigit{2}}\ {\isasymle}\ {\isadigit{2}}\ {\isacharcircum}{\kern0pt}\ {\isacharparenleft}{\kern0pt}r{\isacharplus}{\kern0pt}{\isadigit{2}}{\isacharparenright}{\kern0pt}{\isachardoublequoteclose}\ \isacommand{by}\isamarkupfalse%
\ simp\isanewline
\ \ \isacommand{have}\isamarkupfalse%
\ c{\isacharcolon}{\kern0pt}{\isachardoublequoteopen}log\ {\isadigit{2}}\ {\isacharparenleft}{\kern0pt}real{\isacharunderscore}{\kern0pt}of{\isacharunderscore}{\kern0pt}int\ {\isacharparenleft}{\kern0pt}{\isasymbar}m{\isasymbar}\ {\isacharplus}{\kern0pt}\ {\isadigit{2}}{\isacharparenright}{\kern0pt}{\isacharparenright}{\kern0pt}\ {\isasymle}\ r{\isacharplus}{\kern0pt}{\isadigit{2}}{\isachardoublequoteclose}\isanewline
\ \ \ \ \isacommand{apply}\isamarkupfalse%
\ {\isacharparenleft}{\kern0pt}subst\ Transcendental{\isachardot}{\kern0pt}log{\isacharunderscore}{\kern0pt}le{\isacharunderscore}{\kern0pt}iff{\isacharcomma}{\kern0pt}\ simp{\isacharcomma}{\kern0pt}\ simp{\isacharparenright}{\kern0pt}\isanewline
\ \ \ \ \isacommand{apply}\isamarkupfalse%
\ {\isacharparenleft}{\kern0pt}subst\ powr{\isacharunderscore}{\kern0pt}realpow{\isacharcomma}{\kern0pt}\ simp{\isacharparenright}{\kern0pt}\isanewline
\ \ \ \ \isacommand{by}\isamarkupfalse%
\ {\isacharparenleft}{\kern0pt}metis\ of{\isacharunderscore}{\kern0pt}int{\isacharunderscore}{\kern0pt}le{\isacharunderscore}{\kern0pt}iff\ of{\isacharunderscore}{\kern0pt}int{\isacharunderscore}{\kern0pt}numeral\ of{\isacharunderscore}{\kern0pt}int{\isacharunderscore}{\kern0pt}power\ b{\isacharparenright}{\kern0pt}\isanewline
\isanewline
\ \ \isacommand{have}\isamarkupfalse%
\ {\isachardoublequoteopen}real{\isacharunderscore}{\kern0pt}of{\isacharunderscore}{\kern0pt}int\ {\isacharparenleft}{\kern0pt}abs\ e\ {\isacharplus}{\kern0pt}\ {\isadigit{1}}{\isacharparenright}{\kern0pt}\ {\isasymle}\ real{\isacharunderscore}{\kern0pt}of{\isacharunderscore}{\kern0pt}int\ {\isasymbar}{\isasymlfloor}log\ {\isadigit{2}}\ {\isasymbar}x{\isasymbar}{\isasymrfloor}{\isasymbar}\ {\isacharplus}{\kern0pt}\ \ real{\isacharunderscore}{\kern0pt}of{\isacharunderscore}{\kern0pt}int\ r\ {\isacharplus}{\kern0pt}\ {\isadigit{1}}{\isachardoublequoteclose}\isanewline
\ \ \ \ \isacommand{by}\isamarkupfalse%
\ {\isacharparenleft}{\kern0pt}simp\ add{\isacharcolon}{\kern0pt}e{\isacharunderscore}{\kern0pt}def{\isacharparenright}{\kern0pt}\isanewline
\ \ \isacommand{also}\isamarkupfalse%
\ \isacommand{have}\isamarkupfalse%
\ {\isachardoublequoteopen}{\isachardot}{\kern0pt}{\isachardot}{\kern0pt}{\isachardot}{\kern0pt}\ {\isasymle}\ {\isadigit{1}}\ {\isacharplus}{\kern0pt}\ abs\ {\isacharparenleft}{\kern0pt}log\ {\isadigit{2}}\ {\isacharparenleft}{\kern0pt}abs\ x{\isacharparenright}{\kern0pt}{\isacharparenright}{\kern0pt}\ {\isacharplus}{\kern0pt}\ real{\isacharunderscore}{\kern0pt}of{\isacharunderscore}{\kern0pt}int\ r\ {\isacharplus}{\kern0pt}\ {\isadigit{1}}{\isachardoublequoteclose}\isanewline
\ \ \ \ \isacommand{apply}\isamarkupfalse%
\ {\isacharparenleft}{\kern0pt}simp{\isacharparenright}{\kern0pt}\isanewline
\ \ \ \ \isacommand{apply}\isamarkupfalse%
\ {\isacharparenleft}{\kern0pt}subst\ abs{\isacharunderscore}{\kern0pt}le{\isacharunderscore}{\kern0pt}iff{\isacharparenright}{\kern0pt}\isanewline
\ \ \ \ \isacommand{by}\isamarkupfalse%
\ {\isacharparenleft}{\kern0pt}rule\ conjI{\isacharcomma}{\kern0pt}\ linarith{\isacharcomma}{\kern0pt}\ linarith{\isacharparenright}{\kern0pt}\isanewline
\ \ \isacommand{also}\isamarkupfalse%
\ \isacommand{have}\isamarkupfalse%
\ {\isachardoublequoteopen}{\isachardot}{\kern0pt}{\isachardot}{\kern0pt}{\isachardot}{\kern0pt}\ {\isasymle}\ {\isacharparenleft}{\kern0pt}real{\isacharunderscore}{\kern0pt}of{\isacharunderscore}{\kern0pt}int\ r{\isacharplus}{\kern0pt}\ {\isadigit{1}}{\isacharparenright}{\kern0pt}\ {\isacharasterisk}{\kern0pt}\ {\isacharparenleft}{\kern0pt}{\isadigit{2}}\ {\isacharplus}{\kern0pt}\ abs\ {\isacharparenleft}{\kern0pt}log\ {\isadigit{2}}\ {\isacharparenleft}{\kern0pt}abs\ x{\isacharparenright}{\kern0pt}{\isacharparenright}{\kern0pt}{\isacharparenright}{\kern0pt}{\isachardoublequoteclose}\isanewline
\ \ \ \ \isacommand{by}\isamarkupfalse%
\ {\isacharparenleft}{\kern0pt}simp\ add{\isacharcolon}{\kern0pt}distrib{\isacharunderscore}{\kern0pt}left\ distrib{\isacharunderscore}{\kern0pt}right{\isacharparenright}{\kern0pt}\isanewline
\ \ \isacommand{finally}\isamarkupfalse%
\ \isacommand{have}\isamarkupfalse%
\ d{\isacharcolon}{\kern0pt}{\isachardoublequoteopen}real{\isacharunderscore}{\kern0pt}of{\isacharunderscore}{\kern0pt}int\ {\isacharparenleft}{\kern0pt}abs\ e\ {\isacharplus}{\kern0pt}\ {\isadigit{1}}{\isacharparenright}{\kern0pt}\ {\isasymle}\ {\isacharparenleft}{\kern0pt}real{\isacharunderscore}{\kern0pt}of{\isacharunderscore}{\kern0pt}int\ r{\isacharplus}{\kern0pt}\ {\isadigit{1}}{\isacharparenright}{\kern0pt}\ {\isacharasterisk}{\kern0pt}\ {\isacharparenleft}{\kern0pt}{\isadigit{2}}\ {\isacharplus}{\kern0pt}\ abs\ {\isacharparenleft}{\kern0pt}log\ {\isadigit{2}}\ {\isacharparenleft}{\kern0pt}abs\ x{\isacharparenright}{\kern0pt}{\isacharparenright}{\kern0pt}{\isacharparenright}{\kern0pt}{\isachardoublequoteclose}\ \isacommand{by}\isamarkupfalse%
\ simp\isanewline
\isanewline
\ \ \isacommand{have}\isamarkupfalse%
\ {\isachardoublequoteopen}log\ {\isadigit{2}}\ {\isacharparenleft}{\kern0pt}real{\isacharunderscore}{\kern0pt}of{\isacharunderscore}{\kern0pt}int\ {\isacharparenleft}{\kern0pt}abs\ e\ {\isacharplus}{\kern0pt}\ {\isadigit{1}}{\isacharparenright}{\kern0pt}{\isacharparenright}{\kern0pt}\ {\isasymle}\ log\ {\isadigit{2}}\ {\isacharparenleft}{\kern0pt}real{\isacharunderscore}{\kern0pt}of{\isacharunderscore}{\kern0pt}int\ r\ {\isacharplus}{\kern0pt}\ {\isadigit{1}}{\isacharparenright}{\kern0pt}\ {\isacharplus}{\kern0pt}\ log\ {\isadigit{2}}\ {\isacharparenleft}{\kern0pt}{\isadigit{2}}\ {\isacharplus}{\kern0pt}\ abs\ {\isacharparenleft}{\kern0pt}log\ {\isadigit{2}}\ {\isacharparenleft}{\kern0pt}abs\ x{\isacharparenright}{\kern0pt}{\isacharparenright}{\kern0pt}{\isacharparenright}{\kern0pt}{\isachardoublequoteclose}\isanewline
\ \ \ \ \isacommand{apply}\isamarkupfalse%
\ {\isacharparenleft}{\kern0pt}subst\ log{\isacharunderscore}{\kern0pt}mult{\isacharbrackleft}{\kern0pt}symmetric{\isacharbrackright}{\kern0pt}{\isacharcomma}{\kern0pt}\ simp{\isacharcomma}{\kern0pt}\ simp{\isacharcomma}{\kern0pt}\ simp{\isacharcomma}{\kern0pt}\ simp{\isacharparenright}{\kern0pt}\isanewline
\ \ \ \ \isacommand{using}\isamarkupfalse%
\ d\ \isacommand{by}\isamarkupfalse%
\ simp\isanewline
\ \ \isacommand{also}\isamarkupfalse%
\ \isacommand{have}\isamarkupfalse%
\ {\isachardoublequoteopen}{\isachardot}{\kern0pt}{\isachardot}{\kern0pt}{\isachardot}{\kern0pt}\ {\isasymle}\ r\ {\isacharplus}{\kern0pt}\ log\ {\isadigit{2}}\ {\isacharparenleft}{\kern0pt}{\isadigit{2}}\ {\isacharplus}{\kern0pt}\ abs\ {\isacharparenleft}{\kern0pt}log\ {\isadigit{2}}\ {\isacharparenleft}{\kern0pt}abs\ x{\isacharparenright}{\kern0pt}{\isacharparenright}{\kern0pt}{\isacharparenright}{\kern0pt}{\isachardoublequoteclose}\isanewline
\ \ \ \ \isacommand{apply}\isamarkupfalse%
\ {\isacharparenleft}{\kern0pt}rule\ add{\isacharunderscore}{\kern0pt}mono{\isacharparenright}{\kern0pt}\isanewline
\ \ \ \ \isacommand{using}\isamarkupfalse%
\ log{\isacharunderscore}{\kern0pt}est\ \isacommand{apply}\isamarkupfalse%
\ {\isacharparenleft}{\kern0pt}simp\ add{\isacharcolon}{\kern0pt}add{\isachardot}{\kern0pt}commute{\isacharparenright}{\kern0pt}\isanewline
\ \ \ \ \isacommand{by}\isamarkupfalse%
\ simp\isanewline
\ \ \isacommand{finally}\isamarkupfalse%
\ \isacommand{have}\isamarkupfalse%
\ e{\isacharcolon}{\kern0pt}\ {\isachardoublequoteopen}log\ {\isadigit{2}}\ {\isacharparenleft}{\kern0pt}real{\isacharunderscore}{\kern0pt}of{\isacharunderscore}{\kern0pt}int\ {\isacharparenleft}{\kern0pt}abs\ e\ {\isacharplus}{\kern0pt}\ {\isadigit{1}}{\isacharparenright}{\kern0pt}{\isacharparenright}{\kern0pt}\ {\isasymle}\ r\ {\isacharplus}{\kern0pt}\ log\ {\isadigit{2}}\ {\isacharparenleft}{\kern0pt}{\isadigit{2}}\ {\isacharplus}{\kern0pt}\ abs\ {\isacharparenleft}{\kern0pt}log\ {\isadigit{2}}\ {\isacharparenleft}{\kern0pt}abs\ x{\isacharparenright}{\kern0pt}{\isacharparenright}{\kern0pt}{\isacharparenright}{\kern0pt}{\isachardoublequoteclose}\ \isacommand{by}\isamarkupfalse%
\ simp\isanewline
\isanewline
\ \ \isacommand{have}\isamarkupfalse%
\ {\isachardoublequoteopen}{\isacharquery}{\kern0pt}lhs\ {\isasymle}\ ereal\ {\isacharparenleft}{\kern0pt}{\isadigit{4}}\ {\isacharplus}{\kern0pt}\ {\isacharparenleft}{\kern0pt}{\isadigit{2}}\ {\isacharasterisk}{\kern0pt}\ log\ {\isadigit{2}}\ {\isacharparenleft}{\kern0pt}real{\isacharunderscore}{\kern0pt}of{\isacharunderscore}{\kern0pt}int\ {\isacharparenleft}{\kern0pt}{\isasymbar}m{\isasymbar}\ {\isacharplus}{\kern0pt}\ {\isadigit{2}}{\isacharparenright}{\kern0pt}{\isacharparenright}{\kern0pt}\ {\isacharplus}{\kern0pt}\ {\isadigit{2}}\ {\isacharasterisk}{\kern0pt}\ log\ {\isadigit{2}}\ {\isacharparenleft}{\kern0pt}real{\isacharunderscore}{\kern0pt}of{\isacharunderscore}{\kern0pt}int\ {\isacharparenleft}{\kern0pt}{\isasymbar}e{\isasymbar}\ {\isacharplus}{\kern0pt}\ {\isadigit{1}}{\isacharparenright}{\kern0pt}{\isacharparenright}{\kern0pt}{\isacharparenright}{\kern0pt}{\isacharparenright}{\kern0pt}{\isachardoublequoteclose}\isanewline
\ \ \ \ \isacommand{apply}\isamarkupfalse%
\ {\isacharparenleft}{\kern0pt}simp\ add{\isacharcolon}{\kern0pt}truncate{\isacharunderscore}{\kern0pt}down{\isacharunderscore}{\kern0pt}def\ round{\isacharunderscore}{\kern0pt}down{\isacharunderscore}{\kern0pt}def\ m{\isacharunderscore}{\kern0pt}def{\isacharbrackleft}{\kern0pt}symmetric{\isacharbrackright}{\kern0pt}{\isacharparenright}{\kern0pt}\isanewline
\ \ \ \ \isacommand{apply}\isamarkupfalse%
\ {\isacharparenleft}{\kern0pt}subst\ a{\isacharcomma}{\kern0pt}\ subst\ of{\isacharunderscore}{\kern0pt}int{\isacharunderscore}{\kern0pt}diff{\isacharbrackleft}{\kern0pt}symmetric{\isacharbrackright}{\kern0pt}{\isacharcomma}{\kern0pt}\ subst\ e{\isacharunderscore}{\kern0pt}def{\isacharbrackleft}{\kern0pt}symmetric{\isacharbrackright}{\kern0pt}{\isacharparenright}{\kern0pt}\isanewline
\ \ \ \ \isacommand{using}\isamarkupfalse%
\ float{\isacharunderscore}{\kern0pt}bit{\isacharunderscore}{\kern0pt}count\ \isacommand{by}\isamarkupfalse%
\ simp\isanewline
\ \ \isacommand{also}\isamarkupfalse%
\ \isacommand{have}\isamarkupfalse%
\ {\isachardoublequoteopen}{\isachardot}{\kern0pt}{\isachardot}{\kern0pt}{\isachardot}{\kern0pt}\ {\isasymle}\ ereal\ {\isacharparenleft}{\kern0pt}{\isadigit{4}}\ {\isacharplus}{\kern0pt}\ {\isacharparenleft}{\kern0pt}{\isadigit{2}}\ {\isacharasterisk}{\kern0pt}\ real\ {\isacharparenleft}{\kern0pt}r{\isacharplus}{\kern0pt}{\isadigit{2}}{\isacharparenright}{\kern0pt}\ {\isacharplus}{\kern0pt}\ {\isadigit{2}}\ {\isacharasterisk}{\kern0pt}\ {\isacharparenleft}{\kern0pt}r\ {\isacharplus}{\kern0pt}\ log\ {\isadigit{2}}\ {\isacharparenleft}{\kern0pt}{\isadigit{2}}\ {\isacharplus}{\kern0pt}\ abs\ {\isacharparenleft}{\kern0pt}log\ {\isadigit{2}}\ {\isacharparenleft}{\kern0pt}abs\ x{\isacharparenright}{\kern0pt}{\isacharparenright}{\kern0pt}{\isacharparenright}{\kern0pt}{\isacharparenright}{\kern0pt}{\isacharparenright}{\kern0pt}{\isacharparenright}{\kern0pt}{\isachardoublequoteclose}\isanewline
\ \ \ \ \isacommand{apply}\isamarkupfalse%
\ {\isacharparenleft}{\kern0pt}subst\ ereal{\isacharunderscore}{\kern0pt}less{\isacharunderscore}{\kern0pt}eq{\isacharparenright}{\kern0pt}\isanewline
\ \ \ \ \isacommand{apply}\isamarkupfalse%
\ {\isacharparenleft}{\kern0pt}rule\ add{\isacharunderscore}{\kern0pt}mono{\isacharcomma}{\kern0pt}\ simp{\isacharparenright}{\kern0pt}\isanewline
\ \ \ \ \isacommand{apply}\isamarkupfalse%
\ {\isacharparenleft}{\kern0pt}rule\ add{\isacharunderscore}{\kern0pt}mono{\isacharcomma}{\kern0pt}\ rule\ mult{\isacharunderscore}{\kern0pt}left{\isacharunderscore}{\kern0pt}mono{\isacharcomma}{\kern0pt}\ metis\ c{\isacharcomma}{\kern0pt}\ simp{\isacharparenright}{\kern0pt}\isanewline
\ \ \ \ \isacommand{by}\isamarkupfalse%
\ {\isacharparenleft}{\kern0pt}rule\ mult{\isacharunderscore}{\kern0pt}left{\isacharunderscore}{\kern0pt}mono{\isacharcomma}{\kern0pt}\ metis\ e{\isacharcomma}{\kern0pt}\ simp{\isacharparenright}{\kern0pt}\isanewline
\ \ \isacommand{also}\isamarkupfalse%
\ \isacommand{have}\isamarkupfalse%
\ {\isachardoublequoteopen}{\isachardot}{\kern0pt}{\isachardot}{\kern0pt}{\isachardot}{\kern0pt}\ {\isacharequal}{\kern0pt}\ {\isacharquery}{\kern0pt}rhs{\isachardoublequoteclose}\ \ \isacommand{by}\isamarkupfalse%
\ simp\isanewline
\ \ \isacommand{finally}\isamarkupfalse%
\ \isacommand{show}\isamarkupfalse%
\ {\isacharquery}{\kern0pt}thesis\ \isacommand{by}\isamarkupfalse%
\ simp\isanewline
\isacommand{qed}\isamarkupfalse%
%
\endisatagproof
{\isafoldproof}%
%
\isadelimproof
\isanewline
%
\endisadelimproof
%
\isadelimtheory
\isanewline
%
\endisadelimtheory
%
\isatagtheory
\isacommand{end}\isamarkupfalse%
%
\endisatagtheory
{\isafoldtheory}%
%
\isadelimtheory
%
\endisadelimtheory
%
\end{isabellebody}%
\endinput
%:%file=Float_Ext.tex%:%
%:%11=1%:%
%:%23=3%:%
%:%31=5%:%
%:%32=5%:%
%:%33=6%:%
%:%34=7%:%
%:%41=7%:%
%:%42=8%:%
%:%43=9%:%
%:%44=9%:%
%:%45=10%:%
%:%48=11%:%
%:%52=11%:%
%:%53=11%:%
%:%54=11%:%
%:%59=11%:%
%:%62=12%:%
%:%63=13%:%
%:%64=13%:%
%:%65=14%:%
%:%72=15%:%
%:%73=15%:%
%:%74=16%:%
%:%75=16%:%
%:%76=17%:%
%:%77=17%:%
%:%78=18%:%
%:%79=18%:%
%:%80=19%:%
%:%81=19%:%
%:%82=19%:%
%:%83=20%:%
%:%84=20%:%
%:%85=21%:%
%:%86=21%:%
%:%87=22%:%
%:%88=22%:%
%:%89=23%:%
%:%90=23%:%
%:%91=24%:%
%:%92=24%:%
%:%93=25%:%
%:%94=25%:%
%:%95=26%:%
%:%96=26%:%
%:%97=27%:%
%:%98=27%:%
%:%99=28%:%
%:%100=28%:%
%:%101=29%:%
%:%102=29%:%
%:%103=29%:%
%:%104=29%:%
%:%105=30%:%
%:%106=30%:%
%:%107=31%:%
%:%108=31%:%
%:%109=32%:%
%:%110=32%:%
%:%111=32%:%
%:%112=32%:%
%:%113=33%:%
%:%119=33%:%
%:%122=34%:%
%:%123=35%:%
%:%124=35%:%
%:%125=36%:%
%:%126=37%:%
%:%129=38%:%
%:%133=38%:%
%:%134=38%:%
%:%135=39%:%
%:%136=39%:%
%:%141=39%:%
%:%144=40%:%
%:%145=41%:%
%:%146=41%:%
%:%147=42%:%
%:%148=43%:%
%:%155=44%:%
%:%156=44%:%
%:%157=45%:%
%:%158=45%:%
%:%159=46%:%
%:%160=46%:%
%:%161=47%:%
%:%162=47%:%
%:%163=47%:%
%:%164=48%:%
%:%165=48%:%
%:%166=49%:%
%:%167=49%:%
%:%168=50%:%
%:%169=50%:%
%:%170=51%:%
%:%171=51%:%
%:%172=51%:%
%:%173=51%:%
%:%174=52%:%
%:%175=52%:%
%:%176=52%:%
%:%177=52%:%
%:%178=53%:%
%:%179=53%:%
%:%180=53%:%
%:%181=53%:%
%:%182=54%:%
%:%183=55%:%
%:%184=55%:%
%:%185=56%:%
%:%186=56%:%
%:%187=57%:%
%:%188=57%:%
%:%189=57%:%
%:%190=58%:%
%:%191=58%:%
%:%192=59%:%
%:%193=59%:%
%:%194=60%:%
%:%195=60%:%
%:%196=61%:%
%:%197=61%:%
%:%198=61%:%
%:%199=61%:%
%:%200=62%:%
%:%201=62%:%
%:%202=62%:%
%:%203=62%:%
%:%204=63%:%
%:%205=63%:%
%:%206=63%:%
%:%207=63%:%
%:%208=64%:%
%:%209=65%:%
%:%210=65%:%
%:%211=66%:%
%:%212=66%:%
%:%213=66%:%
%:%214=67%:%
%:%220=67%:%
%:%223=68%:%
%:%224=69%:%
%:%225=69%:%
%:%226=70%:%
%:%227=71%:%
%:%228=72%:%
%:%229=73%:%
%:%230=73%:%
%:%233=74%:%
%:%237=74%:%
%:%238=74%:%
%:%239=75%:%
%:%240=75%:%
%:%241=76%:%
%:%242=76%:%
%:%243=77%:%
%:%244=77%:%
%:%245=78%:%
%:%246=78%:%
%:%255=80%:%
%:%257=82%:%
%:%258=82%:%
%:%259=83%:%
%:%260=84%:%
%:%261=84%:%
%:%262=85%:%
%:%269=86%:%
%:%270=86%:%
%:%271=87%:%
%:%272=87%:%
%:%273=88%:%
%:%274=88%:%
%:%275=89%:%
%:%276=89%:%
%:%277=90%:%
%:%278=90%:%
%:%279=91%:%
%:%280=91%:%
%:%281=92%:%
%:%282=92%:%
%:%283=93%:%
%:%284=93%:%
%:%285=94%:%
%:%286=94%:%
%:%287=95%:%
%:%288=95%:%
%:%289=96%:%
%:%290=96%:%
%:%291=97%:%
%:%292=97%:%
%:%293=98%:%
%:%294=98%:%
%:%295=99%:%
%:%296=99%:%
%:%297=100%:%
%:%298=100%:%
%:%299=100%:%
%:%300=101%:%
%:%301=101%:%
%:%302=102%:%
%:%303=102%:%
%:%304=102%:%
%:%305=103%:%
%:%306=103%:%
%:%307=104%:%
%:%313=104%:%
%:%316=105%:%
%:%317=106%:%
%:%318=106%:%
%:%319=107%:%
%:%326=108%:%
%:%327=108%:%
%:%328=109%:%
%:%329=109%:%
%:%330=110%:%
%:%331=111%:%
%:%332=111%:%
%:%333=112%:%
%:%334=112%:%
%:%335=113%:%
%:%336=113%:%
%:%337=114%:%
%:%338=115%:%
%:%339=115%:%
%:%340=116%:%
%:%341=116%:%
%:%342=117%:%
%:%343=117%:%
%:%344=118%:%
%:%345=118%:%
%:%346=118%:%
%:%347=119%:%
%:%348=119%:%
%:%349=119%:%
%:%350=120%:%
%:%351=120%:%
%:%352=121%:%
%:%353=121%:%
%:%354=121%:%
%:%355=122%:%
%:%356=122%:%
%:%357=122%:%
%:%358=123%:%
%:%359=123%:%
%:%360=124%:%
%:%361=124%:%
%:%362=124%:%
%:%363=125%:%
%:%364=125%:%
%:%365=126%:%
%:%366=126%:%
%:%367=127%:%
%:%368=127%:%
%:%369=127%:%
%:%370=128%:%
%:%371=128%:%
%:%372=129%:%
%:%373=129%:%
%:%374=130%:%
%:%375=130%:%
%:%376=130%:%
%:%377=131%:%
%:%378=131%:%
%:%379=132%:%
%:%380=132%:%
%:%381=132%:%
%:%382=133%:%
%:%383=133%:%
%:%384=134%:%
%:%385=134%:%
%:%386=134%:%
%:%387=135%:%
%:%388=135%:%
%:%389=136%:%
%:%390=136%:%
%:%391=137%:%
%:%392=138%:%
%:%393=138%:%
%:%394=138%:%
%:%395=139%:%
%:%396=139%:%
%:%397=140%:%
%:%398=140%:%
%:%399=141%:%
%:%400=141%:%
%:%401=142%:%
%:%402=142%:%
%:%403=143%:%
%:%404=143%:%
%:%405=144%:%
%:%406=144%:%
%:%407=144%:%
%:%408=145%:%
%:%409=145%:%
%:%410=146%:%
%:%411=146%:%
%:%412=147%:%
%:%413=147%:%
%:%414=148%:%
%:%415=148%:%
%:%416=149%:%
%:%417=149%:%
%:%418=149%:%
%:%419=150%:%
%:%420=150%:%
%:%421=151%:%
%:%422=151%:%
%:%423=151%:%
%:%424=152%:%
%:%425=152%:%
%:%426=153%:%
%:%427=153%:%
%:%428=154%:%
%:%429=154%:%
%:%430=155%:%
%:%431=155%:%
%:%432=155%:%
%:%433=156%:%
%:%434=156%:%
%:%435=157%:%
%:%436=157%:%
%:%437=157%:%
%:%438=158%:%
%:%439=158%:%
%:%440=159%:%
%:%441=159%:%
%:%442=160%:%
%:%443=160%:%
%:%444=161%:%
%:%445=161%:%
%:%446=162%:%
%:%447=162%:%
%:%448=162%:%
%:%449=162%:%
%:%450=162%:%
%:%451=163%:%
%:%452=163%:%
%:%453=163%:%
%:%454=164%:%
%:%455=164%:%
%:%456=164%:%
%:%457=165%:%
%:%458=165%:%
%:%459=166%:%
%:%460=167%:%
%:%461=167%:%
%:%462=167%:%
%:%463=168%:%
%:%464=168%:%
%:%465=169%:%
%:%466=169%:%
%:%467=169%:%
%:%468=170%:%
%:%469=170%:%
%:%470=171%:%
%:%471=171%:%
%:%472=171%:%
%:%473=171%:%
%:%474=172%:%
%:%480=172%:%
%:%483=173%:%
%:%484=174%:%
%:%485=174%:%
%:%486=175%:%
%:%487=176%:%
%:%490=177%:%
%:%494=177%:%
%:%495=177%:%
%:%500=177%:%
%:%503=178%:%
%:%504=179%:%
%:%505=179%:%
%:%506=180%:%
%:%507=181%:%
%:%508=182%:%
%:%509=183%:%
%:%516=184%:%
%:%517=184%:%
%:%518=185%:%
%:%519=185%:%
%:%520=186%:%
%:%521=186%:%
%:%522=187%:%
%:%523=187%:%
%:%524=188%:%
%:%525=188%:%
%:%526=188%:%
%:%527=188%:%
%:%528=188%:%
%:%529=189%:%
%:%530=189%:%
%:%531=190%:%
%:%532=190%:%
%:%533=190%:%
%:%534=191%:%
%:%535=191%:%
%:%536=191%:%
%:%537=192%:%
%:%538=193%:%
%:%539=193%:%
%:%540=194%:%
%:%541=194%:%
%:%542=195%:%
%:%543=196%:%
%:%544=196%:%
%:%545=197%:%
%:%546=197%:%
%:%547=198%:%
%:%548=198%:%
%:%549=198%:%
%:%550=199%:%
%:%551=199%:%
%:%552=200%:%
%:%553=200%:%
%:%554=201%:%
%:%555=202%:%
%:%556=202%:%
%:%557=203%:%
%:%558=203%:%
%:%559=204%:%
%:%560=204%:%
%:%561=205%:%
%:%562=205%:%
%:%563=206%:%
%:%564=206%:%
%:%565=207%:%
%:%566=207%:%
%:%567=207%:%
%:%568=208%:%
%:%569=208%:%
%:%570=209%:%
%:%571=209%:%
%:%572=210%:%
%:%573=210%:%
%:%574=211%:%
%:%575=211%:%
%:%576=212%:%
%:%577=212%:%
%:%578=213%:%
%:%579=213%:%
%:%580=213%:%
%:%581=214%:%
%:%582=214%:%
%:%583=215%:%
%:%584=215%:%
%:%585=215%:%
%:%586=216%:%
%:%587=216%:%
%:%588=217%:%
%:%589=217%:%
%:%590=217%:%
%:%591=218%:%
%:%592=218%:%
%:%593=219%:%
%:%594=219%:%
%:%595=220%:%
%:%596=220%:%
%:%597=220%:%
%:%598=221%:%
%:%599=221%:%
%:%600=222%:%
%:%601=222%:%
%:%602=222%:%
%:%603=222%:%
%:%604=223%:%
%:%605=223%:%
%:%606=224%:%
%:%607=224%:%
%:%608=225%:%
%:%609=225%:%
%:%610=225%:%
%:%611=226%:%
%:%612=226%:%
%:%613=226%:%
%:%614=226%:%
%:%615=227%:%
%:%616=227%:%
%:%617=228%:%
%:%618=229%:%
%:%619=229%:%
%:%620=230%:%
%:%621=230%:%
%:%622=231%:%
%:%623=231%:%
%:%624=231%:%
%:%626=233%:%
%:%627=234%:%
%:%628=234%:%
%:%629=235%:%
%:%630=235%:%
%:%631=235%:%
%:%632=236%:%
%:%633=237%:%
%:%634=237%:%
%:%635=238%:%
%:%636=238%:%
%:%637=238%:%
%:%639=240%:%
%:%640=241%:%
%:%641=241%:%
%:%642=242%:%
%:%643=242%:%
%:%644=243%:%
%:%645=243%:%
%:%646=244%:%
%:%647=244%:%
%:%648=245%:%
%:%649=245%:%
%:%650=246%:%
%:%651=246%:%
%:%652=247%:%
%:%653=247%:%
%:%654=248%:%
%:%655=248%:%
%:%656=248%:%
%:%657=249%:%
%:%658=250%:%
%:%659=250%:%
%:%660=251%:%
%:%661=251%:%
%:%662=252%:%
%:%663=252%:%
%:%664=253%:%
%:%665=253%:%
%:%666=254%:%
%:%667=254%:%
%:%668=255%:%
%:%669=255%:%
%:%670=256%:%
%:%671=256%:%
%:%672=256%:%
%:%673=257%:%
%:%674=257%:%
%:%675=258%:%
%:%676=258%:%
%:%677=258%:%
%:%678=259%:%
%:%679=259%:%
%:%680=260%:%
%:%681=260%:%
%:%682=260%:%
%:%683=260%:%
%:%684=261%:%
%:%685=261%:%
%:%686=262%:%
%:%687=262%:%
%:%688=263%:%
%:%689=263%:%
%:%690=264%:%
%:%691=264%:%
%:%692=265%:%
%:%693=265%:%
%:%694=265%:%
%:%695=266%:%
%:%696=266%:%
%:%697=266%:%
%:%698=266%:%
%:%699=267%:%
%:%705=267%:%
%:%708=268%:%
%:%709=269%:%
%:%710=269%:%
%:%711=270%:%
%:%714=271%:%
%:%718=271%:%
%:%719=271%:%
%:%720=272%:%
%:%721=272%:%
%:%726=272%:%
%:%729=273%:%
%:%730=274%:%
%:%731=274%:%
%:%738=275%:%
%:%739=275%:%
%:%740=276%:%
%:%741=276%:%
%:%742=277%:%
%:%743=277%:%
%:%744=277%:%
%:%745=278%:%
%:%746=278%:%
%:%747=279%:%
%:%748=279%:%
%:%749=280%:%
%:%750=280%:%
%:%751=281%:%
%:%757=281%:%
%:%760=282%:%
%:%761=283%:%
%:%762=283%:%
%:%763=284%:%
%:%764=285%:%
%:%771=286%:%
%:%772=286%:%
%:%773=287%:%
%:%774=287%:%
%:%775=288%:%
%:%776=288%:%
%:%777=289%:%
%:%778=290%:%
%:%779=290%:%
%:%780=290%:%
%:%781=291%:%
%:%782=291%:%
%:%783=292%:%
%:%784=292%:%
%:%785=293%:%
%:%786=293%:%
%:%787=293%:%
%:%788=294%:%
%:%789=294%:%
%:%790=295%:%
%:%791=295%:%
%:%792=295%:%
%:%793=296%:%
%:%794=296%:%
%:%795=297%:%
%:%796=297%:%
%:%797=297%:%
%:%798=297%:%
%:%799=298%:%
%:%800=298%:%
%:%801=299%:%
%:%802=299%:%
%:%803=300%:%
%:%804=300%:%
%:%805=301%:%
%:%806=301%:%
%:%807=302%:%
%:%808=303%:%
%:%809=303%:%
%:%810=304%:%
%:%811=304%:%
%:%812=305%:%
%:%813=305%:%
%:%814=305%:%
%:%815=306%:%
%:%816=306%:%
%:%817=307%:%
%:%818=307%:%
%:%819=308%:%
%:%820=308%:%
%:%821=309%:%
%:%822=309%:%
%:%823=309%:%
%:%824=310%:%
%:%825=310%:%
%:%826=311%:%
%:%827=311%:%
%:%828=311%:%
%:%829=311%:%
%:%830=312%:%
%:%831=313%:%
%:%832=313%:%
%:%833=314%:%
%:%834=314%:%
%:%835=315%:%
%:%836=315%:%
%:%837=315%:%
%:%838=316%:%
%:%839=316%:%
%:%840=316%:%
%:%841=317%:%
%:%842=317%:%
%:%843=318%:%
%:%844=318%:%
%:%845=318%:%
%:%846=319%:%
%:%847=319%:%
%:%848=320%:%
%:%849=320%:%
%:%850=320%:%
%:%851=320%:%
%:%852=321%:%
%:%853=322%:%
%:%854=322%:%
%:%855=323%:%
%:%856=323%:%
%:%857=324%:%
%:%858=324%:%
%:%859=325%:%
%:%860=325%:%
%:%861=325%:%
%:%862=326%:%
%:%863=326%:%
%:%864=326%:%
%:%865=327%:%
%:%866=327%:%
%:%867=328%:%
%:%868=328%:%
%:%869=329%:%
%:%870=329%:%
%:%871=330%:%
%:%872=330%:%
%:%873=331%:%
%:%874=331%:%
%:%875=331%:%
%:%876=331%:%
%:%877=332%:%
%:%878=332%:%
%:%879=332%:%
%:%880=332%:%
%:%881=333%:%
%:%887=333%:%
%:%892=334%:%
%:%897=335%:%

%
\begin{isabellebody}%
\setisabellecontext{List{\isacharunderscore}{\kern0pt}Ext}%
%
\isadelimdocument
%
\endisadelimdocument
%
\isatagdocument
%
\isamarkupsection{Lists%
}
\isamarkuptrue%
%
\endisatagdocument
{\isafolddocument}%
%
\isadelimdocument
%
\endisadelimdocument
%
\isadelimtheory
%
\endisadelimtheory
%
\isatagtheory
\isacommand{theory}\isamarkupfalse%
\ List{\isacharunderscore}{\kern0pt}Ext\isanewline
\ \ \isakeyword{imports}\ Main\ {\isachardoublequoteopen}HOL{\isachardot}{\kern0pt}List{\isachardoublequoteclose}\isanewline
\isakeyword{begin}%
\endisatagtheory
{\isafoldtheory}%
%
\isadelimtheory
%
\endisadelimtheory
%
\begin{isamarkuptext}%
This section contains results about lists in addition to "HOL.List"%
\end{isamarkuptext}\isamarkuptrue%
\isacommand{lemma}\isamarkupfalse%
\ count{\isacharunderscore}{\kern0pt}list{\isacharunderscore}{\kern0pt}gr{\isacharunderscore}{\kern0pt}{\isadigit{1}}{\isacharcolon}{\kern0pt}\isanewline
\ \ {\isachardoublequoteopen}{\isacharparenleft}{\kern0pt}x\ {\isasymin}\ set\ xs{\isacharparenright}{\kern0pt}\ {\isacharequal}{\kern0pt}\ {\isacharparenleft}{\kern0pt}count{\isacharunderscore}{\kern0pt}list\ xs\ x\ {\isasymge}\ {\isadigit{1}}{\isacharparenright}{\kern0pt}{\isachardoublequoteclose}\isanewline
%
\isadelimproof
\ \ %
\endisadelimproof
%
\isatagproof
\isacommand{by}\isamarkupfalse%
\ {\isacharparenleft}{\kern0pt}induction\ xs{\isacharcomma}{\kern0pt}\ simp{\isacharcomma}{\kern0pt}\ simp{\isacharparenright}{\kern0pt}%
\endisatagproof
{\isafoldproof}%
%
\isadelimproof
\isanewline
%
\endisadelimproof
\isanewline
\isacommand{lemma}\isamarkupfalse%
\ count{\isacharunderscore}{\kern0pt}list{\isacharunderscore}{\kern0pt}append{\isacharcolon}{\kern0pt}\ {\isachardoublequoteopen}count{\isacharunderscore}{\kern0pt}list\ {\isacharparenleft}{\kern0pt}xs{\isacharat}{\kern0pt}ys{\isacharparenright}{\kern0pt}\ v\ {\isacharequal}{\kern0pt}\ count{\isacharunderscore}{\kern0pt}list\ xs\ v\ {\isacharplus}{\kern0pt}\ count{\isacharunderscore}{\kern0pt}list\ ys\ v{\isachardoublequoteclose}\isanewline
%
\isadelimproof
\ \ %
\endisadelimproof
%
\isatagproof
\isacommand{by}\isamarkupfalse%
\ {\isacharparenleft}{\kern0pt}induction\ xs{\isacharcomma}{\kern0pt}\ simp{\isacharcomma}{\kern0pt}\ simp{\isacharparenright}{\kern0pt}%
\endisatagproof
{\isafoldproof}%
%
\isadelimproof
\isanewline
%
\endisadelimproof
\isanewline
\isacommand{lemma}\isamarkupfalse%
\ count{\isacharunderscore}{\kern0pt}list{\isacharunderscore}{\kern0pt}card{\isacharcolon}{\kern0pt}\ {\isachardoublequoteopen}count{\isacharunderscore}{\kern0pt}list\ xs\ x\ {\isacharequal}{\kern0pt}\ card\ {\isacharbraceleft}{\kern0pt}k{\isachardot}{\kern0pt}\ k\ {\isacharless}{\kern0pt}\ length\ xs\ {\isasymand}\ xs\ {\isacharbang}{\kern0pt}\ k\ {\isacharequal}{\kern0pt}\ x{\isacharbraceright}{\kern0pt}{\isachardoublequoteclose}\isanewline
%
\isadelimproof
%
\endisadelimproof
%
\isatagproof
\isacommand{proof}\isamarkupfalse%
\ {\isacharminus}{\kern0pt}\isanewline
\ \ \isacommand{have}\isamarkupfalse%
\ {\isachardoublequoteopen}count{\isacharunderscore}{\kern0pt}list\ xs\ x\ {\isacharequal}{\kern0pt}\ length\ {\isacharparenleft}{\kern0pt}filter\ {\isacharparenleft}{\kern0pt}{\isacharparenleft}{\kern0pt}{\isacharequal}{\kern0pt}{\isacharparenright}{\kern0pt}\ x{\isacharparenright}{\kern0pt}\ xs{\isacharparenright}{\kern0pt}{\isachardoublequoteclose}\isanewline
\ \ \ \ \isacommand{by}\isamarkupfalse%
\ {\isacharparenleft}{\kern0pt}induction\ xs{\isacharcomma}{\kern0pt}\ simp{\isacharcomma}{\kern0pt}\ simp{\isacharparenright}{\kern0pt}\isanewline
\ \ \isacommand{also}\isamarkupfalse%
\ \isacommand{have}\isamarkupfalse%
\ {\isachardoublequoteopen}{\isachardot}{\kern0pt}{\isachardot}{\kern0pt}{\isachardot}{\kern0pt}\ {\isacharequal}{\kern0pt}\ card\ {\isacharbraceleft}{\kern0pt}k{\isachardot}{\kern0pt}\ k\ {\isacharless}{\kern0pt}\ length\ xs\ {\isasymand}\ xs\ {\isacharbang}{\kern0pt}\ k\ {\isacharequal}{\kern0pt}\ x{\isacharbraceright}{\kern0pt}{\isachardoublequoteclose}\isanewline
\ \ \ \ \isacommand{apply}\isamarkupfalse%
\ {\isacharparenleft}{\kern0pt}subst\ length{\isacharunderscore}{\kern0pt}filter{\isacharunderscore}{\kern0pt}conv{\isacharunderscore}{\kern0pt}card{\isacharparenright}{\kern0pt}\isanewline
\ \ \ \ \isacommand{by}\isamarkupfalse%
\ metis\isanewline
\ \ \isacommand{finally}\isamarkupfalse%
\ \isacommand{show}\isamarkupfalse%
\ {\isacharquery}{\kern0pt}thesis\ \isacommand{by}\isamarkupfalse%
\ simp\isanewline
\isacommand{qed}\isamarkupfalse%
%
\endisatagproof
{\isafoldproof}%
%
\isadelimproof
\isanewline
%
\endisadelimproof
\isanewline
\isacommand{lemma}\isamarkupfalse%
\ card{\isacharunderscore}{\kern0pt}gr{\isacharunderscore}{\kern0pt}{\isadigit{1}}{\isacharunderscore}{\kern0pt}iff{\isacharcolon}{\kern0pt}\isanewline
\ \ \isakeyword{assumes}\ {\isachardoublequoteopen}finite\ S{\isachardoublequoteclose}\isanewline
\ \ \isakeyword{assumes}\ {\isachardoublequoteopen}x\ {\isasymin}\ S{\isachardoublequoteclose}\isanewline
\ \ \isakeyword{assumes}\ {\isachardoublequoteopen}y\ {\isasymin}\ S{\isachardoublequoteclose}\isanewline
\ \ \isakeyword{assumes}\ {\isachardoublequoteopen}x\ {\isasymnoteq}\ y{\isachardoublequoteclose}\isanewline
\ \ \isakeyword{shows}\ {\isachardoublequoteopen}card\ S\ {\isachargreater}{\kern0pt}\ {\isadigit{1}}{\isachardoublequoteclose}\isanewline
%
\isadelimproof
\ \ %
\endisadelimproof
%
\isatagproof
\isacommand{using}\isamarkupfalse%
\ assms\ card{\isacharunderscore}{\kern0pt}le{\isacharunderscore}{\kern0pt}Suc{\isadigit{0}}{\isacharunderscore}{\kern0pt}iff{\isacharunderscore}{\kern0pt}eq\ leI\ \isacommand{by}\isamarkupfalse%
\ auto%
\endisatagproof
{\isafoldproof}%
%
\isadelimproof
\isanewline
%
\endisadelimproof
\isanewline
\isacommand{lemma}\isamarkupfalse%
\ count{\isacharunderscore}{\kern0pt}list{\isacharunderscore}{\kern0pt}ge{\isacharunderscore}{\kern0pt}{\isadigit{2}}{\isacharunderscore}{\kern0pt}iff{\isacharcolon}{\kern0pt}\isanewline
\ \ \isakeyword{assumes}\ {\isachardoublequoteopen}y\ {\isacharless}{\kern0pt}\ z{\isachardoublequoteclose}\isanewline
\ \ \isakeyword{assumes}\ {\isachardoublequoteopen}z\ {\isacharless}{\kern0pt}\ length\ xs{\isachardoublequoteclose}\isanewline
\ \ \isakeyword{assumes}\ {\isachardoublequoteopen}xs\ {\isacharbang}{\kern0pt}\ y\ {\isacharequal}{\kern0pt}\ xs\ {\isacharbang}{\kern0pt}\ z{\isachardoublequoteclose}\isanewline
\ \ \isakeyword{shows}\ {\isachardoublequoteopen}count{\isacharunderscore}{\kern0pt}list\ xs\ {\isacharparenleft}{\kern0pt}xs\ {\isacharbang}{\kern0pt}\ y{\isacharparenright}{\kern0pt}\ {\isachargreater}{\kern0pt}\ {\isadigit{1}}{\isachardoublequoteclose}\isanewline
%
\isadelimproof
\ \ %
\endisadelimproof
%
\isatagproof
\isacommand{apply}\isamarkupfalse%
\ {\isacharparenleft}{\kern0pt}subst\ count{\isacharunderscore}{\kern0pt}list{\isacharunderscore}{\kern0pt}card{\isacharparenright}{\kern0pt}\isanewline
\ \ \isacommand{apply}\isamarkupfalse%
\ {\isacharparenleft}{\kern0pt}rule\ card{\isacharunderscore}{\kern0pt}gr{\isacharunderscore}{\kern0pt}{\isadigit{1}}{\isacharunderscore}{\kern0pt}iff{\isacharbrackleft}{\kern0pt}\isakeyword{where}\ x{\isacharequal}{\kern0pt}{\isachardoublequoteopen}y{\isachardoublequoteclose}\ \isakeyword{and}\ y{\isacharequal}{\kern0pt}{\isachardoublequoteopen}z{\isachardoublequoteclose}{\isacharbrackright}{\kern0pt}{\isacharparenright}{\kern0pt}\isanewline
\ \ \isacommand{using}\isamarkupfalse%
\ assms\ \isacommand{by}\isamarkupfalse%
\ simp{\isacharplus}{\kern0pt}%
\endisatagproof
{\isafoldproof}%
%
\isadelimproof
\isanewline
%
\endisadelimproof
%
\isadelimtheory
\isanewline
%
\endisadelimtheory
%
\isatagtheory
\isacommand{end}\isamarkupfalse%
%
\endisatagtheory
{\isafoldtheory}%
%
\isadelimtheory
%
\endisadelimtheory
%
\end{isabellebody}%
\endinput
%:%file=List_Ext.tex%:%
%:%11=1%:%
%:%27=3%:%
%:%28=3%:%
%:%29=4%:%
%:%30=5%:%
%:%39=7%:%
%:%41=9%:%
%:%42=9%:%
%:%43=10%:%
%:%46=11%:%
%:%50=11%:%
%:%51=11%:%
%:%56=11%:%
%:%59=12%:%
%:%60=13%:%
%:%61=13%:%
%:%64=14%:%
%:%68=14%:%
%:%69=14%:%
%:%74=14%:%
%:%77=15%:%
%:%78=16%:%
%:%79=16%:%
%:%86=17%:%
%:%87=17%:%
%:%88=18%:%
%:%89=18%:%
%:%90=19%:%
%:%91=19%:%
%:%92=20%:%
%:%93=20%:%
%:%94=20%:%
%:%95=21%:%
%:%96=21%:%
%:%97=22%:%
%:%98=22%:%
%:%99=23%:%
%:%100=23%:%
%:%101=23%:%
%:%102=23%:%
%:%103=24%:%
%:%109=24%:%
%:%112=25%:%
%:%113=26%:%
%:%114=26%:%
%:%115=27%:%
%:%116=28%:%
%:%117=29%:%
%:%118=30%:%
%:%119=31%:%
%:%122=32%:%
%:%126=32%:%
%:%127=32%:%
%:%128=32%:%
%:%133=32%:%
%:%136=33%:%
%:%137=34%:%
%:%138=34%:%
%:%139=35%:%
%:%140=36%:%
%:%141=37%:%
%:%142=38%:%
%:%145=39%:%
%:%149=39%:%
%:%150=39%:%
%:%151=40%:%
%:%152=40%:%
%:%153=41%:%
%:%154=41%:%
%:%155=41%:%
%:%160=41%:%
%:%165=42%:%
%:%170=43%:%

%
\begin{isabellebody}%
\setisabellecontext{Frequency{\isacharunderscore}{\kern0pt}Moments}%
%
\isadelimdocument
%
\endisadelimdocument
%
\isatagdocument
%
\isamarkupsection{Frequency Moments%
}
\isamarkuptrue%
%
\endisatagdocument
{\isafolddocument}%
%
\isadelimdocument
%
\endisadelimdocument
%
\isadelimtheory
%
\endisadelimtheory
%
\isatagtheory
\isacommand{theory}\isamarkupfalse%
\ Frequency{\isacharunderscore}{\kern0pt}Moments\isanewline
\ \ \isakeyword{imports}\ Main\ HOL{\isachardot}{\kern0pt}List\ HOL{\isachardot}{\kern0pt}Rat\ List{\isacharunderscore}{\kern0pt}Ext\isanewline
\isakeyword{begin}%
\endisatagtheory
{\isafoldtheory}%
%
\isadelimtheory
%
\endisadelimtheory
%
\begin{isamarkuptext}%
This section contains a definition of the frequency moments of a stream.%
\end{isamarkuptext}\isamarkuptrue%
\isacommand{definition}\isamarkupfalse%
\ F\ \isakeyword{where}\isanewline
\ \ {\isachardoublequoteopen}F\ k\ xs\ {\isacharequal}{\kern0pt}\ {\isacharparenleft}{\kern0pt}{\isasymSum}\ x\ {\isasymin}\ set\ xs{\isachardot}{\kern0pt}\ {\isacharparenleft}{\kern0pt}rat{\isacharunderscore}{\kern0pt}of{\isacharunderscore}{\kern0pt}nat\ {\isacharparenleft}{\kern0pt}count{\isacharunderscore}{\kern0pt}list\ xs\ x{\isacharparenright}{\kern0pt}{\isacharcircum}{\kern0pt}k{\isacharparenright}{\kern0pt}{\isacharparenright}{\kern0pt}{\isachardoublequoteclose}\isanewline
\isanewline
\isacommand{lemma}\isamarkupfalse%
\ F{\isacharunderscore}{\kern0pt}gr{\isacharunderscore}{\kern0pt}{\isadigit{0}}{\isacharcolon}{\kern0pt}\isanewline
\ \ \isakeyword{assumes}\ {\isachardoublequoteopen}as\ {\isasymnoteq}\ {\isacharbrackleft}{\kern0pt}{\isacharbrackright}{\kern0pt}{\isachardoublequoteclose}\isanewline
\ \ \isakeyword{shows}\ {\isachardoublequoteopen}F\ k\ as\ {\isachargreater}{\kern0pt}\ {\isadigit{0}}{\isachardoublequoteclose}\isanewline
%
\isadelimproof
%
\endisadelimproof
%
\isatagproof
\isacommand{proof}\isamarkupfalse%
\ {\isacharminus}{\kern0pt}\isanewline
\ \ \isacommand{have}\isamarkupfalse%
\ {\isachardoublequoteopen}rat{\isacharunderscore}{\kern0pt}of{\isacharunderscore}{\kern0pt}nat\ {\isadigit{1}}\ {\isasymle}\ rat{\isacharunderscore}{\kern0pt}of{\isacharunderscore}{\kern0pt}nat\ {\isacharparenleft}{\kern0pt}card\ {\isacharparenleft}{\kern0pt}set\ as{\isacharparenright}{\kern0pt}{\isacharparenright}{\kern0pt}{\isachardoublequoteclose}\isanewline
\ \ \ \ \isacommand{apply}\isamarkupfalse%
\ {\isacharparenleft}{\kern0pt}rule\ of{\isacharunderscore}{\kern0pt}nat{\isacharunderscore}{\kern0pt}mono{\isacharparenright}{\kern0pt}\isanewline
\ \ \ \ \isacommand{using}\isamarkupfalse%
\ assms\ card{\isacharunderscore}{\kern0pt}{\isadigit{0}}{\isacharunderscore}{\kern0pt}eq{\isacharbrackleft}{\kern0pt}\isakeyword{where}\ A{\isacharequal}{\kern0pt}{\isachardoublequoteopen}set\ as{\isachardoublequoteclose}{\isacharbrackright}{\kern0pt}\ \isanewline
\ \ \ \ \isacommand{by}\isamarkupfalse%
\ {\isacharparenleft}{\kern0pt}metis\ List{\isachardot}{\kern0pt}finite{\isacharunderscore}{\kern0pt}set\ One{\isacharunderscore}{\kern0pt}nat{\isacharunderscore}{\kern0pt}def\ Suc{\isacharunderscore}{\kern0pt}leI\ neq{\isadigit{0}}{\isacharunderscore}{\kern0pt}conv\ set{\isacharunderscore}{\kern0pt}empty{\isacharparenright}{\kern0pt}\isanewline
\ \ \isacommand{also}\isamarkupfalse%
\ \isacommand{have}\isamarkupfalse%
\ {\isachardoublequoteopen}{\isachardot}{\kern0pt}{\isachardot}{\kern0pt}{\isachardot}{\kern0pt}\ {\isasymle}\ F\ k\ as{\isachardoublequoteclose}\isanewline
\ \ \ \ \isacommand{apply}\isamarkupfalse%
\ {\isacharparenleft}{\kern0pt}simp\ add{\isacharcolon}{\kern0pt}F{\isacharunderscore}{\kern0pt}def{\isacharparenright}{\kern0pt}\isanewline
\ \ \ \ \isacommand{apply}\isamarkupfalse%
\ {\isacharparenleft}{\kern0pt}rule\ sum{\isacharunderscore}{\kern0pt}mono{\isacharbrackleft}{\kern0pt}\isakeyword{where}\ K{\isacharequal}{\kern0pt}{\isachardoublequoteopen}set\ as{\isachardoublequoteclose}\ \isakeyword{and}\ f{\isacharequal}{\kern0pt}{\isachardoublequoteopen}{\isasymlambda}{\isacharunderscore}{\kern0pt}{\isachardot}{\kern0pt}{\isacharparenleft}{\kern0pt}{\isadigit{1}}{\isacharcolon}{\kern0pt}{\isacharcolon}{\kern0pt}rat{\isacharparenright}{\kern0pt}{\isachardoublequoteclose}{\isacharcomma}{\kern0pt}\ simplified{\isacharbrackright}{\kern0pt}{\isacharparenright}{\kern0pt}\isanewline
\ \ \ \ \isacommand{by}\isamarkupfalse%
\ {\isacharparenleft}{\kern0pt}metis\ \ count{\isacharunderscore}{\kern0pt}list{\isacharunderscore}{\kern0pt}gr{\isacharunderscore}{\kern0pt}{\isadigit{1}}\ \ of{\isacharunderscore}{\kern0pt}nat{\isacharunderscore}{\kern0pt}{\isadigit{1}}\ of{\isacharunderscore}{\kern0pt}nat{\isacharunderscore}{\kern0pt}power{\isacharunderscore}{\kern0pt}le{\isacharunderscore}{\kern0pt}of{\isacharunderscore}{\kern0pt}nat{\isacharunderscore}{\kern0pt}cancel{\isacharunderscore}{\kern0pt}iff\ one{\isacharunderscore}{\kern0pt}le{\isacharunderscore}{\kern0pt}power{\isacharparenright}{\kern0pt}\isanewline
\ \ \isacommand{finally}\isamarkupfalse%
\ \isacommand{show}\isamarkupfalse%
\ \ {\isachardoublequoteopen}F\ k\ as\ {\isachargreater}{\kern0pt}\ {\isadigit{0}}{\isachardoublequoteclose}\ \isacommand{by}\isamarkupfalse%
\ simp\isanewline
\isacommand{qed}\isamarkupfalse%
%
\endisatagproof
{\isafoldproof}%
%
\isadelimproof
\isanewline
%
\endisadelimproof
%
\isadelimtheory
\isanewline
%
\endisadelimtheory
%
\isatagtheory
\isacommand{end}\isamarkupfalse%
%
\endisatagtheory
{\isafoldtheory}%
%
\isadelimtheory
%
\endisadelimtheory
%
\end{isabellebody}%
\endinput
%:%file=Frequency_Moments.tex%:%
%:%11=1%:%
%:%27=3%:%
%:%28=3%:%
%:%29=4%:%
%:%30=5%:%
%:%39=7%:%
%:%41=9%:%
%:%42=9%:%
%:%43=10%:%
%:%44=11%:%
%:%45=12%:%
%:%46=12%:%
%:%47=13%:%
%:%48=14%:%
%:%55=15%:%
%:%56=15%:%
%:%57=16%:%
%:%58=16%:%
%:%59=17%:%
%:%60=17%:%
%:%61=18%:%
%:%62=18%:%
%:%63=19%:%
%:%64=19%:%
%:%65=20%:%
%:%66=20%:%
%:%67=20%:%
%:%68=21%:%
%:%69=21%:%
%:%70=22%:%
%:%71=22%:%
%:%72=23%:%
%:%73=23%:%
%:%74=24%:%
%:%75=24%:%
%:%76=24%:%
%:%77=24%:%
%:%78=25%:%
%:%84=25%:%
%:%89=26%:%
%:%94=27%:%

%
\begin{isabellebody}%
\setisabellecontext{Primes{\isacharunderscore}{\kern0pt}Ext}%
%
\isadelimdocument
%
\endisadelimdocument
%
\isatagdocument
%
\isamarkupsection{Primes%
}
\isamarkuptrue%
%
\endisatagdocument
{\isafolddocument}%
%
\isadelimdocument
%
\endisadelimdocument
%
\begin{isamarkuptext}%
This section introduces a function that finds the smallest primes above a given threshold.%
\end{isamarkuptext}\isamarkuptrue%
%
\isadelimtheory
%
\endisadelimtheory
%
\isatagtheory
\isacommand{theory}\isamarkupfalse%
\ Primes{\isacharunderscore}{\kern0pt}Ext\isanewline
\isakeyword{imports}\ Main\ {\isachardoublequoteopen}HOL{\isacharminus}{\kern0pt}Computational{\isacharunderscore}{\kern0pt}Algebra{\isachardot}{\kern0pt}Primes{\isachardoublequoteclose}\ {\isachardoublequoteopen}Bertrands{\isacharunderscore}{\kern0pt}Postulate{\isachardot}{\kern0pt}Bertrand{\isachardoublequoteclose}\ \isanewline
\isakeyword{begin}%
\endisatagtheory
{\isafoldtheory}%
%
\isadelimtheory
%
\endisadelimtheory
\isanewline
\isanewline
\isacommand{lemma}\isamarkupfalse%
\ inf{\isacharunderscore}{\kern0pt}primes{\isacharcolon}{\kern0pt}\ {\isachardoublequoteopen}wf\ {\isacharparenleft}{\kern0pt}{\isacharparenleft}{\kern0pt}{\isasymlambda}n{\isachardot}{\kern0pt}\ {\isacharparenleft}{\kern0pt}Suc\ n{\isacharcomma}{\kern0pt}\ n{\isacharparenright}{\kern0pt}{\isacharparenright}{\kern0pt}\ {\isacharbackquote}{\kern0pt}\ {\isacharbraceleft}{\kern0pt}n{\isachardot}{\kern0pt}\ {\isasymnot}\ {\isacharparenleft}{\kern0pt}prime\ n{\isacharparenright}{\kern0pt}{\isacharbraceright}{\kern0pt}{\isacharparenright}{\kern0pt}{\isachardoublequoteclose}\ {\isacharparenleft}{\kern0pt}\isakeyword{is}\ {\isachardoublequoteopen}wf\ {\isacharquery}{\kern0pt}S{\isachardoublequoteclose}{\isacharparenright}{\kern0pt}\ \isanewline
%
\isadelimproof
%
\endisadelimproof
%
\isatagproof
\isacommand{proof}\isamarkupfalse%
\ {\isacharparenleft}{\kern0pt}rule\ wfI{\isacharunderscore}{\kern0pt}min{\isacharparenright}{\kern0pt}\isanewline
\ \ \isacommand{fix}\isamarkupfalse%
\ x\ {\isacharcolon}{\kern0pt}{\isacharcolon}{\kern0pt}\ nat\isanewline
\ \ \isacommand{fix}\isamarkupfalse%
\ Q\ {\isacharcolon}{\kern0pt}{\isacharcolon}{\kern0pt}\ {\isachardoublequoteopen}nat\ set{\isachardoublequoteclose}\isanewline
\ \ \isacommand{assume}\isamarkupfalse%
\ a{\isacharcolon}{\kern0pt}{\isachardoublequoteopen}x\ {\isasymin}\ Q{\isachardoublequoteclose}\isanewline
\isanewline
\ \ \isacommand{have}\isamarkupfalse%
\ {\isachardoublequoteopen}{\isasymexists}z\ {\isasymin}\ Q{\isachardot}{\kern0pt}\ prime\ z\ {\isasymor}\ Suc\ z\ {\isasymnotin}\ Q{\isachardoublequoteclose}\ \isanewline
\ \ \isacommand{proof}\isamarkupfalse%
\ {\isacharparenleft}{\kern0pt}cases\ {\isachardoublequoteopen}{\isasymexists}z\ {\isasymin}\ Q{\isachardot}{\kern0pt}\ Suc\ z\ {\isasymnotin}\ Q{\isachardoublequoteclose}{\isacharparenright}{\kern0pt}\isanewline
\ \ \ \ \isacommand{case}\isamarkupfalse%
\ True\isanewline
\ \ \ \ \isacommand{then}\isamarkupfalse%
\ \isacommand{show}\isamarkupfalse%
\ {\isacharquery}{\kern0pt}thesis\ \isacommand{by}\isamarkupfalse%
\ auto\isanewline
\ \ \isacommand{next}\isamarkupfalse%
\isanewline
\ \ \ \ \isacommand{case}\isamarkupfalse%
\ False\isanewline
\ \ \ \ \isacommand{hence}\isamarkupfalse%
\ b{\isacharcolon}{\kern0pt}{\isachardoublequoteopen}{\isasymAnd}z{\isachardot}{\kern0pt}\ z\ {\isasymin}\ Q\ {\isasymLongrightarrow}\ Suc\ z\ {\isasymin}\ Q{\isachardoublequoteclose}\ \isacommand{by}\isamarkupfalse%
\ blast\isanewline
\ \ \ \ \isacommand{have}\isamarkupfalse%
\ c{\isacharcolon}{\kern0pt}{\isachardoublequoteopen}{\isasymAnd}k{\isachardot}{\kern0pt}\ k\ {\isacharplus}{\kern0pt}\ x\ {\isasymin}\ Q{\isachardoublequoteclose}\isanewline
\ \ \ \ \isacommand{proof}\isamarkupfalse%
\ {\isacharminus}{\kern0pt}\isanewline
\ \ \ \ \ \ \isacommand{fix}\isamarkupfalse%
\ k\isanewline
\ \ \ \ \ \ \isacommand{show}\isamarkupfalse%
\ {\isachardoublequoteopen}k{\isacharplus}{\kern0pt}x\ {\isasymin}\ Q{\isachardoublequoteclose}\isanewline
\ \ \ \ \ \ \ \ \isacommand{by}\isamarkupfalse%
\ {\isacharparenleft}{\kern0pt}induction\ {\isachardoublequoteopen}k{\isachardoublequoteclose}{\isacharcomma}{\kern0pt}\ simp\ add{\isacharcolon}{\kern0pt}a{\isacharcomma}{\kern0pt}\ simp\ add{\isacharcolon}{\kern0pt}b{\isacharparenright}{\kern0pt}\isanewline
\ \ \ \ \isacommand{qed}\isamarkupfalse%
\isanewline
\ \ \ \ \isacommand{show}\isamarkupfalse%
\ {\isacharquery}{\kern0pt}thesis\ \isanewline
\ \ \ \ \ \ \isacommand{apply}\isamarkupfalse%
\ {\isacharparenleft}{\kern0pt}cases\ {\isachardoublequoteopen}{\isasymexists}z\ {\isasymin}\ Q{\isachardot}{\kern0pt}\ prime\ z{\isachardoublequoteclose}{\isacharparenright}{\kern0pt}\isanewline
\ \ \ \ \ \ \ \isacommand{apply}\isamarkupfalse%
\ blast\isanewline
\ \ \ \ \ \ \ \ \isacommand{by}\isamarkupfalse%
\ {\isacharparenleft}{\kern0pt}metis\ add{\isachardot}{\kern0pt}commute\ less{\isacharunderscore}{\kern0pt}natE\ bigger{\isacharunderscore}{\kern0pt}prime\ c{\isacharparenright}{\kern0pt}\isanewline
\ \ \isacommand{qed}\isamarkupfalse%
\isanewline
\ \ \isacommand{thus}\isamarkupfalse%
\ {\isachardoublequoteopen}{\isasymexists}z\ {\isasymin}\ Q{\isachardot}{\kern0pt}\ {\isasymforall}y{\isachardot}{\kern0pt}\ {\isacharparenleft}{\kern0pt}y{\isacharcomma}{\kern0pt}z{\isacharparenright}{\kern0pt}\ {\isasymin}\ {\isacharquery}{\kern0pt}S\ {\isasymlongrightarrow}\ y\ {\isasymnotin}\ Q{\isachardoublequoteclose}\ \isacommand{by}\isamarkupfalse%
\ blast\isanewline
\isacommand{qed}\isamarkupfalse%
%
\endisatagproof
{\isafoldproof}%
%
\isadelimproof
\isanewline
%
\endisadelimproof
\isanewline
\isacommand{function}\isamarkupfalse%
\ find{\isacharunderscore}{\kern0pt}prime{\isacharunderscore}{\kern0pt}above\ {\isacharcolon}{\kern0pt}{\isacharcolon}{\kern0pt}\ {\isachardoublequoteopen}nat\ {\isasymRightarrow}\ nat{\isachardoublequoteclose}\ \isakeyword{where}\isanewline
\ \ {\isachardoublequoteopen}find{\isacharunderscore}{\kern0pt}prime{\isacharunderscore}{\kern0pt}above\ n\ {\isacharequal}{\kern0pt}\ {\isacharparenleft}{\kern0pt}if\ prime\ n\ then\ n\ else\ find{\isacharunderscore}{\kern0pt}prime{\isacharunderscore}{\kern0pt}above\ {\isacharparenleft}{\kern0pt}Suc\ n{\isacharparenright}{\kern0pt}{\isacharparenright}{\kern0pt}{\isachardoublequoteclose}\isanewline
%
\isadelimproof
\ \ %
\endisadelimproof
%
\isatagproof
\isacommand{by}\isamarkupfalse%
\ auto%
\endisatagproof
{\isafoldproof}%
%
\isadelimproof
\isanewline
%
\endisadelimproof
\isacommand{termination}\isamarkupfalse%
\isanewline
%
\isadelimproof
\ \ %
\endisadelimproof
%
\isatagproof
\isacommand{apply}\isamarkupfalse%
\ {\isacharparenleft}{\kern0pt}relation\ {\isachardoublequoteopen}{\isacharparenleft}{\kern0pt}{\isasymlambda}n{\isachardot}{\kern0pt}\ {\isacharparenleft}{\kern0pt}Suc\ n{\isacharcomma}{\kern0pt}\ n{\isacharparenright}{\kern0pt}{\isacharparenright}{\kern0pt}\ {\isacharbackquote}{\kern0pt}\ {\isacharbraceleft}{\kern0pt}n{\isachardot}{\kern0pt}\ {\isasymnot}\ {\isacharparenleft}{\kern0pt}prime\ n{\isacharparenright}{\kern0pt}{\isacharbraceright}{\kern0pt}{\isachardoublequoteclose}{\isacharparenright}{\kern0pt}\isanewline
\ \ \isacommand{using}\isamarkupfalse%
\ inf{\isacharunderscore}{\kern0pt}primes\ \isacommand{apply}\isamarkupfalse%
\ blast\isanewline
\ \ \isacommand{by}\isamarkupfalse%
\ simp%
\endisatagproof
{\isafoldproof}%
%
\isadelimproof
\isanewline
%
\endisadelimproof
\isanewline
\isacommand{declare}\isamarkupfalse%
\ find{\isacharunderscore}{\kern0pt}prime{\isacharunderscore}{\kern0pt}above{\isachardot}{\kern0pt}simps\ {\isacharbrackleft}{\kern0pt}simp\ del{\isacharbrackright}{\kern0pt}\isanewline
\isanewline
\isacommand{lemma}\isamarkupfalse%
\ find{\isacharunderscore}{\kern0pt}prime{\isacharunderscore}{\kern0pt}above{\isacharunderscore}{\kern0pt}is{\isacharunderscore}{\kern0pt}prime{\isacharcolon}{\kern0pt}\isanewline
\ \ {\isachardoublequoteopen}prime\ {\isacharparenleft}{\kern0pt}find{\isacharunderscore}{\kern0pt}prime{\isacharunderscore}{\kern0pt}above\ n{\isacharparenright}{\kern0pt}{\isachardoublequoteclose}\isanewline
%
\isadelimproof
\ \ %
\endisadelimproof
%
\isatagproof
\isacommand{apply}\isamarkupfalse%
\ {\isacharparenleft}{\kern0pt}induction\ n\ rule{\isacharcolon}{\kern0pt}find{\isacharunderscore}{\kern0pt}prime{\isacharunderscore}{\kern0pt}above{\isachardot}{\kern0pt}induct{\isacharparenright}{\kern0pt}\isanewline
\ \ \isacommand{by}\isamarkupfalse%
\ {\isacharparenleft}{\kern0pt}simp\ add{\isacharcolon}{\kern0pt}\ find{\isacharunderscore}{\kern0pt}prime{\isacharunderscore}{\kern0pt}above{\isachardot}{\kern0pt}simps{\isacharparenright}{\kern0pt}{\isacharplus}{\kern0pt}%
\endisatagproof
{\isafoldproof}%
%
\isadelimproof
\isanewline
%
\endisadelimproof
\isanewline
\isacommand{lemma}\isamarkupfalse%
\ find{\isacharunderscore}{\kern0pt}prime{\isacharunderscore}{\kern0pt}above{\isacharunderscore}{\kern0pt}min{\isacharcolon}{\kern0pt}\isanewline
\ \ {\isachardoublequoteopen}find{\isacharunderscore}{\kern0pt}prime{\isacharunderscore}{\kern0pt}above\ n\ {\isasymge}\ {\isadigit{2}}{\isachardoublequoteclose}\isanewline
%
\isadelimproof
\ \ %
\endisadelimproof
%
\isatagproof
\isacommand{by}\isamarkupfalse%
\ {\isacharparenleft}{\kern0pt}metis\ find{\isacharunderscore}{\kern0pt}prime{\isacharunderscore}{\kern0pt}above{\isacharunderscore}{\kern0pt}is{\isacharunderscore}{\kern0pt}prime\ prime{\isacharunderscore}{\kern0pt}ge{\isacharunderscore}{\kern0pt}{\isadigit{2}}{\isacharunderscore}{\kern0pt}nat{\isacharparenright}{\kern0pt}%
\endisatagproof
{\isafoldproof}%
%
\isadelimproof
\isanewline
%
\endisadelimproof
\isanewline
\isacommand{lemma}\isamarkupfalse%
\ find{\isacharunderscore}{\kern0pt}prime{\isacharunderscore}{\kern0pt}above{\isacharunderscore}{\kern0pt}lower{\isacharunderscore}{\kern0pt}bound{\isacharcolon}{\kern0pt}\isanewline
\ \ {\isachardoublequoteopen}find{\isacharunderscore}{\kern0pt}prime{\isacharunderscore}{\kern0pt}above\ n\ {\isasymge}\ n{\isachardoublequoteclose}\isanewline
%
\isadelimproof
\ \ %
\endisadelimproof
%
\isatagproof
\isacommand{apply}\isamarkupfalse%
\ {\isacharparenleft}{\kern0pt}induction\ n\ rule{\isacharcolon}{\kern0pt}find{\isacharunderscore}{\kern0pt}prime{\isacharunderscore}{\kern0pt}above{\isachardot}{\kern0pt}induct{\isacharparenright}{\kern0pt}\isanewline
\ \ \isacommand{by}\isamarkupfalse%
\ {\isacharparenleft}{\kern0pt}metis\ find{\isacharunderscore}{\kern0pt}prime{\isacharunderscore}{\kern0pt}above{\isachardot}{\kern0pt}simps\ linorder{\isacharunderscore}{\kern0pt}le{\isacharunderscore}{\kern0pt}cases\ not{\isacharunderscore}{\kern0pt}less{\isacharunderscore}{\kern0pt}eq{\isacharunderscore}{\kern0pt}eq{\isacharparenright}{\kern0pt}%
\endisatagproof
{\isafoldproof}%
%
\isadelimproof
\isanewline
%
\endisadelimproof
\isanewline
\isacommand{lemma}\isamarkupfalse%
\ find{\isacharunderscore}{\kern0pt}prime{\isacharunderscore}{\kern0pt}above{\isacharunderscore}{\kern0pt}upper{\isacharunderscore}{\kern0pt}boundI{\isacharcolon}{\kern0pt}\isanewline
\ \ \isakeyword{assumes}\ {\isachardoublequoteopen}prime\ m{\isachardoublequoteclose}\isanewline
\ \ \isakeyword{shows}\ {\isachardoublequoteopen}n\ {\isasymle}\ m\ {\isasymLongrightarrow}\ find{\isacharunderscore}{\kern0pt}prime{\isacharunderscore}{\kern0pt}above\ n\ {\isasymle}\ m{\isachardoublequoteclose}\isanewline
%
\isadelimproof
%
\endisadelimproof
%
\isatagproof
\isacommand{proof}\isamarkupfalse%
\ {\isacharparenleft}{\kern0pt}induction\ n\ rule{\isacharcolon}{\kern0pt}find{\isacharunderscore}{\kern0pt}prime{\isacharunderscore}{\kern0pt}above{\isachardot}{\kern0pt}induct{\isacharparenright}{\kern0pt}\isanewline
\ \ \isacommand{case}\isamarkupfalse%
\ {\isacharparenleft}{\kern0pt}{\isadigit{1}}\ n{\isacharparenright}{\kern0pt}\isanewline
\ \ \isacommand{have}\isamarkupfalse%
\ a{\isacharcolon}{\kern0pt}{\isachardoublequoteopen}{\isasymnot}prime\ n\ {\isasymLongrightarrow}\ Suc\ n\ {\isasymle}\ m{\isachardoublequoteclose}\isanewline
\ \ \ \ \isacommand{by}\isamarkupfalse%
\ {\isacharparenleft}{\kern0pt}metis\ assms\ {\isachardoublequoteopen}{\isadigit{1}}{\isachardot}{\kern0pt}prems{\isachardoublequoteclose}\ not{\isacharunderscore}{\kern0pt}less{\isacharunderscore}{\kern0pt}eq{\isacharunderscore}{\kern0pt}eq\ le{\isacharunderscore}{\kern0pt}antisym{\isacharparenright}{\kern0pt}\isanewline
\ \ \isacommand{show}\isamarkupfalse%
\ {\isacharquery}{\kern0pt}case\ \isacommand{using}\isamarkupfalse%
\ {\isadigit{1}}\ \isanewline
\ \ \ \ \isacommand{apply}\isamarkupfalse%
\ {\isacharparenleft}{\kern0pt}cases\ {\isachardoublequoteopen}prime\ n{\isachardoublequoteclose}{\isacharparenright}{\kern0pt}\isanewline
\ \ \ \ \ \isacommand{apply}\isamarkupfalse%
\ {\isacharparenleft}{\kern0pt}subst\ find{\isacharunderscore}{\kern0pt}prime{\isacharunderscore}{\kern0pt}above{\isachardot}{\kern0pt}simps{\isacharparenright}{\kern0pt}\isanewline
\ \ \ \ \isacommand{using}\isamarkupfalse%
\ assms{\isacharparenleft}{\kern0pt}{\isadigit{1}}{\isacharparenright}{\kern0pt}\ \isacommand{apply}\isamarkupfalse%
\ simp\isanewline
\ \ \ \ \isacommand{by}\isamarkupfalse%
\ {\isacharparenleft}{\kern0pt}metis\ a\ find{\isacharunderscore}{\kern0pt}prime{\isacharunderscore}{\kern0pt}above{\isachardot}{\kern0pt}simps{\isacharparenright}{\kern0pt}\isanewline
\isacommand{qed}\isamarkupfalse%
%
\endisatagproof
{\isafoldproof}%
%
\isadelimproof
\isanewline
%
\endisadelimproof
\isanewline
\isacommand{lemma}\isamarkupfalse%
\ find{\isacharunderscore}{\kern0pt}prime{\isacharunderscore}{\kern0pt}above{\isacharunderscore}{\kern0pt}upper{\isacharunderscore}{\kern0pt}bound{\isacharcolon}{\kern0pt}\isanewline
\ \ {\isachardoublequoteopen}find{\isacharunderscore}{\kern0pt}prime{\isacharunderscore}{\kern0pt}above\ n\ {\isasymle}\ {\isadigit{2}}{\isacharasterisk}{\kern0pt}n{\isacharplus}{\kern0pt}{\isadigit{2}}{\isachardoublequoteclose}\isanewline
%
\isadelimproof
%
\endisadelimproof
%
\isatagproof
\isacommand{proof}\isamarkupfalse%
\ {\isacharparenleft}{\kern0pt}cases\ {\isachardoublequoteopen}n\ {\isasymle}\ {\isadigit{1}}{\isachardoublequoteclose}{\isacharparenright}{\kern0pt}\isanewline
\ \ \isacommand{case}\isamarkupfalse%
\ True\isanewline
\ \ \isacommand{have}\isamarkupfalse%
\ {\isachardoublequoteopen}find{\isacharunderscore}{\kern0pt}prime{\isacharunderscore}{\kern0pt}above\ n\ {\isasymle}\ {\isadigit{2}}{\isachardoublequoteclose}\isanewline
\ \ \ \ \isacommand{apply}\isamarkupfalse%
\ {\isacharparenleft}{\kern0pt}rule\ find{\isacharunderscore}{\kern0pt}prime{\isacharunderscore}{\kern0pt}above{\isacharunderscore}{\kern0pt}upper{\isacharunderscore}{\kern0pt}boundI{\isacharcomma}{\kern0pt}\ simp{\isacharparenright}{\kern0pt}\ \isacommand{using}\isamarkupfalse%
\ True\ \isacommand{by}\isamarkupfalse%
\ simp\ \isanewline
\ \ \isacommand{then}\isamarkupfalse%
\ \isacommand{show}\isamarkupfalse%
\ {\isacharquery}{\kern0pt}thesis\ \isacommand{using}\isamarkupfalse%
\ trans{\isacharunderscore}{\kern0pt}le{\isacharunderscore}{\kern0pt}add{\isadigit{2}}\ \isacommand{by}\isamarkupfalse%
\ blast\isanewline
\isacommand{next}\isamarkupfalse%
\isanewline
\ \ \isacommand{case}\isamarkupfalse%
\ False\isanewline
\ \ \isacommand{hence}\isamarkupfalse%
\ a{\isacharcolon}{\kern0pt}{\isachardoublequoteopen}n\ {\isachargreater}{\kern0pt}\ {\isadigit{1}}{\isachardoublequoteclose}\ \isacommand{by}\isamarkupfalse%
\ auto\isanewline
\ \ \isacommand{then}\isamarkupfalse%
\ \isacommand{obtain}\isamarkupfalse%
\ p\ \isakeyword{where}\ p{\isacharunderscore}{\kern0pt}bound{\isacharcolon}{\kern0pt}\ {\isachardoublequoteopen}p\ {\isasymin}\ {\isacharbraceleft}{\kern0pt}n{\isacharless}{\kern0pt}{\isachardot}{\kern0pt}{\isachardot}{\kern0pt}{\isacharless}{\kern0pt}{\isadigit{2}}{\isacharasterisk}{\kern0pt}n{\isacharbraceright}{\kern0pt}{\isachardoublequoteclose}\ \isakeyword{and}\ p{\isacharunderscore}{\kern0pt}prime{\isacharcolon}{\kern0pt}\ {\isachardoublequoteopen}prime\ p{\isachardoublequoteclose}\ \isanewline
\ \ \ \ \isacommand{using}\isamarkupfalse%
\ bertrand\ \isacommand{by}\isamarkupfalse%
\ metis\isanewline
\ \ \isacommand{have}\isamarkupfalse%
\ {\isachardoublequoteopen}find{\isacharunderscore}{\kern0pt}prime{\isacharunderscore}{\kern0pt}above\ n\ {\isasymle}\ p{\isachardoublequoteclose}\isanewline
\ \ \ \ \isacommand{apply}\isamarkupfalse%
\ {\isacharparenleft}{\kern0pt}rule\ find{\isacharunderscore}{\kern0pt}prime{\isacharunderscore}{\kern0pt}above{\isacharunderscore}{\kern0pt}upper{\isacharunderscore}{\kern0pt}boundI{\isacharparenright}{\kern0pt}\isanewline
\ \ \ \ \ \isacommand{apply}\isamarkupfalse%
\ {\isacharparenleft}{\kern0pt}metis\ p{\isacharunderscore}{\kern0pt}prime{\isacharparenright}{\kern0pt}\isanewline
\ \ \ \ \isacommand{using}\isamarkupfalse%
\ p{\isacharunderscore}{\kern0pt}bound\ \isacommand{by}\isamarkupfalse%
\ simp\isanewline
\ \ \isacommand{thus}\isamarkupfalse%
\ {\isacharquery}{\kern0pt}thesis\ \isacommand{using}\isamarkupfalse%
\ p{\isacharunderscore}{\kern0pt}bound\ \isanewline
\ \ \ \ \isacommand{by}\isamarkupfalse%
\ {\isacharparenleft}{\kern0pt}metis\ greaterThanLessThan{\isacharunderscore}{\kern0pt}iff\ nat{\isacharunderscore}{\kern0pt}le{\isacharunderscore}{\kern0pt}iff{\isacharunderscore}{\kern0pt}add\ nat{\isacharunderscore}{\kern0pt}less{\isacharunderscore}{\kern0pt}le\ trans{\isacharunderscore}{\kern0pt}le{\isacharunderscore}{\kern0pt}add{\isadigit{1}}{\isacharparenright}{\kern0pt}\isanewline
\isacommand{qed}\isamarkupfalse%
%
\endisatagproof
{\isafoldproof}%
%
\isadelimproof
\isanewline
%
\endisadelimproof
%
\isadelimtheory
\isanewline
%
\endisadelimtheory
%
\isatagtheory
\isacommand{end}\isamarkupfalse%
%
\endisatagtheory
{\isafoldtheory}%
%
\isadelimtheory
%
\endisadelimtheory
%
\end{isabellebody}%
\endinput
%:%file=Primes_Ext.tex%:%
%:%11=1%:%
%:%23=3%:%
%:%31=5%:%
%:%32=5%:%
%:%33=6%:%
%:%34=7%:%
%:%41=7%:%
%:%42=8%:%
%:%43=9%:%
%:%44=9%:%
%:%51=10%:%
%:%52=10%:%
%:%53=11%:%
%:%54=11%:%
%:%55=12%:%
%:%56=12%:%
%:%57=13%:%
%:%58=13%:%
%:%59=14%:%
%:%60=15%:%
%:%61=15%:%
%:%62=16%:%
%:%63=16%:%
%:%64=17%:%
%:%65=17%:%
%:%66=18%:%
%:%67=18%:%
%:%68=18%:%
%:%69=18%:%
%:%70=19%:%
%:%71=19%:%
%:%72=20%:%
%:%73=20%:%
%:%74=21%:%
%:%75=21%:%
%:%76=21%:%
%:%77=22%:%
%:%78=22%:%
%:%79=23%:%
%:%80=23%:%
%:%81=24%:%
%:%82=24%:%
%:%83=25%:%
%:%84=25%:%
%:%85=26%:%
%:%86=26%:%
%:%87=27%:%
%:%88=27%:%
%:%89=28%:%
%:%90=28%:%
%:%91=29%:%
%:%92=29%:%
%:%93=30%:%
%:%94=30%:%
%:%95=31%:%
%:%96=31%:%
%:%97=32%:%
%:%98=32%:%
%:%99=33%:%
%:%100=33%:%
%:%101=33%:%
%:%102=34%:%
%:%108=34%:%
%:%111=35%:%
%:%112=36%:%
%:%113=36%:%
%:%114=37%:%
%:%117=38%:%
%:%121=38%:%
%:%122=38%:%
%:%127=38%:%
%:%130=39%:%
%:%131=39%:%
%:%134=40%:%
%:%138=40%:%
%:%139=40%:%
%:%140=41%:%
%:%141=41%:%
%:%142=41%:%
%:%143=42%:%
%:%144=42%:%
%:%149=42%:%
%:%152=43%:%
%:%153=44%:%
%:%154=44%:%
%:%155=45%:%
%:%156=46%:%
%:%157=46%:%
%:%158=47%:%
%:%161=48%:%
%:%165=48%:%
%:%166=48%:%
%:%167=49%:%
%:%168=49%:%
%:%173=49%:%
%:%176=50%:%
%:%177=51%:%
%:%178=51%:%
%:%179=52%:%
%:%182=53%:%
%:%186=53%:%
%:%187=53%:%
%:%192=53%:%
%:%195=54%:%
%:%196=55%:%
%:%197=55%:%
%:%198=56%:%
%:%201=57%:%
%:%205=57%:%
%:%206=57%:%
%:%207=58%:%
%:%208=58%:%
%:%213=58%:%
%:%216=59%:%
%:%217=60%:%
%:%218=60%:%
%:%219=61%:%
%:%220=62%:%
%:%227=63%:%
%:%228=63%:%
%:%229=64%:%
%:%230=64%:%
%:%231=65%:%
%:%232=65%:%
%:%233=66%:%
%:%234=66%:%
%:%235=67%:%
%:%236=67%:%
%:%237=67%:%
%:%238=68%:%
%:%239=68%:%
%:%240=69%:%
%:%241=69%:%
%:%242=70%:%
%:%243=70%:%
%:%244=70%:%
%:%245=71%:%
%:%246=71%:%
%:%247=72%:%
%:%253=72%:%
%:%256=73%:%
%:%257=74%:%
%:%258=74%:%
%:%259=75%:%
%:%266=76%:%
%:%267=76%:%
%:%268=77%:%
%:%269=77%:%
%:%270=78%:%
%:%271=78%:%
%:%272=79%:%
%:%273=79%:%
%:%274=79%:%
%:%275=79%:%
%:%276=80%:%
%:%277=80%:%
%:%278=80%:%
%:%279=80%:%
%:%280=80%:%
%:%281=81%:%
%:%282=81%:%
%:%283=82%:%
%:%284=82%:%
%:%285=83%:%
%:%286=83%:%
%:%287=83%:%
%:%288=84%:%
%:%289=84%:%
%:%290=84%:%
%:%291=85%:%
%:%292=85%:%
%:%293=85%:%
%:%294=86%:%
%:%295=86%:%
%:%296=87%:%
%:%297=87%:%
%:%298=88%:%
%:%299=88%:%
%:%300=89%:%
%:%301=89%:%
%:%302=89%:%
%:%303=90%:%
%:%304=90%:%
%:%305=90%:%
%:%306=91%:%
%:%307=91%:%
%:%308=92%:%
%:%314=92%:%
%:%319=93%:%
%:%324=94%:%

%
\begin{isabellebody}%
\setisabellecontext{Multiset{\isacharunderscore}{\kern0pt}Ext}%
%
\isadelimdocument
%
\endisadelimdocument
%
\isatagdocument
%
\isamarkupsection{Multisets%
}
\isamarkuptrue%
%
\endisatagdocument
{\isafolddocument}%
%
\isadelimdocument
%
\endisadelimdocument
%
\isadelimtheory
%
\endisadelimtheory
%
\isatagtheory
\isacommand{theory}\isamarkupfalse%
\ Multiset{\isacharunderscore}{\kern0pt}Ext\isanewline
\ \ \isakeyword{imports}\ Main\ {\isachardoublequoteopen}HOL{\isachardot}{\kern0pt}Real{\isachardoublequoteclose}\ {\isachardoublequoteopen}HOL{\isacharminus}{\kern0pt}Library{\isachardot}{\kern0pt}Multiset{\isachardoublequoteclose}\isanewline
\isakeyword{begin}%
\endisatagtheory
{\isafoldtheory}%
%
\isadelimtheory
%
\endisadelimtheory
%
\begin{isamarkuptext}%
This section contains results about multisets in addition to "HOL.Multiset"%
\end{isamarkuptext}\isamarkuptrue%
%
\begin{isamarkuptext}%
This is a induction scheme over the distinct elements of a multisets: 
We can represent each multiset as a sum like: 
\isa{replicate{\isacharunderscore}{\kern0pt}mset\ n\isactrlsub {\isadigit{1}}\ x\isactrlsub {\isadigit{1}}\ {\isacharplus}{\kern0pt}\ replicate{\isacharunderscore}{\kern0pt}mset\ n\isactrlsub {\isadigit{2}}\ x\isactrlsub {\isadigit{2}}\ {\isacharplus}{\kern0pt}\ {\isachardot}{\kern0pt}{\isachardot}{\kern0pt}{\isachardot}{\kern0pt}\ {\isacharplus}{\kern0pt}\ replicate{\isacharunderscore}{\kern0pt}mset\ n\isactrlsub k\ x\isactrlsub k} where the 
\isa{x\isactrlsub i} are distinct.%
\end{isamarkuptext}\isamarkuptrue%
\isacommand{lemma}\isamarkupfalse%
\ disj{\isacharunderscore}{\kern0pt}induct{\isacharunderscore}{\kern0pt}mset{\isacharcolon}{\kern0pt}\isanewline
\ \ \isakeyword{assumes}\ {\isachardoublequoteopen}P\ {\isacharbraceleft}{\kern0pt}{\isacharhash}{\kern0pt}{\isacharbraceright}{\kern0pt}{\isachardoublequoteclose}\isanewline
\ \ \isakeyword{assumes}\ {\isachardoublequoteopen}{\isasymAnd}n\ M\ x{\isachardot}{\kern0pt}\ P\ M\ {\isasymLongrightarrow}\ {\isasymnot}{\isacharparenleft}{\kern0pt}x\ {\isasymin}{\isacharhash}{\kern0pt}\ M{\isacharparenright}{\kern0pt}\ {\isasymLongrightarrow}\ n\ {\isachargreater}{\kern0pt}\ {\isadigit{0}}\ {\isasymLongrightarrow}\ P\ {\isacharparenleft}{\kern0pt}M\ {\isacharplus}{\kern0pt}\ replicate{\isacharunderscore}{\kern0pt}mset\ n\ x{\isacharparenright}{\kern0pt}{\isachardoublequoteclose}\isanewline
\ \ \isakeyword{shows}\ {\isachardoublequoteopen}P\ M{\isachardoublequoteclose}\isanewline
%
\isadelimproof
%
\endisadelimproof
%
\isatagproof
\isacommand{proof}\isamarkupfalse%
\ {\isacharparenleft}{\kern0pt}induction\ {\isachardoublequoteopen}size\ M{\isachardoublequoteclose}\ arbitrary{\isacharcolon}{\kern0pt}\ M\ rule{\isacharcolon}{\kern0pt}nat{\isacharunderscore}{\kern0pt}less{\isacharunderscore}{\kern0pt}induct{\isacharparenright}{\kern0pt}\isanewline
\ \ \isacommand{case}\isamarkupfalse%
\ {\isadigit{1}}\isanewline
\ \ \isacommand{show}\isamarkupfalse%
\ {\isacharquery}{\kern0pt}case\isanewline
\ \ \isacommand{proof}\isamarkupfalse%
\ {\isacharparenleft}{\kern0pt}cases\ {\isachardoublequoteopen}M\ {\isacharequal}{\kern0pt}\ {\isacharbraceleft}{\kern0pt}{\isacharhash}{\kern0pt}{\isacharbraceright}{\kern0pt}{\isachardoublequoteclose}{\isacharparenright}{\kern0pt}\isanewline
\ \ \ \ \isacommand{case}\isamarkupfalse%
\ True\isanewline
\ \ \ \ \isacommand{then}\isamarkupfalse%
\ \isacommand{show}\isamarkupfalse%
\ {\isacharquery}{\kern0pt}thesis\ \isacommand{using}\isamarkupfalse%
\ assms\ \isacommand{by}\isamarkupfalse%
\ simp\isanewline
\ \ \isacommand{next}\isamarkupfalse%
\isanewline
\ \ \ \ \isacommand{case}\isamarkupfalse%
\ False\isanewline
\ \ \ \ \isacommand{then}\isamarkupfalse%
\ \isacommand{obtain}\isamarkupfalse%
\ x\ \isakeyword{where}\ x{\isacharunderscore}{\kern0pt}def{\isacharcolon}{\kern0pt}\ {\isachardoublequoteopen}x\ {\isasymin}{\isacharhash}{\kern0pt}\ M{\isachardoublequoteclose}\ \isacommand{using}\isamarkupfalse%
\ multiset{\isacharunderscore}{\kern0pt}nonemptyE\ \isacommand{by}\isamarkupfalse%
\ auto\isanewline
\ \ \ \ \isacommand{define}\isamarkupfalse%
\ M{\isadigit{1}}\ \isakeyword{where}\ {\isachardoublequoteopen}M{\isadigit{1}}\ {\isacharequal}{\kern0pt}\ M\ {\isacharminus}{\kern0pt}\ replicate{\isacharunderscore}{\kern0pt}mset\ {\isacharparenleft}{\kern0pt}count\ M\ x{\isacharparenright}{\kern0pt}\ x{\isachardoublequoteclose}\isanewline
\ \ \ \ \isacommand{then}\isamarkupfalse%
\ \isacommand{have}\isamarkupfalse%
\ M{\isacharunderscore}{\kern0pt}def{\isacharcolon}{\kern0pt}\ {\isachardoublequoteopen}M\ {\isacharequal}{\kern0pt}\ M{\isadigit{1}}\ {\isacharplus}{\kern0pt}\ replicate{\isacharunderscore}{\kern0pt}mset\ {\isacharparenleft}{\kern0pt}count\ M\ x{\isacharparenright}{\kern0pt}\ x{\isachardoublequoteclose}\isanewline
\ \ \ \ \ \ \isacommand{by}\isamarkupfalse%
\ {\isacharparenleft}{\kern0pt}metis\ count{\isacharunderscore}{\kern0pt}le{\isacharunderscore}{\kern0pt}replicate{\isacharunderscore}{\kern0pt}mset{\isacharunderscore}{\kern0pt}subset{\isacharunderscore}{\kern0pt}eq\ dual{\isacharunderscore}{\kern0pt}order{\isachardot}{\kern0pt}refl\ subset{\isacharunderscore}{\kern0pt}mset{\isachardot}{\kern0pt}diff{\isacharunderscore}{\kern0pt}add{\isacharparenright}{\kern0pt}\isanewline
\ \ \ \ \isacommand{have}\isamarkupfalse%
\ {\isachardoublequoteopen}size\ M{\isadigit{1}}\ {\isacharless}{\kern0pt}\ size\ M{\isachardoublequoteclose}\isanewline
\ \ \ \ \ \ \isacommand{by}\isamarkupfalse%
\ {\isacharparenleft}{\kern0pt}metis\ M{\isacharunderscore}{\kern0pt}def\ x{\isacharunderscore}{\kern0pt}def\ count{\isacharunderscore}{\kern0pt}greater{\isacharunderscore}{\kern0pt}zero{\isacharunderscore}{\kern0pt}iff\ less{\isacharunderscore}{\kern0pt}add{\isacharunderscore}{\kern0pt}same{\isacharunderscore}{\kern0pt}cancel{\isadigit{1}}\ size{\isacharunderscore}{\kern0pt}replicate{\isacharunderscore}{\kern0pt}mset\ size{\isacharunderscore}{\kern0pt}union{\isacharparenright}{\kern0pt}\isanewline
\ \ \ \ \isacommand{hence}\isamarkupfalse%
\ {\isachardoublequoteopen}P\ M{\isadigit{1}}{\isachardoublequoteclose}\ \isacommand{using}\isamarkupfalse%
\ {\isadigit{1}}\ \isacommand{by}\isamarkupfalse%
\ blast\isanewline
\ \ \ \ \isacommand{then}\isamarkupfalse%
\ \isacommand{show}\isamarkupfalse%
\ {\isachardoublequoteopen}P\ M{\isachardoublequoteclose}\ \isanewline
\ \ \ \ \ \ \isacommand{apply}\isamarkupfalse%
\ {\isacharparenleft}{\kern0pt}subst\ M{\isacharunderscore}{\kern0pt}def{\isacharcomma}{\kern0pt}\ rule\ assms{\isacharparenleft}{\kern0pt}{\isadigit{2}}{\isacharparenright}{\kern0pt}{\isacharcomma}{\kern0pt}\ simp{\isacharparenright}{\kern0pt}\isanewline
\ \ \ \ \ \ \isacommand{by}\isamarkupfalse%
\ {\isacharparenleft}{\kern0pt}simp\ add{\isacharcolon}{\kern0pt}M{\isadigit{1}}{\isacharunderscore}{\kern0pt}def\ x{\isacharunderscore}{\kern0pt}def\ count{\isacharunderscore}{\kern0pt}eq{\isacharunderscore}{\kern0pt}zero{\isacharunderscore}{\kern0pt}iff{\isacharbrackleft}{\kern0pt}symmetric{\isacharbrackright}{\kern0pt}{\isacharparenright}{\kern0pt}{\isacharplus}{\kern0pt}\isanewline
\ \ \isacommand{qed}\isamarkupfalse%
\isanewline
\isacommand{qed}\isamarkupfalse%
%
\endisatagproof
{\isafoldproof}%
%
\isadelimproof
\isanewline
%
\endisadelimproof
\isanewline
\isacommand{lemma}\isamarkupfalse%
\ prod{\isacharunderscore}{\kern0pt}mset{\isacharunderscore}{\kern0pt}conv{\isacharcolon}{\kern0pt}\ \isanewline
\ \ \isakeyword{fixes}\ f\ {\isacharcolon}{\kern0pt}{\isacharcolon}{\kern0pt}\ {\isachardoublequoteopen}{\isacharprime}{\kern0pt}a\ {\isasymRightarrow}\ {\isacharprime}{\kern0pt}b{\isacharcolon}{\kern0pt}{\isacharcolon}{\kern0pt}{\isacharbraceleft}{\kern0pt}comm{\isacharunderscore}{\kern0pt}monoid{\isacharunderscore}{\kern0pt}mult{\isacharbraceright}{\kern0pt}{\isachardoublequoteclose}\isanewline
\ \ \isakeyword{shows}\ {\isachardoublequoteopen}prod{\isacharunderscore}{\kern0pt}mset\ {\isacharparenleft}{\kern0pt}image{\isacharunderscore}{\kern0pt}mset\ f\ A{\isacharparenright}{\kern0pt}\ {\isacharequal}{\kern0pt}\ prod\ {\isacharparenleft}{\kern0pt}{\isasymlambda}x{\isachardot}{\kern0pt}\ f\ x{\isacharcircum}{\kern0pt}{\isacharparenleft}{\kern0pt}count\ A\ x{\isacharparenright}{\kern0pt}{\isacharparenright}{\kern0pt}\ {\isacharparenleft}{\kern0pt}set{\isacharunderscore}{\kern0pt}mset\ A{\isacharparenright}{\kern0pt}{\isachardoublequoteclose}\isanewline
%
\isadelimproof
%
\endisadelimproof
%
\isatagproof
\isacommand{proof}\isamarkupfalse%
\ {\isacharparenleft}{\kern0pt}induction\ A\ rule{\isacharcolon}{\kern0pt}\ disj{\isacharunderscore}{\kern0pt}induct{\isacharunderscore}{\kern0pt}mset{\isacharparenright}{\kern0pt}\isanewline
\ \ \isacommand{case}\isamarkupfalse%
\ {\isadigit{1}}\isanewline
\ \ \isacommand{then}\isamarkupfalse%
\ \isacommand{show}\isamarkupfalse%
\ {\isacharquery}{\kern0pt}case\ \isacommand{by}\isamarkupfalse%
\ simp\isanewline
\isacommand{next}\isamarkupfalse%
\isanewline
\ \ \isacommand{case}\isamarkupfalse%
\ {\isacharparenleft}{\kern0pt}{\isadigit{2}}\ n\ M\ x{\isacharparenright}{\kern0pt}\isanewline
\ \ \isacommand{moreover}\isamarkupfalse%
\ \isacommand{have}\isamarkupfalse%
\ {\isachardoublequoteopen}count\ M\ x\ {\isacharequal}{\kern0pt}\ {\isadigit{0}}{\isachardoublequoteclose}\ \isacommand{using}\isamarkupfalse%
\ {\isadigit{2}}\ \isacommand{by}\isamarkupfalse%
\ {\isacharparenleft}{\kern0pt}simp\ add{\isacharcolon}{\kern0pt}\ count{\isacharunderscore}{\kern0pt}eq{\isacharunderscore}{\kern0pt}zero{\isacharunderscore}{\kern0pt}iff{\isacharparenright}{\kern0pt}\isanewline
\ \ \isacommand{moreover}\isamarkupfalse%
\ \isacommand{have}\isamarkupfalse%
\ {\isachardoublequoteopen}{\isasymAnd}y{\isachardot}{\kern0pt}\ y\ {\isasymin}\ set{\isacharunderscore}{\kern0pt}mset\ M\ {\isasymLongrightarrow}\ y\ {\isasymnoteq}\ x{\isachardoublequoteclose}\ \isacommand{using}\isamarkupfalse%
\ {\isadigit{2}}\ \isacommand{by}\isamarkupfalse%
\ blast\isanewline
\ \ \isacommand{ultimately}\isamarkupfalse%
\ \isacommand{show}\isamarkupfalse%
\ {\isacharquery}{\kern0pt}case\ \isacommand{by}\isamarkupfalse%
\ {\isacharparenleft}{\kern0pt}simp\ add{\isacharcolon}{\kern0pt}algebra{\isacharunderscore}{\kern0pt}simps{\isacharparenright}{\kern0pt}\ \isanewline
\isacommand{qed}\isamarkupfalse%
%
\endisatagproof
{\isafoldproof}%
%
\isadelimproof
\isanewline
%
\endisadelimproof
\isanewline
\isacommand{lemma}\isamarkupfalse%
\ sum{\isacharunderscore}{\kern0pt}collapse{\isacharcolon}{\kern0pt}\ \isanewline
\ \ \isakeyword{fixes}\ f\ {\isacharcolon}{\kern0pt}{\isacharcolon}{\kern0pt}\ {\isachardoublequoteopen}{\isacharprime}{\kern0pt}a\ {\isasymRightarrow}\ {\isacharprime}{\kern0pt}b{\isacharcolon}{\kern0pt}{\isacharcolon}{\kern0pt}{\isacharbraceleft}{\kern0pt}comm{\isacharunderscore}{\kern0pt}monoid{\isacharunderscore}{\kern0pt}add{\isacharbraceright}{\kern0pt}{\isachardoublequoteclose}\isanewline
\ \ \isakeyword{assumes}\ {\isachardoublequoteopen}finite\ A{\isachardoublequoteclose}\isanewline
\ \ \isakeyword{assumes}\ {\isachardoublequoteopen}z\ {\isasymin}\ A{\isachardoublequoteclose}\isanewline
\ \ \isakeyword{assumes}\ {\isachardoublequoteopen}{\isasymAnd}y{\isachardot}{\kern0pt}\ y\ {\isasymin}\ A\ {\isasymLongrightarrow}\ y\ {\isasymnoteq}\ z\ {\isasymLongrightarrow}\ f\ y\ {\isacharequal}{\kern0pt}\ {\isadigit{0}}{\isachardoublequoteclose}\isanewline
\ \ \isakeyword{shows}\ {\isachardoublequoteopen}sum\ f\ A\ {\isacharequal}{\kern0pt}\ f\ z{\isachardoublequoteclose}\isanewline
%
\isadelimproof
\ \ %
\endisadelimproof
%
\isatagproof
\isacommand{using}\isamarkupfalse%
\ sum{\isachardot}{\kern0pt}union{\isacharunderscore}{\kern0pt}disjoint{\isacharbrackleft}{\kern0pt}\isakeyword{where}\ A{\isacharequal}{\kern0pt}{\isachardoublequoteopen}A{\isacharminus}{\kern0pt}{\isacharbraceleft}{\kern0pt}z{\isacharbraceright}{\kern0pt}{\isachardoublequoteclose}\ \isakeyword{and}\ B{\isacharequal}{\kern0pt}{\isachardoublequoteopen}{\isacharbraceleft}{\kern0pt}z{\isacharbraceright}{\kern0pt}{\isachardoublequoteclose}\ \isakeyword{and}\ g{\isacharequal}{\kern0pt}{\isachardoublequoteopen}f{\isachardoublequoteclose}{\isacharbrackright}{\kern0pt}\isanewline
\ \ \isacommand{by}\isamarkupfalse%
\ {\isacharparenleft}{\kern0pt}simp\ add{\isacharcolon}{\kern0pt}\ assms\ sum{\isachardot}{\kern0pt}insert{\isacharunderscore}{\kern0pt}if{\isacharparenright}{\kern0pt}%
\endisatagproof
{\isafoldproof}%
%
\isadelimproof
%
\endisadelimproof
%
\begin{isamarkuptext}%
There is a version \isa{sum{\isacharunderscore}{\kern0pt}list{\isacharunderscore}{\kern0pt}map{\isacharunderscore}{\kern0pt}eq{\isacharunderscore}{\kern0pt}sum{\isacharunderscore}{\kern0pt}count} but it doesn't work
if the function maps into the reals.%
\end{isamarkuptext}\isamarkuptrue%
\isacommand{lemma}\isamarkupfalse%
\ sum{\isacharunderscore}{\kern0pt}list{\isacharunderscore}{\kern0pt}eval{\isacharcolon}{\kern0pt}\isanewline
\ \ \isakeyword{fixes}\ f\ {\isacharcolon}{\kern0pt}{\isacharcolon}{\kern0pt}\ {\isachardoublequoteopen}{\isacharprime}{\kern0pt}a\ {\isasymRightarrow}\ {\isacharprime}{\kern0pt}b{\isacharcolon}{\kern0pt}{\isacharcolon}{\kern0pt}{\isacharbraceleft}{\kern0pt}ring{\isacharcomma}{\kern0pt}semiring{\isacharunderscore}{\kern0pt}{\isadigit{1}}{\isacharbraceright}{\kern0pt}{\isachardoublequoteclose}\isanewline
\ \ \isakeyword{shows}\ {\isachardoublequoteopen}sum{\isacharunderscore}{\kern0pt}list\ {\isacharparenleft}{\kern0pt}map\ f\ xs{\isacharparenright}{\kern0pt}\ {\isacharequal}{\kern0pt}\ {\isacharparenleft}{\kern0pt}{\isasymSum}x\ {\isasymin}\ set\ xs{\isachardot}{\kern0pt}\ of{\isacharunderscore}{\kern0pt}nat\ {\isacharparenleft}{\kern0pt}count{\isacharunderscore}{\kern0pt}list\ xs\ x{\isacharparenright}{\kern0pt}\ {\isacharasterisk}{\kern0pt}\ f\ x{\isacharparenright}{\kern0pt}{\isachardoublequoteclose}\isanewline
%
\isadelimproof
%
\endisadelimproof
%
\isatagproof
\isacommand{proof}\isamarkupfalse%
\ {\isacharminus}{\kern0pt}\isanewline
\ \ \isacommand{define}\isamarkupfalse%
\ M\ \isakeyword{where}\ {\isachardoublequoteopen}M\ {\isacharequal}{\kern0pt}\ mset\ xs{\isachardoublequoteclose}\isanewline
\ \ \isacommand{have}\isamarkupfalse%
\ {\isachardoublequoteopen}sum{\isacharunderscore}{\kern0pt}mset\ {\isacharparenleft}{\kern0pt}image{\isacharunderscore}{\kern0pt}mset\ f\ M{\isacharparenright}{\kern0pt}\ {\isacharequal}{\kern0pt}\ {\isacharparenleft}{\kern0pt}{\isasymSum}x\ {\isasymin}\ set{\isacharunderscore}{\kern0pt}mset\ M{\isachardot}{\kern0pt}\ of{\isacharunderscore}{\kern0pt}nat\ {\isacharparenleft}{\kern0pt}count\ M\ x{\isacharparenright}{\kern0pt}\ {\isacharasterisk}{\kern0pt}\ f\ x{\isacharparenright}{\kern0pt}{\isachardoublequoteclose}\isanewline
\ \ \isacommand{proof}\isamarkupfalse%
\ {\isacharparenleft}{\kern0pt}induction\ {\isachardoublequoteopen}M{\isachardoublequoteclose}\ rule{\isacharcolon}{\kern0pt}disj{\isacharunderscore}{\kern0pt}induct{\isacharunderscore}{\kern0pt}mset{\isacharparenright}{\kern0pt}\isanewline
\ \ \ \ \isacommand{case}\isamarkupfalse%
\ {\isadigit{1}}\isanewline
\ \ \ \ \isacommand{then}\isamarkupfalse%
\ \isacommand{show}\isamarkupfalse%
\ {\isacharquery}{\kern0pt}case\ \isacommand{by}\isamarkupfalse%
\ simp\isanewline
\ \ \isacommand{next}\isamarkupfalse%
\isanewline
\ \ \ \ \isacommand{case}\isamarkupfalse%
\ {\isacharparenleft}{\kern0pt}{\isadigit{2}}\ n\ M\ x{\isacharparenright}{\kern0pt}\isanewline
\ \ \ \ \isacommand{have}\isamarkupfalse%
\ a{\isacharcolon}{\kern0pt}{\isachardoublequoteopen}{\isasymAnd}y{\isachardot}{\kern0pt}\ y\ {\isasymin}\ set{\isacharunderscore}{\kern0pt}mset\ M\ {\isasymLongrightarrow}\ y\ {\isasymnoteq}\ x{\isachardoublequoteclose}\ \isacommand{using}\isamarkupfalse%
\ {\isadigit{2}}{\isacharparenleft}{\kern0pt}{\isadigit{2}}{\isacharparenright}{\kern0pt}\ \isacommand{by}\isamarkupfalse%
\ blast\isanewline
\ \ \ \ \isacommand{show}\isamarkupfalse%
\ {\isacharquery}{\kern0pt}case\ \isacommand{using}\isamarkupfalse%
\ {\isadigit{2}}\ \isacommand{by}\isamarkupfalse%
\ {\isacharparenleft}{\kern0pt}simp\ add{\isacharcolon}{\kern0pt}a\ \ count{\isacharunderscore}{\kern0pt}eq{\isacharunderscore}{\kern0pt}zero{\isacharunderscore}{\kern0pt}iff{\isacharbrackleft}{\kern0pt}symmetric{\isacharbrackright}{\kern0pt}{\isacharparenright}{\kern0pt}\isanewline
\ \ \isacommand{qed}\isamarkupfalse%
\isanewline
\ \ \isacommand{moreover}\isamarkupfalse%
\ \isacommand{have}\isamarkupfalse%
\ {\isachardoublequoteopen}{\isasymAnd}x{\isachardot}{\kern0pt}\ count{\isacharunderscore}{\kern0pt}list\ xs\ x\ {\isacharequal}{\kern0pt}\ count\ {\isacharparenleft}{\kern0pt}mset\ xs{\isacharparenright}{\kern0pt}\ x{\isachardoublequoteclose}\ \isanewline
\ \ \ \ \isacommand{by}\isamarkupfalse%
\ {\isacharparenleft}{\kern0pt}induction\ xs{\isacharcomma}{\kern0pt}\ simp{\isacharcomma}{\kern0pt}\ simp{\isacharparenright}{\kern0pt}\isanewline
\ \ \isacommand{ultimately}\isamarkupfalse%
\ \isacommand{show}\isamarkupfalse%
\ {\isacharquery}{\kern0pt}thesis\isanewline
\ \ \ \ \isacommand{by}\isamarkupfalse%
\ {\isacharparenleft}{\kern0pt}simp\ add{\isacharcolon}{\kern0pt}M{\isacharunderscore}{\kern0pt}def\ sum{\isacharunderscore}{\kern0pt}mset{\isacharunderscore}{\kern0pt}sum{\isacharunderscore}{\kern0pt}list{\isacharbrackleft}{\kern0pt}symmetric{\isacharbrackright}{\kern0pt}{\isacharparenright}{\kern0pt}\isanewline
\isacommand{qed}\isamarkupfalse%
%
\endisatagproof
{\isafoldproof}%
%
\isadelimproof
\isanewline
%
\endisadelimproof
\isanewline
\isacommand{lemma}\isamarkupfalse%
\ prod{\isacharunderscore}{\kern0pt}list{\isacharunderscore}{\kern0pt}eval{\isacharcolon}{\kern0pt}\isanewline
\ \ \isakeyword{fixes}\ f\ {\isacharcolon}{\kern0pt}{\isacharcolon}{\kern0pt}\ {\isachardoublequoteopen}{\isacharprime}{\kern0pt}a\ {\isasymRightarrow}\ {\isacharprime}{\kern0pt}b{\isacharcolon}{\kern0pt}{\isacharcolon}{\kern0pt}{\isacharbraceleft}{\kern0pt}ring{\isacharcomma}{\kern0pt}semiring{\isacharunderscore}{\kern0pt}{\isadigit{1}}{\isacharcomma}{\kern0pt}comm{\isacharunderscore}{\kern0pt}monoid{\isacharunderscore}{\kern0pt}mult{\isacharbraceright}{\kern0pt}{\isachardoublequoteclose}\isanewline
\ \ \isakeyword{shows}\ {\isachardoublequoteopen}prod{\isacharunderscore}{\kern0pt}list\ {\isacharparenleft}{\kern0pt}map\ f\ xs{\isacharparenright}{\kern0pt}\ {\isacharequal}{\kern0pt}\ {\isacharparenleft}{\kern0pt}{\isasymProd}x\ {\isasymin}\ set\ xs{\isachardot}{\kern0pt}\ {\isacharparenleft}{\kern0pt}f\ x{\isacharparenright}{\kern0pt}{\isacharcircum}{\kern0pt}{\isacharparenleft}{\kern0pt}count{\isacharunderscore}{\kern0pt}list\ xs\ x{\isacharparenright}{\kern0pt}{\isacharparenright}{\kern0pt}{\isachardoublequoteclose}\isanewline
%
\isadelimproof
%
\endisadelimproof
%
\isatagproof
\isacommand{proof}\isamarkupfalse%
\ {\isacharminus}{\kern0pt}\isanewline
\ \ \isacommand{define}\isamarkupfalse%
\ M\ \isakeyword{where}\ {\isachardoublequoteopen}M\ {\isacharequal}{\kern0pt}\ mset\ xs{\isachardoublequoteclose}\isanewline
\ \ \isacommand{have}\isamarkupfalse%
\ {\isachardoublequoteopen}prod{\isacharunderscore}{\kern0pt}mset\ {\isacharparenleft}{\kern0pt}image{\isacharunderscore}{\kern0pt}mset\ f\ M{\isacharparenright}{\kern0pt}\ {\isacharequal}{\kern0pt}\ {\isacharparenleft}{\kern0pt}{\isasymProd}x\ {\isasymin}\ set{\isacharunderscore}{\kern0pt}mset\ M{\isachardot}{\kern0pt}\ f\ x\ {\isacharcircum}{\kern0pt}\ {\isacharparenleft}{\kern0pt}count\ M\ x{\isacharparenright}{\kern0pt}{\isacharparenright}{\kern0pt}{\isachardoublequoteclose}\isanewline
\ \ \isacommand{proof}\isamarkupfalse%
\ {\isacharparenleft}{\kern0pt}induction\ {\isachardoublequoteopen}M{\isachardoublequoteclose}\ rule{\isacharcolon}{\kern0pt}disj{\isacharunderscore}{\kern0pt}induct{\isacharunderscore}{\kern0pt}mset{\isacharparenright}{\kern0pt}\isanewline
\ \ \ \ \isacommand{case}\isamarkupfalse%
\ {\isadigit{1}}\isanewline
\ \ \ \ \isacommand{then}\isamarkupfalse%
\ \isacommand{show}\isamarkupfalse%
\ {\isacharquery}{\kern0pt}case\ \isacommand{by}\isamarkupfalse%
\ simp\isanewline
\ \ \isacommand{next}\isamarkupfalse%
\isanewline
\ \ \ \ \isacommand{case}\isamarkupfalse%
\ {\isacharparenleft}{\kern0pt}{\isadigit{2}}\ n\ M\ x{\isacharparenright}{\kern0pt}\isanewline
\ \ \ \ \isacommand{have}\isamarkupfalse%
\ a{\isacharcolon}{\kern0pt}{\isachardoublequoteopen}{\isasymAnd}y{\isachardot}{\kern0pt}\ y\ {\isasymin}\ set{\isacharunderscore}{\kern0pt}mset\ M\ {\isasymLongrightarrow}\ y\ {\isasymnoteq}\ x{\isachardoublequoteclose}\ \isacommand{using}\isamarkupfalse%
\ {\isadigit{2}}{\isacharparenleft}{\kern0pt}{\isadigit{2}}{\isacharparenright}{\kern0pt}\ \isacommand{by}\isamarkupfalse%
\ blast\isanewline
\ \ \ \ \isacommand{have}\isamarkupfalse%
\ b{\isacharcolon}{\kern0pt}{\isachardoublequoteopen}count\ M\ x\ {\isacharequal}{\kern0pt}\ {\isadigit{0}}{\isachardoublequoteclose}\ \isacommand{apply}\isamarkupfalse%
\ {\isacharparenleft}{\kern0pt}subst\ \ count{\isacharunderscore}{\kern0pt}eq{\isacharunderscore}{\kern0pt}zero{\isacharunderscore}{\kern0pt}iff{\isacharparenright}{\kern0pt}\ \isacommand{using}\isamarkupfalse%
\ {\isadigit{2}}\ \isacommand{by}\isamarkupfalse%
\ blast\ \isanewline
\ \ \ \ \isacommand{show}\isamarkupfalse%
\ {\isacharquery}{\kern0pt}case\ \isacommand{using}\isamarkupfalse%
\ {\isadigit{2}}\ \ \isacommand{by}\isamarkupfalse%
\ {\isacharparenleft}{\kern0pt}simp\ add{\isacharcolon}{\kern0pt}a\ b\ mult{\isachardot}{\kern0pt}commute{\isacharparenright}{\kern0pt}\isanewline
\ \ \isacommand{qed}\isamarkupfalse%
\isanewline
\ \ \isacommand{moreover}\isamarkupfalse%
\ \isacommand{have}\isamarkupfalse%
\ {\isachardoublequoteopen}{\isasymAnd}x{\isachardot}{\kern0pt}\ count{\isacharunderscore}{\kern0pt}list\ xs\ x\ {\isacharequal}{\kern0pt}\ count\ {\isacharparenleft}{\kern0pt}mset\ xs{\isacharparenright}{\kern0pt}\ x{\isachardoublequoteclose}\ \isanewline
\ \ \ \ \isacommand{by}\isamarkupfalse%
\ {\isacharparenleft}{\kern0pt}induction\ xs{\isacharcomma}{\kern0pt}\ simp{\isacharcomma}{\kern0pt}\ simp{\isacharparenright}{\kern0pt}\isanewline
\ \ \isacommand{ultimately}\isamarkupfalse%
\ \isacommand{show}\isamarkupfalse%
\ {\isacharquery}{\kern0pt}thesis\isanewline
\ \ \ \ \isacommand{by}\isamarkupfalse%
\ {\isacharparenleft}{\kern0pt}simp\ add{\isacharcolon}{\kern0pt}M{\isacharunderscore}{\kern0pt}def\ prod{\isacharunderscore}{\kern0pt}mset{\isacharunderscore}{\kern0pt}prod{\isacharunderscore}{\kern0pt}list{\isacharbrackleft}{\kern0pt}symmetric{\isacharbrackright}{\kern0pt}{\isacharparenright}{\kern0pt}\isanewline
\isacommand{qed}\isamarkupfalse%
%
\endisatagproof
{\isafoldproof}%
%
\isadelimproof
\isanewline
%
\endisadelimproof
\isanewline
\isacommand{lemma}\isamarkupfalse%
\ sorted{\isacharunderscore}{\kern0pt}sorted{\isacharunderscore}{\kern0pt}list{\isacharunderscore}{\kern0pt}of{\isacharunderscore}{\kern0pt}multiset{\isacharcolon}{\kern0pt}\ {\isachardoublequoteopen}sorted\ {\isacharparenleft}{\kern0pt}sorted{\isacharunderscore}{\kern0pt}list{\isacharunderscore}{\kern0pt}of{\isacharunderscore}{\kern0pt}multiset\ M{\isacharparenright}{\kern0pt}{\isachardoublequoteclose}\isanewline
%
\isadelimproof
\ \ %
\endisadelimproof
%
\isatagproof
\isacommand{by}\isamarkupfalse%
\ {\isacharparenleft}{\kern0pt}induction\ M{\isacharcomma}{\kern0pt}\ simp{\isacharcomma}{\kern0pt}\ simp\ add{\isacharcolon}{\kern0pt}sorted{\isacharunderscore}{\kern0pt}insort{\isacharparenright}{\kern0pt}%
\endisatagproof
{\isafoldproof}%
%
\isadelimproof
\ \isanewline
%
\endisadelimproof
\isanewline
\isacommand{lemma}\isamarkupfalse%
\ count{\isacharunderscore}{\kern0pt}mset{\isacharcolon}{\kern0pt}\ {\isachardoublequoteopen}count\ {\isacharparenleft}{\kern0pt}mset\ xs{\isacharparenright}{\kern0pt}\ a\ {\isacharequal}{\kern0pt}\ count{\isacharunderscore}{\kern0pt}list\ xs\ a{\isachardoublequoteclose}\isanewline
%
\isadelimproof
\ \ %
\endisadelimproof
%
\isatagproof
\isacommand{by}\isamarkupfalse%
\ {\isacharparenleft}{\kern0pt}induction\ xs{\isacharcomma}{\kern0pt}\ simp{\isacharcomma}{\kern0pt}\ simp{\isacharparenright}{\kern0pt}%
\endisatagproof
{\isafoldproof}%
%
\isadelimproof
\isanewline
%
\endisadelimproof
\isanewline
\isacommand{lemma}\isamarkupfalse%
\ swap{\isacharunderscore}{\kern0pt}filter{\isacharunderscore}{\kern0pt}image{\isacharcolon}{\kern0pt}\ {\isachardoublequoteopen}filter{\isacharunderscore}{\kern0pt}mset\ g\ {\isacharparenleft}{\kern0pt}image{\isacharunderscore}{\kern0pt}mset\ f\ A{\isacharparenright}{\kern0pt}\ {\isacharequal}{\kern0pt}\ image{\isacharunderscore}{\kern0pt}mset\ f\ {\isacharparenleft}{\kern0pt}filter{\isacharunderscore}{\kern0pt}mset\ {\isacharparenleft}{\kern0pt}g\ {\isasymcirc}\ f{\isacharparenright}{\kern0pt}\ A{\isacharparenright}{\kern0pt}{\isachardoublequoteclose}\isanewline
%
\isadelimproof
\ \ %
\endisadelimproof
%
\isatagproof
\isacommand{by}\isamarkupfalse%
\ {\isacharparenleft}{\kern0pt}induction\ A{\isacharcomma}{\kern0pt}\ simp{\isacharcomma}{\kern0pt}\ simp{\isacharparenright}{\kern0pt}%
\endisatagproof
{\isafoldproof}%
%
\isadelimproof
\isanewline
%
\endisadelimproof
\isanewline
\isacommand{lemma}\isamarkupfalse%
\ list{\isacharunderscore}{\kern0pt}eq{\isacharunderscore}{\kern0pt}iff{\isacharcolon}{\kern0pt}\isanewline
\ \ \isakeyword{assumes}\ {\isachardoublequoteopen}mset\ xs\ {\isacharequal}{\kern0pt}\ mset\ ys{\isachardoublequoteclose}\isanewline
\ \ \isakeyword{assumes}\ {\isachardoublequoteopen}sorted\ xs{\isachardoublequoteclose}\isanewline
\ \ \isakeyword{assumes}\ {\isachardoublequoteopen}sorted\ ys{\isachardoublequoteclose}\isanewline
\ \ \isakeyword{shows}\ {\isachardoublequoteopen}xs\ {\isacharequal}{\kern0pt}\ ys{\isachardoublequoteclose}\ \isanewline
%
\isadelimproof
\ \ %
\endisadelimproof
%
\isatagproof
\isacommand{using}\isamarkupfalse%
\ assms\ properties{\isacharunderscore}{\kern0pt}for{\isacharunderscore}{\kern0pt}sort\ \isacommand{by}\isamarkupfalse%
\ blast%
\endisatagproof
{\isafoldproof}%
%
\isadelimproof
\isanewline
%
\endisadelimproof
\isanewline
\isacommand{lemma}\isamarkupfalse%
\ sorted{\isacharunderscore}{\kern0pt}list{\isacharunderscore}{\kern0pt}of{\isacharunderscore}{\kern0pt}multiset{\isacharunderscore}{\kern0pt}image{\isacharunderscore}{\kern0pt}commute{\isacharcolon}{\kern0pt}\isanewline
\ \ \isakeyword{assumes}\ {\isachardoublequoteopen}mono\ f{\isachardoublequoteclose}\isanewline
\ \ \isakeyword{shows}\ {\isachardoublequoteopen}sorted{\isacharunderscore}{\kern0pt}list{\isacharunderscore}{\kern0pt}of{\isacharunderscore}{\kern0pt}multiset\ {\isacharparenleft}{\kern0pt}image{\isacharunderscore}{\kern0pt}mset\ f\ M{\isacharparenright}{\kern0pt}\ {\isacharequal}{\kern0pt}\ map\ f\ {\isacharparenleft}{\kern0pt}sorted{\isacharunderscore}{\kern0pt}list{\isacharunderscore}{\kern0pt}of{\isacharunderscore}{\kern0pt}multiset\ M{\isacharparenright}{\kern0pt}{\isachardoublequoteclose}\ {\isacharparenleft}{\kern0pt}\isakeyword{is}\ {\isachardoublequoteopen}{\isacharquery}{\kern0pt}A\ {\isacharequal}{\kern0pt}\ {\isacharquery}{\kern0pt}B{\isachardoublequoteclose}{\isacharparenright}{\kern0pt}\isanewline
%
\isadelimproof
\ \ %
\endisadelimproof
%
\isatagproof
\isacommand{apply}\isamarkupfalse%
\ {\isacharparenleft}{\kern0pt}rule\ list{\isacharunderscore}{\kern0pt}eq{\isacharunderscore}{\kern0pt}iff{\isacharcomma}{\kern0pt}\ simp{\isacharparenright}{\kern0pt}\isanewline
\ \ \ \isacommand{apply}\isamarkupfalse%
\ {\isacharparenleft}{\kern0pt}simp\ add{\isacharcolon}{\kern0pt}sorted{\isacharunderscore}{\kern0pt}sorted{\isacharunderscore}{\kern0pt}list{\isacharunderscore}{\kern0pt}of{\isacharunderscore}{\kern0pt}multiset{\isacharparenright}{\kern0pt}\isanewline
\ \ \isacommand{apply}\isamarkupfalse%
\ {\isacharparenleft}{\kern0pt}subst\ sorted{\isacharunderscore}{\kern0pt}wrt{\isacharunderscore}{\kern0pt}map{\isacharparenright}{\kern0pt}\isanewline
\ \ \isacommand{by}\isamarkupfalse%
\ {\isacharparenleft}{\kern0pt}metis\ {\isacharparenleft}{\kern0pt}no{\isacharunderscore}{\kern0pt}types{\isacharcomma}{\kern0pt}\ lifting{\isacharparenright}{\kern0pt}\ monoE\ sorted{\isacharunderscore}{\kern0pt}sorted{\isacharunderscore}{\kern0pt}list{\isacharunderscore}{\kern0pt}of{\isacharunderscore}{\kern0pt}multiset\ sorted{\isacharunderscore}{\kern0pt}wrt{\isacharunderscore}{\kern0pt}mono{\isacharunderscore}{\kern0pt}rel\ assms{\isacharparenright}{\kern0pt}%
\endisatagproof
{\isafoldproof}%
%
\isadelimproof
\isanewline
%
\endisadelimproof
%
\isadelimtheory
\isanewline
%
\endisadelimtheory
%
\isatagtheory
\isacommand{end}\isamarkupfalse%
%
\endisatagtheory
{\isafoldtheory}%
%
\isadelimtheory
%
\endisadelimtheory
%
\end{isabellebody}%
\endinput
%:%file=Multiset_Ext.tex%:%
%:%11=1%:%
%:%27=3%:%
%:%28=3%:%
%:%29=4%:%
%:%30=5%:%
%:%39=7%:%
%:%43=9%:%
%:%44=10%:%
%:%45=11%:%
%:%46=12%:%
%:%48=14%:%
%:%49=14%:%
%:%50=15%:%
%:%51=16%:%
%:%52=17%:%
%:%59=18%:%
%:%60=18%:%
%:%61=19%:%
%:%62=19%:%
%:%63=20%:%
%:%64=20%:%
%:%65=21%:%
%:%66=21%:%
%:%67=22%:%
%:%68=22%:%
%:%69=23%:%
%:%70=23%:%
%:%71=23%:%
%:%72=23%:%
%:%73=23%:%
%:%74=24%:%
%:%75=24%:%
%:%76=25%:%
%:%77=25%:%
%:%78=26%:%
%:%79=26%:%
%:%80=26%:%
%:%81=26%:%
%:%82=26%:%
%:%83=27%:%
%:%84=27%:%
%:%85=28%:%
%:%86=28%:%
%:%87=28%:%
%:%88=29%:%
%:%89=29%:%
%:%90=30%:%
%:%91=30%:%
%:%92=31%:%
%:%93=31%:%
%:%94=32%:%
%:%95=32%:%
%:%96=32%:%
%:%97=32%:%
%:%98=33%:%
%:%99=33%:%
%:%100=33%:%
%:%101=34%:%
%:%102=34%:%
%:%103=35%:%
%:%104=35%:%
%:%105=36%:%
%:%106=36%:%
%:%107=37%:%
%:%113=37%:%
%:%116=38%:%
%:%117=39%:%
%:%118=39%:%
%:%119=40%:%
%:%120=41%:%
%:%127=42%:%
%:%128=42%:%
%:%129=43%:%
%:%130=43%:%
%:%131=44%:%
%:%132=44%:%
%:%133=44%:%
%:%134=44%:%
%:%135=45%:%
%:%136=45%:%
%:%137=46%:%
%:%138=46%:%
%:%139=47%:%
%:%140=47%:%
%:%141=47%:%
%:%142=47%:%
%:%143=47%:%
%:%144=48%:%
%:%145=48%:%
%:%146=48%:%
%:%147=48%:%
%:%148=48%:%
%:%149=49%:%
%:%150=49%:%
%:%151=49%:%
%:%152=49%:%
%:%153=50%:%
%:%159=50%:%
%:%162=51%:%
%:%163=52%:%
%:%164=52%:%
%:%165=53%:%
%:%166=54%:%
%:%167=55%:%
%:%168=56%:%
%:%169=57%:%
%:%172=58%:%
%:%176=58%:%
%:%177=58%:%
%:%178=59%:%
%:%179=59%:%
%:%188=61%:%
%:%189=62%:%
%:%191=64%:%
%:%192=64%:%
%:%193=65%:%
%:%194=66%:%
%:%201=67%:%
%:%202=67%:%
%:%203=68%:%
%:%204=68%:%
%:%205=69%:%
%:%206=69%:%
%:%207=70%:%
%:%208=70%:%
%:%209=71%:%
%:%210=71%:%
%:%211=72%:%
%:%212=72%:%
%:%213=72%:%
%:%214=72%:%
%:%215=73%:%
%:%216=73%:%
%:%217=74%:%
%:%218=74%:%
%:%219=75%:%
%:%220=75%:%
%:%221=75%:%
%:%222=75%:%
%:%223=76%:%
%:%224=76%:%
%:%225=76%:%
%:%226=76%:%
%:%227=77%:%
%:%228=77%:%
%:%229=78%:%
%:%230=78%:%
%:%231=78%:%
%:%232=79%:%
%:%233=79%:%
%:%234=80%:%
%:%235=80%:%
%:%236=80%:%
%:%237=81%:%
%:%238=81%:%
%:%239=82%:%
%:%245=82%:%
%:%248=83%:%
%:%249=84%:%
%:%250=84%:%
%:%251=85%:%
%:%252=86%:%
%:%259=87%:%
%:%260=87%:%
%:%261=88%:%
%:%262=88%:%
%:%263=89%:%
%:%264=89%:%
%:%265=90%:%
%:%266=90%:%
%:%267=91%:%
%:%268=91%:%
%:%269=92%:%
%:%270=92%:%
%:%271=92%:%
%:%272=92%:%
%:%273=93%:%
%:%274=93%:%
%:%275=94%:%
%:%276=94%:%
%:%277=95%:%
%:%278=95%:%
%:%279=95%:%
%:%280=95%:%
%:%281=96%:%
%:%282=96%:%
%:%283=96%:%
%:%284=96%:%
%:%285=96%:%
%:%286=97%:%
%:%287=97%:%
%:%288=97%:%
%:%289=97%:%
%:%290=98%:%
%:%291=98%:%
%:%292=99%:%
%:%293=99%:%
%:%294=99%:%
%:%295=100%:%
%:%296=100%:%
%:%297=101%:%
%:%298=101%:%
%:%299=101%:%
%:%300=102%:%
%:%301=102%:%
%:%302=103%:%
%:%308=103%:%
%:%311=104%:%
%:%312=105%:%
%:%313=105%:%
%:%316=106%:%
%:%320=106%:%
%:%321=106%:%
%:%326=106%:%
%:%329=107%:%
%:%330=108%:%
%:%331=108%:%
%:%334=109%:%
%:%338=109%:%
%:%339=109%:%
%:%344=109%:%
%:%347=110%:%
%:%348=111%:%
%:%349=111%:%
%:%352=112%:%
%:%356=112%:%
%:%357=112%:%
%:%362=112%:%
%:%365=113%:%
%:%366=114%:%
%:%367=114%:%
%:%368=115%:%
%:%369=116%:%
%:%370=117%:%
%:%371=118%:%
%:%374=119%:%
%:%378=119%:%
%:%379=119%:%
%:%380=119%:%
%:%385=119%:%
%:%388=120%:%
%:%389=121%:%
%:%390=121%:%
%:%391=122%:%
%:%392=123%:%
%:%395=124%:%
%:%399=124%:%
%:%400=124%:%
%:%401=125%:%
%:%402=125%:%
%:%403=126%:%
%:%404=126%:%
%:%405=127%:%
%:%406=127%:%
%:%411=127%:%
%:%416=128%:%
%:%421=129%:%

%
\begin{isabellebody}%
\setisabellecontext{Probability{\isacharunderscore}{\kern0pt}Ext}%
%
\isadelimdocument
%
\endisadelimdocument
%
\isatagdocument
%
\isamarkupsection{Probability Spaces%
}
\isamarkuptrue%
%
\endisatagdocument
{\isafolddocument}%
%
\isadelimdocument
%
\endisadelimdocument
%
\begin{isamarkuptext}%
Some additional results about probability spaces in addition to "HOL-Probability".%
\end{isamarkuptext}\isamarkuptrue%
%
\isadelimtheory
%
\endisadelimtheory
%
\isatagtheory
\isacommand{theory}\isamarkupfalse%
\ Probability{\isacharunderscore}{\kern0pt}Ext\isanewline
\ \ \isakeyword{imports}\ Main\ {\isachardoublequoteopen}HOL{\isacharminus}{\kern0pt}Probability{\isachardot}{\kern0pt}Independent{\isacharunderscore}{\kern0pt}Family{\isachardoublequoteclose}\ Multiset{\isacharunderscore}{\kern0pt}Ext\ {\isachardoublequoteopen}HOL{\isacharminus}{\kern0pt}Probability{\isachardot}{\kern0pt}Stream{\isacharunderscore}{\kern0pt}Space{\isachardoublequoteclose}\isanewline
\ {\isachardoublequoteopen}HOL{\isacharminus}{\kern0pt}Probability{\isachardot}{\kern0pt}Probability{\isacharunderscore}{\kern0pt}Mass{\isacharunderscore}{\kern0pt}Function{\isachardoublequoteclose}\isanewline
\isakeyword{begin}%
\endisatagtheory
{\isafoldtheory}%
%
\isadelimtheory
%
\endisadelimtheory
\isanewline
\isanewline
\isacommand{lemma}\isamarkupfalse%
\ measure{\isacharunderscore}{\kern0pt}inters{\isacharcolon}{\kern0pt}\ {\isachardoublequoteopen}measure\ M\ {\isacharparenleft}{\kern0pt}E\ {\isasyminter}\ space\ M{\isacharparenright}{\kern0pt}\ {\isacharequal}{\kern0pt}\ {\isasymP}{\isacharparenleft}{\kern0pt}x\ in\ M{\isachardot}{\kern0pt}\ x\ {\isasymin}\ E{\isacharparenright}{\kern0pt}{\isachardoublequoteclose}\isanewline
%
\isadelimproof
\ \ %
\endisadelimproof
%
\isatagproof
\isacommand{by}\isamarkupfalse%
\ {\isacharparenleft}{\kern0pt}simp\ add{\isacharcolon}{\kern0pt}\ Collect{\isacharunderscore}{\kern0pt}conj{\isacharunderscore}{\kern0pt}eq\ inf{\isacharunderscore}{\kern0pt}commute{\isacharparenright}{\kern0pt}%
\endisatagproof
{\isafoldproof}%
%
\isadelimproof
\isanewline
%
\endisadelimproof
\isanewline
\isacommand{lemma}\isamarkupfalse%
\ set{\isacharunderscore}{\kern0pt}comp{\isacharunderscore}{\kern0pt}subsetI{\isacharcolon}{\kern0pt}\ {\isachardoublequoteopen}{\isacharparenleft}{\kern0pt}{\isasymAnd}x{\isachardot}{\kern0pt}\ P\ x\ {\isasymLongrightarrow}\ f\ x\ {\isasymin}\ B{\isacharparenright}{\kern0pt}\ {\isasymLongrightarrow}\ {\isacharbraceleft}{\kern0pt}f\ x{\isacharbar}{\kern0pt}x{\isachardot}{\kern0pt}\ P\ x{\isacharbraceright}{\kern0pt}\ {\isasymsubseteq}\ B{\isachardoublequoteclose}\isanewline
%
\isadelimproof
\ \ %
\endisadelimproof
%
\isatagproof
\isacommand{by}\isamarkupfalse%
\ blast%
\endisatagproof
{\isafoldproof}%
%
\isadelimproof
\isanewline
%
\endisadelimproof
\isanewline
\isacommand{lemma}\isamarkupfalse%
\ set{\isacharunderscore}{\kern0pt}comp{\isacharunderscore}{\kern0pt}cong{\isacharcolon}{\kern0pt}\ \isanewline
\ \ \isakeyword{assumes}\ {\isachardoublequoteopen}{\isasymAnd}x{\isachardot}{\kern0pt}\ P\ x\ {\isasymLongrightarrow}\ f\ x\ {\isacharequal}{\kern0pt}\ h\ {\isacharparenleft}{\kern0pt}g\ x{\isacharparenright}{\kern0pt}{\isachardoublequoteclose}\isanewline
\ \ \isakeyword{shows}\ {\isachardoublequoteopen}{\isacharbraceleft}{\kern0pt}f\ x{\isacharbar}{\kern0pt}\ x{\isachardot}{\kern0pt}\ P\ x{\isacharbraceright}{\kern0pt}\ {\isacharequal}{\kern0pt}\ h\ {\isacharbackquote}{\kern0pt}\ {\isacharbraceleft}{\kern0pt}g\ x{\isacharbar}{\kern0pt}\ x{\isachardot}{\kern0pt}\ P\ x{\isacharbraceright}{\kern0pt}{\isachardoublequoteclose}\isanewline
%
\isadelimproof
\ \ %
\endisadelimproof
%
\isatagproof
\isacommand{using}\isamarkupfalse%
\ assms\ \isacommand{by}\isamarkupfalse%
\ {\isacharparenleft}{\kern0pt}simp\ add{\isacharcolon}{\kern0pt}\ setcompr{\isacharunderscore}{\kern0pt}eq{\isacharunderscore}{\kern0pt}image{\isacharcomma}{\kern0pt}\ auto{\isacharparenright}{\kern0pt}%
\endisatagproof
{\isafoldproof}%
%
\isadelimproof
\isanewline
%
\endisadelimproof
\isanewline
\isacommand{lemma}\isamarkupfalse%
\ indep{\isacharunderscore}{\kern0pt}sets{\isacharunderscore}{\kern0pt}distr{\isacharcolon}{\kern0pt}\isanewline
\ \ \isakeyword{assumes}\ {\isachardoublequoteopen}f\ {\isasymin}\ measurable\ M\ N{\isachardoublequoteclose}\isanewline
\ \ \isakeyword{assumes}\ {\isachardoublequoteopen}prob{\isacharunderscore}{\kern0pt}space\ M{\isachardoublequoteclose}\isanewline
\ \ \isakeyword{assumes}\ {\isachardoublequoteopen}prob{\isacharunderscore}{\kern0pt}space{\isachardot}{\kern0pt}indep{\isacharunderscore}{\kern0pt}sets\ M\ {\isacharparenleft}{\kern0pt}{\isasymlambda}i{\isachardot}{\kern0pt}\ {\isacharparenleft}{\kern0pt}{\isasymlambda}a{\isachardot}{\kern0pt}\ f\ {\isacharminus}{\kern0pt}{\isacharbackquote}{\kern0pt}\ a\ {\isasyminter}\ space\ M{\isacharparenright}{\kern0pt}\ {\isacharbackquote}{\kern0pt}\ A\ i{\isacharparenright}{\kern0pt}\ I{\isachardoublequoteclose}\isanewline
\ \ \isakeyword{assumes}\ {\isachardoublequoteopen}{\isasymAnd}i{\isachardot}{\kern0pt}\ i\ {\isasymin}\ I\ {\isasymLongrightarrow}\ A\ i\ {\isasymsubseteq}\ sets\ N{\isachardoublequoteclose}\isanewline
\ \ \isakeyword{shows}\ {\isachardoublequoteopen}prob{\isacharunderscore}{\kern0pt}space{\isachardot}{\kern0pt}indep{\isacharunderscore}{\kern0pt}sets\ {\isacharparenleft}{\kern0pt}distr\ M\ N\ f{\isacharparenright}{\kern0pt}\ A\ I{\isachardoublequoteclose}\isanewline
%
\isadelimproof
%
\endisadelimproof
%
\isatagproof
\isacommand{proof}\isamarkupfalse%
\ {\isacharminus}{\kern0pt}\isanewline
\ \ \isacommand{define}\isamarkupfalse%
\ F\ \isakeyword{where}\ {\isachardoublequoteopen}F\ {\isacharequal}{\kern0pt}\ {\isacharparenleft}{\kern0pt}{\isasymlambda}i{\isachardot}{\kern0pt}\ {\isacharparenleft}{\kern0pt}{\isasymlambda}a{\isachardot}{\kern0pt}\ f\ {\isacharminus}{\kern0pt}{\isacharbackquote}{\kern0pt}\ a\ {\isasyminter}\ space\ M{\isacharparenright}{\kern0pt}\ {\isacharbackquote}{\kern0pt}\ A\ i{\isacharparenright}{\kern0pt}{\isachardoublequoteclose}\isanewline
\ \ \isacommand{have}\isamarkupfalse%
\ indep{\isacharunderscore}{\kern0pt}F{\isacharcolon}{\kern0pt}\ {\isachardoublequoteopen}prob{\isacharunderscore}{\kern0pt}space{\isachardot}{\kern0pt}indep{\isacharunderscore}{\kern0pt}sets\ M\ F\ I{\isachardoublequoteclose}\isanewline
\ \ \ \ \isacommand{using}\isamarkupfalse%
\ F{\isacharunderscore}{\kern0pt}def\ assms{\isacharparenleft}{\kern0pt}{\isadigit{3}}{\isacharparenright}{\kern0pt}\ \isacommand{by}\isamarkupfalse%
\ simp\isanewline
\isanewline
\ \ \isacommand{have}\isamarkupfalse%
\ sets{\isacharunderscore}{\kern0pt}A{\isacharcolon}{\kern0pt}\ {\isachardoublequoteopen}{\isasymAnd}i{\isachardot}{\kern0pt}\ i\ {\isasymin}\ I\ {\isasymLongrightarrow}\ A\ i\ {\isasymsubseteq}\ sets\ N{\isachardoublequoteclose}\isanewline
\ \ \ \ \isacommand{using}\isamarkupfalse%
\ assms{\isacharparenleft}{\kern0pt}{\isadigit{4}}{\isacharparenright}{\kern0pt}\ \isacommand{by}\isamarkupfalse%
\ blast\isanewline
\isanewline
\ \ \isacommand{have}\isamarkupfalse%
\ indep{\isacharunderscore}{\kern0pt}A{\isacharcolon}{\kern0pt}\ {\isachardoublequoteopen}{\isasymAnd}A{\isacharprime}{\kern0pt}\ J{\isachardot}{\kern0pt}\ J\ {\isasymnoteq}\ {\isacharbraceleft}{\kern0pt}{\isacharbraceright}{\kern0pt}\ {\isasymLongrightarrow}\ J\ {\isasymsubseteq}\ I\ {\isasymLongrightarrow}\ finite\ J\ {\isasymLongrightarrow}\ \isanewline
\ \ {\isasymforall}j{\isasymin}J{\isachardot}{\kern0pt}\ A{\isacharprime}{\kern0pt}\ j\ {\isasymin}\ A\ j\ {\isasymLongrightarrow}\ measure\ {\isacharparenleft}{\kern0pt}distr\ M\ N\ f{\isacharparenright}{\kern0pt}\ {\isacharparenleft}{\kern0pt}{\isasymInter}\ {\isacharparenleft}{\kern0pt}A{\isacharprime}{\kern0pt}\ {\isacharbackquote}{\kern0pt}\ J{\isacharparenright}{\kern0pt}{\isacharparenright}{\kern0pt}\ {\isacharequal}{\kern0pt}\ {\isacharparenleft}{\kern0pt}{\isasymProd}j{\isasymin}J{\isachardot}{\kern0pt}\ measure\ {\isacharparenleft}{\kern0pt}distr\ M\ N\ f{\isacharparenright}{\kern0pt}\ {\isacharparenleft}{\kern0pt}A{\isacharprime}{\kern0pt}\ j{\isacharparenright}{\kern0pt}{\isacharparenright}{\kern0pt}{\isachardoublequoteclose}\isanewline
\ \ \isacommand{proof}\isamarkupfalse%
\ {\isacharminus}{\kern0pt}\isanewline
\ \ \ \ \isacommand{fix}\isamarkupfalse%
\ A{\isacharprime}{\kern0pt}\ J\isanewline
\ \ \ \ \isacommand{assume}\isamarkupfalse%
\ a{\isadigit{1}}{\isacharcolon}{\kern0pt}{\isachardoublequoteopen}J\ {\isasymsubseteq}\ I{\isachardoublequoteclose}\isanewline
\ \ \ \ \isacommand{assume}\isamarkupfalse%
\ a{\isadigit{2}}{\isacharcolon}{\kern0pt}{\isachardoublequoteopen}finite\ J{\isachardoublequoteclose}\isanewline
\ \ \ \ \isacommand{assume}\isamarkupfalse%
\ a{\isadigit{3}}{\isacharcolon}{\kern0pt}{\isachardoublequoteopen}J\ {\isasymnoteq}\ {\isacharbraceleft}{\kern0pt}{\isacharbraceright}{\kern0pt}{\isachardoublequoteclose}\isanewline
\ \ \ \ \isacommand{assume}\isamarkupfalse%
\ a{\isadigit{4}}{\isacharcolon}{\kern0pt}{\isachardoublequoteopen}{\isasymforall}j\ {\isasymin}\ J{\isachardot}{\kern0pt}\ A{\isacharprime}{\kern0pt}\ j\ {\isasymin}\ A\ j{\isachardoublequoteclose}\isanewline
\isanewline
\ \ \ \ \isacommand{define}\isamarkupfalse%
\ F{\isacharprime}{\kern0pt}\ \isakeyword{where}\ {\isachardoublequoteopen}F{\isacharprime}{\kern0pt}\ {\isacharequal}{\kern0pt}\ {\isacharparenleft}{\kern0pt}{\isasymlambda}i{\isachardot}{\kern0pt}\ f\ {\isacharminus}{\kern0pt}{\isacharbackquote}{\kern0pt}\ A{\isacharprime}{\kern0pt}\ i\ {\isasyminter}\ space\ M{\isacharparenright}{\kern0pt}{\isachardoublequoteclose}\isanewline
\isanewline
\ \ \ \ \isacommand{have}\isamarkupfalse%
\ {\isachardoublequoteopen}{\isasymInter}\ {\isacharparenleft}{\kern0pt}F{\isacharprime}{\kern0pt}\ {\isacharbackquote}{\kern0pt}\ J{\isacharparenright}{\kern0pt}\ {\isacharequal}{\kern0pt}\ f\ {\isacharminus}{\kern0pt}{\isacharbackquote}{\kern0pt}\ {\isacharparenleft}{\kern0pt}{\isasymInter}\ {\isacharparenleft}{\kern0pt}A{\isacharprime}{\kern0pt}\ {\isacharbackquote}{\kern0pt}\ J{\isacharparenright}{\kern0pt}{\isacharparenright}{\kern0pt}\ {\isasyminter}\ space\ M{\isachardoublequoteclose}\ \isanewline
\ \ \ \ \ \ \isacommand{apply}\isamarkupfalse%
\ {\isacharparenleft}{\kern0pt}rule\ order{\isacharunderscore}{\kern0pt}antisym{\isacharparenright}{\kern0pt}\isanewline
\ \ \ \ \ \ \isacommand{apply}\isamarkupfalse%
\ {\isacharparenleft}{\kern0pt}rule\ subsetI{\isacharcomma}{\kern0pt}\ simp\ add{\isacharcolon}{\kern0pt}F{\isacharprime}{\kern0pt}{\isacharunderscore}{\kern0pt}def\ a{\isadigit{3}}{\isacharparenright}{\kern0pt}\isanewline
\ \ \ \ \ \ \isacommand{by}\isamarkupfalse%
\ {\isacharparenleft}{\kern0pt}rule\ subsetI{\isacharcomma}{\kern0pt}\ simp\ add{\isacharcolon}{\kern0pt}F{\isacharprime}{\kern0pt}{\isacharunderscore}{\kern0pt}def\ a{\isadigit{3}}{\isacharparenright}{\kern0pt}\isanewline
\ \ \ \ \isacommand{moreover}\isamarkupfalse%
\ \isacommand{have}\isamarkupfalse%
\ {\isachardoublequoteopen}{\isasymInter}\ {\isacharparenleft}{\kern0pt}A{\isacharprime}{\kern0pt}\ {\isacharbackquote}{\kern0pt}\ J{\isacharparenright}{\kern0pt}\ {\isasymin}\ sets\ N{\isachardoublequoteclose}\ \isanewline
\ \ \ \ \ \ \isacommand{using}\isamarkupfalse%
\ a{\isadigit{4}}\ a{\isadigit{1}}\ sets{\isacharunderscore}{\kern0pt}A\ \isanewline
\ \ \ \ \ \ \isacommand{by}\isamarkupfalse%
\ {\isacharparenleft}{\kern0pt}metis\ a{\isadigit{2}}\ a{\isadigit{3}}\ sets{\isachardot}{\kern0pt}finite{\isacharunderscore}{\kern0pt}INT\ subset{\isacharunderscore}{\kern0pt}iff{\isacharparenright}{\kern0pt}\isanewline
\ \ \ \ \isacommand{ultimately}\isamarkupfalse%
\ \isacommand{have}\isamarkupfalse%
\ r{\isadigit{1}}{\isacharcolon}{\kern0pt}\ {\isachardoublequoteopen}measure\ {\isacharparenleft}{\kern0pt}distr\ M\ N\ f{\isacharparenright}{\kern0pt}\ {\isacharparenleft}{\kern0pt}{\isasymInter}\ {\isacharparenleft}{\kern0pt}A{\isacharprime}{\kern0pt}\ {\isacharbackquote}{\kern0pt}\ J{\isacharparenright}{\kern0pt}{\isacharparenright}{\kern0pt}\ {\isacharequal}{\kern0pt}\ measure\ M\ {\isacharparenleft}{\kern0pt}{\isasymInter}\ {\isacharparenleft}{\kern0pt}F{\isacharprime}{\kern0pt}\ {\isacharbackquote}{\kern0pt}\ J{\isacharparenright}{\kern0pt}{\isacharparenright}{\kern0pt}{\isachardoublequoteclose}\ \isanewline
\ \ \ \ \ \ \isacommand{using}\isamarkupfalse%
\ assms{\isacharparenleft}{\kern0pt}{\isadigit{1}}{\isacharparenright}{\kern0pt}\ measure{\isacharunderscore}{\kern0pt}distr\ \isacommand{by}\isamarkupfalse%
\ metis\isanewline
\isanewline
\ \ \ \ \isacommand{have}\isamarkupfalse%
\ {\isachardoublequoteopen}{\isasymAnd}j{\isachardot}{\kern0pt}\ j\ {\isasymin}\ J\ {\isasymLongrightarrow}\ F{\isacharprime}{\kern0pt}\ j\ {\isasymin}\ F\ j{\isachardoublequoteclose}\isanewline
\ \ \ \ \ \ \isacommand{using}\isamarkupfalse%
\ a{\isadigit{4}}\ F{\isacharprime}{\kern0pt}{\isacharunderscore}{\kern0pt}def\ F{\isacharunderscore}{\kern0pt}def\ \isacommand{by}\isamarkupfalse%
\ blast\isanewline
\ \ \ \ \isacommand{hence}\isamarkupfalse%
\ r{\isadigit{2}}{\isacharcolon}{\kern0pt}{\isachardoublequoteopen}measure\ M\ {\isacharparenleft}{\kern0pt}{\isasymInter}\ {\isacharparenleft}{\kern0pt}F{\isacharprime}{\kern0pt}\ {\isacharbackquote}{\kern0pt}\ J{\isacharparenright}{\kern0pt}{\isacharparenright}{\kern0pt}\ {\isacharequal}{\kern0pt}\ {\isacharparenleft}{\kern0pt}{\isasymProd}j{\isasymin}\ J{\isachardot}{\kern0pt}\ measure\ M\ {\isacharparenleft}{\kern0pt}F{\isacharprime}{\kern0pt}\ j{\isacharparenright}{\kern0pt}{\isacharparenright}{\kern0pt}{\isachardoublequoteclose}\isanewline
\ \ \ \ \ \ \isacommand{using}\isamarkupfalse%
\ indep{\isacharunderscore}{\kern0pt}F\ prob{\isacharunderscore}{\kern0pt}space{\isachardot}{\kern0pt}indep{\isacharunderscore}{\kern0pt}setsD\ assms{\isacharparenleft}{\kern0pt}{\isadigit{2}}{\isacharparenright}{\kern0pt}\ a{\isadigit{1}}\ a{\isadigit{2}}\ a{\isadigit{3}}\ \isacommand{by}\isamarkupfalse%
\ metis\isanewline
\isanewline
\ \ \ \ \isacommand{have}\isamarkupfalse%
\ {\isachardoublequoteopen}{\isasymAnd}j{\isachardot}{\kern0pt}\ j\ {\isasymin}\ J\ {\isasymLongrightarrow}\ F{\isacharprime}{\kern0pt}\ j\ {\isacharequal}{\kern0pt}\ \ f\ {\isacharminus}{\kern0pt}{\isacharbackquote}{\kern0pt}\ A{\isacharprime}{\kern0pt}\ j\ \ {\isasyminter}\ space\ M{\isachardoublequoteclose}\ \isanewline
\ \ \ \ \ \ \isacommand{by}\isamarkupfalse%
\ {\isacharparenleft}{\kern0pt}simp\ add{\isacharcolon}{\kern0pt}F{\isacharprime}{\kern0pt}{\isacharunderscore}{\kern0pt}def{\isacharparenright}{\kern0pt}\isanewline
\ \ \ \ \isacommand{moreover}\isamarkupfalse%
\ \isacommand{have}\isamarkupfalse%
\ {\isachardoublequoteopen}{\isasymAnd}j{\isachardot}{\kern0pt}\ j\ {\isasymin}\ J\ {\isasymLongrightarrow}\ A{\isacharprime}{\kern0pt}\ j\ {\isasymin}\ sets\ N{\isachardoublequoteclose}\ \isanewline
\ \ \ \ \ \ \isacommand{using}\isamarkupfalse%
\ a{\isadigit{4}}\ a{\isadigit{1}}\ sets{\isacharunderscore}{\kern0pt}A\ \isacommand{by}\isamarkupfalse%
\ blast\isanewline
\ \ \ \ \isacommand{ultimately}\isamarkupfalse%
\ \isacommand{have}\isamarkupfalse%
\ r{\isadigit{3}}{\isacharcolon}{\kern0pt}{\isachardoublequoteopen}{\isasymAnd}j{\isachardot}{\kern0pt}\ j\ {\isasymin}\ J\ {\isasymLongrightarrow}\ measure\ M\ {\isacharparenleft}{\kern0pt}F{\isacharprime}{\kern0pt}\ j{\isacharparenright}{\kern0pt}\ {\isacharequal}{\kern0pt}\ measure\ {\isacharparenleft}{\kern0pt}distr\ M\ N\ f{\isacharparenright}{\kern0pt}\ {\isacharparenleft}{\kern0pt}A{\isacharprime}{\kern0pt}\ j{\isacharparenright}{\kern0pt}{\isachardoublequoteclose}\isanewline
\ \ \ \ \ \ \isacommand{using}\isamarkupfalse%
\ assms{\isacharparenleft}{\kern0pt}{\isadigit{1}}{\isacharparenright}{\kern0pt}\ measure{\isacharunderscore}{\kern0pt}distr\ \isacommand{by}\isamarkupfalse%
\ metis\isanewline
\isanewline
\ \ \ \ \isacommand{show}\isamarkupfalse%
\ {\isachardoublequoteopen}measure\ {\isacharparenleft}{\kern0pt}distr\ M\ N\ f{\isacharparenright}{\kern0pt}\ {\isacharparenleft}{\kern0pt}{\isasymInter}\ {\isacharparenleft}{\kern0pt}A{\isacharprime}{\kern0pt}\ {\isacharbackquote}{\kern0pt}\ J{\isacharparenright}{\kern0pt}{\isacharparenright}{\kern0pt}\ {\isacharequal}{\kern0pt}\ {\isacharparenleft}{\kern0pt}{\isasymProd}j{\isasymin}J{\isachardot}{\kern0pt}\ measure\ {\isacharparenleft}{\kern0pt}distr\ M\ N\ f{\isacharparenright}{\kern0pt}\ {\isacharparenleft}{\kern0pt}A{\isacharprime}{\kern0pt}\ j{\isacharparenright}{\kern0pt}{\isacharparenright}{\kern0pt}{\isachardoublequoteclose}\isanewline
\ \ \ \ \ \ \isacommand{using}\isamarkupfalse%
\ r{\isadigit{1}}\ r{\isadigit{2}}\ r{\isadigit{3}}\ \isacommand{by}\isamarkupfalse%
\ auto\isanewline
\ \ \isacommand{qed}\isamarkupfalse%
\isanewline
\isanewline
\ \ \isacommand{show}\isamarkupfalse%
\ {\isacharquery}{\kern0pt}thesis\ \isanewline
\ \ \ \ \isacommand{apply}\isamarkupfalse%
\ {\isacharparenleft}{\kern0pt}rule\ prob{\isacharunderscore}{\kern0pt}space{\isachardot}{\kern0pt}indep{\isacharunderscore}{\kern0pt}setsI{\isacharparenright}{\kern0pt}\isanewline
\ \ \ \ \isacommand{using}\isamarkupfalse%
\ assms\ \isacommand{apply}\isamarkupfalse%
\ {\isacharparenleft}{\kern0pt}simp\ add{\isacharcolon}{\kern0pt}prob{\isacharunderscore}{\kern0pt}space{\isachardot}{\kern0pt}prob{\isacharunderscore}{\kern0pt}space{\isacharunderscore}{\kern0pt}distr{\isacharparenright}{\kern0pt}\isanewline
\ \ \ \ \isacommand{apply}\isamarkupfalse%
\ {\isacharparenleft}{\kern0pt}simp\ add{\isacharcolon}{\kern0pt}sets{\isacharunderscore}{\kern0pt}A{\isacharparenright}{\kern0pt}\isanewline
\ \ \ \ \isacommand{using}\isamarkupfalse%
\ indep{\isacharunderscore}{\kern0pt}A\ \isacommand{by}\isamarkupfalse%
\ blast\isanewline
\isacommand{qed}\isamarkupfalse%
%
\endisatagproof
{\isafoldproof}%
%
\isadelimproof
\isanewline
%
\endisadelimproof
\isanewline
\isacommand{lemma}\isamarkupfalse%
\ indep{\isacharunderscore}{\kern0pt}vars{\isacharunderscore}{\kern0pt}distr{\isacharcolon}{\kern0pt}\isanewline
\ \ \isakeyword{assumes}\ {\isachardoublequoteopen}f\ {\isasymin}\ measurable\ M\ N{\isachardoublequoteclose}\isanewline
\ \ \isakeyword{assumes}\ {\isachardoublequoteopen}{\isasymAnd}i{\isachardot}{\kern0pt}\ i\ {\isasymin}\ I\ {\isasymLongrightarrow}\ X{\isacharprime}{\kern0pt}\ i\ {\isasymin}\ measurable\ N\ {\isacharparenleft}{\kern0pt}M{\isacharprime}{\kern0pt}\ i{\isacharparenright}{\kern0pt}{\isachardoublequoteclose}\isanewline
\ \ \isakeyword{assumes}\ {\isachardoublequoteopen}prob{\isacharunderscore}{\kern0pt}space{\isachardot}{\kern0pt}indep{\isacharunderscore}{\kern0pt}vars\ M\ M{\isacharprime}{\kern0pt}\ {\isacharparenleft}{\kern0pt}{\isasymlambda}i{\isachardot}{\kern0pt}\ {\isacharparenleft}{\kern0pt}X{\isacharprime}{\kern0pt}\ i{\isacharparenright}{\kern0pt}\ {\isasymcirc}\ f{\isacharparenright}{\kern0pt}\ I{\isachardoublequoteclose}\isanewline
\ \ \isakeyword{assumes}\ {\isachardoublequoteopen}prob{\isacharunderscore}{\kern0pt}space\ M{\isachardoublequoteclose}\isanewline
\ \ \isakeyword{shows}\ {\isachardoublequoteopen}prob{\isacharunderscore}{\kern0pt}space{\isachardot}{\kern0pt}indep{\isacharunderscore}{\kern0pt}vars\ {\isacharparenleft}{\kern0pt}distr\ M\ N\ f{\isacharparenright}{\kern0pt}\ M{\isacharprime}{\kern0pt}\ X{\isacharprime}{\kern0pt}\ I{\isachardoublequoteclose}\isanewline
%
\isadelimproof
%
\endisadelimproof
%
\isatagproof
\isacommand{proof}\isamarkupfalse%
\ {\isacharminus}{\kern0pt}\isanewline
\ \ \isacommand{have}\isamarkupfalse%
\ b{\isadigit{1}}{\isacharcolon}{\kern0pt}\ {\isachardoublequoteopen}f\ {\isasymin}\ space\ M\ {\isasymrightarrow}\ space\ N{\isachardoublequoteclose}\ \isacommand{using}\isamarkupfalse%
\ assms{\isacharparenleft}{\kern0pt}{\isadigit{1}}{\isacharparenright}{\kern0pt}\ \isacommand{by}\isamarkupfalse%
\ {\isacharparenleft}{\kern0pt}simp\ add{\isacharcolon}{\kern0pt}measurable{\isacharunderscore}{\kern0pt}def{\isacharparenright}{\kern0pt}\isanewline
\ \ \isacommand{have}\isamarkupfalse%
\ a{\isacharcolon}{\kern0pt}{\isachardoublequoteopen}{\isasymAnd}i{\isachardot}{\kern0pt}\ i\ {\isasymin}\ I\ {\isasymLongrightarrow}\ {\isacharbraceleft}{\kern0pt}{\isacharparenleft}{\kern0pt}X{\isacharprime}{\kern0pt}\ i\ {\isasymcirc}\ f{\isacharparenright}{\kern0pt}\ {\isacharminus}{\kern0pt}{\isacharbackquote}{\kern0pt}\ A\ {\isasyminter}\ space\ M\ {\isacharbar}{\kern0pt}A{\isachardot}{\kern0pt}\ A\ {\isasymin}\ sets\ {\isacharparenleft}{\kern0pt}M{\isacharprime}{\kern0pt}\ i{\isacharparenright}{\kern0pt}{\isacharbraceright}{\kern0pt}\ {\isacharequal}{\kern0pt}\ {\isacharparenleft}{\kern0pt}{\isasymlambda}a{\isachardot}{\kern0pt}\ f\ {\isacharminus}{\kern0pt}{\isacharbackquote}{\kern0pt}\ a\ {\isasyminter}\ space\ M{\isacharparenright}{\kern0pt}\ {\isacharbackquote}{\kern0pt}\ {\isacharbraceleft}{\kern0pt}X{\isacharprime}{\kern0pt}\ i\ {\isacharminus}{\kern0pt}{\isacharbackquote}{\kern0pt}\ A\ {\isasyminter}\ space\ N\ {\isacharbar}{\kern0pt}A{\isachardot}{\kern0pt}\ A\ {\isasymin}\ sets\ {\isacharparenleft}{\kern0pt}M{\isacharprime}{\kern0pt}\ i{\isacharparenright}{\kern0pt}{\isacharbraceright}{\kern0pt}{\isachardoublequoteclose}\isanewline
\ \ \ \ \isacommand{apply}\isamarkupfalse%
\ {\isacharparenleft}{\kern0pt}rule\ set{\isacharunderscore}{\kern0pt}comp{\isacharunderscore}{\kern0pt}cong{\isacharparenright}{\kern0pt}\isanewline
\ \ \ \ \isacommand{apply}\isamarkupfalse%
\ {\isacharparenleft}{\kern0pt}rule\ order{\isacharunderscore}{\kern0pt}antisym{\isacharcomma}{\kern0pt}\ rule\ subsetI{\isacharcomma}{\kern0pt}\ simp{\isacharparenright}{\kern0pt}\ \isacommand{using}\isamarkupfalse%
\ b{\isadigit{1}}\ \isacommand{apply}\isamarkupfalse%
\ fast\isanewline
\ \ \ \ \isacommand{by}\isamarkupfalse%
\ {\isacharparenleft}{\kern0pt}rule\ subsetI{\isacharcomma}{\kern0pt}\ simp{\isacharparenright}{\kern0pt}\ \isanewline
\ \ \isacommand{show}\isamarkupfalse%
\ {\isacharquery}{\kern0pt}thesis\ \isanewline
\ \ \isacommand{using}\isamarkupfalse%
\ assms\ \isacommand{apply}\isamarkupfalse%
\ {\isacharparenleft}{\kern0pt}simp\ add{\isacharcolon}{\kern0pt}prob{\isacharunderscore}{\kern0pt}space{\isachardot}{\kern0pt}indep{\isacharunderscore}{\kern0pt}vars{\isacharunderscore}{\kern0pt}def{\isadigit{2}}\ prob{\isacharunderscore}{\kern0pt}space{\isachardot}{\kern0pt}prob{\isacharunderscore}{\kern0pt}space{\isacharunderscore}{\kern0pt}distr{\isacharparenright}{\kern0pt}\isanewline
\ \ \ \isacommand{apply}\isamarkupfalse%
\ {\isacharparenleft}{\kern0pt}rule\ indep{\isacharunderscore}{\kern0pt}sets{\isacharunderscore}{\kern0pt}distr{\isacharparenright}{\kern0pt}\isanewline
\ \ \isacommand{apply}\isamarkupfalse%
\ {\isacharparenleft}{\kern0pt}simp\ add{\isacharcolon}{\kern0pt}a\ cong{\isacharcolon}{\kern0pt}prob{\isacharunderscore}{\kern0pt}space{\isachardot}{\kern0pt}indep{\isacharunderscore}{\kern0pt}sets{\isacharunderscore}{\kern0pt}cong{\isacharparenright}{\kern0pt}\isanewline
\ \ \isacommand{apply}\isamarkupfalse%
\ {\isacharparenleft}{\kern0pt}simp\ add{\isacharcolon}{\kern0pt}a\ cong{\isacharcolon}{\kern0pt}prob{\isacharunderscore}{\kern0pt}space{\isachardot}{\kern0pt}indep{\isacharunderscore}{\kern0pt}sets{\isacharunderscore}{\kern0pt}cong{\isacharparenright}{\kern0pt}\isanewline
\ \ \ \isacommand{apply}\isamarkupfalse%
\ {\isacharparenleft}{\kern0pt}simp\ add{\isacharcolon}{\kern0pt}a\ cong{\isacharcolon}{\kern0pt}prob{\isacharunderscore}{\kern0pt}space{\isachardot}{\kern0pt}indep{\isacharunderscore}{\kern0pt}sets{\isacharunderscore}{\kern0pt}cong{\isacharparenright}{\kern0pt}\isanewline
\ \ \isacommand{using}\isamarkupfalse%
\ assms{\isacharparenleft}{\kern0pt}{\isadigit{2}}{\isacharparenright}{\kern0pt}\ measurable{\isacharunderscore}{\kern0pt}sets\ \isacommand{by}\isamarkupfalse%
\ blast\isanewline
\isacommand{qed}\isamarkupfalse%
%
\endisatagproof
{\isafoldproof}%
%
\isadelimproof
%
\endisadelimproof
%
\begin{isamarkuptext}%
Random variables that depend on disjoint sets of the components of a product space are
independent.%
\end{isamarkuptext}\isamarkuptrue%
\isacommand{lemma}\isamarkupfalse%
\ make{\isacharunderscore}{\kern0pt}ext{\isacharcolon}{\kern0pt}\ \isanewline
\ \ \isakeyword{assumes}\ {\isachardoublequoteopen}{\isasymAnd}x{\isachardot}{\kern0pt}\ P\ x\ {\isacharequal}{\kern0pt}\ P\ {\isacharparenleft}{\kern0pt}restrict\ x\ I{\isacharparenright}{\kern0pt}{\isachardoublequoteclose}\ \isanewline
\ \ \isakeyword{shows}\ {\isachardoublequoteopen}{\isacharparenleft}{\kern0pt}{\isasymforall}x\ {\isasymin}\ Pi\ I\ A{\isachardot}{\kern0pt}\ P\ x{\isacharparenright}{\kern0pt}\ {\isacharequal}{\kern0pt}\ {\isacharparenleft}{\kern0pt}{\isasymforall}x\ {\isasymin}\ PiE\ I\ A{\isachardot}{\kern0pt}\ P\ x{\isacharparenright}{\kern0pt}{\isachardoublequoteclose}\isanewline
%
\isadelimproof
\ \ %
\endisadelimproof
%
\isatagproof
\isacommand{apply}\isamarkupfalse%
\ {\isacharparenleft}{\kern0pt}simp\ add{\isacharcolon}{\kern0pt}PiE{\isacharunderscore}{\kern0pt}def\ Pi{\isacharunderscore}{\kern0pt}def{\isacharparenright}{\kern0pt}\isanewline
\ \ \isacommand{apply}\isamarkupfalse%
\ {\isacharparenleft}{\kern0pt}rule\ order{\isacharunderscore}{\kern0pt}antisym{\isacharparenright}{\kern0pt}\isanewline
\ \ \ \isacommand{apply}\isamarkupfalse%
\ {\isacharparenleft}{\kern0pt}simp\ add{\isacharcolon}{\kern0pt}Pi{\isacharunderscore}{\kern0pt}def{\isacharparenright}{\kern0pt}\isanewline
\ \ \isacommand{using}\isamarkupfalse%
\ assms\ \isacommand{by}\isamarkupfalse%
\ fastforce%
\endisatagproof
{\isafoldproof}%
%
\isadelimproof
\isanewline
%
\endisadelimproof
\isanewline
\isacommand{lemma}\isamarkupfalse%
\ PiE{\isacharunderscore}{\kern0pt}reindex{\isacharcolon}{\kern0pt}\isanewline
\ \ \isakeyword{assumes}\ {\isachardoublequoteopen}inj{\isacharunderscore}{\kern0pt}on\ f\ I{\isachardoublequoteclose}\isanewline
\ \ \isakeyword{shows}\ {\isachardoublequoteopen}PiE\ I\ {\isacharparenleft}{\kern0pt}A\ {\isasymcirc}\ f{\isacharparenright}{\kern0pt}\ {\isacharequal}{\kern0pt}\ {\isacharparenleft}{\kern0pt}{\isasymlambda}a{\isachardot}{\kern0pt}\ restrict\ {\isacharparenleft}{\kern0pt}a\ {\isasymcirc}\ f{\isacharparenright}{\kern0pt}\ I{\isacharparenright}{\kern0pt}\ {\isacharbackquote}{\kern0pt}\ PiE\ {\isacharparenleft}{\kern0pt}f\ {\isacharbackquote}{\kern0pt}\ I{\isacharparenright}{\kern0pt}\ A{\isachardoublequoteclose}\ {\isacharparenleft}{\kern0pt}\isakeyword{is}\ {\isachardoublequoteopen}{\isacharquery}{\kern0pt}lhs\ {\isacharequal}{\kern0pt}\ {\isacharquery}{\kern0pt}f\ {\isacharbackquote}{\kern0pt}\ {\isacharquery}{\kern0pt}rhs{\isachardoublequoteclose}{\isacharparenright}{\kern0pt}\isanewline
%
\isadelimproof
%
\endisadelimproof
%
\isatagproof
\isacommand{proof}\isamarkupfalse%
\ {\isacharminus}{\kern0pt}\isanewline
\ \ \isacommand{have}\isamarkupfalse%
\ {\isachardoublequoteopen}{\isacharquery}{\kern0pt}lhs\ {\isasymsubseteq}\ {\isacharquery}{\kern0pt}f{\isacharbackquote}{\kern0pt}\ {\isacharquery}{\kern0pt}rhs{\isachardoublequoteclose}\isanewline
\ \ \isacommand{proof}\isamarkupfalse%
\ {\isacharparenleft}{\kern0pt}rule\ subsetI{\isacharparenright}{\kern0pt}\isanewline
\ \ \ \ \isacommand{fix}\isamarkupfalse%
\ x\isanewline
\ \ \ \ \isacommand{assume}\isamarkupfalse%
\ a{\isacharcolon}{\kern0pt}{\isachardoublequoteopen}x\ {\isasymin}\ Pi\isactrlsub E\ I\ {\isacharparenleft}{\kern0pt}A\ {\isasymcirc}\ f{\isacharparenright}{\kern0pt}{\isachardoublequoteclose}\isanewline
\ \ \ \ \isacommand{define}\isamarkupfalse%
\ y\ \isakeyword{where}\ y{\isacharunderscore}{\kern0pt}def{\isacharcolon}{\kern0pt}\ {\isachardoublequoteopen}y\ {\isacharequal}{\kern0pt}\ {\isacharparenleft}{\kern0pt}{\isasymlambda}k{\isachardot}{\kern0pt}\ if\ k\ {\isasymin}\ f\ {\isacharbackquote}{\kern0pt}\ I\ then\ x\ {\isacharparenleft}{\kern0pt}the{\isacharunderscore}{\kern0pt}inv{\isacharunderscore}{\kern0pt}into\ I\ f\ k{\isacharparenright}{\kern0pt}\ else\ undefined{\isacharparenright}{\kern0pt}{\isachardoublequoteclose}\isanewline
\ \ \ \ \isacommand{have}\isamarkupfalse%
\ b{\isacharcolon}{\kern0pt}{\isachardoublequoteopen}y\ {\isasymin}\ PiE\ {\isacharparenleft}{\kern0pt}f\ {\isacharbackquote}{\kern0pt}\ I{\isacharparenright}{\kern0pt}\ A{\isachardoublequoteclose}\ \isanewline
\ \ \ \ \ \ \isacommand{apply}\isamarkupfalse%
\ {\isacharparenleft}{\kern0pt}rule\ PiE{\isacharunderscore}{\kern0pt}I{\isacharparenright}{\kern0pt}\isanewline
\ \ \ \ \ \ \isacommand{using}\isamarkupfalse%
\ a\ \isacommand{apply}\isamarkupfalse%
\ {\isacharparenleft}{\kern0pt}simp\ add{\isacharcolon}{\kern0pt}y{\isacharunderscore}{\kern0pt}def\ PiE{\isacharunderscore}{\kern0pt}iff{\isacharparenright}{\kern0pt}\isanewline
\ \ \ \ \ \ \ \isacommand{apply}\isamarkupfalse%
\ {\isacharparenleft}{\kern0pt}metis\ imageE\ assms\ the{\isacharunderscore}{\kern0pt}inv{\isacharunderscore}{\kern0pt}into{\isacharunderscore}{\kern0pt}f{\isacharunderscore}{\kern0pt}eq{\isacharparenright}{\kern0pt}\isanewline
\ \ \ \ \ \ \isacommand{using}\isamarkupfalse%
\ a\ \isacommand{by}\isamarkupfalse%
\ {\isacharparenleft}{\kern0pt}simp\ add{\isacharcolon}{\kern0pt}y{\isacharunderscore}{\kern0pt}def\ PiE{\isacharunderscore}{\kern0pt}iff\ extensional{\isacharunderscore}{\kern0pt}def{\isacharparenright}{\kern0pt}\isanewline
\ \ \ \ \isacommand{have}\isamarkupfalse%
\ c{\isacharcolon}{\kern0pt}\ {\isachardoublequoteopen}x\ {\isacharequal}{\kern0pt}\ {\isacharparenleft}{\kern0pt}{\isasymlambda}a{\isachardot}{\kern0pt}\ restrict\ {\isacharparenleft}{\kern0pt}a\ {\isasymcirc}\ f{\isacharparenright}{\kern0pt}\ I{\isacharparenright}{\kern0pt}\ y{\isachardoublequoteclose}\ \isanewline
\ \ \ \ \ \ \isacommand{apply}\isamarkupfalse%
\ {\isacharparenleft}{\kern0pt}rule\ ext{\isacharparenright}{\kern0pt}\isanewline
\ \ \ \ \ \ \isacommand{using}\isamarkupfalse%
\ a\ \isacommand{apply}\isamarkupfalse%
\ {\isacharparenleft}{\kern0pt}simp\ add{\isacharcolon}{\kern0pt}y{\isacharunderscore}{\kern0pt}def\ PiE{\isacharunderscore}{\kern0pt}iff{\isacharparenright}{\kern0pt}\isanewline
\ \ \ \ \ \ \isacommand{apply}\isamarkupfalse%
\ {\isacharparenleft}{\kern0pt}rule\ conjI{\isacharparenright}{\kern0pt}\isanewline
\ \ \ \ \ \ \isacommand{using}\isamarkupfalse%
\ assms\ the{\isacharunderscore}{\kern0pt}inv{\isacharunderscore}{\kern0pt}into{\isacharunderscore}{\kern0pt}f{\isacharunderscore}{\kern0pt}eq\ \isanewline
\ \ \ \ \ \ \isacommand{apply}\isamarkupfalse%
\ {\isacharparenleft}{\kern0pt}simp\ add{\isacharcolon}{\kern0pt}\ the{\isacharunderscore}{\kern0pt}inv{\isacharunderscore}{\kern0pt}into{\isacharunderscore}{\kern0pt}f{\isacharunderscore}{\kern0pt}eq{\isacharparenright}{\kern0pt}\isanewline
\ \ \ \ \ \ \isacommand{by}\isamarkupfalse%
\ {\isacharparenleft}{\kern0pt}meson\ extensional{\isacharunderscore}{\kern0pt}arb{\isacharparenright}{\kern0pt}\isanewline
\ \ \ \ \isacommand{show}\isamarkupfalse%
\ {\isachardoublequoteopen}x\ {\isasymin}\ {\isacharquery}{\kern0pt}f\ {\isacharbackquote}{\kern0pt}\ {\isacharquery}{\kern0pt}rhs{\isachardoublequoteclose}\ \isacommand{using}\isamarkupfalse%
\ b\ c\ \isacommand{by}\isamarkupfalse%
\ blast\isanewline
\ \ \isacommand{qed}\isamarkupfalse%
\isanewline
\ \ \isacommand{moreover}\isamarkupfalse%
\ \isacommand{have}\isamarkupfalse%
\ {\isachardoublequoteopen}{\isacharquery}{\kern0pt}f\ {\isacharbackquote}{\kern0pt}\ {\isacharquery}{\kern0pt}rhs\ {\isasymsubseteq}\ {\isacharquery}{\kern0pt}lhs{\isachardoublequoteclose}\isanewline
\ \ \ \ \isacommand{apply}\isamarkupfalse%
\ {\isacharparenleft}{\kern0pt}rule\ image{\isacharunderscore}{\kern0pt}subsetI{\isacharparenright}{\kern0pt}\isanewline
\ \ \ \ \isacommand{by}\isamarkupfalse%
\ {\isacharparenleft}{\kern0pt}simp\ add{\isacharcolon}{\kern0pt}Pi{\isacharunderscore}{\kern0pt}def\ PiE{\isacharunderscore}{\kern0pt}def{\isacharparenright}{\kern0pt}\isanewline
\ \ \isacommand{ultimately}\isamarkupfalse%
\ \isacommand{show}\isamarkupfalse%
\ {\isacharquery}{\kern0pt}thesis\ \isacommand{by}\isamarkupfalse%
\ blast\isanewline
\isacommand{qed}\isamarkupfalse%
%
\endisatagproof
{\isafoldproof}%
%
\isadelimproof
\isanewline
%
\endisadelimproof
\isanewline
\isacommand{lemma}\isamarkupfalse%
\ {\isacharparenleft}{\kern0pt}\isakeyword{in}\ prob{\isacharunderscore}{\kern0pt}space{\isacharparenright}{\kern0pt}\ indep{\isacharunderscore}{\kern0pt}sets{\isacharunderscore}{\kern0pt}reindex{\isacharcolon}{\kern0pt}\isanewline
\ \ \isakeyword{assumes}\ {\isachardoublequoteopen}inj{\isacharunderscore}{\kern0pt}on\ f\ I{\isachardoublequoteclose}\isanewline
\ \ \isakeyword{shows}\ {\isachardoublequoteopen}indep{\isacharunderscore}{\kern0pt}sets\ A\ {\isacharparenleft}{\kern0pt}f\ {\isacharbackquote}{\kern0pt}\ I{\isacharparenright}{\kern0pt}\ {\isacharequal}{\kern0pt}\ indep{\isacharunderscore}{\kern0pt}sets\ {\isacharparenleft}{\kern0pt}{\isasymlambda}i{\isachardot}{\kern0pt}\ A\ {\isacharparenleft}{\kern0pt}f\ i{\isacharparenright}{\kern0pt}{\isacharparenright}{\kern0pt}\ I{\isachardoublequoteclose}\isanewline
%
\isadelimproof
%
\endisadelimproof
%
\isatagproof
\isacommand{proof}\isamarkupfalse%
\ {\isacharminus}{\kern0pt}\isanewline
\ \ \isacommand{have}\isamarkupfalse%
\ a{\isacharcolon}{\kern0pt}{\isachardoublequoteopen}{\isasymAnd}J\ g{\isachardot}{\kern0pt}\ J\ {\isasymsubseteq}\ I\ {\isasymLongrightarrow}\ {\isacharparenleft}{\kern0pt}{\isasymProd}j\ {\isasymin}\ f\ {\isacharbackquote}{\kern0pt}\ J{\isachardot}{\kern0pt}\ g\ j{\isacharparenright}{\kern0pt}\ {\isacharequal}{\kern0pt}\ {\isacharparenleft}{\kern0pt}{\isasymProd}j\ {\isasymin}\ J{\isachardot}{\kern0pt}\ g\ {\isacharparenleft}{\kern0pt}f\ j{\isacharparenright}{\kern0pt}{\isacharparenright}{\kern0pt}{\isachardoublequoteclose}\isanewline
\ \ \ \ \isacommand{by}\isamarkupfalse%
\ {\isacharparenleft}{\kern0pt}metis\ assms\ prod{\isachardot}{\kern0pt}reindex{\isacharunderscore}{\kern0pt}cong\ subset{\isacharunderscore}{\kern0pt}inj{\isacharunderscore}{\kern0pt}on{\isacharparenright}{\kern0pt}\isanewline
\isanewline
\ \ \isacommand{have}\isamarkupfalse%
\ {\isachardoublequoteopen}{\isasymAnd}J{\isachardot}{\kern0pt}\ J\ {\isasymsubseteq}\ I\ {\isasymLongrightarrow}\ {\isacharparenleft}{\kern0pt}{\isasymPi}\isactrlsub E\ i\ {\isasymin}\ J{\isachardot}{\kern0pt}\ A\ {\isacharparenleft}{\kern0pt}f\ i{\isacharparenright}{\kern0pt}{\isacharparenright}{\kern0pt}\ {\isacharequal}{\kern0pt}\ {\isacharparenleft}{\kern0pt}{\isasymlambda}a{\isachardot}{\kern0pt}\ restrict\ {\isacharparenleft}{\kern0pt}a\ {\isasymcirc}\ f{\isacharparenright}{\kern0pt}\ J{\isacharparenright}{\kern0pt}\ {\isacharbackquote}{\kern0pt}\ PiE\ {\isacharparenleft}{\kern0pt}f\ {\isacharbackquote}{\kern0pt}\ J{\isacharparenright}{\kern0pt}\ A{\isachardoublequoteclose}\isanewline
\ \ \ \ \isacommand{apply}\isamarkupfalse%
\ {\isacharparenleft}{\kern0pt}subst\ PiE{\isacharunderscore}{\kern0pt}reindex{\isacharbrackleft}{\kern0pt}symmetric{\isacharbrackright}{\kern0pt}{\isacharparenright}{\kern0pt}\isanewline
\ \ \ \ \isacommand{using}\isamarkupfalse%
\ assms\ inj{\isacharunderscore}{\kern0pt}on{\isacharunderscore}{\kern0pt}subset\ \isacommand{apply}\isamarkupfalse%
\ blast\isanewline
\ \ \ \ \isacommand{by}\isamarkupfalse%
\ {\isacharparenleft}{\kern0pt}simp\ add{\isacharcolon}{\kern0pt}\ comp{\isacharunderscore}{\kern0pt}def{\isacharparenright}{\kern0pt}\isanewline
\isanewline
\ \ \isacommand{hence}\isamarkupfalse%
\ b{\isacharcolon}{\kern0pt}{\isachardoublequoteopen}{\isasymAnd}P\ J{\isachardot}{\kern0pt}\ J\ {\isasymsubseteq}\ I\ {\isasymLongrightarrow}\ \ {\isacharparenleft}{\kern0pt}{\isasymAnd}x{\isachardot}{\kern0pt}\ P\ x\ {\isacharequal}{\kern0pt}\ P\ {\isacharparenleft}{\kern0pt}restrict\ x\ J{\isacharparenright}{\kern0pt}{\isacharparenright}{\kern0pt}\ {\isasymLongrightarrow}\ {\isacharparenleft}{\kern0pt}{\isasymforall}A{\isacharprime}{\kern0pt}{\isasymin}PiE\ {\isacharparenleft}{\kern0pt}f\ {\isacharbackquote}{\kern0pt}\ J{\isacharparenright}{\kern0pt}\ A{\isachardot}{\kern0pt}\ P\ {\isacharparenleft}{\kern0pt}A{\isacharprime}{\kern0pt}\ {\isasymcirc}\ f{\isacharparenright}{\kern0pt}{\isacharparenright}{\kern0pt}\ {\isacharequal}{\kern0pt}\ {\isacharparenleft}{\kern0pt}{\isasymforall}A{\isacharprime}{\kern0pt}\ {\isasymin}\ {\isasymPi}\isactrlsub E\ i\ {\isasymin}\ J{\isachardot}{\kern0pt}\ A\ {\isacharparenleft}{\kern0pt}f\ i{\isacharparenright}{\kern0pt}{\isachardot}{\kern0pt}\ P\ A{\isacharprime}{\kern0pt}{\isacharparenright}{\kern0pt}{\isachardoublequoteclose}\isanewline
\ \ \ \ \isacommand{by}\isamarkupfalse%
\ {\isacharparenleft}{\kern0pt}simp{\isacharparenright}{\kern0pt}\isanewline
\isanewline
\ \ \isacommand{have}\isamarkupfalse%
\ c{\isacharcolon}{\kern0pt}{\isachardoublequoteopen}{\isasymAnd}J{\isachardot}{\kern0pt}\ J\ {\isasymsubseteq}\ I\ {\isasymLongrightarrow}\ finite\ {\isacharparenleft}{\kern0pt}f\ {\isacharbackquote}{\kern0pt}\ J{\isacharparenright}{\kern0pt}\ {\isacharequal}{\kern0pt}\ finite\ J{\isachardoublequoteclose}\ \isanewline
\ \ \ \ \isacommand{by}\isamarkupfalse%
\ {\isacharparenleft}{\kern0pt}meson\ assms\ finite{\isacharunderscore}{\kern0pt}image{\isacharunderscore}{\kern0pt}iff\ inj{\isacharunderscore}{\kern0pt}on{\isacharunderscore}{\kern0pt}subset{\isacharparenright}{\kern0pt}\isanewline
\isanewline
\ \ \isacommand{show}\isamarkupfalse%
\ {\isacharquery}{\kern0pt}thesis\isanewline
\ \ \ \ \isacommand{apply}\isamarkupfalse%
\ {\isacharparenleft}{\kern0pt}simp\ add{\isacharcolon}{\kern0pt}indep{\isacharunderscore}{\kern0pt}sets{\isacharunderscore}{\kern0pt}def\ all{\isacharunderscore}{\kern0pt}subset{\isacharunderscore}{\kern0pt}image\ a\ c{\isacharparenright}{\kern0pt}\isanewline
\ \ \ \ \isacommand{apply}\isamarkupfalse%
\ {\isacharparenleft}{\kern0pt}subst\ make{\isacharunderscore}{\kern0pt}ext{\isacharparenright}{\kern0pt}\ \isacommand{apply}\isamarkupfalse%
\ {\isacharparenleft}{\kern0pt}simp\ cong{\isacharcolon}{\kern0pt}restrict{\isacharunderscore}{\kern0pt}cong{\isacharparenright}{\kern0pt}\isanewline
\ \ \ \ \isacommand{apply}\isamarkupfalse%
\ {\isacharparenleft}{\kern0pt}subst\ make{\isacharunderscore}{\kern0pt}ext{\isacharparenright}{\kern0pt}\ \isacommand{apply}\isamarkupfalse%
\ {\isacharparenleft}{\kern0pt}simp\ cong{\isacharcolon}{\kern0pt}restrict{\isacharunderscore}{\kern0pt}cong{\isacharparenright}{\kern0pt}\isanewline
\ \ \ \ \isacommand{by}\isamarkupfalse%
\ {\isacharparenleft}{\kern0pt}simp\ add{\isacharcolon}{\kern0pt}b{\isacharbrackleft}{\kern0pt}symmetric{\isacharbrackright}{\kern0pt}{\isacharparenright}{\kern0pt}\isanewline
\isacommand{qed}\isamarkupfalse%
%
\endisatagproof
{\isafoldproof}%
%
\isadelimproof
\isanewline
%
\endisadelimproof
\isanewline
\isacommand{lemma}\isamarkupfalse%
\ {\isacharparenleft}{\kern0pt}\isakeyword{in}\ prob{\isacharunderscore}{\kern0pt}space{\isacharparenright}{\kern0pt}\ indep{\isacharunderscore}{\kern0pt}vars{\isacharunderscore}{\kern0pt}reindex{\isacharcolon}{\kern0pt}\isanewline
\ \ \isakeyword{assumes}\ {\isachardoublequoteopen}inj{\isacharunderscore}{\kern0pt}on\ f\ I{\isachardoublequoteclose}\isanewline
\ \ \isakeyword{assumes}\ {\isachardoublequoteopen}indep{\isacharunderscore}{\kern0pt}vars\ M{\isacharprime}{\kern0pt}\ X{\isacharprime}{\kern0pt}\ {\isacharparenleft}{\kern0pt}f\ {\isacharbackquote}{\kern0pt}\ I{\isacharparenright}{\kern0pt}{\isachardoublequoteclose}\isanewline
\ \ \isakeyword{shows}\ {\isachardoublequoteopen}indep{\isacharunderscore}{\kern0pt}vars\ {\isacharparenleft}{\kern0pt}M{\isacharprime}{\kern0pt}\ {\isasymcirc}\ f{\isacharparenright}{\kern0pt}\ {\isacharparenleft}{\kern0pt}{\isasymlambda}k\ {\isasymomega}{\isachardot}{\kern0pt}\ X{\isacharprime}{\kern0pt}\ {\isacharparenleft}{\kern0pt}f\ k{\isacharparenright}{\kern0pt}\ {\isasymomega}{\isacharparenright}{\kern0pt}\ I{\isachardoublequoteclose}\isanewline
%
\isadelimproof
\ \ %
\endisadelimproof
%
\isatagproof
\isacommand{using}\isamarkupfalse%
\ assms\ \isacommand{by}\isamarkupfalse%
\ {\isacharparenleft}{\kern0pt}simp\ add{\isacharcolon}{\kern0pt}indep{\isacharunderscore}{\kern0pt}vars{\isacharunderscore}{\kern0pt}def{\isadigit{2}}\ indep{\isacharunderscore}{\kern0pt}sets{\isacharunderscore}{\kern0pt}reindex{\isacharparenright}{\kern0pt}%
\endisatagproof
{\isafoldproof}%
%
\isadelimproof
\isanewline
%
\endisadelimproof
\isanewline
\isacommand{lemma}\isamarkupfalse%
\ {\isacharparenleft}{\kern0pt}\isakeyword{in}\ prob{\isacharunderscore}{\kern0pt}space{\isacharparenright}{\kern0pt}\ variance{\isacharunderscore}{\kern0pt}divide{\isacharcolon}{\kern0pt}\isanewline
\ \ \isakeyword{fixes}\ f\ {\isacharcolon}{\kern0pt}{\isacharcolon}{\kern0pt}\ {\isachardoublequoteopen}{\isacharprime}{\kern0pt}a\ {\isasymRightarrow}\ real{\isachardoublequoteclose}\isanewline
\ \ \isakeyword{assumes}\ {\isachardoublequoteopen}integrable\ M\ f{\isachardoublequoteclose}\isanewline
\ \ \isakeyword{shows}\ {\isachardoublequoteopen}variance\ {\isacharparenleft}{\kern0pt}{\isasymlambda}{\isasymomega}{\isachardot}{\kern0pt}\ f\ {\isasymomega}\ {\isacharslash}{\kern0pt}\ r{\isacharparenright}{\kern0pt}\ {\isacharequal}{\kern0pt}\ variance\ f\ {\isacharslash}{\kern0pt}\ r{\isacharcircum}{\kern0pt}{\isadigit{2}}{\isachardoublequoteclose}\isanewline
%
\isadelimproof
\ \ %
\endisadelimproof
%
\isatagproof
\isacommand{apply}\isamarkupfalse%
\ {\isacharparenleft}{\kern0pt}subst\ Bochner{\isacharunderscore}{\kern0pt}Integration{\isachardot}{\kern0pt}integral{\isacharunderscore}{\kern0pt}divide{\isacharbrackleft}{\kern0pt}OF\ assms{\isacharparenleft}{\kern0pt}{\isadigit{1}}{\isacharparenright}{\kern0pt}{\isacharbrackright}{\kern0pt}{\isacharparenright}{\kern0pt}\isanewline
\ \ \isacommand{apply}\isamarkupfalse%
\ {\isacharparenleft}{\kern0pt}subst\ diff{\isacharunderscore}{\kern0pt}divide{\isacharunderscore}{\kern0pt}distrib{\isacharbrackleft}{\kern0pt}symmetric{\isacharbrackright}{\kern0pt}{\isacharparenright}{\kern0pt}\isanewline
\ \ \isacommand{using}\isamarkupfalse%
\ assms\ \isacommand{by}\isamarkupfalse%
\ {\isacharparenleft}{\kern0pt}simp\ add{\isacharcolon}{\kern0pt}power{\isadigit{2}}{\isacharunderscore}{\kern0pt}eq{\isacharunderscore}{\kern0pt}square\ algebra{\isacharunderscore}{\kern0pt}simps{\isacharparenright}{\kern0pt}%
\endisatagproof
{\isafoldproof}%
%
\isadelimproof
\isanewline
%
\endisadelimproof
\isanewline
\isacommand{lemma}\isamarkupfalse%
\ pmf{\isacharunderscore}{\kern0pt}eq{\isacharcolon}{\kern0pt}\isanewline
\ \ \isakeyword{assumes}\ {\isachardoublequoteopen}{\isasymAnd}x{\isachardot}{\kern0pt}\ x\ {\isasymin}\ set{\isacharunderscore}{\kern0pt}pmf\ {\isasymOmega}\ {\isasymLongrightarrow}\ {\isacharparenleft}{\kern0pt}x\ {\isasymin}\ P{\isacharparenright}{\kern0pt}\ {\isacharequal}{\kern0pt}\ {\isacharparenleft}{\kern0pt}x\ {\isasymin}\ Q{\isacharparenright}{\kern0pt}{\isachardoublequoteclose}\isanewline
\ \ \isakeyword{shows}\ {\isachardoublequoteopen}measure\ {\isacharparenleft}{\kern0pt}measure{\isacharunderscore}{\kern0pt}pmf\ {\isasymOmega}{\isacharparenright}{\kern0pt}\ P\ {\isacharequal}{\kern0pt}\ measure\ {\isacharparenleft}{\kern0pt}measure{\isacharunderscore}{\kern0pt}pmf\ {\isasymOmega}{\isacharparenright}{\kern0pt}\ Q{\isachardoublequoteclose}\isanewline
%
\isadelimproof
\ \ \ \ %
\endisadelimproof
%
\isatagproof
\isacommand{apply}\isamarkupfalse%
\ {\isacharparenleft}{\kern0pt}rule\ measure{\isacharunderscore}{\kern0pt}eq{\isacharunderscore}{\kern0pt}AE{\isacharparenright}{\kern0pt}\isanewline
\ \ \ \ \ \ \isacommand{apply}\isamarkupfalse%
\ {\isacharparenleft}{\kern0pt}subst\ AE{\isacharunderscore}{\kern0pt}measure{\isacharunderscore}{\kern0pt}pmf{\isacharunderscore}{\kern0pt}iff{\isacharparenright}{\kern0pt}\isanewline
\ \ \ \ \isacommand{using}\isamarkupfalse%
\ assms\ \isacommand{by}\isamarkupfalse%
\ auto%
\endisatagproof
{\isafoldproof}%
%
\isadelimproof
\isanewline
%
\endisadelimproof
\isanewline
\isacommand{lemma}\isamarkupfalse%
\ pmf{\isacharunderscore}{\kern0pt}mono{\isacharunderscore}{\kern0pt}{\isadigit{1}}{\isacharcolon}{\kern0pt}\isanewline
\ \ \isakeyword{assumes}\ {\isachardoublequoteopen}{\isasymAnd}x{\isachardot}{\kern0pt}\ x\ {\isasymin}\ P\ {\isasymLongrightarrow}\ x\ {\isasymin}\ set{\isacharunderscore}{\kern0pt}pmf\ {\isasymOmega}\ {\isasymLongrightarrow}\ x\ {\isasymin}\ Q{\isachardoublequoteclose}\isanewline
\ \ \isakeyword{shows}\ {\isachardoublequoteopen}measure\ {\isacharparenleft}{\kern0pt}measure{\isacharunderscore}{\kern0pt}pmf\ {\isasymOmega}{\isacharparenright}{\kern0pt}\ P\ {\isasymle}\ measure\ {\isacharparenleft}{\kern0pt}measure{\isacharunderscore}{\kern0pt}pmf\ {\isasymOmega}{\isacharparenright}{\kern0pt}\ Q{\isachardoublequoteclose}\isanewline
%
\isadelimproof
%
\endisadelimproof
%
\isatagproof
\isacommand{proof}\isamarkupfalse%
\ {\isacharminus}{\kern0pt}\isanewline
\ \ \isacommand{have}\isamarkupfalse%
\ {\isachardoublequoteopen}measure\ {\isacharparenleft}{\kern0pt}measure{\isacharunderscore}{\kern0pt}pmf\ {\isasymOmega}{\isacharparenright}{\kern0pt}\ P\ {\isacharequal}{\kern0pt}\ measure\ {\isacharparenleft}{\kern0pt}measure{\isacharunderscore}{\kern0pt}pmf\ {\isasymOmega}{\isacharparenright}{\kern0pt}\ {\isacharparenleft}{\kern0pt}P\ {\isasyminter}\ set{\isacharunderscore}{\kern0pt}pmf\ {\isasymOmega}{\isacharparenright}{\kern0pt}{\isachardoublequoteclose}\ \isanewline
\ \ \ \ \isacommand{by}\isamarkupfalse%
\ {\isacharparenleft}{\kern0pt}rule\ pmf{\isacharunderscore}{\kern0pt}eq{\isacharcomma}{\kern0pt}\ simp{\isacharparenright}{\kern0pt}\isanewline
\ \ \isacommand{also}\isamarkupfalse%
\ \isacommand{have}\isamarkupfalse%
\ {\isachardoublequoteopen}{\isachardot}{\kern0pt}{\isachardot}{\kern0pt}{\isachardot}{\kern0pt}\ {\isasymle}\ \ measure\ {\isacharparenleft}{\kern0pt}measure{\isacharunderscore}{\kern0pt}pmf\ {\isasymOmega}{\isacharparenright}{\kern0pt}\ Q{\isachardoublequoteclose}\isanewline
\ \ \isacommand{apply}\isamarkupfalse%
\ {\isacharparenleft}{\kern0pt}rule\ finite{\isacharunderscore}{\kern0pt}measure{\isachardot}{\kern0pt}finite{\isacharunderscore}{\kern0pt}measure{\isacharunderscore}{\kern0pt}mono{\isacharcomma}{\kern0pt}\ simp{\isacharparenright}{\kern0pt}\isanewline
\ \ \ \ \ \isacommand{apply}\isamarkupfalse%
\ {\isacharparenleft}{\kern0pt}rule\ subsetI{\isacharparenright}{\kern0pt}\ \isacommand{using}\isamarkupfalse%
\ assms\ \isacommand{apply}\isamarkupfalse%
\ blast\isanewline
\ \ \ \ \isacommand{by}\isamarkupfalse%
\ simp\isanewline
\ \ \isacommand{finally}\isamarkupfalse%
\ \isacommand{show}\isamarkupfalse%
\ {\isacharquery}{\kern0pt}thesis\ \isacommand{by}\isamarkupfalse%
\ simp\isanewline
\isacommand{qed}\isamarkupfalse%
%
\endisatagproof
{\isafoldproof}%
%
\isadelimproof
\isanewline
%
\endisadelimproof
\isanewline
\isacommand{definition}\isamarkupfalse%
\ {\isacharparenleft}{\kern0pt}\isakeyword{in}\ prob{\isacharunderscore}{\kern0pt}space{\isacharparenright}{\kern0pt}\ covariance\ \isakeyword{where}\ \isanewline
\ \ {\isachardoublequoteopen}covariance\ f\ g\ {\isacharequal}{\kern0pt}\ expectation\ {\isacharparenleft}{\kern0pt}{\isasymlambda}{\isasymomega}{\isachardot}{\kern0pt}\ {\isacharparenleft}{\kern0pt}f\ {\isasymomega}\ {\isacharminus}{\kern0pt}\ expectation\ f{\isacharparenright}{\kern0pt}\ {\isacharasterisk}{\kern0pt}\ {\isacharparenleft}{\kern0pt}g\ {\isasymomega}\ {\isacharminus}{\kern0pt}\ expectation\ g{\isacharparenright}{\kern0pt}{\isacharparenright}{\kern0pt}{\isachardoublequoteclose}\isanewline
\isanewline
\isacommand{lemma}\isamarkupfalse%
\ {\isacharparenleft}{\kern0pt}\isakeyword{in}\ prob{\isacharunderscore}{\kern0pt}space{\isacharparenright}{\kern0pt}\ real{\isacharunderscore}{\kern0pt}prod{\isacharunderscore}{\kern0pt}integrable{\isacharcolon}{\kern0pt}\isanewline
\ \ \isakeyword{fixes}\ f\ g\ {\isacharcolon}{\kern0pt}{\isacharcolon}{\kern0pt}\ {\isachardoublequoteopen}{\isacharprime}{\kern0pt}a\ {\isasymRightarrow}\ real{\isachardoublequoteclose}\isanewline
\ \ \isakeyword{assumes}\ {\isacharbrackleft}{\kern0pt}measurable{\isacharbrackright}{\kern0pt}{\isacharcolon}{\kern0pt}\ {\isachardoublequoteopen}f\ {\isasymin}\ borel{\isacharunderscore}{\kern0pt}measurable\ M{\isachardoublequoteclose}\ {\isachardoublequoteopen}g\ {\isasymin}\ borel{\isacharunderscore}{\kern0pt}measurable\ M{\isachardoublequoteclose}\isanewline
\ \ \isakeyword{assumes}\ sq{\isacharunderscore}{\kern0pt}int{\isacharcolon}{\kern0pt}\ {\isachardoublequoteopen}integrable\ M\ {\isacharparenleft}{\kern0pt}{\isasymlambda}{\isasymomega}{\isachardot}{\kern0pt}\ f\ {\isasymomega}{\isacharcircum}{\kern0pt}{\isadigit{2}}{\isacharparenright}{\kern0pt}{\isachardoublequoteclose}\ {\isachardoublequoteopen}integrable\ M\ {\isacharparenleft}{\kern0pt}{\isasymlambda}{\isasymomega}{\isachardot}{\kern0pt}\ g\ {\isasymomega}{\isacharcircum}{\kern0pt}{\isadigit{2}}{\isacharparenright}{\kern0pt}{\isachardoublequoteclose}\isanewline
\ \ \isakeyword{shows}\ {\isachardoublequoteopen}integrable\ M\ {\isacharparenleft}{\kern0pt}{\isasymlambda}{\isasymomega}{\isachardot}{\kern0pt}\ f\ {\isasymomega}\ {\isacharasterisk}{\kern0pt}\ g\ {\isasymomega}{\isacharparenright}{\kern0pt}{\isachardoublequoteclose}\isanewline
%
\isadelimproof
\ \ %
\endisadelimproof
%
\isatagproof
\isacommand{unfolding}\isamarkupfalse%
\ integrable{\isacharunderscore}{\kern0pt}iff{\isacharunderscore}{\kern0pt}bounded\isanewline
\isacommand{proof}\isamarkupfalse%
\isanewline
\ \ \isacommand{have}\isamarkupfalse%
\ {\isachardoublequoteopen}{\isacharparenleft}{\kern0pt}{\isasymintegral}\isactrlsup {\isacharplus}{\kern0pt}\ {\isasymomega}{\isachardot}{\kern0pt}\ ennreal\ {\isacharparenleft}{\kern0pt}norm\ {\isacharparenleft}{\kern0pt}f\ {\isasymomega}\ {\isacharasterisk}{\kern0pt}\ g\ {\isasymomega}{\isacharparenright}{\kern0pt}{\isacharparenright}{\kern0pt}\ {\isasympartial}M{\isacharparenright}{\kern0pt}\isactrlsup {\isadigit{2}}\ {\isacharequal}{\kern0pt}\ {\isacharparenleft}{\kern0pt}{\isasymintegral}\isactrlsup {\isacharplus}{\kern0pt}\ {\isasymomega}{\isachardot}{\kern0pt}\ ennreal\ {\isasymbar}f\ {\isasymomega}{\isasymbar}\ {\isacharasterisk}{\kern0pt}\ ennreal\ {\isasymbar}g\ {\isasymomega}{\isasymbar}\ {\isasympartial}M{\isacharparenright}{\kern0pt}\isactrlsup {\isadigit{2}}{\isachardoublequoteclose}\ \isanewline
\ \ \ \ \isacommand{by}\isamarkupfalse%
\ {\isacharparenleft}{\kern0pt}simp\ add{\isacharcolon}{\kern0pt}\ abs{\isacharunderscore}{\kern0pt}mult\ ennreal{\isacharunderscore}{\kern0pt}mult{\isacharparenright}{\kern0pt}\isanewline
\ \ \isacommand{also}\isamarkupfalse%
\ \isacommand{have}\isamarkupfalse%
\ {\isachardoublequoteopen}{\isachardot}{\kern0pt}{\isachardot}{\kern0pt}{\isachardot}{\kern0pt}\ {\isasymle}\ \ {\isacharparenleft}{\kern0pt}{\isasymintegral}\isactrlsup {\isacharplus}{\kern0pt}\ {\isasymomega}{\isachardot}{\kern0pt}\ ennreal\ {\isasymbar}f\ {\isasymomega}{\isasymbar}{\isacharcircum}{\kern0pt}{\isadigit{2}}\ {\isasympartial}M{\isacharparenright}{\kern0pt}\ {\isacharasterisk}{\kern0pt}\ {\isacharparenleft}{\kern0pt}{\isasymintegral}\isactrlsup {\isacharplus}{\kern0pt}\ {\isasymomega}{\isachardot}{\kern0pt}\ ennreal\ {\isasymbar}g\ {\isasymomega}{\isasymbar}{\isacharcircum}{\kern0pt}{\isadigit{2}}\ {\isasympartial}M{\isacharparenright}{\kern0pt}{\isachardoublequoteclose}\isanewline
\ \ \ \ \isacommand{apply}\isamarkupfalse%
\ {\isacharparenleft}{\kern0pt}rule\ Cauchy{\isacharunderscore}{\kern0pt}Schwarz{\isacharunderscore}{\kern0pt}nn{\isacharunderscore}{\kern0pt}integral{\isacharparenright}{\kern0pt}\ \isacommand{by}\isamarkupfalse%
\ auto\isanewline
\ \ \isacommand{also}\isamarkupfalse%
\ \isacommand{have}\isamarkupfalse%
\ {\isachardoublequoteopen}{\isachardot}{\kern0pt}{\isachardot}{\kern0pt}{\isachardot}{\kern0pt}\ {\isacharless}{\kern0pt}\ {\isasyminfinity}{\isachardoublequoteclose}\ \isanewline
\ \ \ \ \isacommand{using}\isamarkupfalse%
\ sq{\isacharunderscore}{\kern0pt}int\ \isacommand{by}\isamarkupfalse%
\ {\isacharparenleft}{\kern0pt}auto\ simp{\isacharcolon}{\kern0pt}\ integrable{\isacharunderscore}{\kern0pt}iff{\isacharunderscore}{\kern0pt}bounded\ ennreal{\isacharunderscore}{\kern0pt}power\ ennreal{\isacharunderscore}{\kern0pt}mult{\isacharunderscore}{\kern0pt}less{\isacharunderscore}{\kern0pt}top{\isacharparenright}{\kern0pt}\isanewline
\ \ \isacommand{finally}\isamarkupfalse%
\ \isacommand{have}\isamarkupfalse%
\ {\isachardoublequoteopen}{\isacharparenleft}{\kern0pt}{\isasymintegral}\isactrlsup {\isacharplus}{\kern0pt}\ x{\isachardot}{\kern0pt}\ ennreal\ {\isacharparenleft}{\kern0pt}norm\ {\isacharparenleft}{\kern0pt}f\ x\ {\isacharasterisk}{\kern0pt}\ g\ x{\isacharparenright}{\kern0pt}{\isacharparenright}{\kern0pt}\ {\isasympartial}M{\isacharparenright}{\kern0pt}\isactrlsup {\isadigit{2}}\ {\isacharless}{\kern0pt}\ {\isasyminfinity}{\isachardoublequoteclose}\ \isanewline
\ \ \ \ \isacommand{by}\isamarkupfalse%
\ simp\isanewline
\ \ \isacommand{thus}\isamarkupfalse%
\ {\isachardoublequoteopen}{\isacharparenleft}{\kern0pt}{\isasymintegral}\isactrlsup {\isacharplus}{\kern0pt}\ x{\isachardot}{\kern0pt}\ ennreal\ {\isacharparenleft}{\kern0pt}norm\ {\isacharparenleft}{\kern0pt}f\ x\ {\isacharasterisk}{\kern0pt}\ g\ x{\isacharparenright}{\kern0pt}{\isacharparenright}{\kern0pt}\ {\isasympartial}M{\isacharparenright}{\kern0pt}\ {\isacharless}{\kern0pt}\ {\isasyminfinity}{\isachardoublequoteclose}\ \isanewline
\ \ \ \ \isacommand{by}\isamarkupfalse%
\ {\isacharparenleft}{\kern0pt}simp\ add{\isacharcolon}{\kern0pt}\ power{\isacharunderscore}{\kern0pt}less{\isacharunderscore}{\kern0pt}top{\isacharunderscore}{\kern0pt}ennreal{\isacharparenright}{\kern0pt}\isanewline
\isacommand{qed}\isamarkupfalse%
\ auto%
\endisatagproof
{\isafoldproof}%
%
\isadelimproof
\isanewline
%
\endisadelimproof
\isanewline
\isacommand{lemma}\isamarkupfalse%
\ {\isacharparenleft}{\kern0pt}\isakeyword{in}\ prob{\isacharunderscore}{\kern0pt}space{\isacharparenright}{\kern0pt}\ covariance{\isacharunderscore}{\kern0pt}eq{\isacharcolon}{\kern0pt}\isanewline
\ \ \isakeyword{fixes}\ f\ {\isacharcolon}{\kern0pt}{\isacharcolon}{\kern0pt}\ {\isachardoublequoteopen}{\isacharprime}{\kern0pt}a\ {\isasymRightarrow}\ real{\isachardoublequoteclose}\isanewline
\ \ \isakeyword{assumes}\ {\isachardoublequoteopen}f\ {\isasymin}\ borel{\isacharunderscore}{\kern0pt}measurable\ M{\isachardoublequoteclose}\ {\isachardoublequoteopen}g\ {\isasymin}\ borel{\isacharunderscore}{\kern0pt}measurable\ M{\isachardoublequoteclose}\isanewline
\ \ \isakeyword{assumes}\ {\isachardoublequoteopen}integrable\ M\ {\isacharparenleft}{\kern0pt}{\isasymlambda}{\isasymomega}{\isachardot}{\kern0pt}\ f\ {\isasymomega}{\isacharcircum}{\kern0pt}{\isadigit{2}}{\isacharparenright}{\kern0pt}{\isachardoublequoteclose}\ {\isachardoublequoteopen}integrable\ M\ {\isacharparenleft}{\kern0pt}{\isasymlambda}{\isasymomega}{\isachardot}{\kern0pt}\ g\ {\isasymomega}{\isacharcircum}{\kern0pt}{\isadigit{2}}{\isacharparenright}{\kern0pt}{\isachardoublequoteclose}\isanewline
\ \ \isakeyword{shows}\ {\isachardoublequoteopen}covariance\ f\ g\ {\isacharequal}{\kern0pt}\ expectation\ {\isacharparenleft}{\kern0pt}{\isasymlambda}{\isasymomega}{\isachardot}{\kern0pt}\ f\ {\isasymomega}\ {\isacharasterisk}{\kern0pt}\ g\ {\isasymomega}{\isacharparenright}{\kern0pt}\ {\isacharminus}{\kern0pt}\ expectation\ f\ {\isacharasterisk}{\kern0pt}\ expectation\ g{\isachardoublequoteclose}\isanewline
%
\isadelimproof
%
\endisadelimproof
%
\isatagproof
\isacommand{proof}\isamarkupfalse%
\ {\isacharminus}{\kern0pt}\isanewline
\ \ \isacommand{have}\isamarkupfalse%
\ {\isachardoublequoteopen}integrable\ M\ f{\isachardoublequoteclose}\ \isacommand{using}\isamarkupfalse%
\ square{\isacharunderscore}{\kern0pt}integrable{\isacharunderscore}{\kern0pt}imp{\isacharunderscore}{\kern0pt}integrable\ assms\ \isacommand{by}\isamarkupfalse%
\ auto\isanewline
\ \ \isacommand{moreover}\isamarkupfalse%
\ \isacommand{have}\isamarkupfalse%
\ {\isachardoublequoteopen}integrable\ M\ g{\isachardoublequoteclose}\ \isacommand{using}\isamarkupfalse%
\ square{\isacharunderscore}{\kern0pt}integrable{\isacharunderscore}{\kern0pt}imp{\isacharunderscore}{\kern0pt}integrable\ assms\ \isacommand{by}\isamarkupfalse%
\ auto\isanewline
\ \ \isacommand{ultimately}\isamarkupfalse%
\ \isacommand{show}\isamarkupfalse%
\ {\isacharquery}{\kern0pt}thesis\isanewline
\ \ \ \ \isacommand{using}\isamarkupfalse%
\ assms\ real{\isacharunderscore}{\kern0pt}prod{\isacharunderscore}{\kern0pt}integrable\isanewline
\ \ \ \ \isacommand{by}\isamarkupfalse%
\ {\isacharparenleft}{\kern0pt}simp\ add{\isacharcolon}{\kern0pt}covariance{\isacharunderscore}{\kern0pt}def\ algebra{\isacharunderscore}{\kern0pt}simps\ prob{\isacharunderscore}{\kern0pt}space{\isacharparenright}{\kern0pt}\isanewline
\isacommand{qed}\isamarkupfalse%
%
\endisatagproof
{\isafoldproof}%
%
\isadelimproof
\isanewline
%
\endisadelimproof
\isanewline
\isacommand{lemma}\isamarkupfalse%
\ {\isacharparenleft}{\kern0pt}\isakeyword{in}\ prob{\isacharunderscore}{\kern0pt}space{\isacharparenright}{\kern0pt}\ covar{\isacharunderscore}{\kern0pt}integrable{\isacharcolon}{\kern0pt}\isanewline
\ \ \isakeyword{fixes}\ f\ g\ {\isacharcolon}{\kern0pt}{\isacharcolon}{\kern0pt}\ {\isachardoublequoteopen}{\isacharprime}{\kern0pt}a\ {\isasymRightarrow}\ real{\isachardoublequoteclose}\isanewline
\ \ \isakeyword{assumes}\ {\isachardoublequoteopen}f\ {\isasymin}\ borel{\isacharunderscore}{\kern0pt}measurable\ M{\isachardoublequoteclose}\ {\isachardoublequoteopen}g\ {\isasymin}\ borel{\isacharunderscore}{\kern0pt}measurable\ M{\isachardoublequoteclose}\isanewline
\ \ \isakeyword{assumes}\ {\isachardoublequoteopen}integrable\ M\ {\isacharparenleft}{\kern0pt}{\isasymlambda}{\isasymomega}{\isachardot}{\kern0pt}\ f\ {\isasymomega}{\isacharcircum}{\kern0pt}{\isadigit{2}}{\isacharparenright}{\kern0pt}{\isachardoublequoteclose}\ {\isachardoublequoteopen}integrable\ M\ {\isacharparenleft}{\kern0pt}{\isasymlambda}{\isasymomega}{\isachardot}{\kern0pt}\ g\ {\isasymomega}{\isacharcircum}{\kern0pt}{\isadigit{2}}{\isacharparenright}{\kern0pt}{\isachardoublequoteclose}\isanewline
\ \ \isakeyword{shows}\ {\isachardoublequoteopen}integrable\ M\ {\isacharparenleft}{\kern0pt}{\isasymlambda}{\isasymomega}{\isachardot}{\kern0pt}\ {\isacharparenleft}{\kern0pt}f\ {\isasymomega}\ {\isacharminus}{\kern0pt}\ expectation\ f{\isacharparenright}{\kern0pt}\ {\isacharasterisk}{\kern0pt}\ {\isacharparenleft}{\kern0pt}g\ {\isasymomega}\ {\isacharminus}{\kern0pt}\ expectation\ g{\isacharparenright}{\kern0pt}{\isacharparenright}{\kern0pt}{\isachardoublequoteclose}\isanewline
%
\isadelimproof
%
\endisadelimproof
%
\isatagproof
\isacommand{proof}\isamarkupfalse%
\ {\isacharminus}{\kern0pt}\isanewline
\ \ \isacommand{have}\isamarkupfalse%
\ {\isachardoublequoteopen}integrable\ M\ f{\isachardoublequoteclose}\ \isacommand{using}\isamarkupfalse%
\ square{\isacharunderscore}{\kern0pt}integrable{\isacharunderscore}{\kern0pt}imp{\isacharunderscore}{\kern0pt}integrable\ assms\ \isacommand{by}\isamarkupfalse%
\ auto\isanewline
\ \ \isacommand{moreover}\isamarkupfalse%
\ \isacommand{have}\isamarkupfalse%
\ {\isachardoublequoteopen}integrable\ M\ g{\isachardoublequoteclose}\ \isacommand{using}\isamarkupfalse%
\ square{\isacharunderscore}{\kern0pt}integrable{\isacharunderscore}{\kern0pt}imp{\isacharunderscore}{\kern0pt}integrable\ assms\ \isacommand{by}\isamarkupfalse%
\ auto\isanewline
\ \ \isacommand{ultimately}\isamarkupfalse%
\ \isacommand{show}\isamarkupfalse%
\ {\isacharquery}{\kern0pt}thesis\ \isacommand{using}\isamarkupfalse%
\ assms\ real{\isacharunderscore}{\kern0pt}prod{\isacharunderscore}{\kern0pt}integrable\ \isacommand{by}\isamarkupfalse%
\ {\isacharparenleft}{\kern0pt}simp\ add{\isacharcolon}{\kern0pt}\ algebra{\isacharunderscore}{\kern0pt}simps{\isacharparenright}{\kern0pt}\isanewline
\isacommand{qed}\isamarkupfalse%
%
\endisatagproof
{\isafoldproof}%
%
\isadelimproof
\isanewline
%
\endisadelimproof
\isanewline
\isacommand{lemma}\isamarkupfalse%
\ {\isacharparenleft}{\kern0pt}\isakeyword{in}\ prob{\isacharunderscore}{\kern0pt}space{\isacharparenright}{\kern0pt}\ sum{\isacharunderscore}{\kern0pt}square{\isacharunderscore}{\kern0pt}int{\isacharcolon}{\kern0pt}\isanewline
\ \ \isakeyword{fixes}\ f\ {\isacharcolon}{\kern0pt}{\isacharcolon}{\kern0pt}\ {\isachardoublequoteopen}{\isacharprime}{\kern0pt}b\ {\isasymRightarrow}\ {\isacharprime}{\kern0pt}a\ {\isasymRightarrow}\ real{\isachardoublequoteclose}\isanewline
\ \ \isakeyword{assumes}\ {\isachardoublequoteopen}finite\ I{\isachardoublequoteclose}\isanewline
\ \ \isakeyword{assumes}\ {\isachardoublequoteopen}{\isasymAnd}i{\isachardot}{\kern0pt}\ i\ {\isasymin}\ I\ {\isasymLongrightarrow}\ f\ i\ {\isasymin}\ borel{\isacharunderscore}{\kern0pt}measurable\ M{\isachardoublequoteclose}\isanewline
\ \ \isakeyword{assumes}\ {\isachardoublequoteopen}{\isasymAnd}i{\isachardot}{\kern0pt}\ i\ {\isasymin}\ I\ {\isasymLongrightarrow}\ integrable\ M\ {\isacharparenleft}{\kern0pt}{\isasymlambda}{\isasymomega}{\isachardot}{\kern0pt}\ f\ i\ {\isasymomega}{\isacharcircum}{\kern0pt}{\isadigit{2}}{\isacharparenright}{\kern0pt}{\isachardoublequoteclose}\isanewline
\ \ \isakeyword{shows}\ {\isachardoublequoteopen}integrable\ M\ {\isacharparenleft}{\kern0pt}{\isasymlambda}{\isasymomega}{\isachardot}{\kern0pt}\ {\isacharparenleft}{\kern0pt}{\isasymSum}i\ {\isasymin}\ I{\isachardot}{\kern0pt}\ f\ i\ {\isasymomega}{\isacharparenright}{\kern0pt}\isactrlsup {\isadigit{2}}{\isacharparenright}{\kern0pt}{\isachardoublequoteclose}\isanewline
%
\isadelimproof
\ \ %
\endisadelimproof
%
\isatagproof
\isacommand{apply}\isamarkupfalse%
\ {\isacharparenleft}{\kern0pt}simp\ add{\isacharcolon}{\kern0pt}power{\isadigit{2}}{\isacharunderscore}{\kern0pt}eq{\isacharunderscore}{\kern0pt}square\ sum{\isacharunderscore}{\kern0pt}distrib{\isacharunderscore}{\kern0pt}left\ sum{\isacharunderscore}{\kern0pt}distrib{\isacharunderscore}{\kern0pt}right{\isacharparenright}{\kern0pt}\isanewline
\ \ \isacommand{apply}\isamarkupfalse%
\ {\isacharparenleft}{\kern0pt}rule\ Bochner{\isacharunderscore}{\kern0pt}Integration{\isachardot}{\kern0pt}integrable{\isacharunderscore}{\kern0pt}sum{\isacharparenright}{\kern0pt}\isanewline
\ \ \isacommand{apply}\isamarkupfalse%
\ {\isacharparenleft}{\kern0pt}rule\ Bochner{\isacharunderscore}{\kern0pt}Integration{\isachardot}{\kern0pt}integrable{\isacharunderscore}{\kern0pt}sum{\isacharparenright}{\kern0pt}\isanewline
\ \ \isacommand{apply}\isamarkupfalse%
\ {\isacharparenleft}{\kern0pt}rule\ real{\isacharunderscore}{\kern0pt}prod{\isacharunderscore}{\kern0pt}integrable{\isacharparenright}{\kern0pt}\isanewline
\ \ \isacommand{using}\isamarkupfalse%
\ assms\ \isacommand{by}\isamarkupfalse%
\ auto%
\endisatagproof
{\isafoldproof}%
%
\isadelimproof
\ \isanewline
%
\endisadelimproof
\isanewline
\isacommand{lemma}\isamarkupfalse%
\ {\isacharparenleft}{\kern0pt}\isakeyword{in}\ prob{\isacharunderscore}{\kern0pt}space{\isacharparenright}{\kern0pt}\ var{\isacharunderscore}{\kern0pt}sum{\isacharunderscore}{\kern0pt}{\isadigit{1}}{\isacharcolon}{\kern0pt}\isanewline
\ \ \isakeyword{fixes}\ f\ {\isacharcolon}{\kern0pt}{\isacharcolon}{\kern0pt}\ {\isachardoublequoteopen}{\isacharprime}{\kern0pt}b\ {\isasymRightarrow}\ {\isacharprime}{\kern0pt}a\ {\isasymRightarrow}\ real{\isachardoublequoteclose}\isanewline
\ \ \isakeyword{assumes}\ {\isachardoublequoteopen}finite\ I{\isachardoublequoteclose}\isanewline
\ \ \isakeyword{assumes}\ {\isachardoublequoteopen}{\isasymAnd}i{\isachardot}{\kern0pt}\ i\ {\isasymin}\ I\ {\isasymLongrightarrow}\ f\ i\ {\isasymin}\ borel{\isacharunderscore}{\kern0pt}measurable\ M{\isachardoublequoteclose}\isanewline
\ \ \isakeyword{assumes}\ {\isachardoublequoteopen}{\isasymAnd}i{\isachardot}{\kern0pt}\ i\ {\isasymin}\ I\ {\isasymLongrightarrow}\ integrable\ M\ {\isacharparenleft}{\kern0pt}{\isasymlambda}{\isasymomega}{\isachardot}{\kern0pt}\ f\ i\ {\isasymomega}{\isacharcircum}{\kern0pt}{\isadigit{2}}{\isacharparenright}{\kern0pt}{\isachardoublequoteclose}\isanewline
\ \ \isakeyword{shows}\ \isanewline
\ \ \ \ {\isachardoublequoteopen}variance\ {\isacharparenleft}{\kern0pt}{\isasymlambda}{\isasymomega}{\isachardot}{\kern0pt}\ {\isacharparenleft}{\kern0pt}{\isasymSum}i\ {\isasymin}\ I{\isachardot}{\kern0pt}\ f\ i\ {\isasymomega}{\isacharparenright}{\kern0pt}{\isacharparenright}{\kern0pt}\ {\isacharequal}{\kern0pt}\ {\isacharparenleft}{\kern0pt}{\isasymSum}i\ {\isasymin}\ I{\isachardot}{\kern0pt}\ {\isacharparenleft}{\kern0pt}{\isasymSum}j\ {\isasymin}\ I{\isachardot}{\kern0pt}\ covariance\ {\isacharparenleft}{\kern0pt}f\ i{\isacharparenright}{\kern0pt}\ {\isacharparenleft}{\kern0pt}f\ j{\isacharparenright}{\kern0pt}{\isacharparenright}{\kern0pt}{\isacharparenright}{\kern0pt}{\isachardoublequoteclose}\ {\isacharparenleft}{\kern0pt}\isakeyword{is}\ {\isachardoublequoteopen}{\isacharquery}{\kern0pt}lhs\ {\isacharequal}{\kern0pt}\ {\isacharquery}{\kern0pt}rhs{\isachardoublequoteclose}{\isacharparenright}{\kern0pt}\isanewline
%
\isadelimproof
%
\endisadelimproof
%
\isatagproof
\isacommand{proof}\isamarkupfalse%
\ {\isacharminus}{\kern0pt}\isanewline
\ \ \isacommand{have}\isamarkupfalse%
\ a{\isacharcolon}{\kern0pt}{\isachardoublequoteopen}{\isasymAnd}i\ j{\isachardot}{\kern0pt}\ i\ {\isasymin}\ I\ {\isasymLongrightarrow}\ j\ {\isasymin}\ I\ {\isasymLongrightarrow}\ integrable\ M\ {\isacharparenleft}{\kern0pt}{\isasymlambda}{\isasymomega}{\isachardot}{\kern0pt}\ {\isacharparenleft}{\kern0pt}f\ i\ {\isasymomega}\ {\isacharminus}{\kern0pt}\ expectation\ {\isacharparenleft}{\kern0pt}f\ i{\isacharparenright}{\kern0pt}{\isacharparenright}{\kern0pt}\ {\isacharasterisk}{\kern0pt}\ {\isacharparenleft}{\kern0pt}f\ j\ {\isasymomega}\ {\isacharminus}{\kern0pt}\ expectation\ {\isacharparenleft}{\kern0pt}f\ j{\isacharparenright}{\kern0pt}{\isacharparenright}{\kern0pt}{\isacharparenright}{\kern0pt}{\isachardoublequoteclose}\ \isanewline
\ \ \ \ \isacommand{using}\isamarkupfalse%
\ assms\ covar{\isacharunderscore}{\kern0pt}integrable\ \isacommand{by}\isamarkupfalse%
\ simp\isanewline
\ \ \isacommand{have}\isamarkupfalse%
\ {\isachardoublequoteopen}{\isacharquery}{\kern0pt}lhs\ {\isacharequal}{\kern0pt}\ expectation\ {\isacharparenleft}{\kern0pt}{\isasymlambda}{\isasymomega}{\isachardot}{\kern0pt}\ {\isacharparenleft}{\kern0pt}{\isasymSum}i{\isasymin}I{\isachardot}{\kern0pt}\ f\ i\ {\isasymomega}\ {\isacharminus}{\kern0pt}\ expectation\ {\isacharparenleft}{\kern0pt}f\ i{\isacharparenright}{\kern0pt}{\isacharparenright}{\kern0pt}\isactrlsup {\isadigit{2}}{\isacharparenright}{\kern0pt}{\isachardoublequoteclose}\isanewline
\ \ \ \ \isacommand{apply}\isamarkupfalse%
\ {\isacharparenleft}{\kern0pt}subst\ Bochner{\isacharunderscore}{\kern0pt}Integration{\isachardot}{\kern0pt}integral{\isacharunderscore}{\kern0pt}sum{\isacharparenright}{\kern0pt}\isanewline
\ \ \ \ \isacommand{apply}\isamarkupfalse%
\ {\isacharparenleft}{\kern0pt}simp\ add{\isacharcolon}{\kern0pt}\ square{\isacharunderscore}{\kern0pt}integrable{\isacharunderscore}{\kern0pt}imp{\isacharunderscore}{\kern0pt}integrable{\isacharbrackleft}{\kern0pt}OF\ assms{\isacharparenleft}{\kern0pt}{\isadigit{2}}{\isacharparenright}{\kern0pt}\ assms{\isacharparenleft}{\kern0pt}{\isadigit{3}}{\isacharparenright}{\kern0pt}{\isacharbrackright}{\kern0pt}{\isacharparenright}{\kern0pt}\isanewline
\ \ \ \ \isacommand{by}\isamarkupfalse%
\ {\isacharparenleft}{\kern0pt}subst\ sum{\isacharunderscore}{\kern0pt}subtractf{\isacharbrackleft}{\kern0pt}symmetric{\isacharbrackright}{\kern0pt}{\isacharcomma}{\kern0pt}\ simp{\isacharparenright}{\kern0pt}\isanewline
\ \ \isacommand{also}\isamarkupfalse%
\ \isacommand{have}\isamarkupfalse%
\ {\isachardoublequoteopen}{\isachardot}{\kern0pt}{\isachardot}{\kern0pt}{\isachardot}{\kern0pt}\ {\isacharequal}{\kern0pt}\ expectation\ {\isacharparenleft}{\kern0pt}{\isasymlambda}{\isasymomega}{\isachardot}{\kern0pt}\ {\isacharparenleft}{\kern0pt}{\isasymSum}i\ {\isasymin}\ I{\isachardot}{\kern0pt}\ {\isacharparenleft}{\kern0pt}{\isasymSum}j\ {\isasymin}\ I{\isachardot}{\kern0pt}\ {\isacharparenleft}{\kern0pt}f\ i\ {\isasymomega}\ {\isacharminus}{\kern0pt}\ expectation\ {\isacharparenleft}{\kern0pt}f\ i{\isacharparenright}{\kern0pt}{\isacharparenright}{\kern0pt}\ {\isacharasterisk}{\kern0pt}\ \ {\isacharparenleft}{\kern0pt}f\ j\ {\isasymomega}\ {\isacharminus}{\kern0pt}\ expectation\ {\isacharparenleft}{\kern0pt}f\ j{\isacharparenright}{\kern0pt}{\isacharparenright}{\kern0pt}{\isacharparenright}{\kern0pt}{\isacharparenright}{\kern0pt}{\isacharparenright}{\kern0pt}{\isachardoublequoteclose}\isanewline
\ \ \ \ \isacommand{apply}\isamarkupfalse%
\ {\isacharparenleft}{\kern0pt}simp\ add{\isacharcolon}{\kern0pt}\ power{\isadigit{2}}{\isacharunderscore}{\kern0pt}eq{\isacharunderscore}{\kern0pt}square\ sum{\isacharunderscore}{\kern0pt}distrib{\isacharunderscore}{\kern0pt}right\ sum{\isacharunderscore}{\kern0pt}distrib{\isacharunderscore}{\kern0pt}left{\isacharparenright}{\kern0pt}\isanewline
\ \ \ \ \isacommand{apply}\isamarkupfalse%
\ {\isacharparenleft}{\kern0pt}rule\ Bochner{\isacharunderscore}{\kern0pt}Integration{\isachardot}{\kern0pt}integral{\isacharunderscore}{\kern0pt}cong{\isacharcomma}{\kern0pt}\ simp{\isacharparenright}{\kern0pt}\isanewline
\ \ \ \ \isacommand{apply}\isamarkupfalse%
\ {\isacharparenleft}{\kern0pt}rule\ sum{\isachardot}{\kern0pt}cong{\isacharcomma}{\kern0pt}\ simp{\isacharparenright}{\kern0pt}{\isacharplus}{\kern0pt}\isanewline
\ \ \ \ \isacommand{by}\isamarkupfalse%
\ {\isacharparenleft}{\kern0pt}simp\ add{\isacharcolon}{\kern0pt}mult{\isachardot}{\kern0pt}commute{\isacharparenright}{\kern0pt}\isanewline
\ \ \isacommand{also}\isamarkupfalse%
\ \isacommand{have}\isamarkupfalse%
\ {\isachardoublequoteopen}{\isachardot}{\kern0pt}{\isachardot}{\kern0pt}{\isachardot}{\kern0pt}\ {\isacharequal}{\kern0pt}\ {\isacharparenleft}{\kern0pt}{\isasymSum}i\ {\isasymin}\ I{\isachardot}{\kern0pt}\ {\isacharparenleft}{\kern0pt}{\isasymSum}j\ {\isasymin}\ I{\isachardot}{\kern0pt}\ covariance\ {\isacharparenleft}{\kern0pt}f\ i{\isacharparenright}{\kern0pt}\ {\isacharparenleft}{\kern0pt}f\ j{\isacharparenright}{\kern0pt}{\isacharparenright}{\kern0pt}{\isacharparenright}{\kern0pt}{\isachardoublequoteclose}\isanewline
\ \ \ \ \isacommand{using}\isamarkupfalse%
\ a\ \isacommand{by}\isamarkupfalse%
\ {\isacharparenleft}{\kern0pt}simp\ add{\isacharcolon}{\kern0pt}\ Bochner{\isacharunderscore}{\kern0pt}Integration{\isachardot}{\kern0pt}integral{\isacharunderscore}{\kern0pt}sum\ covariance{\isacharunderscore}{\kern0pt}def{\isacharparenright}{\kern0pt}\ \isanewline
\ \ \isacommand{finally}\isamarkupfalse%
\ \isacommand{show}\isamarkupfalse%
\ {\isacharquery}{\kern0pt}thesis\ \isacommand{by}\isamarkupfalse%
\ simp\isanewline
\isacommand{qed}\isamarkupfalse%
%
\endisatagproof
{\isafoldproof}%
%
\isadelimproof
\isanewline
%
\endisadelimproof
\isanewline
\isacommand{lemma}\isamarkupfalse%
\ {\isacharparenleft}{\kern0pt}\isakeyword{in}\ prob{\isacharunderscore}{\kern0pt}space{\isacharparenright}{\kern0pt}\ covar{\isacharunderscore}{\kern0pt}self{\isacharunderscore}{\kern0pt}eq{\isacharcolon}{\kern0pt}\isanewline
\ \ \isakeyword{fixes}\ f\ {\isacharcolon}{\kern0pt}{\isacharcolon}{\kern0pt}\ {\isachardoublequoteopen}{\isacharprime}{\kern0pt}a\ {\isasymRightarrow}\ real{\isachardoublequoteclose}\isanewline
\ \ \isakeyword{shows}\ {\isachardoublequoteopen}covariance\ f\ f\ {\isacharequal}{\kern0pt}\ variance\ f{\isachardoublequoteclose}\isanewline
%
\isadelimproof
\ \ %
\endisadelimproof
%
\isatagproof
\isacommand{by}\isamarkupfalse%
\ {\isacharparenleft}{\kern0pt}simp\ add{\isacharcolon}{\kern0pt}covariance{\isacharunderscore}{\kern0pt}def\ power{\isadigit{2}}{\isacharunderscore}{\kern0pt}eq{\isacharunderscore}{\kern0pt}square{\isacharparenright}{\kern0pt}%
\endisatagproof
{\isafoldproof}%
%
\isadelimproof
\isanewline
%
\endisadelimproof
\isanewline
\isacommand{lemma}\isamarkupfalse%
\ {\isacharparenleft}{\kern0pt}\isakeyword{in}\ prob{\isacharunderscore}{\kern0pt}space{\isacharparenright}{\kern0pt}\ covar{\isacharunderscore}{\kern0pt}indep{\isacharunderscore}{\kern0pt}eq{\isacharunderscore}{\kern0pt}zero{\isacharcolon}{\kern0pt}\isanewline
\ \ \isakeyword{fixes}\ f\ g\ {\isacharcolon}{\kern0pt}{\isacharcolon}{\kern0pt}\ {\isachardoublequoteopen}{\isacharprime}{\kern0pt}a\ {\isasymRightarrow}\ real{\isachardoublequoteclose}\isanewline
\ \ \isakeyword{assumes}\ {\isachardoublequoteopen}integrable\ M\ f{\isachardoublequoteclose}\isanewline
\ \ \isakeyword{assumes}\ {\isachardoublequoteopen}integrable\ M\ g{\isachardoublequoteclose}\isanewline
\ \ \isakeyword{assumes}\ {\isachardoublequoteopen}indep{\isacharunderscore}{\kern0pt}var\ borel\ f\ borel\ g{\isachardoublequoteclose}\isanewline
\ \ \isakeyword{shows}\ {\isachardoublequoteopen}covariance\ f\ g\ {\isacharequal}{\kern0pt}\ {\isadigit{0}}{\isachardoublequoteclose}\isanewline
%
\isadelimproof
%
\endisadelimproof
%
\isatagproof
\isacommand{proof}\isamarkupfalse%
\ {\isacharminus}{\kern0pt}\isanewline
\ \ \isacommand{have}\isamarkupfalse%
\ a{\isacharcolon}{\kern0pt}{\isachardoublequoteopen}indep{\isacharunderscore}{\kern0pt}var\ borel\ {\isacharparenleft}{\kern0pt}{\isacharparenleft}{\kern0pt}{\isasymlambda}t{\isachardot}{\kern0pt}\ t\ {\isacharminus}{\kern0pt}\ expectation\ f{\isacharparenright}{\kern0pt}\ {\isasymcirc}\ f{\isacharparenright}{\kern0pt}\ borel\ \ {\isacharparenleft}{\kern0pt}{\isacharparenleft}{\kern0pt}{\isasymlambda}t{\isachardot}{\kern0pt}\ t\ {\isacharminus}{\kern0pt}\ expectation\ g{\isacharparenright}{\kern0pt}\ {\isasymcirc}\ g{\isacharparenright}{\kern0pt}{\isachardoublequoteclose}\isanewline
\ \ \ \ \isacommand{by}\isamarkupfalse%
\ {\isacharparenleft}{\kern0pt}rule\ indep{\isacharunderscore}{\kern0pt}var{\isacharunderscore}{\kern0pt}compose{\isacharbrackleft}{\kern0pt}OF\ assms{\isacharparenleft}{\kern0pt}{\isadigit{3}}{\isacharparenright}{\kern0pt}{\isacharbrackright}{\kern0pt}{\isacharcomma}{\kern0pt}\ simp{\isacharcomma}{\kern0pt}\ simp{\isacharparenright}{\kern0pt}\isanewline
\isanewline
\ \ \isacommand{show}\isamarkupfalse%
\ {\isacharquery}{\kern0pt}thesis\isanewline
\ \ \ \ \isacommand{apply}\isamarkupfalse%
\ {\isacharparenleft}{\kern0pt}simp\ add{\isacharcolon}{\kern0pt}covariance{\isacharunderscore}{\kern0pt}def{\isacharparenright}{\kern0pt}\isanewline
\ \ \ \ \isacommand{apply}\isamarkupfalse%
\ {\isacharparenleft}{\kern0pt}subst\ indep{\isacharunderscore}{\kern0pt}var{\isacharunderscore}{\kern0pt}lebesgue{\isacharunderscore}{\kern0pt}integral{\isacharparenright}{\kern0pt}\isanewline
\ \ \ \ \isacommand{using}\isamarkupfalse%
\ a\ assms\ \isacommand{by}\isamarkupfalse%
\ {\isacharparenleft}{\kern0pt}simp\ add{\isacharcolon}{\kern0pt}comp{\isacharunderscore}{\kern0pt}def\ prob{\isacharunderscore}{\kern0pt}space{\isacharparenright}{\kern0pt}{\isacharplus}{\kern0pt}\isanewline
\isacommand{qed}\isamarkupfalse%
%
\endisatagproof
{\isafoldproof}%
%
\isadelimproof
\isanewline
%
\endisadelimproof
\isanewline
\isacommand{lemma}\isamarkupfalse%
\ {\isacharparenleft}{\kern0pt}\isakeyword{in}\ prob{\isacharunderscore}{\kern0pt}space{\isacharparenright}{\kern0pt}\ var{\isacharunderscore}{\kern0pt}sum{\isacharunderscore}{\kern0pt}{\isadigit{2}}{\isacharcolon}{\kern0pt}\isanewline
\ \ \isakeyword{fixes}\ f\ {\isacharcolon}{\kern0pt}{\isacharcolon}{\kern0pt}\ {\isachardoublequoteopen}{\isacharprime}{\kern0pt}b\ {\isasymRightarrow}\ {\isacharprime}{\kern0pt}a\ {\isasymRightarrow}\ real{\isachardoublequoteclose}\isanewline
\ \ \isakeyword{assumes}\ {\isachardoublequoteopen}finite\ I{\isachardoublequoteclose}\isanewline
\ \ \isakeyword{assumes}\ {\isachardoublequoteopen}{\isasymAnd}i{\isachardot}{\kern0pt}\ i\ {\isasymin}\ I\ {\isasymLongrightarrow}\ f\ i\ {\isasymin}\ borel{\isacharunderscore}{\kern0pt}measurable\ M{\isachardoublequoteclose}\isanewline
\ \ \isakeyword{assumes}\ {\isachardoublequoteopen}{\isasymAnd}i{\isachardot}{\kern0pt}\ i\ {\isasymin}\ I\ {\isasymLongrightarrow}\ integrable\ M\ {\isacharparenleft}{\kern0pt}{\isasymlambda}{\isasymomega}{\isachardot}{\kern0pt}\ f\ i\ {\isasymomega}{\isacharcircum}{\kern0pt}{\isadigit{2}}{\isacharparenright}{\kern0pt}{\isachardoublequoteclose}\isanewline
\ \ \isakeyword{shows}\ {\isachardoublequoteopen}variance\ {\isacharparenleft}{\kern0pt}{\isasymlambda}{\isasymomega}{\isachardot}{\kern0pt}\ {\isacharparenleft}{\kern0pt}{\isasymSum}i\ {\isasymin}\ I{\isachardot}{\kern0pt}\ f\ i\ {\isasymomega}{\isacharparenright}{\kern0pt}{\isacharparenright}{\kern0pt}\ {\isacharequal}{\kern0pt}\ \isanewline
\ \ \ \ \ \ {\isacharparenleft}{\kern0pt}{\isasymSum}i\ {\isasymin}\ I{\isachardot}{\kern0pt}\ variance\ {\isacharparenleft}{\kern0pt}f\ i{\isacharparenright}{\kern0pt}{\isacharparenright}{\kern0pt}\ {\isacharplus}{\kern0pt}\ {\isacharparenleft}{\kern0pt}{\isasymSum}i\ {\isasymin}\ I{\isachardot}{\kern0pt}\ {\isasymSum}j\ {\isasymin}\ I\ {\isacharminus}{\kern0pt}\ {\isacharbraceleft}{\kern0pt}i{\isacharbraceright}{\kern0pt}{\isachardot}{\kern0pt}\ covariance\ {\isacharparenleft}{\kern0pt}f\ i{\isacharparenright}{\kern0pt}\ {\isacharparenleft}{\kern0pt}f\ j{\isacharparenright}{\kern0pt}{\isacharparenright}{\kern0pt}{\isachardoublequoteclose}\isanewline
%
\isadelimproof
\ \ %
\endisadelimproof
%
\isatagproof
\isacommand{apply}\isamarkupfalse%
\ {\isacharparenleft}{\kern0pt}subst\ var{\isacharunderscore}{\kern0pt}sum{\isacharunderscore}{\kern0pt}{\isadigit{1}}{\isacharbrackleft}{\kern0pt}OF\ assms{\isacharparenleft}{\kern0pt}{\isadigit{1}}{\isacharparenright}{\kern0pt}\ assms{\isacharparenleft}{\kern0pt}{\isadigit{2}}{\isacharparenright}{\kern0pt}\ assms{\isacharparenleft}{\kern0pt}{\isadigit{3}}{\isacharparenright}{\kern0pt}{\isacharbrackright}{\kern0pt}{\isacharcomma}{\kern0pt}\ simp{\isacharparenright}{\kern0pt}\isanewline
\ \ \isacommand{apply}\isamarkupfalse%
\ {\isacharparenleft}{\kern0pt}subst\ covar{\isacharunderscore}{\kern0pt}self{\isacharunderscore}{\kern0pt}eq{\isacharbrackleft}{\kern0pt}symmetric{\isacharbrackright}{\kern0pt}{\isacharparenright}{\kern0pt}\isanewline
\ \ \isacommand{apply}\isamarkupfalse%
\ {\isacharparenleft}{\kern0pt}subst\ sum{\isachardot}{\kern0pt}distrib{\isacharbrackleft}{\kern0pt}symmetric{\isacharbrackright}{\kern0pt}{\isacharparenright}{\kern0pt}\isanewline
\ \ \isacommand{apply}\isamarkupfalse%
\ {\isacharparenleft}{\kern0pt}rule\ sum{\isachardot}{\kern0pt}cong{\isacharcomma}{\kern0pt}\ simp{\isacharparenright}{\kern0pt}\isanewline
\ \ \isacommand{apply}\isamarkupfalse%
\ {\isacharparenleft}{\kern0pt}subst\ sum{\isachardot}{\kern0pt}insert{\isacharbrackleft}{\kern0pt}symmetric{\isacharbrackright}{\kern0pt}{\isacharcomma}{\kern0pt}\ simp\ add{\isacharcolon}{\kern0pt}assms{\isacharcomma}{\kern0pt}\ simp{\isacharparenright}{\kern0pt}\isanewline
\ \ \isacommand{by}\isamarkupfalse%
\ {\isacharparenleft}{\kern0pt}rule\ sum{\isachardot}{\kern0pt}cong{\isacharcomma}{\kern0pt}\ simp\ add{\isacharcolon}{\kern0pt}insert{\isacharunderscore}{\kern0pt}absorb{\isacharcomma}{\kern0pt}\ simp{\isacharparenright}{\kern0pt}%
\endisatagproof
{\isafoldproof}%
%
\isadelimproof
\isanewline
%
\endisadelimproof
\isanewline
\isacommand{lemma}\isamarkupfalse%
\ {\isacharparenleft}{\kern0pt}\isakeyword{in}\ prob{\isacharunderscore}{\kern0pt}space{\isacharparenright}{\kern0pt}\ var{\isacharunderscore}{\kern0pt}sum{\isacharunderscore}{\kern0pt}pairwise{\isacharunderscore}{\kern0pt}indep{\isacharcolon}{\kern0pt}\isanewline
\ \ \isakeyword{fixes}\ f\ {\isacharcolon}{\kern0pt}{\isacharcolon}{\kern0pt}\ {\isachardoublequoteopen}{\isacharprime}{\kern0pt}b\ {\isasymRightarrow}\ {\isacharprime}{\kern0pt}a\ {\isasymRightarrow}\ real{\isachardoublequoteclose}\isanewline
\ \ \isakeyword{assumes}\ {\isachardoublequoteopen}finite\ I{\isachardoublequoteclose}\isanewline
\ \ \isakeyword{assumes}\ {\isachardoublequoteopen}{\isasymAnd}i{\isachardot}{\kern0pt}\ i\ {\isasymin}\ I\ {\isasymLongrightarrow}\ f\ i\ {\isasymin}\ borel{\isacharunderscore}{\kern0pt}measurable\ M{\isachardoublequoteclose}\isanewline
\ \ \isakeyword{assumes}\ {\isachardoublequoteopen}{\isasymAnd}i{\isachardot}{\kern0pt}\ i\ {\isasymin}\ I\ {\isasymLongrightarrow}\ integrable\ M\ {\isacharparenleft}{\kern0pt}{\isasymlambda}{\isasymomega}{\isachardot}{\kern0pt}\ f\ i\ {\isasymomega}{\isacharcircum}{\kern0pt}{\isadigit{2}}{\isacharparenright}{\kern0pt}{\isachardoublequoteclose}\isanewline
\ \ \isakeyword{assumes}\ {\isachardoublequoteopen}{\isasymAnd}i\ j{\isachardot}{\kern0pt}\ i\ {\isasymin}\ I\ {\isasymLongrightarrow}\ j\ {\isasymin}\ I\ {\isasymLongrightarrow}\ i\ {\isasymnoteq}\ j\ {\isasymLongrightarrow}\ indep{\isacharunderscore}{\kern0pt}var\ borel\ {\isacharparenleft}{\kern0pt}f\ i{\isacharparenright}{\kern0pt}\ borel\ {\isacharparenleft}{\kern0pt}f\ j{\isacharparenright}{\kern0pt}{\isachardoublequoteclose}\isanewline
\ \ \isakeyword{shows}\ {\isachardoublequoteopen}variance\ {\isacharparenleft}{\kern0pt}{\isasymlambda}{\isasymomega}{\isachardot}{\kern0pt}\ {\isacharparenleft}{\kern0pt}{\isasymSum}i\ {\isasymin}\ I{\isachardot}{\kern0pt}\ f\ i\ {\isasymomega}{\isacharparenright}{\kern0pt}{\isacharparenright}{\kern0pt}\ {\isacharequal}{\kern0pt}\ {\isacharparenleft}{\kern0pt}{\isasymSum}i\ {\isasymin}\ I{\isachardot}{\kern0pt}\ variance\ {\isacharparenleft}{\kern0pt}f\ i{\isacharparenright}{\kern0pt}{\isacharparenright}{\kern0pt}{\isachardoublequoteclose}\isanewline
%
\isadelimproof
%
\endisadelimproof
%
\isatagproof
\isacommand{proof}\isamarkupfalse%
\ {\isacharminus}{\kern0pt}\isanewline
\ \ \isacommand{have}\isamarkupfalse%
\ {\isachardoublequoteopen}{\isasymAnd}i\ j{\isachardot}{\kern0pt}\ i\ {\isasymin}\ I\ {\isasymLongrightarrow}\ j\ {\isasymin}\ I\ {\isacharminus}{\kern0pt}\ {\isacharbraceleft}{\kern0pt}i{\isacharbraceright}{\kern0pt}\ {\isasymLongrightarrow}\ covariance\ {\isacharparenleft}{\kern0pt}f\ i{\isacharparenright}{\kern0pt}\ {\isacharparenleft}{\kern0pt}f\ j{\isacharparenright}{\kern0pt}\ {\isacharequal}{\kern0pt}\ {\isadigit{0}}{\isachardoublequoteclose}\ \isanewline
\ \ \ \ \isacommand{apply}\isamarkupfalse%
\ {\isacharparenleft}{\kern0pt}rule\ covar{\isacharunderscore}{\kern0pt}indep{\isacharunderscore}{\kern0pt}eq{\isacharunderscore}{\kern0pt}zero{\isacharparenright}{\kern0pt}\isanewline
\ \ \ \ \isacommand{using}\isamarkupfalse%
\ assms\ square{\isacharunderscore}{\kern0pt}integrable{\isacharunderscore}{\kern0pt}imp{\isacharunderscore}{\kern0pt}integrable{\isacharbrackleft}{\kern0pt}OF\ assms{\isacharparenleft}{\kern0pt}{\isadigit{2}}{\isacharparenright}{\kern0pt}\ assms{\isacharparenleft}{\kern0pt}{\isadigit{3}}{\isacharparenright}{\kern0pt}{\isacharbrackright}{\kern0pt}\ \isacommand{by}\isamarkupfalse%
\ auto\isanewline
\isanewline
\ \ \isacommand{hence}\isamarkupfalse%
\ a{\isacharcolon}{\kern0pt}{\isachardoublequoteopen}{\isacharparenleft}{\kern0pt}{\isasymSum}i\ {\isasymin}\ I{\isachardot}{\kern0pt}\ {\isasymSum}j\ {\isasymin}\ I\ {\isacharminus}{\kern0pt}\ {\isacharbraceleft}{\kern0pt}i{\isacharbraceright}{\kern0pt}{\isachardot}{\kern0pt}\ covariance\ {\isacharparenleft}{\kern0pt}f\ i{\isacharparenright}{\kern0pt}\ {\isacharparenleft}{\kern0pt}f\ j{\isacharparenright}{\kern0pt}{\isacharparenright}{\kern0pt}\ {\isacharequal}{\kern0pt}\ {\isadigit{0}}{\isachardoublequoteclose}\isanewline
\ \ \ \ \isacommand{by}\isamarkupfalse%
\ simp\isanewline
\isanewline
\ \ \isacommand{show}\isamarkupfalse%
\ {\isacharquery}{\kern0pt}thesis\isanewline
\ \ \ \ \isacommand{by}\isamarkupfalse%
\ {\isacharparenleft}{\kern0pt}subst\ var{\isacharunderscore}{\kern0pt}sum{\isacharunderscore}{\kern0pt}{\isadigit{2}}{\isacharbrackleft}{\kern0pt}OF\ assms{\isacharparenleft}{\kern0pt}{\isadigit{1}}{\isacharparenright}{\kern0pt}\ assms{\isacharparenleft}{\kern0pt}{\isadigit{2}}{\isacharparenright}{\kern0pt}\ assms{\isacharparenleft}{\kern0pt}{\isadigit{3}}{\isacharparenright}{\kern0pt}{\isacharbrackright}{\kern0pt}{\isacharcomma}{\kern0pt}\ simp{\isacharcomma}{\kern0pt}\ simp\ add{\isacharcolon}{\kern0pt}a{\isacharparenright}{\kern0pt}\isanewline
\isacommand{qed}\isamarkupfalse%
%
\endisatagproof
{\isafoldproof}%
%
\isadelimproof
\isanewline
%
\endisadelimproof
\isanewline
\isacommand{lemma}\isamarkupfalse%
\ {\isacharparenleft}{\kern0pt}\isakeyword{in}\ prob{\isacharunderscore}{\kern0pt}space{\isacharparenright}{\kern0pt}\ indep{\isacharunderscore}{\kern0pt}var{\isacharunderscore}{\kern0pt}from{\isacharunderscore}{\kern0pt}indep{\isacharunderscore}{\kern0pt}vars{\isacharcolon}{\kern0pt}\isanewline
\ \ \isakeyword{assumes}\ {\isachardoublequoteopen}i\ {\isasymnoteq}\ j{\isachardoublequoteclose}\isanewline
\ \ \isakeyword{assumes}\ {\isachardoublequoteopen}indep{\isacharunderscore}{\kern0pt}vars\ {\isacharparenleft}{\kern0pt}{\isasymlambda}{\isacharunderscore}{\kern0pt}{\isachardot}{\kern0pt}\ M{\isacharprime}{\kern0pt}{\isacharparenright}{\kern0pt}\ f\ {\isacharbraceleft}{\kern0pt}i{\isacharcomma}{\kern0pt}\ j{\isacharbraceright}{\kern0pt}{\isachardoublequoteclose}\ \isanewline
\ \ \isakeyword{shows}\ {\isachardoublequoteopen}indep{\isacharunderscore}{\kern0pt}var\ M{\isacharprime}{\kern0pt}\ {\isacharparenleft}{\kern0pt}f\ i{\isacharparenright}{\kern0pt}\ M{\isacharprime}{\kern0pt}\ {\isacharparenleft}{\kern0pt}f\ j{\isacharparenright}{\kern0pt}{\isachardoublequoteclose}\isanewline
%
\isadelimproof
%
\endisadelimproof
%
\isatagproof
\isacommand{proof}\isamarkupfalse%
\ {\isacharminus}{\kern0pt}\isanewline
\ \ \isacommand{have}\isamarkupfalse%
\ a{\isacharcolon}{\kern0pt}{\isachardoublequoteopen}inj\ {\isacharparenleft}{\kern0pt}case{\isacharunderscore}{\kern0pt}bool\ i\ j{\isacharparenright}{\kern0pt}{\isachardoublequoteclose}\ \isacommand{using}\isamarkupfalse%
\ assms{\isacharparenleft}{\kern0pt}{\isadigit{1}}{\isacharparenright}{\kern0pt}\ \isanewline
\ \ \ \ \isacommand{by}\isamarkupfalse%
\ {\isacharparenleft}{\kern0pt}simp\ add{\isacharcolon}{\kern0pt}\ bool{\isachardot}{\kern0pt}case{\isacharunderscore}{\kern0pt}eq{\isacharunderscore}{\kern0pt}if\ inj{\isacharunderscore}{\kern0pt}def{\isacharparenright}{\kern0pt}\isanewline
\ \ \isacommand{have}\isamarkupfalse%
\ b{\isacharcolon}{\kern0pt}{\isachardoublequoteopen}range\ {\isacharparenleft}{\kern0pt}case{\isacharunderscore}{\kern0pt}bool\ i\ j{\isacharparenright}{\kern0pt}\ {\isacharequal}{\kern0pt}\ {\isacharbraceleft}{\kern0pt}i{\isacharcomma}{\kern0pt}j{\isacharbraceright}{\kern0pt}{\isachardoublequoteclose}\ \isanewline
\ \ \ \ \isacommand{by}\isamarkupfalse%
\ {\isacharparenleft}{\kern0pt}simp\ add{\isacharcolon}{\kern0pt}\ UNIV{\isacharunderscore}{\kern0pt}bool\ insert{\isacharunderscore}{\kern0pt}commute{\isacharparenright}{\kern0pt}\isanewline
\ \ \isacommand{have}\isamarkupfalse%
\ c{\isacharcolon}{\kern0pt}{\isachardoublequoteopen}indep{\isacharunderscore}{\kern0pt}vars\ {\isacharparenleft}{\kern0pt}{\isasymlambda}{\isacharunderscore}{\kern0pt}{\isachardot}{\kern0pt}\ M{\isacharprime}{\kern0pt}{\isacharparenright}{\kern0pt}\ f\ {\isacharparenleft}{\kern0pt}range\ {\isacharparenleft}{\kern0pt}case{\isacharunderscore}{\kern0pt}bool\ i\ j{\isacharparenright}{\kern0pt}{\isacharparenright}{\kern0pt}{\isachardoublequoteclose}\ \isacommand{using}\isamarkupfalse%
\ assms{\isacharparenleft}{\kern0pt}{\isadigit{2}}{\isacharparenright}{\kern0pt}\ b\ \isacommand{by}\isamarkupfalse%
\ simp\isanewline
\isanewline
\ \ \isacommand{have}\isamarkupfalse%
\ {\isachardoublequoteopen}True\ {\isacharequal}{\kern0pt}\ indep{\isacharunderscore}{\kern0pt}vars\ {\isacharparenleft}{\kern0pt}{\isasymlambda}x{\isachardot}{\kern0pt}\ M{\isacharprime}{\kern0pt}{\isacharparenright}{\kern0pt}\ {\isacharparenleft}{\kern0pt}{\isasymlambda}x{\isachardot}{\kern0pt}\ f\ {\isacharparenleft}{\kern0pt}case{\isacharunderscore}{\kern0pt}bool\ i\ j\ x{\isacharparenright}{\kern0pt}{\isacharparenright}{\kern0pt}\ UNIV{\isachardoublequoteclose}\ \isanewline
\ \ \ \ \isacommand{using}\isamarkupfalse%
\ indep{\isacharunderscore}{\kern0pt}vars{\isacharunderscore}{\kern0pt}reindex{\isacharbrackleft}{\kern0pt}OF\ a\ c{\isacharbrackright}{\kern0pt}\isanewline
\ \ \ \ \isacommand{by}\isamarkupfalse%
\ {\isacharparenleft}{\kern0pt}simp\ add{\isacharcolon}{\kern0pt}comp{\isacharunderscore}{\kern0pt}def{\isacharparenright}{\kern0pt}\isanewline
\ \ \isacommand{also}\isamarkupfalse%
\ \isacommand{have}\isamarkupfalse%
\ {\isachardoublequoteopen}{\isachardot}{\kern0pt}{\isachardot}{\kern0pt}{\isachardot}{\kern0pt}\ {\isacharequal}{\kern0pt}\ indep{\isacharunderscore}{\kern0pt}vars\ {\isacharparenleft}{\kern0pt}{\isasymlambda}x{\isachardot}{\kern0pt}\ case{\isacharunderscore}{\kern0pt}bool\ M{\isacharprime}{\kern0pt}\ M{\isacharprime}{\kern0pt}\ x{\isacharparenright}{\kern0pt}\ {\isacharparenleft}{\kern0pt}{\isasymlambda}x{\isachardot}{\kern0pt}\ case{\isacharunderscore}{\kern0pt}bool\ {\isacharparenleft}{\kern0pt}f\ i{\isacharparenright}{\kern0pt}\ {\isacharparenleft}{\kern0pt}f\ j{\isacharparenright}{\kern0pt}\ x{\isacharparenright}{\kern0pt}\ UNIV{\isachardoublequoteclose}\isanewline
\ \ \ \ \isacommand{apply}\isamarkupfalse%
\ {\isacharparenleft}{\kern0pt}rule\ indep{\isacharunderscore}{\kern0pt}vars{\isacharunderscore}{\kern0pt}cong{\isacharcomma}{\kern0pt}\ simp{\isacharparenright}{\kern0pt}\isanewline
\ \ \ \ \isacommand{apply}\isamarkupfalse%
\ {\isacharparenleft}{\kern0pt}metis\ bool{\isachardot}{\kern0pt}case{\isacharunderscore}{\kern0pt}distrib{\isacharparenright}{\kern0pt}\isanewline
\ \ \ \ \isacommand{by}\isamarkupfalse%
\ {\isacharparenleft}{\kern0pt}simp\ add{\isacharcolon}{\kern0pt}\ bool{\isachardot}{\kern0pt}case{\isacharunderscore}{\kern0pt}eq{\isacharunderscore}{\kern0pt}if{\isacharparenright}{\kern0pt}\isanewline
\ \ \isacommand{also}\isamarkupfalse%
\ \isacommand{have}\isamarkupfalse%
\ {\isachardoublequoteopen}{\isachardot}{\kern0pt}{\isachardot}{\kern0pt}{\isachardot}{\kern0pt}\ {\isacharequal}{\kern0pt}\ {\isacharquery}{\kern0pt}thesis{\isachardoublequoteclose}\isanewline
\ \ \ \ \isacommand{apply}\isamarkupfalse%
\ {\isacharparenleft}{\kern0pt}subst\ indep{\isacharunderscore}{\kern0pt}var{\isacharunderscore}{\kern0pt}def{\isacharparenright}{\kern0pt}\ \isacommand{by}\isamarkupfalse%
\ simp\isanewline
\ \ \isacommand{finally}\isamarkupfalse%
\ \isacommand{show}\isamarkupfalse%
\ {\isacharquery}{\kern0pt}thesis\ \isacommand{by}\isamarkupfalse%
\ simp\isanewline
\isacommand{qed}\isamarkupfalse%
%
\endisatagproof
{\isafoldproof}%
%
\isadelimproof
\isanewline
%
\endisadelimproof
\isanewline
\isacommand{lemma}\isamarkupfalse%
\ {\isacharparenleft}{\kern0pt}\isakeyword{in}\ prob{\isacharunderscore}{\kern0pt}space{\isacharparenright}{\kern0pt}\ var{\isacharunderscore}{\kern0pt}sum{\isacharunderscore}{\kern0pt}pairwise{\isacharunderscore}{\kern0pt}indep{\isacharunderscore}{\kern0pt}{\isadigit{2}}{\isacharcolon}{\kern0pt}\isanewline
\ \ \isakeyword{fixes}\ f\ {\isacharcolon}{\kern0pt}{\isacharcolon}{\kern0pt}\ {\isachardoublequoteopen}{\isacharprime}{\kern0pt}b\ {\isasymRightarrow}\ {\isacharprime}{\kern0pt}a\ {\isasymRightarrow}\ real{\isachardoublequoteclose}\isanewline
\ \ \isakeyword{assumes}\ {\isachardoublequoteopen}finite\ I{\isachardoublequoteclose}\isanewline
\ \ \isakeyword{assumes}\ {\isachardoublequoteopen}{\isasymAnd}i{\isachardot}{\kern0pt}\ i\ {\isasymin}\ I\ {\isasymLongrightarrow}\ f\ i\ {\isasymin}\ borel{\isacharunderscore}{\kern0pt}measurable\ M{\isachardoublequoteclose}\isanewline
\ \ \isakeyword{assumes}\ {\isachardoublequoteopen}{\isasymAnd}i{\isachardot}{\kern0pt}\ i\ {\isasymin}\ I\ {\isasymLongrightarrow}\ integrable\ M\ {\isacharparenleft}{\kern0pt}{\isasymlambda}{\isasymomega}{\isachardot}{\kern0pt}\ f\ i\ {\isasymomega}{\isacharcircum}{\kern0pt}{\isadigit{2}}{\isacharparenright}{\kern0pt}{\isachardoublequoteclose}\isanewline
\ \ \isakeyword{assumes}\ {\isachardoublequoteopen}{\isasymAnd}J{\isachardot}{\kern0pt}\ J\ {\isasymsubseteq}\ I\ {\isasymLongrightarrow}\ card\ J\ {\isacharequal}{\kern0pt}\ {\isadigit{2}}\ {\isasymLongrightarrow}\ indep{\isacharunderscore}{\kern0pt}vars\ {\isacharparenleft}{\kern0pt}{\isasymlambda}\ {\isacharunderscore}{\kern0pt}{\isachardot}{\kern0pt}\ borel{\isacharparenright}{\kern0pt}\ f\ J{\isachardoublequoteclose}\isanewline
\ \ \isakeyword{shows}\ {\isachardoublequoteopen}variance\ {\isacharparenleft}{\kern0pt}{\isasymlambda}{\isasymomega}{\isachardot}{\kern0pt}\ {\isacharparenleft}{\kern0pt}{\isasymSum}i\ {\isasymin}\ I{\isachardot}{\kern0pt}\ f\ i\ {\isasymomega}{\isacharparenright}{\kern0pt}{\isacharparenright}{\kern0pt}\ {\isacharequal}{\kern0pt}\ {\isacharparenleft}{\kern0pt}{\isasymSum}i\ {\isasymin}\ I{\isachardot}{\kern0pt}\ variance\ {\isacharparenleft}{\kern0pt}f\ i{\isacharparenright}{\kern0pt}{\isacharparenright}{\kern0pt}{\isachardoublequoteclose}\isanewline
%
\isadelimproof
\ \ %
\endisadelimproof
%
\isatagproof
\isacommand{apply}\isamarkupfalse%
\ {\isacharparenleft}{\kern0pt}rule\ var{\isacharunderscore}{\kern0pt}sum{\isacharunderscore}{\kern0pt}pairwise{\isacharunderscore}{\kern0pt}indep{\isacharbrackleft}{\kern0pt}OF\ assms{\isacharparenleft}{\kern0pt}{\isadigit{1}}{\isacharparenright}{\kern0pt}\ assms{\isacharparenleft}{\kern0pt}{\isadigit{2}}{\isacharparenright}{\kern0pt}\ assms{\isacharparenleft}{\kern0pt}{\isadigit{3}}{\isacharparenright}{\kern0pt}{\isacharbrackright}{\kern0pt}{\isacharcomma}{\kern0pt}\ simp{\isacharcomma}{\kern0pt}\ simp{\isacharparenright}{\kern0pt}\isanewline
\ \ \isacommand{apply}\isamarkupfalse%
\ {\isacharparenleft}{\kern0pt}rule\ indep{\isacharunderscore}{\kern0pt}var{\isacharunderscore}{\kern0pt}from{\isacharunderscore}{\kern0pt}indep{\isacharunderscore}{\kern0pt}vars{\isacharcomma}{\kern0pt}\ simp{\isacharparenright}{\kern0pt}\isanewline
\ \ \isacommand{by}\isamarkupfalse%
\ {\isacharparenleft}{\kern0pt}rule\ assms{\isacharparenleft}{\kern0pt}{\isadigit{4}}{\isacharparenright}{\kern0pt}{\isacharcomma}{\kern0pt}\ simp{\isacharcomma}{\kern0pt}\ simp{\isacharparenright}{\kern0pt}%
\endisatagproof
{\isafoldproof}%
%
\isadelimproof
\isanewline
%
\endisadelimproof
\isanewline
\isacommand{lemma}\isamarkupfalse%
\ {\isacharparenleft}{\kern0pt}\isakeyword{in}\ prob{\isacharunderscore}{\kern0pt}space{\isacharparenright}{\kern0pt}\ var{\isacharunderscore}{\kern0pt}sum{\isacharunderscore}{\kern0pt}all{\isacharunderscore}{\kern0pt}indep{\isacharcolon}{\kern0pt}\isanewline
\ \ \isakeyword{fixes}\ f\ {\isacharcolon}{\kern0pt}{\isacharcolon}{\kern0pt}\ {\isachardoublequoteopen}{\isacharprime}{\kern0pt}b\ {\isasymRightarrow}\ {\isacharprime}{\kern0pt}a\ {\isasymRightarrow}\ real{\isachardoublequoteclose}\isanewline
\ \ \isakeyword{assumes}\ {\isachardoublequoteopen}finite\ I{\isachardoublequoteclose}\isanewline
\ \ \isakeyword{assumes}\ {\isachardoublequoteopen}{\isasymAnd}i{\isachardot}{\kern0pt}\ i\ {\isasymin}\ I\ {\isasymLongrightarrow}\ f\ i\ {\isasymin}\ borel{\isacharunderscore}{\kern0pt}measurable\ M{\isachardoublequoteclose}\isanewline
\ \ \isakeyword{assumes}\ {\isachardoublequoteopen}{\isasymAnd}i{\isachardot}{\kern0pt}\ i\ {\isasymin}\ I\ {\isasymLongrightarrow}\ integrable\ M\ {\isacharparenleft}{\kern0pt}{\isasymlambda}{\isasymomega}{\isachardot}{\kern0pt}\ f\ i\ {\isasymomega}{\isacharcircum}{\kern0pt}{\isadigit{2}}{\isacharparenright}{\kern0pt}{\isachardoublequoteclose}\isanewline
\ \ \isakeyword{assumes}\ {\isachardoublequoteopen}indep{\isacharunderscore}{\kern0pt}vars\ {\isacharparenleft}{\kern0pt}{\isasymlambda}\ {\isacharunderscore}{\kern0pt}{\isachardot}{\kern0pt}\ borel{\isacharparenright}{\kern0pt}\ f\ I{\isachardoublequoteclose}\isanewline
\ \ \isakeyword{shows}\ {\isachardoublequoteopen}variance\ {\isacharparenleft}{\kern0pt}{\isasymlambda}{\isasymomega}{\isachardot}{\kern0pt}\ {\isacharparenleft}{\kern0pt}{\isasymSum}i\ {\isasymin}\ I{\isachardot}{\kern0pt}\ f\ i\ {\isasymomega}{\isacharparenright}{\kern0pt}{\isacharparenright}{\kern0pt}\ {\isacharequal}{\kern0pt}\ {\isacharparenleft}{\kern0pt}{\isasymSum}i\ {\isasymin}\ I{\isachardot}{\kern0pt}\ variance\ {\isacharparenleft}{\kern0pt}f\ i{\isacharparenright}{\kern0pt}{\isacharparenright}{\kern0pt}{\isachardoublequoteclose}\isanewline
%
\isadelimproof
\ \ %
\endisadelimproof
%
\isatagproof
\isacommand{apply}\isamarkupfalse%
\ {\isacharparenleft}{\kern0pt}rule\ var{\isacharunderscore}{\kern0pt}sum{\isacharunderscore}{\kern0pt}pairwise{\isacharunderscore}{\kern0pt}indep{\isacharunderscore}{\kern0pt}{\isadigit{2}}{\isacharbrackleft}{\kern0pt}OF\ assms{\isacharparenleft}{\kern0pt}{\isadigit{1}}{\isacharparenright}{\kern0pt}\ assms{\isacharparenleft}{\kern0pt}{\isadigit{2}}{\isacharparenright}{\kern0pt}\ assms{\isacharparenleft}{\kern0pt}{\isadigit{3}}{\isacharparenright}{\kern0pt}{\isacharbrackright}{\kern0pt}{\isacharcomma}{\kern0pt}\ simp{\isacharcomma}{\kern0pt}\ simp{\isacharparenright}{\kern0pt}\isanewline
\ \ \isacommand{using}\isamarkupfalse%
\ indep{\isacharunderscore}{\kern0pt}vars{\isacharunderscore}{\kern0pt}subset{\isacharbrackleft}{\kern0pt}OF\ assms{\isacharparenleft}{\kern0pt}{\isadigit{4}}{\isacharparenright}{\kern0pt}{\isacharbrackright}{\kern0pt}\ \isacommand{by}\isamarkupfalse%
\ simp%
\endisatagproof
{\isafoldproof}%
%
\isadelimproof
\isanewline
%
\endisadelimproof
%
\isadelimtheory
\isanewline
%
\endisadelimtheory
%
\isatagtheory
\isacommand{end}\isamarkupfalse%
%
\endisatagtheory
{\isafoldtheory}%
%
\isadelimtheory
%
\endisadelimtheory
%
\end{isabellebody}%
\endinput
%:%file=Probability_Ext.tex%:%
%:%11=1%:%
%:%23=3%:%
%:%31=5%:%
%:%32=5%:%
%:%33=6%:%
%:%34=7%:%
%:%35=8%:%
%:%42=8%:%
%:%43=9%:%
%:%44=10%:%
%:%45=10%:%
%:%48=11%:%
%:%52=11%:%
%:%53=11%:%
%:%58=11%:%
%:%61=12%:%
%:%62=13%:%
%:%63=13%:%
%:%66=14%:%
%:%70=14%:%
%:%71=14%:%
%:%76=14%:%
%:%79=15%:%
%:%80=16%:%
%:%81=16%:%
%:%82=17%:%
%:%83=18%:%
%:%86=19%:%
%:%90=19%:%
%:%91=19%:%
%:%92=19%:%
%:%97=19%:%
%:%100=20%:%
%:%101=21%:%
%:%102=21%:%
%:%103=22%:%
%:%104=23%:%
%:%105=24%:%
%:%106=25%:%
%:%107=26%:%
%:%114=27%:%
%:%115=27%:%
%:%116=28%:%
%:%117=28%:%
%:%118=29%:%
%:%119=29%:%
%:%120=30%:%
%:%121=30%:%
%:%122=30%:%
%:%123=31%:%
%:%124=32%:%
%:%125=32%:%
%:%126=33%:%
%:%127=33%:%
%:%128=33%:%
%:%129=34%:%
%:%130=35%:%
%:%131=35%:%
%:%132=36%:%
%:%133=37%:%
%:%134=37%:%
%:%135=38%:%
%:%136=38%:%
%:%137=39%:%
%:%138=39%:%
%:%139=40%:%
%:%140=40%:%
%:%141=41%:%
%:%142=41%:%
%:%143=42%:%
%:%144=42%:%
%:%145=43%:%
%:%146=44%:%
%:%147=44%:%
%:%148=45%:%
%:%149=46%:%
%:%150=46%:%
%:%151=47%:%
%:%152=47%:%
%:%153=48%:%
%:%154=48%:%
%:%155=49%:%
%:%156=49%:%
%:%157=50%:%
%:%158=50%:%
%:%159=50%:%
%:%160=51%:%
%:%161=51%:%
%:%162=52%:%
%:%163=52%:%
%:%164=53%:%
%:%165=53%:%
%:%166=53%:%
%:%167=54%:%
%:%168=54%:%
%:%169=54%:%
%:%170=55%:%
%:%171=56%:%
%:%172=56%:%
%:%173=57%:%
%:%174=57%:%
%:%175=57%:%
%:%176=58%:%
%:%177=58%:%
%:%178=59%:%
%:%179=59%:%
%:%180=59%:%
%:%181=60%:%
%:%182=61%:%
%:%183=61%:%
%:%184=62%:%
%:%185=62%:%
%:%186=63%:%
%:%187=63%:%
%:%188=63%:%
%:%189=64%:%
%:%190=64%:%
%:%191=64%:%
%:%192=65%:%
%:%193=65%:%
%:%194=65%:%
%:%195=66%:%
%:%196=66%:%
%:%197=66%:%
%:%198=67%:%
%:%199=68%:%
%:%200=68%:%
%:%201=69%:%
%:%202=69%:%
%:%203=69%:%
%:%204=70%:%
%:%205=70%:%
%:%206=71%:%
%:%207=72%:%
%:%208=72%:%
%:%209=73%:%
%:%210=73%:%
%:%211=74%:%
%:%212=74%:%
%:%213=74%:%
%:%214=75%:%
%:%215=75%:%
%:%216=76%:%
%:%217=76%:%
%:%218=76%:%
%:%219=77%:%
%:%225=77%:%
%:%228=78%:%
%:%229=79%:%
%:%230=79%:%
%:%231=80%:%
%:%232=81%:%
%:%233=82%:%
%:%234=83%:%
%:%235=84%:%
%:%242=85%:%
%:%243=85%:%
%:%244=86%:%
%:%245=86%:%
%:%246=86%:%
%:%247=86%:%
%:%248=87%:%
%:%249=87%:%
%:%250=88%:%
%:%251=88%:%
%:%252=89%:%
%:%253=89%:%
%:%254=89%:%
%:%255=89%:%
%:%256=90%:%
%:%257=90%:%
%:%258=91%:%
%:%259=91%:%
%:%260=92%:%
%:%261=92%:%
%:%262=92%:%
%:%263=93%:%
%:%264=93%:%
%:%265=94%:%
%:%266=94%:%
%:%267=95%:%
%:%268=95%:%
%:%269=96%:%
%:%270=96%:%
%:%271=97%:%
%:%272=97%:%
%:%273=97%:%
%:%274=98%:%
%:%284=101%:%
%:%285=102%:%
%:%287=104%:%
%:%288=104%:%
%:%289=105%:%
%:%290=106%:%
%:%293=107%:%
%:%297=107%:%
%:%298=107%:%
%:%299=108%:%
%:%300=108%:%
%:%301=109%:%
%:%302=109%:%
%:%303=110%:%
%:%304=110%:%
%:%305=110%:%
%:%310=110%:%
%:%313=111%:%
%:%314=112%:%
%:%315=112%:%
%:%316=113%:%
%:%317=114%:%
%:%324=115%:%
%:%325=115%:%
%:%326=116%:%
%:%327=116%:%
%:%328=117%:%
%:%329=117%:%
%:%330=118%:%
%:%331=118%:%
%:%332=119%:%
%:%333=119%:%
%:%334=120%:%
%:%335=120%:%
%:%336=121%:%
%:%337=121%:%
%:%338=122%:%
%:%339=122%:%
%:%340=123%:%
%:%341=123%:%
%:%342=123%:%
%:%343=124%:%
%:%344=124%:%
%:%345=125%:%
%:%346=125%:%
%:%347=125%:%
%:%348=126%:%
%:%349=126%:%
%:%350=127%:%
%:%351=127%:%
%:%352=128%:%
%:%353=128%:%
%:%354=128%:%
%:%355=129%:%
%:%356=129%:%
%:%357=130%:%
%:%358=130%:%
%:%359=131%:%
%:%360=131%:%
%:%361=132%:%
%:%362=132%:%
%:%363=133%:%
%:%364=133%:%
%:%365=133%:%
%:%366=133%:%
%:%367=134%:%
%:%368=134%:%
%:%369=135%:%
%:%370=135%:%
%:%371=135%:%
%:%372=136%:%
%:%373=136%:%
%:%374=137%:%
%:%375=137%:%
%:%376=138%:%
%:%377=138%:%
%:%378=138%:%
%:%379=138%:%
%:%380=139%:%
%:%386=139%:%
%:%389=140%:%
%:%390=141%:%
%:%391=141%:%
%:%392=142%:%
%:%393=143%:%
%:%400=144%:%
%:%401=144%:%
%:%402=145%:%
%:%403=145%:%
%:%404=146%:%
%:%405=146%:%
%:%406=147%:%
%:%407=148%:%
%:%408=148%:%
%:%409=149%:%
%:%410=149%:%
%:%411=150%:%
%:%412=150%:%
%:%413=150%:%
%:%414=151%:%
%:%415=151%:%
%:%416=152%:%
%:%417=153%:%
%:%418=153%:%
%:%419=154%:%
%:%420=154%:%
%:%421=155%:%
%:%422=156%:%
%:%423=156%:%
%:%424=157%:%
%:%425=157%:%
%:%426=158%:%
%:%427=159%:%
%:%428=159%:%
%:%429=160%:%
%:%430=160%:%
%:%431=161%:%
%:%432=161%:%
%:%433=161%:%
%:%434=162%:%
%:%435=162%:%
%:%436=162%:%
%:%437=163%:%
%:%438=163%:%
%:%439=164%:%
%:%445=164%:%
%:%448=165%:%
%:%449=166%:%
%:%450=166%:%
%:%451=167%:%
%:%452=168%:%
%:%453=169%:%
%:%456=170%:%
%:%460=170%:%
%:%461=170%:%
%:%462=170%:%
%:%467=170%:%
%:%470=171%:%
%:%471=172%:%
%:%472=172%:%
%:%473=173%:%
%:%474=174%:%
%:%475=175%:%
%:%478=176%:%
%:%482=176%:%
%:%483=176%:%
%:%484=177%:%
%:%485=177%:%
%:%486=178%:%
%:%487=178%:%
%:%488=178%:%
%:%493=178%:%
%:%496=179%:%
%:%497=180%:%
%:%498=180%:%
%:%499=181%:%
%:%500=182%:%
%:%503=183%:%
%:%507=183%:%
%:%508=183%:%
%:%509=184%:%
%:%510=184%:%
%:%511=185%:%
%:%512=185%:%
%:%513=185%:%
%:%518=185%:%
%:%521=186%:%
%:%522=187%:%
%:%523=187%:%
%:%524=188%:%
%:%525=189%:%
%:%532=190%:%
%:%533=190%:%
%:%534=191%:%
%:%535=191%:%
%:%536=192%:%
%:%537=192%:%
%:%538=193%:%
%:%539=193%:%
%:%540=193%:%
%:%541=194%:%
%:%542=194%:%
%:%543=195%:%
%:%544=195%:%
%:%545=195%:%
%:%546=195%:%
%:%547=196%:%
%:%548=196%:%
%:%549=197%:%
%:%550=197%:%
%:%551=197%:%
%:%552=197%:%
%:%553=198%:%
%:%559=198%:%
%:%562=199%:%
%:%563=200%:%
%:%564=200%:%
%:%565=201%:%
%:%566=202%:%
%:%567=203%:%
%:%568=203%:%
%:%569=204%:%
%:%570=205%:%
%:%571=206%:%
%:%572=207%:%
%:%575=208%:%
%:%579=208%:%
%:%580=208%:%
%:%581=209%:%
%:%582=209%:%
%:%583=210%:%
%:%584=210%:%
%:%585=211%:%
%:%586=211%:%
%:%587=212%:%
%:%588=212%:%
%:%589=212%:%
%:%590=213%:%
%:%591=213%:%
%:%592=213%:%
%:%593=214%:%
%:%594=214%:%
%:%595=214%:%
%:%596=215%:%
%:%597=215%:%
%:%598=215%:%
%:%599=216%:%
%:%600=216%:%
%:%601=216%:%
%:%602=217%:%
%:%603=217%:%
%:%604=218%:%
%:%605=218%:%
%:%606=219%:%
%:%607=219%:%
%:%608=220%:%
%:%609=220%:%
%:%614=220%:%
%:%617=221%:%
%:%618=222%:%
%:%619=222%:%
%:%620=223%:%
%:%621=224%:%
%:%622=225%:%
%:%623=226%:%
%:%630=227%:%
%:%631=227%:%
%:%632=228%:%
%:%633=228%:%
%:%634=228%:%
%:%635=228%:%
%:%636=229%:%
%:%637=229%:%
%:%638=229%:%
%:%639=229%:%
%:%640=229%:%
%:%641=230%:%
%:%642=230%:%
%:%643=230%:%
%:%644=231%:%
%:%645=231%:%
%:%646=232%:%
%:%647=232%:%
%:%648=233%:%
%:%654=233%:%
%:%657=234%:%
%:%658=235%:%
%:%659=235%:%
%:%660=236%:%
%:%661=237%:%
%:%662=238%:%
%:%663=239%:%
%:%670=240%:%
%:%671=240%:%
%:%672=241%:%
%:%673=241%:%
%:%674=241%:%
%:%675=241%:%
%:%676=242%:%
%:%677=242%:%
%:%678=242%:%
%:%679=242%:%
%:%680=242%:%
%:%681=243%:%
%:%682=243%:%
%:%683=243%:%
%:%684=243%:%
%:%685=243%:%
%:%686=244%:%
%:%692=244%:%
%:%695=245%:%
%:%696=246%:%
%:%697=246%:%
%:%698=247%:%
%:%699=248%:%
%:%700=249%:%
%:%701=250%:%
%:%702=251%:%
%:%705=252%:%
%:%709=252%:%
%:%710=252%:%
%:%711=253%:%
%:%712=253%:%
%:%713=254%:%
%:%714=254%:%
%:%715=255%:%
%:%716=255%:%
%:%717=256%:%
%:%718=256%:%
%:%719=256%:%
%:%724=256%:%
%:%727=257%:%
%:%728=258%:%
%:%729=258%:%
%:%730=259%:%
%:%731=260%:%
%:%732=261%:%
%:%733=262%:%
%:%734=263%:%
%:%735=264%:%
%:%742=265%:%
%:%743=265%:%
%:%744=266%:%
%:%745=266%:%
%:%746=267%:%
%:%747=267%:%
%:%748=267%:%
%:%749=268%:%
%:%750=268%:%
%:%751=269%:%
%:%752=269%:%
%:%753=270%:%
%:%754=270%:%
%:%755=271%:%
%:%756=271%:%
%:%757=272%:%
%:%758=272%:%
%:%759=272%:%
%:%760=273%:%
%:%761=273%:%
%:%762=274%:%
%:%763=274%:%
%:%764=275%:%
%:%765=275%:%
%:%766=276%:%
%:%767=276%:%
%:%768=277%:%
%:%769=277%:%
%:%770=277%:%
%:%771=278%:%
%:%772=278%:%
%:%773=278%:%
%:%774=279%:%
%:%775=279%:%
%:%776=279%:%
%:%777=279%:%
%:%778=280%:%
%:%784=280%:%
%:%787=281%:%
%:%788=282%:%
%:%789=282%:%
%:%790=283%:%
%:%791=284%:%
%:%794=285%:%
%:%798=285%:%
%:%799=285%:%
%:%804=285%:%
%:%807=286%:%
%:%808=287%:%
%:%809=287%:%
%:%810=288%:%
%:%811=289%:%
%:%812=290%:%
%:%813=291%:%
%:%814=292%:%
%:%821=293%:%
%:%822=293%:%
%:%823=294%:%
%:%824=294%:%
%:%825=295%:%
%:%826=295%:%
%:%827=296%:%
%:%828=297%:%
%:%829=297%:%
%:%830=298%:%
%:%831=298%:%
%:%832=299%:%
%:%833=299%:%
%:%834=300%:%
%:%835=300%:%
%:%836=300%:%
%:%837=301%:%
%:%843=301%:%
%:%846=302%:%
%:%847=303%:%
%:%848=303%:%
%:%849=304%:%
%:%850=305%:%
%:%851=306%:%
%:%852=307%:%
%:%853=308%:%
%:%854=309%:%
%:%857=310%:%
%:%861=310%:%
%:%862=310%:%
%:%863=311%:%
%:%864=311%:%
%:%865=312%:%
%:%866=312%:%
%:%867=313%:%
%:%868=313%:%
%:%869=314%:%
%:%870=314%:%
%:%871=315%:%
%:%872=315%:%
%:%877=315%:%
%:%880=316%:%
%:%881=317%:%
%:%882=317%:%
%:%883=318%:%
%:%884=319%:%
%:%885=320%:%
%:%886=321%:%
%:%887=322%:%
%:%888=323%:%
%:%895=324%:%
%:%896=324%:%
%:%897=325%:%
%:%898=325%:%
%:%899=326%:%
%:%900=326%:%
%:%901=327%:%
%:%902=327%:%
%:%903=327%:%
%:%904=328%:%
%:%905=329%:%
%:%906=329%:%
%:%907=330%:%
%:%908=330%:%
%:%909=331%:%
%:%910=332%:%
%:%911=332%:%
%:%912=333%:%
%:%913=333%:%
%:%914=334%:%
%:%920=334%:%
%:%923=335%:%
%:%924=336%:%
%:%925=336%:%
%:%926=337%:%
%:%927=338%:%
%:%928=339%:%
%:%935=340%:%
%:%936=340%:%
%:%937=341%:%
%:%938=341%:%
%:%939=341%:%
%:%940=342%:%
%:%941=342%:%
%:%942=343%:%
%:%943=343%:%
%:%944=344%:%
%:%945=344%:%
%:%946=345%:%
%:%947=345%:%
%:%948=345%:%
%:%949=345%:%
%:%950=346%:%
%:%951=347%:%
%:%952=347%:%
%:%953=348%:%
%:%954=348%:%
%:%955=349%:%
%:%956=349%:%
%:%957=350%:%
%:%958=350%:%
%:%959=350%:%
%:%960=351%:%
%:%961=351%:%
%:%962=352%:%
%:%963=352%:%
%:%964=353%:%
%:%965=353%:%
%:%966=354%:%
%:%967=354%:%
%:%968=354%:%
%:%969=355%:%
%:%970=355%:%
%:%971=355%:%
%:%972=356%:%
%:%973=356%:%
%:%974=356%:%
%:%975=356%:%
%:%976=357%:%
%:%982=357%:%
%:%985=358%:%
%:%986=359%:%
%:%987=359%:%
%:%988=360%:%
%:%989=361%:%
%:%990=362%:%
%:%991=363%:%
%:%992=364%:%
%:%993=365%:%
%:%996=366%:%
%:%1000=366%:%
%:%1001=366%:%
%:%1002=367%:%
%:%1003=367%:%
%:%1004=368%:%
%:%1005=368%:%
%:%1010=368%:%
%:%1013=369%:%
%:%1014=370%:%
%:%1015=370%:%
%:%1016=371%:%
%:%1017=372%:%
%:%1018=373%:%
%:%1019=374%:%
%:%1020=375%:%
%:%1021=376%:%
%:%1024=377%:%
%:%1028=377%:%
%:%1029=377%:%
%:%1030=378%:%
%:%1031=378%:%
%:%1032=378%:%
%:%1037=378%:%
%:%1042=379%:%
%:%1047=380%:%

%
\begin{isabellebody}%
\setisabellecontext{Median}%
%
\isadelimdocument
%
\endisadelimdocument
%
\isatagdocument
%
\isamarkupsection{Median%
}
\isamarkuptrue%
%
\endisatagdocument
{\isafolddocument}%
%
\isadelimdocument
%
\endisadelimdocument
%
\isadelimtheory
%
\endisadelimtheory
%
\isatagtheory
\isacommand{theory}\isamarkupfalse%
\ Median\isanewline
\ \ \isakeyword{imports}\ Main\ {\isachardoublequoteopen}HOL{\isacharminus}{\kern0pt}Probability{\isachardot}{\kern0pt}Hoeffding{\isachardoublequoteclose}\ {\isachardoublequoteopen}HOL{\isacharminus}{\kern0pt}Library{\isachardot}{\kern0pt}Multiset{\isachardoublequoteclose}\ Probability{\isacharunderscore}{\kern0pt}Ext\ {\isachardoublequoteopen}HOL{\isachardot}{\kern0pt}List{\isachardoublequoteclose}\isanewline
\isakeyword{begin}%
\endisatagtheory
{\isafoldtheory}%
%
\isadelimtheory
%
\endisadelimtheory
%
\begin{isamarkuptext}%
This section includes an amplification result for estimation algorithms using the median
method.%
\end{isamarkuptext}\isamarkuptrue%
\isacommand{fun}\isamarkupfalse%
\ sort{\isacharunderscore}{\kern0pt}primitive\ \isakeyword{where}\isanewline
\ \ {\isachardoublequoteopen}sort{\isacharunderscore}{\kern0pt}primitive\ i\ j\ f\ k\ {\isacharequal}{\kern0pt}\ {\isacharparenleft}{\kern0pt}if\ k\ {\isacharequal}{\kern0pt}\ i\ then\ min\ {\isacharparenleft}{\kern0pt}f\ i{\isacharparenright}{\kern0pt}\ {\isacharparenleft}{\kern0pt}f\ j{\isacharparenright}{\kern0pt}\ else\ {\isacharparenleft}{\kern0pt}if\ k\ {\isacharequal}{\kern0pt}\ j\ then\ max\ {\isacharparenleft}{\kern0pt}f\ i{\isacharparenright}{\kern0pt}\ {\isacharparenleft}{\kern0pt}f\ j{\isacharparenright}{\kern0pt}\ else\ f\ k{\isacharparenright}{\kern0pt}{\isacharparenright}{\kern0pt}{\isachardoublequoteclose}\isanewline
\isanewline
\isacommand{fun}\isamarkupfalse%
\ sort{\isacharunderscore}{\kern0pt}map\ \isakeyword{where}\isanewline
\ \ {\isachardoublequoteopen}sort{\isacharunderscore}{\kern0pt}map\ f\ n\ {\isacharequal}{\kern0pt}\ fold\ id\ {\isacharbrackleft}{\kern0pt}sort{\isacharunderscore}{\kern0pt}primitive\ j\ i{\isachardot}{\kern0pt}\ i\ {\isacharless}{\kern0pt}{\isacharminus}{\kern0pt}\ {\isacharbrackleft}{\kern0pt}{\isadigit{0}}{\isachardot}{\kern0pt}{\isachardot}{\kern0pt}{\isacharless}{\kern0pt}n{\isacharbrackright}{\kern0pt}{\isacharcomma}{\kern0pt}\ j\ {\isacharless}{\kern0pt}{\isacharminus}{\kern0pt}\ {\isacharbrackleft}{\kern0pt}{\isadigit{0}}{\isachardot}{\kern0pt}{\isachardot}{\kern0pt}{\isacharless}{\kern0pt}i{\isacharbrackright}{\kern0pt}{\isacharbrackright}{\kern0pt}\ f{\isachardoublequoteclose}\isanewline
\isanewline
\isacommand{lemma}\isamarkupfalse%
\ sort{\isacharunderscore}{\kern0pt}map{\isacharunderscore}{\kern0pt}ind{\isacharcolon}{\kern0pt}\isanewline
\ \ {\isachardoublequoteopen}sort{\isacharunderscore}{\kern0pt}map\ f\ {\isacharparenleft}{\kern0pt}Suc\ n{\isacharparenright}{\kern0pt}\ {\isacharequal}{\kern0pt}\ fold\ id\ {\isacharbrackleft}{\kern0pt}sort{\isacharunderscore}{\kern0pt}primitive\ j\ n{\isachardot}{\kern0pt}\ j\ {\isacharless}{\kern0pt}{\isacharminus}{\kern0pt}\ {\isacharbrackleft}{\kern0pt}{\isadigit{0}}{\isachardot}{\kern0pt}{\isachardot}{\kern0pt}{\isacharless}{\kern0pt}n{\isacharbrackright}{\kern0pt}{\isacharbrackright}{\kern0pt}\ {\isacharparenleft}{\kern0pt}sort{\isacharunderscore}{\kern0pt}map\ f\ n{\isacharparenright}{\kern0pt}{\isachardoublequoteclose}\isanewline
%
\isadelimproof
\ \ %
\endisadelimproof
%
\isatagproof
\isacommand{by}\isamarkupfalse%
\ simp%
\endisatagproof
{\isafoldproof}%
%
\isadelimproof
\isanewline
%
\endisadelimproof
\isanewline
\isacommand{lemma}\isamarkupfalse%
\ sort{\isacharunderscore}{\kern0pt}map{\isacharunderscore}{\kern0pt}strict{\isacharunderscore}{\kern0pt}mono{\isacharcolon}{\kern0pt}\isanewline
\ \ \isakeyword{fixes}\ f\ {\isacharcolon}{\kern0pt}{\isacharcolon}{\kern0pt}\ {\isachardoublequoteopen}nat\ {\isasymRightarrow}\ {\isacharprime}{\kern0pt}b\ {\isacharcolon}{\kern0pt}{\isacharcolon}{\kern0pt}\ linorder{\isachardoublequoteclose}\isanewline
\ \ \isakeyword{shows}\ {\isachardoublequoteopen}j\ {\isacharless}{\kern0pt}\ n\ {\isasymLongrightarrow}\ i\ {\isacharless}{\kern0pt}\ j\ {\isasymLongrightarrow}\ sort{\isacharunderscore}{\kern0pt}map\ f\ n\ i\ {\isasymle}\ sort{\isacharunderscore}{\kern0pt}map\ f\ n\ j{\isachardoublequoteclose}\isanewline
%
\isadelimproof
%
\endisadelimproof
%
\isatagproof
\isacommand{proof}\isamarkupfalse%
\ {\isacharparenleft}{\kern0pt}induction\ n\ arbitrary{\isacharcolon}{\kern0pt}\ i\ j{\isacharparenright}{\kern0pt}\isanewline
\ \ \isacommand{case}\isamarkupfalse%
\ {\isadigit{0}}\isanewline
\ \ \isacommand{then}\isamarkupfalse%
\ \isacommand{show}\isamarkupfalse%
\ {\isacharquery}{\kern0pt}case\ \isacommand{by}\isamarkupfalse%
\ simp\isanewline
\isacommand{next}\isamarkupfalse%
\isanewline
\ \ \isacommand{case}\isamarkupfalse%
\ {\isacharparenleft}{\kern0pt}Suc\ n{\isacharparenright}{\kern0pt}\isanewline
\ \ \isacommand{define}\isamarkupfalse%
\ g\ \isakeyword{where}\ {\isachardoublequoteopen}g\ {\isacharequal}{\kern0pt}\ {\isacharparenleft}{\kern0pt}{\isasymlambda}k{\isachardot}{\kern0pt}\ fold\ id\ {\isacharbrackleft}{\kern0pt}sort{\isacharunderscore}{\kern0pt}primitive\ j\ n{\isachardot}{\kern0pt}\ j\ {\isacharless}{\kern0pt}{\isacharminus}{\kern0pt}\ {\isacharbrackleft}{\kern0pt}{\isadigit{0}}{\isachardot}{\kern0pt}{\isachardot}{\kern0pt}{\isacharless}{\kern0pt}k{\isacharbrackright}{\kern0pt}{\isacharbrackright}{\kern0pt}\ {\isacharparenleft}{\kern0pt}sort{\isacharunderscore}{\kern0pt}map\ f\ n{\isacharparenright}{\kern0pt}{\isacharparenright}{\kern0pt}{\isachardoublequoteclose}\isanewline
\ \ \isacommand{define}\isamarkupfalse%
\ k\ \isakeyword{where}\ {\isachardoublequoteopen}k\ {\isacharequal}{\kern0pt}\ n{\isachardoublequoteclose}\isanewline
\ \ \isacommand{have}\isamarkupfalse%
\ a{\isacharcolon}{\kern0pt}{\isachardoublequoteopen}{\isacharparenleft}{\kern0pt}{\isasymforall}i\ j{\isachardot}{\kern0pt}\ j\ {\isacharless}{\kern0pt}\ n\ {\isasymlongrightarrow}\ i\ {\isacharless}{\kern0pt}\ j\ {\isasymlongrightarrow}\ g\ k\ i\ {\isasymle}\ g\ k\ j{\isacharparenright}{\kern0pt}\ {\isasymand}\ {\isacharparenleft}{\kern0pt}{\isasymforall}l{\isachardot}{\kern0pt}\ l\ {\isacharless}{\kern0pt}\ k\ {\isasymlongrightarrow}\ g\ k\ l\ {\isasymle}\ g\ k\ n{\isacharparenright}{\kern0pt}{\isachardoublequoteclose}\isanewline
\ \ \isacommand{proof}\isamarkupfalse%
\ {\isacharparenleft}{\kern0pt}induction\ k{\isacharparenright}{\kern0pt}\isanewline
\ \ \ \ \isacommand{case}\isamarkupfalse%
\ {\isadigit{0}}\isanewline
\ \ \ \ \isacommand{then}\isamarkupfalse%
\ \isacommand{show}\isamarkupfalse%
\ {\isacharquery}{\kern0pt}case\ \isacommand{using}\isamarkupfalse%
\ Suc\ \isacommand{by}\isamarkupfalse%
\ {\isacharparenleft}{\kern0pt}simp\ add{\isacharcolon}{\kern0pt}g{\isacharunderscore}{\kern0pt}def\ del{\isacharcolon}{\kern0pt}sort{\isacharunderscore}{\kern0pt}map{\isachardot}{\kern0pt}simps{\isacharparenright}{\kern0pt}\isanewline
\ \ \isacommand{next}\isamarkupfalse%
\isanewline
\ \ \ \ \isacommand{case}\isamarkupfalse%
\ {\isacharparenleft}{\kern0pt}Suc\ k{\isacharparenright}{\kern0pt}\isanewline
\ \ \ \ \isacommand{have}\isamarkupfalse%
\ {\isachardoublequoteopen}g\ {\isacharparenleft}{\kern0pt}Suc\ k{\isacharparenright}{\kern0pt}\ {\isacharequal}{\kern0pt}\ sort{\isacharunderscore}{\kern0pt}primitive\ k\ n\ {\isacharparenleft}{\kern0pt}g\ k{\isacharparenright}{\kern0pt}{\isachardoublequoteclose}\ \isanewline
\ \ \ \ \ \ \isacommand{by}\isamarkupfalse%
\ {\isacharparenleft}{\kern0pt}simp\ add{\isacharcolon}{\kern0pt}g{\isacharunderscore}{\kern0pt}def{\isacharparenright}{\kern0pt}\isanewline
\ \ \ \ \isacommand{then}\isamarkupfalse%
\ \isacommand{show}\isamarkupfalse%
\ {\isacharquery}{\kern0pt}case\ \isacommand{using}\isamarkupfalse%
\ Suc\isanewline
\ \ \ \ \ \ \isacommand{apply}\isamarkupfalse%
\ {\isacharparenleft}{\kern0pt}cases\ {\isachardoublequoteopen}g\ k\ k\ {\isasymle}\ g\ k\ n{\isachardoublequoteclose}{\isacharparenright}{\kern0pt}\isanewline
\ \ \ \ \ \ \ \isacommand{apply}\isamarkupfalse%
\ {\isacharparenleft}{\kern0pt}simp\ add{\isacharcolon}{\kern0pt}min{\isacharunderscore}{\kern0pt}def\ max{\isacharunderscore}{\kern0pt}def{\isacharparenright}{\kern0pt}\isanewline
\ \ \ \ \ \ \isacommand{using}\isamarkupfalse%
\ less{\isacharunderscore}{\kern0pt}antisym\ \isacommand{apply}\isamarkupfalse%
\ blast\isanewline
\ \ \ \ \ \ \isacommand{apply}\isamarkupfalse%
\ {\isacharparenleft}{\kern0pt}cases\ {\isachardoublequoteopen}g\ k\ n\ {\isasymle}\ g\ k\ k{\isachardoublequoteclose}{\isacharparenright}{\kern0pt}\isanewline
\ \ \ \ \ \ \ \isacommand{apply}\isamarkupfalse%
\ {\isacharparenleft}{\kern0pt}simp\ add{\isacharcolon}{\kern0pt}min{\isacharunderscore}{\kern0pt}def\ max{\isacharunderscore}{\kern0pt}def{\isacharparenright}{\kern0pt}\isanewline
\ \ \ \ \ \ \ \isacommand{apply}\isamarkupfalse%
\ {\isacharparenleft}{\kern0pt}metis\ less{\isacharunderscore}{\kern0pt}antisym\ max{\isachardot}{\kern0pt}coboundedI{\isadigit{2}}\ max{\isachardot}{\kern0pt}orderE{\isacharparenright}{\kern0pt}\isanewline
\ \ \ \ \ \ \isacommand{by}\isamarkupfalse%
\ simp\isanewline
\ \ \isacommand{qed}\isamarkupfalse%
\isanewline
\isanewline
\ \ \isacommand{hence}\isamarkupfalse%
\ {\isachardoublequoteopen}{\isasymAnd}i\ j{\isachardot}{\kern0pt}\ j\ {\isacharless}{\kern0pt}\ Suc\ n\ {\isasymLongrightarrow}\ i\ {\isacharless}{\kern0pt}\ j\ {\isasymLongrightarrow}\ g\ n\ i\ {\isasymle}\ g\ n\ j{\isachardoublequoteclose}\isanewline
\ \ \ \ \isacommand{apply}\isamarkupfalse%
\ {\isacharparenleft}{\kern0pt}simp\ add{\isacharcolon}{\kern0pt}k{\isacharunderscore}{\kern0pt}def{\isacharparenright}{\kern0pt}\ \isacommand{using}\isamarkupfalse%
\ less{\isacharunderscore}{\kern0pt}antisym\ \isacommand{by}\isamarkupfalse%
\ blast\isanewline
\ \ \isacommand{moreover}\isamarkupfalse%
\ \isacommand{have}\isamarkupfalse%
\ {\isachardoublequoteopen}sort{\isacharunderscore}{\kern0pt}map\ f\ {\isacharparenleft}{\kern0pt}Suc\ n{\isacharparenright}{\kern0pt}\ {\isacharequal}{\kern0pt}\ g\ n{\isachardoublequoteclose}\ \isanewline
\ \ \ \ \isacommand{by}\isamarkupfalse%
\ {\isacharparenleft}{\kern0pt}simp\ add{\isacharcolon}{\kern0pt}sort{\isacharunderscore}{\kern0pt}map{\isacharunderscore}{\kern0pt}ind\ g{\isacharunderscore}{\kern0pt}def\ del{\isacharcolon}{\kern0pt}sort{\isacharunderscore}{\kern0pt}map{\isachardot}{\kern0pt}simps{\isacharparenright}{\kern0pt}\isanewline
\ \ \isacommand{ultimately}\isamarkupfalse%
\ \isacommand{show}\isamarkupfalse%
\ {\isacharquery}{\kern0pt}case\isanewline
\ \ \ \ \isacommand{apply}\isamarkupfalse%
\ {\isacharparenleft}{\kern0pt}simp\ del{\isacharcolon}{\kern0pt}sort{\isacharunderscore}{\kern0pt}map{\isachardot}{\kern0pt}simps{\isacharparenright}{\kern0pt}\isanewline
\ \ \ \ \isacommand{using}\isamarkupfalse%
\ Suc\ \isacommand{by}\isamarkupfalse%
\ blast\isanewline
\isacommand{qed}\isamarkupfalse%
%
\endisatagproof
{\isafoldproof}%
%
\isadelimproof
\isanewline
%
\endisadelimproof
\isanewline
\isacommand{lemma}\isamarkupfalse%
\ sort{\isacharunderscore}{\kern0pt}map{\isacharunderscore}{\kern0pt}mono{\isacharcolon}{\kern0pt}\isanewline
\ \ \isakeyword{fixes}\ f\ {\isacharcolon}{\kern0pt}{\isacharcolon}{\kern0pt}\ {\isachardoublequoteopen}nat\ {\isasymRightarrow}\ {\isacharprime}{\kern0pt}b\ {\isacharcolon}{\kern0pt}{\isacharcolon}{\kern0pt}\ linorder{\isachardoublequoteclose}\isanewline
\ \ \isakeyword{shows}\ {\isachardoublequoteopen}j\ {\isacharless}{\kern0pt}\ n\ {\isasymLongrightarrow}\ i\ {\isasymle}\ j\ {\isasymLongrightarrow}\ sort{\isacharunderscore}{\kern0pt}map\ f\ n\ i\ {\isasymle}\ sort{\isacharunderscore}{\kern0pt}map\ f\ n\ j{\isachardoublequoteclose}\isanewline
%
\isadelimproof
\ \ %
\endisadelimproof
%
\isatagproof
\isacommand{using}\isamarkupfalse%
\ sort{\isacharunderscore}{\kern0pt}map{\isacharunderscore}{\kern0pt}strict{\isacharunderscore}{\kern0pt}mono\ \isanewline
\ \ \isacommand{by}\isamarkupfalse%
\ {\isacharparenleft}{\kern0pt}metis\ eq{\isacharunderscore}{\kern0pt}iff\ le{\isacharunderscore}{\kern0pt}imp{\isacharunderscore}{\kern0pt}less{\isacharunderscore}{\kern0pt}or{\isacharunderscore}{\kern0pt}eq{\isacharparenright}{\kern0pt}%
\endisatagproof
{\isafoldproof}%
%
\isadelimproof
\isanewline
%
\endisadelimproof
\isanewline
\isacommand{lemma}\isamarkupfalse%
\ sort{\isacharunderscore}{\kern0pt}map{\isacharunderscore}{\kern0pt}perm{\isacharcolon}{\kern0pt}\isanewline
\ \ \isakeyword{fixes}\ f\ {\isacharcolon}{\kern0pt}{\isacharcolon}{\kern0pt}\ {\isachardoublequoteopen}nat\ {\isasymRightarrow}\ {\isacharprime}{\kern0pt}b\ {\isacharcolon}{\kern0pt}{\isacharcolon}{\kern0pt}\ linorder{\isachardoublequoteclose}\isanewline
\ \ \isakeyword{shows}\ {\isachardoublequoteopen}image{\isacharunderscore}{\kern0pt}mset\ {\isacharparenleft}{\kern0pt}sort{\isacharunderscore}{\kern0pt}map\ f\ n{\isacharparenright}{\kern0pt}\ {\isacharparenleft}{\kern0pt}mset\ {\isacharbrackleft}{\kern0pt}{\isadigit{0}}{\isachardot}{\kern0pt}{\isachardot}{\kern0pt}{\isacharless}{\kern0pt}n{\isacharbrackright}{\kern0pt}{\isacharparenright}{\kern0pt}\ {\isacharequal}{\kern0pt}\ image{\isacharunderscore}{\kern0pt}mset\ f\ {\isacharparenleft}{\kern0pt}mset\ {\isacharbrackleft}{\kern0pt}{\isadigit{0}}{\isachardot}{\kern0pt}{\isachardot}{\kern0pt}{\isacharless}{\kern0pt}n{\isacharbrackright}{\kern0pt}{\isacharparenright}{\kern0pt}{\isachardoublequoteclose}\isanewline
%
\isadelimproof
%
\endisadelimproof
%
\isatagproof
\isacommand{proof}\isamarkupfalse%
\ {\isacharminus}{\kern0pt}\isanewline
\ \ \isacommand{define}\isamarkupfalse%
\ is{\isacharunderscore}{\kern0pt}swap\ \isakeyword{where}\ {\isachardoublequoteopen}is{\isacharunderscore}{\kern0pt}swap\ {\isacharequal}{\kern0pt}\ {\isacharparenleft}{\kern0pt}{\isasymlambda}{\isacharparenleft}{\kern0pt}ts\ {\isacharcolon}{\kern0pt}{\isacharcolon}{\kern0pt}\ {\isacharparenleft}{\kern0pt}{\isacharparenleft}{\kern0pt}nat\ {\isasymRightarrow}\ {\isacharprime}{\kern0pt}b{\isacharparenright}{\kern0pt}\ {\isasymRightarrow}\ nat\ {\isasymRightarrow}\ {\isacharprime}{\kern0pt}b{\isacharparenright}{\kern0pt}{\isacharparenright}{\kern0pt}{\isachardot}{\kern0pt}\ {\isasymexists}i\ {\isacharless}{\kern0pt}\ n{\isachardot}{\kern0pt}\ {\isasymexists}j\ {\isacharless}{\kern0pt}\ n{\isachardot}{\kern0pt}\ ts\ {\isacharequal}{\kern0pt}\ sort{\isacharunderscore}{\kern0pt}primitive\ i\ j{\isacharparenright}{\kern0pt}{\isachardoublequoteclose}\isanewline
\ \ \isacommand{define}\isamarkupfalse%
\ t\ {\isacharcolon}{\kern0pt}{\isacharcolon}{\kern0pt}\ {\isachardoublequoteopen}{\isacharparenleft}{\kern0pt}{\isacharparenleft}{\kern0pt}nat\ {\isasymRightarrow}\ {\isacharprime}{\kern0pt}b{\isacharparenright}{\kern0pt}\ {\isasymRightarrow}\ nat\ {\isasymRightarrow}\ {\isacharprime}{\kern0pt}b{\isacharparenright}{\kern0pt}\ list{\isachardoublequoteclose}\ \isanewline
\ \ \ \ \isakeyword{where}\ {\isachardoublequoteopen}t\ {\isacharequal}{\kern0pt}\ {\isacharbrackleft}{\kern0pt}sort{\isacharunderscore}{\kern0pt}primitive\ j\ i{\isachardot}{\kern0pt}\ i\ {\isacharless}{\kern0pt}{\isacharminus}{\kern0pt}\ {\isacharbrackleft}{\kern0pt}{\isadigit{0}}{\isachardot}{\kern0pt}{\isachardot}{\kern0pt}{\isacharless}{\kern0pt}n{\isacharbrackright}{\kern0pt}{\isacharcomma}{\kern0pt}\ j\ {\isacharless}{\kern0pt}{\isacharminus}{\kern0pt}\ {\isacharbrackleft}{\kern0pt}{\isadigit{0}}{\isachardot}{\kern0pt}{\isachardot}{\kern0pt}{\isacharless}{\kern0pt}i{\isacharbrackright}{\kern0pt}{\isacharbrackright}{\kern0pt}{\isachardoublequoteclose}\isanewline
\isanewline
\ \ \isacommand{have}\isamarkupfalse%
\ a{\isacharcolon}{\kern0pt}\ {\isachardoublequoteopen}{\isasymAnd}x\ f{\isachardot}{\kern0pt}\ is{\isacharunderscore}{\kern0pt}swap\ x\ {\isasymLongrightarrow}\ image{\isacharunderscore}{\kern0pt}mset\ {\isacharparenleft}{\kern0pt}x\ f{\isacharparenright}{\kern0pt}\ {\isacharparenleft}{\kern0pt}mset{\isacharunderscore}{\kern0pt}set\ {\isacharbraceleft}{\kern0pt}{\isadigit{0}}{\isachardot}{\kern0pt}{\isachardot}{\kern0pt}{\isacharless}{\kern0pt}n{\isacharbraceright}{\kern0pt}{\isacharparenright}{\kern0pt}\ {\isacharequal}{\kern0pt}\ image{\isacharunderscore}{\kern0pt}mset\ f\ {\isacharparenleft}{\kern0pt}mset{\isacharunderscore}{\kern0pt}set\ {\isacharbraceleft}{\kern0pt}{\isadigit{0}}{\isachardot}{\kern0pt}{\isachardot}{\kern0pt}{\isacharless}{\kern0pt}n{\isacharbraceright}{\kern0pt}{\isacharparenright}{\kern0pt}{\isachardoublequoteclose}\isanewline
\ \ \isacommand{proof}\isamarkupfalse%
\ {\isacharminus}{\kern0pt}\isanewline
\ \ \ \ \isacommand{fix}\isamarkupfalse%
\ x\isanewline
\ \ \ \ \isacommand{fix}\isamarkupfalse%
\ f\ {\isacharcolon}{\kern0pt}{\isacharcolon}{\kern0pt}\ {\isachardoublequoteopen}nat\ {\isasymRightarrow}\ {\isacharprime}{\kern0pt}b\ {\isacharcolon}{\kern0pt}{\isacharcolon}{\kern0pt}\ linorder{\isachardoublequoteclose}\isanewline
\ \ \ \ \isacommand{assume}\isamarkupfalse%
\ {\isachardoublequoteopen}is{\isacharunderscore}{\kern0pt}swap\ x{\isachardoublequoteclose}\isanewline
\ \ \ \ \isacommand{then}\isamarkupfalse%
\ \isacommand{obtain}\isamarkupfalse%
\ i\ j\ \isakeyword{where}\ x{\isacharunderscore}{\kern0pt}def{\isacharcolon}{\kern0pt}\ {\isachardoublequoteopen}x\ {\isacharequal}{\kern0pt}\ sort{\isacharunderscore}{\kern0pt}primitive\ i\ j{\isachardoublequoteclose}\ \isakeyword{and}\ i{\isacharunderscore}{\kern0pt}bound{\isacharcolon}{\kern0pt}\ {\isachardoublequoteopen}i\ {\isacharless}{\kern0pt}\ n{\isachardoublequoteclose}\ \isakeyword{and}\ j{\isacharunderscore}{\kern0pt}bound{\isacharcolon}{\kern0pt}{\isachardoublequoteopen}j\ {\isacharless}{\kern0pt}\ n{\isachardoublequoteclose}\ \isanewline
\ \ \ \ \ \ \isacommand{using}\isamarkupfalse%
\ is{\isacharunderscore}{\kern0pt}swap{\isacharunderscore}{\kern0pt}def\ \isacommand{by}\isamarkupfalse%
\ blast\isanewline
\ \ \ \ \isacommand{define}\isamarkupfalse%
\ inv\ \isakeyword{where}\ {\isachardoublequoteopen}inv\ {\isacharequal}{\kern0pt}\ mset{\isacharunderscore}{\kern0pt}set\ {\isacharbraceleft}{\kern0pt}k{\isachardot}{\kern0pt}\ k\ {\isacharless}{\kern0pt}\ n\ {\isasymand}\ k\ {\isasymnoteq}\ i\ {\isasymand}\ k\ {\isasymnoteq}\ j{\isacharbraceright}{\kern0pt}{\isachardoublequoteclose}\isanewline
\ \ \ \ \isacommand{have}\isamarkupfalse%
\ b{\isacharcolon}{\kern0pt}{\isachardoublequoteopen}{\isacharbraceleft}{\kern0pt}{\isadigit{0}}{\isachardot}{\kern0pt}{\isachardot}{\kern0pt}{\isacharless}{\kern0pt}n{\isacharbraceright}{\kern0pt}\ {\isacharequal}{\kern0pt}\ {\isacharbraceleft}{\kern0pt}k{\isachardot}{\kern0pt}\ k\ {\isacharless}{\kern0pt}\ n\ {\isasymand}\ k\ {\isasymnoteq}\ i\ {\isasymand}\ k\ {\isasymnoteq}\ j{\isacharbraceright}{\kern0pt}\ {\isasymunion}\ {\isacharbraceleft}{\kern0pt}i{\isacharcomma}{\kern0pt}j{\isacharbraceright}{\kern0pt}{\isachardoublequoteclose}\isanewline
\ \ \ \ \ \ \isacommand{apply}\isamarkupfalse%
\ {\isacharparenleft}{\kern0pt}rule\ order{\isacharunderscore}{\kern0pt}antisym{\isacharcomma}{\kern0pt}\ rule\ subsetI{\isacharcomma}{\kern0pt}\ simp{\isacharcomma}{\kern0pt}\ blast{\isacharcomma}{\kern0pt}\ rule\ subsetI{\isacharcomma}{\kern0pt}\ simp{\isacharparenright}{\kern0pt}\isanewline
\ \ \ \ \ \ \isacommand{using}\isamarkupfalse%
\ i{\isacharunderscore}{\kern0pt}bound\ j{\isacharunderscore}{\kern0pt}bound\ \isacommand{by}\isamarkupfalse%
\ meson\isanewline
\ \ \ \ \isacommand{have}\isamarkupfalse%
\ c{\isacharcolon}{\kern0pt}{\isachardoublequoteopen}{\isasymAnd}k{\isachardot}{\kern0pt}\ k\ {\isasymin}{\isacharhash}{\kern0pt}\ inv\ {\isasymLongrightarrow}\ {\isacharparenleft}{\kern0pt}x\ f{\isacharparenright}{\kern0pt}\ k\ {\isacharequal}{\kern0pt}\ f\ k{\isachardoublequoteclose}\ \isanewline
\ \ \ \ \ \ \isacommand{by}\isamarkupfalse%
\ {\isacharparenleft}{\kern0pt}simp\ add{\isacharcolon}{\kern0pt}x{\isacharunderscore}{\kern0pt}def\ inv{\isacharunderscore}{\kern0pt}def{\isacharparenright}{\kern0pt}\isanewline
\ \ \ \ \isacommand{have}\isamarkupfalse%
\ {\isachardoublequoteopen}image{\isacharunderscore}{\kern0pt}mset\ {\isacharparenleft}{\kern0pt}x\ f{\isacharparenright}{\kern0pt}\ inv\ {\isacharequal}{\kern0pt}\ image{\isacharunderscore}{\kern0pt}mset\ f\ inv{\isachardoublequoteclose}\isanewline
\ \ \ \ \ \ \isacommand{apply}\isamarkupfalse%
\ {\isacharparenleft}{\kern0pt}rule\ multiset{\isacharunderscore}{\kern0pt}eqI{\isacharparenright}{\kern0pt}\isanewline
\ \ \ \ \ \ \isacommand{using}\isamarkupfalse%
\ c\ multiset{\isachardot}{\kern0pt}map{\isacharunderscore}{\kern0pt}cong{\isadigit{0}}\ \isacommand{by}\isamarkupfalse%
\ force\isanewline
\ \ \ \ \isacommand{moreover}\isamarkupfalse%
\ \isacommand{have}\isamarkupfalse%
\ {\isachardoublequoteopen}image{\isacharunderscore}{\kern0pt}mset\ {\isacharparenleft}{\kern0pt}x\ f{\isacharparenright}{\kern0pt}\ {\isacharparenleft}{\kern0pt}mset{\isacharunderscore}{\kern0pt}set\ {\isacharbraceleft}{\kern0pt}i{\isacharcomma}{\kern0pt}j{\isacharbraceright}{\kern0pt}{\isacharparenright}{\kern0pt}\ {\isacharequal}{\kern0pt}\ image{\isacharunderscore}{\kern0pt}mset\ f\ {\isacharparenleft}{\kern0pt}mset{\isacharunderscore}{\kern0pt}set\ {\isacharbraceleft}{\kern0pt}i{\isacharcomma}{\kern0pt}j{\isacharbraceright}{\kern0pt}{\isacharparenright}{\kern0pt}{\isachardoublequoteclose}\isanewline
\ \ \ \ \ \ \isacommand{apply}\isamarkupfalse%
\ {\isacharparenleft}{\kern0pt}cases\ {\isachardoublequoteopen}i\ {\isacharequal}{\kern0pt}\ j{\isachardoublequoteclose}{\isacharparenright}{\kern0pt}\isanewline
\ \ \ \ \ \ \isacommand{by}\isamarkupfalse%
\ {\isacharparenleft}{\kern0pt}simp\ add{\isacharcolon}{\kern0pt}x{\isacharunderscore}{\kern0pt}def\ max{\isacharunderscore}{\kern0pt}def\ min{\isacharunderscore}{\kern0pt}def{\isacharparenright}{\kern0pt}{\isacharplus}{\kern0pt}\isanewline
\ \ \ \ \isacommand{moreover}\isamarkupfalse%
\ \isacommand{have}\isamarkupfalse%
\ {\isachardoublequoteopen}mset{\isacharunderscore}{\kern0pt}set\ {\isacharbraceleft}{\kern0pt}{\isadigit{0}}{\isachardot}{\kern0pt}{\isachardot}{\kern0pt}{\isacharless}{\kern0pt}n{\isacharbraceright}{\kern0pt}\ {\isacharequal}{\kern0pt}\ inv\ {\isacharplus}{\kern0pt}\ mset{\isacharunderscore}{\kern0pt}set\ {\isacharbraceleft}{\kern0pt}i{\isacharcomma}{\kern0pt}j{\isacharbraceright}{\kern0pt}{\isachardoublequoteclose}\isanewline
\ \ \ \ \ \ \isacommand{by}\isamarkupfalse%
\ {\isacharparenleft}{\kern0pt}simp\ only{\isacharcolon}{\kern0pt}inv{\isacharunderscore}{\kern0pt}def\ b{\isacharcomma}{\kern0pt}\ rule\ mset{\isacharunderscore}{\kern0pt}set{\isacharunderscore}{\kern0pt}Union{\isacharcomma}{\kern0pt}\ simp{\isacharcomma}{\kern0pt}\ simp{\isacharcomma}{\kern0pt}\ simp{\isacharparenright}{\kern0pt}\ \isanewline
\ \ \ \ \isacommand{ultimately}\isamarkupfalse%
\ \isacommand{show}\isamarkupfalse%
\ {\isachardoublequoteopen}image{\isacharunderscore}{\kern0pt}mset\ {\isacharparenleft}{\kern0pt}x\ f{\isacharparenright}{\kern0pt}\ {\isacharparenleft}{\kern0pt}mset{\isacharunderscore}{\kern0pt}set\ {\isacharbraceleft}{\kern0pt}{\isadigit{0}}{\isachardot}{\kern0pt}{\isachardot}{\kern0pt}{\isacharless}{\kern0pt}n{\isacharbraceright}{\kern0pt}{\isacharparenright}{\kern0pt}\ {\isacharequal}{\kern0pt}\ image{\isacharunderscore}{\kern0pt}mset\ f\ {\isacharparenleft}{\kern0pt}mset{\isacharunderscore}{\kern0pt}set\ {\isacharbraceleft}{\kern0pt}{\isadigit{0}}{\isachardot}{\kern0pt}{\isachardot}{\kern0pt}{\isacharless}{\kern0pt}n{\isacharbraceright}{\kern0pt}{\isacharparenright}{\kern0pt}{\isachardoublequoteclose}\isanewline
\ \ \ \ \ \ \isacommand{by}\isamarkupfalse%
\ simp\isanewline
\ \ \isacommand{qed}\isamarkupfalse%
\isanewline
\isanewline
\ \ \isacommand{have}\isamarkupfalse%
\ {\isachardoublequoteopen}{\isacharparenleft}{\kern0pt}{\isasymforall}x\ {\isasymin}\ set\ t{\isachardot}{\kern0pt}\ is{\isacharunderscore}{\kern0pt}swap\ x{\isacharparenright}{\kern0pt}\ {\isasymLongrightarrow}\ image{\isacharunderscore}{\kern0pt}mset\ {\isacharparenleft}{\kern0pt}fold\ id\ t\ f{\isacharparenright}{\kern0pt}\ {\isacharparenleft}{\kern0pt}mset\ {\isacharbrackleft}{\kern0pt}{\isadigit{0}}{\isachardot}{\kern0pt}{\isachardot}{\kern0pt}{\isacharless}{\kern0pt}n{\isacharbrackright}{\kern0pt}{\isacharparenright}{\kern0pt}\ {\isacharequal}{\kern0pt}\ image{\isacharunderscore}{\kern0pt}mset\ f\ {\isacharparenleft}{\kern0pt}mset\ {\isacharbrackleft}{\kern0pt}{\isadigit{0}}{\isachardot}{\kern0pt}{\isachardot}{\kern0pt}{\isacharless}{\kern0pt}n{\isacharbrackright}{\kern0pt}{\isacharparenright}{\kern0pt}{\isachardoublequoteclose}\isanewline
\ \ \ \ \isacommand{by}\isamarkupfalse%
\ {\isacharparenleft}{\kern0pt}induction\ t\ arbitrary{\isacharcolon}{\kern0pt}f{\isacharcomma}{\kern0pt}\ simp{\isacharcomma}{\kern0pt}\ simp\ add{\isacharcolon}{\kern0pt}a{\isacharparenright}{\kern0pt}\ \isanewline
\ \ \isacommand{moreover}\isamarkupfalse%
\ \isacommand{have}\isamarkupfalse%
\ {\isachardoublequoteopen}{\isasymAnd}x{\isachardot}{\kern0pt}\ x\ {\isasymin}\ set\ t\ {\isasymLongrightarrow}\ is{\isacharunderscore}{\kern0pt}swap\ x{\isachardoublequoteclose}\ \isanewline
\ \ \ \ \isacommand{apply}\isamarkupfalse%
\ {\isacharparenleft}{\kern0pt}simp\ add{\isacharcolon}{\kern0pt}t{\isacharunderscore}{\kern0pt}def\ is{\isacharunderscore}{\kern0pt}swap{\isacharunderscore}{\kern0pt}def{\isacharparenright}{\kern0pt}\ \isanewline
\ \ \ \ \isacommand{by}\isamarkupfalse%
\ {\isacharparenleft}{\kern0pt}meson\ atLeastLessThan{\isacharunderscore}{\kern0pt}iff\ imageE\ less{\isacharunderscore}{\kern0pt}imp{\isacharunderscore}{\kern0pt}le\ less{\isacharunderscore}{\kern0pt}le{\isacharunderscore}{\kern0pt}trans{\isacharparenright}{\kern0pt}\ \ \isanewline
\ \ \isacommand{ultimately}\isamarkupfalse%
\ \isacommand{have}\isamarkupfalse%
\ {\isachardoublequoteopen}image{\isacharunderscore}{\kern0pt}mset\ {\isacharparenleft}{\kern0pt}fold\ id\ t\ f{\isacharparenright}{\kern0pt}\ {\isacharparenleft}{\kern0pt}mset\ {\isacharbrackleft}{\kern0pt}{\isadigit{0}}{\isachardot}{\kern0pt}{\isachardot}{\kern0pt}{\isacharless}{\kern0pt}n{\isacharbrackright}{\kern0pt}{\isacharparenright}{\kern0pt}\ {\isacharequal}{\kern0pt}\ image{\isacharunderscore}{\kern0pt}mset\ f\ {\isacharparenleft}{\kern0pt}mset\ {\isacharbrackleft}{\kern0pt}{\isadigit{0}}{\isachardot}{\kern0pt}{\isachardot}{\kern0pt}{\isacharless}{\kern0pt}n{\isacharbrackright}{\kern0pt}{\isacharparenright}{\kern0pt}{\isachardoublequoteclose}\ \isacommand{by}\isamarkupfalse%
\ blast\isanewline
\ \ \isacommand{then}\isamarkupfalse%
\ \isacommand{show}\isamarkupfalse%
\ {\isacharquery}{\kern0pt}thesis\ \isacommand{by}\isamarkupfalse%
\ {\isacharparenleft}{\kern0pt}simp\ add{\isacharcolon}{\kern0pt}t{\isacharunderscore}{\kern0pt}def{\isacharparenright}{\kern0pt}\isanewline
\isacommand{qed}\isamarkupfalse%
%
\endisatagproof
{\isafoldproof}%
%
\isadelimproof
\isanewline
%
\endisadelimproof
\isanewline
\isacommand{lemma}\isamarkupfalse%
\ sort{\isacharunderscore}{\kern0pt}map{\isacharunderscore}{\kern0pt}eq{\isacharunderscore}{\kern0pt}sort{\isacharcolon}{\kern0pt}\isanewline
\ \ \isakeyword{fixes}\ f\ {\isacharcolon}{\kern0pt}{\isacharcolon}{\kern0pt}\ {\isachardoublequoteopen}nat\ {\isasymRightarrow}\ {\isacharparenleft}{\kern0pt}{\isacharprime}{\kern0pt}b\ {\isacharcolon}{\kern0pt}{\isacharcolon}{\kern0pt}\ linorder{\isacharparenright}{\kern0pt}{\isachardoublequoteclose}\isanewline
\ \ \isakeyword{shows}\ {\isachardoublequoteopen}map\ {\isacharparenleft}{\kern0pt}sort{\isacharunderscore}{\kern0pt}map\ f\ n{\isacharparenright}{\kern0pt}\ {\isacharbrackleft}{\kern0pt}{\isadigit{0}}{\isachardot}{\kern0pt}{\isachardot}{\kern0pt}{\isacharless}{\kern0pt}n{\isacharbrackright}{\kern0pt}\ {\isacharequal}{\kern0pt}\ sort\ {\isacharparenleft}{\kern0pt}map\ f\ {\isacharbrackleft}{\kern0pt}{\isadigit{0}}{\isachardot}{\kern0pt}{\isachardot}{\kern0pt}{\isacharless}{\kern0pt}n{\isacharbrackright}{\kern0pt}{\isacharparenright}{\kern0pt}{\isachardoublequoteclose}\ {\isacharparenleft}{\kern0pt}\isakeyword{is}\ {\isachardoublequoteopen}{\isacharquery}{\kern0pt}A\ {\isacharequal}{\kern0pt}\ {\isacharquery}{\kern0pt}B{\isachardoublequoteclose}{\isacharparenright}{\kern0pt}\isanewline
%
\isadelimproof
%
\endisadelimproof
%
\isatagproof
\isacommand{proof}\isamarkupfalse%
\ {\isacharminus}{\kern0pt}\isanewline
\ \ \isacommand{have}\isamarkupfalse%
\ {\isachardoublequoteopen}mset\ {\isacharquery}{\kern0pt}A\ {\isacharequal}{\kern0pt}\ mset\ {\isacharquery}{\kern0pt}B{\isachardoublequoteclose}\isanewline
\ \ \ \ \isacommand{using}\isamarkupfalse%
\ sort{\isacharunderscore}{\kern0pt}map{\isacharunderscore}{\kern0pt}perm{\isacharbrackleft}{\kern0pt}\isakeyword{where}\ f{\isacharequal}{\kern0pt}{\isachardoublequoteopen}f{\isachardoublequoteclose}\ \isakeyword{and}\ n{\isacharequal}{\kern0pt}{\isachardoublequoteopen}n{\isachardoublequoteclose}{\isacharbrackright}{\kern0pt}\isanewline
\ \ \ \ \isacommand{by}\isamarkupfalse%
\ {\isacharparenleft}{\kern0pt}simp\ del{\isacharcolon}{\kern0pt}sort{\isacharunderscore}{\kern0pt}map{\isachardot}{\kern0pt}simps{\isacharparenright}{\kern0pt}\isanewline
\ \ \isacommand{moreover}\isamarkupfalse%
\ \isacommand{have}\isamarkupfalse%
\ {\isachardoublequoteopen}sorted\ {\isacharquery}{\kern0pt}B{\isachardoublequoteclose}\isanewline
\ \ \ \ \isacommand{by}\isamarkupfalse%
\ simp\isanewline
\ \ \isacommand{moreover}\isamarkupfalse%
\ \isacommand{have}\isamarkupfalse%
\ {\isachardoublequoteopen}sorted\ {\isacharquery}{\kern0pt}A{\isachardoublequoteclose}\isanewline
\ \ \ \ \isacommand{apply}\isamarkupfalse%
\ {\isacharparenleft}{\kern0pt}subst\ sorted{\isacharunderscore}{\kern0pt}wrt{\isacharunderscore}{\kern0pt}iff{\isacharunderscore}{\kern0pt}nth{\isacharunderscore}{\kern0pt}less{\isacharparenright}{\kern0pt}\isanewline
\ \ \ \ \isacommand{apply}\isamarkupfalse%
\ {\isacharparenleft}{\kern0pt}simp\ del{\isacharcolon}{\kern0pt}sort{\isacharunderscore}{\kern0pt}map{\isachardot}{\kern0pt}simps{\isacharparenright}{\kern0pt}\isanewline
\ \ \ \ \isacommand{using}\isamarkupfalse%
\ sort{\isacharunderscore}{\kern0pt}map{\isacharunderscore}{\kern0pt}mono\ \ \ \ \isanewline
\ \ \ \ \isacommand{by}\isamarkupfalse%
\ {\isacharparenleft}{\kern0pt}metis\ nat{\isacharunderscore}{\kern0pt}less{\isacharunderscore}{\kern0pt}le{\isacharparenright}{\kern0pt}\isanewline
\ \ \isacommand{ultimately}\isamarkupfalse%
\ \isacommand{show}\isamarkupfalse%
\ {\isachardoublequoteopen}{\isacharquery}{\kern0pt}A\ {\isacharequal}{\kern0pt}\ {\isacharquery}{\kern0pt}B{\isachardoublequoteclose}\ \isanewline
\ \ \ \ \isacommand{using}\isamarkupfalse%
\ list{\isacharunderscore}{\kern0pt}eq{\isacharunderscore}{\kern0pt}iff\ \isacommand{by}\isamarkupfalse%
\ blast\isanewline
\isacommand{qed}\isamarkupfalse%
%
\endisatagproof
{\isafoldproof}%
%
\isadelimproof
\isanewline
%
\endisadelimproof
\isanewline
\isacommand{definition}\isamarkupfalse%
\ median\ \isakeyword{where}\isanewline
\ \ {\isachardoublequoteopen}median\ n\ f\ {\isacharequal}{\kern0pt}\ \ sort\ {\isacharparenleft}{\kern0pt}map\ f\ {\isacharbrackleft}{\kern0pt}{\isadigit{0}}{\isachardot}{\kern0pt}{\isachardot}{\kern0pt}{\isacharless}{\kern0pt}n{\isacharbrackright}{\kern0pt}{\isacharparenright}{\kern0pt}\ {\isacharbang}{\kern0pt}\ {\isacharparenleft}{\kern0pt}n\ div\ {\isadigit{2}}{\isacharparenright}{\kern0pt}{\isachardoublequoteclose}\isanewline
\isanewline
\isacommand{lemma}\isamarkupfalse%
\ median{\isacharunderscore}{\kern0pt}alt{\isacharunderscore}{\kern0pt}def{\isacharcolon}{\kern0pt}\isanewline
\ \ \isakeyword{assumes}\ {\isachardoublequoteopen}n\ {\isachargreater}{\kern0pt}\ {\isadigit{0}}{\isachardoublequoteclose}\isanewline
\ \ \isakeyword{shows}\ {\isachardoublequoteopen}median\ n\ f\ {\isacharequal}{\kern0pt}\ {\isacharparenleft}{\kern0pt}sort{\isacharunderscore}{\kern0pt}map\ f\ n{\isacharparenright}{\kern0pt}\ {\isacharparenleft}{\kern0pt}n\ div\ {\isadigit{2}}{\isacharparenright}{\kern0pt}{\isachardoublequoteclose}\isanewline
%
\isadelimproof
\ \ %
\endisadelimproof
%
\isatagproof
\isacommand{using}\isamarkupfalse%
\ assms\isanewline
\ \ \isacommand{by}\isamarkupfalse%
\ {\isacharparenleft}{\kern0pt}simp\ add{\isacharcolon}{\kern0pt}median{\isacharunderscore}{\kern0pt}def\ sort{\isacharunderscore}{\kern0pt}map{\isacharunderscore}{\kern0pt}eq{\isacharunderscore}{\kern0pt}sort{\isacharbrackleft}{\kern0pt}symmetric{\isacharbrackright}{\kern0pt}\ del{\isacharcolon}{\kern0pt}sort{\isacharunderscore}{\kern0pt}map{\isachardot}{\kern0pt}simps{\isacharparenright}{\kern0pt}%
\endisatagproof
{\isafoldproof}%
%
\isadelimproof
\isanewline
%
\endisadelimproof
\isanewline
\isacommand{definition}\isamarkupfalse%
\ up{\isacharunderscore}{\kern0pt}ray\ {\isacharcolon}{\kern0pt}{\isacharcolon}{\kern0pt}\ {\isachardoublequoteopen}{\isacharparenleft}{\kern0pt}{\isacharprime}{\kern0pt}a\ {\isacharcolon}{\kern0pt}{\isacharcolon}{\kern0pt}\ linorder{\isacharparenright}{\kern0pt}\ set\ {\isasymRightarrow}\ bool{\isachardoublequoteclose}\ \isakeyword{where}\isanewline
\ \ {\isachardoublequoteopen}up{\isacharunderscore}{\kern0pt}ray\ I\ {\isacharequal}{\kern0pt}\ {\isacharparenleft}{\kern0pt}{\isasymforall}x\ y{\isachardot}{\kern0pt}\ x\ {\isasymin}\ I\ {\isasymlongrightarrow}\ x\ {\isasymle}\ y\ {\isasymlongrightarrow}\ y\ {\isasymin}\ I{\isacharparenright}{\kern0pt}{\isachardoublequoteclose}\isanewline
\isanewline
\isacommand{lemma}\isamarkupfalse%
\ up{\isacharunderscore}{\kern0pt}ray{\isacharunderscore}{\kern0pt}borel{\isacharcolon}{\kern0pt}\isanewline
\ \ \isakeyword{assumes}\ {\isachardoublequoteopen}up{\isacharunderscore}{\kern0pt}ray\ {\isacharparenleft}{\kern0pt}I\ {\isacharcolon}{\kern0pt}{\isacharcolon}{\kern0pt}\ {\isacharparenleft}{\kern0pt}{\isacharparenleft}{\kern0pt}{\isacharprime}{\kern0pt}a\ {\isacharcolon}{\kern0pt}{\isacharcolon}{\kern0pt}\ linorder{\isacharunderscore}{\kern0pt}topology{\isacharparenright}{\kern0pt}\ set{\isacharparenright}{\kern0pt}{\isacharparenright}{\kern0pt}{\isachardoublequoteclose}\isanewline
\ \ \isakeyword{shows}\ {\isachardoublequoteopen}I\ {\isasymin}\ borel{\isachardoublequoteclose}\isanewline
%
\isadelimproof
%
\endisadelimproof
%
\isatagproof
\isacommand{proof}\isamarkupfalse%
\ {\isacharparenleft}{\kern0pt}cases\ {\isachardoublequoteopen}closed\ I{\isachardoublequoteclose}{\isacharparenright}{\kern0pt}\isanewline
\ \ \isacommand{case}\isamarkupfalse%
\ True\isanewline
\ \ \isacommand{then}\isamarkupfalse%
\ \isacommand{show}\isamarkupfalse%
\ {\isacharquery}{\kern0pt}thesis\ \isacommand{using}\isamarkupfalse%
\ borel{\isacharunderscore}{\kern0pt}closed\ \isacommand{by}\isamarkupfalse%
\ blast\isanewline
\isacommand{next}\isamarkupfalse%
\isanewline
\ \ \isacommand{case}\isamarkupfalse%
\ False\isanewline
\ \ \isacommand{hence}\isamarkupfalse%
\ b{\isacharcolon}{\kern0pt}{\isachardoublequoteopen}{\isasymnot}\ closed\ I{\isachardoublequoteclose}\ \isacommand{by}\isamarkupfalse%
\ blast\isanewline
\isanewline
\ \ \isacommand{have}\isamarkupfalse%
\ {\isachardoublequoteopen}open\ I{\isachardoublequoteclose}\isanewline
\ \ \isacommand{proof}\isamarkupfalse%
\ {\isacharparenleft}{\kern0pt}rule\ Topological{\isacharunderscore}{\kern0pt}Spaces{\isachardot}{\kern0pt}openI{\isacharparenright}{\kern0pt}\isanewline
\ \ \ \ \isacommand{fix}\isamarkupfalse%
\ x\isanewline
\ \ \ \ \isacommand{assume}\isamarkupfalse%
\ c{\isacharcolon}{\kern0pt}{\isachardoublequoteopen}x\ {\isasymin}\ I{\isachardoublequoteclose}\isanewline
\ \ \ \ \isacommand{show}\isamarkupfalse%
\ {\isachardoublequoteopen}{\isasymexists}T{\isachardot}{\kern0pt}\ open\ T\ {\isasymand}\ x\ {\isasymin}\ T\ {\isasymand}\ T\ {\isasymsubseteq}\ I{\isachardoublequoteclose}\isanewline
\ \ \ \ \isacommand{proof}\isamarkupfalse%
\ {\isacharparenleft}{\kern0pt}cases\ {\isachardoublequoteopen}{\isasymexists}y{\isachardot}{\kern0pt}\ y\ {\isacharless}{\kern0pt}\ x\ {\isasymand}\ y\ {\isasymin}\ I{\isachardoublequoteclose}{\isacharparenright}{\kern0pt}\isanewline
\ \ \ \ \ \ \isacommand{case}\isamarkupfalse%
\ True\isanewline
\ \ \ \ \ \ \isacommand{then}\isamarkupfalse%
\ \isacommand{obtain}\isamarkupfalse%
\ y\ \isakeyword{where}\ a{\isacharcolon}{\kern0pt}{\isachardoublequoteopen}y\ {\isacharless}{\kern0pt}\ x\ {\isasymand}\ y\ {\isasymin}\ I{\isachardoublequoteclose}\ \isacommand{by}\isamarkupfalse%
\ blast\isanewline
\ \ \ \ \ \ \isacommand{have}\isamarkupfalse%
\ {\isachardoublequoteopen}open\ {\isacharbraceleft}{\kern0pt}y{\isacharless}{\kern0pt}{\isachardot}{\kern0pt}{\isachardot}{\kern0pt}{\isacharbraceright}{\kern0pt}{\isachardoublequoteclose}\ \isacommand{by}\isamarkupfalse%
\ simp\isanewline
\ \ \ \ \ \ \isacommand{moreover}\isamarkupfalse%
\ \isacommand{have}\isamarkupfalse%
\ {\isachardoublequoteopen}x\ {\isasymin}\ {\isacharbraceleft}{\kern0pt}y{\isacharless}{\kern0pt}{\isachardot}{\kern0pt}{\isachardot}{\kern0pt}{\isacharbraceright}{\kern0pt}{\isachardoublequoteclose}\ \isacommand{using}\isamarkupfalse%
\ a\ \isacommand{by}\isamarkupfalse%
\ simp\isanewline
\ \ \ \ \ \ \isacommand{moreover}\isamarkupfalse%
\ \isacommand{have}\isamarkupfalse%
\ {\isachardoublequoteopen}{\isacharbraceleft}{\kern0pt}y{\isacharless}{\kern0pt}{\isachardot}{\kern0pt}{\isachardot}{\kern0pt}{\isacharbraceright}{\kern0pt}\ {\isasymsubseteq}\ I{\isachardoublequoteclose}\isanewline
\ \ \ \ \ \ \ \ \isacommand{apply}\isamarkupfalse%
\ {\isacharparenleft}{\kern0pt}rule\ subsetI{\isacharparenright}{\kern0pt}\isanewline
\ \ \ \ \ \ \ \ \isacommand{using}\isamarkupfalse%
\ a\ assms{\isacharparenleft}{\kern0pt}{\isadigit{1}}{\isacharparenright}{\kern0pt}\ \isacommand{apply}\isamarkupfalse%
\ {\isacharparenleft}{\kern0pt}simp\ add{\isacharcolon}{\kern0pt}\ up{\isacharunderscore}{\kern0pt}ray{\isacharunderscore}{\kern0pt}def{\isacharparenright}{\kern0pt}\ \isanewline
\ \ \ \ \ \ \ \ \isacommand{by}\isamarkupfalse%
\ {\isacharparenleft}{\kern0pt}metis\ less{\isacharunderscore}{\kern0pt}le{\isacharunderscore}{\kern0pt}not{\isacharunderscore}{\kern0pt}le{\isacharparenright}{\kern0pt}\isanewline
\ \ \ \ \ \ \isacommand{ultimately}\isamarkupfalse%
\ \isacommand{show}\isamarkupfalse%
\ {\isacharquery}{\kern0pt}thesis\ \isacommand{by}\isamarkupfalse%
\ blast\isanewline
\ \ \ \ \isacommand{next}\isamarkupfalse%
\isanewline
\ \ \ \ \ \ \isacommand{case}\isamarkupfalse%
\ False\isanewline
\ \ \ \ \ \ \isacommand{hence}\isamarkupfalse%
\ {\isachardoublequoteopen}I\ {\isasymsubseteq}\ {\isacharbraceleft}{\kern0pt}x{\isachardot}{\kern0pt}{\isachardot}{\kern0pt}{\isacharbraceright}{\kern0pt}{\isachardoublequoteclose}\ \isacommand{using}\isamarkupfalse%
\ linorder{\isacharunderscore}{\kern0pt}not{\isacharunderscore}{\kern0pt}less\ \isacommand{by}\isamarkupfalse%
\ auto\isanewline
\ \ \ \ \ \ \isacommand{moreover}\isamarkupfalse%
\ \isacommand{have}\isamarkupfalse%
\ {\isachardoublequoteopen}{\isacharbraceleft}{\kern0pt}x{\isachardot}{\kern0pt}{\isachardot}{\kern0pt}{\isacharbraceright}{\kern0pt}\ {\isasymsubseteq}\ I{\isachardoublequoteclose}\isanewline
\ \ \ \ \ \ \ \ \isacommand{using}\isamarkupfalse%
\ c\ assms{\isacharparenleft}{\kern0pt}{\isadigit{1}}{\isacharparenright}{\kern0pt}\ \ \isacommand{apply}\isamarkupfalse%
\ {\isacharparenleft}{\kern0pt}simp\ add{\isacharcolon}{\kern0pt}\ up{\isacharunderscore}{\kern0pt}ray{\isacharunderscore}{\kern0pt}def{\isacharparenright}{\kern0pt}\ \isanewline
\ \ \ \ \ \ \ \ \isacommand{by}\isamarkupfalse%
\ blast\isanewline
\ \ \ \ \ \ \isacommand{ultimately}\isamarkupfalse%
\ \isacommand{have}\isamarkupfalse%
\ {\isachardoublequoteopen}I\ {\isacharequal}{\kern0pt}\ {\isacharbraceleft}{\kern0pt}x{\isachardot}{\kern0pt}{\isachardot}{\kern0pt}{\isacharbraceright}{\kern0pt}{\isachardoublequoteclose}\ \isanewline
\ \ \ \ \ \ \ \ \isacommand{by}\isamarkupfalse%
\ {\isacharparenleft}{\kern0pt}rule\ order{\isacharunderscore}{\kern0pt}antisym{\isacharparenright}{\kern0pt}\isanewline
\ \ \ \ \ \ \isacommand{moreover}\isamarkupfalse%
\ \isacommand{have}\isamarkupfalse%
\ {\isachardoublequoteopen}closed\ {\isacharbraceleft}{\kern0pt}x{\isachardot}{\kern0pt}{\isachardot}{\kern0pt}{\isacharbraceright}{\kern0pt}{\isachardoublequoteclose}\ \isacommand{by}\isamarkupfalse%
\ simp\isanewline
\ \ \ \ \ \ \isacommand{ultimately}\isamarkupfalse%
\ \isacommand{have}\isamarkupfalse%
\ {\isachardoublequoteopen}False{\isachardoublequoteclose}\ \isacommand{using}\isamarkupfalse%
\ b\ \isacommand{by}\isamarkupfalse%
\ auto\isanewline
\ \ \ \ \ \ \isacommand{then}\isamarkupfalse%
\ \isacommand{show}\isamarkupfalse%
\ {\isacharquery}{\kern0pt}thesis\ \isacommand{by}\isamarkupfalse%
\ simp\isanewline
\ \ \ \ \isacommand{qed}\isamarkupfalse%
\isanewline
\ \ \isacommand{qed}\isamarkupfalse%
\isanewline
\ \ \isacommand{then}\isamarkupfalse%
\ \isacommand{show}\isamarkupfalse%
\ {\isacharquery}{\kern0pt}thesis\ \isacommand{by}\isamarkupfalse%
\ simp\isanewline
\isacommand{qed}\isamarkupfalse%
%
\endisatagproof
{\isafoldproof}%
%
\isadelimproof
\isanewline
%
\endisadelimproof
\isanewline
\isacommand{definition}\isamarkupfalse%
\ down{\isacharunderscore}{\kern0pt}ray\ {\isacharcolon}{\kern0pt}{\isacharcolon}{\kern0pt}\ {\isachardoublequoteopen}{\isacharparenleft}{\kern0pt}{\isacharprime}{\kern0pt}a\ {\isacharcolon}{\kern0pt}{\isacharcolon}{\kern0pt}\ linorder{\isacharparenright}{\kern0pt}\ set\ {\isasymRightarrow}\ bool{\isachardoublequoteclose}\ \isakeyword{where}\isanewline
\ \ {\isachardoublequoteopen}down{\isacharunderscore}{\kern0pt}ray\ I\ {\isacharequal}{\kern0pt}\ {\isacharparenleft}{\kern0pt}{\isasymforall}x\ y{\isachardot}{\kern0pt}\ y\ {\isasymin}\ I\ {\isasymlongrightarrow}\ x\ {\isasymle}\ y\ {\isasymlongrightarrow}\ x\ {\isasymin}\ I{\isacharparenright}{\kern0pt}{\isachardoublequoteclose}\isanewline
\isanewline
\isacommand{lemma}\isamarkupfalse%
\ down{\isacharunderscore}{\kern0pt}ray{\isacharunderscore}{\kern0pt}borel{\isacharcolon}{\kern0pt}\isanewline
\ \ \isakeyword{assumes}\ {\isachardoublequoteopen}down{\isacharunderscore}{\kern0pt}ray\ {\isacharparenleft}{\kern0pt}I\ {\isacharcolon}{\kern0pt}{\isacharcolon}{\kern0pt}\ {\isacharparenleft}{\kern0pt}{\isacharparenleft}{\kern0pt}{\isacharprime}{\kern0pt}a\ {\isacharcolon}{\kern0pt}{\isacharcolon}{\kern0pt}\ linorder{\isacharunderscore}{\kern0pt}topology{\isacharparenright}{\kern0pt}\ set{\isacharparenright}{\kern0pt}{\isacharparenright}{\kern0pt}{\isachardoublequoteclose}\isanewline
\ \ \isakeyword{shows}\ {\isachardoublequoteopen}I\ {\isasymin}\ borel{\isachardoublequoteclose}\isanewline
%
\isadelimproof
%
\endisadelimproof
%
\isatagproof
\isacommand{proof}\isamarkupfalse%
\ {\isacharminus}{\kern0pt}\isanewline
\ \ \isacommand{have}\isamarkupfalse%
\ {\isachardoublequoteopen}up{\isacharunderscore}{\kern0pt}ray\ {\isacharparenleft}{\kern0pt}{\isacharminus}{\kern0pt}I{\isacharparenright}{\kern0pt}{\isachardoublequoteclose}\isanewline
\ \ \ \ \isacommand{using}\isamarkupfalse%
\ assms\ \isacommand{apply}\isamarkupfalse%
\ {\isacharparenleft}{\kern0pt}simp\ add{\isacharcolon}{\kern0pt}\ up{\isacharunderscore}{\kern0pt}ray{\isacharunderscore}{\kern0pt}def\ down{\isacharunderscore}{\kern0pt}ray{\isacharunderscore}{\kern0pt}def{\isacharparenright}{\kern0pt}\ \isacommand{by}\isamarkupfalse%
\ blast\isanewline
\ \ \isacommand{hence}\isamarkupfalse%
\ {\isachardoublequoteopen}{\isacharparenleft}{\kern0pt}{\isacharminus}{\kern0pt}I{\isacharparenright}{\kern0pt}\ {\isasymin}\ borel{\isachardoublequoteclose}\ \isacommand{using}\isamarkupfalse%
\ up{\isacharunderscore}{\kern0pt}ray{\isacharunderscore}{\kern0pt}borel\ \isacommand{by}\isamarkupfalse%
\ blast\isanewline
\ \ \isacommand{thus}\isamarkupfalse%
\ {\isachardoublequoteopen}I\ {\isasymin}\ borel{\isachardoublequoteclose}\ \isanewline
\ \ \ \ \isacommand{by}\isamarkupfalse%
\ {\isacharparenleft}{\kern0pt}metis\ borel{\isacharunderscore}{\kern0pt}comp\ double{\isacharunderscore}{\kern0pt}complement{\isacharparenright}{\kern0pt}\isanewline
\isacommand{qed}\isamarkupfalse%
%
\endisatagproof
{\isafoldproof}%
%
\isadelimproof
\isanewline
%
\endisadelimproof
\isanewline
\isacommand{definition}\isamarkupfalse%
\ interval\ {\isacharcolon}{\kern0pt}{\isacharcolon}{\kern0pt}\ {\isachardoublequoteopen}{\isacharparenleft}{\kern0pt}{\isacharprime}{\kern0pt}a\ {\isacharcolon}{\kern0pt}{\isacharcolon}{\kern0pt}\ linorder{\isacharparenright}{\kern0pt}\ set\ {\isasymRightarrow}\ bool{\isachardoublequoteclose}\ \isakeyword{where}\isanewline
\ \ {\isachardoublequoteopen}interval\ I\ {\isacharequal}{\kern0pt}\ {\isacharparenleft}{\kern0pt}{\isasymforall}x\ y\ z{\isachardot}{\kern0pt}\ x\ {\isasymin}\ I\ {\isasymlongrightarrow}\ z\ {\isasymin}\ I\ {\isasymlongrightarrow}\ x\ {\isasymle}\ y\ {\isasymlongrightarrow}\ y\ {\isasymle}\ z\ {\isasymlongrightarrow}\ y\ {\isasymin}\ I{\isacharparenright}{\kern0pt}{\isachardoublequoteclose}\isanewline
\isanewline
\isacommand{lemma}\isamarkupfalse%
\ interval{\isacharunderscore}{\kern0pt}borel{\isacharcolon}{\kern0pt}\isanewline
\ \ \isakeyword{assumes}\ {\isachardoublequoteopen}interval\ {\isacharparenleft}{\kern0pt}I\ {\isacharcolon}{\kern0pt}{\isacharcolon}{\kern0pt}\ {\isacharparenleft}{\kern0pt}{\isacharparenleft}{\kern0pt}{\isacharprime}{\kern0pt}a\ {\isacharcolon}{\kern0pt}{\isacharcolon}{\kern0pt}\ linorder{\isacharunderscore}{\kern0pt}topology{\isacharparenright}{\kern0pt}\ set{\isacharparenright}{\kern0pt}{\isacharparenright}{\kern0pt}{\isachardoublequoteclose}\isanewline
\ \ \isakeyword{shows}\ {\isachardoublequoteopen}I\ {\isasymin}\ borel{\isachardoublequoteclose}\isanewline
%
\isadelimproof
%
\endisadelimproof
%
\isatagproof
\isacommand{proof}\isamarkupfalse%
\ {\isacharparenleft}{\kern0pt}cases\ {\isachardoublequoteopen}I\ {\isacharequal}{\kern0pt}\ {\isacharbraceleft}{\kern0pt}{\isacharbraceright}{\kern0pt}{\isachardoublequoteclose}{\isacharparenright}{\kern0pt}\isanewline
\ \ \isacommand{case}\isamarkupfalse%
\ True\isanewline
\ \ \isacommand{then}\isamarkupfalse%
\ \isacommand{show}\isamarkupfalse%
\ {\isacharquery}{\kern0pt}thesis\ \isacommand{by}\isamarkupfalse%
\ simp\isanewline
\isacommand{next}\isamarkupfalse%
\isanewline
\ \ \isacommand{case}\isamarkupfalse%
\ False\isanewline
\ \ \isacommand{then}\isamarkupfalse%
\ \isacommand{obtain}\isamarkupfalse%
\ x\ \isakeyword{where}\ a{\isacharcolon}{\kern0pt}{\isachardoublequoteopen}x\ {\isasymin}\ I{\isachardoublequoteclose}\ \isacommand{by}\isamarkupfalse%
\ blast\isanewline
\ \ \isacommand{have}\isamarkupfalse%
\ {\isachardoublequoteopen}{\isasymAnd}y\ z{\isachardot}{\kern0pt}\ y\ {\isasymin}\ I\ {\isasymunion}\ {\isacharbraceleft}{\kern0pt}x{\isachardot}{\kern0pt}{\isachardot}{\kern0pt}{\isacharbraceright}{\kern0pt}\ {\isasymLongrightarrow}\ y\ {\isasymle}\ z\ {\isasymLongrightarrow}\ z\ {\isasymin}\ I\ {\isasymunion}\ {\isacharbraceleft}{\kern0pt}x{\isachardot}{\kern0pt}{\isachardot}{\kern0pt}{\isacharbraceright}{\kern0pt}{\isachardoublequoteclose}\ \isanewline
\ \ \ \ \isacommand{by}\isamarkupfalse%
\ {\isacharparenleft}{\kern0pt}metis\ assms\ a\ interval{\isacharunderscore}{\kern0pt}def\ \ IntE\ UnE\ Un{\isacharunderscore}{\kern0pt}Int{\isacharunderscore}{\kern0pt}eq{\isacharparenleft}{\kern0pt}{\isadigit{1}}{\isacharparenright}{\kern0pt}\ Un{\isacharunderscore}{\kern0pt}Int{\isacharunderscore}{\kern0pt}eq{\isacharparenleft}{\kern0pt}{\isadigit{2}}{\isacharparenright}{\kern0pt}\ atLeast{\isacharunderscore}{\kern0pt}iff\ nle{\isacharunderscore}{\kern0pt}le\ order{\isachardot}{\kern0pt}trans{\isacharparenright}{\kern0pt}\isanewline
\ \ \isacommand{hence}\isamarkupfalse%
\ {\isachardoublequoteopen}up{\isacharunderscore}{\kern0pt}ray\ {\isacharparenleft}{\kern0pt}I\ {\isasymunion}\ {\isacharbraceleft}{\kern0pt}x{\isachardot}{\kern0pt}{\isachardot}{\kern0pt}{\isacharbraceright}{\kern0pt}{\isacharparenright}{\kern0pt}{\isachardoublequoteclose}\isanewline
\ \ \ \ \isacommand{using}\isamarkupfalse%
\ up{\isacharunderscore}{\kern0pt}ray{\isacharunderscore}{\kern0pt}def\ \isacommand{by}\isamarkupfalse%
\ blast\isanewline
\ \ \isacommand{hence}\isamarkupfalse%
\ b{\isacharcolon}{\kern0pt}{\isachardoublequoteopen}I\ {\isasymunion}\ {\isacharbraceleft}{\kern0pt}x{\isachardot}{\kern0pt}{\isachardot}{\kern0pt}{\isacharbraceright}{\kern0pt}\ {\isasymin}\ borel{\isachardoublequoteclose}\ \isanewline
\ \ \ \ \isacommand{using}\isamarkupfalse%
\ up{\isacharunderscore}{\kern0pt}ray{\isacharunderscore}{\kern0pt}borel\ \isacommand{by}\isamarkupfalse%
\ blast\isanewline
\isanewline
\ \ \isacommand{have}\isamarkupfalse%
\ {\isachardoublequoteopen}{\isasymAnd}y\ z{\isachardot}{\kern0pt}\ y\ {\isasymin}\ I\ {\isasymunion}\ {\isacharbraceleft}{\kern0pt}{\isachardot}{\kern0pt}{\isachardot}{\kern0pt}x{\isacharbraceright}{\kern0pt}\ {\isasymLongrightarrow}\ z\ {\isasymle}\ y\ {\isasymLongrightarrow}\ z\ {\isasymin}\ I\ {\isasymunion}\ {\isacharbraceleft}{\kern0pt}{\isachardot}{\kern0pt}{\isachardot}{\kern0pt}x{\isacharbraceright}{\kern0pt}{\isachardoublequoteclose}\ \isanewline
\ \ \ \ \isacommand{by}\isamarkupfalse%
\ {\isacharparenleft}{\kern0pt}metis\ assms\ a\ interval{\isacharunderscore}{\kern0pt}def\ UnE\ UnI{\isadigit{1}}\ UnI{\isadigit{2}}\ atMost{\isacharunderscore}{\kern0pt}iff\ dual{\isacharunderscore}{\kern0pt}order{\isachardot}{\kern0pt}trans\ linorder{\isacharunderscore}{\kern0pt}le{\isacharunderscore}{\kern0pt}cases{\isacharparenright}{\kern0pt}\isanewline
\ \ \isacommand{hence}\isamarkupfalse%
\ {\isachardoublequoteopen}down{\isacharunderscore}{\kern0pt}ray\ {\isacharparenleft}{\kern0pt}I\ {\isasymunion}\ {\isacharbraceleft}{\kern0pt}{\isachardot}{\kern0pt}{\isachardot}{\kern0pt}x{\isacharbraceright}{\kern0pt}{\isacharparenright}{\kern0pt}{\isachardoublequoteclose}\isanewline
\ \ \ \ \isacommand{using}\isamarkupfalse%
\ down{\isacharunderscore}{\kern0pt}ray{\isacharunderscore}{\kern0pt}def\ \isacommand{by}\isamarkupfalse%
\ blast\isanewline
\ \ \isacommand{hence}\isamarkupfalse%
\ c{\isacharcolon}{\kern0pt}{\isachardoublequoteopen}I\ {\isasymunion}\ {\isacharbraceleft}{\kern0pt}{\isachardot}{\kern0pt}{\isachardot}{\kern0pt}x{\isacharbraceright}{\kern0pt}\ {\isasymin}\ borel{\isachardoublequoteclose}\isanewline
\ \ \ \ \isacommand{using}\isamarkupfalse%
\ down{\isacharunderscore}{\kern0pt}ray{\isacharunderscore}{\kern0pt}borel\ \isacommand{by}\isamarkupfalse%
\ blast\isanewline
\isanewline
\ \ \isacommand{have}\isamarkupfalse%
\ {\isachardoublequoteopen}I\ {\isacharequal}{\kern0pt}\ {\isacharparenleft}{\kern0pt}I\ {\isasymunion}\ {\isacharbraceleft}{\kern0pt}x{\isachardot}{\kern0pt}{\isachardot}{\kern0pt}{\isacharbraceright}{\kern0pt}{\isacharparenright}{\kern0pt}\ {\isasyminter}\ {\isacharparenleft}{\kern0pt}I\ {\isasymunion}\ {\isacharbraceleft}{\kern0pt}{\isachardot}{\kern0pt}{\isachardot}{\kern0pt}x{\isacharbraceright}{\kern0pt}{\isacharparenright}{\kern0pt}{\isachardoublequoteclose}\isanewline
\ \ \ \ \isacommand{using}\isamarkupfalse%
\ a\ \isacommand{by}\isamarkupfalse%
\ fastforce\ \ \ \ \isanewline
\isanewline
\ \ \isacommand{then}\isamarkupfalse%
\ \isacommand{show}\isamarkupfalse%
\ {\isacharquery}{\kern0pt}thesis\ \isacommand{using}\isamarkupfalse%
\ b\ c\isanewline
\ \ \ \ \isacommand{by}\isamarkupfalse%
\ {\isacharparenleft}{\kern0pt}metis\ sets{\isachardot}{\kern0pt}Int{\isacharparenright}{\kern0pt}\isanewline
\isacommand{qed}\isamarkupfalse%
%
\endisatagproof
{\isafoldproof}%
%
\isadelimproof
\isanewline
%
\endisadelimproof
\isanewline
\isacommand{lemma}\isamarkupfalse%
\ interval{\isacharunderscore}{\kern0pt}rule{\isacharcolon}{\kern0pt}\isanewline
\ \ \isakeyword{assumes}\ {\isachardoublequoteopen}interval\ I{\isachardoublequoteclose}\isanewline
\ \ \isakeyword{assumes}\ {\isachardoublequoteopen}a\ {\isasymle}\ x{\isachardoublequoteclose}\ {\isachardoublequoteopen}x\ {\isasymle}\ b{\isachardoublequoteclose}\isanewline
\ \ \isakeyword{assumes}\ {\isachardoublequoteopen}a\ {\isasymin}\ I{\isachardoublequoteclose}\isanewline
\ \ \isakeyword{assumes}\ {\isachardoublequoteopen}b\ {\isasymin}\ I{\isachardoublequoteclose}\isanewline
\ \ \isakeyword{shows}\ {\isachardoublequoteopen}x\ {\isasymin}\ I{\isachardoublequoteclose}\isanewline
%
\isadelimproof
\ \ %
\endisadelimproof
%
\isatagproof
\isacommand{using}\isamarkupfalse%
\ assms{\isacharparenleft}{\kern0pt}{\isadigit{1}}{\isacharparenright}{\kern0pt}\ \isacommand{apply}\isamarkupfalse%
\ {\isacharparenleft}{\kern0pt}simp\ add{\isacharcolon}{\kern0pt}interval{\isacharunderscore}{\kern0pt}def{\isacharparenright}{\kern0pt}\isanewline
\ \ \isacommand{using}\isamarkupfalse%
\ assms\ \isacommand{by}\isamarkupfalse%
\ blast%
\endisatagproof
{\isafoldproof}%
%
\isadelimproof
\isanewline
%
\endisadelimproof
\isanewline
\isacommand{lemma}\isamarkupfalse%
\ sorted{\isacharunderscore}{\kern0pt}int{\isacharcolon}{\kern0pt}\isanewline
\ \ \isakeyword{assumes}\ {\isachardoublequoteopen}interval\ I{\isachardoublequoteclose}\isanewline
\ \ \isakeyword{assumes}\ {\isachardoublequoteopen}sorted\ xs{\isachardoublequoteclose}\isanewline
\ \ \isakeyword{assumes}\ {\isachardoublequoteopen}k\ {\isacharless}{\kern0pt}\ length\ xs{\isachardoublequoteclose}\ {\isachardoublequoteopen}i\ {\isasymle}\ j{\isachardoublequoteclose}\ {\isachardoublequoteopen}j\ {\isasymle}\ k\ {\isachardoublequoteclose}\isanewline
\ \ \isakeyword{assumes}\ {\isachardoublequoteopen}xs\ {\isacharbang}{\kern0pt}\ i\ {\isasymin}\ I{\isachardoublequoteclose}\ {\isachardoublequoteopen}xs\ {\isacharbang}{\kern0pt}\ k\ {\isasymin}\ I{\isachardoublequoteclose}\isanewline
\ \ \isakeyword{shows}\ {\isachardoublequoteopen}xs\ {\isacharbang}{\kern0pt}\ j\ {\isasymin}\ I{\isachardoublequoteclose}\isanewline
%
\isadelimproof
\ \ %
\endisadelimproof
%
\isatagproof
\isacommand{apply}\isamarkupfalse%
\ {\isacharparenleft}{\kern0pt}rule\ interval{\isacharunderscore}{\kern0pt}rule{\isacharbrackleft}{\kern0pt}\isakeyword{where}\ a{\isacharequal}{\kern0pt}{\isachardoublequoteopen}xs\ {\isacharbang}{\kern0pt}\ i{\isachardoublequoteclose}\ \isakeyword{and}\ b{\isacharequal}{\kern0pt}{\isachardoublequoteopen}xs\ {\isacharbang}{\kern0pt}\ k{\isachardoublequoteclose}{\isacharbrackright}{\kern0pt}{\isacharparenright}{\kern0pt}\isanewline
\ \ \isacommand{using}\isamarkupfalse%
\ assms\ \isacommand{by}\isamarkupfalse%
\ {\isacharparenleft}{\kern0pt}simp\ add{\isacharcolon}{\kern0pt}\ sorted{\isacharunderscore}{\kern0pt}nth{\isacharunderscore}{\kern0pt}mono{\isacharparenright}{\kern0pt}{\isacharplus}{\kern0pt}%
\endisatagproof
{\isafoldproof}%
%
\isadelimproof
\isanewline
%
\endisadelimproof
\isanewline
\isacommand{lemma}\isamarkupfalse%
\ mid{\isacharunderscore}{\kern0pt}in{\isacharunderscore}{\kern0pt}interval{\isacharcolon}{\kern0pt}\isanewline
\ \ \isakeyword{assumes}\ {\isachardoublequoteopen}{\isadigit{2}}{\isacharasterisk}{\kern0pt}length\ {\isacharparenleft}{\kern0pt}filter\ {\isacharparenleft}{\kern0pt}{\isasymlambda}x{\isachardot}{\kern0pt}\ x\ {\isasymin}\ I{\isacharparenright}{\kern0pt}\ xs{\isacharparenright}{\kern0pt}\ {\isachargreater}{\kern0pt}\ length\ xs{\isachardoublequoteclose}\isanewline
\ \ \isakeyword{assumes}\ {\isachardoublequoteopen}interval\ I{\isachardoublequoteclose}\isanewline
\ \ \isakeyword{assumes}\ {\isachardoublequoteopen}sorted\ xs{\isachardoublequoteclose}\isanewline
\ \ \isakeyword{shows}\ {\isachardoublequoteopen}xs\ {\isacharbang}{\kern0pt}\ {\isacharparenleft}{\kern0pt}length\ xs\ div\ {\isadigit{2}}{\isacharparenright}{\kern0pt}\ {\isasymin}\ I{\isachardoublequoteclose}\isanewline
%
\isadelimproof
%
\endisadelimproof
%
\isatagproof
\isacommand{proof}\isamarkupfalse%
\ {\isacharminus}{\kern0pt}\isanewline
\ \ \isacommand{have}\isamarkupfalse%
\ {\isachardoublequoteopen}length\ {\isacharparenleft}{\kern0pt}filter\ {\isacharparenleft}{\kern0pt}{\isasymlambda}x{\isachardot}{\kern0pt}\ x\ {\isasymin}\ I{\isacharparenright}{\kern0pt}\ xs{\isacharparenright}{\kern0pt}\ {\isachargreater}{\kern0pt}\ {\isadigit{0}}{\isachardoublequoteclose}\ \ \isacommand{using}\isamarkupfalse%
\ assms{\isacharparenleft}{\kern0pt}{\isadigit{1}}{\isacharparenright}{\kern0pt}\ \isacommand{by}\isamarkupfalse%
\ linarith\isanewline
\ \ \isacommand{then}\isamarkupfalse%
\ \isacommand{obtain}\isamarkupfalse%
\ v\ \isakeyword{where}\ v{\isacharunderscore}{\kern0pt}{\isadigit{1}}{\isacharcolon}{\kern0pt}\ {\isachardoublequoteopen}v\ {\isacharless}{\kern0pt}\ length\ xs{\isachardoublequoteclose}\ \isakeyword{and}\ v{\isacharunderscore}{\kern0pt}{\isadigit{2}}{\isacharcolon}{\kern0pt}\ {\isachardoublequoteopen}xs\ {\isacharbang}{\kern0pt}\ v\ {\isasymin}\ I{\isachardoublequoteclose}\ \isanewline
\ \ \ \ \isacommand{by}\isamarkupfalse%
\ {\isacharparenleft}{\kern0pt}metis\ filter{\isacharunderscore}{\kern0pt}False\ in{\isacharunderscore}{\kern0pt}set{\isacharunderscore}{\kern0pt}conv{\isacharunderscore}{\kern0pt}nth\ length{\isacharunderscore}{\kern0pt}greater{\isacharunderscore}{\kern0pt}{\isadigit{0}}{\isacharunderscore}{\kern0pt}conv{\isacharparenright}{\kern0pt}\isanewline
\isanewline
\ \ \isacommand{define}\isamarkupfalse%
\ J\ \isakeyword{where}\ {\isachardoublequoteopen}J\ {\isacharequal}{\kern0pt}\ {\isacharbraceleft}{\kern0pt}k{\isachardot}{\kern0pt}\ k\ {\isacharless}{\kern0pt}\ length\ xs\ {\isasymand}\ xs\ {\isacharbang}{\kern0pt}\ k\ {\isasymin}\ I{\isacharbraceright}{\kern0pt}{\isachardoublequoteclose}\isanewline
\isanewline
\ \ \isacommand{have}\isamarkupfalse%
\ card{\isacharunderscore}{\kern0pt}J{\isacharunderscore}{\kern0pt}min{\isacharcolon}{\kern0pt}\ {\isachardoublequoteopen}{\isadigit{2}}{\isacharasterisk}{\kern0pt}card\ J\ {\isachargreater}{\kern0pt}\ length\ xs{\isachardoublequoteclose}\isanewline
\ \ \ \ \isacommand{using}\isamarkupfalse%
\ assms{\isacharparenleft}{\kern0pt}{\isadigit{1}}{\isacharparenright}{\kern0pt}\ \isacommand{by}\isamarkupfalse%
\ {\isacharparenleft}{\kern0pt}simp\ add{\isacharcolon}{\kern0pt}J{\isacharunderscore}{\kern0pt}def\ length{\isacharunderscore}{\kern0pt}filter{\isacharunderscore}{\kern0pt}conv{\isacharunderscore}{\kern0pt}card{\isacharparenright}{\kern0pt}\isanewline
\isanewline
\ \ \isacommand{consider}\isamarkupfalse%
\isanewline
\ \ \ \ {\isacharparenleft}{\kern0pt}a{\isacharparenright}{\kern0pt}\ {\isachardoublequoteopen}xs\ {\isacharbang}{\kern0pt}\ {\isacharparenleft}{\kern0pt}length\ xs\ div\ {\isadigit{2}}{\isacharparenright}{\kern0pt}\ {\isasymin}\ I{\isachardoublequoteclose}\ {\isacharbar}{\kern0pt}\isanewline
\ \ \ \ {\isacharparenleft}{\kern0pt}b{\isacharparenright}{\kern0pt}\ {\isachardoublequoteopen}xs\ {\isacharbang}{\kern0pt}\ {\isacharparenleft}{\kern0pt}length\ xs\ div\ {\isadigit{2}}{\isacharparenright}{\kern0pt}\ {\isasymnotin}\ I\ {\isasymand}\ v\ {\isachargreater}{\kern0pt}\ {\isacharparenleft}{\kern0pt}length\ xs\ div\ {\isadigit{2}}{\isacharparenright}{\kern0pt}{\isachardoublequoteclose}\ {\isacharbar}{\kern0pt}\isanewline
\ \ \ \ {\isacharparenleft}{\kern0pt}c{\isacharparenright}{\kern0pt}\ {\isachardoublequoteopen}xs\ {\isacharbang}{\kern0pt}\ {\isacharparenleft}{\kern0pt}length\ xs\ div\ {\isadigit{2}}{\isacharparenright}{\kern0pt}\ {\isasymnotin}\ I\ {\isasymand}\ v\ {\isacharless}{\kern0pt}\ {\isacharparenleft}{\kern0pt}length\ xs\ div\ {\isadigit{2}}{\isacharparenright}{\kern0pt}{\isachardoublequoteclose}\isanewline
\ \ \ \ \isacommand{by}\isamarkupfalse%
\ {\isacharparenleft}{\kern0pt}metis\ linorder{\isacharunderscore}{\kern0pt}cases\ v{\isacharunderscore}{\kern0pt}{\isadigit{2}}{\isacharparenright}{\kern0pt}\isanewline
\ \ \isacommand{thus}\isamarkupfalse%
\ {\isacharquery}{\kern0pt}thesis\isanewline
\ \ \isacommand{proof}\isamarkupfalse%
\ {\isacharparenleft}{\kern0pt}cases{\isacharparenright}{\kern0pt}\isanewline
\ \ \ \ \isacommand{case}\isamarkupfalse%
\ a\isanewline
\ \ \ \ \isacommand{then}\isamarkupfalse%
\ \isacommand{show}\isamarkupfalse%
\ {\isacharquery}{\kern0pt}thesis\ \isacommand{by}\isamarkupfalse%
\ simp\isanewline
\ \ \isacommand{next}\isamarkupfalse%
\isanewline
\ \ \ \ \isacommand{case}\isamarkupfalse%
\ b\isanewline
\ \ \ \ \isacommand{have}\isamarkupfalse%
\ p{\isacharcolon}{\kern0pt}{\isachardoublequoteopen}{\isasymAnd}k{\isachardot}{\kern0pt}\ k\ {\isasymle}\ length\ xs\ div\ {\isadigit{2}}\ {\isasymLongrightarrow}\ xs\ {\isacharbang}{\kern0pt}\ k\ {\isasymnotin}\ I{\isachardoublequoteclose}\isanewline
\ \ \ \ \ \ \isacommand{using}\isamarkupfalse%
\ b\ v{\isacharunderscore}{\kern0pt}{\isadigit{2}}\ sorted{\isacharunderscore}{\kern0pt}int{\isacharbrackleft}{\kern0pt}OF\ assms{\isacharparenleft}{\kern0pt}{\isadigit{2}}{\isacharparenright}{\kern0pt}\ assms{\isacharparenleft}{\kern0pt}{\isadigit{3}}{\isacharparenright}{\kern0pt}\ v{\isacharunderscore}{\kern0pt}{\isadigit{1}}{\isacharcomma}{\kern0pt}\ \isakeyword{where}\ j{\isacharequal}{\kern0pt}{\isachardoublequoteopen}length\ xs\ div\ {\isadigit{2}}{\isachardoublequoteclose}{\isacharbrackright}{\kern0pt}\ \isacommand{apply}\isamarkupfalse%
\ simp\ \isacommand{by}\isamarkupfalse%
\ blast\isanewline
\ \ \ \ \isacommand{have}\isamarkupfalse%
\ {\isachardoublequoteopen}card\ J\ {\isasymle}\ card\ {\isacharbraceleft}{\kern0pt}Suc\ {\isacharparenleft}{\kern0pt}length\ xs\ div\ {\isadigit{2}}{\isacharparenright}{\kern0pt}{\isachardot}{\kern0pt}{\isachardot}{\kern0pt}{\isacharless}{\kern0pt}length\ xs{\isacharbraceright}{\kern0pt}{\isachardoublequoteclose}\isanewline
\ \ \ \ \ \ \isacommand{apply}\isamarkupfalse%
\ {\isacharparenleft}{\kern0pt}rule\ card{\isacharunderscore}{\kern0pt}mono{\isacharcomma}{\kern0pt}\ simp{\isacharparenright}{\kern0pt}\isanewline
\ \ \ \ \ \ \isacommand{apply}\isamarkupfalse%
\ {\isacharparenleft}{\kern0pt}rule\ subsetI{\isacharcomma}{\kern0pt}\ simp\ add{\isacharcolon}{\kern0pt}J{\isacharunderscore}{\kern0pt}def\ not{\isacharunderscore}{\kern0pt}less{\isacharunderscore}{\kern0pt}eq{\isacharunderscore}{\kern0pt}eq{\isacharbrackleft}{\kern0pt}symmetric{\isacharbrackright}{\kern0pt}{\isacharparenright}{\kern0pt}\isanewline
\ \ \ \ \ \ \isacommand{using}\isamarkupfalse%
\ p\ \isacommand{by}\isamarkupfalse%
\ metis\isanewline
\ \ \ \ \isacommand{hence}\isamarkupfalse%
\ {\isachardoublequoteopen}card\ J\ {\isasymle}\ length\ xs\ {\isacharminus}{\kern0pt}\ {\isacharparenleft}{\kern0pt}Suc\ {\isacharparenleft}{\kern0pt}length\ xs\ div\ {\isadigit{2}}{\isacharparenright}{\kern0pt}{\isacharparenright}{\kern0pt}{\isachardoublequoteclose}\isanewline
\ \ \ \ \ \ \isacommand{using}\isamarkupfalse%
\ card{\isacharunderscore}{\kern0pt}atLeastLessThan\ \isacommand{by}\isamarkupfalse%
\ metis\isanewline
\ \ \ \ \isacommand{hence}\isamarkupfalse%
\ {\isachardoublequoteopen}length\ xs\ {\isasymle}\ {\isadigit{2}}{\isacharasterisk}{\kern0pt}{\isacharparenleft}{\kern0pt}\ length\ xs\ {\isacharminus}{\kern0pt}\ {\isacharparenleft}{\kern0pt}Suc\ {\isacharparenleft}{\kern0pt}length\ xs\ div\ {\isadigit{2}}{\isacharparenright}{\kern0pt}{\isacharparenright}{\kern0pt}{\isacharparenright}{\kern0pt}{\isachardoublequoteclose}\isanewline
\ \ \ \ \ \ \isacommand{using}\isamarkupfalse%
\ card{\isacharunderscore}{\kern0pt}J{\isacharunderscore}{\kern0pt}min\ \isacommand{by}\isamarkupfalse%
\ linarith\isanewline
\ \ \ \ \isacommand{hence}\isamarkupfalse%
\ {\isachardoublequoteopen}False{\isachardoublequoteclose}\isanewline
\ \ \ \ \ \ \isacommand{apply}\isamarkupfalse%
\ {\isacharparenleft}{\kern0pt}simp\ add{\isacharcolon}{\kern0pt}nat{\isacharunderscore}{\kern0pt}distrib{\isacharparenright}{\kern0pt}\isanewline
\ \ \ \ \ \ \isacommand{apply}\isamarkupfalse%
\ {\isacharparenleft}{\kern0pt}subst\ {\isacharparenleft}{\kern0pt}asm{\isacharparenright}{\kern0pt}\ le{\isacharunderscore}{\kern0pt}diff{\isacharunderscore}{\kern0pt}conv{\isadigit{2}}{\isacharparenright}{\kern0pt}\isanewline
\ \ \ \ \ \ \isacommand{using}\isamarkupfalse%
\ b\ v{\isacharunderscore}{\kern0pt}{\isadigit{1}}\ \isacommand{apply}\isamarkupfalse%
\ linarith\isanewline
\ \ \ \ \ \ \isacommand{by}\isamarkupfalse%
\ simp\isanewline
\ \ \ \ \isacommand{then}\isamarkupfalse%
\ \isacommand{show}\isamarkupfalse%
\ {\isacharquery}{\kern0pt}thesis\ \isacommand{by}\isamarkupfalse%
\ simp\isanewline
\ \ \isacommand{next}\isamarkupfalse%
\isanewline
\ \ \ \ \isacommand{case}\isamarkupfalse%
\ c\isanewline
\ \ \ \ \isacommand{have}\isamarkupfalse%
\ p{\isacharcolon}{\kern0pt}{\isachardoublequoteopen}{\isasymAnd}k{\isachardot}{\kern0pt}\ k\ {\isasymge}\ length\ xs\ div\ {\isadigit{2}}\ {\isasymLongrightarrow}\ k\ {\isacharless}{\kern0pt}\ length\ xs\ {\isasymLongrightarrow}\ xs\ {\isacharbang}{\kern0pt}\ k\ {\isasymnotin}\ I{\isachardoublequoteclose}\isanewline
\ \ \ \ \ \ \isacommand{using}\isamarkupfalse%
\ c\ v{\isacharunderscore}{\kern0pt}{\isadigit{1}}\ v{\isacharunderscore}{\kern0pt}{\isadigit{2}}\ sorted{\isacharunderscore}{\kern0pt}int{\isacharbrackleft}{\kern0pt}OF\ assms{\isacharparenleft}{\kern0pt}{\isadigit{2}}{\isacharparenright}{\kern0pt}\ assms{\isacharparenleft}{\kern0pt}{\isadigit{3}}{\isacharparenright}{\kern0pt}{\isacharcomma}{\kern0pt}\ \isakeyword{where}\ i\ {\isacharequal}{\kern0pt}{\isachardoublequoteopen}v{\isachardoublequoteclose}\ \isakeyword{and}\ j{\isacharequal}{\kern0pt}{\isachardoublequoteopen}length\ xs\ div\ {\isadigit{2}}{\isachardoublequoteclose}{\isacharbrackright}{\kern0pt}\ \isacommand{apply}\isamarkupfalse%
\ simp\ \isacommand{by}\isamarkupfalse%
\ blast\isanewline
\ \ \ \ \isacommand{have}\isamarkupfalse%
\ {\isachardoublequoteopen}card\ J\ {\isasymle}\ card\ {\isacharbraceleft}{\kern0pt}{\isadigit{0}}{\isachardot}{\kern0pt}{\isachardot}{\kern0pt}{\isacharless}{\kern0pt}{\isacharparenleft}{\kern0pt}length\ xs\ div\ {\isadigit{2}}{\isacharparenright}{\kern0pt}{\isacharbraceright}{\kern0pt}{\isachardoublequoteclose}\isanewline
\ \ \ \ \ \ \isacommand{apply}\isamarkupfalse%
\ {\isacharparenleft}{\kern0pt}rule\ card{\isacharunderscore}{\kern0pt}mono{\isacharcomma}{\kern0pt}\ simp{\isacharparenright}{\kern0pt}\isanewline
\ \ \ \ \ \ \isacommand{apply}\isamarkupfalse%
\ {\isacharparenleft}{\kern0pt}rule\ subsetI{\isacharcomma}{\kern0pt}\ simp\ add{\isacharcolon}{\kern0pt}J{\isacharunderscore}{\kern0pt}def\ not{\isacharunderscore}{\kern0pt}less{\isacharunderscore}{\kern0pt}eq{\isacharunderscore}{\kern0pt}eq{\isacharbrackleft}{\kern0pt}symmetric{\isacharbrackright}{\kern0pt}{\isacharparenright}{\kern0pt}\isanewline
\ \ \ \ \ \ \isacommand{using}\isamarkupfalse%
\ p\ linorder{\isacharunderscore}{\kern0pt}le{\isacharunderscore}{\kern0pt}less{\isacharunderscore}{\kern0pt}linear\ \isacommand{by}\isamarkupfalse%
\ blast\isanewline
\ \ \ \ \isacommand{hence}\isamarkupfalse%
\ {\isachardoublequoteopen}card\ J\ {\isasymle}\ {\isacharparenleft}{\kern0pt}length\ xs\ div\ {\isadigit{2}}{\isacharparenright}{\kern0pt}{\isachardoublequoteclose}\isanewline
\ \ \ \ \ \ \isacommand{using}\isamarkupfalse%
\ card{\isacharunderscore}{\kern0pt}atLeastLessThan\ \isacommand{by}\isamarkupfalse%
\ simp\isanewline
\ \ \ \ \isacommand{then}\isamarkupfalse%
\ \isacommand{show}\isamarkupfalse%
\ {\isacharquery}{\kern0pt}thesis\ \isacommand{using}\isamarkupfalse%
\ card{\isacharunderscore}{\kern0pt}J{\isacharunderscore}{\kern0pt}min\ \isacommand{by}\isamarkupfalse%
\ linarith\isanewline
\ \ \isacommand{qed}\isamarkupfalse%
\isanewline
\isacommand{qed}\isamarkupfalse%
%
\endisatagproof
{\isafoldproof}%
%
\isadelimproof
\isanewline
%
\endisadelimproof
\isanewline
\isacommand{lemma}\isamarkupfalse%
\ median{\isacharunderscore}{\kern0pt}est{\isacharcolon}{\kern0pt}\isanewline
\ \ \isakeyword{assumes}\ {\isachardoublequoteopen}interval\ I{\isachardoublequoteclose}\isanewline
\ \ \isakeyword{assumes}\ {\isachardoublequoteopen}{\isadigit{2}}{\isacharasterisk}{\kern0pt}card\ {\isacharbraceleft}{\kern0pt}k{\isachardot}{\kern0pt}\ k\ {\isacharless}{\kern0pt}\ n\ {\isasymand}\ f\ k\ {\isasymin}\ I{\isacharbraceright}{\kern0pt}\ {\isachargreater}{\kern0pt}\ n{\isachardoublequoteclose}\isanewline
\ \ \isakeyword{shows}\ {\isachardoublequoteopen}median\ n\ f\ {\isasymin}\ I{\isachardoublequoteclose}\isanewline
%
\isadelimproof
%
\endisadelimproof
%
\isatagproof
\isacommand{proof}\isamarkupfalse%
\ {\isacharminus}{\kern0pt}\isanewline
\ \ \isacommand{have}\isamarkupfalse%
\ a{\isacharcolon}{\kern0pt}{\isachardoublequoteopen}{\isacharbraceleft}{\kern0pt}k{\isachardot}{\kern0pt}\ k\ {\isacharless}{\kern0pt}\ n\ {\isasymand}\ f\ k\ {\isasymin}\ I{\isacharbraceright}{\kern0pt}\ {\isacharequal}{\kern0pt}\ {\isacharbraceleft}{\kern0pt}i{\isachardot}{\kern0pt}\ i\ {\isacharless}{\kern0pt}\ n\ {\isasymand}\ map\ f\ {\isacharbrackleft}{\kern0pt}{\isadigit{0}}{\isachardot}{\kern0pt}{\isachardot}{\kern0pt}{\isacharless}{\kern0pt}n{\isacharbrackright}{\kern0pt}\ {\isacharbang}{\kern0pt}\ i\ {\isasymin}\ I{\isacharbraceright}{\kern0pt}{\isachardoublequoteclose}\isanewline
\ \ \ \ \isacommand{apply}\isamarkupfalse%
\ {\isacharparenleft}{\kern0pt}rule\ order{\isacharunderscore}{\kern0pt}antisym{\isacharparenright}{\kern0pt}\isanewline
\ \ \ \ \ \isacommand{apply}\isamarkupfalse%
\ {\isacharparenleft}{\kern0pt}rule\ subsetI{\isacharcomma}{\kern0pt}\ simp{\isacharparenright}{\kern0pt}\isanewline
\ \ \ \ \isacommand{apply}\isamarkupfalse%
\ {\isacharparenleft}{\kern0pt}rule\ subsetI{\isacharcomma}{\kern0pt}\ simp{\isacharparenright}{\kern0pt}\ \isanewline
\ \ \ \ \isacommand{by}\isamarkupfalse%
\ {\isacharparenleft}{\kern0pt}metis\ add{\isacharunderscore}{\kern0pt}{\isadigit{0}}\ diff{\isacharunderscore}{\kern0pt}zero\ nth{\isacharunderscore}{\kern0pt}map{\isacharunderscore}{\kern0pt}upt{\isacharparenright}{\kern0pt}\isanewline
\isanewline
\ \ \isacommand{show}\isamarkupfalse%
\ {\isacharquery}{\kern0pt}thesis\isanewline
\ \ \ \ \isacommand{apply}\isamarkupfalse%
\ {\isacharparenleft}{\kern0pt}simp\ add{\isacharcolon}{\kern0pt}median{\isacharunderscore}{\kern0pt}def{\isacharparenright}{\kern0pt}\isanewline
\ \ \ \ \isacommand{apply}\isamarkupfalse%
\ {\isacharparenleft}{\kern0pt}rule\ mid{\isacharunderscore}{\kern0pt}in{\isacharunderscore}{\kern0pt}interval{\isacharbrackleft}{\kern0pt}\isakeyword{where}\ I{\isacharequal}{\kern0pt}{\isachardoublequoteopen}I{\isachardoublequoteclose}\ \isakeyword{and}\ xs{\isacharequal}{\kern0pt}{\isachardoublequoteopen}sort\ {\isacharparenleft}{\kern0pt}map\ f\ {\isacharbrackleft}{\kern0pt}{\isadigit{0}}{\isachardot}{\kern0pt}{\isachardot}{\kern0pt}{\isacharless}{\kern0pt}n{\isacharbrackright}{\kern0pt}{\isacharparenright}{\kern0pt}{\isachardoublequoteclose}{\isacharcomma}{\kern0pt}\ simplified{\isacharbrackright}{\kern0pt}{\isacharparenright}{\kern0pt}\isanewline
\ \ \ \ \ \isacommand{using}\isamarkupfalse%
\ assms\ a\ \isacommand{apply}\isamarkupfalse%
\ {\isacharparenleft}{\kern0pt}simp\ add{\isacharcolon}{\kern0pt}filter{\isacharunderscore}{\kern0pt}sort\ comp{\isacharunderscore}{\kern0pt}def\ length{\isacharunderscore}{\kern0pt}filter{\isacharunderscore}{\kern0pt}conv{\isacharunderscore}{\kern0pt}card{\isacharparenright}{\kern0pt}\isanewline
\ \ \ \ \isacommand{by}\isamarkupfalse%
\ {\isacharparenleft}{\kern0pt}simp\ add{\isacharcolon}{\kern0pt}assms{\isacharparenright}{\kern0pt}\isanewline
\isacommand{qed}\isamarkupfalse%
%
\endisatagproof
{\isafoldproof}%
%
\isadelimproof
\isanewline
%
\endisadelimproof
\isanewline
\isacommand{lemma}\isamarkupfalse%
\ median{\isacharunderscore}{\kern0pt}measurable{\isacharcolon}{\kern0pt}\isanewline
\ \ \isakeyword{fixes}\ X\ {\isacharcolon}{\kern0pt}{\isacharcolon}{\kern0pt}\ {\isachardoublequoteopen}nat\ {\isasymRightarrow}\ {\isacharprime}{\kern0pt}a\ {\isasymRightarrow}\ {\isacharparenleft}{\kern0pt}{\isacharprime}{\kern0pt}b\ {\isacharcolon}{\kern0pt}{\isacharcolon}{\kern0pt}\ {\isacharbraceleft}{\kern0pt}linorder{\isacharcomma}{\kern0pt}\ topological{\isacharunderscore}{\kern0pt}space{\isacharcomma}{\kern0pt}\ linorder{\isacharunderscore}{\kern0pt}topology{\isacharcomma}{\kern0pt}\ second{\isacharunderscore}{\kern0pt}countable{\isacharunderscore}{\kern0pt}topology{\isacharbraceright}{\kern0pt}{\isacharparenright}{\kern0pt}{\isachardoublequoteclose}\isanewline
\ \ \isakeyword{assumes}\ {\isachardoublequoteopen}n\ {\isasymge}\ {\isadigit{1}}{\isachardoublequoteclose}\ \isanewline
\ \ \isakeyword{assumes}\ {\isachardoublequoteopen}{\isasymAnd}i{\isachardot}{\kern0pt}\ i\ {\isacharless}{\kern0pt}\ n\ {\isasymLongrightarrow}\ X\ i\ {\isasymin}\ measurable\ M\ borel{\isachardoublequoteclose}\isanewline
\ \ \isakeyword{shows}\ {\isachardoublequoteopen}{\isacharparenleft}{\kern0pt}{\isasymlambda}x{\isachardot}{\kern0pt}\ median\ n\ {\isacharparenleft}{\kern0pt}{\isasymlambda}i{\isachardot}{\kern0pt}\ X\ i\ x{\isacharparenright}{\kern0pt}{\isacharparenright}{\kern0pt}\ {\isasymin}\ measurable\ M\ borel{\isachardoublequoteclose}\isanewline
%
\isadelimproof
%
\endisadelimproof
%
\isatagproof
\isacommand{proof}\isamarkupfalse%
\ {\isacharminus}{\kern0pt}\isanewline
\ \ \isacommand{have}\isamarkupfalse%
\ n{\isacharunderscore}{\kern0pt}ge{\isacharunderscore}{\kern0pt}{\isadigit{0}}{\isacharcolon}{\kern0pt}{\isachardoublequoteopen}n\ {\isachargreater}{\kern0pt}\ {\isadigit{0}}{\isachardoublequoteclose}\ \isacommand{using}\isamarkupfalse%
\ assms\ \isacommand{by}\isamarkupfalse%
\ simp\isanewline
\ \ \isacommand{define}\isamarkupfalse%
\ is{\isacharunderscore}{\kern0pt}swap\ \isakeyword{where}\ {\isachardoublequoteopen}is{\isacharunderscore}{\kern0pt}swap\ {\isacharequal}{\kern0pt}\ {\isacharparenleft}{\kern0pt}{\isasymlambda}{\isacharparenleft}{\kern0pt}ts\ {\isacharcolon}{\kern0pt}{\isacharcolon}{\kern0pt}\ {\isacharparenleft}{\kern0pt}{\isacharparenleft}{\kern0pt}nat\ {\isasymRightarrow}\ {\isacharprime}{\kern0pt}b{\isacharparenright}{\kern0pt}\ {\isasymRightarrow}\ nat\ {\isasymRightarrow}\ {\isacharprime}{\kern0pt}b{\isacharparenright}{\kern0pt}{\isacharparenright}{\kern0pt}{\isachardot}{\kern0pt}\ {\isasymexists}i\ {\isacharless}{\kern0pt}\ n{\isachardot}{\kern0pt}\ {\isasymexists}j\ {\isacharless}{\kern0pt}\ n{\isachardot}{\kern0pt}\ ts\ {\isacharequal}{\kern0pt}\ sort{\isacharunderscore}{\kern0pt}primitive\ i\ j{\isacharparenright}{\kern0pt}{\isachardoublequoteclose}\isanewline
\ \ \isacommand{define}\isamarkupfalse%
\ t\ {\isacharcolon}{\kern0pt}{\isacharcolon}{\kern0pt}\ {\isachardoublequoteopen}{\isacharparenleft}{\kern0pt}{\isacharparenleft}{\kern0pt}nat\ {\isasymRightarrow}\ {\isacharprime}{\kern0pt}b{\isacharparenright}{\kern0pt}\ {\isasymRightarrow}\ nat\ {\isasymRightarrow}\ {\isacharprime}{\kern0pt}b{\isacharparenright}{\kern0pt}\ list{\isachardoublequoteclose}\ \isanewline
\ \ \ \ \isakeyword{where}\ {\isachardoublequoteopen}t\ {\isacharequal}{\kern0pt}\ {\isacharbrackleft}{\kern0pt}sort{\isacharunderscore}{\kern0pt}primitive\ j\ i{\isachardot}{\kern0pt}\ i\ {\isacharless}{\kern0pt}{\isacharminus}{\kern0pt}\ {\isacharbrackleft}{\kern0pt}{\isadigit{0}}{\isachardot}{\kern0pt}{\isachardot}{\kern0pt}{\isacharless}{\kern0pt}n{\isacharbrackright}{\kern0pt}{\isacharcomma}{\kern0pt}\ j\ {\isacharless}{\kern0pt}{\isacharminus}{\kern0pt}\ {\isacharbrackleft}{\kern0pt}{\isadigit{0}}{\isachardot}{\kern0pt}{\isachardot}{\kern0pt}{\isacharless}{\kern0pt}i{\isacharbrackright}{\kern0pt}{\isacharbrackright}{\kern0pt}{\isachardoublequoteclose}\isanewline
\isanewline
\ \ \isacommand{define}\isamarkupfalse%
\ meas{\isacharunderscore}{\kern0pt}ptw\ {\isacharcolon}{\kern0pt}{\isacharcolon}{\kern0pt}\ {\isachardoublequoteopen}{\isacharparenleft}{\kern0pt}nat\ {\isasymRightarrow}\ {\isacharprime}{\kern0pt}a\ {\isasymRightarrow}\ {\isacharprime}{\kern0pt}b{\isacharparenright}{\kern0pt}\ {\isasymRightarrow}\ bool{\isachardoublequoteclose}\isanewline
\ \ \ \ \isakeyword{where}\ {\isachardoublequoteopen}meas{\isacharunderscore}{\kern0pt}ptw\ {\isacharequal}{\kern0pt}\ {\isacharparenleft}{\kern0pt}{\isasymlambda}f{\isachardot}{\kern0pt}\ {\isacharparenleft}{\kern0pt}{\isasymforall}k{\isachardot}{\kern0pt}\ k\ {\isacharless}{\kern0pt}\ n\ {\isasymlongrightarrow}\ f\ k\ {\isasymin}\ borel{\isacharunderscore}{\kern0pt}measurable\ M{\isacharparenright}{\kern0pt}{\isacharparenright}{\kern0pt}{\isachardoublequoteclose}\isanewline
\isanewline
\ \ \isacommand{have}\isamarkupfalse%
\ ind{\isacharunderscore}{\kern0pt}step{\isacharcolon}{\kern0pt}\isanewline
\ \ \ \ {\isachardoublequoteopen}{\isasymAnd}x\ {\isacharparenleft}{\kern0pt}g\ {\isacharcolon}{\kern0pt}{\isacharcolon}{\kern0pt}\ nat\ {\isasymRightarrow}\ {\isacharprime}{\kern0pt}a\ {\isasymRightarrow}\ {\isacharprime}{\kern0pt}b{\isacharparenright}{\kern0pt}{\isachardot}{\kern0pt}\ meas{\isacharunderscore}{\kern0pt}ptw\ g\ {\isasymLongrightarrow}\ is{\isacharunderscore}{\kern0pt}swap\ x\ {\isasymLongrightarrow}\ meas{\isacharunderscore}{\kern0pt}ptw\ {\isacharparenleft}{\kern0pt}{\isasymlambda}k\ {\isasymomega}{\isachardot}{\kern0pt}\ x\ {\isacharparenleft}{\kern0pt}{\isasymlambda}i{\isachardot}{\kern0pt}\ g\ i\ {\isasymomega}{\isacharparenright}{\kern0pt}\ k{\isacharparenright}{\kern0pt}{\isachardoublequoteclose}\isanewline
\ \ \isacommand{proof}\isamarkupfalse%
\ {\isacharminus}{\kern0pt}\isanewline
\ \ \ \ \isacommand{fix}\isamarkupfalse%
\ x\ g\isanewline
\ \ \ \ \isacommand{assume}\isamarkupfalse%
\ {\isachardoublequoteopen}meas{\isacharunderscore}{\kern0pt}ptw\ g{\isachardoublequoteclose}\isanewline
\ \ \ \ \isacommand{hence}\isamarkupfalse%
\ a{\isacharcolon}{\kern0pt}{\isachardoublequoteopen}{\isasymAnd}k{\isachardot}{\kern0pt}\ k\ {\isacharless}{\kern0pt}\ n\ {\isasymLongrightarrow}\ g\ k\ {\isasymin}\ borel{\isacharunderscore}{\kern0pt}measurable\ M{\isachardoublequoteclose}\ \isacommand{by}\isamarkupfalse%
\ {\isacharparenleft}{\kern0pt}simp\ add{\isacharcolon}{\kern0pt}meas{\isacharunderscore}{\kern0pt}ptw{\isacharunderscore}{\kern0pt}def{\isacharparenright}{\kern0pt}\isanewline
\ \ \ \ \isacommand{assume}\isamarkupfalse%
\ {\isachardoublequoteopen}is{\isacharunderscore}{\kern0pt}swap\ x{\isachardoublequoteclose}\isanewline
\ \ \ \ \isacommand{then}\isamarkupfalse%
\ \isacommand{obtain}\isamarkupfalse%
\ i\ j\ \isakeyword{where}\ x{\isacharunderscore}{\kern0pt}def{\isacharcolon}{\kern0pt}{\isachardoublequoteopen}x{\isacharequal}{\kern0pt}sort{\isacharunderscore}{\kern0pt}primitive\ i\ j{\isachardoublequoteclose}\ \isakeyword{and}\ i{\isacharunderscore}{\kern0pt}le{\isacharcolon}{\kern0pt}{\isachardoublequoteopen}i\ {\isacharless}{\kern0pt}\ n{\isachardoublequoteclose}\ \isakeyword{and}\ j{\isacharunderscore}{\kern0pt}le{\isacharcolon}{\kern0pt}{\isachardoublequoteopen}j\ {\isacharless}{\kern0pt}\ n{\isachardoublequoteclose}\isanewline
\ \ \ \ \ \ \isacommand{apply}\isamarkupfalse%
\ {\isacharparenleft}{\kern0pt}simp\ add{\isacharcolon}{\kern0pt}is{\isacharunderscore}{\kern0pt}swap{\isacharunderscore}{\kern0pt}def{\isacharparenright}{\kern0pt}\ \isacommand{by}\isamarkupfalse%
\ blast\isanewline
\ \ \ \ \isacommand{have}\isamarkupfalse%
\ {\isachardoublequoteopen}{\isasymAnd}k{\isachardot}{\kern0pt}\ k\ {\isacharless}{\kern0pt}\ n\ {\isasymLongrightarrow}\ {\isacharparenleft}{\kern0pt}{\isasymlambda}{\isasymomega}{\isachardot}{\kern0pt}\ x\ {\isacharparenleft}{\kern0pt}{\isasymlambda}i{\isachardot}{\kern0pt}\ g\ i\ {\isasymomega}{\isacharparenright}{\kern0pt}\ k{\isacharparenright}{\kern0pt}\ {\isasymin}\ borel{\isacharunderscore}{\kern0pt}measurable\ M{\isachardoublequoteclose}\isanewline
\ \ \ \ \isacommand{proof}\isamarkupfalse%
\ {\isacharminus}{\kern0pt}\isanewline
\ \ \ \ \ \ \isacommand{fix}\isamarkupfalse%
\ k\isanewline
\ \ \ \ \ \ \isacommand{assume}\isamarkupfalse%
\ {\isachardoublequoteopen}k\ {\isacharless}{\kern0pt}\ n{\isachardoublequoteclose}\isanewline
\ \ \ \ \ \ \isacommand{thus}\isamarkupfalse%
\ {\isachardoublequoteopen}\ {\isacharparenleft}{\kern0pt}{\isasymlambda}{\isasymomega}{\isachardot}{\kern0pt}\ x\ {\isacharparenleft}{\kern0pt}{\isasymlambda}i{\isachardot}{\kern0pt}\ g\ i\ {\isasymomega}{\isacharparenright}{\kern0pt}\ k{\isacharparenright}{\kern0pt}\ {\isasymin}\ borel{\isacharunderscore}{\kern0pt}measurable\ M{\isachardoublequoteclose}\isanewline
\ \ \ \ \ \ \ \ \isacommand{apply}\isamarkupfalse%
\ {\isacharparenleft}{\kern0pt}simp\ add{\isacharcolon}{\kern0pt}x{\isacharunderscore}{\kern0pt}def{\isacharparenright}{\kern0pt}\isanewline
\ \ \ \ \ \ \ \ \isacommand{apply}\isamarkupfalse%
\ {\isacharparenleft}{\kern0pt}cases\ {\isachardoublequoteopen}k\ {\isacharequal}{\kern0pt}\ i{\isachardoublequoteclose}{\isacharcomma}{\kern0pt}\ simp{\isacharparenright}{\kern0pt}\isanewline
\ \ \ \ \ \ \ \ \isacommand{using}\isamarkupfalse%
\ a\ i{\isacharunderscore}{\kern0pt}le\ j{\isacharunderscore}{\kern0pt}le\ borel{\isacharunderscore}{\kern0pt}measurable{\isacharunderscore}{\kern0pt}min\ \isacommand{apply}\isamarkupfalse%
\ blast\isanewline
\ \ \ \ \ \ \ \ \isacommand{apply}\isamarkupfalse%
\ {\isacharparenleft}{\kern0pt}cases\ {\isachardoublequoteopen}k\ {\isacharequal}{\kern0pt}\ j{\isachardoublequoteclose}{\isacharcomma}{\kern0pt}\ simp{\isacharparenright}{\kern0pt}\isanewline
\ \ \ \ \ \ \ \ \isacommand{using}\isamarkupfalse%
\ a\ i{\isacharunderscore}{\kern0pt}le\ j{\isacharunderscore}{\kern0pt}le\ borel{\isacharunderscore}{\kern0pt}measurable{\isacharunderscore}{\kern0pt}max\ \isacommand{apply}\isamarkupfalse%
\ blast\isanewline
\ \ \ \ \ \ \ \ \isacommand{using}\isamarkupfalse%
\ a\ \isacommand{by}\isamarkupfalse%
\ simp\isanewline
\ \ \ \ \isacommand{qed}\isamarkupfalse%
\isanewline
\ \ \ \ \isacommand{thus}\isamarkupfalse%
\ {\isachardoublequoteopen}meas{\isacharunderscore}{\kern0pt}ptw\ {\isacharparenleft}{\kern0pt}{\isasymlambda}k\ {\isasymomega}{\isachardot}{\kern0pt}\ x\ {\isacharparenleft}{\kern0pt}{\isasymlambda}i{\isachardot}{\kern0pt}\ g\ i\ {\isasymomega}{\isacharparenright}{\kern0pt}\ k{\isacharparenright}{\kern0pt}{\isachardoublequoteclose}\ \isanewline
\ \ \ \ \ \ \isacommand{by}\isamarkupfalse%
\ {\isacharparenleft}{\kern0pt}simp\ add{\isacharcolon}{\kern0pt}meas{\isacharunderscore}{\kern0pt}ptw{\isacharunderscore}{\kern0pt}def{\isacharparenright}{\kern0pt}\isanewline
\ \ \isacommand{qed}\isamarkupfalse%
\isanewline
\isanewline
\ \ \isacommand{have}\isamarkupfalse%
\ {\isachardoublequoteopen}{\isacharparenleft}{\kern0pt}{\isasymforall}x\ {\isasymin}\ set\ t{\isachardot}{\kern0pt}\ is{\isacharunderscore}{\kern0pt}swap\ x{\isacharparenright}{\kern0pt}\ {\isasymLongrightarrow}\ meas{\isacharunderscore}{\kern0pt}ptw\ {\isacharparenleft}{\kern0pt}{\isasymlambda}\ k\ {\isasymomega}{\isachardot}{\kern0pt}\ {\isacharparenleft}{\kern0pt}fold\ id\ t\ {\isacharparenleft}{\kern0pt}{\isasymlambda}k{\isachardot}{\kern0pt}\ X\ k\ {\isasymomega}{\isacharparenright}{\kern0pt}{\isacharparenright}{\kern0pt}\ k{\isacharparenright}{\kern0pt}{\isachardoublequoteclose}\isanewline
\ \ \isacommand{proof}\isamarkupfalse%
\ {\isacharparenleft}{\kern0pt}induction\ t\ rule{\isacharcolon}{\kern0pt}rev{\isacharunderscore}{\kern0pt}induct{\isacharparenright}{\kern0pt}\isanewline
\ \ \ \ \isacommand{case}\isamarkupfalse%
\ Nil\isanewline
\ \ \ \ \isacommand{then}\isamarkupfalse%
\ \isacommand{show}\isamarkupfalse%
\ {\isacharquery}{\kern0pt}case\ \isacommand{using}\isamarkupfalse%
\ assms\ \isacommand{by}\isamarkupfalse%
\ {\isacharparenleft}{\kern0pt}simp\ add{\isacharcolon}{\kern0pt}meas{\isacharunderscore}{\kern0pt}ptw{\isacharunderscore}{\kern0pt}def{\isacharparenright}{\kern0pt}\isanewline
\ \ \isacommand{next}\isamarkupfalse%
\isanewline
\ \ \ \ \isacommand{case}\isamarkupfalse%
\ {\isacharparenleft}{\kern0pt}snoc\ x\ xs{\isacharparenright}{\kern0pt}\isanewline
\ \ \ \ \isacommand{have}\isamarkupfalse%
\ a{\isacharcolon}{\kern0pt}{\isachardoublequoteopen}meas{\isacharunderscore}{\kern0pt}ptw\ {\isacharparenleft}{\kern0pt}{\isasymlambda}k\ {\isasymomega}{\isachardot}{\kern0pt}\ fold\ {\isacharparenleft}{\kern0pt}{\isasymlambda}a{\isachardot}{\kern0pt}\ a{\isacharparenright}{\kern0pt}\ xs\ {\isacharparenleft}{\kern0pt}{\isasymlambda}k{\isachardot}{\kern0pt}\ X\ k\ {\isasymomega}{\isacharparenright}{\kern0pt}\ k{\isacharparenright}{\kern0pt}{\isachardoublequoteclose}\ \isacommand{using}\isamarkupfalse%
\ snoc\ \isacommand{by}\isamarkupfalse%
\ simp\isanewline
\ \ \ \ \isacommand{have}\isamarkupfalse%
\ b{\isacharcolon}{\kern0pt}{\isachardoublequoteopen}is{\isacharunderscore}{\kern0pt}swap\ x{\isachardoublequoteclose}\ \isacommand{using}\isamarkupfalse%
\ snoc\ \isacommand{by}\isamarkupfalse%
\ simp\isanewline
\ \ \ \ \isacommand{show}\isamarkupfalse%
\ {\isacharquery}{\kern0pt}case\ \isacommand{apply}\isamarkupfalse%
\ simp\isanewline
\ \ \ \ \ \ \isacommand{using}\isamarkupfalse%
\ ind{\isacharunderscore}{\kern0pt}step{\isacharbrackleft}{\kern0pt}OF\ a\ b{\isacharbrackright}{\kern0pt}\ \isacommand{by}\isamarkupfalse%
\ simp\isanewline
\ \ \isacommand{qed}\isamarkupfalse%
\isanewline
\ \ \isacommand{moreover}\isamarkupfalse%
\ \isacommand{have}\isamarkupfalse%
\ {\isachardoublequoteopen}{\isasymAnd}x{\isachardot}{\kern0pt}\ x\ {\isasymin}\ set\ t\ {\isasymLongrightarrow}\ is{\isacharunderscore}{\kern0pt}swap\ x{\isachardoublequoteclose}\ \isanewline
\ \ \ \ \isacommand{apply}\isamarkupfalse%
\ {\isacharparenleft}{\kern0pt}simp\ add{\isacharcolon}{\kern0pt}t{\isacharunderscore}{\kern0pt}def\ is{\isacharunderscore}{\kern0pt}swap{\isacharunderscore}{\kern0pt}def{\isacharparenright}{\kern0pt}\ \isanewline
\ \ \ \ \isacommand{by}\isamarkupfalse%
\ {\isacharparenleft}{\kern0pt}meson\ atLeastLessThan{\isacharunderscore}{\kern0pt}iff\ imageE\ less{\isacharunderscore}{\kern0pt}imp{\isacharunderscore}{\kern0pt}le\ less{\isacharunderscore}{\kern0pt}le{\isacharunderscore}{\kern0pt}trans{\isacharparenright}{\kern0pt}\ \ \isanewline
\ \ \isacommand{moreover}\isamarkupfalse%
\ \isacommand{have}\isamarkupfalse%
\ {\isachardoublequoteopen}n\ div\ {\isadigit{2}}\ {\isacharless}{\kern0pt}\ n{\isachardoublequoteclose}\ \isacommand{using}\isamarkupfalse%
\ n{\isacharunderscore}{\kern0pt}ge{\isacharunderscore}{\kern0pt}{\isadigit{0}}\ \isacommand{by}\isamarkupfalse%
\ simp\isanewline
\ \ \isacommand{ultimately}\isamarkupfalse%
\ \isacommand{show}\isamarkupfalse%
\ {\isacharquery}{\kern0pt}thesis\isanewline
\ \ \ \ \isacommand{apply}\isamarkupfalse%
\ {\isacharparenleft}{\kern0pt}subst\ median{\isacharunderscore}{\kern0pt}alt{\isacharunderscore}{\kern0pt}def{\isacharbrackleft}{\kern0pt}OF\ n{\isacharunderscore}{\kern0pt}ge{\isacharunderscore}{\kern0pt}{\isadigit{0}}{\isacharbrackright}{\kern0pt}{\isacharparenright}{\kern0pt}\isanewline
\ \ \ \ \isacommand{by}\isamarkupfalse%
\ {\isacharparenleft}{\kern0pt}simp\ add{\isacharcolon}{\kern0pt}t{\isacharunderscore}{\kern0pt}def{\isacharbrackleft}{\kern0pt}symmetric{\isacharbrackright}{\kern0pt}\ meas{\isacharunderscore}{\kern0pt}ptw{\isacharunderscore}{\kern0pt}def{\isacharparenright}{\kern0pt}\isanewline
\isacommand{qed}\isamarkupfalse%
%
\endisatagproof
{\isafoldproof}%
%
\isadelimproof
\isanewline
%
\endisadelimproof
\isanewline
\isacommand{lemma}\isamarkupfalse%
\ {\isacharparenleft}{\kern0pt}\isakeyword{in}\ prob{\isacharunderscore}{\kern0pt}space{\isacharparenright}{\kern0pt}\ median{\isacharunderscore}{\kern0pt}bound{\isacharcolon}{\kern0pt}\isanewline
\ \ \isakeyword{fixes}\ n\ {\isacharcolon}{\kern0pt}{\isacharcolon}{\kern0pt}\ nat\isanewline
\ \ \isakeyword{fixes}\ I\ {\isacharcolon}{\kern0pt}{\isacharcolon}{\kern0pt}\ {\isachardoublequoteopen}{\isacharparenleft}{\kern0pt}{\isacharprime}{\kern0pt}b\ {\isacharcolon}{\kern0pt}{\isacharcolon}{\kern0pt}\ {\isacharbraceleft}{\kern0pt}linorder{\isacharunderscore}{\kern0pt}topology{\isacharcomma}{\kern0pt}\ second{\isacharunderscore}{\kern0pt}countable{\isacharunderscore}{\kern0pt}topology{\isacharbraceright}{\kern0pt}{\isacharparenright}{\kern0pt}\ set{\isachardoublequoteclose}\isanewline
\ \ \isakeyword{assumes}\ {\isachardoublequoteopen}interval\ I{\isachardoublequoteclose}\isanewline
\ \ \isakeyword{assumes}\ {\isachardoublequoteopen}{\isasymalpha}\ {\isachargreater}{\kern0pt}\ {\isadigit{0}}{\isachardoublequoteclose}\isanewline
\ \ \isakeyword{assumes}\ {\isachardoublequoteopen}{\isasymepsilon}\ {\isasymin}\ {\isacharbraceleft}{\kern0pt}{\isadigit{0}}{\isacharless}{\kern0pt}{\isachardot}{\kern0pt}{\isachardot}{\kern0pt}{\isacharless}{\kern0pt}{\isadigit{1}}{\isacharbraceright}{\kern0pt}{\isachardoublequoteclose}\isanewline
\ \ \isakeyword{assumes}\ {\isachardoublequoteopen}indep{\isacharunderscore}{\kern0pt}vars\ {\isacharparenleft}{\kern0pt}{\isasymlambda}{\isacharunderscore}{\kern0pt}{\isachardot}{\kern0pt}\ borel{\isacharparenright}{\kern0pt}\ X\ {\isacharbraceleft}{\kern0pt}{\isadigit{0}}{\isachardot}{\kern0pt}{\isachardot}{\kern0pt}{\isacharless}{\kern0pt}n{\isacharbraceright}{\kern0pt}{\isachardoublequoteclose}\isanewline
\ \ \isakeyword{assumes}\ {\isachardoublequoteopen}n\ {\isasymge}\ {\isacharminus}{\kern0pt}\ ln\ {\isasymepsilon}\ {\isacharslash}{\kern0pt}\ {\isacharparenleft}{\kern0pt}{\isadigit{2}}\ {\isacharasterisk}{\kern0pt}\ {\isasymalpha}\isactrlsup {\isadigit{2}}{\isacharparenright}{\kern0pt}{\isachardoublequoteclose}\isanewline
\ \ \isakeyword{assumes}\ {\isachardoublequoteopen}{\isasymAnd}i{\isachardot}{\kern0pt}\ i\ {\isacharless}{\kern0pt}\ n\ {\isasymLongrightarrow}\ {\isasymP}{\isacharparenleft}{\kern0pt}{\isasymomega}\ in\ M{\isachardot}{\kern0pt}\ X\ i\ {\isasymomega}\ {\isasymin}\ I{\isacharparenright}{\kern0pt}\ {\isasymge}\ {\isadigit{1}}{\isacharslash}{\kern0pt}{\isadigit{2}}{\isacharplus}{\kern0pt}{\isasymalpha}{\isachardoublequoteclose}\ \isanewline
\ \ \isakeyword{shows}\ {\isachardoublequoteopen}{\isasymP}{\isacharparenleft}{\kern0pt}{\isasymomega}\ in\ M{\isachardot}{\kern0pt}\ median\ n\ {\isacharparenleft}{\kern0pt}{\isasymlambda}i{\isachardot}{\kern0pt}\ X\ i\ {\isasymomega}{\isacharparenright}{\kern0pt}\ {\isasymin}\ I{\isacharparenright}{\kern0pt}\ {\isasymge}\ {\isadigit{1}}{\isacharminus}{\kern0pt}{\isasymepsilon}{\isachardoublequoteclose}\ {\isacharparenleft}{\kern0pt}\isakeyword{is}\ {\isachardoublequoteopen}{\isasymP}{\isacharparenleft}{\kern0pt}{\isasymomega}\ in\ M{\isachardot}{\kern0pt}\ {\isacharquery}{\kern0pt}lhs\ {\isasymomega}{\isacharparenright}{\kern0pt}\ {\isasymge}\ {\isacharquery}{\kern0pt}C{\isachardoublequoteclose}{\isacharparenright}{\kern0pt}\ \isanewline
%
\isadelimproof
%
\endisadelimproof
%
\isatagproof
\isacommand{proof}\isamarkupfalse%
\ {\isacharminus}{\kern0pt}\isanewline
\ \ \isacommand{define}\isamarkupfalse%
\ Y\ {\isacharcolon}{\kern0pt}{\isacharcolon}{\kern0pt}\ {\isachardoublequoteopen}nat\ {\isasymRightarrow}\ {\isacharprime}{\kern0pt}a\ {\isasymRightarrow}\ real{\isachardoublequoteclose}\ \isakeyword{where}\ {\isachardoublequoteopen}Y\ {\isacharequal}{\kern0pt}\ {\isacharparenleft}{\kern0pt}{\isasymlambda}i{\isachardot}{\kern0pt}\ indicator\ I\ {\isasymcirc}\ {\isacharparenleft}{\kern0pt}X\ i{\isacharparenright}{\kern0pt}{\isacharparenright}{\kern0pt}{\isachardoublequoteclose}\isanewline
\isanewline
\ \ \isacommand{define}\isamarkupfalse%
\ t\ \isakeyword{where}\ {\isachardoublequoteopen}t\ {\isacharequal}{\kern0pt}\ {\isacharparenleft}{\kern0pt}{\isasymSum}i\ {\isacharequal}{\kern0pt}\ {\isadigit{0}}{\isachardot}{\kern0pt}{\isachardot}{\kern0pt}{\isacharless}{\kern0pt}n{\isachardot}{\kern0pt}\ expectation\ {\isacharparenleft}{\kern0pt}Y\ i{\isacharparenright}{\kern0pt}{\isacharparenright}{\kern0pt}\ {\isacharminus}{\kern0pt}\ n{\isacharslash}{\kern0pt}{\isadigit{2}}{\isachardoublequoteclose}\isanewline
\ \ \isacommand{have}\isamarkupfalse%
\ {\isachardoublequoteopen}{\isadigit{0}}\ {\isacharless}{\kern0pt}\ {\isacharminus}{\kern0pt}ln\ {\isasymepsilon}\ {\isacharslash}{\kern0pt}\ {\isacharparenleft}{\kern0pt}{\isadigit{2}}\ {\isacharasterisk}{\kern0pt}\ {\isasymalpha}\isactrlsup {\isadigit{2}}{\isacharparenright}{\kern0pt}{\isachardoublequoteclose}\ \ \isanewline
\ \ \ \ \isacommand{apply}\isamarkupfalse%
\ {\isacharparenleft}{\kern0pt}rule\ divide{\isacharunderscore}{\kern0pt}pos{\isacharunderscore}{\kern0pt}pos{\isacharparenright}{\kern0pt}\isanewline
\ \ \ \ \isacommand{apply}\isamarkupfalse%
\ {\isacharparenleft}{\kern0pt}simp{\isacharcomma}{\kern0pt}\ subst\ ln{\isacharunderscore}{\kern0pt}less{\isacharunderscore}{\kern0pt}zero{\isacharunderscore}{\kern0pt}iff{\isacharparenright}{\kern0pt}\isanewline
\ \ \ \ \isacommand{using}\isamarkupfalse%
\ assms\ \isacommand{by}\isamarkupfalse%
\ auto\isanewline
\ \ \isacommand{also}\isamarkupfalse%
\ \isacommand{have}\isamarkupfalse%
\ {\isachardoublequoteopen}{\isachardot}{\kern0pt}{\isachardot}{\kern0pt}{\isachardot}{\kern0pt}\ {\isasymle}\ real\ n{\isachardoublequoteclose}\ \isacommand{using}\isamarkupfalse%
\ assms\ \isacommand{by}\isamarkupfalse%
\ simp\isanewline
\ \ \isacommand{finally}\isamarkupfalse%
\ \isacommand{have}\isamarkupfalse%
\ {\isachardoublequoteopen}real\ n\ {\isachargreater}{\kern0pt}\ {\isadigit{0}}{\isachardoublequoteclose}\ \isacommand{by}\isamarkupfalse%
\ simp\isanewline
\ \ \isacommand{hence}\isamarkupfalse%
\ n{\isacharunderscore}{\kern0pt}ge{\isacharunderscore}{\kern0pt}{\isadigit{1}}{\isacharcolon}{\kern0pt}{\isachardoublequoteopen}n\ {\isasymge}\ {\isadigit{1}}{\isachardoublequoteclose}\ \isacommand{by}\isamarkupfalse%
\ linarith\isanewline
\ \ \isacommand{hence}\isamarkupfalse%
\ n{\isacharunderscore}{\kern0pt}ge{\isacharunderscore}{\kern0pt}{\isadigit{0}}{\isacharcolon}{\kern0pt}{\isachardoublequoteopen}n\ {\isachargreater}{\kern0pt}\ {\isadigit{0}}{\isachardoublequoteclose}\ \isacommand{by}\isamarkupfalse%
\ simp\isanewline
\isanewline
\ \ \isacommand{have}\isamarkupfalse%
\ ind{\isacharunderscore}{\kern0pt}comp{\isacharcolon}{\kern0pt}\ {\isachardoublequoteopen}{\isasymAnd}i{\isachardot}{\kern0pt}\ indicator\ I\ {\isasymcirc}\ {\isacharparenleft}{\kern0pt}X\ i{\isacharparenright}{\kern0pt}\ {\isacharequal}{\kern0pt}\ indicator\ {\isacharbraceleft}{\kern0pt}{\isasymomega}{\isachardot}{\kern0pt}\ X\ i\ {\isasymomega}\ {\isasymin}\ I{\isacharbraceright}{\kern0pt}{\isachardoublequoteclose}\isanewline
\ \ \ \ \isacommand{apply}\isamarkupfalse%
\ {\isacharparenleft}{\kern0pt}rule\ ext{\isacharparenright}{\kern0pt}\isanewline
\ \ \ \ \isacommand{by}\isamarkupfalse%
\ {\isacharparenleft}{\kern0pt}simp\ add{\isacharcolon}{\kern0pt}indicator{\isacharunderscore}{\kern0pt}def\ comp{\isacharunderscore}{\kern0pt}def{\isacharparenright}{\kern0pt}\isanewline
\isanewline
\ \ \isacommand{have}\isamarkupfalse%
\ {\isachardoublequoteopen}{\isasymalpha}\ {\isacharasterisk}{\kern0pt}\ n\ {\isasymle}\ {\isacharparenleft}{\kern0pt}{\isasymSum}\ i\ {\isacharequal}{\kern0pt}{\isadigit{0}}{\isachardot}{\kern0pt}{\isachardot}{\kern0pt}{\isacharless}{\kern0pt}n{\isachardot}{\kern0pt}\ {\isadigit{1}}{\isacharslash}{\kern0pt}{\isadigit{2}}\ {\isacharplus}{\kern0pt}\ {\isasymalpha}{\isacharparenright}{\kern0pt}\ {\isacharminus}{\kern0pt}\ n{\isacharslash}{\kern0pt}{\isadigit{2}}{\isachardoublequoteclose}\isanewline
\ \ \ \ \isacommand{by}\isamarkupfalse%
\ {\isacharparenleft}{\kern0pt}simp\ add{\isacharcolon}{\kern0pt}algebra{\isacharunderscore}{\kern0pt}simps{\isacharparenright}{\kern0pt}\isanewline
\ \ \isacommand{also}\isamarkupfalse%
\ \isacommand{have}\isamarkupfalse%
\ {\isachardoublequoteopen}{\isachardot}{\kern0pt}{\isachardot}{\kern0pt}{\isachardot}{\kern0pt}\ {\isasymle}\ {\isacharparenleft}{\kern0pt}{\isasymSum}\ i\ {\isacharequal}{\kern0pt}\ {\isadigit{0}}{\isachardot}{\kern0pt}{\isachardot}{\kern0pt}{\isacharless}{\kern0pt}n{\isachardot}{\kern0pt}\ expectation\ {\isacharparenleft}{\kern0pt}Y\ i{\isacharparenright}{\kern0pt}{\isacharparenright}{\kern0pt}\ {\isacharminus}{\kern0pt}\ n{\isacharslash}{\kern0pt}{\isadigit{2}}{\isachardoublequoteclose}\isanewline
\ \ \ \ \isacommand{apply}\isamarkupfalse%
\ {\isacharparenleft}{\kern0pt}rule\ diff{\isacharunderscore}{\kern0pt}right{\isacharunderscore}{\kern0pt}mono{\isacharcomma}{\kern0pt}\ rule\ sum{\isacharunderscore}{\kern0pt}mono{\isacharparenright}{\kern0pt}\isanewline
\ \ \ \ \isacommand{using}\isamarkupfalse%
\ assms{\isacharparenleft}{\kern0pt}{\isadigit{6}}{\isacharparenright}{\kern0pt}\ \isacommand{by}\isamarkupfalse%
\ {\isacharparenleft}{\kern0pt}simp\ add{\isacharcolon}{\kern0pt}Y{\isacharunderscore}{\kern0pt}def\ ind{\isacharunderscore}{\kern0pt}comp\ measure{\isacharunderscore}{\kern0pt}inters{\isacharparenright}{\kern0pt}\ \isanewline
\ \ \isacommand{also}\isamarkupfalse%
\ \isacommand{have}\isamarkupfalse%
\ {\isachardoublequoteopen}{\isachardot}{\kern0pt}{\isachardot}{\kern0pt}{\isachardot}{\kern0pt}\ {\isacharequal}{\kern0pt}\ t{\isachardoublequoteclose}\ \isacommand{by}\isamarkupfalse%
\ {\isacharparenleft}{\kern0pt}simp\ add{\isacharcolon}{\kern0pt}t{\isacharunderscore}{\kern0pt}def{\isacharparenright}{\kern0pt}\isanewline
\ \ \isacommand{finally}\isamarkupfalse%
\ \isacommand{have}\isamarkupfalse%
\ t{\isacharunderscore}{\kern0pt}ge{\isacharunderscore}{\kern0pt}a{\isacharcolon}{\kern0pt}\ {\isachardoublequoteopen}t\ {\isasymge}\ {\isasymalpha}\ {\isacharasterisk}{\kern0pt}\ n{\isachardoublequoteclose}\ \isacommand{by}\isamarkupfalse%
\ simp\isanewline
\isanewline
\ \ \isacommand{have}\isamarkupfalse%
\ d{\isacharcolon}{\kern0pt}\ {\isachardoublequoteopen}{\isadigit{0}}\ {\isasymle}\ {\isasymalpha}\ {\isacharasterisk}{\kern0pt}\ n{\isachardoublequoteclose}\ \isanewline
\ \ \ \ \isacommand{apply}\isamarkupfalse%
\ {\isacharparenleft}{\kern0pt}rule\ mult{\isacharunderscore}{\kern0pt}nonneg{\isacharunderscore}{\kern0pt}nonneg{\isacharparenright}{\kern0pt}\isanewline
\ \ \ \ \isacommand{using}\isamarkupfalse%
\ assms{\isacharparenleft}{\kern0pt}{\isadigit{2}}{\isacharparenright}{\kern0pt}\ n{\isacharunderscore}{\kern0pt}ge{\isacharunderscore}{\kern0pt}{\isadigit{0}}\ \isacommand{by}\isamarkupfalse%
\ simp{\isacharplus}{\kern0pt}\isanewline
\ \ \isacommand{also}\isamarkupfalse%
\ \isacommand{have}\isamarkupfalse%
\ {\isachardoublequoteopen}{\isachardot}{\kern0pt}{\isachardot}{\kern0pt}{\isachardot}{\kern0pt}\ {\isasymle}\ t{\isachardoublequoteclose}\ \isacommand{using}\isamarkupfalse%
\ t{\isacharunderscore}{\kern0pt}ge{\isacharunderscore}{\kern0pt}a\ \isacommand{by}\isamarkupfalse%
\ simp\isanewline
\ \ \isacommand{finally}\isamarkupfalse%
\ \isacommand{have}\isamarkupfalse%
\ t{\isacharunderscore}{\kern0pt}ge{\isacharunderscore}{\kern0pt}{\isadigit{0}}{\isacharcolon}{\kern0pt}\ {\isachardoublequoteopen}t\ {\isasymge}\ {\isadigit{0}}{\isachardoublequoteclose}\ \isacommand{by}\isamarkupfalse%
\ simp\isanewline
\isanewline
\ \ \isacommand{have}\isamarkupfalse%
\ \ {\isachardoublequoteopen}{\isacharparenleft}{\kern0pt}{\isasymalpha}\ {\isacharasterisk}{\kern0pt}\ n{\isacharparenright}{\kern0pt}\isactrlsup {\isadigit{2}}\ {\isasymle}\ t\isactrlsup {\isadigit{2}}{\isachardoublequoteclose}\ \isacommand{using}\isamarkupfalse%
\ t{\isacharunderscore}{\kern0pt}ge{\isacharunderscore}{\kern0pt}a\ d\ power{\isacharunderscore}{\kern0pt}mono\ \isacommand{by}\isamarkupfalse%
\ blast\isanewline
\ \ \isacommand{hence}\isamarkupfalse%
\ t{\isacharunderscore}{\kern0pt}ge{\isacharunderscore}{\kern0pt}a{\isacharunderscore}{\kern0pt}sq{\isacharcolon}{\kern0pt}\ {\isachardoublequoteopen}{\isasymalpha}\isactrlsup {\isadigit{2}}\ {\isacharasterisk}{\kern0pt}\ real\ n\ {\isacharasterisk}{\kern0pt}\ real\ n\ {\isasymle}\ t\isactrlsup {\isadigit{2}}{\isachardoublequoteclose}\isanewline
\ \ \ \ \isacommand{by}\isamarkupfalse%
\ {\isacharparenleft}{\kern0pt}simp\ add{\isacharcolon}{\kern0pt}algebra{\isacharunderscore}{\kern0pt}simps\ power{\isadigit{2}}{\isacharunderscore}{\kern0pt}eq{\isacharunderscore}{\kern0pt}square{\isacharparenright}{\kern0pt}\isanewline
\isanewline
\ \ \isacommand{have}\isamarkupfalse%
\ Y{\isacharunderscore}{\kern0pt}indep{\isacharcolon}{\kern0pt}\ {\isachardoublequoteopen}indep{\isacharunderscore}{\kern0pt}vars\ {\isacharparenleft}{\kern0pt}{\isasymlambda}{\isacharunderscore}{\kern0pt}{\isachardot}{\kern0pt}\ borel{\isacharparenright}{\kern0pt}\ Y\ {\isacharbraceleft}{\kern0pt}{\isadigit{0}}{\isachardot}{\kern0pt}{\isachardot}{\kern0pt}{\isacharless}{\kern0pt}n{\isacharbraceright}{\kern0pt}{\isachardoublequoteclose}\isanewline
\ \ \ \ \isacommand{apply}\isamarkupfalse%
\ {\isacharparenleft}{\kern0pt}subst\ Y{\isacharunderscore}{\kern0pt}def{\isacharparenright}{\kern0pt}\isanewline
\ \ \ \ \isacommand{apply}\isamarkupfalse%
\ {\isacharparenleft}{\kern0pt}rule\ indep{\isacharunderscore}{\kern0pt}vars{\isacharunderscore}{\kern0pt}compose{\isacharbrackleft}{\kern0pt}\isakeyword{where}\ M{\isacharprime}{\kern0pt}{\isacharequal}{\kern0pt}{\isachardoublequoteopen}{\isacharparenleft}{\kern0pt}{\isasymlambda}{\isacharunderscore}{\kern0pt}{\isachardot}{\kern0pt}\ borel{\isacharparenright}{\kern0pt}{\isachardoublequoteclose}{\isacharbrackright}{\kern0pt}{\isacharparenright}{\kern0pt}\isanewline
\ \ \ \ \ \isacommand{apply}\isamarkupfalse%
\ {\isacharparenleft}{\kern0pt}metis\ assms{\isacharparenleft}{\kern0pt}{\isadigit{4}}{\isacharparenright}{\kern0pt}{\isacharparenright}{\kern0pt}\isanewline
\ \ \ \ \isacommand{using}\isamarkupfalse%
\ interval{\isacharunderscore}{\kern0pt}borel{\isacharbrackleft}{\kern0pt}OF\ assms{\isacharparenleft}{\kern0pt}{\isadigit{1}}{\isacharparenright}{\kern0pt}{\isacharbrackright}{\kern0pt}\ \isacommand{by}\isamarkupfalse%
\ simp\isanewline
\ \isanewline
\ \ \isacommand{hence}\isamarkupfalse%
\ b{\isacharcolon}{\kern0pt}{\isachardoublequoteopen}Hoeffding{\isacharunderscore}{\kern0pt}ineq\ M\ {\isacharbraceleft}{\kern0pt}{\isadigit{0}}{\isachardot}{\kern0pt}{\isachardot}{\kern0pt}{\isacharless}{\kern0pt}n{\isacharbraceright}{\kern0pt}\ Y\ {\isacharparenleft}{\kern0pt}{\isasymlambda}i{\isachardot}{\kern0pt}\ {\isadigit{0}}{\isacharparenright}{\kern0pt}\ {\isacharparenleft}{\kern0pt}{\isasymlambda}i{\isachardot}{\kern0pt}\ {\isadigit{1}}{\isacharparenright}{\kern0pt}{\isachardoublequoteclose}\ \isanewline
\ \ \ \ \isacommand{apply}\isamarkupfalse%
\ {\isacharparenleft}{\kern0pt}simp\ add{\isacharcolon}{\kern0pt}Hoeffding{\isacharunderscore}{\kern0pt}ineq{\isacharunderscore}{\kern0pt}def\ indep{\isacharunderscore}{\kern0pt}interval{\isacharunderscore}{\kern0pt}bounded{\isacharunderscore}{\kern0pt}random{\isacharunderscore}{\kern0pt}variables{\isacharunderscore}{\kern0pt}def{\isacharparenright}{\kern0pt}\isanewline
\ \ \ \ \isacommand{by}\isamarkupfalse%
\ {\isacharparenleft}{\kern0pt}simp\ add{\isacharcolon}{\kern0pt}prob{\isacharunderscore}{\kern0pt}space{\isacharunderscore}{\kern0pt}axioms\ indep{\isacharunderscore}{\kern0pt}interval{\isacharunderscore}{\kern0pt}bounded{\isacharunderscore}{\kern0pt}random{\isacharunderscore}{\kern0pt}variables{\isacharunderscore}{\kern0pt}axioms{\isacharunderscore}{\kern0pt}def\ Y{\isacharunderscore}{\kern0pt}def\ Y{\isacharunderscore}{\kern0pt}indep{\isacharparenright}{\kern0pt}\isanewline
\isanewline
\ \ \isacommand{have}\isamarkupfalse%
\ c{\isacharcolon}{\kern0pt}\ {\isachardoublequoteopen}{\isasymAnd}{\isasymomega}{\isachardot}{\kern0pt}\ {\isacharparenleft}{\kern0pt}{\isasymSum}i\ {\isacharequal}{\kern0pt}\ {\isadigit{0}}{\isachardot}{\kern0pt}{\isachardot}{\kern0pt}{\isacharless}{\kern0pt}n{\isachardot}{\kern0pt}\ Y\ i\ {\isasymomega}{\isacharparenright}{\kern0pt}\ {\isachargreater}{\kern0pt}\ n{\isacharslash}{\kern0pt}{\isadigit{2}}\ {\isasymLongrightarrow}\ median\ n\ {\isacharparenleft}{\kern0pt}{\isasymlambda}i{\isachardot}{\kern0pt}\ X\ i\ {\isasymomega}{\isacharparenright}{\kern0pt}\ {\isasymin}\ I{\isachardoublequoteclose}\isanewline
\ \ \isacommand{proof}\isamarkupfalse%
\ {\isacharminus}{\kern0pt}\isanewline
\ \ \ \ \isacommand{fix}\isamarkupfalse%
\ {\isasymomega}\isanewline
\ \ \ \ \isacommand{assume}\isamarkupfalse%
\ {\isachardoublequoteopen}{\isacharparenleft}{\kern0pt}{\isasymSum}i\ {\isacharequal}{\kern0pt}\ {\isadigit{0}}{\isachardot}{\kern0pt}{\isachardot}{\kern0pt}{\isacharless}{\kern0pt}n{\isachardot}{\kern0pt}\ Y\ i\ {\isasymomega}{\isacharparenright}{\kern0pt}\ {\isachargreater}{\kern0pt}\ n{\isacharslash}{\kern0pt}{\isadigit{2}}{\isachardoublequoteclose}\isanewline
\ \ \ \ \isacommand{hence}\isamarkupfalse%
\ {\isachardoublequoteopen}n\ {\isacharless}{\kern0pt}\ {\isadigit{2}}\ {\isacharasterisk}{\kern0pt}\ card\ {\isacharparenleft}{\kern0pt}{\isacharbraceleft}{\kern0pt}{\isadigit{0}}{\isachardot}{\kern0pt}{\isachardot}{\kern0pt}{\isacharless}{\kern0pt}n{\isacharbraceright}{\kern0pt}\ {\isasyminter}\ {\isacharbraceleft}{\kern0pt}i{\isachardot}{\kern0pt}\ X\ i\ {\isasymomega}\ {\isasymin}\ I{\isacharbraceright}{\kern0pt}{\isacharparenright}{\kern0pt}{\isachardoublequoteclose}\ \isanewline
\ \ \ \ \ \ \isacommand{by}\isamarkupfalse%
\ {\isacharparenleft}{\kern0pt}simp\ add{\isacharcolon}{\kern0pt}Y{\isacharunderscore}{\kern0pt}def\ indicator{\isacharunderscore}{\kern0pt}def{\isacharparenright}{\kern0pt}\ \isanewline
\ \ \ \ \isacommand{also}\isamarkupfalse%
\ \isacommand{have}\isamarkupfalse%
\ {\isachardoublequoteopen}{\isachardot}{\kern0pt}{\isachardot}{\kern0pt}{\isachardot}{\kern0pt}\ {\isacharequal}{\kern0pt}\ {\isadigit{2}}\ {\isacharasterisk}{\kern0pt}\ card\ {\isacharbraceleft}{\kern0pt}i{\isachardot}{\kern0pt}\ i\ {\isacharless}{\kern0pt}\ n\ {\isasymand}\ X\ i\ {\isasymomega}\ {\isasymin}\ I{\isacharbraceright}{\kern0pt}{\isachardoublequoteclose}\isanewline
\ \ \ \ \ \ \isacommand{apply}\isamarkupfalse%
\ {\isacharparenleft}{\kern0pt}simp{\isacharparenright}{\kern0pt}\isanewline
\ \ \ \ \ \ \isacommand{apply}\isamarkupfalse%
\ {\isacharparenleft}{\kern0pt}rule\ arg{\isacharunderscore}{\kern0pt}cong{\isacharbrackleft}{\kern0pt}\isakeyword{where}\ f{\isacharequal}{\kern0pt}{\isachardoublequoteopen}card{\isachardoublequoteclose}{\isacharbrackright}{\kern0pt}{\isacharparenright}{\kern0pt}\isanewline
\ \ \ \ \ \ \isacommand{by}\isamarkupfalse%
\ {\isacharparenleft}{\kern0pt}rule\ order{\isacharunderscore}{\kern0pt}antisym{\isacharcomma}{\kern0pt}\ rule\ subsetI{\isacharcomma}{\kern0pt}\ simp{\isacharcomma}{\kern0pt}\ rule\ subsetI{\isacharcomma}{\kern0pt}\ simp{\isacharparenright}{\kern0pt}\isanewline
\ \ \ \ \isacommand{finally}\isamarkupfalse%
\ \isacommand{have}\isamarkupfalse%
\ {\isachardoublequoteopen}{\isadigit{2}}\ {\isacharasterisk}{\kern0pt}\ card\ {\isacharbraceleft}{\kern0pt}i{\isachardot}{\kern0pt}\ i\ {\isacharless}{\kern0pt}\ n\ {\isasymand}\ X\ i\ {\isasymomega}\ {\isasymin}\ I{\isacharbraceright}{\kern0pt}\ {\isachargreater}{\kern0pt}\ n{\isachardoublequoteclose}\ \isacommand{by}\isamarkupfalse%
\ simp\isanewline
\ \ \ \ \isacommand{thus}\isamarkupfalse%
\ {\isachardoublequoteopen}median\ n\ {\isacharparenleft}{\kern0pt}{\isasymlambda}i{\isachardot}{\kern0pt}\ X\ i\ {\isasymomega}{\isacharparenright}{\kern0pt}\ {\isasymin}\ I{\isachardoublequoteclose}\isanewline
\ \ \ \ \ \ \isacommand{using}\isamarkupfalse%
\ median{\isacharunderscore}{\kern0pt}est{\isacharbrackleft}{\kern0pt}OF\ assms{\isacharparenleft}{\kern0pt}{\isadigit{1}}{\isacharparenright}{\kern0pt}{\isacharbrackright}{\kern0pt}\ \isacommand{by}\isamarkupfalse%
\ simp\isanewline
\ \ \isacommand{qed}\isamarkupfalse%
\isanewline
\isanewline
\ \ \isacommand{have}\isamarkupfalse%
\ {\isachardoublequoteopen}{\isadigit{1}}\ {\isacharminus}{\kern0pt}\ {\isasymepsilon}\ {\isasymle}\ {\isadigit{1}}{\isacharminus}{\kern0pt}\ exp\ {\isacharparenleft}{\kern0pt}{\isacharminus}{\kern0pt}\ {\isacharparenleft}{\kern0pt}{\isadigit{2}}\ {\isacharasterisk}{\kern0pt}\ {\isasymalpha}\isactrlsup {\isadigit{2}}\ {\isacharasterisk}{\kern0pt}\ real\ n{\isacharparenright}{\kern0pt}{\isacharparenright}{\kern0pt}{\isachardoublequoteclose}\ \isanewline
\ \ \ \ \isacommand{apply}\isamarkupfalse%
\ simp\isanewline
\ \ \ \ \isacommand{apply}\isamarkupfalse%
\ {\isacharparenleft}{\kern0pt}subst\ ln{\isacharunderscore}{\kern0pt}ge{\isacharunderscore}{\kern0pt}iff{\isacharbrackleft}{\kern0pt}symmetric{\isacharbrackright}{\kern0pt}{\isacharparenright}{\kern0pt}\isanewline
\ \ \ \ \isacommand{using}\isamarkupfalse%
\ assms{\isacharparenleft}{\kern0pt}{\isadigit{3}}{\isacharparenright}{\kern0pt}\ \isacommand{apply}\isamarkupfalse%
\ simp\isanewline
\ \ \ \ \isacommand{using}\isamarkupfalse%
\ assms{\isacharparenleft}{\kern0pt}{\isadigit{5}}{\isacharparenright}{\kern0pt}\ \isacommand{apply}\isamarkupfalse%
\ {\isacharparenleft}{\kern0pt}subst\ {\isacharparenleft}{\kern0pt}asm{\isacharparenright}{\kern0pt}\ pos{\isacharunderscore}{\kern0pt}divide{\isacharunderscore}{\kern0pt}le{\isacharunderscore}{\kern0pt}eq{\isacharparenright}{\kern0pt}\ \isanewline
\ \ \ \ \ \isacommand{apply}\isamarkupfalse%
\ {\isacharparenleft}{\kern0pt}simp\ add{\isacharcolon}{\kern0pt}\ assms{\isacharparenleft}{\kern0pt}{\isadigit{2}}{\isacharparenright}{\kern0pt}\ power{\isadigit{2}}{\isacharunderscore}{\kern0pt}eq{\isacharunderscore}{\kern0pt}square{\isacharparenright}{\kern0pt}\isanewline
\ \ \ \ \isacommand{by}\isamarkupfalse%
\ {\isacharparenleft}{\kern0pt}simp\ add{\isacharcolon}{\kern0pt}\ mult{\isacharunderscore}{\kern0pt}of{\isacharunderscore}{\kern0pt}nat{\isacharunderscore}{\kern0pt}commute{\isacharparenright}{\kern0pt}\isanewline
\ \ \isacommand{also}\isamarkupfalse%
\ \isacommand{have}\isamarkupfalse%
\ {\isachardoublequoteopen}{\isachardot}{\kern0pt}{\isachardot}{\kern0pt}{\isachardot}{\kern0pt}\ {\isasymle}\ {\isadigit{1}}{\isacharminus}{\kern0pt}\ exp\ {\isacharparenleft}{\kern0pt}{\isacharminus}{\kern0pt}\ {\isacharparenleft}{\kern0pt}{\isadigit{2}}\ {\isacharasterisk}{\kern0pt}\ t\isactrlsup {\isadigit{2}}\ {\isacharslash}{\kern0pt}\ real\ n{\isacharparenright}{\kern0pt}{\isacharparenright}{\kern0pt}{\isachardoublequoteclose}\ \isanewline
\ \ \ \ \isacommand{apply}\isamarkupfalse%
\ simp\isanewline
\ \ \ \ \isacommand{apply}\isamarkupfalse%
\ {\isacharparenleft}{\kern0pt}subst\ pos{\isacharunderscore}{\kern0pt}le{\isacharunderscore}{\kern0pt}divide{\isacharunderscore}{\kern0pt}eq{\isacharparenright}{\kern0pt}\ \isacommand{using}\isamarkupfalse%
\ n{\isacharunderscore}{\kern0pt}ge{\isacharunderscore}{\kern0pt}{\isadigit{0}}\ \isacommand{apply}\isamarkupfalse%
\ simp\isanewline
\ \ \ \ \isacommand{using}\isamarkupfalse%
\ t{\isacharunderscore}{\kern0pt}ge{\isacharunderscore}{\kern0pt}a{\isacharunderscore}{\kern0pt}sq\ \isacommand{by}\isamarkupfalse%
\ linarith\isanewline
\ \ \isacommand{also}\isamarkupfalse%
\ \isacommand{have}\isamarkupfalse%
\ {\isachardoublequoteopen}{\isachardot}{\kern0pt}{\isachardot}{\kern0pt}{\isachardot}{\kern0pt}\ {\isasymle}\ {\isadigit{1}}\ {\isacharminus}{\kern0pt}\ {\isasymP}{\isacharparenleft}{\kern0pt}{\isasymomega}\ in\ M{\isachardot}{\kern0pt}\ {\isacharparenleft}{\kern0pt}{\isasymSum}i\ {\isacharequal}{\kern0pt}\ {\isadigit{0}}{\isachardot}{\kern0pt}{\isachardot}{\kern0pt}{\isacharless}{\kern0pt}n{\isachardot}{\kern0pt}\ Y\ i\ {\isasymomega}{\isacharparenright}{\kern0pt}\ {\isasymle}\ n{\isacharslash}{\kern0pt}{\isadigit{2}}{\isacharparenright}{\kern0pt}{\isachardoublequoteclose}\ \isanewline
\ \ \ \ \isacommand{using}\isamarkupfalse%
\ Hoeffding{\isacharunderscore}{\kern0pt}ineq{\isachardot}{\kern0pt}Hoeffding{\isacharunderscore}{\kern0pt}ineq{\isacharunderscore}{\kern0pt}le{\isacharbrackleft}{\kern0pt}OF\ b{\isacharcomma}{\kern0pt}\ \isakeyword{where}\ {\isasymepsilon}{\isacharequal}{\kern0pt}{\isachardoublequoteopen}t{\isachardoublequoteclose}{\isacharcomma}{\kern0pt}\ simplified{\isacharbrackright}{\kern0pt}\ n{\isacharunderscore}{\kern0pt}ge{\isacharunderscore}{\kern0pt}{\isadigit{0}}\ t{\isacharunderscore}{\kern0pt}ge{\isacharunderscore}{\kern0pt}{\isadigit{0}}\isanewline
\ \ \ \ \isacommand{by}\isamarkupfalse%
\ {\isacharparenleft}{\kern0pt}simp\ add{\isacharcolon}{\kern0pt}t{\isacharunderscore}{\kern0pt}def{\isacharparenright}{\kern0pt}\ \isanewline
\ \ \isacommand{also}\isamarkupfalse%
\ \isacommand{have}\isamarkupfalse%
\ {\isachardoublequoteopen}{\isachardot}{\kern0pt}{\isachardot}{\kern0pt}{\isachardot}{\kern0pt}\ {\isacharequal}{\kern0pt}\ {\isasymP}{\isacharparenleft}{\kern0pt}{\isasymomega}\ in\ M{\isachardot}{\kern0pt}\ {\isacharparenleft}{\kern0pt}{\isasymSum}i\ {\isacharequal}{\kern0pt}\ {\isadigit{0}}{\isachardot}{\kern0pt}{\isachardot}{\kern0pt}{\isacharless}{\kern0pt}n{\isachardot}{\kern0pt}\ Y\ i\ {\isasymomega}{\isacharparenright}{\kern0pt}\ {\isachargreater}{\kern0pt}\ n{\isacharslash}{\kern0pt}{\isadigit{2}}{\isacharparenright}{\kern0pt}{\isachardoublequoteclose}\ \isanewline
\ \ \ \ \isacommand{apply}\isamarkupfalse%
\ {\isacharparenleft}{\kern0pt}subst\ prob{\isacharunderscore}{\kern0pt}compl{\isacharbrackleft}{\kern0pt}symmetric{\isacharbrackright}{\kern0pt}{\isacharparenright}{\kern0pt}\isanewline
\ \ \ \ \ \isacommand{apply}\isamarkupfalse%
\ measurable\isanewline
\ \ \ \ \isacommand{using}\isamarkupfalse%
\ Y{\isacharunderscore}{\kern0pt}indep\ \isacommand{apply}\isamarkupfalse%
\ {\isacharparenleft}{\kern0pt}simp\ add{\isacharcolon}{\kern0pt}indep{\isacharunderscore}{\kern0pt}vars{\isacharunderscore}{\kern0pt}def{\isacharparenright}{\kern0pt}\isanewline
\ \ \ \ \isacommand{apply}\isamarkupfalse%
\ {\isacharparenleft}{\kern0pt}rule\ arg{\isacharunderscore}{\kern0pt}cong{\isadigit{2}}{\isacharbrackleft}{\kern0pt}\isakeyword{where}\ f{\isacharequal}{\kern0pt}{\isachardoublequoteopen}measure{\isachardoublequoteclose}{\isacharbrackright}{\kern0pt}{\isacharcomma}{\kern0pt}\ simp{\isacharparenright}{\kern0pt}\isanewline
\ \ \ \ \isacommand{by}\isamarkupfalse%
\ {\isacharparenleft}{\kern0pt}rule\ order{\isacharunderscore}{\kern0pt}antisym{\isacharcomma}{\kern0pt}\ rule\ subsetI{\isacharcomma}{\kern0pt}\ simp\ add{\isacharcolon}{\kern0pt}not{\isacharunderscore}{\kern0pt}le{\isacharcomma}{\kern0pt}\ rule\ subsetI{\isacharcomma}{\kern0pt}\ simp\ add{\isacharcolon}{\kern0pt}not{\isacharunderscore}{\kern0pt}le{\isacharparenright}{\kern0pt}\isanewline
\ \ \isacommand{also}\isamarkupfalse%
\ \isacommand{have}\isamarkupfalse%
\ {\isachardoublequoteopen}{\isachardot}{\kern0pt}{\isachardot}{\kern0pt}{\isachardot}{\kern0pt}\ {\isasymle}\ {\isasymP}{\isacharparenleft}{\kern0pt}{\isasymomega}\ in\ M{\isachardot}{\kern0pt}\ median\ n\ {\isacharparenleft}{\kern0pt}{\isasymlambda}i{\isachardot}{\kern0pt}\ X\ i\ {\isasymomega}{\isacharparenright}{\kern0pt}\ {\isasymin}\ I{\isacharparenright}{\kern0pt}{\isachardoublequoteclose}\isanewline
\ \ \ \ \isacommand{apply}\isamarkupfalse%
\ {\isacharparenleft}{\kern0pt}rule\ finite{\isacharunderscore}{\kern0pt}measure{\isacharunderscore}{\kern0pt}mono{\isacharparenright}{\kern0pt}\isanewline
\ \ \ \ \ \isacommand{apply}\isamarkupfalse%
\ {\isacharparenleft}{\kern0pt}rule\ subsetI{\isacharparenright}{\kern0pt}\ \isacommand{using}\isamarkupfalse%
\ c\ \isacommand{apply}\isamarkupfalse%
\ simp\ \isanewline
\ \ \ \ \isacommand{using}\isamarkupfalse%
\ interval{\isacharunderscore}{\kern0pt}borel{\isacharbrackleft}{\kern0pt}OF\ assms{\isacharparenleft}{\kern0pt}{\isadigit{1}}{\isacharparenright}{\kern0pt}{\isacharbrackright}{\kern0pt}\ \isacommand{apply}\isamarkupfalse%
\ measurable\isanewline
\ \ \ \ \isacommand{apply}\isamarkupfalse%
\ {\isacharparenleft}{\kern0pt}rule\ median{\isacharunderscore}{\kern0pt}measurable{\isacharbrackleft}{\kern0pt}OF\ n{\isacharunderscore}{\kern0pt}ge{\isacharunderscore}{\kern0pt}{\isadigit{1}}{\isacharbrackright}{\kern0pt}{\isacharparenright}{\kern0pt}\isanewline
\ \ \ \ \isacommand{using}\isamarkupfalse%
\ assms{\isacharparenleft}{\kern0pt}{\isadigit{4}}{\isacharparenright}{\kern0pt}\ \isacommand{by}\isamarkupfalse%
\ {\isacharparenleft}{\kern0pt}simp\ add{\isacharcolon}{\kern0pt}indep{\isacharunderscore}{\kern0pt}vars{\isacharunderscore}{\kern0pt}def{\isacharparenright}{\kern0pt}\isanewline
\ \ \isacommand{finally}\isamarkupfalse%
\ \isacommand{show}\isamarkupfalse%
\ {\isacharquery}{\kern0pt}thesis\ \isacommand{by}\isamarkupfalse%
\ simp\isanewline
\isacommand{qed}\isamarkupfalse%
%
\endisatagproof
{\isafoldproof}%
%
\isadelimproof
\isanewline
%
\endisadelimproof
\isanewline
\isacommand{lemma}\isamarkupfalse%
\ {\isacharparenleft}{\kern0pt}\isakeyword{in}\ prob{\isacharunderscore}{\kern0pt}space{\isacharparenright}{\kern0pt}\ median{\isacharunderscore}{\kern0pt}bound{\isacharunderscore}{\kern0pt}{\isadigit{1}}{\isacharcolon}{\kern0pt}\isanewline
\ \ \isakeyword{fixes}\ a\ b\ {\isacharcolon}{\kern0pt}{\isacharcolon}{\kern0pt}\ real\isanewline
\ \ \isakeyword{fixes}\ n\ {\isacharcolon}{\kern0pt}{\isacharcolon}{\kern0pt}\ nat\isanewline
\ \ \isakeyword{assumes}\ {\isachardoublequoteopen}{\isasymalpha}\ {\isachargreater}{\kern0pt}\ {\isadigit{0}}{\isachardoublequoteclose}\isanewline
\ \ \isakeyword{assumes}\ {\isachardoublequoteopen}{\isasymepsilon}\ {\isasymin}\ {\isacharbraceleft}{\kern0pt}{\isadigit{0}}{\isacharless}{\kern0pt}{\isachardot}{\kern0pt}{\isachardot}{\kern0pt}{\isacharless}{\kern0pt}{\isadigit{1}}{\isacharbraceright}{\kern0pt}{\isachardoublequoteclose}\isanewline
\ \ \isakeyword{assumes}\ {\isachardoublequoteopen}indep{\isacharunderscore}{\kern0pt}vars\ {\isacharparenleft}{\kern0pt}{\isasymlambda}{\isacharunderscore}{\kern0pt}{\isachardot}{\kern0pt}\ borel{\isacharparenright}{\kern0pt}\ X\ {\isacharbraceleft}{\kern0pt}{\isadigit{0}}{\isachardot}{\kern0pt}{\isachardot}{\kern0pt}{\isacharless}{\kern0pt}n{\isacharbraceright}{\kern0pt}{\isachardoublequoteclose}\isanewline
\ \ \isakeyword{assumes}\ {\isachardoublequoteopen}n\ {\isasymge}\ {\isacharminus}{\kern0pt}\ ln\ {\isasymepsilon}\ {\isacharslash}{\kern0pt}\ {\isacharparenleft}{\kern0pt}{\isadigit{2}}\ {\isacharasterisk}{\kern0pt}\ {\isasymalpha}\isactrlsup {\isadigit{2}}{\isacharparenright}{\kern0pt}{\isachardoublequoteclose}\isanewline
\ \ \isakeyword{assumes}\ {\isachardoublequoteopen}{\isasymAnd}i{\isachardot}{\kern0pt}\ i\ {\isacharless}{\kern0pt}\ n\ {\isasymLongrightarrow}\ {\isasymP}{\isacharparenleft}{\kern0pt}{\isasymomega}\ in\ M{\isachardot}{\kern0pt}\ X\ i\ {\isasymomega}\ {\isasymin}\ {\isacharbraceleft}{\kern0pt}a{\isachardot}{\kern0pt}{\isachardot}{\kern0pt}b{\isacharbraceright}{\kern0pt}{\isacharparenright}{\kern0pt}\ {\isasymge}\ {\isadigit{1}}{\isacharslash}{\kern0pt}{\isadigit{2}}{\isacharplus}{\kern0pt}{\isasymalpha}{\isachardoublequoteclose}\ \isanewline
\ \ \isakeyword{shows}\ {\isachardoublequoteopen}{\isasymP}{\isacharparenleft}{\kern0pt}{\isasymomega}\ in\ M{\isachardot}{\kern0pt}\ median\ n\ {\isacharparenleft}{\kern0pt}{\isasymlambda}i{\isachardot}{\kern0pt}\ X\ i\ {\isasymomega}{\isacharparenright}{\kern0pt}\ {\isasymin}\ {\isacharbraceleft}{\kern0pt}a{\isachardot}{\kern0pt}{\isachardot}{\kern0pt}b{\isacharbraceright}{\kern0pt}{\isacharparenright}{\kern0pt}\ {\isasymge}\ {\isadigit{1}}{\isacharminus}{\kern0pt}{\isasymepsilon}{\isachardoublequoteclose}\ {\isacharparenleft}{\kern0pt}\isakeyword{is}\ {\isachardoublequoteopen}{\isasymP}{\isacharparenleft}{\kern0pt}{\isasymomega}\ in\ M{\isachardot}{\kern0pt}\ {\isacharquery}{\kern0pt}lhs\ {\isasymomega}{\isacharparenright}{\kern0pt}\ {\isasymge}\ {\isacharquery}{\kern0pt}C{\isachardoublequoteclose}{\isacharparenright}{\kern0pt}\ \isanewline
%
\isadelimproof
\ \ %
\endisadelimproof
%
\isatagproof
\isacommand{apply}\isamarkupfalse%
\ {\isacharparenleft}{\kern0pt}rule\ median{\isacharunderscore}{\kern0pt}bound{\isacharbrackleft}{\kern0pt}OF\ {\isacharunderscore}{\kern0pt}\ assms{\isacharparenleft}{\kern0pt}{\isadigit{1}}{\isacharparenright}{\kern0pt}\ assms{\isacharparenleft}{\kern0pt}{\isadigit{2}}{\isacharparenright}{\kern0pt}\ assms{\isacharparenleft}{\kern0pt}{\isadigit{3}}{\isacharparenright}{\kern0pt}\ assms{\isacharparenleft}{\kern0pt}{\isadigit{4}}{\isacharparenright}{\kern0pt}\ assms{\isacharparenleft}{\kern0pt}{\isadigit{5}}{\isacharparenright}{\kern0pt}{\isacharbrackright}{\kern0pt}{\isacharparenright}{\kern0pt}\isanewline
\ \ \isacommand{by}\isamarkupfalse%
\ {\isacharparenleft}{\kern0pt}simp\ add{\isacharcolon}{\kern0pt}interval{\isacharunderscore}{\kern0pt}def{\isacharparenright}{\kern0pt}{\isacharplus}{\kern0pt}%
\endisatagproof
{\isafoldproof}%
%
\isadelimproof
\isanewline
%
\endisadelimproof
\isanewline
\isacommand{lemma}\isamarkupfalse%
\ {\isacharparenleft}{\kern0pt}\isakeyword{in}\ prob{\isacharunderscore}{\kern0pt}space{\isacharparenright}{\kern0pt}\ median{\isacharunderscore}{\kern0pt}bound{\isacharunderscore}{\kern0pt}{\isadigit{2}}{\isacharcolon}{\kern0pt}\isanewline
\ \ \isakeyword{fixes}\ {\isasymmu}\ {\isacharcolon}{\kern0pt}{\isacharcolon}{\kern0pt}\ real\isanewline
\ \ \isakeyword{fixes}\ {\isasymdelta}\ {\isacharcolon}{\kern0pt}{\isacharcolon}{\kern0pt}\ real\isanewline
\ \ \isakeyword{assumes}\ {\isachardoublequoteopen}{\isasymepsilon}\ {\isasymin}\ {\isacharbraceleft}{\kern0pt}{\isadigit{0}}{\isacharless}{\kern0pt}{\isachardot}{\kern0pt}{\isachardot}{\kern0pt}{\isacharless}{\kern0pt}{\isadigit{1}}{\isacharbraceright}{\kern0pt}{\isachardoublequoteclose}\isanewline
\ \ \isakeyword{assumes}\ {\isachardoublequoteopen}indep{\isacharunderscore}{\kern0pt}vars\ {\isacharparenleft}{\kern0pt}{\isasymlambda}{\isacharunderscore}{\kern0pt}{\isachardot}{\kern0pt}\ borel{\isacharparenright}{\kern0pt}\ X\ {\isacharbraceleft}{\kern0pt}{\isadigit{0}}{\isachardot}{\kern0pt}{\isachardot}{\kern0pt}{\isacharless}{\kern0pt}n{\isacharbraceright}{\kern0pt}{\isachardoublequoteclose}\isanewline
\ \ \isakeyword{assumes}\ {\isachardoublequoteopen}n\ {\isasymge}\ {\isacharminus}{\kern0pt}{\isadigit{1}}{\isadigit{8}}\ {\isacharasterisk}{\kern0pt}\ ln\ {\isasymepsilon}{\isachardoublequoteclose}\isanewline
\ \ \isakeyword{assumes}\ {\isachardoublequoteopen}{\isasymAnd}i{\isachardot}{\kern0pt}\ i\ {\isacharless}{\kern0pt}\ n\ {\isasymLongrightarrow}\ {\isasymP}{\isacharparenleft}{\kern0pt}{\isasymomega}\ in\ M{\isachardot}{\kern0pt}\ abs\ {\isacharparenleft}{\kern0pt}X\ i\ {\isasymomega}\ {\isacharminus}{\kern0pt}\ {\isasymmu}{\isacharparenright}{\kern0pt}\ {\isachargreater}{\kern0pt}\ {\isasymdelta}{\isacharparenright}{\kern0pt}\ {\isasymle}\ {\isadigit{1}}{\isacharslash}{\kern0pt}{\isadigit{3}}{\isachardoublequoteclose}\ \isanewline
\ \ \isakeyword{shows}\ {\isachardoublequoteopen}{\isasymP}{\isacharparenleft}{\kern0pt}{\isasymomega}\ in\ M{\isachardot}{\kern0pt}\ abs\ {\isacharparenleft}{\kern0pt}median\ n\ {\isacharparenleft}{\kern0pt}{\isasymlambda}i{\isachardot}{\kern0pt}\ X\ i\ {\isasymomega}{\isacharparenright}{\kern0pt}\ {\isacharminus}{\kern0pt}\ {\isasymmu}{\isacharparenright}{\kern0pt}\ {\isasymle}\ {\isasymdelta}{\isacharparenright}{\kern0pt}\ {\isasymge}\ {\isadigit{1}}{\isacharminus}{\kern0pt}{\isasymepsilon}{\isachardoublequoteclose}\isanewline
%
\isadelimproof
%
\endisadelimproof
%
\isatagproof
\isacommand{proof}\isamarkupfalse%
\ {\isacharminus}{\kern0pt}\isanewline
\ \ \isacommand{have}\isamarkupfalse%
\ b{\isacharcolon}{\kern0pt}{\isachardoublequoteopen}{\isasymAnd}i{\isachardot}{\kern0pt}\ i\ {\isacharless}{\kern0pt}\ n\ {\isasymLongrightarrow}\ space\ M\ {\isacharminus}{\kern0pt}\ {\isacharbraceleft}{\kern0pt}{\isasymomega}\ {\isasymin}\ space\ M{\isachardot}{\kern0pt}\ X\ i\ {\isasymomega}\ {\isasymin}\ {\isacharbraceleft}{\kern0pt}{\isasymmu}\ {\isacharminus}{\kern0pt}\ {\isasymdelta}{\isachardot}{\kern0pt}{\isachardot}{\kern0pt}{\isasymmu}\ {\isacharplus}{\kern0pt}\ {\isasymdelta}{\isacharbraceright}{\kern0pt}{\isacharbraceright}{\kern0pt}\ {\isacharequal}{\kern0pt}\ \ {\isacharbraceleft}{\kern0pt}{\isasymomega}\ {\isasymin}\ space\ M{\isachardot}{\kern0pt}\ abs\ {\isacharparenleft}{\kern0pt}X\ i\ {\isasymomega}\ {\isacharminus}{\kern0pt}\ {\isasymmu}{\isacharparenright}{\kern0pt}\ {\isachargreater}{\kern0pt}\ {\isasymdelta}{\isacharbraceright}{\kern0pt}{\isachardoublequoteclose}\isanewline
\ \ \ \ \isacommand{apply}\isamarkupfalse%
\ {\isacharparenleft}{\kern0pt}rule\ order{\isacharunderscore}{\kern0pt}antisym{\isacharparenright}{\kern0pt}\isanewline
\ \ \ \ \isacommand{apply}\isamarkupfalse%
\ {\isacharparenleft}{\kern0pt}rule\ subsetI{\isacharcomma}{\kern0pt}\ simp{\isacharcomma}{\kern0pt}\ linarith{\isacharparenright}{\kern0pt}\isanewline
\ \ \ \ \isacommand{by}\isamarkupfalse%
\ {\isacharparenleft}{\kern0pt}rule\ subsetI{\isacharcomma}{\kern0pt}\ simp{\isacharcomma}{\kern0pt}\ linarith{\isacharparenright}{\kern0pt}\isanewline
\isanewline
\ \ \isacommand{have}\isamarkupfalse%
\ {\isachardoublequoteopen}{\isasymAnd}i{\isachardot}{\kern0pt}\ i\ {\isacharless}{\kern0pt}\ n\ {\isasymLongrightarrow}\ {\isadigit{1}}\ {\isacharminus}{\kern0pt}\ {\isasymP}{\isacharparenleft}{\kern0pt}{\isasymomega}\ in\ M{\isachardot}{\kern0pt}\ X\ i\ {\isasymomega}\ {\isasymin}\ {\isacharbraceleft}{\kern0pt}{\isasymmu}{\isacharminus}{\kern0pt}\ {\isasymdelta}{\isachardot}{\kern0pt}{\isachardot}{\kern0pt}{\isasymmu}{\isacharplus}{\kern0pt}{\isasymdelta}{\isacharbraceright}{\kern0pt}{\isacharparenright}{\kern0pt}\ {\isasymle}\ {\isadigit{1}}{\isacharslash}{\kern0pt}{\isadigit{3}}{\isachardoublequoteclose}\isanewline
\ \ \ \ \isacommand{apply}\isamarkupfalse%
\ {\isacharparenleft}{\kern0pt}subst\ prob{\isacharunderscore}{\kern0pt}compl{\isacharbrackleft}{\kern0pt}symmetric{\isacharbrackright}{\kern0pt}{\isacharparenright}{\kern0pt}\isanewline
\ \ \ \ \ \isacommand{apply}\isamarkupfalse%
\ {\isacharparenleft}{\kern0pt}measurable{\isacharparenright}{\kern0pt}\isanewline
\ \ \ \ \isacommand{using}\isamarkupfalse%
\ assms{\isacharparenleft}{\kern0pt}{\isadigit{2}}{\isacharparenright}{\kern0pt}\ \isacommand{apply}\isamarkupfalse%
\ {\isacharparenleft}{\kern0pt}simp\ add{\isacharcolon}{\kern0pt}indep{\isacharunderscore}{\kern0pt}vars{\isacharunderscore}{\kern0pt}def{\isacharparenright}{\kern0pt}\isanewline
\ \ \ \ \isacommand{apply}\isamarkupfalse%
\ {\isacharparenleft}{\kern0pt}subst\ b{\isacharcomma}{\kern0pt}\ simp{\isacharparenright}{\kern0pt}\isanewline
\ \ \ \ \isacommand{using}\isamarkupfalse%
\ assms{\isacharparenleft}{\kern0pt}{\isadigit{4}}{\isacharparenright}{\kern0pt}\ \isacommand{by}\isamarkupfalse%
\ simp\isanewline
\isanewline
\ \ \isacommand{hence}\isamarkupfalse%
\ a{\isacharcolon}{\kern0pt}{\isachardoublequoteopen}{\isasymAnd}i{\isachardot}{\kern0pt}\ i\ {\isacharless}{\kern0pt}\ n\ {\isasymLongrightarrow}\ {\isasymP}{\isacharparenleft}{\kern0pt}{\isasymomega}\ in\ M{\isachardot}{\kern0pt}\ X\ i\ {\isasymomega}\ {\isasymin}\ {\isacharbraceleft}{\kern0pt}{\isasymmu}{\isacharminus}{\kern0pt}\ {\isasymdelta}{\isachardot}{\kern0pt}{\isachardot}{\kern0pt}{\isasymmu}{\isacharplus}{\kern0pt}{\isasymdelta}{\isacharbraceright}{\kern0pt}{\isacharparenright}{\kern0pt}\ {\isasymge}\ {\isadigit{2}}{\isacharslash}{\kern0pt}{\isadigit{3}}{\isachardoublequoteclose}\ \isacommand{by}\isamarkupfalse%
\ simp\isanewline
\ \ \isanewline
\ \ \isacommand{have}\isamarkupfalse%
\ {\isachardoublequoteopen}{\isadigit{1}}{\isacharminus}{\kern0pt}{\isasymepsilon}\ {\isasymle}\ {\isasymP}{\isacharparenleft}{\kern0pt}{\isasymomega}\ in\ M{\isachardot}{\kern0pt}\ median\ n\ {\isacharparenleft}{\kern0pt}{\isasymlambda}i{\isachardot}{\kern0pt}\ X\ i\ {\isasymomega}{\isacharparenright}{\kern0pt}\ {\isasymin}\ {\isacharbraceleft}{\kern0pt}{\isasymmu}{\isacharminus}{\kern0pt}{\isasymdelta}{\isachardot}{\kern0pt}{\isachardot}{\kern0pt}{\isasymmu}{\isacharplus}{\kern0pt}{\isasymdelta}{\isacharbraceright}{\kern0pt}{\isacharparenright}{\kern0pt}{\isachardoublequoteclose}\isanewline
\ \ \ \ \isacommand{apply}\isamarkupfalse%
\ {\isacharparenleft}{\kern0pt}rule\ median{\isacharunderscore}{\kern0pt}bound{\isacharunderscore}{\kern0pt}{\isadigit{1}}{\isacharbrackleft}{\kern0pt}OF\ {\isacharunderscore}{\kern0pt}\ assms{\isacharparenleft}{\kern0pt}{\isadigit{1}}{\isacharparenright}{\kern0pt}\ assms{\isacharparenleft}{\kern0pt}{\isadigit{2}}{\isacharparenright}{\kern0pt}{\isacharcomma}{\kern0pt}\ \isakeyword{where}\ {\isasymalpha}{\isacharequal}{\kern0pt}{\isachardoublequoteopen}{\isadigit{1}}{\isacharslash}{\kern0pt}{\isadigit{6}}{\isachardoublequoteclose}{\isacharbrackright}{\kern0pt}{\isacharcomma}{\kern0pt}\ simp{\isacharparenright}{\kern0pt}\ \isanewline
\ \ \ \ \ \isacommand{apply}\isamarkupfalse%
\ {\isacharparenleft}{\kern0pt}simp\ add{\isacharcolon}{\kern0pt}power{\isadigit{2}}{\isacharunderscore}{\kern0pt}eq{\isacharunderscore}{\kern0pt}square{\isacharparenright}{\kern0pt}\isanewline
\ \ \ \ \isacommand{using}\isamarkupfalse%
\ assms{\isacharparenleft}{\kern0pt}{\isadigit{3}}{\isacharparenright}{\kern0pt}\ \isacommand{apply}\isamarkupfalse%
\ simp\isanewline
\ \ \ \ \isacommand{using}\isamarkupfalse%
\ a\ \isacommand{by}\isamarkupfalse%
\ simp\isanewline
\ \ \isacommand{also}\isamarkupfalse%
\ \isacommand{have}\isamarkupfalse%
\ {\isachardoublequoteopen}{\isachardot}{\kern0pt}{\isachardot}{\kern0pt}{\isachardot}{\kern0pt}\ {\isacharequal}{\kern0pt}\ {\isasymP}{\isacharparenleft}{\kern0pt}{\isasymomega}\ in\ M{\isachardot}{\kern0pt}\ abs\ {\isacharparenleft}{\kern0pt}median\ n\ {\isacharparenleft}{\kern0pt}{\isasymlambda}i{\isachardot}{\kern0pt}\ X\ i\ {\isasymomega}{\isacharparenright}{\kern0pt}\ {\isacharminus}{\kern0pt}\ {\isasymmu}{\isacharparenright}{\kern0pt}\ {\isasymle}\ {\isasymdelta}{\isacharparenright}{\kern0pt}{\isachardoublequoteclose}\isanewline
\ \ \ \ \isacommand{apply}\isamarkupfalse%
\ {\isacharparenleft}{\kern0pt}rule\ arg{\isacharunderscore}{\kern0pt}cong{\isadigit{2}}{\isacharbrackleft}{\kern0pt}\isakeyword{where}\ f{\isacharequal}{\kern0pt}{\isachardoublequoteopen}measure{\isachardoublequoteclose}{\isacharbrackright}{\kern0pt}{\isacharcomma}{\kern0pt}\ simp{\isacharparenright}{\kern0pt}\isanewline
\ \ \ \ \isacommand{apply}\isamarkupfalse%
\ {\isacharparenleft}{\kern0pt}rule\ order{\isacharunderscore}{\kern0pt}antisym{\isacharparenright}{\kern0pt}\isanewline
\ \ \ \ \isacommand{apply}\isamarkupfalse%
\ {\isacharparenleft}{\kern0pt}rule\ subsetI{\isacharcomma}{\kern0pt}\ simp{\isacharcomma}{\kern0pt}\ linarith{\isacharparenright}{\kern0pt}\isanewline
\ \ \ \ \isacommand{by}\isamarkupfalse%
\ {\isacharparenleft}{\kern0pt}rule\ subsetI{\isacharcomma}{\kern0pt}\ simp{\isacharcomma}{\kern0pt}\ linarith{\isacharparenright}{\kern0pt}\isanewline
\ \ \isacommand{finally}\isamarkupfalse%
\ \isacommand{show}\isamarkupfalse%
\ {\isacharquery}{\kern0pt}thesis\ \isacommand{by}\isamarkupfalse%
\ simp\isanewline
\isacommand{qed}\isamarkupfalse%
%
\endisatagproof
{\isafoldproof}%
%
\isadelimproof
\isanewline
%
\endisadelimproof
\isanewline
\isacommand{lemma}\isamarkupfalse%
\ sorted{\isacharunderscore}{\kern0pt}mono{\isacharunderscore}{\kern0pt}map{\isacharcolon}{\kern0pt}\ \isanewline
\ \ \isakeyword{assumes}\ {\isachardoublequoteopen}sorted\ xs{\isachardoublequoteclose}\isanewline
\ \ \isakeyword{assumes}\ {\isachardoublequoteopen}mono\ f{\isachardoublequoteclose}\isanewline
\ \ \isakeyword{shows}\ {\isachardoublequoteopen}sorted\ {\isacharparenleft}{\kern0pt}map\ f\ xs{\isacharparenright}{\kern0pt}{\isachardoublequoteclose}\isanewline
%
\isadelimproof
\ \ %
\endisadelimproof
%
\isatagproof
\isacommand{using}\isamarkupfalse%
\ assms\ \isacommand{apply}\isamarkupfalse%
\ {\isacharparenleft}{\kern0pt}simp\ add{\isacharcolon}{\kern0pt}sorted{\isacharunderscore}{\kern0pt}wrt{\isacharunderscore}{\kern0pt}map{\isacharparenright}{\kern0pt}\isanewline
\ \ \isacommand{apply}\isamarkupfalse%
\ {\isacharparenleft}{\kern0pt}rule\ sorted{\isacharunderscore}{\kern0pt}wrt{\isacharunderscore}{\kern0pt}mono{\isacharunderscore}{\kern0pt}rel{\isacharbrackleft}{\kern0pt}\isakeyword{where}\ P{\isacharequal}{\kern0pt}{\isachardoublequoteopen}{\isacharparenleft}{\kern0pt}{\isasymle}{\isacharparenright}{\kern0pt}{\isachardoublequoteclose}{\isacharbrackright}{\kern0pt}{\isacharparenright}{\kern0pt}\isanewline
\ \ \isacommand{by}\isamarkupfalse%
\ {\isacharparenleft}{\kern0pt}simp\ add{\isacharcolon}{\kern0pt}mono{\isacharunderscore}{\kern0pt}def{\isacharcomma}{\kern0pt}\ simp{\isacharparenright}{\kern0pt}%
\endisatagproof
{\isafoldproof}%
%
\isadelimproof
\isanewline
%
\endisadelimproof
\isanewline
\isacommand{lemma}\isamarkupfalse%
\ map{\isacharunderscore}{\kern0pt}sort{\isacharcolon}{\kern0pt}\isanewline
\ \ \isakeyword{assumes}\ {\isachardoublequoteopen}mono\ f{\isachardoublequoteclose}\isanewline
\ \ \isakeyword{shows}\ {\isachardoublequoteopen}sort\ {\isacharparenleft}{\kern0pt}map\ f\ xs{\isacharparenright}{\kern0pt}\ {\isacharequal}{\kern0pt}\ map\ f\ {\isacharparenleft}{\kern0pt}sort\ xs{\isacharparenright}{\kern0pt}{\isachardoublequoteclose}\isanewline
%
\isadelimproof
\ \ %
\endisadelimproof
%
\isatagproof
\isacommand{apply}\isamarkupfalse%
\ {\isacharparenleft}{\kern0pt}rule\ properties{\isacharunderscore}{\kern0pt}for{\isacharunderscore}{\kern0pt}sort{\isacharparenright}{\kern0pt}\isanewline
\ \ \ \isacommand{apply}\isamarkupfalse%
\ simp\isanewline
\ \ \isacommand{by}\isamarkupfalse%
\ {\isacharparenleft}{\kern0pt}rule\ sorted{\isacharunderscore}{\kern0pt}mono{\isacharunderscore}{\kern0pt}map{\isacharcomma}{\kern0pt}\ simp{\isacharcomma}{\kern0pt}\ simp\ add{\isacharcolon}{\kern0pt}assms{\isacharparenright}{\kern0pt}%
\endisatagproof
{\isafoldproof}%
%
\isadelimproof
\isanewline
%
\endisadelimproof
\isanewline
\isacommand{lemma}\isamarkupfalse%
\ median{\isacharunderscore}{\kern0pt}cong{\isacharcolon}{\kern0pt}\isanewline
\ \ \isakeyword{assumes}\ {\isachardoublequoteopen}{\isasymAnd}i{\isachardot}{\kern0pt}\ i\ {\isacharless}{\kern0pt}\ n\ {\isasymLongrightarrow}\ f\ i\ {\isacharequal}{\kern0pt}\ g\ i{\isachardoublequoteclose}\isanewline
\ \ \isakeyword{shows}\ {\isachardoublequoteopen}median\ n\ f\ {\isacharequal}{\kern0pt}\ median\ n\ g{\isachardoublequoteclose}\isanewline
%
\isadelimproof
\ \ %
\endisadelimproof
%
\isatagproof
\isacommand{apply}\isamarkupfalse%
\ {\isacharparenleft}{\kern0pt}cases\ {\isachardoublequoteopen}n\ {\isacharequal}{\kern0pt}\ {\isadigit{0}}{\isachardoublequoteclose}{\isacharcomma}{\kern0pt}\ simp\ add{\isacharcolon}{\kern0pt}median{\isacharunderscore}{\kern0pt}def{\isacharparenright}{\kern0pt}\isanewline
\ \ \isacommand{apply}\isamarkupfalse%
\ {\isacharparenleft}{\kern0pt}simp\ add{\isacharcolon}{\kern0pt}median{\isacharunderscore}{\kern0pt}def{\isacharparenright}{\kern0pt}\isanewline
\ \ \isacommand{apply}\isamarkupfalse%
\ {\isacharparenleft}{\kern0pt}rule\ arg{\isacharunderscore}{\kern0pt}cong{\isadigit{2}}{\isacharbrackleft}{\kern0pt}\isakeyword{where}\ f{\isacharequal}{\kern0pt}{\isachardoublequoteopen}{\isacharparenleft}{\kern0pt}{\isacharbang}{\kern0pt}{\isacharparenright}{\kern0pt}{\isachardoublequoteclose}{\isacharbrackright}{\kern0pt}{\isacharparenright}{\kern0pt}\isanewline
\ \ \ \isacommand{apply}\isamarkupfalse%
\ {\isacharparenleft}{\kern0pt}rule\ arg{\isacharunderscore}{\kern0pt}cong{\isacharbrackleft}{\kern0pt}\isakeyword{where}\ f{\isacharequal}{\kern0pt}{\isachardoublequoteopen}sort{\isachardoublequoteclose}{\isacharbrackright}{\kern0pt}{\isacharparenright}{\kern0pt}\isanewline
\ \ \isacommand{by}\isamarkupfalse%
\ {\isacharparenleft}{\kern0pt}rule\ map{\isacharunderscore}{\kern0pt}cong{\isacharcomma}{\kern0pt}\ simp{\isacharcomma}{\kern0pt}\ simp\ add{\isacharcolon}{\kern0pt}assms{\isacharcomma}{\kern0pt}\ simp{\isacharparenright}{\kern0pt}%
\endisatagproof
{\isafoldproof}%
%
\isadelimproof
\isanewline
%
\endisadelimproof
\isanewline
\isacommand{lemma}\isamarkupfalse%
\ median{\isacharunderscore}{\kern0pt}restrict{\isacharcolon}{\kern0pt}\ \isanewline
\ \ \isakeyword{assumes}\ {\isachardoublequoteopen}n\ {\isachargreater}{\kern0pt}\ {\isadigit{0}}{\isachardoublequoteclose}\isanewline
\ \ \isakeyword{shows}\ {\isachardoublequoteopen}median\ n\ {\isacharparenleft}{\kern0pt}{\isasymlambda}i\ {\isasymin}\ {\isacharbraceleft}{\kern0pt}{\isadigit{0}}{\isachardot}{\kern0pt}{\isachardot}{\kern0pt}{\isacharless}{\kern0pt}n{\isacharbraceright}{\kern0pt}{\isachardot}{\kern0pt}f\ i{\isacharparenright}{\kern0pt}\ {\isacharequal}{\kern0pt}\ median\ n\ f{\isachardoublequoteclose}\isanewline
%
\isadelimproof
\ \ %
\endisadelimproof
%
\isatagproof
\isacommand{by}\isamarkupfalse%
\ {\isacharparenleft}{\kern0pt}rule\ median{\isacharunderscore}{\kern0pt}cong{\isacharcomma}{\kern0pt}\ simp{\isacharparenright}{\kern0pt}%
\endisatagproof
{\isafoldproof}%
%
\isadelimproof
\isanewline
%
\endisadelimproof
\isanewline
\isacommand{lemma}\isamarkupfalse%
\ median{\isacharunderscore}{\kern0pt}rat{\isacharcolon}{\kern0pt}\isanewline
\ \ \isakeyword{assumes}\ {\isachardoublequoteopen}n\ {\isachargreater}{\kern0pt}\ {\isadigit{0}}{\isachardoublequoteclose}\isanewline
\ \ \isakeyword{shows}\ {\isachardoublequoteopen}real{\isacharunderscore}{\kern0pt}of{\isacharunderscore}{\kern0pt}rat\ {\isacharparenleft}{\kern0pt}median\ n\ f{\isacharparenright}{\kern0pt}\ {\isacharequal}{\kern0pt}\ median\ n\ {\isacharparenleft}{\kern0pt}{\isasymlambda}i{\isachardot}{\kern0pt}\ real{\isacharunderscore}{\kern0pt}of{\isacharunderscore}{\kern0pt}rat\ {\isacharparenleft}{\kern0pt}f\ i{\isacharparenright}{\kern0pt}{\isacharparenright}{\kern0pt}{\isachardoublequoteclose}\isanewline
%
\isadelimproof
%
\endisadelimproof
%
\isatagproof
\isacommand{proof}\isamarkupfalse%
\ {\isacharminus}{\kern0pt}\isanewline
\ \ \isacommand{have}\isamarkupfalse%
\ a{\isacharcolon}{\kern0pt}{\isachardoublequoteopen}map\ {\isacharparenleft}{\kern0pt}{\isasymlambda}i{\isachardot}{\kern0pt}\ real{\isacharunderscore}{\kern0pt}of{\isacharunderscore}{\kern0pt}rat\ {\isacharparenleft}{\kern0pt}f\ i{\isacharparenright}{\kern0pt}{\isacharparenright}{\kern0pt}\ {\isacharbrackleft}{\kern0pt}{\isadigit{0}}{\isachardot}{\kern0pt}{\isachardot}{\kern0pt}{\isacharless}{\kern0pt}n{\isacharbrackright}{\kern0pt}\ {\isacharequal}{\kern0pt}\ \isanewline
\ \ \ \ map\ real{\isacharunderscore}{\kern0pt}of{\isacharunderscore}{\kern0pt}rat\ {\isacharparenleft}{\kern0pt}map\ {\isacharparenleft}{\kern0pt}{\isasymlambda}i{\isachardot}{\kern0pt}\ f\ i{\isacharparenright}{\kern0pt}\ {\isacharbrackleft}{\kern0pt}{\isadigit{0}}{\isachardot}{\kern0pt}{\isachardot}{\kern0pt}{\isacharless}{\kern0pt}n{\isacharbrackright}{\kern0pt}{\isacharparenright}{\kern0pt}{\isachardoublequoteclose}\isanewline
\ \ \ \ \isacommand{by}\isamarkupfalse%
\ {\isacharparenleft}{\kern0pt}simp{\isacharparenright}{\kern0pt}\isanewline
\ \ \isacommand{show}\isamarkupfalse%
\ {\isacharquery}{\kern0pt}thesis\ \isanewline
\ \ \ \ \isacommand{apply}\isamarkupfalse%
\ {\isacharparenleft}{\kern0pt}simp\ add{\isacharcolon}{\kern0pt}a\ median{\isacharunderscore}{\kern0pt}def\ del{\isacharcolon}{\kern0pt}map{\isacharunderscore}{\kern0pt}map{\isacharparenright}{\kern0pt}\isanewline
\ \ \ \ \isacommand{apply}\isamarkupfalse%
\ {\isacharparenleft}{\kern0pt}subst\ map{\isacharunderscore}{\kern0pt}sort{\isacharbrackleft}{\kern0pt}\isakeyword{where}\ f{\isacharequal}{\kern0pt}{\isachardoublequoteopen}real{\isacharunderscore}{\kern0pt}of{\isacharunderscore}{\kern0pt}rat{\isachardoublequoteclose}{\isacharbrackright}{\kern0pt}{\isacharcomma}{\kern0pt}\ simp\ add{\isacharcolon}{\kern0pt}mono{\isacharunderscore}{\kern0pt}def\ of{\isacharunderscore}{\kern0pt}rat{\isacharunderscore}{\kern0pt}less{\isacharunderscore}{\kern0pt}eq{\isacharparenright}{\kern0pt}\isanewline
\ \ \ \ \isacommand{apply}\isamarkupfalse%
\ {\isacharparenleft}{\kern0pt}subst\ nth{\isacharunderscore}{\kern0pt}map{\isacharbrackleft}{\kern0pt}\isakeyword{where}\ f{\isacharequal}{\kern0pt}{\isachardoublequoteopen}real{\isacharunderscore}{\kern0pt}of{\isacharunderscore}{\kern0pt}rat{\isachardoublequoteclose}{\isacharbrackright}{\kern0pt}{\isacharparenright}{\kern0pt}\ \isacommand{using}\isamarkupfalse%
\ assms\ \isanewline
\ \ \ \ \isacommand{apply}\isamarkupfalse%
\ fastforce\isanewline
\ \ \ \ \isacommand{by}\isamarkupfalse%
\ simp\isanewline
\isacommand{qed}\isamarkupfalse%
%
\endisatagproof
{\isafoldproof}%
%
\isadelimproof
\isanewline
%
\endisadelimproof
\isanewline
\isacommand{lemma}\isamarkupfalse%
\ median{\isacharunderscore}{\kern0pt}const{\isacharcolon}{\kern0pt}\isanewline
\ \ \isakeyword{assumes}\ {\isachardoublequoteopen}k\ {\isachargreater}{\kern0pt}\ {\isadigit{0}}{\isachardoublequoteclose}\isanewline
\ \ \isakeyword{shows}\ {\isachardoublequoteopen}median\ k\ {\isacharparenleft}{\kern0pt}{\isasymlambda}i\ {\isasymin}\ {\isacharbraceleft}{\kern0pt}{\isadigit{0}}{\isachardot}{\kern0pt}{\isachardot}{\kern0pt}{\isacharless}{\kern0pt}k{\isacharbraceright}{\kern0pt}{\isachardot}{\kern0pt}\ a{\isacharparenright}{\kern0pt}\ {\isacharequal}{\kern0pt}\ a{\isachardoublequoteclose}\isanewline
%
\isadelimproof
%
\endisadelimproof
%
\isatagproof
\isacommand{proof}\isamarkupfalse%
\ {\isacharminus}{\kern0pt}\isanewline
\ \ \isacommand{have}\isamarkupfalse%
\ b{\isacharcolon}{\kern0pt}\ {\isachardoublequoteopen}sorted\ {\isacharparenleft}{\kern0pt}map\ {\isacharparenleft}{\kern0pt}{\isasymlambda}{\isacharunderscore}{\kern0pt}{\isachardot}{\kern0pt}\ a{\isacharparenright}{\kern0pt}\ {\isacharbrackleft}{\kern0pt}{\isadigit{0}}{\isachardot}{\kern0pt}{\isachardot}{\kern0pt}{\isacharless}{\kern0pt}k{\isacharbrackright}{\kern0pt}{\isacharparenright}{\kern0pt}{\isachardoublequoteclose}\ \isanewline
\ \ \ \ \isacommand{by}\isamarkupfalse%
\ {\isacharparenleft}{\kern0pt}subst\ sorted{\isacharunderscore}{\kern0pt}wrt{\isacharunderscore}{\kern0pt}map{\isacharcomma}{\kern0pt}\ simp{\isacharparenright}{\kern0pt}\isanewline
\ \ \isacommand{have}\isamarkupfalse%
\ a{\isacharcolon}{\kern0pt}\ {\isachardoublequoteopen}sort\ {\isacharparenleft}{\kern0pt}map\ {\isacharparenleft}{\kern0pt}{\isasymlambda}{\isacharunderscore}{\kern0pt}{\isachardot}{\kern0pt}\ a{\isacharparenright}{\kern0pt}\ {\isacharbrackleft}{\kern0pt}{\isadigit{0}}{\isachardot}{\kern0pt}{\isachardot}{\kern0pt}{\isacharless}{\kern0pt}k{\isacharbrackright}{\kern0pt}{\isacharparenright}{\kern0pt}\ {\isacharequal}{\kern0pt}\ map\ {\isacharparenleft}{\kern0pt}{\isasymlambda}{\isacharunderscore}{\kern0pt}{\isachardot}{\kern0pt}\ a{\isacharparenright}{\kern0pt}\ {\isacharbrackleft}{\kern0pt}{\isadigit{0}}{\isachardot}{\kern0pt}{\isachardot}{\kern0pt}{\isacharless}{\kern0pt}k{\isacharbrackright}{\kern0pt}{\isachardoublequoteclose}\isanewline
\ \ \ \ \isacommand{by}\isamarkupfalse%
\ {\isacharparenleft}{\kern0pt}subst\ sorted{\isacharunderscore}{\kern0pt}sort{\isacharunderscore}{\kern0pt}id{\isacharbrackleft}{\kern0pt}OF\ b{\isacharbrackright}{\kern0pt}{\isacharcomma}{\kern0pt}\ simp{\isacharparenright}{\kern0pt}\isanewline
\ \ \isacommand{have}\isamarkupfalse%
\ {\isachardoublequoteopen}median\ k\ {\isacharparenleft}{\kern0pt}{\isasymlambda}i\ {\isasymin}\ {\isacharbraceleft}{\kern0pt}{\isadigit{0}}{\isachardot}{\kern0pt}{\isachardot}{\kern0pt}{\isacharless}{\kern0pt}k{\isacharbraceright}{\kern0pt}{\isachardot}{\kern0pt}\ a{\isacharparenright}{\kern0pt}\ {\isacharequal}{\kern0pt}\ median\ k\ {\isacharparenleft}{\kern0pt}{\isasymlambda}{\isacharunderscore}{\kern0pt}{\isachardot}{\kern0pt}\ a{\isacharparenright}{\kern0pt}{\isachardoublequoteclose}\isanewline
\ \ \ \ \isacommand{by}\isamarkupfalse%
\ {\isacharparenleft}{\kern0pt}subst\ median{\isacharunderscore}{\kern0pt}restrict{\isacharbrackleft}{\kern0pt}OF\ assms{\isacharparenleft}{\kern0pt}{\isadigit{1}}{\isacharparenright}{\kern0pt}{\isacharbrackright}{\kern0pt}{\isacharcomma}{\kern0pt}\ simp{\isacharparenright}{\kern0pt}\isanewline
\ \ \isacommand{also}\isamarkupfalse%
\ \isacommand{have}\isamarkupfalse%
\ {\isachardoublequoteopen}{\isachardot}{\kern0pt}{\isachardot}{\kern0pt}{\isachardot}{\kern0pt}\ {\isacharequal}{\kern0pt}\ a{\isachardoublequoteclose}\isanewline
\ \ \ \ \isacommand{apply}\isamarkupfalse%
\ {\isacharparenleft}{\kern0pt}simp\ add{\isacharcolon}{\kern0pt}median{\isacharunderscore}{\kern0pt}def\ a{\isacharparenright}{\kern0pt}\isanewline
\ \ \ \ \isacommand{apply}\isamarkupfalse%
\ {\isacharparenleft}{\kern0pt}subst\ nth{\isacharunderscore}{\kern0pt}map{\isacharparenright}{\kern0pt}\isanewline
\ \ \ \ \isacommand{using}\isamarkupfalse%
\ assms\ \isacommand{by}\isamarkupfalse%
\ simp{\isacharplus}{\kern0pt}\isanewline
\ \ \isacommand{finally}\isamarkupfalse%
\ \isacommand{show}\isamarkupfalse%
\ {\isacharquery}{\kern0pt}thesis\ \isacommand{by}\isamarkupfalse%
\ simp\isanewline
\isacommand{qed}\isamarkupfalse%
%
\endisatagproof
{\isafoldproof}%
%
\isadelimproof
\isanewline
%
\endisadelimproof
%
\isadelimtheory
\isanewline
%
\endisadelimtheory
%
\isatagtheory
\isacommand{end}\isamarkupfalse%
%
\endisatagtheory
{\isafoldtheory}%
%
\isadelimtheory
%
\endisadelimtheory
%
\end{isabellebody}%
\endinput
%:%file=Median.tex%:%
%:%11=1%:%
%:%27=3%:%
%:%28=3%:%
%:%29=4%:%
%:%30=5%:%
%:%39=7%:%
%:%40=8%:%
%:%42=10%:%
%:%43=10%:%
%:%44=11%:%
%:%45=12%:%
%:%46=13%:%
%:%47=13%:%
%:%48=14%:%
%:%49=15%:%
%:%50=16%:%
%:%51=16%:%
%:%52=17%:%
%:%55=18%:%
%:%59=18%:%
%:%60=18%:%
%:%65=18%:%
%:%68=19%:%
%:%69=20%:%
%:%70=20%:%
%:%71=21%:%
%:%72=22%:%
%:%79=23%:%
%:%80=23%:%
%:%81=24%:%
%:%82=24%:%
%:%83=25%:%
%:%84=25%:%
%:%85=25%:%
%:%86=25%:%
%:%87=26%:%
%:%88=26%:%
%:%89=27%:%
%:%90=27%:%
%:%91=28%:%
%:%92=28%:%
%:%93=29%:%
%:%94=29%:%
%:%95=30%:%
%:%96=30%:%
%:%97=31%:%
%:%98=31%:%
%:%99=32%:%
%:%100=32%:%
%:%101=33%:%
%:%102=33%:%
%:%103=33%:%
%:%104=33%:%
%:%105=33%:%
%:%106=34%:%
%:%107=34%:%
%:%108=35%:%
%:%109=35%:%
%:%110=36%:%
%:%111=36%:%
%:%112=37%:%
%:%113=37%:%
%:%114=38%:%
%:%115=38%:%
%:%116=38%:%
%:%117=38%:%
%:%118=39%:%
%:%119=39%:%
%:%120=40%:%
%:%121=40%:%
%:%122=41%:%
%:%123=41%:%
%:%124=41%:%
%:%125=42%:%
%:%126=42%:%
%:%127=43%:%
%:%128=43%:%
%:%129=44%:%
%:%130=44%:%
%:%131=45%:%
%:%132=45%:%
%:%133=46%:%
%:%134=46%:%
%:%135=47%:%
%:%136=48%:%
%:%137=48%:%
%:%138=49%:%
%:%139=49%:%
%:%140=49%:%
%:%141=49%:%
%:%142=50%:%
%:%143=50%:%
%:%144=50%:%
%:%145=51%:%
%:%146=51%:%
%:%147=52%:%
%:%148=52%:%
%:%149=52%:%
%:%150=53%:%
%:%151=53%:%
%:%152=54%:%
%:%153=54%:%
%:%154=54%:%
%:%155=55%:%
%:%161=55%:%
%:%164=56%:%
%:%165=57%:%
%:%166=57%:%
%:%167=58%:%
%:%168=59%:%
%:%171=60%:%
%:%175=60%:%
%:%176=60%:%
%:%177=61%:%
%:%178=61%:%
%:%183=61%:%
%:%186=62%:%
%:%187=63%:%
%:%188=63%:%
%:%189=64%:%
%:%190=65%:%
%:%197=66%:%
%:%198=66%:%
%:%199=67%:%
%:%200=67%:%
%:%201=68%:%
%:%202=68%:%
%:%203=69%:%
%:%204=70%:%
%:%205=71%:%
%:%206=71%:%
%:%207=72%:%
%:%208=72%:%
%:%209=73%:%
%:%210=73%:%
%:%211=74%:%
%:%212=74%:%
%:%213=75%:%
%:%214=75%:%
%:%215=76%:%
%:%216=76%:%
%:%217=76%:%
%:%218=77%:%
%:%219=77%:%
%:%220=77%:%
%:%221=78%:%
%:%222=78%:%
%:%223=79%:%
%:%224=79%:%
%:%225=80%:%
%:%226=80%:%
%:%227=81%:%
%:%228=81%:%
%:%229=81%:%
%:%230=82%:%
%:%231=82%:%
%:%232=83%:%
%:%233=83%:%
%:%234=84%:%
%:%235=84%:%
%:%236=85%:%
%:%237=85%:%
%:%238=86%:%
%:%239=86%:%
%:%240=86%:%
%:%241=87%:%
%:%242=87%:%
%:%243=87%:%
%:%244=88%:%
%:%245=88%:%
%:%246=89%:%
%:%247=89%:%
%:%248=90%:%
%:%249=90%:%
%:%250=90%:%
%:%251=91%:%
%:%252=91%:%
%:%253=92%:%
%:%254=92%:%
%:%255=92%:%
%:%256=93%:%
%:%257=93%:%
%:%258=94%:%
%:%259=94%:%
%:%260=95%:%
%:%261=96%:%
%:%262=96%:%
%:%263=97%:%
%:%264=97%:%
%:%265=98%:%
%:%266=98%:%
%:%267=98%:%
%:%268=99%:%
%:%269=99%:%
%:%270=100%:%
%:%271=100%:%
%:%272=101%:%
%:%273=101%:%
%:%274=101%:%
%:%275=101%:%
%:%276=102%:%
%:%277=102%:%
%:%278=102%:%
%:%279=102%:%
%:%280=103%:%
%:%286=103%:%
%:%289=104%:%
%:%290=105%:%
%:%291=105%:%
%:%292=106%:%
%:%293=107%:%
%:%300=108%:%
%:%301=108%:%
%:%302=109%:%
%:%303=109%:%
%:%304=110%:%
%:%305=110%:%
%:%306=111%:%
%:%307=111%:%
%:%308=112%:%
%:%309=112%:%
%:%310=112%:%
%:%311=113%:%
%:%312=113%:%
%:%313=114%:%
%:%314=114%:%
%:%315=114%:%
%:%316=115%:%
%:%317=115%:%
%:%318=116%:%
%:%319=116%:%
%:%320=117%:%
%:%321=117%:%
%:%322=118%:%
%:%323=118%:%
%:%324=119%:%
%:%325=119%:%
%:%326=119%:%
%:%327=120%:%
%:%328=120%:%
%:%329=120%:%
%:%330=121%:%
%:%336=121%:%
%:%339=122%:%
%:%340=123%:%
%:%341=123%:%
%:%342=124%:%
%:%343=125%:%
%:%344=126%:%
%:%345=126%:%
%:%346=127%:%
%:%347=128%:%
%:%350=129%:%
%:%354=129%:%
%:%355=129%:%
%:%356=130%:%
%:%357=130%:%
%:%362=130%:%
%:%365=131%:%
%:%366=132%:%
%:%367=132%:%
%:%368=133%:%
%:%369=134%:%
%:%370=135%:%
%:%371=135%:%
%:%372=136%:%
%:%373=137%:%
%:%380=138%:%
%:%381=138%:%
%:%382=139%:%
%:%383=139%:%
%:%384=140%:%
%:%385=140%:%
%:%386=140%:%
%:%387=140%:%
%:%388=140%:%
%:%389=141%:%
%:%390=141%:%
%:%391=142%:%
%:%392=142%:%
%:%393=143%:%
%:%394=143%:%
%:%395=143%:%
%:%396=144%:%
%:%397=145%:%
%:%398=145%:%
%:%399=146%:%
%:%400=146%:%
%:%401=147%:%
%:%402=147%:%
%:%403=148%:%
%:%404=148%:%
%:%405=149%:%
%:%406=149%:%
%:%407=150%:%
%:%408=150%:%
%:%409=151%:%
%:%410=151%:%
%:%411=152%:%
%:%412=152%:%
%:%413=152%:%
%:%414=152%:%
%:%415=153%:%
%:%416=153%:%
%:%417=153%:%
%:%418=154%:%
%:%419=154%:%
%:%420=154%:%
%:%421=154%:%
%:%422=154%:%
%:%423=155%:%
%:%424=155%:%
%:%425=155%:%
%:%426=156%:%
%:%427=156%:%
%:%428=157%:%
%:%429=157%:%
%:%430=157%:%
%:%431=158%:%
%:%432=158%:%
%:%433=159%:%
%:%434=159%:%
%:%435=159%:%
%:%436=159%:%
%:%437=160%:%
%:%438=160%:%
%:%439=161%:%
%:%440=161%:%
%:%441=162%:%
%:%442=162%:%
%:%443=162%:%
%:%444=162%:%
%:%445=163%:%
%:%446=163%:%
%:%447=163%:%
%:%448=164%:%
%:%449=164%:%
%:%450=164%:%
%:%451=165%:%
%:%452=165%:%
%:%453=166%:%
%:%454=166%:%
%:%455=166%:%
%:%456=167%:%
%:%457=167%:%
%:%458=168%:%
%:%459=168%:%
%:%460=168%:%
%:%461=168%:%
%:%462=169%:%
%:%463=169%:%
%:%464=169%:%
%:%465=169%:%
%:%466=169%:%
%:%467=170%:%
%:%468=170%:%
%:%469=170%:%
%:%470=170%:%
%:%471=171%:%
%:%472=171%:%
%:%473=172%:%
%:%474=172%:%
%:%475=173%:%
%:%476=173%:%
%:%477=173%:%
%:%478=173%:%
%:%479=174%:%
%:%485=174%:%
%:%488=175%:%
%:%489=176%:%
%:%490=176%:%
%:%491=177%:%
%:%492=178%:%
%:%493=179%:%
%:%494=179%:%
%:%495=180%:%
%:%496=181%:%
%:%503=182%:%
%:%504=182%:%
%:%505=183%:%
%:%506=183%:%
%:%507=184%:%
%:%508=184%:%
%:%509=184%:%
%:%510=184%:%
%:%511=185%:%
%:%512=185%:%
%:%513=185%:%
%:%514=185%:%
%:%515=186%:%
%:%516=186%:%
%:%517=187%:%
%:%518=187%:%
%:%519=188%:%
%:%525=188%:%
%:%528=189%:%
%:%529=190%:%
%:%530=190%:%
%:%531=191%:%
%:%532=192%:%
%:%533=193%:%
%:%534=193%:%
%:%535=194%:%
%:%536=195%:%
%:%543=196%:%
%:%544=196%:%
%:%545=197%:%
%:%546=197%:%
%:%547=198%:%
%:%548=198%:%
%:%549=198%:%
%:%550=198%:%
%:%551=199%:%
%:%552=199%:%
%:%553=200%:%
%:%554=200%:%
%:%555=201%:%
%:%556=201%:%
%:%557=201%:%
%:%558=201%:%
%:%559=202%:%
%:%560=202%:%
%:%561=203%:%
%:%562=203%:%
%:%563=204%:%
%:%564=204%:%
%:%565=205%:%
%:%566=205%:%
%:%567=205%:%
%:%568=206%:%
%:%569=206%:%
%:%570=207%:%
%:%571=207%:%
%:%572=207%:%
%:%573=208%:%
%:%574=209%:%
%:%575=209%:%
%:%576=210%:%
%:%577=210%:%
%:%578=211%:%
%:%579=211%:%
%:%580=212%:%
%:%581=212%:%
%:%582=212%:%
%:%583=213%:%
%:%584=213%:%
%:%585=214%:%
%:%586=214%:%
%:%587=214%:%
%:%588=215%:%
%:%589=216%:%
%:%590=216%:%
%:%591=217%:%
%:%592=217%:%
%:%593=217%:%
%:%594=218%:%
%:%595=219%:%
%:%596=219%:%
%:%597=219%:%
%:%598=219%:%
%:%599=220%:%
%:%600=220%:%
%:%601=221%:%
%:%607=221%:%
%:%610=222%:%
%:%611=223%:%
%:%612=223%:%
%:%613=224%:%
%:%614=225%:%
%:%615=226%:%
%:%616=227%:%
%:%617=228%:%
%:%620=229%:%
%:%624=229%:%
%:%625=229%:%
%:%626=229%:%
%:%627=230%:%
%:%628=230%:%
%:%629=230%:%
%:%634=230%:%
%:%637=231%:%
%:%638=232%:%
%:%639=232%:%
%:%640=233%:%
%:%641=234%:%
%:%642=235%:%
%:%643=236%:%
%:%644=237%:%
%:%647=238%:%
%:%651=238%:%
%:%652=238%:%
%:%653=239%:%
%:%654=239%:%
%:%655=239%:%
%:%660=239%:%
%:%663=240%:%
%:%664=241%:%
%:%665=241%:%
%:%666=242%:%
%:%667=243%:%
%:%668=244%:%
%:%669=245%:%
%:%676=246%:%
%:%677=246%:%
%:%678=247%:%
%:%679=247%:%
%:%680=247%:%
%:%681=247%:%
%:%682=248%:%
%:%683=248%:%
%:%684=248%:%
%:%685=249%:%
%:%686=249%:%
%:%687=250%:%
%:%688=251%:%
%:%689=251%:%
%:%690=252%:%
%:%691=253%:%
%:%692=253%:%
%:%693=254%:%
%:%694=254%:%
%:%695=254%:%
%:%696=255%:%
%:%697=256%:%
%:%698=256%:%
%:%699=257%:%
%:%700=258%:%
%:%701=259%:%
%:%702=260%:%
%:%703=260%:%
%:%704=261%:%
%:%705=261%:%
%:%706=262%:%
%:%707=262%:%
%:%708=263%:%
%:%709=263%:%
%:%710=264%:%
%:%711=264%:%
%:%712=264%:%
%:%713=264%:%
%:%714=265%:%
%:%715=265%:%
%:%716=266%:%
%:%717=266%:%
%:%718=267%:%
%:%719=267%:%
%:%720=268%:%
%:%721=268%:%
%:%722=268%:%
%:%723=268%:%
%:%724=269%:%
%:%725=269%:%
%:%726=270%:%
%:%727=270%:%
%:%728=271%:%
%:%729=271%:%
%:%730=272%:%
%:%731=272%:%
%:%732=272%:%
%:%733=273%:%
%:%734=273%:%
%:%735=274%:%
%:%736=274%:%
%:%737=274%:%
%:%738=275%:%
%:%739=275%:%
%:%740=276%:%
%:%741=276%:%
%:%742=276%:%
%:%743=277%:%
%:%744=277%:%
%:%745=278%:%
%:%746=278%:%
%:%747=279%:%
%:%748=279%:%
%:%749=280%:%
%:%750=280%:%
%:%751=280%:%
%:%752=281%:%
%:%753=281%:%
%:%754=282%:%
%:%755=282%:%
%:%756=282%:%
%:%757=282%:%
%:%758=283%:%
%:%759=283%:%
%:%760=284%:%
%:%761=284%:%
%:%762=285%:%
%:%763=285%:%
%:%764=286%:%
%:%765=286%:%
%:%766=286%:%
%:%767=286%:%
%:%768=287%:%
%:%769=287%:%
%:%770=288%:%
%:%771=288%:%
%:%772=289%:%
%:%773=289%:%
%:%774=290%:%
%:%775=290%:%
%:%776=290%:%
%:%777=291%:%
%:%778=291%:%
%:%779=292%:%
%:%780=292%:%
%:%781=292%:%
%:%782=293%:%
%:%783=293%:%
%:%784=293%:%
%:%785=293%:%
%:%786=293%:%
%:%787=294%:%
%:%788=294%:%
%:%789=295%:%
%:%795=295%:%
%:%798=296%:%
%:%799=297%:%
%:%800=297%:%
%:%801=298%:%
%:%802=299%:%
%:%803=300%:%
%:%810=301%:%
%:%811=301%:%
%:%812=302%:%
%:%813=302%:%
%:%814=303%:%
%:%815=303%:%
%:%816=304%:%
%:%817=304%:%
%:%818=305%:%
%:%819=305%:%
%:%820=306%:%
%:%821=306%:%
%:%822=307%:%
%:%823=308%:%
%:%824=308%:%
%:%825=309%:%
%:%826=309%:%
%:%827=310%:%
%:%828=310%:%
%:%829=311%:%
%:%830=311%:%
%:%831=311%:%
%:%832=312%:%
%:%833=312%:%
%:%834=313%:%
%:%840=313%:%
%:%843=314%:%
%:%844=315%:%
%:%845=315%:%
%:%846=316%:%
%:%847=317%:%
%:%848=318%:%
%:%849=319%:%
%:%856=320%:%
%:%857=320%:%
%:%858=321%:%
%:%859=321%:%
%:%860=321%:%
%:%861=321%:%
%:%862=322%:%
%:%863=322%:%
%:%864=323%:%
%:%865=323%:%
%:%866=324%:%
%:%867=325%:%
%:%868=326%:%
%:%869=326%:%
%:%870=327%:%
%:%871=328%:%
%:%872=329%:%
%:%873=329%:%
%:%874=330%:%
%:%875=331%:%
%:%876=331%:%
%:%877=332%:%
%:%878=332%:%
%:%879=333%:%
%:%880=333%:%
%:%881=334%:%
%:%882=334%:%
%:%883=334%:%
%:%884=335%:%
%:%885=335%:%
%:%886=336%:%
%:%887=336%:%
%:%888=336%:%
%:%889=337%:%
%:%890=337%:%
%:%891=337%:%
%:%892=338%:%
%:%893=338%:%
%:%894=339%:%
%:%895=339%:%
%:%896=340%:%
%:%897=340%:%
%:%898=341%:%
%:%899=341%:%
%:%900=342%:%
%:%901=342%:%
%:%902=343%:%
%:%903=343%:%
%:%904=344%:%
%:%905=344%:%
%:%906=345%:%
%:%907=345%:%
%:%908=345%:%
%:%909=346%:%
%:%910=346%:%
%:%911=347%:%
%:%912=347%:%
%:%913=347%:%
%:%914=348%:%
%:%915=348%:%
%:%916=348%:%
%:%917=349%:%
%:%918=349%:%
%:%919=350%:%
%:%920=350%:%
%:%921=351%:%
%:%922=351%:%
%:%923=352%:%
%:%924=352%:%
%:%925=353%:%
%:%926=354%:%
%:%927=354%:%
%:%928=355%:%
%:%929=355%:%
%:%930=356%:%
%:%931=356%:%
%:%932=357%:%
%:%933=357%:%
%:%934=357%:%
%:%935=357%:%
%:%936=357%:%
%:%937=358%:%
%:%938=358%:%
%:%939=359%:%
%:%940=359%:%
%:%941=360%:%
%:%942=360%:%
%:%943=360%:%
%:%944=360%:%
%:%945=361%:%
%:%946=361%:%
%:%947=361%:%
%:%948=361%:%
%:%949=362%:%
%:%950=362%:%
%:%951=362%:%
%:%952=363%:%
%:%953=363%:%
%:%954=363%:%
%:%955=364%:%
%:%956=364%:%
%:%957=365%:%
%:%958=365%:%
%:%959=365%:%
%:%960=366%:%
%:%961=366%:%
%:%962=367%:%
%:%963=367%:%
%:%964=368%:%
%:%965=368%:%
%:%966=368%:%
%:%967=368%:%
%:%968=368%:%
%:%969=369%:%
%:%970=369%:%
%:%971=369%:%
%:%972=370%:%
%:%973=370%:%
%:%974=371%:%
%:%975=371%:%
%:%976=372%:%
%:%982=372%:%
%:%985=373%:%
%:%986=374%:%
%:%987=374%:%
%:%988=375%:%
%:%989=376%:%
%:%990=377%:%
%:%991=378%:%
%:%992=379%:%
%:%993=380%:%
%:%994=381%:%
%:%995=382%:%
%:%996=383%:%
%:%1003=384%:%
%:%1004=384%:%
%:%1005=385%:%
%:%1006=385%:%
%:%1007=386%:%
%:%1008=387%:%
%:%1009=387%:%
%:%1010=388%:%
%:%1011=388%:%
%:%1012=389%:%
%:%1013=389%:%
%:%1014=390%:%
%:%1015=390%:%
%:%1016=391%:%
%:%1017=391%:%
%:%1018=391%:%
%:%1019=392%:%
%:%1020=392%:%
%:%1021=392%:%
%:%1022=392%:%
%:%1023=392%:%
%:%1024=393%:%
%:%1025=393%:%
%:%1026=393%:%
%:%1027=393%:%
%:%1028=394%:%
%:%1029=394%:%
%:%1030=394%:%
%:%1031=395%:%
%:%1032=395%:%
%:%1033=395%:%
%:%1034=396%:%
%:%1035=397%:%
%:%1036=397%:%
%:%1037=398%:%
%:%1038=398%:%
%:%1039=399%:%
%:%1040=399%:%
%:%1041=400%:%
%:%1042=401%:%
%:%1043=401%:%
%:%1044=402%:%
%:%1045=402%:%
%:%1046=403%:%
%:%1047=403%:%
%:%1048=403%:%
%:%1049=404%:%
%:%1050=404%:%
%:%1051=405%:%
%:%1052=405%:%
%:%1053=405%:%
%:%1054=406%:%
%:%1055=406%:%
%:%1056=406%:%
%:%1057=406%:%
%:%1058=407%:%
%:%1059=407%:%
%:%1060=407%:%
%:%1061=407%:%
%:%1062=408%:%
%:%1063=409%:%
%:%1064=409%:%
%:%1065=410%:%
%:%1066=410%:%
%:%1067=411%:%
%:%1068=411%:%
%:%1069=411%:%
%:%1070=412%:%
%:%1071=412%:%
%:%1072=412%:%
%:%1073=412%:%
%:%1074=412%:%
%:%1075=413%:%
%:%1076=413%:%
%:%1077=413%:%
%:%1078=413%:%
%:%1079=414%:%
%:%1080=415%:%
%:%1081=415%:%
%:%1082=415%:%
%:%1083=415%:%
%:%1084=416%:%
%:%1085=416%:%
%:%1086=417%:%
%:%1087=417%:%
%:%1088=418%:%
%:%1089=419%:%
%:%1090=419%:%
%:%1091=420%:%
%:%1092=420%:%
%:%1093=421%:%
%:%1094=421%:%
%:%1095=422%:%
%:%1096=422%:%
%:%1097=423%:%
%:%1098=423%:%
%:%1099=423%:%
%:%1100=424%:%
%:%1101=425%:%
%:%1102=425%:%
%:%1103=426%:%
%:%1104=426%:%
%:%1105=427%:%
%:%1106=427%:%
%:%1107=428%:%
%:%1108=429%:%
%:%1109=429%:%
%:%1110=430%:%
%:%1111=430%:%
%:%1112=431%:%
%:%1113=431%:%
%:%1114=432%:%
%:%1115=432%:%
%:%1116=433%:%
%:%1117=433%:%
%:%1118=434%:%
%:%1119=434%:%
%:%1120=435%:%
%:%1121=435%:%
%:%1122=435%:%
%:%1123=436%:%
%:%1124=436%:%
%:%1125=437%:%
%:%1126=437%:%
%:%1127=438%:%
%:%1128=438%:%
%:%1129=439%:%
%:%1130=439%:%
%:%1131=439%:%
%:%1132=439%:%
%:%1133=440%:%
%:%1134=440%:%
%:%1135=441%:%
%:%1136=441%:%
%:%1137=441%:%
%:%1138=442%:%
%:%1139=442%:%
%:%1140=443%:%
%:%1141=444%:%
%:%1142=444%:%
%:%1143=445%:%
%:%1144=445%:%
%:%1145=446%:%
%:%1146=446%:%
%:%1147=447%:%
%:%1148=447%:%
%:%1149=447%:%
%:%1150=448%:%
%:%1151=448%:%
%:%1152=448%:%
%:%1153=449%:%
%:%1154=449%:%
%:%1155=450%:%
%:%1156=450%:%
%:%1157=451%:%
%:%1158=451%:%
%:%1159=451%:%
%:%1160=452%:%
%:%1161=452%:%
%:%1162=453%:%
%:%1163=453%:%
%:%1164=453%:%
%:%1165=453%:%
%:%1166=454%:%
%:%1167=454%:%
%:%1168=454%:%
%:%1169=455%:%
%:%1170=455%:%
%:%1171=455%:%
%:%1172=456%:%
%:%1173=456%:%
%:%1174=457%:%
%:%1175=457%:%
%:%1176=458%:%
%:%1177=458%:%
%:%1178=458%:%
%:%1179=459%:%
%:%1180=459%:%
%:%1181=460%:%
%:%1182=460%:%
%:%1183=461%:%
%:%1184=461%:%
%:%1185=461%:%
%:%1186=462%:%
%:%1187=462%:%
%:%1188=463%:%
%:%1189=463%:%
%:%1190=464%:%
%:%1191=464%:%
%:%1192=464%:%
%:%1193=465%:%
%:%1194=465%:%
%:%1195=466%:%
%:%1196=466%:%
%:%1197=466%:%
%:%1198=466%:%
%:%1199=467%:%
%:%1200=467%:%
%:%1201=467%:%
%:%1202=468%:%
%:%1203=468%:%
%:%1204=469%:%
%:%1205=469%:%
%:%1206=469%:%
%:%1207=470%:%
%:%1208=470%:%
%:%1209=470%:%
%:%1210=470%:%
%:%1211=471%:%
%:%1217=471%:%
%:%1220=472%:%
%:%1221=473%:%
%:%1222=473%:%
%:%1223=474%:%
%:%1224=475%:%
%:%1225=476%:%
%:%1226=477%:%
%:%1227=478%:%
%:%1228=479%:%
%:%1229=480%:%
%:%1230=481%:%
%:%1233=482%:%
%:%1237=482%:%
%:%1238=482%:%
%:%1239=483%:%
%:%1240=483%:%
%:%1245=483%:%
%:%1248=484%:%
%:%1249=485%:%
%:%1250=485%:%
%:%1251=486%:%
%:%1252=487%:%
%:%1253=488%:%
%:%1254=489%:%
%:%1255=490%:%
%:%1256=491%:%
%:%1257=492%:%
%:%1264=493%:%
%:%1265=493%:%
%:%1266=494%:%
%:%1267=494%:%
%:%1268=495%:%
%:%1269=495%:%
%:%1270=496%:%
%:%1271=496%:%
%:%1272=497%:%
%:%1273=497%:%
%:%1274=498%:%
%:%1275=499%:%
%:%1276=499%:%
%:%1277=500%:%
%:%1278=500%:%
%:%1279=501%:%
%:%1280=501%:%
%:%1281=502%:%
%:%1282=502%:%
%:%1283=502%:%
%:%1284=503%:%
%:%1285=503%:%
%:%1286=504%:%
%:%1287=504%:%
%:%1288=504%:%
%:%1289=505%:%
%:%1290=506%:%
%:%1291=506%:%
%:%1292=506%:%
%:%1293=507%:%
%:%1294=508%:%
%:%1295=508%:%
%:%1296=509%:%
%:%1297=509%:%
%:%1298=510%:%
%:%1299=510%:%
%:%1300=511%:%
%:%1301=511%:%
%:%1302=511%:%
%:%1303=512%:%
%:%1304=512%:%
%:%1305=512%:%
%:%1306=513%:%
%:%1307=513%:%
%:%1308=513%:%
%:%1309=514%:%
%:%1310=514%:%
%:%1311=515%:%
%:%1312=515%:%
%:%1313=516%:%
%:%1314=516%:%
%:%1315=517%:%
%:%1316=517%:%
%:%1317=518%:%
%:%1318=518%:%
%:%1319=518%:%
%:%1320=518%:%
%:%1321=519%:%
%:%1327=519%:%
%:%1330=520%:%
%:%1331=521%:%
%:%1332=521%:%
%:%1333=522%:%
%:%1334=523%:%
%:%1335=524%:%
%:%1338=525%:%
%:%1342=525%:%
%:%1343=525%:%
%:%1344=525%:%
%:%1345=526%:%
%:%1346=526%:%
%:%1347=527%:%
%:%1348=527%:%
%:%1353=527%:%
%:%1356=528%:%
%:%1357=529%:%
%:%1358=529%:%
%:%1359=530%:%
%:%1360=531%:%
%:%1363=532%:%
%:%1367=532%:%
%:%1368=532%:%
%:%1369=533%:%
%:%1370=533%:%
%:%1371=534%:%
%:%1372=534%:%
%:%1377=534%:%
%:%1380=535%:%
%:%1381=536%:%
%:%1382=536%:%
%:%1383=537%:%
%:%1384=538%:%
%:%1387=539%:%
%:%1391=539%:%
%:%1392=539%:%
%:%1393=540%:%
%:%1394=540%:%
%:%1395=541%:%
%:%1396=541%:%
%:%1397=542%:%
%:%1398=542%:%
%:%1399=543%:%
%:%1400=543%:%
%:%1405=543%:%
%:%1408=544%:%
%:%1409=545%:%
%:%1410=545%:%
%:%1411=546%:%
%:%1412=547%:%
%:%1415=548%:%
%:%1419=548%:%
%:%1420=548%:%
%:%1425=548%:%
%:%1428=549%:%
%:%1429=550%:%
%:%1430=550%:%
%:%1431=551%:%
%:%1432=552%:%
%:%1439=553%:%
%:%1440=553%:%
%:%1441=554%:%
%:%1442=554%:%
%:%1443=555%:%
%:%1444=556%:%
%:%1445=556%:%
%:%1446=557%:%
%:%1447=557%:%
%:%1448=558%:%
%:%1449=558%:%
%:%1450=559%:%
%:%1451=559%:%
%:%1452=560%:%
%:%1453=560%:%
%:%1454=560%:%
%:%1455=561%:%
%:%1456=561%:%
%:%1457=562%:%
%:%1458=562%:%
%:%1459=563%:%
%:%1465=563%:%
%:%1468=564%:%
%:%1469=565%:%
%:%1470=565%:%
%:%1471=566%:%
%:%1472=567%:%
%:%1479=568%:%
%:%1480=568%:%
%:%1481=569%:%
%:%1482=569%:%
%:%1483=570%:%
%:%1484=570%:%
%:%1485=571%:%
%:%1486=571%:%
%:%1487=572%:%
%:%1488=572%:%
%:%1489=573%:%
%:%1490=573%:%
%:%1491=574%:%
%:%1492=574%:%
%:%1493=575%:%
%:%1494=575%:%
%:%1495=575%:%
%:%1496=576%:%
%:%1497=576%:%
%:%1498=577%:%
%:%1499=577%:%
%:%1500=578%:%
%:%1501=578%:%
%:%1502=578%:%
%:%1503=579%:%
%:%1504=579%:%
%:%1505=579%:%
%:%1506=579%:%
%:%1507=580%:%
%:%1513=580%:%
%:%1518=581%:%
%:%1523=582%:%

%
\begin{isabellebody}%
\setisabellecontext{Set{\isacharunderscore}{\kern0pt}Ext}%
%
\isadelimtheory
%
\endisadelimtheory
%
\isatagtheory
\isacommand{theory}\isamarkupfalse%
\ Set{\isacharunderscore}{\kern0pt}Ext\isanewline
\isakeyword{imports}\ Main\isanewline
\isakeyword{begin}%
\endisatagtheory
{\isafoldtheory}%
%
\isadelimtheory
%
\endisadelimtheory
%
\begin{isamarkuptext}%
This is like \isa{card{\isacharunderscore}{\kern0pt}vimage{\isacharunderscore}{\kern0pt}inj} but supports \isa{inj{\isacharunderscore}{\kern0pt}on} instead.%
\end{isamarkuptext}\isamarkuptrue%
\isacommand{lemma}\isamarkupfalse%
\ card{\isacharunderscore}{\kern0pt}vimage{\isacharunderscore}{\kern0pt}inj{\isacharunderscore}{\kern0pt}on{\isacharcolon}{\kern0pt}\isanewline
\ \ \isakeyword{assumes}\ {\isachardoublequoteopen}inj{\isacharunderscore}{\kern0pt}on\ f\ B{\isachardoublequoteclose}\isanewline
\ \ \isakeyword{assumes}\ {\isachardoublequoteopen}A\ {\isasymsubseteq}\ f\ {\isacharbackquote}{\kern0pt}\ B{\isachardoublequoteclose}\isanewline
\ \ \isakeyword{shows}\ {\isachardoublequoteopen}card\ {\isacharparenleft}{\kern0pt}f\ {\isacharminus}{\kern0pt}{\isacharbackquote}{\kern0pt}\ A\ {\isasyminter}\ B{\isacharparenright}{\kern0pt}\ {\isacharequal}{\kern0pt}\ card\ A{\isachardoublequoteclose}\isanewline
%
\isadelimproof
%
\endisadelimproof
%
\isatagproof
\isacommand{proof}\isamarkupfalse%
\ {\isacharminus}{\kern0pt}\isanewline
\ \ \isacommand{have}\isamarkupfalse%
\ {\isachardoublequoteopen}A\ {\isacharequal}{\kern0pt}\ f\ {\isacharbackquote}{\kern0pt}\ {\isacharparenleft}{\kern0pt}f\ {\isacharminus}{\kern0pt}{\isacharbackquote}{\kern0pt}\ A\ {\isasyminter}\ B{\isacharparenright}{\kern0pt}{\isachardoublequoteclose}\ \isacommand{using}\isamarkupfalse%
\ assms{\isacharparenleft}{\kern0pt}{\isadigit{2}}{\isacharparenright}{\kern0pt}\ \isacommand{by}\isamarkupfalse%
\ auto\isanewline
\ \ \isacommand{thus}\isamarkupfalse%
\ {\isacharquery}{\kern0pt}thesis\ \isacommand{using}\isamarkupfalse%
\ assms\ card{\isacharunderscore}{\kern0pt}image\ \isanewline
\ \ \ \ \isacommand{by}\isamarkupfalse%
\ {\isacharparenleft}{\kern0pt}metis\ inf{\isacharunderscore}{\kern0pt}le{\isadigit{2}}\ inj{\isacharunderscore}{\kern0pt}on{\isacharunderscore}{\kern0pt}subset{\isacharparenright}{\kern0pt}\isanewline
\isacommand{qed}\isamarkupfalse%
%
\endisatagproof
{\isafoldproof}%
%
\isadelimproof
\isanewline
%
\endisadelimproof
\isanewline
\isacommand{lemma}\isamarkupfalse%
\ card{\isacharunderscore}{\kern0pt}ordered{\isacharunderscore}{\kern0pt}pairs{\isacharcolon}{\kern0pt}\isanewline
\ \ \isakeyword{fixes}\ M\ {\isacharcolon}{\kern0pt}{\isacharcolon}{\kern0pt}\ {\isachardoublequoteopen}{\isacharparenleft}{\kern0pt}{\isacharprime}{\kern0pt}a\ {\isacharcolon}{\kern0pt}{\isacharcolon}{\kern0pt}linorder{\isacharparenright}{\kern0pt}\ set{\isachardoublequoteclose}\ \isanewline
\ \ \isakeyword{assumes}\ {\isachardoublequoteopen}finite\ M{\isachardoublequoteclose}\isanewline
\ \ \isakeyword{shows}\ {\isachardoublequoteopen}{\isadigit{2}}\ {\isacharasterisk}{\kern0pt}\ card\ {\isacharbraceleft}{\kern0pt}{\isacharparenleft}{\kern0pt}x{\isacharcomma}{\kern0pt}y{\isacharparenright}{\kern0pt}\ {\isasymin}\ M\ {\isasymtimes}\ M{\isachardot}{\kern0pt}\ x\ {\isacharless}{\kern0pt}\ y{\isacharbraceright}{\kern0pt}\ {\isacharequal}{\kern0pt}\ card\ M\ {\isacharasterisk}{\kern0pt}\ {\isacharparenleft}{\kern0pt}card\ M\ {\isacharminus}{\kern0pt}\ {\isadigit{1}}{\isacharparenright}{\kern0pt}{\isachardoublequoteclose}\isanewline
%
\isadelimproof
%
\endisadelimproof
%
\isatagproof
\isacommand{proof}\isamarkupfalse%
\ {\isacharminus}{\kern0pt}\isanewline
\ \ \isacommand{have}\isamarkupfalse%
\ {\isachardoublequoteopen}{\isadigit{2}}\ {\isacharasterisk}{\kern0pt}\ card\ {\isacharbraceleft}{\kern0pt}{\isacharparenleft}{\kern0pt}x{\isacharcomma}{\kern0pt}y{\isacharparenright}{\kern0pt}\ {\isasymin}\ M\ {\isasymtimes}\ M{\isachardot}{\kern0pt}\ x\ {\isacharless}{\kern0pt}\ y{\isacharbraceright}{\kern0pt}\ {\isacharequal}{\kern0pt}\isanewline
\ \ \ \ card\ {\isacharbraceleft}{\kern0pt}{\isacharparenleft}{\kern0pt}x{\isacharcomma}{\kern0pt}y{\isacharparenright}{\kern0pt}\ {\isasymin}\ M\ {\isasymtimes}\ M{\isachardot}{\kern0pt}\ x\ {\isacharless}{\kern0pt}\ y{\isacharbraceright}{\kern0pt}\ {\isacharplus}{\kern0pt}\ card\ {\isacharparenleft}{\kern0pt}{\isacharparenleft}{\kern0pt}{\isasymlambda}x{\isachardot}{\kern0pt}\ {\isacharparenleft}{\kern0pt}snd\ x{\isacharcomma}{\kern0pt}\ fst\ x{\isacharparenright}{\kern0pt}{\isacharparenright}{\kern0pt}{\isacharbackquote}{\kern0pt}{\isacharbraceleft}{\kern0pt}{\isacharparenleft}{\kern0pt}x{\isacharcomma}{\kern0pt}y{\isacharparenright}{\kern0pt}\ {\isasymin}\ M\ {\isasymtimes}\ M{\isachardot}{\kern0pt}\ x\ {\isacharless}{\kern0pt}\ y{\isacharbraceright}{\kern0pt}{\isacharparenright}{\kern0pt}{\isachardoublequoteclose}\isanewline
\ \ \ \ \isacommand{apply}\isamarkupfalse%
\ {\isacharparenleft}{\kern0pt}subst\ card{\isacharunderscore}{\kern0pt}image{\isacharparenright}{\kern0pt}\isanewline
\ \ \ \ \isacommand{apply}\isamarkupfalse%
\ {\isacharparenleft}{\kern0pt}rule\ inj{\isacharunderscore}{\kern0pt}onI{\isacharcomma}{\kern0pt}\ simp\ add{\isacharcolon}{\kern0pt}case{\isacharunderscore}{\kern0pt}prod{\isacharunderscore}{\kern0pt}beta\ prod{\isacharunderscore}{\kern0pt}eq{\isacharunderscore}{\kern0pt}iff{\isacharparenright}{\kern0pt}\isanewline
\ \ \ \ \isacommand{by}\isamarkupfalse%
\ simp\isanewline
\ \ \isacommand{also}\isamarkupfalse%
\ \isacommand{have}\isamarkupfalse%
\ {\isachardoublequoteopen}{\isachardot}{\kern0pt}{\isachardot}{\kern0pt}{\isachardot}{\kern0pt}\ {\isacharequal}{\kern0pt}\ card\ {\isacharbraceleft}{\kern0pt}{\isacharparenleft}{\kern0pt}x{\isacharcomma}{\kern0pt}y{\isacharparenright}{\kern0pt}\ {\isasymin}\ M\ {\isasymtimes}\ M{\isachardot}{\kern0pt}\ x\ {\isacharless}{\kern0pt}\ y{\isacharbraceright}{\kern0pt}\ {\isacharplus}{\kern0pt}\ card\ {\isacharbraceleft}{\kern0pt}{\isacharparenleft}{\kern0pt}x{\isacharcomma}{\kern0pt}y{\isacharparenright}{\kern0pt}\ {\isasymin}\ M\ {\isasymtimes}\ M{\isachardot}{\kern0pt}\ y\ {\isacharless}{\kern0pt}\ x{\isacharbraceright}{\kern0pt}{\isachardoublequoteclose}\isanewline
\ \ \ \ \isacommand{apply}\isamarkupfalse%
\ {\isacharparenleft}{\kern0pt}rule\ arg{\isacharunderscore}{\kern0pt}cong{\isadigit{2}}{\isacharbrackleft}{\kern0pt}\isakeyword{where}\ f{\isacharequal}{\kern0pt}{\isachardoublequoteopen}{\isacharparenleft}{\kern0pt}{\isacharplus}{\kern0pt}{\isacharparenright}{\kern0pt}{\isachardoublequoteclose}{\isacharbrackright}{\kern0pt}{\isacharcomma}{\kern0pt}\ simp{\isacharparenright}{\kern0pt}\isanewline
\ \ \ \ \isacommand{apply}\isamarkupfalse%
\ {\isacharparenleft}{\kern0pt}rule\ arg{\isacharunderscore}{\kern0pt}cong{\isacharbrackleft}{\kern0pt}\isakeyword{where}\ f{\isacharequal}{\kern0pt}{\isachardoublequoteopen}card{\isachardoublequoteclose}{\isacharbrackright}{\kern0pt}{\isacharparenright}{\kern0pt}\isanewline
\ \ \ \ \isacommand{apply}\isamarkupfalse%
\ {\isacharparenleft}{\kern0pt}rule\ order{\isacharunderscore}{\kern0pt}antisym{\isacharparenright}{\kern0pt}\isanewline
\ \ \ \ \ \isacommand{apply}\isamarkupfalse%
\ {\isacharparenleft}{\kern0pt}rule\ image{\isacharunderscore}{\kern0pt}subsetI{\isacharcomma}{\kern0pt}\ simp\ add{\isacharcolon}{\kern0pt}case{\isacharunderscore}{\kern0pt}prod{\isacharunderscore}{\kern0pt}beta{\isacharparenright}{\kern0pt}\isanewline
\ \ \ \ \isacommand{apply}\isamarkupfalse%
\ {\isacharparenleft}{\kern0pt}rule\ subsetI{\isacharcomma}{\kern0pt}\ simp{\isacharparenright}{\kern0pt}\ \isanewline
\ \ \ \ \isacommand{using}\isamarkupfalse%
\ image{\isacharunderscore}{\kern0pt}iff\ \isacommand{by}\isamarkupfalse%
\ fastforce\ \isanewline
\ \ \isacommand{also}\isamarkupfalse%
\ \isacommand{have}\isamarkupfalse%
\ {\isachardoublequoteopen}{\isachardot}{\kern0pt}{\isachardot}{\kern0pt}{\isachardot}{\kern0pt}\ {\isacharequal}{\kern0pt}\ card\ {\isacharparenleft}{\kern0pt}{\isacharbraceleft}{\kern0pt}{\isacharparenleft}{\kern0pt}x{\isacharcomma}{\kern0pt}y{\isacharparenright}{\kern0pt}\ {\isasymin}\ M\ {\isasymtimes}\ M{\isachardot}{\kern0pt}\ x\ {\isacharless}{\kern0pt}\ y{\isacharbraceright}{\kern0pt}\ {\isasymunion}\ {\isacharbraceleft}{\kern0pt}{\isacharparenleft}{\kern0pt}x{\isacharcomma}{\kern0pt}y{\isacharparenright}{\kern0pt}\ {\isasymin}\ M\ {\isasymtimes}\ M{\isachardot}{\kern0pt}\ y\ {\isacharless}{\kern0pt}\ x{\isacharbraceright}{\kern0pt}{\isacharparenright}{\kern0pt}{\isachardoublequoteclose}\isanewline
\ \ \ \ \isacommand{apply}\isamarkupfalse%
\ {\isacharparenleft}{\kern0pt}rule\ card{\isacharunderscore}{\kern0pt}Un{\isacharunderscore}{\kern0pt}disjoint{\isacharbrackleft}{\kern0pt}symmetric{\isacharbrackright}{\kern0pt}{\isacharparenright}{\kern0pt}\isanewline
\ \ \ \ \isacommand{apply}\isamarkupfalse%
\ {\isacharparenleft}{\kern0pt}rule\ finite{\isacharunderscore}{\kern0pt}subset{\isacharbrackleft}{\kern0pt}\isakeyword{where}\ B{\isacharequal}{\kern0pt}{\isachardoublequoteopen}M\ {\isasymtimes}\ M{\isachardoublequoteclose}{\isacharbrackright}{\kern0pt}{\isacharcomma}{\kern0pt}\ rule\ subsetI{\isacharcomma}{\kern0pt}\ simp\ add{\isacharcolon}{\kern0pt}case{\isacharunderscore}{\kern0pt}prod{\isacharunderscore}{\kern0pt}beta\ mem{\isacharunderscore}{\kern0pt}Times{\isacharunderscore}{\kern0pt}iff{\isacharparenright}{\kern0pt}\isanewline
\ \ \ \ \isacommand{using}\isamarkupfalse%
\ assms\ \isacommand{apply}\isamarkupfalse%
\ simp\isanewline
\ \ \ \ \isacommand{apply}\isamarkupfalse%
\ {\isacharparenleft}{\kern0pt}rule\ finite{\isacharunderscore}{\kern0pt}subset{\isacharbrackleft}{\kern0pt}\isakeyword{where}\ B{\isacharequal}{\kern0pt}{\isachardoublequoteopen}M\ {\isasymtimes}\ M{\isachardoublequoteclose}{\isacharbrackright}{\kern0pt}{\isacharcomma}{\kern0pt}\ rule\ subsetI{\isacharcomma}{\kern0pt}\ simp\ add{\isacharcolon}{\kern0pt}case{\isacharunderscore}{\kern0pt}prod{\isacharunderscore}{\kern0pt}beta\ mem{\isacharunderscore}{\kern0pt}Times{\isacharunderscore}{\kern0pt}iff{\isacharparenright}{\kern0pt}\isanewline
\ \ \ \ \isacommand{using}\isamarkupfalse%
\ assms\ \isacommand{apply}\isamarkupfalse%
\ simp\isanewline
\ \ \ \ \isacommand{apply}\isamarkupfalse%
\ {\isacharparenleft}{\kern0pt}rule\ order{\isacharunderscore}{\kern0pt}antisym{\isacharcomma}{\kern0pt}\ rule\ subsetI{\isacharcomma}{\kern0pt}\ simp\ add{\isacharcolon}{\kern0pt}case{\isacharunderscore}{\kern0pt}prod{\isacharunderscore}{\kern0pt}beta{\isacharcomma}{\kern0pt}\ force{\isacharparenright}{\kern0pt}\ \isanewline
\ \ \ \ \isacommand{by}\isamarkupfalse%
\ simp\isanewline
\ \ \isacommand{also}\isamarkupfalse%
\ \isacommand{have}\isamarkupfalse%
\ {\isachardoublequoteopen}{\isachardot}{\kern0pt}{\isachardot}{\kern0pt}{\isachardot}{\kern0pt}\ {\isacharequal}{\kern0pt}\ card\ {\isacharparenleft}{\kern0pt}{\isacharparenleft}{\kern0pt}M\ {\isasymtimes}\ M{\isacharparenright}{\kern0pt}\ {\isacharminus}{\kern0pt}\ {\isacharbraceleft}{\kern0pt}{\isacharparenleft}{\kern0pt}x{\isacharcomma}{\kern0pt}y{\isacharparenright}{\kern0pt}\ {\isasymin}\ M\ {\isasymtimes}\ M{\isachardot}{\kern0pt}\ x\ {\isacharequal}{\kern0pt}\ y{\isacharbraceright}{\kern0pt}{\isacharparenright}{\kern0pt}{\isachardoublequoteclose}\isanewline
\ \ \ \ \isacommand{apply}\isamarkupfalse%
\ {\isacharparenleft}{\kern0pt}rule\ arg{\isacharunderscore}{\kern0pt}cong{\isacharbrackleft}{\kern0pt}\isakeyword{where}\ f{\isacharequal}{\kern0pt}{\isachardoublequoteopen}card{\isachardoublequoteclose}{\isacharbrackright}{\kern0pt}{\isacharparenright}{\kern0pt}\isanewline
\ \ \ \ \isacommand{apply}\isamarkupfalse%
\ {\isacharparenleft}{\kern0pt}rule\ order{\isacharunderscore}{\kern0pt}antisym{\isacharcomma}{\kern0pt}\ rule\ subsetI{\isacharcomma}{\kern0pt}\ simp\ add{\isacharcolon}{\kern0pt}case{\isacharunderscore}{\kern0pt}prod{\isacharunderscore}{\kern0pt}beta{\isacharcomma}{\kern0pt}\ force{\isacharparenright}{\kern0pt}\isanewline
\ \ \ \ \isacommand{by}\isamarkupfalse%
\ {\isacharparenleft}{\kern0pt}rule\ subsetI{\isacharcomma}{\kern0pt}\ simp\ add{\isacharcolon}{\kern0pt}case{\isacharunderscore}{\kern0pt}prod{\isacharunderscore}{\kern0pt}beta{\isacharcomma}{\kern0pt}\ force{\isacharparenright}{\kern0pt}\isanewline
\ \ \isacommand{also}\isamarkupfalse%
\ \isacommand{have}\isamarkupfalse%
\ {\isachardoublequoteopen}{\isachardot}{\kern0pt}{\isachardot}{\kern0pt}{\isachardot}{\kern0pt}\ {\isacharequal}{\kern0pt}\ card\ {\isacharparenleft}{\kern0pt}M\ {\isasymtimes}\ M{\isacharparenright}{\kern0pt}\ {\isacharminus}{\kern0pt}\ card\ {\isacharbraceleft}{\kern0pt}{\isacharparenleft}{\kern0pt}x{\isacharcomma}{\kern0pt}y{\isacharparenright}{\kern0pt}\ {\isasymin}\ M\ {\isasymtimes}\ M{\isachardot}{\kern0pt}\ x\ {\isacharequal}{\kern0pt}\ y{\isacharbraceright}{\kern0pt}{\isachardoublequoteclose}\isanewline
\ \ \ \ \isacommand{apply}\isamarkupfalse%
\ {\isacharparenleft}{\kern0pt}rule\ card{\isacharunderscore}{\kern0pt}Diff{\isacharunderscore}{\kern0pt}subset{\isacharparenright}{\kern0pt}\isanewline
\ \ \ \ \isacommand{apply}\isamarkupfalse%
\ {\isacharparenleft}{\kern0pt}rule\ finite{\isacharunderscore}{\kern0pt}subset{\isacharbrackleft}{\kern0pt}\isakeyword{where}\ B{\isacharequal}{\kern0pt}{\isachardoublequoteopen}M\ {\isasymtimes}\ M{\isachardoublequoteclose}{\isacharbrackright}{\kern0pt}{\isacharcomma}{\kern0pt}\ rule\ subsetI{\isacharcomma}{\kern0pt}\ simp\ add{\isacharcolon}{\kern0pt}case{\isacharunderscore}{\kern0pt}prod{\isacharunderscore}{\kern0pt}beta\ mem{\isacharunderscore}{\kern0pt}Times{\isacharunderscore}{\kern0pt}iff{\isacharparenright}{\kern0pt}\isanewline
\ \ \ \ \isacommand{using}\isamarkupfalse%
\ assms\ \isacommand{apply}\isamarkupfalse%
\ simp\isanewline
\ \ \ \ \isacommand{by}\isamarkupfalse%
\ {\isacharparenleft}{\kern0pt}rule\ subsetI{\isacharcomma}{\kern0pt}\ simp\ add{\isacharcolon}{\kern0pt}case{\isacharunderscore}{\kern0pt}prod{\isacharunderscore}{\kern0pt}beta\ mem{\isacharunderscore}{\kern0pt}Times{\isacharunderscore}{\kern0pt}iff{\isacharparenright}{\kern0pt}\isanewline
\ \ \isacommand{also}\isamarkupfalse%
\ \isacommand{have}\isamarkupfalse%
\ {\isachardoublequoteopen}{\isachardot}{\kern0pt}{\isachardot}{\kern0pt}{\isachardot}{\kern0pt}\ {\isacharequal}{\kern0pt}\ card\ M\ {\isacharcircum}{\kern0pt}\ {\isadigit{2}}\ {\isacharminus}{\kern0pt}\ card\ {\isacharparenleft}{\kern0pt}{\isacharparenleft}{\kern0pt}{\isasymlambda}x{\isachardot}{\kern0pt}\ {\isacharparenleft}{\kern0pt}x{\isacharcomma}{\kern0pt}x{\isacharparenright}{\kern0pt}{\isacharparenright}{\kern0pt}\ {\isacharbackquote}{\kern0pt}\ M{\isacharparenright}{\kern0pt}{\isachardoublequoteclose}\isanewline
\ \ \ \ \isacommand{apply}\isamarkupfalse%
\ {\isacharparenleft}{\kern0pt}rule\ arg{\isacharunderscore}{\kern0pt}cong{\isadigit{2}}{\isacharbrackleft}{\kern0pt}\isakeyword{where}\ f{\isacharequal}{\kern0pt}{\isachardoublequoteopen}{\isacharparenleft}{\kern0pt}{\isacharminus}{\kern0pt}{\isacharparenright}{\kern0pt}{\isachardoublequoteclose}{\isacharbrackright}{\kern0pt}{\isacharparenright}{\kern0pt}\isanewline
\ \ \ \ \isacommand{using}\isamarkupfalse%
\ assms\ \isacommand{apply}\isamarkupfalse%
\ {\isacharparenleft}{\kern0pt}simp\ add{\isacharcolon}{\kern0pt}power{\isadigit{2}}{\isacharunderscore}{\kern0pt}eq{\isacharunderscore}{\kern0pt}square{\isacharparenright}{\kern0pt}\isanewline
\ \ \ \ \isacommand{apply}\isamarkupfalse%
\ {\isacharparenleft}{\kern0pt}rule\ arg{\isacharunderscore}{\kern0pt}cong{\isacharbrackleft}{\kern0pt}\isakeyword{where}\ f{\isacharequal}{\kern0pt}{\isachardoublequoteopen}card{\isachardoublequoteclose}{\isacharbrackright}{\kern0pt}{\isacharparenright}{\kern0pt}\isanewline
\ \ \ \ \isacommand{apply}\isamarkupfalse%
\ {\isacharparenleft}{\kern0pt}rule\ order{\isacharunderscore}{\kern0pt}antisym{\isacharcomma}{\kern0pt}\ rule\ subsetI{\isacharcomma}{\kern0pt}\ simp\ add{\isacharcolon}{\kern0pt}case{\isacharunderscore}{\kern0pt}prod{\isacharunderscore}{\kern0pt}beta{\isacharcomma}{\kern0pt}\ force{\isacharparenright}{\kern0pt}\isanewline
\ \ \ \ \isacommand{by}\isamarkupfalse%
\ {\isacharparenleft}{\kern0pt}rule\ image{\isacharunderscore}{\kern0pt}subsetI{\isacharcomma}{\kern0pt}\ simp{\isacharparenright}{\kern0pt}\isanewline
\ \ \isacommand{also}\isamarkupfalse%
\ \isacommand{have}\isamarkupfalse%
\ {\isachardoublequoteopen}{\isachardot}{\kern0pt}{\isachardot}{\kern0pt}{\isachardot}{\kern0pt}\ {\isacharequal}{\kern0pt}\ card\ M\ {\isacharcircum}{\kern0pt}\ {\isadigit{2}}\ {\isacharminus}{\kern0pt}\ card\ M{\isachardoublequoteclose}\isanewline
\ \ \ \ \isacommand{apply}\isamarkupfalse%
\ {\isacharparenleft}{\kern0pt}rule\ arg{\isacharunderscore}{\kern0pt}cong{\isadigit{2}}{\isacharbrackleft}{\kern0pt}\isakeyword{where}\ f{\isacharequal}{\kern0pt}{\isachardoublequoteopen}{\isacharparenleft}{\kern0pt}{\isacharminus}{\kern0pt}{\isacharparenright}{\kern0pt}{\isachardoublequoteclose}{\isacharbrackright}{\kern0pt}{\isacharcomma}{\kern0pt}\ simp{\isacharparenright}{\kern0pt}\isanewline
\ \ \ \ \isacommand{apply}\isamarkupfalse%
\ {\isacharparenleft}{\kern0pt}rule\ card{\isacharunderscore}{\kern0pt}image{\isacharparenright}{\kern0pt}\isanewline
\ \ \ \ \isacommand{by}\isamarkupfalse%
\ {\isacharparenleft}{\kern0pt}rule\ inj{\isacharunderscore}{\kern0pt}onI{\isacharcomma}{\kern0pt}\ simp{\isacharparenright}{\kern0pt}\isanewline
\ \ \isacommand{also}\isamarkupfalse%
\ \isacommand{have}\isamarkupfalse%
\ {\isachardoublequoteopen}{\isachardot}{\kern0pt}{\isachardot}{\kern0pt}{\isachardot}{\kern0pt}\ {\isacharequal}{\kern0pt}\ card\ M\ {\isacharasterisk}{\kern0pt}\ {\isacharparenleft}{\kern0pt}card\ M\ {\isacharminus}{\kern0pt}\ {\isadigit{1}}{\isacharparenright}{\kern0pt}{\isachardoublequoteclose}\isanewline
\ \ \ \ \isacommand{apply}\isamarkupfalse%
\ {\isacharparenleft}{\kern0pt}cases\ {\isachardoublequoteopen}card\ M\ {\isasymge}\ {\isadigit{0}}{\isachardoublequoteclose}{\isacharcomma}{\kern0pt}\ simp\ add{\isacharcolon}{\kern0pt}power{\isadigit{2}}{\isacharunderscore}{\kern0pt}eq{\isacharunderscore}{\kern0pt}square\ algebra{\isacharunderscore}{\kern0pt}simps{\isacharparenright}{\kern0pt}\isanewline
\ \ \ \ \isacommand{by}\isamarkupfalse%
\ simp\isanewline
\ \ \isacommand{finally}\isamarkupfalse%
\ \isacommand{show}\isamarkupfalse%
\ {\isacharquery}{\kern0pt}thesis\ \isacommand{by}\isamarkupfalse%
\ simp\isanewline
\isacommand{qed}\isamarkupfalse%
%
\endisatagproof
{\isafoldproof}%
%
\isadelimproof
\isanewline
%
\endisadelimproof
%
\isadelimtheory
\isanewline
%
\endisadelimtheory
%
\isatagtheory
\isacommand{end}\isamarkupfalse%
%
\endisatagtheory
{\isafoldtheory}%
%
\isadelimtheory
%
\endisadelimtheory
%
\end{isabellebody}%
\endinput
%:%file=Set_Ext.tex%:%
%:%10=1%:%
%:%11=1%:%
%:%12=2%:%
%:%13=3%:%
%:%22=5%:%
%:%24=6%:%
%:%25=6%:%
%:%26=7%:%
%:%27=8%:%
%:%28=9%:%
%:%35=10%:%
%:%36=10%:%
%:%37=11%:%
%:%38=11%:%
%:%39=11%:%
%:%40=11%:%
%:%41=12%:%
%:%42=12%:%
%:%43=12%:%
%:%44=13%:%
%:%45=13%:%
%:%46=14%:%
%:%52=14%:%
%:%55=15%:%
%:%56=16%:%
%:%57=16%:%
%:%58=17%:%
%:%59=18%:%
%:%60=19%:%
%:%67=20%:%
%:%68=20%:%
%:%69=21%:%
%:%70=21%:%
%:%71=22%:%
%:%72=23%:%
%:%73=23%:%
%:%74=24%:%
%:%75=24%:%
%:%76=25%:%
%:%77=25%:%
%:%78=26%:%
%:%79=26%:%
%:%80=26%:%
%:%81=27%:%
%:%82=27%:%
%:%83=28%:%
%:%84=28%:%
%:%85=29%:%
%:%86=29%:%
%:%87=30%:%
%:%88=30%:%
%:%89=31%:%
%:%90=31%:%
%:%91=32%:%
%:%92=32%:%
%:%93=32%:%
%:%94=33%:%
%:%95=33%:%
%:%96=33%:%
%:%97=34%:%
%:%98=34%:%
%:%99=35%:%
%:%100=35%:%
%:%101=36%:%
%:%102=36%:%
%:%103=36%:%
%:%104=37%:%
%:%105=37%:%
%:%106=38%:%
%:%107=38%:%
%:%108=38%:%
%:%109=39%:%
%:%110=39%:%
%:%111=40%:%
%:%112=40%:%
%:%113=41%:%
%:%114=41%:%
%:%115=41%:%
%:%116=42%:%
%:%117=42%:%
%:%118=43%:%
%:%119=43%:%
%:%120=44%:%
%:%121=44%:%
%:%122=45%:%
%:%123=45%:%
%:%124=45%:%
%:%125=46%:%
%:%126=46%:%
%:%127=47%:%
%:%128=47%:%
%:%129=48%:%
%:%130=48%:%
%:%131=48%:%
%:%132=49%:%
%:%133=49%:%
%:%134=50%:%
%:%135=50%:%
%:%136=50%:%
%:%137=51%:%
%:%138=51%:%
%:%139=52%:%
%:%140=52%:%
%:%141=52%:%
%:%142=53%:%
%:%143=53%:%
%:%144=54%:%
%:%145=54%:%
%:%146=55%:%
%:%147=55%:%
%:%148=56%:%
%:%149=56%:%
%:%150=56%:%
%:%151=57%:%
%:%152=57%:%
%:%153=58%:%
%:%154=58%:%
%:%155=59%:%
%:%156=59%:%
%:%157=60%:%
%:%158=60%:%
%:%159=60%:%
%:%160=61%:%
%:%161=61%:%
%:%162=62%:%
%:%163=62%:%
%:%164=63%:%
%:%165=63%:%
%:%166=63%:%
%:%167=63%:%
%:%168=64%:%
%:%174=64%:%
%:%179=65%:%
%:%184=66%:%

%
\begin{isabellebody}%
\setisabellecontext{OrderStatistics}%
%
\isadelimdocument
%
\endisadelimdocument
%
\isatagdocument
%
\isamarkupsection{Order Statistics%
}
\isamarkuptrue%
%
\endisatagdocument
{\isafolddocument}%
%
\isadelimdocument
%
\endisadelimdocument
%
\isadelimtheory
%
\endisadelimtheory
%
\isatagtheory
\isacommand{theory}\isamarkupfalse%
\ OrderStatistics\isanewline
\ \ \isakeyword{imports}\ Main\ {\isachardoublequoteopen}HOL{\isacharminus}{\kern0pt}Library{\isachardot}{\kern0pt}Multiset{\isachardoublequoteclose}\ List{\isacharunderscore}{\kern0pt}Ext\ Multiset{\isacharunderscore}{\kern0pt}Ext\ Set{\isacharunderscore}{\kern0pt}Ext\isanewline
\isakeyword{begin}%
\endisatagtheory
{\isafoldtheory}%
%
\isadelimtheory
%
\endisadelimtheory
%
\begin{isamarkuptext}%
This section contains definitions and results about order statistics.%
\end{isamarkuptext}\isamarkuptrue%
\isacommand{definition}\isamarkupfalse%
\ rank{\isacharunderscore}{\kern0pt}of\ {\isacharcolon}{\kern0pt}{\isacharcolon}{\kern0pt}\ {\isachardoublequoteopen}{\isacharprime}{\kern0pt}a\ {\isacharcolon}{\kern0pt}{\isacharcolon}{\kern0pt}\ linorder\ {\isasymRightarrow}\ {\isacharprime}{\kern0pt}a\ set\ {\isasymRightarrow}\ nat{\isachardoublequoteclose}\ \isakeyword{where}\ {\isachardoublequoteopen}rank{\isacharunderscore}{\kern0pt}of\ x\ S\ {\isacharequal}{\kern0pt}\ card\ {\isacharbraceleft}{\kern0pt}y\ {\isasymin}\ S{\isachardot}{\kern0pt}\ y\ {\isacharless}{\kern0pt}\ x{\isacharbraceright}{\kern0pt}{\isachardoublequoteclose}%
\begin{isamarkuptext}%
The function \isa{rank{\isacharunderscore}{\kern0pt}of} returns the rank of an element within a set.%
\end{isamarkuptext}\isamarkuptrue%
\isacommand{lemma}\isamarkupfalse%
\ rank{\isacharunderscore}{\kern0pt}mono{\isacharcolon}{\kern0pt}\isanewline
\ \ \isakeyword{assumes}\ {\isachardoublequoteopen}finite\ S{\isachardoublequoteclose}\isanewline
\ \ \isakeyword{shows}\ {\isachardoublequoteopen}x\ {\isasymle}\ y\ {\isasymLongrightarrow}\ rank{\isacharunderscore}{\kern0pt}of\ x\ S\ {\isasymle}\ rank{\isacharunderscore}{\kern0pt}of\ y\ S{\isachardoublequoteclose}\isanewline
%
\isadelimproof
\ \ %
\endisadelimproof
%
\isatagproof
\isacommand{apply}\isamarkupfalse%
\ {\isacharparenleft}{\kern0pt}simp\ add{\isacharcolon}{\kern0pt}rank{\isacharunderscore}{\kern0pt}of{\isacharunderscore}{\kern0pt}def{\isacharparenright}{\kern0pt}\isanewline
\ \ \isacommand{apply}\isamarkupfalse%
\ {\isacharparenleft}{\kern0pt}rule\ card{\isacharunderscore}{\kern0pt}mono{\isacharparenright}{\kern0pt}\isanewline
\ \ \isacommand{using}\isamarkupfalse%
\ assms\ \isacommand{apply}\isamarkupfalse%
\ simp\isanewline
\ \ \isacommand{by}\isamarkupfalse%
\ {\isacharparenleft}{\kern0pt}rule\ subsetI{\isacharcomma}{\kern0pt}\ simp{\isacharcomma}{\kern0pt}\ force{\isacharparenright}{\kern0pt}%
\endisatagproof
{\isafoldproof}%
%
\isadelimproof
\isanewline
%
\endisadelimproof
\isanewline
\isacommand{lemma}\isamarkupfalse%
\ rank{\isacharunderscore}{\kern0pt}mono{\isacharunderscore}{\kern0pt}commute{\isacharcolon}{\kern0pt}\isanewline
\ \ \isakeyword{assumes}\ {\isachardoublequoteopen}finite\ S{\isachardoublequoteclose}\isanewline
\ \ \isakeyword{assumes}\ {\isachardoublequoteopen}S\ {\isasymsubseteq}\ T{\isachardoublequoteclose}\isanewline
\ \ \isakeyword{assumes}\ {\isachardoublequoteopen}strict{\isacharunderscore}{\kern0pt}mono{\isacharunderscore}{\kern0pt}on\ f\ T{\isachardoublequoteclose}\isanewline
\ \ \isakeyword{assumes}\ {\isachardoublequoteopen}x\ {\isasymin}\ T{\isachardoublequoteclose}\isanewline
\ \ \isakeyword{shows}\ {\isachardoublequoteopen}rank{\isacharunderscore}{\kern0pt}of\ x\ S\ {\isacharequal}{\kern0pt}\ rank{\isacharunderscore}{\kern0pt}of\ {\isacharparenleft}{\kern0pt}f\ x{\isacharparenright}{\kern0pt}\ {\isacharparenleft}{\kern0pt}f\ {\isacharbackquote}{\kern0pt}\ S{\isacharparenright}{\kern0pt}{\isachardoublequoteclose}\isanewline
%
\isadelimproof
%
\endisadelimproof
%
\isatagproof
\isacommand{proof}\isamarkupfalse%
\ {\isacharminus}{\kern0pt}\isanewline
\ \ \isacommand{have}\isamarkupfalse%
\ {\isachardoublequoteopen}rank{\isacharunderscore}{\kern0pt}of\ {\isacharparenleft}{\kern0pt}f\ x{\isacharparenright}{\kern0pt}\ {\isacharparenleft}{\kern0pt}f\ {\isacharbackquote}{\kern0pt}\ S{\isacharparenright}{\kern0pt}\ {\isacharequal}{\kern0pt}\ card\ {\isacharparenleft}{\kern0pt}f\ {\isacharbackquote}{\kern0pt}\ {\isacharbraceleft}{\kern0pt}y\ {\isasymin}\ S{\isachardot}{\kern0pt}\ y\ {\isacharless}{\kern0pt}\ x{\isacharbraceright}{\kern0pt}{\isacharparenright}{\kern0pt}{\isachardoublequoteclose}\isanewline
\ \ \ \ \isacommand{apply}\isamarkupfalse%
\ {\isacharparenleft}{\kern0pt}simp\ add{\isacharcolon}{\kern0pt}rank{\isacharunderscore}{\kern0pt}of{\isacharunderscore}{\kern0pt}def{\isacharparenright}{\kern0pt}\isanewline
\ \ \ \ \isacommand{apply}\isamarkupfalse%
\ {\isacharparenleft}{\kern0pt}rule\ arg{\isacharunderscore}{\kern0pt}cong{\isacharbrackleft}{\kern0pt}\isakeyword{where}\ f{\isacharequal}{\kern0pt}{\isachardoublequoteopen}card{\isachardoublequoteclose}{\isacharbrackright}{\kern0pt}{\isacharparenright}{\kern0pt}\isanewline
\ \ \ \ \isacommand{apply}\isamarkupfalse%
\ {\isacharparenleft}{\kern0pt}rule\ order{\isacharunderscore}{\kern0pt}antisym{\isacharparenright}{\kern0pt}\isanewline
\ \ \ \ \isacommand{apply}\isamarkupfalse%
\ {\isacharparenleft}{\kern0pt}rule\ subsetI{\isacharcomma}{\kern0pt}\ simp{\isacharparenright}{\kern0pt}\isanewline
\ \ \ \ \isacommand{using}\isamarkupfalse%
\ assms\ strict{\isacharunderscore}{\kern0pt}mono{\isacharunderscore}{\kern0pt}on{\isacharunderscore}{\kern0pt}leD\ \isacommand{apply}\isamarkupfalse%
\ fastforce\isanewline
\ \ \ \ \isacommand{apply}\isamarkupfalse%
\ {\isacharparenleft}{\kern0pt}rule\ image{\isacharunderscore}{\kern0pt}subsetI{\isacharcomma}{\kern0pt}\ simp{\isacharparenright}{\kern0pt}\ \isanewline
\ \ \ \ \isacommand{using}\isamarkupfalse%
\ assms\ \isacommand{by}\isamarkupfalse%
\ {\isacharparenleft}{\kern0pt}simp\ add{\isacharcolon}{\kern0pt}\ in{\isacharunderscore}{\kern0pt}mono\ strict{\isacharunderscore}{\kern0pt}mono{\isacharunderscore}{\kern0pt}on{\isacharunderscore}{\kern0pt}def{\isacharparenright}{\kern0pt}\isanewline
\ \ \isacommand{also}\isamarkupfalse%
\ \isacommand{have}\isamarkupfalse%
\ {\isachardoublequoteopen}{\isachardot}{\kern0pt}{\isachardot}{\kern0pt}{\isachardot}{\kern0pt}\ {\isacharequal}{\kern0pt}\ card\ {\isacharbraceleft}{\kern0pt}y\ {\isasymin}\ S{\isachardot}{\kern0pt}\ y\ {\isacharless}{\kern0pt}\ x{\isacharbraceright}{\kern0pt}{\isachardoublequoteclose}\isanewline
\ \ \ \ \isacommand{apply}\isamarkupfalse%
\ {\isacharparenleft}{\kern0pt}rule\ card{\isacharunderscore}{\kern0pt}image{\isacharparenright}{\kern0pt}\isanewline
\ \ \ \ \isacommand{apply}\isamarkupfalse%
\ {\isacharparenleft}{\kern0pt}rule\ inj{\isacharunderscore}{\kern0pt}on{\isacharunderscore}{\kern0pt}subset{\isacharbrackleft}{\kern0pt}\isakeyword{where}\ A{\isacharequal}{\kern0pt}{\isachardoublequoteopen}T{\isachardoublequoteclose}{\isacharbrackright}{\kern0pt}{\isacharparenright}{\kern0pt}\isanewline
\ \ \ \ \ \isacommand{apply}\isamarkupfalse%
\ {\isacharparenleft}{\kern0pt}metis\ assms{\isacharparenleft}{\kern0pt}{\isadigit{3}}{\isacharparenright}{\kern0pt}\ strict{\isacharunderscore}{\kern0pt}mono{\isacharunderscore}{\kern0pt}on{\isacharunderscore}{\kern0pt}imp{\isacharunderscore}{\kern0pt}inj{\isacharunderscore}{\kern0pt}on{\isacharparenright}{\kern0pt}\isanewline
\ \ \ \ \isacommand{using}\isamarkupfalse%
\ assms\ \isacommand{by}\isamarkupfalse%
\ blast\isanewline
\ \ \isacommand{also}\isamarkupfalse%
\ \isacommand{have}\isamarkupfalse%
\ {\isachardoublequoteopen}{\isachardot}{\kern0pt}{\isachardot}{\kern0pt}{\isachardot}{\kern0pt}\ {\isacharequal}{\kern0pt}\ rank{\isacharunderscore}{\kern0pt}of\ x\ S{\isachardoublequoteclose}\isanewline
\ \ \ \ \isacommand{by}\isamarkupfalse%
\ {\isacharparenleft}{\kern0pt}simp\ add{\isacharcolon}{\kern0pt}rank{\isacharunderscore}{\kern0pt}of{\isacharunderscore}{\kern0pt}def{\isacharparenright}{\kern0pt}\isanewline
\ \ \isacommand{finally}\isamarkupfalse%
\ \isacommand{show}\isamarkupfalse%
\ {\isacharquery}{\kern0pt}thesis\isanewline
\ \ \ \ \isacommand{by}\isamarkupfalse%
\ simp\isanewline
\isacommand{qed}\isamarkupfalse%
%
\endisatagproof
{\isafoldproof}%
%
\isadelimproof
\isanewline
%
\endisadelimproof
\isanewline
\isanewline
\isacommand{definition}\isamarkupfalse%
\ least\ \isakeyword{where}\ {\isachardoublequoteopen}least\ k\ S\ {\isacharequal}{\kern0pt}\ {\isacharbraceleft}{\kern0pt}y\ {\isasymin}\ S{\isachardot}{\kern0pt}\ rank{\isacharunderscore}{\kern0pt}of\ y\ S\ {\isacharless}{\kern0pt}\ k{\isacharbraceright}{\kern0pt}{\isachardoublequoteclose}%
\begin{isamarkuptext}%
The function \isa{least} returns the k smallest elements of a finite set.%
\end{isamarkuptext}\isamarkuptrue%
\isacommand{lemma}\isamarkupfalse%
\ rank{\isacharunderscore}{\kern0pt}strict{\isacharunderscore}{\kern0pt}mono{\isacharcolon}{\kern0pt}\ \isanewline
\ \ \isakeyword{assumes}\ {\isachardoublequoteopen}finite\ S{\isachardoublequoteclose}\isanewline
\ \ \isakeyword{shows}\ {\isachardoublequoteopen}strict{\isacharunderscore}{\kern0pt}mono{\isacharunderscore}{\kern0pt}on\ {\isacharparenleft}{\kern0pt}{\isasymlambda}x{\isachardot}{\kern0pt}\ rank{\isacharunderscore}{\kern0pt}of\ x\ S{\isacharparenright}{\kern0pt}\ S{\isachardoublequoteclose}\isanewline
%
\isadelimproof
%
\endisadelimproof
%
\isatagproof
\isacommand{proof}\isamarkupfalse%
\ {\isacharminus}{\kern0pt}\isanewline
\ \ \isacommand{have}\isamarkupfalse%
\ {\isachardoublequoteopen}{\isasymAnd}x\ y{\isachardot}{\kern0pt}\ x\ {\isasymin}\ S\ {\isasymLongrightarrow}\ y\ {\isasymin}\ S\ {\isasymLongrightarrow}\ x\ {\isacharless}{\kern0pt}\ y\ {\isasymLongrightarrow}\ rank{\isacharunderscore}{\kern0pt}of\ x\ S\ {\isacharless}{\kern0pt}\ rank{\isacharunderscore}{\kern0pt}of\ y\ S{\isachardoublequoteclose}\isanewline
\ \ \ \ \isacommand{apply}\isamarkupfalse%
\ {\isacharparenleft}{\kern0pt}simp\ add{\isacharcolon}{\kern0pt}rank{\isacharunderscore}{\kern0pt}of{\isacharunderscore}{\kern0pt}def{\isacharparenright}{\kern0pt}\isanewline
\ \ \ \ \isacommand{apply}\isamarkupfalse%
\ {\isacharparenleft}{\kern0pt}rule\ psubset{\isacharunderscore}{\kern0pt}card{\isacharunderscore}{\kern0pt}mono{\isacharparenright}{\kern0pt}\isanewline
\ \ \ \ \ \isacommand{apply}\isamarkupfalse%
\ {\isacharparenleft}{\kern0pt}simp\ add{\isacharcolon}{\kern0pt}assms{\isacharparenright}{\kern0pt}\isanewline
\ \ \ \ \isacommand{apply}\isamarkupfalse%
\ {\isacharparenleft}{\kern0pt}simp\ add{\isacharcolon}{\kern0pt}\ psubset{\isacharunderscore}{\kern0pt}eq{\isacharparenright}{\kern0pt}\isanewline
\ \ \ \ \isacommand{apply}\isamarkupfalse%
\ {\isacharparenleft}{\kern0pt}rule\ conjI{\isacharcomma}{\kern0pt}\ rule\ subsetI{\isacharcomma}{\kern0pt}\ force{\isacharparenright}{\kern0pt}\isanewline
\ \ \ \ \isacommand{by}\isamarkupfalse%
\ blast\isanewline
\isanewline
\ \ \isacommand{thus}\isamarkupfalse%
\ {\isacharquery}{\kern0pt}thesis\isanewline
\ \ \ \ \isacommand{by}\isamarkupfalse%
\ {\isacharparenleft}{\kern0pt}simp\ add{\isacharcolon}{\kern0pt}rank{\isacharunderscore}{\kern0pt}of{\isacharunderscore}{\kern0pt}def\ strict{\isacharunderscore}{\kern0pt}mono{\isacharunderscore}{\kern0pt}on{\isacharunderscore}{\kern0pt}def{\isacharparenright}{\kern0pt}\isanewline
\isacommand{qed}\isamarkupfalse%
%
\endisatagproof
{\isafoldproof}%
%
\isadelimproof
\isanewline
%
\endisadelimproof
\isanewline
\isanewline
\isacommand{lemma}\isamarkupfalse%
\ rank{\isacharunderscore}{\kern0pt}of{\isacharunderscore}{\kern0pt}image{\isacharcolon}{\kern0pt}\isanewline
\ \ \isakeyword{assumes}\ {\isachardoublequoteopen}finite\ S{\isachardoublequoteclose}\isanewline
\ \ \isakeyword{shows}\ {\isachardoublequoteopen}{\isacharparenleft}{\kern0pt}{\isasymlambda}x{\isachardot}{\kern0pt}\ rank{\isacharunderscore}{\kern0pt}of\ x\ S{\isacharparenright}{\kern0pt}\ {\isacharbackquote}{\kern0pt}\ S\ {\isacharequal}{\kern0pt}\ {\isacharbraceleft}{\kern0pt}{\isadigit{0}}{\isachardot}{\kern0pt}{\isachardot}{\kern0pt}{\isacharless}{\kern0pt}card\ S{\isacharbraceright}{\kern0pt}{\isachardoublequoteclose}\isanewline
%
\isadelimproof
\ \ %
\endisadelimproof
%
\isatagproof
\isacommand{apply}\isamarkupfalse%
\ {\isacharparenleft}{\kern0pt}rule\ card{\isacharunderscore}{\kern0pt}seteq{\isacharcomma}{\kern0pt}\ simp{\isacharparenright}{\kern0pt}\isanewline
\ \ \ \isacommand{apply}\isamarkupfalse%
\ {\isacharparenleft}{\kern0pt}rule\ image{\isacharunderscore}{\kern0pt}subsetI{\isacharcomma}{\kern0pt}\ simp\ add{\isacharcolon}{\kern0pt}rank{\isacharunderscore}{\kern0pt}of{\isacharunderscore}{\kern0pt}def{\isacharparenright}{\kern0pt}\isanewline
\ \ \ \isacommand{apply}\isamarkupfalse%
\ {\isacharparenleft}{\kern0pt}rule\ psubset{\isacharunderscore}{\kern0pt}card{\isacharunderscore}{\kern0pt}mono{\isacharcomma}{\kern0pt}\ metis\ assms{\isacharcomma}{\kern0pt}\ blast{\isacharparenright}{\kern0pt}\isanewline
\ \ \isacommand{apply}\isamarkupfalse%
\ simp\isanewline
\ \ \isacommand{apply}\isamarkupfalse%
\ {\isacharparenleft}{\kern0pt}subst\ card{\isacharunderscore}{\kern0pt}image{\isacharparenright}{\kern0pt}\isanewline
\ \ \ \isacommand{apply}\isamarkupfalse%
\ {\isacharparenleft}{\kern0pt}metis\ strict{\isacharunderscore}{\kern0pt}mono{\isacharunderscore}{\kern0pt}on{\isacharunderscore}{\kern0pt}imp{\isacharunderscore}{\kern0pt}inj{\isacharunderscore}{\kern0pt}on\ rank{\isacharunderscore}{\kern0pt}strict{\isacharunderscore}{\kern0pt}mono\ assms{\isacharparenright}{\kern0pt}\ \isanewline
\ \ \isacommand{by}\isamarkupfalse%
\ simp%
\endisatagproof
{\isafoldproof}%
%
\isadelimproof
\isanewline
%
\endisadelimproof
\isanewline
\isacommand{lemma}\isamarkupfalse%
\ card{\isacharunderscore}{\kern0pt}least{\isacharcolon}{\kern0pt}\isanewline
\ \ \isakeyword{assumes}\ {\isachardoublequoteopen}finite\ S{\isachardoublequoteclose}\isanewline
\ \ \isakeyword{shows}\ {\isachardoublequoteopen}card\ {\isacharparenleft}{\kern0pt}least\ k\ S{\isacharparenright}{\kern0pt}\ {\isacharequal}{\kern0pt}\ min\ k\ {\isacharparenleft}{\kern0pt}card\ S{\isacharparenright}{\kern0pt}{\isachardoublequoteclose}\isanewline
%
\isadelimproof
%
\endisadelimproof
%
\isatagproof
\isacommand{proof}\isamarkupfalse%
\ {\isacharparenleft}{\kern0pt}cases\ {\isachardoublequoteopen}card\ S\ {\isacharless}{\kern0pt}\ k{\isachardoublequoteclose}{\isacharparenright}{\kern0pt}\isanewline
\ \ \isacommand{case}\isamarkupfalse%
\ True\isanewline
\ \ \isacommand{have}\isamarkupfalse%
\ {\isachardoublequoteopen}{\isasymAnd}t{\isachardot}{\kern0pt}\ rank{\isacharunderscore}{\kern0pt}of\ t\ S\ {\isasymle}\ card\ S{\isachardoublequoteclose}\ \isanewline
\ \ \ \ \isacommand{apply}\isamarkupfalse%
\ {\isacharparenleft}{\kern0pt}simp\ add{\isacharcolon}{\kern0pt}rank{\isacharunderscore}{\kern0pt}of{\isacharunderscore}{\kern0pt}def{\isacharparenright}{\kern0pt}\isanewline
\ \ \ \ \isacommand{by}\isamarkupfalse%
\ {\isacharparenleft}{\kern0pt}rule\ card{\isacharunderscore}{\kern0pt}mono{\isacharcomma}{\kern0pt}\ metis\ assms{\isacharcomma}{\kern0pt}\ simp{\isacharparenright}{\kern0pt}\isanewline
\ \ \isacommand{hence}\isamarkupfalse%
\ {\isachardoublequoteopen}{\isasymAnd}t{\isachardot}{\kern0pt}\ rank{\isacharunderscore}{\kern0pt}of\ t\ S\ {\isacharless}{\kern0pt}\ k{\isachardoublequoteclose}\ \isanewline
\ \ \ \ \isacommand{by}\isamarkupfalse%
\ {\isacharparenleft}{\kern0pt}metis\ True\ not{\isacharunderscore}{\kern0pt}less{\isacharunderscore}{\kern0pt}iff{\isacharunderscore}{\kern0pt}gr{\isacharunderscore}{\kern0pt}or{\isacharunderscore}{\kern0pt}eq\ order{\isacharunderscore}{\kern0pt}less{\isacharunderscore}{\kern0pt}le{\isacharunderscore}{\kern0pt}trans{\isacharparenright}{\kern0pt}\isanewline
\ \ \isacommand{hence}\isamarkupfalse%
\ {\isachardoublequoteopen}least\ k\ S\ {\isacharequal}{\kern0pt}\ S{\isachardoublequoteclose}\isanewline
\ \ \ \ \isacommand{by}\isamarkupfalse%
\ {\isacharparenleft}{\kern0pt}simp\ add{\isacharcolon}{\kern0pt}least{\isacharunderscore}{\kern0pt}def{\isacharparenright}{\kern0pt}\isanewline
\ \ \isacommand{then}\isamarkupfalse%
\ \isacommand{show}\isamarkupfalse%
\ {\isacharquery}{\kern0pt}thesis\ \isacommand{using}\isamarkupfalse%
\ True\ \isacommand{by}\isamarkupfalse%
\ simp\isanewline
\isacommand{next}\isamarkupfalse%
\isanewline
\ \ \isacommand{case}\isamarkupfalse%
\ False\isanewline
\ \ \isacommand{hence}\isamarkupfalse%
\ a{\isacharcolon}{\kern0pt}{\isachardoublequoteopen}card\ S\ {\isasymge}\ k{\isachardoublequoteclose}\ \isacommand{using}\isamarkupfalse%
\ leI\ \isacommand{by}\isamarkupfalse%
\ blast\isanewline
\ \ \isacommand{have}\isamarkupfalse%
\ {\isachardoublequoteopen}card\ {\isacharparenleft}{\kern0pt}{\isacharparenleft}{\kern0pt}{\isasymlambda}x{\isachardot}{\kern0pt}\ rank{\isacharunderscore}{\kern0pt}of\ x\ S{\isacharparenright}{\kern0pt}\ {\isacharminus}{\kern0pt}{\isacharbackquote}{\kern0pt}\ {\isacharbraceleft}{\kern0pt}{\isadigit{0}}{\isachardot}{\kern0pt}{\isachardot}{\kern0pt}{\isacharless}{\kern0pt}k{\isacharbraceright}{\kern0pt}\ {\isasyminter}\ S{\isacharparenright}{\kern0pt}\ {\isacharequal}{\kern0pt}\ card\ {\isacharbraceleft}{\kern0pt}{\isadigit{0}}{\isachardot}{\kern0pt}{\isachardot}{\kern0pt}{\isacharless}{\kern0pt}k{\isacharbraceright}{\kern0pt}{\isachardoublequoteclose}\isanewline
\ \ \ \ \isacommand{apply}\isamarkupfalse%
\ {\isacharparenleft}{\kern0pt}rule\ card{\isacharunderscore}{\kern0pt}vimage{\isacharunderscore}{\kern0pt}inj{\isacharunderscore}{\kern0pt}on{\isacharparenright}{\kern0pt}\isanewline
\ \ \ \ \ \isacommand{apply}\isamarkupfalse%
\ {\isacharparenleft}{\kern0pt}metis\ strict{\isacharunderscore}{\kern0pt}mono{\isacharunderscore}{\kern0pt}on{\isacharunderscore}{\kern0pt}imp{\isacharunderscore}{\kern0pt}inj{\isacharunderscore}{\kern0pt}on\ rank{\isacharunderscore}{\kern0pt}strict{\isacharunderscore}{\kern0pt}mono\ assms{\isacharparenright}{\kern0pt}\ \isanewline
\ \ \ \ \isacommand{apply}\isamarkupfalse%
\ {\isacharparenleft}{\kern0pt}subst\ rank{\isacharunderscore}{\kern0pt}of{\isacharunderscore}{\kern0pt}image{\isacharcomma}{\kern0pt}\ metis\ assms{\isacharparenright}{\kern0pt}\isanewline
\ \ \ \ \isacommand{using}\isamarkupfalse%
\ a\ \isacommand{by}\isamarkupfalse%
\ simp\isanewline
\ \ \isacommand{hence}\isamarkupfalse%
\ {\isachardoublequoteopen}card\ {\isacharparenleft}{\kern0pt}least\ k\ S{\isacharparenright}{\kern0pt}\ {\isacharequal}{\kern0pt}\ k{\isachardoublequoteclose}\isanewline
\ \ \ \ \isacommand{by}\isamarkupfalse%
\ {\isacharparenleft}{\kern0pt}simp\ add{\isacharcolon}{\kern0pt}\ Collect{\isacharunderscore}{\kern0pt}conj{\isacharunderscore}{\kern0pt}eq\ Int{\isacharunderscore}{\kern0pt}commute\ least{\isacharunderscore}{\kern0pt}def\ vimage{\isacharunderscore}{\kern0pt}def{\isacharparenright}{\kern0pt}\isanewline
\ \ \isacommand{then}\isamarkupfalse%
\ \isacommand{show}\isamarkupfalse%
\ {\isacharquery}{\kern0pt}thesis\ \isacommand{using}\isamarkupfalse%
\ a\ \isacommand{by}\isamarkupfalse%
\ linarith\isanewline
\isacommand{qed}\isamarkupfalse%
%
\endisatagproof
{\isafoldproof}%
%
\isadelimproof
\isanewline
%
\endisadelimproof
\isanewline
\isacommand{lemma}\isamarkupfalse%
\ least{\isacharunderscore}{\kern0pt}subset{\isacharcolon}{\kern0pt}\ {\isachardoublequoteopen}least\ k\ S\ {\isasymsubseteq}\ S{\isachardoublequoteclose}\isanewline
%
\isadelimproof
\ \ %
\endisadelimproof
%
\isatagproof
\isacommand{by}\isamarkupfalse%
\ {\isacharparenleft}{\kern0pt}simp\ add{\isacharcolon}{\kern0pt}least{\isacharunderscore}{\kern0pt}def{\isacharparenright}{\kern0pt}%
\endisatagproof
{\isafoldproof}%
%
\isadelimproof
\isanewline
%
\endisadelimproof
\isanewline
\isacommand{lemma}\isamarkupfalse%
\ preserve{\isacharunderscore}{\kern0pt}rank{\isacharcolon}{\kern0pt}\isanewline
\ \ \isakeyword{assumes}\ {\isachardoublequoteopen}finite\ S{\isachardoublequoteclose}\isanewline
\ \ \isakeyword{shows}\ {\isachardoublequoteopen}rank{\isacharunderscore}{\kern0pt}of\ x\ {\isacharparenleft}{\kern0pt}least\ m\ S{\isacharparenright}{\kern0pt}\ {\isacharequal}{\kern0pt}\ min\ m\ {\isacharparenleft}{\kern0pt}rank{\isacharunderscore}{\kern0pt}of\ x\ S{\isacharparenright}{\kern0pt}{\isachardoublequoteclose}\isanewline
%
\isadelimproof
%
\endisadelimproof
%
\isatagproof
\isacommand{proof}\isamarkupfalse%
\ {\isacharparenleft}{\kern0pt}cases\ {\isachardoublequoteopen}rank{\isacharunderscore}{\kern0pt}of\ x\ S\ {\isasymge}\ m{\isachardoublequoteclose}{\isacharparenright}{\kern0pt}\isanewline
\ \ \isacommand{case}\isamarkupfalse%
\ True\isanewline
\ \ \isacommand{hence}\isamarkupfalse%
\ {\isachardoublequoteopen}{\isacharbraceleft}{\kern0pt}y\ {\isasymin}\ least\ m\ S{\isachardot}{\kern0pt}\ y\ {\isacharless}{\kern0pt}\ x{\isacharbraceright}{\kern0pt}\ {\isacharequal}{\kern0pt}\ least\ m\ S{\isachardoublequoteclose}\ \isanewline
\ \ \ \ \isacommand{apply}\isamarkupfalse%
\ {\isacharparenleft}{\kern0pt}simp\ add{\isacharcolon}{\kern0pt}\ least{\isacharunderscore}{\kern0pt}def{\isacharparenright}{\kern0pt}\isanewline
\ \ \ \ \isacommand{apply}\isamarkupfalse%
\ {\isacharparenleft}{\kern0pt}rule\ Collect{\isacharunderscore}{\kern0pt}cong{\isacharparenright}{\kern0pt}\isanewline
\ \ \ \ \isacommand{using}\isamarkupfalse%
\ rank{\isacharunderscore}{\kern0pt}mono{\isacharbrackleft}{\kern0pt}OF\ assms{\isacharbrackright}{\kern0pt}\isanewline
\ \ \ \ \isacommand{by}\isamarkupfalse%
\ {\isacharparenleft}{\kern0pt}metis\ linorder{\isacharunderscore}{\kern0pt}not{\isacharunderscore}{\kern0pt}less\ order{\isacharunderscore}{\kern0pt}less{\isacharunderscore}{\kern0pt}le{\isacharunderscore}{\kern0pt}trans{\isacharparenright}{\kern0pt}\isanewline
\ \ \isacommand{moreover}\isamarkupfalse%
\ \isacommand{have}\isamarkupfalse%
\ {\isachardoublequoteopen}m\ {\isasymle}\ card\ S{\isachardoublequoteclose}\isanewline
\ \ \ \ \isacommand{apply}\isamarkupfalse%
\ {\isacharparenleft}{\kern0pt}rule\ order{\isacharunderscore}{\kern0pt}trans{\isacharbrackleft}{\kern0pt}\isakeyword{where}\ y{\isacharequal}{\kern0pt}{\isachardoublequoteopen}rank{\isacharunderscore}{\kern0pt}of\ x\ S{\isachardoublequoteclose}{\isacharbrackright}{\kern0pt}{\isacharcomma}{\kern0pt}\ metis\ True{\isacharparenright}{\kern0pt}\isanewline
\ \ \ \ \isacommand{apply}\isamarkupfalse%
\ {\isacharparenleft}{\kern0pt}simp\ add{\isacharcolon}{\kern0pt}rank{\isacharunderscore}{\kern0pt}of{\isacharunderscore}{\kern0pt}def{\isacharparenright}{\kern0pt}\isanewline
\ \ \ \ \isacommand{by}\isamarkupfalse%
\ {\isacharparenleft}{\kern0pt}rule\ card{\isacharunderscore}{\kern0pt}mono{\isacharbrackleft}{\kern0pt}OF\ assms{\isacharbrackright}{\kern0pt}{\isacharcomma}{\kern0pt}\ simp{\isacharparenright}{\kern0pt}\isanewline
\ \ \isacommand{hence}\isamarkupfalse%
\ {\isachardoublequoteopen}card\ {\isacharparenleft}{\kern0pt}least\ m\ S{\isacharparenright}{\kern0pt}\ {\isacharequal}{\kern0pt}\ m{\isachardoublequoteclose}\isanewline
\ \ \ \ \isacommand{apply}\isamarkupfalse%
\ {\isacharparenleft}{\kern0pt}subst\ card{\isacharunderscore}{\kern0pt}least{\isacharbrackleft}{\kern0pt}OF\ assms{\isacharbrackright}{\kern0pt}{\isacharparenright}{\kern0pt}\isanewline
\ \ \ \ \isacommand{by}\isamarkupfalse%
\ simp\isanewline
\ \ \isacommand{ultimately}\isamarkupfalse%
\ \isacommand{show}\isamarkupfalse%
\ {\isacharquery}{\kern0pt}thesis\ \isacommand{using}\isamarkupfalse%
\ True\ \isacommand{by}\isamarkupfalse%
\ {\isacharparenleft}{\kern0pt}simp\ add{\isacharcolon}{\kern0pt}rank{\isacharunderscore}{\kern0pt}of{\isacharunderscore}{\kern0pt}def{\isacharparenright}{\kern0pt}\isanewline
\isacommand{next}\isamarkupfalse%
\isanewline
\ \ \isacommand{case}\isamarkupfalse%
\ False\isanewline
\ \ \isacommand{have}\isamarkupfalse%
\ {\isachardoublequoteopen}rank{\isacharunderscore}{\kern0pt}of\ x\ {\isacharparenleft}{\kern0pt}least\ m\ S{\isacharparenright}{\kern0pt}\ {\isacharequal}{\kern0pt}\ rank{\isacharunderscore}{\kern0pt}of\ x\ S{\isachardoublequoteclose}\ \isanewline
\ \ \ \ \isacommand{apply}\isamarkupfalse%
\ {\isacharparenleft}{\kern0pt}simp\ add{\isacharcolon}{\kern0pt}rank{\isacharunderscore}{\kern0pt}of{\isacharunderscore}{\kern0pt}def{\isacharparenright}{\kern0pt}\isanewline
\ \ \ \ \isacommand{apply}\isamarkupfalse%
\ {\isacharparenleft}{\kern0pt}rule\ arg{\isacharunderscore}{\kern0pt}cong{\isacharbrackleft}{\kern0pt}\isakeyword{where}\ f{\isacharequal}{\kern0pt}{\isachardoublequoteopen}card{\isachardoublequoteclose}{\isacharbrackright}{\kern0pt}{\isacharparenright}{\kern0pt}\isanewline
\ \ \ \ \isacommand{apply}\isamarkupfalse%
\ {\isacharparenleft}{\kern0pt}rule\ Collect{\isacharunderscore}{\kern0pt}cong{\isacharparenright}{\kern0pt}\isanewline
\ \ \ \ \isacommand{apply}\isamarkupfalse%
\ {\isacharparenleft}{\kern0pt}simp\ add{\isacharcolon}{\kern0pt}\ least{\isacharunderscore}{\kern0pt}def{\isacharparenright}{\kern0pt}\isanewline
\ \ \ \ \isacommand{by}\isamarkupfalse%
\ {\isacharparenleft}{\kern0pt}metis\ False\ rank{\isacharunderscore}{\kern0pt}mono{\isacharbrackleft}{\kern0pt}OF\ assms{\isacharbrackright}{\kern0pt}\ less{\isacharunderscore}{\kern0pt}le{\isacharunderscore}{\kern0pt}not{\isacharunderscore}{\kern0pt}le\ min{\isacharunderscore}{\kern0pt}def\ min{\isacharunderscore}{\kern0pt}less{\isacharunderscore}{\kern0pt}iff{\isacharunderscore}{\kern0pt}conj\ nle{\isacharunderscore}{\kern0pt}le{\isacharparenright}{\kern0pt}\isanewline
\ \ \isacommand{thus}\isamarkupfalse%
\ {\isacharquery}{\kern0pt}thesis\ \isacommand{using}\isamarkupfalse%
\ False\ \isacommand{by}\isamarkupfalse%
\ simp\isanewline
\isacommand{qed}\isamarkupfalse%
%
\endisatagproof
{\isafoldproof}%
%
\isadelimproof
\isanewline
%
\endisadelimproof
\isanewline
\isacommand{lemma}\isamarkupfalse%
\ rank{\isacharunderscore}{\kern0pt}insert{\isacharcolon}{\kern0pt}\isanewline
\ \ \isakeyword{assumes}\ {\isachardoublequoteopen}finite\ T{\isachardoublequoteclose}\isanewline
\ \ \isakeyword{shows}\ {\isachardoublequoteopen}rank{\isacharunderscore}{\kern0pt}of\ y\ {\isacharparenleft}{\kern0pt}insert\ v\ T{\isacharparenright}{\kern0pt}\ {\isacharequal}{\kern0pt}\ of{\isacharunderscore}{\kern0pt}bool\ {\isacharparenleft}{\kern0pt}v\ {\isacharless}{\kern0pt}\ y\ {\isasymand}\ v\ {\isasymnotin}\ T{\isacharparenright}{\kern0pt}\ {\isacharplus}{\kern0pt}\ rank{\isacharunderscore}{\kern0pt}of\ y\ T{\isachardoublequoteclose}\isanewline
%
\isadelimproof
%
\endisadelimproof
%
\isatagproof
\isacommand{proof}\isamarkupfalse%
\ {\isacharminus}{\kern0pt}\isanewline
\ \ \isacommand{have}\isamarkupfalse%
\ a{\isacharcolon}{\kern0pt}{\isachardoublequoteopen}v\ {\isasymnotin}\ T\ {\isasymLongrightarrow}\ v\ {\isacharless}{\kern0pt}\ y\ {\isasymLongrightarrow}\ rank{\isacharunderscore}{\kern0pt}of\ y\ {\isacharparenleft}{\kern0pt}insert\ v\ T{\isacharparenright}{\kern0pt}\ {\isacharequal}{\kern0pt}\ Suc\ {\isacharparenleft}{\kern0pt}rank{\isacharunderscore}{\kern0pt}of\ y\ T{\isacharparenright}{\kern0pt}{\isachardoublequoteclose}\isanewline
\ \ \isacommand{proof}\isamarkupfalse%
\ {\isacharminus}{\kern0pt}\isanewline
\ \ \ \ \isacommand{assume}\isamarkupfalse%
\ a{\isacharunderscore}{\kern0pt}{\isadigit{1}}{\isacharcolon}{\kern0pt}\ {\isachardoublequoteopen}v\ {\isasymnotin}\ T{\isachardoublequoteclose}\isanewline
\ \ \ \ \isacommand{assume}\isamarkupfalse%
\ a{\isacharunderscore}{\kern0pt}{\isadigit{2}}{\isacharcolon}{\kern0pt}\ {\isachardoublequoteopen}v\ {\isacharless}{\kern0pt}\ y{\isachardoublequoteclose}\isanewline
\ \ \ \ \isacommand{have}\isamarkupfalse%
\ {\isachardoublequoteopen}rank{\isacharunderscore}{\kern0pt}of\ y\ {\isacharparenleft}{\kern0pt}insert\ v\ T{\isacharparenright}{\kern0pt}\ {\isacharequal}{\kern0pt}\ card\ {\isacharparenleft}{\kern0pt}insert\ v\ {\isacharbraceleft}{\kern0pt}z\ {\isasymin}\ T{\isachardot}{\kern0pt}\ z\ {\isacharless}{\kern0pt}\ y{\isacharbraceright}{\kern0pt}{\isacharparenright}{\kern0pt}{\isachardoublequoteclose}\isanewline
\ \ \ \ \ \ \isacommand{apply}\isamarkupfalse%
\ {\isacharparenleft}{\kern0pt}simp\ add{\isacharcolon}{\kern0pt}\ rank{\isacharunderscore}{\kern0pt}of{\isacharunderscore}{\kern0pt}def{\isacharparenright}{\kern0pt}\ \isanewline
\ \ \ \ \ \ \isacommand{apply}\isamarkupfalse%
\ {\isacharparenleft}{\kern0pt}subst\ insert{\isacharunderscore}{\kern0pt}compr{\isacharparenright}{\kern0pt}\isanewline
\ \ \ \ \ \ \isacommand{by}\isamarkupfalse%
\ {\isacharparenleft}{\kern0pt}metis\ a{\isacharunderscore}{\kern0pt}{\isadigit{2}}\ mem{\isacharunderscore}{\kern0pt}Collect{\isacharunderscore}{\kern0pt}eq{\isacharparenright}{\kern0pt}\isanewline
\ \ \ \ \isacommand{also}\isamarkupfalse%
\ \isacommand{have}\isamarkupfalse%
\ {\isachardoublequoteopen}{\isachardot}{\kern0pt}{\isachardot}{\kern0pt}{\isachardot}{\kern0pt}\ {\isacharequal}{\kern0pt}\ Suc\ {\isacharparenleft}{\kern0pt}card\ {\isacharbraceleft}{\kern0pt}z\ {\isasymin}\ T{\isachardot}{\kern0pt}\ z\ {\isacharless}{\kern0pt}\ y{\isacharbraceright}{\kern0pt}{\isacharparenright}{\kern0pt}{\isachardoublequoteclose}\isanewline
\ \ \ \ \ \ \isacommand{apply}\isamarkupfalse%
\ {\isacharparenleft}{\kern0pt}subst\ card{\isacharunderscore}{\kern0pt}insert{\isacharunderscore}{\kern0pt}disjoint{\isacharparenright}{\kern0pt}\isanewline
\ \ \ \ \ \ \isacommand{using}\isamarkupfalse%
\ assms\ a{\isacharunderscore}{\kern0pt}{\isadigit{1}}\ \isacommand{by}\isamarkupfalse%
\ simp{\isacharplus}{\kern0pt}\isanewline
\ \ \ \ \isacommand{also}\isamarkupfalse%
\ \isacommand{have}\isamarkupfalse%
\ {\isachardoublequoteopen}{\isachardot}{\kern0pt}{\isachardot}{\kern0pt}{\isachardot}{\kern0pt}\ {\isacharequal}{\kern0pt}\ Suc\ {\isacharparenleft}{\kern0pt}rank{\isacharunderscore}{\kern0pt}of\ y\ T{\isacharparenright}{\kern0pt}{\isachardoublequoteclose}\isanewline
\ \ \ \ \ \ \isacommand{by}\isamarkupfalse%
\ {\isacharparenleft}{\kern0pt}simp\ add{\isacharcolon}{\kern0pt}rank{\isacharunderscore}{\kern0pt}of{\isacharunderscore}{\kern0pt}def{\isacharparenright}{\kern0pt}\ \isanewline
\ \ \ \ \isacommand{finally}\isamarkupfalse%
\ \isacommand{show}\isamarkupfalse%
\ {\isachardoublequoteopen}rank{\isacharunderscore}{\kern0pt}of\ y\ {\isacharparenleft}{\kern0pt}insert\ v\ T{\isacharparenright}{\kern0pt}\ {\isacharequal}{\kern0pt}\ Suc\ {\isacharparenleft}{\kern0pt}rank{\isacharunderscore}{\kern0pt}of\ y\ T{\isacharparenright}{\kern0pt}{\isachardoublequoteclose}\isanewline
\ \ \ \ \ \ \isacommand{by}\isamarkupfalse%
\ blast\isanewline
\ \ \isacommand{qed}\isamarkupfalse%
\isanewline
\ \ \isacommand{have}\isamarkupfalse%
\ b{\isacharcolon}{\kern0pt}{\isachardoublequoteopen}v\ {\isasymnotin}\ T\ {\isasymLongrightarrow}\ {\isasymnot}{\isacharparenleft}{\kern0pt}v\ {\isacharless}{\kern0pt}\ y{\isacharparenright}{\kern0pt}\ {\isasymLongrightarrow}\ rank{\isacharunderscore}{\kern0pt}of\ y\ {\isacharparenleft}{\kern0pt}insert\ v\ T{\isacharparenright}{\kern0pt}\ {\isacharequal}{\kern0pt}\ rank{\isacharunderscore}{\kern0pt}of\ y\ T{\isachardoublequoteclose}\isanewline
\ \ \ \ \isacommand{by}\isamarkupfalse%
\ {\isacharparenleft}{\kern0pt}simp\ add{\isacharcolon}{\kern0pt}rank{\isacharunderscore}{\kern0pt}of{\isacharunderscore}{\kern0pt}def{\isacharcomma}{\kern0pt}\ metis{\isacharparenright}{\kern0pt}\isanewline
\ \ \isacommand{have}\isamarkupfalse%
\ c{\isacharcolon}{\kern0pt}{\isachardoublequoteopen}v\ {\isasymin}\ T\ {\isasymLongrightarrow}\ rank{\isacharunderscore}{\kern0pt}of\ y\ {\isacharparenleft}{\kern0pt}insert\ v\ T{\isacharparenright}{\kern0pt}\ {\isacharequal}{\kern0pt}\ rank{\isacharunderscore}{\kern0pt}of\ y\ T{\isachardoublequoteclose}\isanewline
\ \ \ \ \isacommand{by}\isamarkupfalse%
\ {\isacharparenleft}{\kern0pt}simp\ add{\isacharcolon}{\kern0pt}insert{\isacharunderscore}{\kern0pt}absorb{\isacharparenright}{\kern0pt}\isanewline
\isanewline
\ \ \isacommand{show}\isamarkupfalse%
\ {\isacharquery}{\kern0pt}thesis\isanewline
\ \ \ \ \isacommand{apply}\isamarkupfalse%
\ {\isacharparenleft}{\kern0pt}cases\ {\isachardoublequoteopen}v\ {\isasymin}\ T{\isachardoublequoteclose}{\isacharcomma}{\kern0pt}\ simp\ add{\isacharcolon}{\kern0pt}\ c{\isacharparenright}{\kern0pt}\isanewline
\ \ \ \ \isacommand{apply}\isamarkupfalse%
\ {\isacharparenleft}{\kern0pt}cases\ {\isachardoublequoteopen}v\ {\isacharless}{\kern0pt}\ y{\isachardoublequoteclose}{\isacharcomma}{\kern0pt}\ simp\ add{\isacharcolon}{\kern0pt}a{\isacharparenright}{\kern0pt}\isanewline
\ \ \ \ \isacommand{by}\isamarkupfalse%
\ {\isacharparenleft}{\kern0pt}simp\ add{\isacharcolon}{\kern0pt}b{\isacharparenright}{\kern0pt}\isanewline
\isacommand{qed}\isamarkupfalse%
%
\endisatagproof
{\isafoldproof}%
%
\isadelimproof
\isanewline
%
\endisadelimproof
\isanewline
\isacommand{lemma}\isamarkupfalse%
\ least{\isacharunderscore}{\kern0pt}mono{\isacharunderscore}{\kern0pt}commute{\isacharcolon}{\kern0pt}\isanewline
\ \ \isakeyword{assumes}\ {\isachardoublequoteopen}finite\ S{\isachardoublequoteclose}\isanewline
\ \ \isakeyword{assumes}\ {\isachardoublequoteopen}strict{\isacharunderscore}{\kern0pt}mono{\isacharunderscore}{\kern0pt}on\ f\ S{\isachardoublequoteclose}\isanewline
\ \ \isakeyword{shows}\ {\isachardoublequoteopen}f\ {\isacharbackquote}{\kern0pt}\ least\ k\ S\ {\isacharequal}{\kern0pt}\ least\ k\ {\isacharparenleft}{\kern0pt}f\ {\isacharbackquote}{\kern0pt}\ S{\isacharparenright}{\kern0pt}{\isachardoublequoteclose}\isanewline
%
\isadelimproof
%
\endisadelimproof
%
\isatagproof
\isacommand{proof}\isamarkupfalse%
\ {\isacharminus}{\kern0pt}\isanewline
\ \ \isacommand{have}\isamarkupfalse%
\ a{\isacharcolon}{\kern0pt}{\isachardoublequoteopen}inj{\isacharunderscore}{\kern0pt}on\ f\ S{\isachardoublequoteclose}\ \isanewline
\ \ \ \ \isacommand{using}\isamarkupfalse%
\ strict{\isacharunderscore}{\kern0pt}mono{\isacharunderscore}{\kern0pt}on{\isacharunderscore}{\kern0pt}imp{\isacharunderscore}{\kern0pt}inj{\isacharunderscore}{\kern0pt}on{\isacharbrackleft}{\kern0pt}OF\ assms{\isacharparenleft}{\kern0pt}{\isadigit{2}}{\isacharparenright}{\kern0pt}{\isacharbrackright}{\kern0pt}\ \isacommand{by}\isamarkupfalse%
\ simp\isanewline
\ \ \isacommand{have}\isamarkupfalse%
\ b{\isacharcolon}{\kern0pt}\ {\isachardoublequoteopen}card\ {\isacharparenleft}{\kern0pt}least\ k\ {\isacharparenleft}{\kern0pt}f\ {\isacharbackquote}{\kern0pt}\ S{\isacharparenright}{\kern0pt}{\isacharparenright}{\kern0pt}\ {\isasymle}\ card\ {\isacharparenleft}{\kern0pt}f\ {\isacharbackquote}{\kern0pt}\ least\ k\ S{\isacharparenright}{\kern0pt}{\isachardoublequoteclose}\isanewline
\ \ \ \ \isacommand{apply}\isamarkupfalse%
\ {\isacharparenleft}{\kern0pt}subst\ card{\isacharunderscore}{\kern0pt}least{\isacharcomma}{\kern0pt}\ simp\ add{\isacharcolon}{\kern0pt}assms{\isacharparenright}{\kern0pt}\isanewline
\ \ \ \ \isacommand{apply}\isamarkupfalse%
\ {\isacharparenleft}{\kern0pt}subst\ card{\isacharunderscore}{\kern0pt}image{\isacharcomma}{\kern0pt}\ metis\ a{\isacharparenright}{\kern0pt}\isanewline
\ \ \ \ \isacommand{apply}\isamarkupfalse%
\ {\isacharparenleft}{\kern0pt}subst\ card{\isacharunderscore}{\kern0pt}image{\isacharcomma}{\kern0pt}\ rule\ inj{\isacharunderscore}{\kern0pt}on{\isacharunderscore}{\kern0pt}subset{\isacharbrackleft}{\kern0pt}OF\ a{\isacharbrackright}{\kern0pt}{\isacharcomma}{\kern0pt}\ simp\ add{\isacharcolon}{\kern0pt}least{\isacharunderscore}{\kern0pt}def{\isacharparenright}{\kern0pt}\isanewline
\ \ \ \ \isacommand{by}\isamarkupfalse%
\ {\isacharparenleft}{\kern0pt}subst\ card{\isacharunderscore}{\kern0pt}least{\isacharcomma}{\kern0pt}\ simp\ add{\isacharcolon}{\kern0pt}assms{\isacharcomma}{\kern0pt}\ simp{\isacharparenright}{\kern0pt}\isanewline
\isanewline
\ \ \isacommand{show}\isamarkupfalse%
\ {\isacharquery}{\kern0pt}thesis\isanewline
\ \ \ \ \isacommand{apply}\isamarkupfalse%
\ {\isacharparenleft}{\kern0pt}rule\ card{\isacharunderscore}{\kern0pt}seteq{\isacharcomma}{\kern0pt}\ simp\ add{\isacharcolon}{\kern0pt}least{\isacharunderscore}{\kern0pt}def\ assms{\isacharparenright}{\kern0pt}\isanewline
\ \ \ \ \ \isacommand{apply}\isamarkupfalse%
\ {\isacharparenleft}{\kern0pt}rule\ image{\isacharunderscore}{\kern0pt}subsetI{\isacharcomma}{\kern0pt}\ simp\ add{\isacharcolon}{\kern0pt}least{\isacharunderscore}{\kern0pt}def{\isacharparenright}{\kern0pt}\isanewline
\ \ \ \ \ \isacommand{apply}\isamarkupfalse%
\ {\isacharparenleft}{\kern0pt}subst\ rank{\isacharunderscore}{\kern0pt}mono{\isacharunderscore}{\kern0pt}commute{\isacharbrackleft}{\kern0pt}symmetric{\isacharcomma}{\kern0pt}\ \isakeyword{where}\ T{\isacharequal}{\kern0pt}{\isachardoublequoteopen}S{\isachardoublequoteclose}{\isacharbrackright}{\kern0pt}{\isacharcomma}{\kern0pt}\ metis\ assms{\isacharparenleft}{\kern0pt}{\isadigit{1}}{\isacharparenright}{\kern0pt}{\isacharcomma}{\kern0pt}\ simp{\isacharcomma}{\kern0pt}\ metis\ assms{\isacharparenleft}{\kern0pt}{\isadigit{2}}{\isacharparenright}{\kern0pt}{\isacharcomma}{\kern0pt}\ simp{\isacharcomma}{\kern0pt}\ simp{\isacharparenright}{\kern0pt}\isanewline
\ \ \ \ \isacommand{by}\isamarkupfalse%
\ {\isacharparenleft}{\kern0pt}metis\ b{\isacharparenright}{\kern0pt}\isanewline
\isacommand{qed}\isamarkupfalse%
%
\endisatagproof
{\isafoldproof}%
%
\isadelimproof
\isanewline
%
\endisadelimproof
\isanewline
\isacommand{lemma}\isamarkupfalse%
\ least{\isacharunderscore}{\kern0pt}insert{\isacharcolon}{\kern0pt}\ \isanewline
\ \ \isakeyword{assumes}\ {\isachardoublequoteopen}finite\ S{\isachardoublequoteclose}\isanewline
\ \ \isakeyword{shows}\ {\isachardoublequoteopen}least\ k\ {\isacharparenleft}{\kern0pt}insert\ x\ {\isacharparenleft}{\kern0pt}least\ k\ S{\isacharparenright}{\kern0pt}{\isacharparenright}{\kern0pt}\ {\isacharequal}{\kern0pt}\ least\ k\ {\isacharparenleft}{\kern0pt}insert\ x\ S{\isacharparenright}{\kern0pt}{\isachardoublequoteclose}\ {\isacharparenleft}{\kern0pt}\isakeyword{is}\ {\isachardoublequoteopen}{\isacharquery}{\kern0pt}lhs\ {\isacharequal}{\kern0pt}\ {\isacharquery}{\kern0pt}rhs{\isachardoublequoteclose}{\isacharparenright}{\kern0pt}\isanewline
%
\isadelimproof
%
\endisadelimproof
%
\isatagproof
\isacommand{proof}\isamarkupfalse%
\ {\isacharminus}{\kern0pt}\isanewline
\ \ \isacommand{have}\isamarkupfalse%
\ c{\isacharcolon}{\kern0pt}\ {\isachardoublequoteopen}x\ {\isasymin}\ least\ k\ S\ {\isasymLongrightarrow}\ x\ {\isasymin}\ S{\isachardoublequoteclose}\ \isacommand{by}\isamarkupfalse%
\ {\isacharparenleft}{\kern0pt}simp\ add{\isacharcolon}{\kern0pt}least{\isacharunderscore}{\kern0pt}def{\isacharparenright}{\kern0pt}\isanewline
\ \ \isacommand{have}\isamarkupfalse%
\ b{\isacharcolon}{\kern0pt}{\isachardoublequoteopen}min\ k\ {\isacharparenleft}{\kern0pt}card\ {\isacharparenleft}{\kern0pt}insert\ x\ S{\isacharparenright}{\kern0pt}{\isacharparenright}{\kern0pt}\ {\isasymle}\ card\ {\isacharparenleft}{\kern0pt}insert\ x\ {\isacharparenleft}{\kern0pt}least\ k\ S{\isacharparenright}{\kern0pt}{\isacharparenright}{\kern0pt}{\isachardoublequoteclose}\isanewline
\ \ \ \ \isacommand{apply}\isamarkupfalse%
\ {\isacharparenleft}{\kern0pt}cases\ {\isachardoublequoteopen}x\ {\isasymin}\ least\ k\ S{\isachardoublequoteclose}{\isacharparenright}{\kern0pt}\isanewline
\ \ \ \ \ \isacommand{using}\isamarkupfalse%
\ c\ \isacommand{apply}\isamarkupfalse%
\ {\isacharparenleft}{\kern0pt}simp\ add{\isacharcolon}{\kern0pt}\ insert{\isacharunderscore}{\kern0pt}absorb{\isacharparenright}{\kern0pt}\isanewline
\ \ \ \ \ \isacommand{apply}\isamarkupfalse%
\ {\isacharparenleft}{\kern0pt}subst\ card{\isacharunderscore}{\kern0pt}least{\isacharcomma}{\kern0pt}\ simp\ add{\isacharcolon}{\kern0pt}assms\ least{\isacharunderscore}{\kern0pt}def{\isacharcomma}{\kern0pt}\ simp{\isacharparenright}{\kern0pt}\isanewline
\ \ \ \ \isacommand{apply}\isamarkupfalse%
\ {\isacharparenleft}{\kern0pt}subst\ card{\isacharunderscore}{\kern0pt}insert{\isacharunderscore}{\kern0pt}disjoint{\isacharcomma}{\kern0pt}\ simp\ add{\isacharcolon}{\kern0pt}assms\ least{\isacharunderscore}{\kern0pt}def{\isacharcomma}{\kern0pt}\ simp{\isacharparenright}{\kern0pt}\isanewline
\ \ \ \ \isacommand{apply}\isamarkupfalse%
\ {\isacharparenleft}{\kern0pt}cases\ {\isachardoublequoteopen}x\ {\isasymin}\ S{\isachardoublequoteclose}{\isacharparenright}{\kern0pt}\isanewline
\ \ \ \ \ \isacommand{apply}\isamarkupfalse%
\ {\isacharparenleft}{\kern0pt}simp\ add{\isacharcolon}{\kern0pt}insert{\isacharunderscore}{\kern0pt}absorb{\isacharparenright}{\kern0pt}\isanewline
\ \ \ \ \ \isacommand{apply}\isamarkupfalse%
\ {\isacharparenleft}{\kern0pt}subst\ card{\isacharunderscore}{\kern0pt}least{\isacharcomma}{\kern0pt}\ simp\ add{\isacharcolon}{\kern0pt}assms\ least{\isacharunderscore}{\kern0pt}def{\isacharparenright}{\kern0pt}\isanewline
\ \ \ \ \ \isacommand{using}\isamarkupfalse%
\ nat{\isacharunderscore}{\kern0pt}less{\isacharunderscore}{\kern0pt}le\ \isacommand{apply}\isamarkupfalse%
\ blast\isanewline
\ \ \ \ \isacommand{apply}\isamarkupfalse%
\ {\isacharparenleft}{\kern0pt}subst\ card{\isacharunderscore}{\kern0pt}insert{\isacharunderscore}{\kern0pt}disjoint{\isacharcomma}{\kern0pt}\ simp\ add{\isacharcolon}{\kern0pt}assms\ least{\isacharunderscore}{\kern0pt}def{\isacharcomma}{\kern0pt}\ simp{\isacharparenright}{\kern0pt}\isanewline
\ \ \ \ \isacommand{apply}\isamarkupfalse%
\ {\isacharparenleft}{\kern0pt}subst\ card{\isacharunderscore}{\kern0pt}least{\isacharcomma}{\kern0pt}\ simp\ add{\isacharcolon}{\kern0pt}assms\ least{\isacharunderscore}{\kern0pt}def{\isacharparenright}{\kern0pt}\ \isanewline
\ \ \ \ \isacommand{by}\isamarkupfalse%
\ simp\isanewline
\ \ \isacommand{have}\isamarkupfalse%
\ a{\isacharcolon}{\kern0pt}{\isachardoublequoteopen}card\ {\isacharquery}{\kern0pt}rhs\ {\isasymle}\ card\ {\isacharquery}{\kern0pt}lhs{\isachardoublequoteclose}\isanewline
\ \ \ \ \isacommand{apply}\isamarkupfalse%
\ {\isacharparenleft}{\kern0pt}subst\ card{\isacharunderscore}{\kern0pt}least{\isacharcomma}{\kern0pt}\ simp\ add{\isacharcolon}{\kern0pt}assms\ least{\isacharunderscore}{\kern0pt}def{\isacharparenright}{\kern0pt}\isanewline
\ \ \ \ \isacommand{apply}\isamarkupfalse%
\ {\isacharparenleft}{\kern0pt}subst\ card{\isacharunderscore}{\kern0pt}least{\isacharcomma}{\kern0pt}\ simp\ add{\isacharcolon}{\kern0pt}assms\ least{\isacharunderscore}{\kern0pt}def{\isacharparenright}{\kern0pt}\isanewline
\ \ \ \ \isacommand{by}\isamarkupfalse%
\ {\isacharparenleft}{\kern0pt}meson\ b\ min{\isachardot}{\kern0pt}boundedI\ min{\isachardot}{\kern0pt}cobounded{\isadigit{1}}{\isacharparenright}{\kern0pt}\isanewline
\isanewline
\ \ \isacommand{have}\isamarkupfalse%
\ d{\isacharcolon}{\kern0pt}{\isachardoublequoteopen}{\isasymAnd}y{\isachardot}{\kern0pt}\ y\ {\isasymin}\ least\ k\ {\isacharparenleft}{\kern0pt}insert\ x\ {\isacharparenleft}{\kern0pt}least\ k\ S{\isacharparenright}{\kern0pt}{\isacharparenright}{\kern0pt}\ {\isasymLongrightarrow}\ y\ {\isasymin}\ least\ k\ {\isacharparenleft}{\kern0pt}insert\ x\ S{\isacharparenright}{\kern0pt}{\isachardoublequoteclose}\ \isanewline
\ \ \ \ \isacommand{apply}\isamarkupfalse%
\ {\isacharparenleft}{\kern0pt}subst\ least{\isacharunderscore}{\kern0pt}def{\isacharcomma}{\kern0pt}\ subst\ {\isacharparenleft}{\kern0pt}asm{\isacharparenright}{\kern0pt}\ least{\isacharunderscore}{\kern0pt}def{\isacharparenright}{\kern0pt}\isanewline
\ \ \ \ \isacommand{apply}\isamarkupfalse%
\ {\isacharparenleft}{\kern0pt}subst\ rank{\isacharunderscore}{\kern0pt}insert{\isacharbrackleft}{\kern0pt}OF\ assms{\isacharbrackright}{\kern0pt}{\isacharparenright}{\kern0pt}\isanewline
\ \ \ \ \isacommand{apply}\isamarkupfalse%
\ {\isacharparenleft}{\kern0pt}subst\ {\isacharparenleft}{\kern0pt}asm{\isacharparenright}{\kern0pt}\ rank{\isacharunderscore}{\kern0pt}insert{\isacharcomma}{\kern0pt}\ simp\ add{\isacharcolon}{\kern0pt}assms\ least{\isacharunderscore}{\kern0pt}def{\isacharparenright}{\kern0pt}\isanewline
\ \ \ \ \isacommand{apply}\isamarkupfalse%
\ {\isacharparenleft}{\kern0pt}subst\ {\isacharparenleft}{\kern0pt}asm{\isacharparenright}{\kern0pt}\ preserve{\isacharunderscore}{\kern0pt}rank{\isacharcomma}{\kern0pt}\ simp\ add{\isacharcolon}{\kern0pt}assms{\isacharparenright}{\kern0pt}\isanewline
\ \ \ \ \isacommand{apply}\isamarkupfalse%
\ {\isacharparenleft}{\kern0pt}cases\ {\isachardoublequoteopen}x\ {\isasymin}\ least\ k\ S{\isachardoublequoteclose}{\isacharparenright}{\kern0pt}\isanewline
\ \ \ \ \isacommand{apply}\isamarkupfalse%
\ {\isacharparenleft}{\kern0pt}simp{\isacharcomma}{\kern0pt}\ metis\ insert{\isacharunderscore}{\kern0pt}subset\ least{\isacharunderscore}{\kern0pt}subset\ min{\isachardot}{\kern0pt}strict{\isacharunderscore}{\kern0pt}order{\isacharunderscore}{\kern0pt}iff\ min{\isacharunderscore}{\kern0pt}def\ mk{\isacharunderscore}{\kern0pt}disjoint{\isacharunderscore}{\kern0pt}insert{\isacharparenright}{\kern0pt}\isanewline
\ \ \ \ \isacommand{apply}\isamarkupfalse%
\ {\isacharparenleft}{\kern0pt}simp{\isacharparenright}{\kern0pt}\ \isanewline
\ \ \ \ \ \isacommand{using}\isamarkupfalse%
\ least{\isacharunderscore}{\kern0pt}def\ \isacommand{apply}\isamarkupfalse%
\ fastforce\isanewline
\ \ \ \ \isacommand{by}\isamarkupfalse%
\ {\isacharparenleft}{\kern0pt}metis\ insert{\isacharunderscore}{\kern0pt}subset\ least{\isacharunderscore}{\kern0pt}subset\ min{\isacharunderscore}{\kern0pt}def\ mk{\isacharunderscore}{\kern0pt}disjoint{\isacharunderscore}{\kern0pt}insert\ nat{\isacharunderscore}{\kern0pt}neq{\isacharunderscore}{\kern0pt}iff{\isacharparenright}{\kern0pt}\isanewline
\ \ \isanewline
\ \ \isacommand{show}\isamarkupfalse%
\ {\isacharquery}{\kern0pt}thesis\isanewline
\ \ \ \ \isacommand{apply}\isamarkupfalse%
\ {\isacharparenleft}{\kern0pt}rule\ card{\isacharunderscore}{\kern0pt}seteq{\isacharcomma}{\kern0pt}\ simp\ add{\isacharcolon}{\kern0pt}least{\isacharunderscore}{\kern0pt}def\ assms{\isacharparenright}{\kern0pt}\isanewline
\ \ \ \ \ \isacommand{apply}\isamarkupfalse%
\ {\isacharparenleft}{\kern0pt}rule\ subsetI{\isacharcomma}{\kern0pt}\ metis\ d{\isacharparenright}{\kern0pt}\isanewline
\ \ \ \ \isacommand{using}\isamarkupfalse%
\ a\ \isacommand{by}\isamarkupfalse%
\ simp\isanewline
\isacommand{qed}\isamarkupfalse%
%
\endisatagproof
{\isafoldproof}%
%
\isadelimproof
\isanewline
%
\endisadelimproof
\isanewline
\isacommand{definition}\isamarkupfalse%
\ count{\isacharunderscore}{\kern0pt}le\ \isakeyword{where}\ {\isachardoublequoteopen}count{\isacharunderscore}{\kern0pt}le\ x\ M\ {\isacharequal}{\kern0pt}\ size\ {\isacharbraceleft}{\kern0pt}{\isacharhash}{\kern0pt}y\ {\isasymin}{\isacharhash}{\kern0pt}\ M{\isachardot}{\kern0pt}\ y\ {\isasymle}\ x{\isacharhash}{\kern0pt}{\isacharbraceright}{\kern0pt}{\isachardoublequoteclose}\isanewline
\isacommand{definition}\isamarkupfalse%
\ count{\isacharunderscore}{\kern0pt}less\ \isakeyword{where}\ {\isachardoublequoteopen}count{\isacharunderscore}{\kern0pt}less\ x\ M\ {\isacharequal}{\kern0pt}\ size\ {\isacharbraceleft}{\kern0pt}{\isacharhash}{\kern0pt}y\ {\isasymin}{\isacharhash}{\kern0pt}\ M{\isachardot}{\kern0pt}\ y\ {\isacharless}{\kern0pt}\ x{\isacharhash}{\kern0pt}{\isacharbraceright}{\kern0pt}{\isachardoublequoteclose}\isanewline
\isanewline
\isacommand{definition}\isamarkupfalse%
\ nth{\isacharunderscore}{\kern0pt}mset\ {\isacharcolon}{\kern0pt}{\isacharcolon}{\kern0pt}\ {\isachardoublequoteopen}nat\ {\isasymRightarrow}\ {\isacharparenleft}{\kern0pt}{\isacharprime}{\kern0pt}a\ {\isacharcolon}{\kern0pt}{\isacharcolon}{\kern0pt}\ linorder{\isacharparenright}{\kern0pt}\ multiset\ {\isasymRightarrow}\ {\isacharprime}{\kern0pt}a{\isachardoublequoteclose}\ \isakeyword{where}\isanewline
\ \ {\isachardoublequoteopen}nth{\isacharunderscore}{\kern0pt}mset\ k\ M\ {\isacharequal}{\kern0pt}\ sorted{\isacharunderscore}{\kern0pt}list{\isacharunderscore}{\kern0pt}of{\isacharunderscore}{\kern0pt}multiset\ M\ {\isacharbang}{\kern0pt}\ k{\isachardoublequoteclose}\isanewline
\isanewline
\isacommand{lemma}\isamarkupfalse%
\ nth{\isacharunderscore}{\kern0pt}mset{\isacharunderscore}{\kern0pt}bound{\isacharunderscore}{\kern0pt}left{\isacharcolon}{\kern0pt}\isanewline
\ \ \isakeyword{assumes}\ {\isachardoublequoteopen}k\ {\isacharless}{\kern0pt}\ size\ M{\isachardoublequoteclose}\isanewline
\ \ \isakeyword{assumes}\ {\isachardoublequoteopen}count{\isacharunderscore}{\kern0pt}less\ x\ M\ {\isasymle}\ k{\isachardoublequoteclose}\isanewline
\ \ \isakeyword{shows}\ {\isachardoublequoteopen}x\ {\isasymle}\ nth{\isacharunderscore}{\kern0pt}mset\ k\ M{\isachardoublequoteclose}\isanewline
%
\isadelimproof
%
\endisadelimproof
%
\isatagproof
\isacommand{proof}\isamarkupfalse%
\ {\isacharparenleft}{\kern0pt}rule\ ccontr{\isacharparenright}{\kern0pt}\isanewline
\ \ \isacommand{define}\isamarkupfalse%
\ xs\ \isakeyword{where}\ {\isachardoublequoteopen}xs\ {\isacharequal}{\kern0pt}\ sorted{\isacharunderscore}{\kern0pt}list{\isacharunderscore}{\kern0pt}of{\isacharunderscore}{\kern0pt}multiset\ M{\isachardoublequoteclose}\isanewline
\ \ \isacommand{have}\isamarkupfalse%
\ s{\isacharunderscore}{\kern0pt}xs{\isacharcolon}{\kern0pt}\ {\isachardoublequoteopen}sorted\ xs{\isachardoublequoteclose}\ \isacommand{by}\isamarkupfalse%
\ {\isacharparenleft}{\kern0pt}simp\ add{\isacharcolon}{\kern0pt}xs{\isacharunderscore}{\kern0pt}def\ sorted{\isacharunderscore}{\kern0pt}sorted{\isacharunderscore}{\kern0pt}list{\isacharunderscore}{\kern0pt}of{\isacharunderscore}{\kern0pt}multiset{\isacharparenright}{\kern0pt}\isanewline
\ \ \isacommand{have}\isamarkupfalse%
\ l{\isacharunderscore}{\kern0pt}xs{\isacharcolon}{\kern0pt}\ {\isachardoublequoteopen}k\ {\isacharless}{\kern0pt}\ length\ xs{\isachardoublequoteclose}\ \isacommand{apply}\isamarkupfalse%
\ {\isacharparenleft}{\kern0pt}simp\ add{\isacharcolon}{\kern0pt}xs{\isacharunderscore}{\kern0pt}def{\isacharparenright}{\kern0pt}\ \isanewline
\ \ \ \ \isacommand{by}\isamarkupfalse%
\ {\isacharparenleft}{\kern0pt}metis\ size{\isacharunderscore}{\kern0pt}mset\ mset{\isacharunderscore}{\kern0pt}sorted{\isacharunderscore}{\kern0pt}list{\isacharunderscore}{\kern0pt}of{\isacharunderscore}{\kern0pt}multiset\ assms{\isacharparenleft}{\kern0pt}{\isadigit{1}}{\isacharparenright}{\kern0pt}{\isacharparenright}{\kern0pt}\ \ \isanewline
\ \ \isacommand{have}\isamarkupfalse%
\ M{\isacharunderscore}{\kern0pt}xs{\isacharcolon}{\kern0pt}\ {\isachardoublequoteopen}M\ {\isacharequal}{\kern0pt}\ mset\ xs{\isachardoublequoteclose}\ \isacommand{by}\isamarkupfalse%
\ {\isacharparenleft}{\kern0pt}simp\ add{\isacharcolon}{\kern0pt}xs{\isacharunderscore}{\kern0pt}def{\isacharparenright}{\kern0pt}\isanewline
\ \ \isacommand{hence}\isamarkupfalse%
\ a{\isacharcolon}{\kern0pt}{\isachardoublequoteopen}{\isasymAnd}i{\isachardot}{\kern0pt}\ i\ {\isasymle}\ k\ {\isasymLongrightarrow}\ xs\ {\isacharbang}{\kern0pt}\ i\ {\isasymle}\ xs\ {\isacharbang}{\kern0pt}\ k{\isachardoublequoteclose}\isanewline
\ \ \ \ \isacommand{using}\isamarkupfalse%
\ s{\isacharunderscore}{\kern0pt}xs\ l{\isacharunderscore}{\kern0pt}xs\ sorted{\isacharunderscore}{\kern0pt}iff{\isacharunderscore}{\kern0pt}nth{\isacharunderscore}{\kern0pt}mono\ \isacommand{by}\isamarkupfalse%
\ blast\isanewline
\isanewline
\ \ \isacommand{assume}\isamarkupfalse%
\ {\isachardoublequoteopen}{\isasymnot}{\isacharparenleft}{\kern0pt}x\ {\isasymle}\ nth{\isacharunderscore}{\kern0pt}mset\ k\ M{\isacharparenright}{\kern0pt}{\isachardoublequoteclose}\isanewline
\ \ \isacommand{hence}\isamarkupfalse%
\ {\isachardoublequoteopen}x\ {\isachargreater}{\kern0pt}\ nth{\isacharunderscore}{\kern0pt}mset\ k\ M{\isachardoublequoteclose}\ \isacommand{by}\isamarkupfalse%
\ simp\isanewline
\ \ \isacommand{hence}\isamarkupfalse%
\ b{\isacharcolon}{\kern0pt}{\isachardoublequoteopen}x\ {\isachargreater}{\kern0pt}\ xs\ {\isacharbang}{\kern0pt}\ k{\isachardoublequoteclose}\ \isacommand{by}\isamarkupfalse%
\ {\isacharparenleft}{\kern0pt}simp\ add{\isacharcolon}{\kern0pt}nth{\isacharunderscore}{\kern0pt}mset{\isacharunderscore}{\kern0pt}def\ xs{\isacharunderscore}{\kern0pt}def{\isacharbrackleft}{\kern0pt}symmetric{\isacharbrackright}{\kern0pt}{\isacharparenright}{\kern0pt}\isanewline
\isanewline
\ \ \isacommand{have}\isamarkupfalse%
\ {\isachardoublequoteopen}k\ {\isacharless}{\kern0pt}\ card\ {\isacharbraceleft}{\kern0pt}{\isadigit{0}}{\isachardot}{\kern0pt}{\isachardot}{\kern0pt}k{\isacharbraceright}{\kern0pt}{\isachardoublequoteclose}\ \isacommand{by}\isamarkupfalse%
\ simp\isanewline
\ \ \isacommand{also}\isamarkupfalse%
\ \isacommand{have}\isamarkupfalse%
\ {\isachardoublequoteopen}{\isachardot}{\kern0pt}{\isachardot}{\kern0pt}{\isachardot}{\kern0pt}\ {\isasymle}\ card\ {\isacharbraceleft}{\kern0pt}i{\isachardot}{\kern0pt}\ i\ {\isacharless}{\kern0pt}\ length\ xs\ {\isasymand}\ xs\ {\isacharbang}{\kern0pt}\ i\ {\isacharless}{\kern0pt}\ x{\isacharbraceright}{\kern0pt}{\isachardoublequoteclose}\isanewline
\ \ \ \ \isacommand{apply}\isamarkupfalse%
\ {\isacharparenleft}{\kern0pt}rule\ card{\isacharunderscore}{\kern0pt}mono{\isacharcomma}{\kern0pt}\ simp{\isacharparenright}{\kern0pt}\isanewline
\ \ \ \ \isacommand{apply}\isamarkupfalse%
\ {\isacharparenleft}{\kern0pt}rule\ subsetI{\isacharcomma}{\kern0pt}\ simp{\isacharparenright}{\kern0pt}\isanewline
\ \ \ \ \isacommand{using}\isamarkupfalse%
\ a\ b\ l{\isacharunderscore}{\kern0pt}xs\ order{\isacharunderscore}{\kern0pt}le{\isacharunderscore}{\kern0pt}less{\isacharunderscore}{\kern0pt}trans\ \isacommand{by}\isamarkupfalse%
\ auto\isanewline
\ \ \isacommand{also}\isamarkupfalse%
\ \isacommand{have}\isamarkupfalse%
\ {\isachardoublequoteopen}{\isachardot}{\kern0pt}{\isachardot}{\kern0pt}{\isachardot}{\kern0pt}\ {\isacharequal}{\kern0pt}\ count{\isacharunderscore}{\kern0pt}less\ x\ M{\isachardoublequoteclose}\isanewline
\ \ \ \ \isacommand{apply}\isamarkupfalse%
\ {\isacharparenleft}{\kern0pt}simp\ add{\isacharcolon}{\kern0pt}count{\isacharunderscore}{\kern0pt}less{\isacharunderscore}{\kern0pt}def\ M{\isacharunderscore}{\kern0pt}xs{\isacharparenright}{\kern0pt}\isanewline
\ \ \ \ \isacommand{apply}\isamarkupfalse%
\ {\isacharparenleft}{\kern0pt}subst\ mset{\isacharunderscore}{\kern0pt}filter{\isacharbrackleft}{\kern0pt}symmetric{\isacharbrackright}{\kern0pt}{\isacharcomma}{\kern0pt}\ subst\ size{\isacharunderscore}{\kern0pt}mset{\isacharparenright}{\kern0pt}\isanewline
\ \ \ \ \isacommand{by}\isamarkupfalse%
\ {\isacharparenleft}{\kern0pt}subst\ length{\isacharunderscore}{\kern0pt}filter{\isacharunderscore}{\kern0pt}conv{\isacharunderscore}{\kern0pt}card{\isacharcomma}{\kern0pt}\ simp{\isacharparenright}{\kern0pt}\isanewline
\ \ \isacommand{also}\isamarkupfalse%
\ \isacommand{have}\isamarkupfalse%
\ {\isachardoublequoteopen}{\isachardot}{\kern0pt}{\isachardot}{\kern0pt}{\isachardot}{\kern0pt}\ {\isasymle}\ k{\isachardoublequoteclose}\isanewline
\ \ \ \ \isacommand{using}\isamarkupfalse%
\ assms\ \isacommand{by}\isamarkupfalse%
\ simp\isanewline
\ \ \isacommand{finally}\isamarkupfalse%
\ \isacommand{show}\isamarkupfalse%
\ {\isachardoublequoteopen}False{\isachardoublequoteclose}\ \isacommand{by}\isamarkupfalse%
\ simp\isanewline
\isacommand{qed}\isamarkupfalse%
%
\endisatagproof
{\isafoldproof}%
%
\isadelimproof
\isanewline
%
\endisadelimproof
\isanewline
\isacommand{lemma}\isamarkupfalse%
\ nth{\isacharunderscore}{\kern0pt}mset{\isacharunderscore}{\kern0pt}bound{\isacharunderscore}{\kern0pt}left{\isacharunderscore}{\kern0pt}excl{\isacharcolon}{\kern0pt}\isanewline
\ \ \isakeyword{assumes}\ {\isachardoublequoteopen}k\ {\isacharless}{\kern0pt}\ size\ M{\isachardoublequoteclose}\isanewline
\ \ \isakeyword{assumes}\ {\isachardoublequoteopen}count{\isacharunderscore}{\kern0pt}le\ x\ M\ {\isasymle}\ k{\isachardoublequoteclose}\isanewline
\ \ \isakeyword{shows}\ {\isachardoublequoteopen}x\ {\isacharless}{\kern0pt}\ nth{\isacharunderscore}{\kern0pt}mset\ k\ M{\isachardoublequoteclose}\isanewline
%
\isadelimproof
%
\endisadelimproof
%
\isatagproof
\isacommand{proof}\isamarkupfalse%
\ {\isacharparenleft}{\kern0pt}rule\ ccontr{\isacharparenright}{\kern0pt}\isanewline
\ \ \isacommand{define}\isamarkupfalse%
\ xs\ \isakeyword{where}\ {\isachardoublequoteopen}xs\ {\isacharequal}{\kern0pt}\ sorted{\isacharunderscore}{\kern0pt}list{\isacharunderscore}{\kern0pt}of{\isacharunderscore}{\kern0pt}multiset\ M{\isachardoublequoteclose}\isanewline
\ \ \isacommand{have}\isamarkupfalse%
\ s{\isacharunderscore}{\kern0pt}xs{\isacharcolon}{\kern0pt}\ {\isachardoublequoteopen}sorted\ xs{\isachardoublequoteclose}\ \isacommand{by}\isamarkupfalse%
\ {\isacharparenleft}{\kern0pt}simp\ add{\isacharcolon}{\kern0pt}xs{\isacharunderscore}{\kern0pt}def\ sorted{\isacharunderscore}{\kern0pt}sorted{\isacharunderscore}{\kern0pt}list{\isacharunderscore}{\kern0pt}of{\isacharunderscore}{\kern0pt}multiset{\isacharparenright}{\kern0pt}\isanewline
\ \ \isacommand{have}\isamarkupfalse%
\ l{\isacharunderscore}{\kern0pt}xs{\isacharcolon}{\kern0pt}\ {\isachardoublequoteopen}k\ {\isacharless}{\kern0pt}\ length\ xs{\isachardoublequoteclose}\ \isacommand{apply}\isamarkupfalse%
\ {\isacharparenleft}{\kern0pt}simp\ add{\isacharcolon}{\kern0pt}xs{\isacharunderscore}{\kern0pt}def{\isacharparenright}{\kern0pt}\ \isanewline
\ \ \ \ \isacommand{by}\isamarkupfalse%
\ {\isacharparenleft}{\kern0pt}metis\ size{\isacharunderscore}{\kern0pt}mset\ mset{\isacharunderscore}{\kern0pt}sorted{\isacharunderscore}{\kern0pt}list{\isacharunderscore}{\kern0pt}of{\isacharunderscore}{\kern0pt}multiset\ assms{\isacharparenleft}{\kern0pt}{\isadigit{1}}{\isacharparenright}{\kern0pt}{\isacharparenright}{\kern0pt}\ \ \isanewline
\ \ \isacommand{have}\isamarkupfalse%
\ M{\isacharunderscore}{\kern0pt}xs{\isacharcolon}{\kern0pt}\ {\isachardoublequoteopen}M\ {\isacharequal}{\kern0pt}\ mset\ xs{\isachardoublequoteclose}\ \isacommand{by}\isamarkupfalse%
\ {\isacharparenleft}{\kern0pt}simp\ add{\isacharcolon}{\kern0pt}xs{\isacharunderscore}{\kern0pt}def{\isacharparenright}{\kern0pt}\isanewline
\ \ \isacommand{hence}\isamarkupfalse%
\ a{\isacharcolon}{\kern0pt}{\isachardoublequoteopen}{\isasymAnd}i{\isachardot}{\kern0pt}\ i\ {\isasymle}\ k\ {\isasymLongrightarrow}\ xs\ {\isacharbang}{\kern0pt}\ i\ {\isasymle}\ xs\ {\isacharbang}{\kern0pt}\ k{\isachardoublequoteclose}\isanewline
\ \ \ \ \isacommand{using}\isamarkupfalse%
\ s{\isacharunderscore}{\kern0pt}xs\ l{\isacharunderscore}{\kern0pt}xs\ sorted{\isacharunderscore}{\kern0pt}iff{\isacharunderscore}{\kern0pt}nth{\isacharunderscore}{\kern0pt}mono\ \isacommand{by}\isamarkupfalse%
\ blast\isanewline
\isanewline
\ \ \isacommand{assume}\isamarkupfalse%
\ {\isachardoublequoteopen}{\isasymnot}{\isacharparenleft}{\kern0pt}x\ {\isacharless}{\kern0pt}\ nth{\isacharunderscore}{\kern0pt}mset\ k\ M{\isacharparenright}{\kern0pt}{\isachardoublequoteclose}\isanewline
\ \ \isacommand{hence}\isamarkupfalse%
\ {\isachardoublequoteopen}x\ {\isasymge}\ nth{\isacharunderscore}{\kern0pt}mset\ k\ M{\isachardoublequoteclose}\ \isacommand{by}\isamarkupfalse%
\ simp\isanewline
\ \ \isacommand{hence}\isamarkupfalse%
\ b{\isacharcolon}{\kern0pt}{\isachardoublequoteopen}x\ {\isasymge}\ xs\ {\isacharbang}{\kern0pt}\ k{\isachardoublequoteclose}\ \isacommand{by}\isamarkupfalse%
\ {\isacharparenleft}{\kern0pt}simp\ add{\isacharcolon}{\kern0pt}nth{\isacharunderscore}{\kern0pt}mset{\isacharunderscore}{\kern0pt}def\ xs{\isacharunderscore}{\kern0pt}def{\isacharbrackleft}{\kern0pt}symmetric{\isacharbrackright}{\kern0pt}{\isacharparenright}{\kern0pt}\isanewline
\isanewline
\ \ \isacommand{have}\isamarkupfalse%
\ {\isachardoublequoteopen}k{\isacharplus}{\kern0pt}{\isadigit{1}}\ {\isasymle}\ card\ {\isacharbraceleft}{\kern0pt}{\isadigit{0}}{\isachardot}{\kern0pt}{\isachardot}{\kern0pt}k{\isacharbraceright}{\kern0pt}{\isachardoublequoteclose}\ \isacommand{by}\isamarkupfalse%
\ simp\isanewline
\ \ \isacommand{also}\isamarkupfalse%
\ \isacommand{have}\isamarkupfalse%
\ {\isachardoublequoteopen}{\isachardot}{\kern0pt}{\isachardot}{\kern0pt}{\isachardot}{\kern0pt}\ {\isasymle}\ card\ {\isacharbraceleft}{\kern0pt}i{\isachardot}{\kern0pt}\ i\ {\isacharless}{\kern0pt}\ length\ xs\ {\isasymand}\ xs\ {\isacharbang}{\kern0pt}\ i\ {\isasymle}\ xs\ {\isacharbang}{\kern0pt}\ k{\isacharbraceright}{\kern0pt}{\isachardoublequoteclose}\isanewline
\ \ \ \ \isacommand{apply}\isamarkupfalse%
\ {\isacharparenleft}{\kern0pt}rule\ card{\isacharunderscore}{\kern0pt}mono{\isacharcomma}{\kern0pt}\ simp{\isacharparenright}{\kern0pt}\isanewline
\ \ \ \ \isacommand{apply}\isamarkupfalse%
\ {\isacharparenleft}{\kern0pt}rule\ subsetI{\isacharcomma}{\kern0pt}\ simp{\isacharparenright}{\kern0pt}\isanewline
\ \ \ \ \isacommand{using}\isamarkupfalse%
\ a\ b\ l{\isacharunderscore}{\kern0pt}xs\ order{\isacharunderscore}{\kern0pt}le{\isacharunderscore}{\kern0pt}less{\isacharunderscore}{\kern0pt}trans\ \isacommand{by}\isamarkupfalse%
\ auto\isanewline
\ \ \isacommand{also}\isamarkupfalse%
\ \isacommand{have}\isamarkupfalse%
\ {\isachardoublequoteopen}{\isachardot}{\kern0pt}{\isachardot}{\kern0pt}{\isachardot}{\kern0pt}\ {\isasymle}\ card\ {\isacharbraceleft}{\kern0pt}i{\isachardot}{\kern0pt}\ i\ {\isacharless}{\kern0pt}\ length\ xs\ {\isasymand}\ xs\ {\isacharbang}{\kern0pt}\ i\ {\isasymle}\ x{\isacharbraceright}{\kern0pt}{\isachardoublequoteclose}\isanewline
\ \ \ \ \isacommand{apply}\isamarkupfalse%
\ {\isacharparenleft}{\kern0pt}rule\ card{\isacharunderscore}{\kern0pt}mono{\isacharcomma}{\kern0pt}\ simp{\isacharparenright}{\kern0pt}\isanewline
\ \ \ \ \isacommand{apply}\isamarkupfalse%
\ {\isacharparenleft}{\kern0pt}rule\ subsetI{\isacharcomma}{\kern0pt}\ simp{\isacharparenright}{\kern0pt}\ \isacommand{using}\isamarkupfalse%
\ b\ \isanewline
\ \ \ \ \isacommand{by}\isamarkupfalse%
\ force\isanewline
\ \ \isacommand{also}\isamarkupfalse%
\ \isacommand{have}\isamarkupfalse%
\ {\isachardoublequoteopen}{\isachardot}{\kern0pt}{\isachardot}{\kern0pt}{\isachardot}{\kern0pt}\ {\isacharequal}{\kern0pt}\ count{\isacharunderscore}{\kern0pt}le\ x\ M{\isachardoublequoteclose}\isanewline
\ \ \ \ \isacommand{apply}\isamarkupfalse%
\ {\isacharparenleft}{\kern0pt}simp\ add{\isacharcolon}{\kern0pt}count{\isacharunderscore}{\kern0pt}le{\isacharunderscore}{\kern0pt}def\ M{\isacharunderscore}{\kern0pt}xs{\isacharparenright}{\kern0pt}\isanewline
\ \ \ \ \isacommand{apply}\isamarkupfalse%
\ {\isacharparenleft}{\kern0pt}subst\ mset{\isacharunderscore}{\kern0pt}filter{\isacharbrackleft}{\kern0pt}symmetric{\isacharbrackright}{\kern0pt}{\isacharcomma}{\kern0pt}\ subst\ size{\isacharunderscore}{\kern0pt}mset{\isacharparenright}{\kern0pt}\isanewline
\ \ \ \ \isacommand{by}\isamarkupfalse%
\ {\isacharparenleft}{\kern0pt}subst\ length{\isacharunderscore}{\kern0pt}filter{\isacharunderscore}{\kern0pt}conv{\isacharunderscore}{\kern0pt}card{\isacharcomma}{\kern0pt}\ simp{\isacharparenright}{\kern0pt}\isanewline
\ \ \isacommand{also}\isamarkupfalse%
\ \isacommand{have}\isamarkupfalse%
\ {\isachardoublequoteopen}{\isachardot}{\kern0pt}{\isachardot}{\kern0pt}{\isachardot}{\kern0pt}\ {\isasymle}\ k{\isachardoublequoteclose}\isanewline
\ \ \ \ \isacommand{using}\isamarkupfalse%
\ assms\ \isacommand{by}\isamarkupfalse%
\ simp\isanewline
\ \ \isacommand{finally}\isamarkupfalse%
\ \isacommand{show}\isamarkupfalse%
\ {\isachardoublequoteopen}False{\isachardoublequoteclose}\ \isacommand{by}\isamarkupfalse%
\ simp\isanewline
\isacommand{qed}\isamarkupfalse%
%
\endisatagproof
{\isafoldproof}%
%
\isadelimproof
\isanewline
%
\endisadelimproof
\isanewline
\isacommand{lemma}\isamarkupfalse%
\ nth{\isacharunderscore}{\kern0pt}mset{\isacharunderscore}{\kern0pt}bound{\isacharunderscore}{\kern0pt}right{\isacharcolon}{\kern0pt}\isanewline
\ \ \isakeyword{assumes}\ {\isachardoublequoteopen}k\ {\isacharless}{\kern0pt}\ size\ M{\isachardoublequoteclose}\isanewline
\ \ \isakeyword{assumes}\ {\isachardoublequoteopen}count{\isacharunderscore}{\kern0pt}le\ x\ M\ {\isachargreater}{\kern0pt}\ k{\isachardoublequoteclose}\isanewline
\ \ \isakeyword{shows}\ {\isachardoublequoteopen}nth{\isacharunderscore}{\kern0pt}mset\ k\ M\ {\isasymle}\ x{\isachardoublequoteclose}\isanewline
%
\isadelimproof
%
\endisadelimproof
%
\isatagproof
\isacommand{proof}\isamarkupfalse%
\ {\isacharparenleft}{\kern0pt}rule\ ccontr{\isacharparenright}{\kern0pt}\isanewline
\ \ \isacommand{define}\isamarkupfalse%
\ xs\ \isakeyword{where}\ {\isachardoublequoteopen}xs\ {\isacharequal}{\kern0pt}\ sorted{\isacharunderscore}{\kern0pt}list{\isacharunderscore}{\kern0pt}of{\isacharunderscore}{\kern0pt}multiset\ M{\isachardoublequoteclose}\isanewline
\ \ \isacommand{have}\isamarkupfalse%
\ s{\isacharunderscore}{\kern0pt}xs{\isacharcolon}{\kern0pt}\ {\isachardoublequoteopen}sorted\ xs{\isachardoublequoteclose}\ \isacommand{by}\isamarkupfalse%
\ {\isacharparenleft}{\kern0pt}simp\ add{\isacharcolon}{\kern0pt}xs{\isacharunderscore}{\kern0pt}def\ sorted{\isacharunderscore}{\kern0pt}sorted{\isacharunderscore}{\kern0pt}list{\isacharunderscore}{\kern0pt}of{\isacharunderscore}{\kern0pt}multiset{\isacharparenright}{\kern0pt}\isanewline
\ \ \isacommand{have}\isamarkupfalse%
\ l{\isacharunderscore}{\kern0pt}xs{\isacharcolon}{\kern0pt}\ {\isachardoublequoteopen}k\ {\isacharless}{\kern0pt}\ length\ xs{\isachardoublequoteclose}\ \isacommand{apply}\isamarkupfalse%
\ {\isacharparenleft}{\kern0pt}simp\ add{\isacharcolon}{\kern0pt}xs{\isacharunderscore}{\kern0pt}def{\isacharparenright}{\kern0pt}\ \isanewline
\ \ \ \ \isacommand{by}\isamarkupfalse%
\ {\isacharparenleft}{\kern0pt}metis\ size{\isacharunderscore}{\kern0pt}mset\ mset{\isacharunderscore}{\kern0pt}sorted{\isacharunderscore}{\kern0pt}list{\isacharunderscore}{\kern0pt}of{\isacharunderscore}{\kern0pt}multiset\ assms{\isacharparenleft}{\kern0pt}{\isadigit{1}}{\isacharparenright}{\kern0pt}{\isacharparenright}{\kern0pt}\ \ \isanewline
\ \ \isacommand{have}\isamarkupfalse%
\ M{\isacharunderscore}{\kern0pt}xs{\isacharcolon}{\kern0pt}\ {\isachardoublequoteopen}M\ {\isacharequal}{\kern0pt}\ mset\ xs{\isachardoublequoteclose}\ \isacommand{by}\isamarkupfalse%
\ {\isacharparenleft}{\kern0pt}simp\ add{\isacharcolon}{\kern0pt}xs{\isacharunderscore}{\kern0pt}def{\isacharparenright}{\kern0pt}\isanewline
\isanewline
\ \ \isacommand{assume}\isamarkupfalse%
\ {\isachardoublequoteopen}{\isasymnot}{\isacharparenleft}{\kern0pt}nth{\isacharunderscore}{\kern0pt}mset\ k\ M\ {\isasymle}\ x{\isacharparenright}{\kern0pt}{\isachardoublequoteclose}\isanewline
\ \ \isacommand{hence}\isamarkupfalse%
\ {\isachardoublequoteopen}x\ {\isacharless}{\kern0pt}\ nth{\isacharunderscore}{\kern0pt}mset\ k\ M{\isachardoublequoteclose}\ \isacommand{by}\isamarkupfalse%
\ simp\isanewline
\ \ \isacommand{hence}\isamarkupfalse%
\ {\isachardoublequoteopen}x\ {\isacharless}{\kern0pt}\ xs\ {\isacharbang}{\kern0pt}\ k{\isachardoublequoteclose}\ \isanewline
\ \ \ \ \isacommand{by}\isamarkupfalse%
\ {\isacharparenleft}{\kern0pt}simp\ add{\isacharcolon}{\kern0pt}nth{\isacharunderscore}{\kern0pt}mset{\isacharunderscore}{\kern0pt}def\ xs{\isacharunderscore}{\kern0pt}def{\isacharbrackleft}{\kern0pt}symmetric{\isacharbrackright}{\kern0pt}{\isacharparenright}{\kern0pt}\isanewline
\ \ \isacommand{hence}\isamarkupfalse%
\ a{\isacharcolon}{\kern0pt}{\isachardoublequoteopen}{\isasymAnd}i{\isachardot}{\kern0pt}\ i\ {\isacharless}{\kern0pt}\ length\ xs\ {\isasymand}\ xs\ {\isacharbang}{\kern0pt}\ i\ {\isasymle}\ x\ {\isasymLongrightarrow}\ i\ {\isacharless}{\kern0pt}\ k{\isachardoublequoteclose}\isanewline
\ \ \ \ \isacommand{using}\isamarkupfalse%
\ s{\isacharunderscore}{\kern0pt}xs\ l{\isacharunderscore}{\kern0pt}xs\ sorted{\isacharunderscore}{\kern0pt}iff{\isacharunderscore}{\kern0pt}nth{\isacharunderscore}{\kern0pt}mono\ leI\ \isacommand{by}\isamarkupfalse%
\ fastforce\isanewline
\ \ \isacommand{have}\isamarkupfalse%
\ {\isachardoublequoteopen}count{\isacharunderscore}{\kern0pt}le\ x\ M\ {\isasymle}\ card\ {\isacharbraceleft}{\kern0pt}i{\isachardot}{\kern0pt}\ i\ {\isacharless}{\kern0pt}\ length\ xs\ {\isasymand}\ xs\ {\isacharbang}{\kern0pt}\ i\ {\isasymle}\ x{\isacharbraceright}{\kern0pt}{\isachardoublequoteclose}\isanewline
\ \ \ \ \isacommand{apply}\isamarkupfalse%
\ {\isacharparenleft}{\kern0pt}simp\ add{\isacharcolon}{\kern0pt}count{\isacharunderscore}{\kern0pt}le{\isacharunderscore}{\kern0pt}def\ M{\isacharunderscore}{\kern0pt}xs{\isacharparenright}{\kern0pt}\isanewline
\ \ \ \ \isacommand{apply}\isamarkupfalse%
\ {\isacharparenleft}{\kern0pt}subst\ mset{\isacharunderscore}{\kern0pt}filter{\isacharbrackleft}{\kern0pt}symmetric{\isacharbrackright}{\kern0pt}{\isacharcomma}{\kern0pt}\ subst\ size{\isacharunderscore}{\kern0pt}mset{\isacharparenright}{\kern0pt}\isanewline
\ \ \ \ \isacommand{apply}\isamarkupfalse%
\ {\isacharparenleft}{\kern0pt}subst\ length{\isacharunderscore}{\kern0pt}filter{\isacharunderscore}{\kern0pt}conv{\isacharunderscore}{\kern0pt}card{\isacharparenright}{\kern0pt}\isanewline
\ \ \ \ \isacommand{by}\isamarkupfalse%
\ {\isacharparenleft}{\kern0pt}rule\ card{\isacharunderscore}{\kern0pt}mono{\isacharcomma}{\kern0pt}\ simp{\isacharcomma}{\kern0pt}\ simp{\isacharparenright}{\kern0pt}\isanewline
\ \ \isacommand{also}\isamarkupfalse%
\ \isacommand{have}\isamarkupfalse%
\ {\isachardoublequoteopen}{\isachardot}{\kern0pt}{\isachardot}{\kern0pt}{\isachardot}{\kern0pt}\ {\isasymle}\ card\ {\isacharbraceleft}{\kern0pt}i{\isachardot}{\kern0pt}\ i\ {\isacharless}{\kern0pt}\ k{\isacharbraceright}{\kern0pt}{\isachardoublequoteclose}\isanewline
\ \ \ \ \isacommand{apply}\isamarkupfalse%
\ {\isacharparenleft}{\kern0pt}rule\ card{\isacharunderscore}{\kern0pt}mono{\isacharcomma}{\kern0pt}\ simp{\isacharparenright}{\kern0pt}\isanewline
\ \ \ \ \isacommand{by}\isamarkupfalse%
\ {\isacharparenleft}{\kern0pt}rule\ subsetI{\isacharcomma}{\kern0pt}\ simp\ add{\isacharcolon}{\kern0pt}a{\isacharparenright}{\kern0pt}\isanewline
\ \ \isacommand{also}\isamarkupfalse%
\ \isacommand{have}\isamarkupfalse%
\ {\isachardoublequoteopen}{\isachardot}{\kern0pt}{\isachardot}{\kern0pt}{\isachardot}{\kern0pt}\ {\isacharequal}{\kern0pt}\ k{\isachardoublequoteclose}\ \isacommand{by}\isamarkupfalse%
\ simp\isanewline
\ \ \isacommand{finally}\isamarkupfalse%
\ \isacommand{have}\isamarkupfalse%
\ {\isachardoublequoteopen}count{\isacharunderscore}{\kern0pt}le\ x\ M\ {\isasymle}\ k{\isachardoublequoteclose}\ \isacommand{by}\isamarkupfalse%
\ simp\isanewline
\ \ \isacommand{thus}\isamarkupfalse%
\ {\isachardoublequoteopen}False{\isachardoublequoteclose}\ \isacommand{using}\isamarkupfalse%
\ assms\ \isacommand{by}\isamarkupfalse%
\ simp\isanewline
\isacommand{qed}\isamarkupfalse%
%
\endisatagproof
{\isafoldproof}%
%
\isadelimproof
\isanewline
%
\endisadelimproof
\isanewline
\isacommand{lemma}\isamarkupfalse%
\ nth{\isacharunderscore}{\kern0pt}mset{\isacharunderscore}{\kern0pt}commute{\isacharunderscore}{\kern0pt}mono{\isacharcolon}{\kern0pt}\isanewline
\ \ \isakeyword{assumes}\ {\isachardoublequoteopen}mono\ f{\isachardoublequoteclose}\isanewline
\ \ \isakeyword{assumes}\ {\isachardoublequoteopen}k\ {\isacharless}{\kern0pt}\ size\ M{\isachardoublequoteclose}\isanewline
\ \ \isakeyword{shows}\ {\isachardoublequoteopen}f\ {\isacharparenleft}{\kern0pt}nth{\isacharunderscore}{\kern0pt}mset\ k\ M{\isacharparenright}{\kern0pt}\ {\isacharequal}{\kern0pt}\ nth{\isacharunderscore}{\kern0pt}mset\ k\ {\isacharparenleft}{\kern0pt}image{\isacharunderscore}{\kern0pt}mset\ f\ M{\isacharparenright}{\kern0pt}{\isachardoublequoteclose}\isanewline
%
\isadelimproof
%
\endisadelimproof
%
\isatagproof
\isacommand{proof}\isamarkupfalse%
\ {\isacharminus}{\kern0pt}\isanewline
\ \ \isacommand{have}\isamarkupfalse%
\ a{\isacharcolon}{\kern0pt}{\isachardoublequoteopen}k\ {\isacharless}{\kern0pt}\ length\ {\isacharparenleft}{\kern0pt}sorted{\isacharunderscore}{\kern0pt}list{\isacharunderscore}{\kern0pt}of{\isacharunderscore}{\kern0pt}multiset\ M{\isacharparenright}{\kern0pt}{\isachardoublequoteclose}\isanewline
\ \ \ \ \isacommand{by}\isamarkupfalse%
\ {\isacharparenleft}{\kern0pt}metis\ assms{\isacharparenleft}{\kern0pt}{\isadigit{2}}{\isacharparenright}{\kern0pt}\ mset{\isacharunderscore}{\kern0pt}sorted{\isacharunderscore}{\kern0pt}list{\isacharunderscore}{\kern0pt}of{\isacharunderscore}{\kern0pt}multiset\ size{\isacharunderscore}{\kern0pt}mset{\isacharparenright}{\kern0pt}\isanewline
\ \ \isacommand{show}\isamarkupfalse%
\ {\isacharquery}{\kern0pt}thesis\isanewline
\ \ \ \ \isacommand{using}\isamarkupfalse%
\ a\ \isacommand{by}\isamarkupfalse%
\ {\isacharparenleft}{\kern0pt}simp\ add{\isacharcolon}{\kern0pt}nth{\isacharunderscore}{\kern0pt}mset{\isacharunderscore}{\kern0pt}def\ sorted{\isacharunderscore}{\kern0pt}list{\isacharunderscore}{\kern0pt}of{\isacharunderscore}{\kern0pt}multiset{\isacharunderscore}{\kern0pt}image{\isacharunderscore}{\kern0pt}commute{\isacharbrackleft}{\kern0pt}OF\ assms{\isacharparenleft}{\kern0pt}{\isadigit{1}}{\isacharparenright}{\kern0pt}{\isacharbrackright}{\kern0pt}{\isacharparenright}{\kern0pt}\isanewline
\isacommand{qed}\isamarkupfalse%
%
\endisatagproof
{\isafoldproof}%
%
\isadelimproof
\ \isanewline
%
\endisadelimproof
\isanewline
\isacommand{lemma}\isamarkupfalse%
\ nth{\isacharunderscore}{\kern0pt}mset{\isacharunderscore}{\kern0pt}max{\isacharcolon}{\kern0pt}\ \isanewline
\ \ \isakeyword{assumes}\ {\isachardoublequoteopen}size\ A\ {\isachargreater}{\kern0pt}\ k{\isachardoublequoteclose}\isanewline
\ \ \isakeyword{assumes}\ {\isachardoublequoteopen}{\isasymAnd}x{\isachardot}{\kern0pt}\ x\ {\isasymle}\ nth{\isacharunderscore}{\kern0pt}mset\ k\ A\ {\isasymLongrightarrow}\ count\ A\ x\ {\isasymle}\ {\isadigit{1}}{\isachardoublequoteclose}\isanewline
\ \ \isakeyword{shows}\ {\isachardoublequoteopen}nth{\isacharunderscore}{\kern0pt}mset\ k\ A\ {\isacharequal}{\kern0pt}\ Max\ {\isacharparenleft}{\kern0pt}least\ {\isacharparenleft}{\kern0pt}k{\isacharplus}{\kern0pt}{\isadigit{1}}{\isacharparenright}{\kern0pt}\ {\isacharparenleft}{\kern0pt}set{\isacharunderscore}{\kern0pt}mset\ A{\isacharparenright}{\kern0pt}{\isacharparenright}{\kern0pt}{\isachardoublequoteclose}\ \isakeyword{and}\ {\isachardoublequoteopen}card\ {\isacharparenleft}{\kern0pt}least\ {\isacharparenleft}{\kern0pt}k{\isacharplus}{\kern0pt}{\isadigit{1}}{\isacharparenright}{\kern0pt}\ {\isacharparenleft}{\kern0pt}set{\isacharunderscore}{\kern0pt}mset\ A{\isacharparenright}{\kern0pt}{\isacharparenright}{\kern0pt}\ {\isacharequal}{\kern0pt}\ k{\isacharplus}{\kern0pt}{\isadigit{1}}{\isachardoublequoteclose}\isanewline
%
\isadelimproof
%
\endisadelimproof
%
\isatagproof
\isacommand{proof}\isamarkupfalse%
\ {\isacharminus}{\kern0pt}\isanewline
\ \ \isacommand{define}\isamarkupfalse%
\ xs\ \isakeyword{where}\ {\isachardoublequoteopen}xs\ {\isacharequal}{\kern0pt}\ sorted{\isacharunderscore}{\kern0pt}list{\isacharunderscore}{\kern0pt}of{\isacharunderscore}{\kern0pt}multiset\ A{\isachardoublequoteclose}\isanewline
\ \ \isacommand{have}\isamarkupfalse%
\ k{\isacharunderscore}{\kern0pt}bound{\isacharcolon}{\kern0pt}\ {\isachardoublequoteopen}k\ {\isacharless}{\kern0pt}\ length\ xs{\isachardoublequoteclose}\ \isacommand{apply}\isamarkupfalse%
\ {\isacharparenleft}{\kern0pt}simp\ add{\isacharcolon}{\kern0pt}xs{\isacharunderscore}{\kern0pt}def{\isacharparenright}{\kern0pt}\isanewline
\ \ \ \ \isacommand{by}\isamarkupfalse%
\ {\isacharparenleft}{\kern0pt}metis\ size{\isacharunderscore}{\kern0pt}mset\ mset{\isacharunderscore}{\kern0pt}sorted{\isacharunderscore}{\kern0pt}list{\isacharunderscore}{\kern0pt}of{\isacharunderscore}{\kern0pt}multiset\ assms{\isacharparenleft}{\kern0pt}{\isadigit{1}}{\isacharparenright}{\kern0pt}{\isacharparenright}{\kern0pt}\ \ \isanewline
\isanewline
\ \ \isacommand{have}\isamarkupfalse%
\ A{\isacharunderscore}{\kern0pt}def{\isacharcolon}{\kern0pt}\ {\isachardoublequoteopen}A\ {\isacharequal}{\kern0pt}\ mset\ xs{\isachardoublequoteclose}\ \isacommand{by}\isamarkupfalse%
\ {\isacharparenleft}{\kern0pt}simp\ add{\isacharcolon}{\kern0pt}xs{\isacharunderscore}{\kern0pt}def{\isacharparenright}{\kern0pt}\isanewline
\ \ \isacommand{have}\isamarkupfalse%
\ s{\isacharunderscore}{\kern0pt}xs{\isacharcolon}{\kern0pt}\ {\isachardoublequoteopen}sorted\ xs{\isachardoublequoteclose}\ \isacommand{by}\isamarkupfalse%
\ {\isacharparenleft}{\kern0pt}simp\ add{\isacharcolon}{\kern0pt}xs{\isacharunderscore}{\kern0pt}def\ sorted{\isacharunderscore}{\kern0pt}sorted{\isacharunderscore}{\kern0pt}list{\isacharunderscore}{\kern0pt}of{\isacharunderscore}{\kern0pt}multiset{\isacharparenright}{\kern0pt}\isanewline
\ \ \isacommand{have}\isamarkupfalse%
\ a{\isacharunderscore}{\kern0pt}{\isadigit{2}}{\isacharcolon}{\kern0pt}\ {\isachardoublequoteopen}{\isasymAnd}x{\isachardot}{\kern0pt}\ x\ {\isasymle}\ xs\ {\isacharbang}{\kern0pt}\ k\ {\isasymLongrightarrow}\ count{\isacharunderscore}{\kern0pt}list\ xs\ x\ {\isasymle}\ {\isadigit{1}}{\isachardoublequoteclose}\ \isanewline
\ \ \ \ \isacommand{using}\isamarkupfalse%
\ assms{\isacharparenleft}{\kern0pt}{\isadigit{2}}{\isacharparenright}{\kern0pt}\ \isacommand{apply}\isamarkupfalse%
\ {\isacharparenleft}{\kern0pt}simp\ add{\isacharcolon}{\kern0pt}xs{\isacharunderscore}{\kern0pt}def{\isacharbrackleft}{\kern0pt}symmetric{\isacharbrackright}{\kern0pt}\ nth{\isacharunderscore}{\kern0pt}mset{\isacharunderscore}{\kern0pt}def{\isacharparenright}{\kern0pt}\isanewline
\ \ \ \ \isacommand{by}\isamarkupfalse%
\ {\isacharparenleft}{\kern0pt}simp\ add{\isacharcolon}{\kern0pt}A{\isacharunderscore}{\kern0pt}def\ count{\isacharunderscore}{\kern0pt}mset{\isacharparenright}{\kern0pt}\ \isanewline
\isanewline
\ \ \isacommand{have}\isamarkupfalse%
\ inj{\isacharunderscore}{\kern0pt}xs{\isacharcolon}{\kern0pt}\ {\isachardoublequoteopen}inj{\isacharunderscore}{\kern0pt}on\ {\isacharparenleft}{\kern0pt}{\isasymlambda}k{\isachardot}{\kern0pt}\ xs\ {\isacharbang}{\kern0pt}\ k{\isacharparenright}{\kern0pt}\ {\isacharbraceleft}{\kern0pt}{\isadigit{0}}{\isachardot}{\kern0pt}{\isachardot}{\kern0pt}k{\isacharbraceright}{\kern0pt}{\isachardoublequoteclose}\isanewline
\ \ \ \ \isacommand{apply}\isamarkupfalse%
\ {\isacharparenleft}{\kern0pt}rule\ inj{\isacharunderscore}{\kern0pt}onI{\isacharparenright}{\kern0pt}\isanewline
\ \ \ \ \isacommand{apply}\isamarkupfalse%
\ simp\isanewline
\ \ \ \ \isacommand{by}\isamarkupfalse%
\ {\isacharparenleft}{\kern0pt}metis\ {\isacharparenleft}{\kern0pt}full{\isacharunderscore}{\kern0pt}types{\isacharparenright}{\kern0pt}\ count{\isacharunderscore}{\kern0pt}list{\isacharunderscore}{\kern0pt}ge{\isacharunderscore}{\kern0pt}{\isadigit{2}}{\isacharunderscore}{\kern0pt}iff\ k{\isacharunderscore}{\kern0pt}bound\ a{\isacharunderscore}{\kern0pt}{\isadigit{2}}\ \isanewline
\ \ \ \ \ \ \ \ le{\isacharunderscore}{\kern0pt}neq{\isacharunderscore}{\kern0pt}implies{\isacharunderscore}{\kern0pt}less\ linorder{\isacharunderscore}{\kern0pt}not{\isacharunderscore}{\kern0pt}le\ order{\isacharunderscore}{\kern0pt}le{\isacharunderscore}{\kern0pt}less{\isacharunderscore}{\kern0pt}trans\ s{\isacharunderscore}{\kern0pt}xs\ sorted{\isacharunderscore}{\kern0pt}iff{\isacharunderscore}{\kern0pt}nth{\isacharunderscore}{\kern0pt}mono{\isacharparenright}{\kern0pt}\isanewline
\isanewline
\ \ \isacommand{have}\isamarkupfalse%
\ rank{\isacharunderscore}{\kern0pt}conv{\isacharunderscore}{\kern0pt}{\isadigit{2}}{\isacharcolon}{\kern0pt}\ {\isachardoublequoteopen}{\isasymAnd}y{\isachardot}{\kern0pt}\ y\ {\isacharless}{\kern0pt}\ length\ xs\ {\isasymLongrightarrow}\ rank{\isacharunderscore}{\kern0pt}of\ {\isacharparenleft}{\kern0pt}xs\ {\isacharbang}{\kern0pt}\ y{\isacharparenright}{\kern0pt}\ {\isacharparenleft}{\kern0pt}set\ xs{\isacharparenright}{\kern0pt}\ {\isacharless}{\kern0pt}\ k{\isacharplus}{\kern0pt}{\isadigit{1}}\ {\isasymLongrightarrow}\ y\ {\isacharless}{\kern0pt}\ k{\isacharplus}{\kern0pt}{\isadigit{1}}{\isachardoublequoteclose}\isanewline
\ \ \isacommand{proof}\isamarkupfalse%
\ {\isacharparenleft}{\kern0pt}rule\ ccontr{\isacharparenright}{\kern0pt}\isanewline
\ \ \ \ \isacommand{fix}\isamarkupfalse%
\ y\isanewline
\ \ \ \ \isacommand{assume}\isamarkupfalse%
\ b{\isacharcolon}{\kern0pt}{\isachardoublequoteopen}y\ {\isacharless}{\kern0pt}\ length\ xs{\isachardoublequoteclose}\isanewline
\ \ \ \ \isacommand{assume}\isamarkupfalse%
\ {\isachardoublequoteopen}{\isasymnot}y\ {\isacharless}{\kern0pt}\ k\ {\isacharplus}{\kern0pt}{\isadigit{1}}{\isachardoublequoteclose}\isanewline
\ \ \ \ \isacommand{hence}\isamarkupfalse%
\ a{\isacharcolon}{\kern0pt}{\isachardoublequoteopen}k\ {\isacharplus}{\kern0pt}\ {\isadigit{1}}\ {\isasymle}\ y{\isachardoublequoteclose}\ \isacommand{by}\isamarkupfalse%
\ simp\isanewline
\isanewline
\ \ \ \ \isacommand{have}\isamarkupfalse%
\ d{\isacharcolon}{\kern0pt}{\isachardoublequoteopen}Suc\ k\ {\isacharless}{\kern0pt}\ length\ xs{\isachardoublequoteclose}\ \isacommand{using}\isamarkupfalse%
\ a\ b\ \isacommand{by}\isamarkupfalse%
\ simp\isanewline
\isanewline
\ \ \ \ \isacommand{have}\isamarkupfalse%
\ {\isachardoublequoteopen}k{\isacharplus}{\kern0pt}{\isadigit{1}}\ {\isacharequal}{\kern0pt}\ card\ {\isacharparenleft}{\kern0pt}{\isacharparenleft}{\kern0pt}{\isacharbang}{\kern0pt}{\isacharparenright}{\kern0pt}\ xs\ {\isacharbackquote}{\kern0pt}\ {\isacharbraceleft}{\kern0pt}{\isadigit{0}}{\isachardot}{\kern0pt}{\isachardot}{\kern0pt}k{\isacharbraceright}{\kern0pt}{\isacharparenright}{\kern0pt}{\isachardoublequoteclose}\ \isanewline
\ \ \ \ \ \ \isacommand{by}\isamarkupfalse%
\ {\isacharparenleft}{\kern0pt}subst\ card{\isacharunderscore}{\kern0pt}image{\isacharbrackleft}{\kern0pt}OF\ inj{\isacharunderscore}{\kern0pt}xs{\isacharbrackright}{\kern0pt}{\isacharcomma}{\kern0pt}\ simp{\isacharparenright}{\kern0pt}\isanewline
\ \ \ \ \isacommand{also}\isamarkupfalse%
\ \isacommand{have}\isamarkupfalse%
\ {\isachardoublequoteopen}{\isachardot}{\kern0pt}{\isachardot}{\kern0pt}{\isachardot}{\kern0pt}\ {\isasymle}\ rank{\isacharunderscore}{\kern0pt}of\ {\isacharparenleft}{\kern0pt}xs\ {\isacharbang}{\kern0pt}\ {\isacharparenleft}{\kern0pt}k{\isacharplus}{\kern0pt}{\isadigit{1}}{\isacharparenright}{\kern0pt}{\isacharparenright}{\kern0pt}\ {\isacharparenleft}{\kern0pt}set\ xs{\isacharparenright}{\kern0pt}{\isachardoublequoteclose}\isanewline
\ \ \ \ \ \ \isacommand{apply}\isamarkupfalse%
\ {\isacharparenleft}{\kern0pt}simp\ add{\isacharcolon}{\kern0pt}rank{\isacharunderscore}{\kern0pt}of{\isacharunderscore}{\kern0pt}def{\isacharparenright}{\kern0pt}\isanewline
\ \ \ \ \ \ \isacommand{apply}\isamarkupfalse%
\ {\isacharparenleft}{\kern0pt}rule\ card{\isacharunderscore}{\kern0pt}mono{\isacharcomma}{\kern0pt}\ simp{\isacharparenright}{\kern0pt}\isanewline
\ \ \ \ \ \ \isacommand{apply}\isamarkupfalse%
\ {\isacharparenleft}{\kern0pt}rule\ image{\isacharunderscore}{\kern0pt}subsetI{\isacharcomma}{\kern0pt}\ simp{\isacharparenright}{\kern0pt}\ \isanewline
\ \ \ \ \ \ \isacommand{apply}\isamarkupfalse%
\ {\isacharparenleft}{\kern0pt}rule\ conjI{\isacharparenright}{\kern0pt}\ \isacommand{using}\isamarkupfalse%
\ k{\isacharunderscore}{\kern0pt}bound\ \isacommand{apply}\isamarkupfalse%
\ simp\isanewline
\ \ \ \ \ \ \isacommand{by}\isamarkupfalse%
\ {\isacharparenleft}{\kern0pt}metis\ count{\isacharunderscore}{\kern0pt}list{\isacharunderscore}{\kern0pt}ge{\isacharunderscore}{\kern0pt}{\isadigit{2}}{\isacharunderscore}{\kern0pt}iff\ a{\isacharunderscore}{\kern0pt}{\isadigit{2}}\ not{\isacharunderscore}{\kern0pt}le\ le{\isacharunderscore}{\kern0pt}imp{\isacharunderscore}{\kern0pt}less{\isacharunderscore}{\kern0pt}Suc\ s{\isacharunderscore}{\kern0pt}xs\ sorted{\isacharunderscore}{\kern0pt}iff{\isacharunderscore}{\kern0pt}nth{\isacharunderscore}{\kern0pt}mono\ d\ order{\isacharunderscore}{\kern0pt}less{\isacharunderscore}{\kern0pt}le{\isacharparenright}{\kern0pt}\isanewline
\ \ \ \ \isacommand{also}\isamarkupfalse%
\ \isacommand{have}\isamarkupfalse%
\ {\isachardoublequoteopen}{\isachardot}{\kern0pt}{\isachardot}{\kern0pt}{\isachardot}{\kern0pt}\ {\isasymle}\ rank{\isacharunderscore}{\kern0pt}of\ {\isacharparenleft}{\kern0pt}xs\ {\isacharbang}{\kern0pt}\ y{\isacharparenright}{\kern0pt}\ {\isacharparenleft}{\kern0pt}set\ xs{\isacharparenright}{\kern0pt}{\isachardoublequoteclose}\isanewline
\ \ \ \ \ \ \isacommand{apply}\isamarkupfalse%
\ {\isacharparenleft}{\kern0pt}simp\ add{\isacharcolon}{\kern0pt}rank{\isacharunderscore}{\kern0pt}of{\isacharunderscore}{\kern0pt}def{\isacharparenright}{\kern0pt}\isanewline
\ \ \ \ \ \ \isacommand{apply}\isamarkupfalse%
\ {\isacharparenleft}{\kern0pt}rule\ card{\isacharunderscore}{\kern0pt}mono{\isacharcomma}{\kern0pt}\ simp{\isacharparenright}{\kern0pt}\isanewline
\ \ \ \ \ \ \isacommand{apply}\isamarkupfalse%
\ {\isacharparenleft}{\kern0pt}rule\ subsetI{\isacharcomma}{\kern0pt}\ simp{\isacharparenright}{\kern0pt}\isanewline
\ \ \ \ \ \ \isacommand{by}\isamarkupfalse%
\ {\isacharparenleft}{\kern0pt}metis\ Suc{\isacharunderscore}{\kern0pt}eq{\isacharunderscore}{\kern0pt}plus{\isadigit{1}}\ a\ b\ s{\isacharunderscore}{\kern0pt}xs\ order{\isacharunderscore}{\kern0pt}less{\isacharunderscore}{\kern0pt}le{\isacharunderscore}{\kern0pt}trans\ sorted{\isacharunderscore}{\kern0pt}iff{\isacharunderscore}{\kern0pt}nth{\isacharunderscore}{\kern0pt}mono{\isacharparenright}{\kern0pt}\isanewline
\ \ \ \ \isacommand{also}\isamarkupfalse%
\ \isacommand{assume}\isamarkupfalse%
\ {\isachardoublequoteopen}{\isachardot}{\kern0pt}{\isachardot}{\kern0pt}{\isachardot}{\kern0pt}\ {\isacharless}{\kern0pt}\ k{\isacharplus}{\kern0pt}{\isadigit{1}}{\isachardoublequoteclose}\isanewline
\ \ \ \ \isacommand{finally}\isamarkupfalse%
\ \isacommand{show}\isamarkupfalse%
\ {\isachardoublequoteopen}False{\isachardoublequoteclose}\ \isacommand{by}\isamarkupfalse%
\ force\isanewline
\ \ \isacommand{qed}\isamarkupfalse%
\isanewline
\isanewline
\ \ \isacommand{have}\isamarkupfalse%
\ rank{\isacharunderscore}{\kern0pt}conv{\isacharunderscore}{\kern0pt}{\isadigit{1}}{\isacharcolon}{\kern0pt}\ {\isachardoublequoteopen}{\isasymAnd}y{\isachardot}{\kern0pt}\ y\ {\isacharless}{\kern0pt}\ k\ {\isacharplus}{\kern0pt}\ {\isadigit{1}}\ {\isasymLongrightarrow}\ rank{\isacharunderscore}{\kern0pt}of\ {\isacharparenleft}{\kern0pt}xs\ {\isacharbang}{\kern0pt}\ y{\isacharparenright}{\kern0pt}\ {\isacharparenleft}{\kern0pt}set\ xs{\isacharparenright}{\kern0pt}\ {\isacharless}{\kern0pt}\ k{\isacharplus}{\kern0pt}{\isadigit{1}}{\isachardoublequoteclose}\isanewline
\ \ \isacommand{proof}\isamarkupfalse%
\ {\isacharminus}{\kern0pt}\isanewline
\ \ \ \ \isacommand{fix}\isamarkupfalse%
\ y\isanewline
\ \ \ \ \isacommand{have}\isamarkupfalse%
\ {\isachardoublequoteopen}rank{\isacharunderscore}{\kern0pt}of\ {\isacharparenleft}{\kern0pt}xs\ {\isacharbang}{\kern0pt}\ y{\isacharparenright}{\kern0pt}\ {\isacharparenleft}{\kern0pt}set\ xs{\isacharparenright}{\kern0pt}\ {\isasymle}\ card\ {\isacharparenleft}{\kern0pt}{\isacharparenleft}{\kern0pt}{\isasymlambda}k{\isachardot}{\kern0pt}\ xs\ {\isacharbang}{\kern0pt}\ k{\isacharparenright}{\kern0pt}\ {\isacharbackquote}{\kern0pt}\ {\isacharbraceleft}{\kern0pt}k{\isachardot}{\kern0pt}\ k\ {\isacharless}{\kern0pt}\ length\ xs\ {\isasymand}\ xs\ {\isacharbang}{\kern0pt}\ k\ {\isacharless}{\kern0pt}\ xs\ {\isacharbang}{\kern0pt}\ y{\isacharbraceright}{\kern0pt}{\isacharparenright}{\kern0pt}{\isachardoublequoteclose}\isanewline
\ \ \ \ \ \ \isacommand{apply}\isamarkupfalse%
\ {\isacharparenleft}{\kern0pt}simp\ add{\isacharcolon}{\kern0pt}rank{\isacharunderscore}{\kern0pt}of{\isacharunderscore}{\kern0pt}def{\isacharparenright}{\kern0pt}\isanewline
\ \ \ \ \ \ \isacommand{apply}\isamarkupfalse%
\ {\isacharparenleft}{\kern0pt}rule\ card{\isacharunderscore}{\kern0pt}mono{\isacharcomma}{\kern0pt}\ simp{\isacharparenright}{\kern0pt}\isanewline
\ \ \ \ \ \ \isacommand{apply}\isamarkupfalse%
\ {\isacharparenleft}{\kern0pt}rule\ subsetI{\isacharcomma}{\kern0pt}\ simp{\isacharparenright}{\kern0pt}\isanewline
\ \ \ \ \ \ \isacommand{by}\isamarkupfalse%
\ {\isacharparenleft}{\kern0pt}metis\ {\isacharparenleft}{\kern0pt}no{\isacharunderscore}{\kern0pt}types{\isacharcomma}{\kern0pt}\ lifting{\isacharparenright}{\kern0pt}\ imageI\ in{\isacharunderscore}{\kern0pt}set{\isacharunderscore}{\kern0pt}conv{\isacharunderscore}{\kern0pt}nth\ mem{\isacharunderscore}{\kern0pt}Collect{\isacharunderscore}{\kern0pt}eq{\isacharparenright}{\kern0pt}\isanewline
\ \ \ \ \isacommand{also}\isamarkupfalse%
\ \isacommand{have}\isamarkupfalse%
\ {\isachardoublequoteopen}{\isachardot}{\kern0pt}{\isachardot}{\kern0pt}{\isachardot}{\kern0pt}\ {\isasymle}\ card\ {\isacharbraceleft}{\kern0pt}k{\isachardot}{\kern0pt}\ k\ {\isacharless}{\kern0pt}\ length\ xs\ {\isasymand}\ xs\ {\isacharbang}{\kern0pt}\ k\ {\isacharless}{\kern0pt}\ xs\ {\isacharbang}{\kern0pt}\ y{\isacharbraceright}{\kern0pt}{\isachardoublequoteclose}\isanewline
\ \ \ \ \ \ \isacommand{by}\isamarkupfalse%
\ {\isacharparenleft}{\kern0pt}rule\ card{\isacharunderscore}{\kern0pt}image{\isacharunderscore}{\kern0pt}le{\isacharcomma}{\kern0pt}\ simp{\isacharparenright}{\kern0pt}\isanewline
\ \ \ \ \isacommand{also}\isamarkupfalse%
\ \isacommand{have}\isamarkupfalse%
\ {\isachardoublequoteopen}{\isachardot}{\kern0pt}{\isachardot}{\kern0pt}{\isachardot}{\kern0pt}\ {\isasymle}\ card\ {\isacharbraceleft}{\kern0pt}k{\isachardot}{\kern0pt}\ k\ {\isacharless}{\kern0pt}\ y{\isacharbraceright}{\kern0pt}{\isachardoublequoteclose}\isanewline
\ \ \ \ \ \ \isacommand{apply}\isamarkupfalse%
\ {\isacharparenleft}{\kern0pt}rule\ card{\isacharunderscore}{\kern0pt}mono{\isacharcomma}{\kern0pt}\ simp{\isacharparenright}{\kern0pt}\isanewline
\ \ \ \ \ \ \isacommand{apply}\isamarkupfalse%
\ {\isacharparenleft}{\kern0pt}rule\ subsetI{\isacharcomma}{\kern0pt}\ simp{\isacharparenright}{\kern0pt}\isanewline
\ \ \ \ \ \ \isacommand{apply}\isamarkupfalse%
\ {\isacharparenleft}{\kern0pt}rule\ ccontr{\isacharcomma}{\kern0pt}\ simp\ add{\isacharcolon}{\kern0pt}not{\isacharunderscore}{\kern0pt}less{\isacharparenright}{\kern0pt}\isanewline
\ \ \ \ \ \ \isacommand{by}\isamarkupfalse%
\ {\isacharparenleft}{\kern0pt}meson\ leD\ sorted{\isacharunderscore}{\kern0pt}iff{\isacharunderscore}{\kern0pt}nth{\isacharunderscore}{\kern0pt}mono\ s{\isacharunderscore}{\kern0pt}xs{\isacharparenright}{\kern0pt}\isanewline
\ \ \ \ \isacommand{also}\isamarkupfalse%
\ \isacommand{have}\isamarkupfalse%
\ {\isachardoublequoteopen}{\isachardot}{\kern0pt}{\isachardot}{\kern0pt}{\isachardot}{\kern0pt}\ {\isacharequal}{\kern0pt}\ y{\isachardoublequoteclose}\ \isacommand{by}\isamarkupfalse%
\ simp\isanewline
\ \ \ \ \isacommand{also}\isamarkupfalse%
\ \isacommand{assume}\isamarkupfalse%
\ {\isachardoublequoteopen}y\ {\isacharless}{\kern0pt}\ k\ {\isacharplus}{\kern0pt}\ {\isadigit{1}}{\isachardoublequoteclose}\isanewline
\ \ \ \ \isacommand{finally}\isamarkupfalse%
\ \isacommand{show}\isamarkupfalse%
\ {\isachardoublequoteopen}rank{\isacharunderscore}{\kern0pt}of\ {\isacharparenleft}{\kern0pt}xs\ {\isacharbang}{\kern0pt}\ y{\isacharparenright}{\kern0pt}\ {\isacharparenleft}{\kern0pt}set\ xs{\isacharparenright}{\kern0pt}\ {\isacharless}{\kern0pt}\ k{\isacharplus}{\kern0pt}{\isadigit{1}}{\isachardoublequoteclose}\ \isacommand{by}\isamarkupfalse%
\ simp\isanewline
\ \ \isacommand{qed}\isamarkupfalse%
\isanewline
\isanewline
\ \ \isacommand{have}\isamarkupfalse%
\ rank{\isacharunderscore}{\kern0pt}conv{\isacharcolon}{\kern0pt}\ {\isachardoublequoteopen}{\isasymAnd}y{\isachardot}{\kern0pt}\ y\ {\isacharless}{\kern0pt}\ length\ xs\ {\isasymLongrightarrow}\ rank{\isacharunderscore}{\kern0pt}of\ {\isacharparenleft}{\kern0pt}xs\ {\isacharbang}{\kern0pt}\ y{\isacharparenright}{\kern0pt}\ {\isacharparenleft}{\kern0pt}set\ xs{\isacharparenright}{\kern0pt}\ {\isacharless}{\kern0pt}\ k{\isacharplus}{\kern0pt}{\isadigit{1}}\ {\isasymlongleftrightarrow}\ y\ {\isacharless}{\kern0pt}\ k{\isacharplus}{\kern0pt}{\isadigit{1}}{\isachardoublequoteclose}\isanewline
\ \ \ \ \isacommand{using}\isamarkupfalse%
\ rank{\isacharunderscore}{\kern0pt}conv{\isacharunderscore}{\kern0pt}{\isadigit{1}}\ rank{\isacharunderscore}{\kern0pt}conv{\isacharunderscore}{\kern0pt}{\isadigit{2}}\ \isacommand{by}\isamarkupfalse%
\ blast\isanewline
\isanewline
\ \ \isacommand{have}\isamarkupfalse%
\ max{\isacharunderscore}{\kern0pt}{\isadigit{1}}{\isacharcolon}{\kern0pt}\ {\isachardoublequoteopen}{\isasymAnd}y{\isachardot}{\kern0pt}\ y\ {\isasymin}\ least\ {\isacharparenleft}{\kern0pt}k{\isacharplus}{\kern0pt}{\isadigit{1}}{\isacharparenright}{\kern0pt}\ {\isacharparenleft}{\kern0pt}set\ xs{\isacharparenright}{\kern0pt}\ {\isasymLongrightarrow}\ y\ {\isasymle}\ xs\ {\isacharbang}{\kern0pt}\ k{\isachardoublequoteclose}\ \isanewline
\ \ \isacommand{proof}\isamarkupfalse%
\ {\isacharminus}{\kern0pt}\isanewline
\ \ \ \ \isacommand{fix}\isamarkupfalse%
\ y\isanewline
\ \ \ \ \isacommand{assume}\isamarkupfalse%
\ a{\isacharcolon}{\kern0pt}{\isachardoublequoteopen}y\ {\isasymin}\ least\ {\isacharparenleft}{\kern0pt}k{\isacharplus}{\kern0pt}{\isadigit{1}}{\isacharparenright}{\kern0pt}\ {\isacharparenleft}{\kern0pt}set\ xs{\isacharparenright}{\kern0pt}{\isachardoublequoteclose}\isanewline
\ \ \ \ \isacommand{hence}\isamarkupfalse%
\ {\isachardoublequoteopen}y\ {\isasymin}\ set\ xs{\isachardoublequoteclose}\ \isacommand{using}\isamarkupfalse%
\ least{\isacharunderscore}{\kern0pt}subset\ \isacommand{by}\isamarkupfalse%
\ blast\isanewline
\ \ \ \ \isacommand{then}\isamarkupfalse%
\ \isacommand{obtain}\isamarkupfalse%
\ i\ \isakeyword{where}\ i{\isacharunderscore}{\kern0pt}bound{\isacharcolon}{\kern0pt}\ {\isachardoublequoteopen}i\ {\isacharless}{\kern0pt}\ length\ xs{\isachardoublequoteclose}\ \isakeyword{and}\ y{\isacharunderscore}{\kern0pt}def{\isacharcolon}{\kern0pt}\ {\isachardoublequoteopen}y\ {\isacharequal}{\kern0pt}\ xs\ {\isacharbang}{\kern0pt}\ i{\isachardoublequoteclose}\ \isacommand{using}\isamarkupfalse%
\ in{\isacharunderscore}{\kern0pt}set{\isacharunderscore}{\kern0pt}conv{\isacharunderscore}{\kern0pt}nth\ \isacommand{by}\isamarkupfalse%
\ metis\isanewline
\ \ \ \ \isacommand{hence}\isamarkupfalse%
\ {\isachardoublequoteopen}rank{\isacharunderscore}{\kern0pt}of\ {\isacharparenleft}{\kern0pt}xs\ {\isacharbang}{\kern0pt}\ i{\isacharparenright}{\kern0pt}\ {\isacharparenleft}{\kern0pt}set\ xs{\isacharparenright}{\kern0pt}\ {\isacharless}{\kern0pt}\ k{\isacharplus}{\kern0pt}{\isadigit{1}}{\isachardoublequoteclose}\isanewline
\ \ \ \ \ \ \isacommand{using}\isamarkupfalse%
\ a\ y{\isacharunderscore}{\kern0pt}def\ i{\isacharunderscore}{\kern0pt}bound\ \isacommand{by}\isamarkupfalse%
\ {\isacharparenleft}{\kern0pt}simp\ add{\isacharcolon}{\kern0pt}\ least{\isacharunderscore}{\kern0pt}def{\isacharparenright}{\kern0pt}\isanewline
\ \ \ \ \isacommand{hence}\isamarkupfalse%
\ {\isachardoublequoteopen}i\ {\isacharless}{\kern0pt}\ k{\isacharplus}{\kern0pt}{\isadigit{1}}{\isachardoublequoteclose}\isanewline
\ \ \ \ \ \ \isacommand{using}\isamarkupfalse%
\ rank{\isacharunderscore}{\kern0pt}conv\ i{\isacharunderscore}{\kern0pt}bound\ \isacommand{by}\isamarkupfalse%
\ blast\isanewline
\ \ \ \ \isacommand{hence}\isamarkupfalse%
\ {\isachardoublequoteopen}i\ {\isasymle}\ k{\isachardoublequoteclose}\ \isacommand{by}\isamarkupfalse%
\ linarith\isanewline
\ \ \ \ \isacommand{hence}\isamarkupfalse%
\ {\isachardoublequoteopen}xs\ {\isacharbang}{\kern0pt}\ i\ {\isasymle}\ xs\ {\isacharbang}{\kern0pt}\ k{\isachardoublequoteclose}\isanewline
\ \ \ \ \ \ \isacommand{using}\isamarkupfalse%
\ s{\isacharunderscore}{\kern0pt}xs\ i{\isacharunderscore}{\kern0pt}bound\ k{\isacharunderscore}{\kern0pt}bound\ sorted{\isacharunderscore}{\kern0pt}nth{\isacharunderscore}{\kern0pt}mono\ \isacommand{by}\isamarkupfalse%
\ blast\isanewline
\ \ \ \ \isacommand{thus}\isamarkupfalse%
\ {\isachardoublequoteopen}y\ {\isasymle}\ xs\ {\isacharbang}{\kern0pt}\ k{\isachardoublequoteclose}\ \isacommand{using}\isamarkupfalse%
\ y{\isacharunderscore}{\kern0pt}def\ \isacommand{by}\isamarkupfalse%
\ simp\isanewline
\ \ \isacommand{qed}\isamarkupfalse%
\isanewline
\isanewline
\ \ \isacommand{have}\isamarkupfalse%
\ max{\isacharunderscore}{\kern0pt}{\isadigit{2}}{\isacharcolon}{\kern0pt}{\isachardoublequoteopen}xs\ {\isacharbang}{\kern0pt}\ k\ {\isasymin}\ least\ {\isacharparenleft}{\kern0pt}k{\isacharplus}{\kern0pt}{\isadigit{1}}{\isacharparenright}{\kern0pt}\ {\isacharparenleft}{\kern0pt}set\ xs{\isacharparenright}{\kern0pt}{\isachardoublequoteclose}\isanewline
\ \ \ \ \isacommand{apply}\isamarkupfalse%
\ {\isacharparenleft}{\kern0pt}simp\ add{\isacharcolon}{\kern0pt}least{\isacharunderscore}{\kern0pt}def{\isacharparenright}{\kern0pt}\isanewline
\ \ \ \ \isacommand{using}\isamarkupfalse%
\ k{\isacharunderscore}{\kern0pt}bound\ rank{\isacharunderscore}{\kern0pt}conv\ \isacommand{by}\isamarkupfalse%
\ simp\isanewline
\isanewline
\ \ \isacommand{have}\isamarkupfalse%
\ r{\isacharunderscore}{\kern0pt}{\isadigit{1}}{\isacharcolon}{\kern0pt}\ {\isachardoublequoteopen}Max\ {\isacharparenleft}{\kern0pt}least\ {\isacharparenleft}{\kern0pt}k{\isacharplus}{\kern0pt}{\isadigit{1}}{\isacharparenright}{\kern0pt}\ {\isacharparenleft}{\kern0pt}set\ xs{\isacharparenright}{\kern0pt}{\isacharparenright}{\kern0pt}\ {\isacharequal}{\kern0pt}\ xs\ {\isacharbang}{\kern0pt}\ k{\isachardoublequoteclose}\isanewline
\ \ \ \ \isacommand{apply}\isamarkupfalse%
\ {\isacharparenleft}{\kern0pt}rule\ Max{\isacharunderscore}{\kern0pt}eqI{\isacharcomma}{\kern0pt}\ rule\ finite{\isacharunderscore}{\kern0pt}subset{\isacharbrackleft}{\kern0pt}OF\ least{\isacharunderscore}{\kern0pt}subset{\isacharbrackright}{\kern0pt}{\isacharcomma}{\kern0pt}\ simp{\isacharparenright}{\kern0pt}\isanewline
\ \ \ \ \ \isacommand{apply}\isamarkupfalse%
\ {\isacharparenleft}{\kern0pt}metis\ max{\isacharunderscore}{\kern0pt}{\isadigit{1}}{\isacharparenright}{\kern0pt}\isanewline
\ \ \ \ \isacommand{by}\isamarkupfalse%
\ {\isacharparenleft}{\kern0pt}metis\ max{\isacharunderscore}{\kern0pt}{\isadigit{2}}{\isacharparenright}{\kern0pt}\isanewline
\isanewline
\ \ \isacommand{have}\isamarkupfalse%
\ {\isachardoublequoteopen}k\ {\isacharplus}{\kern0pt}\ {\isadigit{1}}\ {\isacharequal}{\kern0pt}\ card\ {\isacharparenleft}{\kern0pt}{\isacharparenleft}{\kern0pt}{\isasymlambda}i{\isachardot}{\kern0pt}\ xs\ {\isacharbang}{\kern0pt}\ i{\isacharparenright}{\kern0pt}\ {\isacharbackquote}{\kern0pt}\ {\isacharbraceleft}{\kern0pt}{\isadigit{0}}{\isachardot}{\kern0pt}{\isachardot}{\kern0pt}k{\isacharbraceright}{\kern0pt}{\isacharparenright}{\kern0pt}{\isachardoublequoteclose}\ \isanewline
\ \ \ \ \isacommand{by}\isamarkupfalse%
\ {\isacharparenleft}{\kern0pt}subst\ card{\isacharunderscore}{\kern0pt}image{\isacharbrackleft}{\kern0pt}OF\ inj{\isacharunderscore}{\kern0pt}xs{\isacharbrackright}{\kern0pt}{\isacharcomma}{\kern0pt}\ simp{\isacharparenright}{\kern0pt}\ \isanewline
\ \ \isacommand{also}\isamarkupfalse%
\ \isacommand{have}\isamarkupfalse%
\ {\isachardoublequoteopen}{\isachardot}{\kern0pt}{\isachardot}{\kern0pt}{\isachardot}{\kern0pt}\ {\isasymle}\ card\ {\isacharparenleft}{\kern0pt}least\ {\isacharparenleft}{\kern0pt}k{\isacharplus}{\kern0pt}{\isadigit{1}}{\isacharparenright}{\kern0pt}\ {\isacharparenleft}{\kern0pt}set\ xs{\isacharparenright}{\kern0pt}{\isacharparenright}{\kern0pt}{\isachardoublequoteclose}\isanewline
\ \ \ \ \isacommand{apply}\isamarkupfalse%
\ {\isacharparenleft}{\kern0pt}rule\ card{\isacharunderscore}{\kern0pt}mono{\isacharcomma}{\kern0pt}\ rule\ finite{\isacharunderscore}{\kern0pt}subset{\isacharbrackleft}{\kern0pt}OF\ least{\isacharunderscore}{\kern0pt}subset{\isacharbrackright}{\kern0pt}{\isacharcomma}{\kern0pt}\ simp{\isacharparenright}{\kern0pt}\isanewline
\ \ \ \ \isacommand{apply}\isamarkupfalse%
\ {\isacharparenleft}{\kern0pt}rule\ image{\isacharunderscore}{\kern0pt}subsetI{\isacharparenright}{\kern0pt}\isanewline
\ \ \ \ \isacommand{apply}\isamarkupfalse%
\ {\isacharparenleft}{\kern0pt}simp\ add{\isacharcolon}{\kern0pt}least{\isacharunderscore}{\kern0pt}def{\isacharparenright}{\kern0pt}\isanewline
\ \ \ \ \isacommand{using}\isamarkupfalse%
\ rank{\isacharunderscore}{\kern0pt}conv\ k{\isacharunderscore}{\kern0pt}bound\ \isacommand{by}\isamarkupfalse%
\ simp\isanewline
\ \ \isacommand{finally}\isamarkupfalse%
\ \isacommand{have}\isamarkupfalse%
\ {\isachardoublequoteopen}card\ {\isacharparenleft}{\kern0pt}least\ {\isacharparenleft}{\kern0pt}k{\isacharplus}{\kern0pt}{\isadigit{1}}{\isacharparenright}{\kern0pt}\ {\isacharparenleft}{\kern0pt}set\ xs{\isacharparenright}{\kern0pt}{\isacharparenright}{\kern0pt}\ {\isasymge}\ k{\isacharplus}{\kern0pt}{\isadigit{1}}{\isachardoublequoteclose}\ \isacommand{by}\isamarkupfalse%
\ simp\isanewline
\ \ \isacommand{moreover}\isamarkupfalse%
\ \isacommand{have}\isamarkupfalse%
\ {\isachardoublequoteopen}card\ {\isacharparenleft}{\kern0pt}least\ {\isacharparenleft}{\kern0pt}k{\isacharplus}{\kern0pt}{\isadigit{1}}{\isacharparenright}{\kern0pt}\ {\isacharparenleft}{\kern0pt}set\ xs{\isacharparenright}{\kern0pt}{\isacharparenright}{\kern0pt}\ {\isasymle}\ k{\isacharplus}{\kern0pt}{\isadigit{1}}{\isachardoublequoteclose}\isanewline
\ \ \ \ \isacommand{by}\isamarkupfalse%
\ {\isacharparenleft}{\kern0pt}subst\ card{\isacharunderscore}{\kern0pt}least{\isacharcomma}{\kern0pt}\ simp{\isacharcomma}{\kern0pt}\ simp{\isacharparenright}{\kern0pt}\isanewline
\ \ \isacommand{ultimately}\isamarkupfalse%
\ \isacommand{have}\isamarkupfalse%
\ r{\isacharunderscore}{\kern0pt}{\isadigit{2}}{\isacharcolon}{\kern0pt}\ {\isachardoublequoteopen}card\ {\isacharparenleft}{\kern0pt}least\ {\isacharparenleft}{\kern0pt}k{\isacharplus}{\kern0pt}{\isadigit{1}}{\isacharparenright}{\kern0pt}\ {\isacharparenleft}{\kern0pt}set\ xs{\isacharparenright}{\kern0pt}{\isacharparenright}{\kern0pt}\ {\isacharequal}{\kern0pt}\ k{\isacharplus}{\kern0pt}{\isadigit{1}}{\isachardoublequoteclose}\ \isacommand{by}\isamarkupfalse%
\ simp\isanewline
\isanewline
\ \ \isacommand{show}\isamarkupfalse%
\ {\isachardoublequoteopen}nth{\isacharunderscore}{\kern0pt}mset\ k\ A\ {\isacharequal}{\kern0pt}\ Max\ {\isacharparenleft}{\kern0pt}least\ {\isacharparenleft}{\kern0pt}k{\isacharplus}{\kern0pt}{\isadigit{1}}{\isacharparenright}{\kern0pt}\ {\isacharparenleft}{\kern0pt}set{\isacharunderscore}{\kern0pt}mset\ A{\isacharparenright}{\kern0pt}{\isacharparenright}{\kern0pt}{\isachardoublequoteclose}\ \isanewline
\ \ \ \ \isacommand{apply}\isamarkupfalse%
\ {\isacharparenleft}{\kern0pt}simp\ add{\isacharcolon}{\kern0pt}nth{\isacharunderscore}{\kern0pt}mset{\isacharunderscore}{\kern0pt}def\ xs{\isacharunderscore}{\kern0pt}def{\isacharbrackleft}{\kern0pt}symmetric{\isacharbrackright}{\kern0pt}\ r{\isacharunderscore}{\kern0pt}{\isadigit{1}}{\isacharbrackleft}{\kern0pt}symmetric{\isacharbrackright}{\kern0pt}{\isacharparenright}{\kern0pt}\isanewline
\ \ \ \ \isacommand{by}\isamarkupfalse%
\ {\isacharparenleft}{\kern0pt}simp\ add{\isacharcolon}{\kern0pt}A{\isacharunderscore}{\kern0pt}def{\isacharparenright}{\kern0pt}\isanewline
\isanewline
\ \ \isacommand{show}\isamarkupfalse%
\ {\isachardoublequoteopen}card\ {\isacharparenleft}{\kern0pt}least\ {\isacharparenleft}{\kern0pt}k{\isacharplus}{\kern0pt}{\isadigit{1}}{\isacharparenright}{\kern0pt}\ {\isacharparenleft}{\kern0pt}set{\isacharunderscore}{\kern0pt}mset\ A{\isacharparenright}{\kern0pt}{\isacharparenright}{\kern0pt}\ {\isacharequal}{\kern0pt}\ k{\isacharplus}{\kern0pt}{\isadigit{1}}{\isachardoublequoteclose}\ \isanewline
\ \ \ \ \isacommand{using}\isamarkupfalse%
\ r{\isacharunderscore}{\kern0pt}{\isadigit{2}}\ \isacommand{by}\isamarkupfalse%
\ {\isacharparenleft}{\kern0pt}simp\ add{\isacharcolon}{\kern0pt}A{\isacharunderscore}{\kern0pt}def{\isacharparenright}{\kern0pt}\isanewline
\isacommand{qed}\isamarkupfalse%
%
\endisatagproof
{\isafoldproof}%
%
\isadelimproof
\isanewline
%
\endisadelimproof
%
\isadelimtheory
\isanewline
%
\endisadelimtheory
%
\isatagtheory
\isacommand{end}\isamarkupfalse%
%
\endisatagtheory
{\isafoldtheory}%
%
\isadelimtheory
%
\endisadelimtheory
%
\end{isabellebody}%
\endinput
%:%file=OrderStatistics.tex%:%
%:%11=1%:%
%:%27=3%:%
%:%28=3%:%
%:%29=4%:%
%:%30=5%:%
%:%39=7%:%
%:%41=10%:%
%:%42=10%:%
%:%44=11%:%
%:%46=13%:%
%:%47=13%:%
%:%48=14%:%
%:%49=15%:%
%:%52=16%:%
%:%56=16%:%
%:%57=16%:%
%:%58=17%:%
%:%59=17%:%
%:%60=18%:%
%:%61=18%:%
%:%62=18%:%
%:%63=19%:%
%:%64=19%:%
%:%69=19%:%
%:%72=20%:%
%:%73=21%:%
%:%74=21%:%
%:%75=22%:%
%:%76=23%:%
%:%77=24%:%
%:%78=25%:%
%:%79=26%:%
%:%86=27%:%
%:%87=27%:%
%:%88=28%:%
%:%89=28%:%
%:%90=29%:%
%:%91=29%:%
%:%92=30%:%
%:%93=30%:%
%:%94=31%:%
%:%95=31%:%
%:%96=32%:%
%:%97=32%:%
%:%98=33%:%
%:%99=33%:%
%:%100=33%:%
%:%101=34%:%
%:%102=34%:%
%:%103=35%:%
%:%104=35%:%
%:%105=35%:%
%:%106=36%:%
%:%107=36%:%
%:%108=36%:%
%:%109=37%:%
%:%110=37%:%
%:%111=38%:%
%:%112=38%:%
%:%113=39%:%
%:%114=39%:%
%:%115=40%:%
%:%116=40%:%
%:%117=40%:%
%:%118=41%:%
%:%119=41%:%
%:%120=41%:%
%:%121=42%:%
%:%122=42%:%
%:%123=43%:%
%:%124=43%:%
%:%125=43%:%
%:%126=44%:%
%:%127=44%:%
%:%128=45%:%
%:%134=45%:%
%:%137=46%:%
%:%138=47%:%
%:%139=48%:%
%:%140=48%:%
%:%142=49%:%
%:%144=51%:%
%:%145=51%:%
%:%146=52%:%
%:%147=53%:%
%:%154=54%:%
%:%155=54%:%
%:%156=55%:%
%:%157=55%:%
%:%158=56%:%
%:%159=56%:%
%:%160=57%:%
%:%161=57%:%
%:%162=58%:%
%:%163=58%:%
%:%164=59%:%
%:%165=59%:%
%:%166=60%:%
%:%167=60%:%
%:%168=61%:%
%:%169=61%:%
%:%170=62%:%
%:%171=63%:%
%:%172=63%:%
%:%173=64%:%
%:%174=64%:%
%:%175=65%:%
%:%181=65%:%
%:%184=66%:%
%:%185=67%:%
%:%186=68%:%
%:%187=68%:%
%:%188=69%:%
%:%189=70%:%
%:%192=71%:%
%:%196=71%:%
%:%197=71%:%
%:%198=72%:%
%:%199=72%:%
%:%200=73%:%
%:%201=73%:%
%:%202=74%:%
%:%203=74%:%
%:%204=75%:%
%:%205=75%:%
%:%206=76%:%
%:%207=76%:%
%:%208=77%:%
%:%209=77%:%
%:%214=77%:%
%:%217=78%:%
%:%218=79%:%
%:%219=79%:%
%:%220=80%:%
%:%221=81%:%
%:%228=82%:%
%:%229=82%:%
%:%230=83%:%
%:%231=83%:%
%:%232=84%:%
%:%233=84%:%
%:%234=85%:%
%:%235=85%:%
%:%236=86%:%
%:%237=86%:%
%:%238=87%:%
%:%239=87%:%
%:%240=88%:%
%:%241=88%:%
%:%242=89%:%
%:%243=89%:%
%:%244=90%:%
%:%245=90%:%
%:%246=91%:%
%:%247=91%:%
%:%248=91%:%
%:%249=91%:%
%:%250=91%:%
%:%251=92%:%
%:%252=92%:%
%:%253=93%:%
%:%254=93%:%
%:%255=94%:%
%:%256=94%:%
%:%257=94%:%
%:%258=94%:%
%:%259=95%:%
%:%260=95%:%
%:%261=96%:%
%:%262=96%:%
%:%263=97%:%
%:%264=97%:%
%:%265=98%:%
%:%266=98%:%
%:%267=99%:%
%:%268=99%:%
%:%269=99%:%
%:%270=100%:%
%:%271=100%:%
%:%272=101%:%
%:%273=101%:%
%:%274=102%:%
%:%275=102%:%
%:%276=102%:%
%:%277=102%:%
%:%278=102%:%
%:%279=103%:%
%:%285=103%:%
%:%288=104%:%
%:%289=105%:%
%:%290=105%:%
%:%293=106%:%
%:%297=106%:%
%:%298=106%:%
%:%303=106%:%
%:%306=107%:%
%:%307=108%:%
%:%308=108%:%
%:%309=109%:%
%:%310=110%:%
%:%317=111%:%
%:%318=111%:%
%:%319=112%:%
%:%320=112%:%
%:%321=113%:%
%:%322=113%:%
%:%323=114%:%
%:%324=114%:%
%:%325=115%:%
%:%326=115%:%
%:%327=116%:%
%:%328=116%:%
%:%329=117%:%
%:%330=117%:%
%:%331=118%:%
%:%332=118%:%
%:%333=118%:%
%:%334=119%:%
%:%335=119%:%
%:%336=120%:%
%:%337=120%:%
%:%338=121%:%
%:%339=121%:%
%:%340=122%:%
%:%341=122%:%
%:%342=123%:%
%:%343=123%:%
%:%344=124%:%
%:%345=124%:%
%:%346=125%:%
%:%347=125%:%
%:%348=125%:%
%:%349=125%:%
%:%350=125%:%
%:%351=126%:%
%:%352=126%:%
%:%353=127%:%
%:%354=127%:%
%:%355=128%:%
%:%356=128%:%
%:%357=129%:%
%:%358=129%:%
%:%359=130%:%
%:%360=130%:%
%:%361=131%:%
%:%362=131%:%
%:%363=132%:%
%:%364=132%:%
%:%365=133%:%
%:%366=133%:%
%:%367=134%:%
%:%368=134%:%
%:%369=134%:%
%:%370=134%:%
%:%371=135%:%
%:%377=135%:%
%:%380=136%:%
%:%381=137%:%
%:%382=137%:%
%:%383=138%:%
%:%384=139%:%
%:%391=140%:%
%:%392=140%:%
%:%393=141%:%
%:%394=141%:%
%:%395=142%:%
%:%396=142%:%
%:%397=143%:%
%:%398=143%:%
%:%399=144%:%
%:%400=144%:%
%:%401=145%:%
%:%402=145%:%
%:%403=146%:%
%:%404=146%:%
%:%405=147%:%
%:%406=147%:%
%:%407=148%:%
%:%408=148%:%
%:%409=149%:%
%:%410=149%:%
%:%411=149%:%
%:%412=150%:%
%:%413=150%:%
%:%414=151%:%
%:%415=151%:%
%:%416=151%:%
%:%417=152%:%
%:%418=152%:%
%:%419=152%:%
%:%420=153%:%
%:%421=153%:%
%:%422=154%:%
%:%423=154%:%
%:%424=154%:%
%:%425=155%:%
%:%426=155%:%
%:%427=156%:%
%:%428=156%:%
%:%429=157%:%
%:%430=157%:%
%:%431=158%:%
%:%432=158%:%
%:%433=159%:%
%:%434=159%:%
%:%435=160%:%
%:%436=160%:%
%:%437=161%:%
%:%438=162%:%
%:%439=162%:%
%:%440=163%:%
%:%441=163%:%
%:%442=164%:%
%:%443=164%:%
%:%444=165%:%
%:%445=165%:%
%:%446=166%:%
%:%452=166%:%
%:%455=167%:%
%:%456=168%:%
%:%457=168%:%
%:%458=169%:%
%:%459=170%:%
%:%460=171%:%
%:%467=172%:%
%:%468=172%:%
%:%469=173%:%
%:%470=173%:%
%:%471=174%:%
%:%472=174%:%
%:%473=174%:%
%:%474=175%:%
%:%475=175%:%
%:%476=176%:%
%:%477=176%:%
%:%478=177%:%
%:%479=177%:%
%:%480=178%:%
%:%481=178%:%
%:%482=179%:%
%:%483=179%:%
%:%484=180%:%
%:%485=181%:%
%:%486=181%:%
%:%487=182%:%
%:%488=182%:%
%:%489=183%:%
%:%490=183%:%
%:%491=184%:%
%:%492=184%:%
%:%493=185%:%
%:%494=185%:%
%:%495=186%:%
%:%501=186%:%
%:%504=187%:%
%:%505=188%:%
%:%506=188%:%
%:%507=189%:%
%:%508=190%:%
%:%515=191%:%
%:%516=191%:%
%:%517=192%:%
%:%518=192%:%
%:%519=192%:%
%:%520=193%:%
%:%521=193%:%
%:%522=194%:%
%:%523=194%:%
%:%524=195%:%
%:%525=195%:%
%:%526=195%:%
%:%527=196%:%
%:%528=196%:%
%:%529=197%:%
%:%530=197%:%
%:%531=198%:%
%:%532=198%:%
%:%533=199%:%
%:%534=199%:%
%:%535=200%:%
%:%536=200%:%
%:%537=201%:%
%:%538=201%:%
%:%539=201%:%
%:%540=202%:%
%:%541=202%:%
%:%542=203%:%
%:%543=203%:%
%:%544=204%:%
%:%545=204%:%
%:%546=205%:%
%:%547=205%:%
%:%548=206%:%
%:%549=206%:%
%:%550=207%:%
%:%551=207%:%
%:%552=208%:%
%:%553=208%:%
%:%554=209%:%
%:%555=210%:%
%:%556=210%:%
%:%557=211%:%
%:%558=211%:%
%:%559=212%:%
%:%560=212%:%
%:%561=213%:%
%:%562=213%:%
%:%563=214%:%
%:%564=214%:%
%:%565=215%:%
%:%566=215%:%
%:%567=216%:%
%:%568=216%:%
%:%569=217%:%
%:%570=217%:%
%:%571=218%:%
%:%572=218%:%
%:%573=218%:%
%:%574=219%:%
%:%575=219%:%
%:%576=220%:%
%:%577=221%:%
%:%578=221%:%
%:%579=222%:%
%:%580=222%:%
%:%581=223%:%
%:%582=223%:%
%:%583=224%:%
%:%584=224%:%
%:%585=224%:%
%:%586=225%:%
%:%592=225%:%
%:%595=226%:%
%:%596=227%:%
%:%597=227%:%
%:%598=228%:%
%:%599=228%:%
%:%600=229%:%
%:%601=230%:%
%:%602=230%:%
%:%603=231%:%
%:%604=232%:%
%:%605=233%:%
%:%606=233%:%
%:%607=234%:%
%:%608=235%:%
%:%609=236%:%
%:%616=237%:%
%:%617=237%:%
%:%618=238%:%
%:%619=238%:%
%:%620=239%:%
%:%621=239%:%
%:%622=239%:%
%:%623=240%:%
%:%624=240%:%
%:%625=240%:%
%:%626=241%:%
%:%627=241%:%
%:%628=242%:%
%:%629=242%:%
%:%630=242%:%
%:%631=243%:%
%:%632=243%:%
%:%633=244%:%
%:%634=244%:%
%:%635=244%:%
%:%636=245%:%
%:%637=246%:%
%:%638=246%:%
%:%639=247%:%
%:%640=247%:%
%:%641=247%:%
%:%642=248%:%
%:%643=248%:%
%:%644=248%:%
%:%645=249%:%
%:%646=250%:%
%:%647=250%:%
%:%648=250%:%
%:%649=251%:%
%:%650=251%:%
%:%651=251%:%
%:%652=252%:%
%:%653=252%:%
%:%654=253%:%
%:%655=253%:%
%:%656=254%:%
%:%657=254%:%
%:%658=254%:%
%:%659=255%:%
%:%660=255%:%
%:%661=255%:%
%:%662=256%:%
%:%663=256%:%
%:%664=257%:%
%:%665=257%:%
%:%666=258%:%
%:%667=258%:%
%:%668=259%:%
%:%669=259%:%
%:%670=259%:%
%:%671=260%:%
%:%672=260%:%
%:%673=260%:%
%:%674=261%:%
%:%675=261%:%
%:%676=261%:%
%:%677=261%:%
%:%678=262%:%
%:%684=262%:%
%:%687=263%:%
%:%688=264%:%
%:%689=264%:%
%:%690=265%:%
%:%691=266%:%
%:%692=267%:%
%:%699=268%:%
%:%700=268%:%
%:%701=269%:%
%:%702=269%:%
%:%703=270%:%
%:%704=270%:%
%:%705=270%:%
%:%706=271%:%
%:%707=271%:%
%:%708=271%:%
%:%709=272%:%
%:%710=272%:%
%:%711=273%:%
%:%712=273%:%
%:%713=273%:%
%:%714=274%:%
%:%715=274%:%
%:%716=275%:%
%:%717=275%:%
%:%718=275%:%
%:%719=276%:%
%:%720=277%:%
%:%721=277%:%
%:%722=278%:%
%:%723=278%:%
%:%724=278%:%
%:%725=279%:%
%:%726=279%:%
%:%727=279%:%
%:%728=280%:%
%:%729=281%:%
%:%730=281%:%
%:%731=281%:%
%:%732=282%:%
%:%733=282%:%
%:%734=282%:%
%:%735=283%:%
%:%736=283%:%
%:%737=284%:%
%:%738=284%:%
%:%739=285%:%
%:%740=285%:%
%:%741=285%:%
%:%742=286%:%
%:%743=286%:%
%:%744=286%:%
%:%745=287%:%
%:%746=287%:%
%:%747=288%:%
%:%748=288%:%
%:%749=288%:%
%:%750=289%:%
%:%751=289%:%
%:%752=290%:%
%:%753=290%:%
%:%754=290%:%
%:%755=291%:%
%:%756=291%:%
%:%757=292%:%
%:%758=292%:%
%:%759=293%:%
%:%760=293%:%
%:%761=294%:%
%:%762=294%:%
%:%763=294%:%
%:%764=295%:%
%:%765=295%:%
%:%766=295%:%
%:%767=296%:%
%:%768=296%:%
%:%769=296%:%
%:%770=296%:%
%:%771=297%:%
%:%777=297%:%
%:%780=298%:%
%:%781=299%:%
%:%782=299%:%
%:%783=300%:%
%:%784=301%:%
%:%785=302%:%
%:%792=303%:%
%:%793=303%:%
%:%794=304%:%
%:%795=304%:%
%:%796=305%:%
%:%797=305%:%
%:%798=305%:%
%:%799=306%:%
%:%800=306%:%
%:%801=306%:%
%:%802=307%:%
%:%803=307%:%
%:%804=308%:%
%:%805=308%:%
%:%806=308%:%
%:%807=309%:%
%:%808=310%:%
%:%809=310%:%
%:%810=311%:%
%:%811=311%:%
%:%812=311%:%
%:%813=312%:%
%:%814=312%:%
%:%815=313%:%
%:%816=313%:%
%:%817=314%:%
%:%818=314%:%
%:%819=315%:%
%:%820=315%:%
%:%821=315%:%
%:%822=316%:%
%:%823=316%:%
%:%824=317%:%
%:%825=317%:%
%:%826=318%:%
%:%827=318%:%
%:%828=319%:%
%:%829=319%:%
%:%830=320%:%
%:%831=320%:%
%:%832=321%:%
%:%833=321%:%
%:%834=321%:%
%:%835=322%:%
%:%836=322%:%
%:%837=323%:%
%:%838=323%:%
%:%839=324%:%
%:%840=324%:%
%:%841=324%:%
%:%842=324%:%
%:%843=325%:%
%:%844=325%:%
%:%845=325%:%
%:%846=325%:%
%:%847=326%:%
%:%848=326%:%
%:%849=326%:%
%:%850=326%:%
%:%851=327%:%
%:%857=327%:%
%:%860=328%:%
%:%861=329%:%
%:%862=329%:%
%:%863=330%:%
%:%864=331%:%
%:%865=332%:%
%:%872=333%:%
%:%873=333%:%
%:%874=334%:%
%:%875=334%:%
%:%876=335%:%
%:%877=335%:%
%:%878=336%:%
%:%879=336%:%
%:%880=337%:%
%:%881=337%:%
%:%882=337%:%
%:%883=338%:%
%:%889=338%:%
%:%892=339%:%
%:%893=340%:%
%:%894=340%:%
%:%895=341%:%
%:%896=342%:%
%:%897=343%:%
%:%904=344%:%
%:%905=344%:%
%:%906=345%:%
%:%907=345%:%
%:%908=346%:%
%:%909=346%:%
%:%910=346%:%
%:%911=347%:%
%:%912=347%:%
%:%913=348%:%
%:%914=349%:%
%:%915=349%:%
%:%916=349%:%
%:%917=350%:%
%:%918=350%:%
%:%919=350%:%
%:%920=351%:%
%:%921=351%:%
%:%922=352%:%
%:%923=352%:%
%:%924=352%:%
%:%925=353%:%
%:%926=353%:%
%:%927=354%:%
%:%928=355%:%
%:%929=355%:%
%:%930=356%:%
%:%931=356%:%
%:%932=357%:%
%:%933=357%:%
%:%934=358%:%
%:%935=358%:%
%:%936=359%:%
%:%937=360%:%
%:%938=361%:%
%:%939=361%:%
%:%940=362%:%
%:%941=362%:%
%:%942=363%:%
%:%943=363%:%
%:%944=364%:%
%:%945=364%:%
%:%946=365%:%
%:%947=365%:%
%:%948=366%:%
%:%949=366%:%
%:%950=366%:%
%:%951=367%:%
%:%952=368%:%
%:%953=368%:%
%:%954=368%:%
%:%955=368%:%
%:%956=369%:%
%:%957=370%:%
%:%958=370%:%
%:%959=371%:%
%:%960=371%:%
%:%961=372%:%
%:%962=372%:%
%:%963=372%:%
%:%964=373%:%
%:%965=373%:%
%:%966=374%:%
%:%967=374%:%
%:%968=375%:%
%:%969=375%:%
%:%970=376%:%
%:%971=376%:%
%:%972=376%:%
%:%973=376%:%
%:%974=377%:%
%:%975=377%:%
%:%976=378%:%
%:%977=378%:%
%:%978=378%:%
%:%979=379%:%
%:%980=379%:%
%:%981=380%:%
%:%982=380%:%
%:%983=381%:%
%:%984=381%:%
%:%985=382%:%
%:%986=382%:%
%:%987=383%:%
%:%988=383%:%
%:%989=383%:%
%:%990=384%:%
%:%991=384%:%
%:%992=384%:%
%:%993=384%:%
%:%994=385%:%
%:%995=385%:%
%:%996=386%:%
%:%997=387%:%
%:%998=387%:%
%:%999=388%:%
%:%1000=388%:%
%:%1001=389%:%
%:%1002=389%:%
%:%1003=390%:%
%:%1004=390%:%
%:%1005=391%:%
%:%1006=391%:%
%:%1007=392%:%
%:%1008=392%:%
%:%1009=393%:%
%:%1010=393%:%
%:%1011=394%:%
%:%1012=394%:%
%:%1013=395%:%
%:%1014=395%:%
%:%1015=395%:%
%:%1016=396%:%
%:%1017=396%:%
%:%1018=397%:%
%:%1019=397%:%
%:%1020=397%:%
%:%1021=398%:%
%:%1022=398%:%
%:%1023=399%:%
%:%1024=399%:%
%:%1025=400%:%
%:%1026=400%:%
%:%1027=401%:%
%:%1028=401%:%
%:%1029=402%:%
%:%1030=402%:%
%:%1031=402%:%
%:%1032=402%:%
%:%1033=403%:%
%:%1034=403%:%
%:%1035=403%:%
%:%1036=404%:%
%:%1037=404%:%
%:%1038=404%:%
%:%1039=404%:%
%:%1040=405%:%
%:%1041=405%:%
%:%1042=406%:%
%:%1043=407%:%
%:%1044=407%:%
%:%1045=408%:%
%:%1046=408%:%
%:%1047=408%:%
%:%1048=409%:%
%:%1049=410%:%
%:%1050=410%:%
%:%1051=411%:%
%:%1052=411%:%
%:%1053=412%:%
%:%1054=412%:%
%:%1055=413%:%
%:%1056=413%:%
%:%1057=414%:%
%:%1058=414%:%
%:%1059=414%:%
%:%1060=414%:%
%:%1061=415%:%
%:%1062=415%:%
%:%1063=415%:%
%:%1064=415%:%
%:%1065=415%:%
%:%1066=416%:%
%:%1067=416%:%
%:%1068=417%:%
%:%1069=417%:%
%:%1070=417%:%
%:%1071=418%:%
%:%1072=418%:%
%:%1073=419%:%
%:%1074=419%:%
%:%1075=419%:%
%:%1076=420%:%
%:%1077=420%:%
%:%1078=420%:%
%:%1079=421%:%
%:%1080=421%:%
%:%1081=422%:%
%:%1082=422%:%
%:%1083=422%:%
%:%1084=423%:%
%:%1085=423%:%
%:%1086=423%:%
%:%1087=423%:%
%:%1088=424%:%
%:%1089=424%:%
%:%1090=425%:%
%:%1091=426%:%
%:%1092=426%:%
%:%1093=427%:%
%:%1094=427%:%
%:%1095=428%:%
%:%1096=428%:%
%:%1097=428%:%
%:%1098=429%:%
%:%1099=430%:%
%:%1100=430%:%
%:%1101=431%:%
%:%1102=431%:%
%:%1103=432%:%
%:%1104=432%:%
%:%1105=433%:%
%:%1106=433%:%
%:%1107=434%:%
%:%1108=435%:%
%:%1109=435%:%
%:%1110=436%:%
%:%1111=436%:%
%:%1112=437%:%
%:%1113=437%:%
%:%1114=437%:%
%:%1115=438%:%
%:%1116=438%:%
%:%1117=439%:%
%:%1118=439%:%
%:%1119=440%:%
%:%1120=440%:%
%:%1121=441%:%
%:%1122=441%:%
%:%1123=441%:%
%:%1124=442%:%
%:%1125=442%:%
%:%1126=442%:%
%:%1127=442%:%
%:%1128=443%:%
%:%1129=443%:%
%:%1130=443%:%
%:%1131=444%:%
%:%1132=444%:%
%:%1133=445%:%
%:%1134=445%:%
%:%1135=445%:%
%:%1136=445%:%
%:%1137=446%:%
%:%1138=447%:%
%:%1139=447%:%
%:%1140=448%:%
%:%1141=448%:%
%:%1142=449%:%
%:%1143=449%:%
%:%1144=450%:%
%:%1145=451%:%
%:%1146=451%:%
%:%1147=452%:%
%:%1148=452%:%
%:%1149=452%:%
%:%1150=453%:%
%:%1156=453%:%
%:%1161=454%:%
%:%1166=455%:%

%
\begin{isabellebody}%
\setisabellecontext{PolynomialCounting}%
%
\isadelimdocument
%
\endisadelimdocument
%
\isatagdocument
%
\isamarkupsection{Counting Polynomials%
}
\isamarkuptrue%
%
\endisatagdocument
{\isafolddocument}%
%
\isadelimdocument
%
\endisadelimdocument
%
\isadelimtheory
%
\endisadelimtheory
%
\isatagtheory
\isacommand{theory}\isamarkupfalse%
\ PolynomialCounting\isanewline
\ \ \isakeyword{imports}\ Main{\isachardoublequoteopen}HOL{\isacharminus}{\kern0pt}Algebra{\isachardot}{\kern0pt}Polynomial{\isacharunderscore}{\kern0pt}Divisibility{\isachardoublequoteclose}\ {\isachardoublequoteopen}HOL{\isacharminus}{\kern0pt}Algebra{\isachardot}{\kern0pt}Polynomials{\isachardoublequoteclose}\ {\isachardoublequoteopen}HOL{\isacharminus}{\kern0pt}Library{\isachardot}{\kern0pt}FuncSet{\isachardoublequoteclose}\isanewline
\ \ \ \ Set{\isacharunderscore}{\kern0pt}Ext\ \isanewline
\isakeyword{begin}%
\endisatagtheory
{\isafoldtheory}%
%
\isadelimtheory
%
\endisadelimtheory
%
\begin{isamarkuptext}%
This section contains results about the count of polynomials with a given degree interpolating
a certain number of points.%
\end{isamarkuptext}\isamarkuptrue%
\isacommand{definition}\isamarkupfalse%
\ bounded{\isacharunderscore}{\kern0pt}degree{\isacharunderscore}{\kern0pt}polynomials\isanewline
\ \ \isakeyword{where}\ {\isachardoublequoteopen}bounded{\isacharunderscore}{\kern0pt}degree{\isacharunderscore}{\kern0pt}polynomials\ F\ n\ {\isacharequal}{\kern0pt}\ {\isacharbraceleft}{\kern0pt}x{\isachardot}{\kern0pt}\ x\ {\isasymin}\ carrier\ {\isacharparenleft}{\kern0pt}poly{\isacharunderscore}{\kern0pt}ring\ F{\isacharparenright}{\kern0pt}\ {\isasymand}\ {\isacharparenleft}{\kern0pt}degree\ x\ {\isacharless}{\kern0pt}\ n\ {\isasymor}\ x\ {\isacharequal}{\kern0pt}\ {\isacharbrackleft}{\kern0pt}{\isacharbrackright}{\kern0pt}{\isacharparenright}{\kern0pt}{\isacharbraceright}{\kern0pt}{\isachardoublequoteclose}\isanewline
\isanewline
\isacommand{lemma}\isamarkupfalse%
\ bounded{\isacharunderscore}{\kern0pt}degree{\isacharunderscore}{\kern0pt}polynomials{\isacharunderscore}{\kern0pt}length{\isacharcolon}{\kern0pt}\isanewline
\ \ {\isachardoublequoteopen}bounded{\isacharunderscore}{\kern0pt}degree{\isacharunderscore}{\kern0pt}polynomials\ F\ n\ {\isacharequal}{\kern0pt}\ {\isacharbraceleft}{\kern0pt}x{\isachardot}{\kern0pt}\ x\ {\isasymin}\ carrier\ {\isacharparenleft}{\kern0pt}poly{\isacharunderscore}{\kern0pt}ring\ F{\isacharparenright}{\kern0pt}\ {\isasymand}\ length\ x\ {\isasymle}\ n{\isacharbraceright}{\kern0pt}{\isachardoublequoteclose}\isanewline
%
\isadelimproof
\ \ %
\endisadelimproof
%
\isatagproof
\isacommand{apply}\isamarkupfalse%
\ {\isacharparenleft}{\kern0pt}rule\ order{\isacharunderscore}{\kern0pt}antisym{\isacharparenright}{\kern0pt}\isanewline
\ \ \isacommand{apply}\isamarkupfalse%
\ {\isacharparenleft}{\kern0pt}rule\ subsetI{\isacharcomma}{\kern0pt}\ simp\ add{\isacharcolon}{\kern0pt}bounded{\isacharunderscore}{\kern0pt}degree{\isacharunderscore}{\kern0pt}polynomials{\isacharunderscore}{\kern0pt}def{\isacharparenright}{\kern0pt}\ \isanewline
\ \ \isacommand{apply}\isamarkupfalse%
\ {\isacharparenleft}{\kern0pt}metis\ Suc{\isacharunderscore}{\kern0pt}pred\ leI\ less{\isacharunderscore}{\kern0pt}Suc{\isacharunderscore}{\kern0pt}eq{\isacharunderscore}{\kern0pt}{\isadigit{0}}{\isacharunderscore}{\kern0pt}disj\ less{\isacharunderscore}{\kern0pt}Suc{\isacharunderscore}{\kern0pt}eq{\isacharunderscore}{\kern0pt}le\ list{\isachardot}{\kern0pt}size{\isacharparenleft}{\kern0pt}{\isadigit{3}}{\isacharparenright}{\kern0pt}{\isacharparenright}{\kern0pt}\isanewline
\ \ \isacommand{apply}\isamarkupfalse%
\ {\isacharparenleft}{\kern0pt}rule\ subsetI{\isacharcomma}{\kern0pt}\ simp\ add{\isacharcolon}{\kern0pt}bounded{\isacharunderscore}{\kern0pt}degree{\isacharunderscore}{\kern0pt}polynomials{\isacharunderscore}{\kern0pt}def{\isacharparenright}{\kern0pt}\ \isanewline
\ \ \isacommand{by}\isamarkupfalse%
\ {\isacharparenleft}{\kern0pt}metis\ diff{\isacharunderscore}{\kern0pt}less\ length{\isacharunderscore}{\kern0pt}greater{\isacharunderscore}{\kern0pt}{\isadigit{0}}{\isacharunderscore}{\kern0pt}conv\ lessI\ less{\isacharunderscore}{\kern0pt}imp{\isacharunderscore}{\kern0pt}diff{\isacharunderscore}{\kern0pt}less\ order{\isachardot}{\kern0pt}not{\isacharunderscore}{\kern0pt}eq{\isacharunderscore}{\kern0pt}order{\isacharunderscore}{\kern0pt}implies{\isacharunderscore}{\kern0pt}strict{\isacharparenright}{\kern0pt}%
\endisatagproof
{\isafoldproof}%
%
\isadelimproof
\isanewline
%
\endisadelimproof
\isanewline
\isacommand{lemma}\isamarkupfalse%
\ fin{\isacharunderscore}{\kern0pt}degree{\isacharunderscore}{\kern0pt}bounded{\isacharcolon}{\kern0pt}\isanewline
\ \ \isakeyword{assumes}\ {\isachardoublequoteopen}ring\ F{\isachardoublequoteclose}\isanewline
\ \ \isakeyword{assumes}\ {\isachardoublequoteopen}finite\ {\isacharparenleft}{\kern0pt}carrier\ F{\isacharparenright}{\kern0pt}{\isachardoublequoteclose}\isanewline
\ \ \isakeyword{shows}\ {\isachardoublequoteopen}finite\ {\isacharparenleft}{\kern0pt}bounded{\isacharunderscore}{\kern0pt}degree{\isacharunderscore}{\kern0pt}polynomials\ F\ n{\isacharparenright}{\kern0pt}{\isachardoublequoteclose}\isanewline
%
\isadelimproof
%
\endisadelimproof
%
\isatagproof
\isacommand{proof}\isamarkupfalse%
\ {\isacharminus}{\kern0pt}\isanewline
\ \ \isacommand{have}\isamarkupfalse%
\ {\isachardoublequoteopen}bounded{\isacharunderscore}{\kern0pt}degree{\isacharunderscore}{\kern0pt}polynomials\ F\ n\ {\isasymsubseteq}\ {\isacharbraceleft}{\kern0pt}p{\isachardot}{\kern0pt}\ set\ p\ {\isasymsubseteq}\ carrier\ F\ {\isasymand}\ length\ p\ {\isasymle}\ n{\isacharbraceright}{\kern0pt}{\isachardoublequoteclose}\isanewline
\ \ \ \ \isacommand{apply}\isamarkupfalse%
\ {\isacharparenleft}{\kern0pt}rule\ subsetI{\isacharparenright}{\kern0pt}\isanewline
\ \ \ \ \isacommand{apply}\isamarkupfalse%
\ {\isacharparenleft}{\kern0pt}simp\ add{\isacharcolon}{\kern0pt}\ bounded{\isacharunderscore}{\kern0pt}degree{\isacharunderscore}{\kern0pt}polynomials{\isacharunderscore}{\kern0pt}length{\isacharparenright}{\kern0pt}\ \isacommand{using}\isamarkupfalse%
\ assms{\isacharparenleft}{\kern0pt}{\isadigit{1}}{\isacharparenright}{\kern0pt}\ \isanewline
\ \ \ \ \isacommand{by}\isamarkupfalse%
\ {\isacharparenleft}{\kern0pt}meson\ ring{\isachardot}{\kern0pt}polynomial{\isacharunderscore}{\kern0pt}incl\ univ{\isacharunderscore}{\kern0pt}poly{\isacharunderscore}{\kern0pt}carrier{\isacharparenright}{\kern0pt}\isanewline
\ \ \isacommand{thus}\isamarkupfalse%
\ {\isacharquery}{\kern0pt}thesis\ \ \isacommand{apply}\isamarkupfalse%
\ {\isacharparenleft}{\kern0pt}rule\ finite{\isacharunderscore}{\kern0pt}subset{\isacharparenright}{\kern0pt}\isanewline
\ \ \ \ \isacommand{using}\isamarkupfalse%
\ assms{\isacharparenleft}{\kern0pt}{\isadigit{2}}{\isacharparenright}{\kern0pt}\ finite{\isacharunderscore}{\kern0pt}lists{\isacharunderscore}{\kern0pt}length{\isacharunderscore}{\kern0pt}le\ \isacommand{by}\isamarkupfalse%
\ auto\isanewline
\isacommand{qed}\isamarkupfalse%
%
\endisatagproof
{\isafoldproof}%
%
\isadelimproof
\isanewline
%
\endisadelimproof
\isanewline
\isacommand{lemma}\isamarkupfalse%
\ fin{\isacharunderscore}{\kern0pt}fixed{\isacharunderscore}{\kern0pt}degree{\isacharcolon}{\kern0pt}\isanewline
\ \ \isakeyword{assumes}\ {\isachardoublequoteopen}ring\ F{\isachardoublequoteclose}\isanewline
\ \ \isakeyword{assumes}\ {\isachardoublequoteopen}finite\ {\isacharparenleft}{\kern0pt}carrier\ F{\isacharparenright}{\kern0pt}{\isachardoublequoteclose}\isanewline
\ \ \isakeyword{shows}\ {\isachardoublequoteopen}finite\ {\isacharbraceleft}{\kern0pt}p{\isachardot}{\kern0pt}\ p\ {\isasymin}\ carrier\ {\isacharparenleft}{\kern0pt}poly{\isacharunderscore}{\kern0pt}ring\ F{\isacharparenright}{\kern0pt}\ {\isasymand}\ length\ p\ {\isacharequal}{\kern0pt}\ n{\isacharbraceright}{\kern0pt}{\isachardoublequoteclose}\isanewline
%
\isadelimproof
%
\endisadelimproof
%
\isatagproof
\isacommand{proof}\isamarkupfalse%
\ {\isacharminus}{\kern0pt}\isanewline
\ \ \isacommand{have}\isamarkupfalse%
\ {\isachardoublequoteopen}{\isacharbraceleft}{\kern0pt}p{\isachardot}{\kern0pt}\ p\ {\isasymin}\ carrier\ {\isacharparenleft}{\kern0pt}poly{\isacharunderscore}{\kern0pt}ring\ F{\isacharparenright}{\kern0pt}\ {\isasymand}\ length\ p\ {\isacharequal}{\kern0pt}\ n{\isacharbraceright}{\kern0pt}\ {\isasymsubseteq}\ bounded{\isacharunderscore}{\kern0pt}degree{\isacharunderscore}{\kern0pt}polynomials\ F\ n{\isachardoublequoteclose}\isanewline
\ \ \ \ \isacommand{by}\isamarkupfalse%
\ {\isacharparenleft}{\kern0pt}rule\ subsetI{\isacharcomma}{\kern0pt}\ simp\ add{\isacharcolon}{\kern0pt}bounded{\isacharunderscore}{\kern0pt}degree{\isacharunderscore}{\kern0pt}polynomials{\isacharunderscore}{\kern0pt}length{\isacharparenright}{\kern0pt}\ \isanewline
\ \ \isacommand{then}\isamarkupfalse%
\ \isacommand{show}\isamarkupfalse%
\ {\isacharquery}{\kern0pt}thesis\isanewline
\ \ \isacommand{using}\isamarkupfalse%
\ fin{\isacharunderscore}{\kern0pt}degree{\isacharunderscore}{\kern0pt}bounded\ assms\ rev{\isacharunderscore}{\kern0pt}finite{\isacharunderscore}{\kern0pt}subset\ \isacommand{by}\isamarkupfalse%
\ blast\isanewline
\isacommand{qed}\isamarkupfalse%
%
\endisatagproof
{\isafoldproof}%
%
\isadelimproof
\isanewline
%
\endisadelimproof
\isanewline
\isacommand{lemma}\isamarkupfalse%
\ nonzero{\isacharunderscore}{\kern0pt}length{\isacharunderscore}{\kern0pt}polynomials{\isacharunderscore}{\kern0pt}count{\isacharcolon}{\kern0pt}\isanewline
\ \ \isakeyword{assumes}\ {\isachardoublequoteopen}ring\ F{\isachardoublequoteclose}\isanewline
\ \ \isakeyword{assumes}\ {\isachardoublequoteopen}finite\ {\isacharparenleft}{\kern0pt}carrier\ F{\isacharparenright}{\kern0pt}{\isachardoublequoteclose}\isanewline
\ \ \isakeyword{shows}\ {\isachardoublequoteopen}card\ {\isacharbraceleft}{\kern0pt}p{\isachardot}{\kern0pt}\ p\ {\isasymin}\ carrier\ {\isacharparenleft}{\kern0pt}poly{\isacharunderscore}{\kern0pt}ring\ F{\isacharparenright}{\kern0pt}\ {\isasymand}\ length\ p\ {\isacharequal}{\kern0pt}\ Suc\ n{\isacharbraceright}{\kern0pt}\ \isanewline
\ \ \ \ \ \ \ \ {\isacharequal}{\kern0pt}\ {\isacharparenleft}{\kern0pt}card\ {\isacharparenleft}{\kern0pt}carrier\ F{\isacharparenright}{\kern0pt}\ {\isacharminus}{\kern0pt}\ {\isadigit{1}}{\isacharparenright}{\kern0pt}\ {\isacharasterisk}{\kern0pt}\ card\ {\isacharparenleft}{\kern0pt}carrier\ F{\isacharparenright}{\kern0pt}\ {\isacharcircum}{\kern0pt}\ n{\isachardoublequoteclose}\isanewline
%
\isadelimproof
%
\endisadelimproof
%
\isatagproof
\isacommand{proof}\isamarkupfalse%
\ {\isacharminus}{\kern0pt}\isanewline
\ \ \isacommand{define}\isamarkupfalse%
\ A\ \isakeyword{where}\ {\isachardoublequoteopen}A\ {\isacharequal}{\kern0pt}\ {\isacharbraceleft}{\kern0pt}p{\isachardot}{\kern0pt}\ p\ {\isasymin}\ {\isacharparenleft}{\kern0pt}carrier\ {\isacharparenleft}{\kern0pt}poly{\isacharunderscore}{\kern0pt}ring\ F{\isacharparenright}{\kern0pt}{\isacharparenright}{\kern0pt}\ {\isasymand}\ length\ p\ {\isacharequal}{\kern0pt}\ Suc\ n{\isacharbraceright}{\kern0pt}{\isachardoublequoteclose}\isanewline
\ \ \isacommand{have}\isamarkupfalse%
\ b{\isacharcolon}{\kern0pt}{\isachardoublequoteopen}A\ {\isacharequal}{\kern0pt}\ {\isacharbraceleft}{\kern0pt}p{\isachardot}{\kern0pt}\ polynomial\isactrlbsub F\isactrlesub \ {\isacharparenleft}{\kern0pt}carrier\ F{\isacharparenright}{\kern0pt}\ p\ {\isasymand}\ length\ p\ {\isacharequal}{\kern0pt}\ Suc\ n{\isacharbraceright}{\kern0pt}{\isachardoublequoteclose}\isanewline
\ \ \ \ \isacommand{apply}\isamarkupfalse%
{\isacharparenleft}{\kern0pt}rule\ order{\isacharunderscore}{\kern0pt}antisym{\isacharcomma}{\kern0pt}\ rule\ subsetI{\isacharparenright}{\kern0pt}\isanewline
\ \ \ \ \isacommand{using}\isamarkupfalse%
\ A{\isacharunderscore}{\kern0pt}def\ assms{\isacharparenleft}{\kern0pt}{\isadigit{1}}{\isacharparenright}{\kern0pt}\ \isacommand{by}\isamarkupfalse%
\ {\isacharparenleft}{\kern0pt}simp\ add{\isacharcolon}{\kern0pt}\ univ{\isacharunderscore}{\kern0pt}poly{\isacharunderscore}{\kern0pt}carrier{\isacharparenright}{\kern0pt}{\isacharplus}{\kern0pt}\isanewline
\ \ \isacommand{have}\isamarkupfalse%
\ c{\isacharcolon}{\kern0pt}{\isachardoublequoteopen}A\ {\isacharequal}{\kern0pt}\ {\isacharbraceleft}{\kern0pt}p{\isachardot}{\kern0pt}\ set\ p\ {\isasymsubseteq}\ carrier\ F\ {\isasymand}\ hd\ p\ {\isasymnoteq}\ {\isasymzero}\isactrlbsub F\isactrlesub \ {\isasymand}\ length\ p\ {\isacharequal}{\kern0pt}\ Suc\ n{\isacharbraceright}{\kern0pt}{\isachardoublequoteclose}\isanewline
\ \ \ \ \isacommand{apply}\isamarkupfalse%
\ {\isacharparenleft}{\kern0pt}rule\ order{\isacharunderscore}{\kern0pt}antisym{\isacharparenright}{\kern0pt}\isanewline
\ \ \ \ \isacommand{apply}\isamarkupfalse%
\ {\isacharparenleft}{\kern0pt}rule\ subsetI{\isacharcomma}{\kern0pt}\ simp\ add{\isacharcolon}{\kern0pt}b\ polynomial{\isacharunderscore}{\kern0pt}def{\isacharcomma}{\kern0pt}\ force{\isacharparenright}{\kern0pt}\isanewline
\ \ \ \ \isacommand{by}\isamarkupfalse%
\ {\isacharparenleft}{\kern0pt}rule\ subsetI{\isacharcomma}{\kern0pt}\ simp\ add{\isacharcolon}{\kern0pt}b\ polynomial{\isacharunderscore}{\kern0pt}def{\isacharparenright}{\kern0pt}\isanewline
\ \ \isacommand{have}\isamarkupfalse%
\ d{\isacharcolon}{\kern0pt}{\isachardoublequoteopen}A\ {\isacharequal}{\kern0pt}\ {\isacharbraceleft}{\kern0pt}p{\isachardot}{\kern0pt}\ {\isasymexists}u\ v{\isachardot}{\kern0pt}\ p{\isacharequal}{\kern0pt}u{\isacharhash}{\kern0pt}v\ {\isasymand}\ set\ v\ {\isasymsubseteq}\ carrier\ F\ {\isasymand}\ u\ {\isasymin}\ carrier\ F\ {\isacharminus}{\kern0pt}\ {\isacharbraceleft}{\kern0pt}{\isasymzero}\isactrlbsub F\isactrlesub {\isacharbraceright}{\kern0pt}\ {\isasymand}\ length\ v\ {\isacharequal}{\kern0pt}\ n{\isacharbraceright}{\kern0pt}{\isachardoublequoteclose}\isanewline
\ \ \ \ \isacommand{apply}\isamarkupfalse%
{\isacharparenleft}{\kern0pt}rule\ order{\isacharunderscore}{\kern0pt}antisym{\isacharcomma}{\kern0pt}\ rule\ subsetI{\isacharparenright}{\kern0pt}\isanewline
\ \ \ \ \ \isacommand{apply}\isamarkupfalse%
\ {\isacharparenleft}{\kern0pt}simp\ add{\isacharcolon}{\kern0pt}c{\isacharparenright}{\kern0pt}\ \isanewline
\ \ \ \ \ \isacommand{apply}\isamarkupfalse%
\ {\isacharparenleft}{\kern0pt}metis\ Suc{\isacharunderscore}{\kern0pt}length{\isacharunderscore}{\kern0pt}conv\ hd{\isacharunderscore}{\kern0pt}Cons{\isacharunderscore}{\kern0pt}tl\ length{\isacharunderscore}{\kern0pt}{\isadigit{0}}{\isacharunderscore}{\kern0pt}conv\ list{\isachardot}{\kern0pt}sel{\isacharparenleft}{\kern0pt}{\isadigit{3}}{\isacharparenright}{\kern0pt}\ list{\isachardot}{\kern0pt}set{\isacharunderscore}{\kern0pt}sel{\isacharparenleft}{\kern0pt}{\isadigit{1}}{\isacharparenright}{\kern0pt}\ nat{\isachardot}{\kern0pt}simps{\isacharparenleft}{\kern0pt}{\isadigit{3}}{\isacharparenright}{\kern0pt}\ \isanewline
\ \ \ \ \ \ \ \ \ \ \ \ order{\isacharunderscore}{\kern0pt}trans\ set{\isacharunderscore}{\kern0pt}subset{\isacharunderscore}{\kern0pt}Cons\ subsetD{\isacharparenright}{\kern0pt}\isanewline
\ \ \ \ \isacommand{apply}\isamarkupfalse%
\ {\isacharparenleft}{\kern0pt}rule\ subsetI{\isacharcomma}{\kern0pt}\ simp\ add{\isacharcolon}{\kern0pt}c{\isacharparenright}{\kern0pt}\ \isacommand{using}\isamarkupfalse%
\ assms{\isacharparenleft}{\kern0pt}{\isadigit{2}}{\isacharparenright}{\kern0pt}\ \isacommand{by}\isamarkupfalse%
\ force\isanewline
\ \ \isacommand{define}\isamarkupfalse%
\ B\ \isakeyword{where}\ {\isachardoublequoteopen}B\ {\isacharequal}{\kern0pt}\ {\isacharbraceleft}{\kern0pt}p{\isachardot}{\kern0pt}\ set\ p\ {\isasymsubseteq}\ carrier\ F\ {\isasymand}\ length\ p\ {\isacharequal}{\kern0pt}\ n{\isacharbraceright}{\kern0pt}{\isachardoublequoteclose}\isanewline
\ \ \isacommand{have}\isamarkupfalse%
\ {\isachardoublequoteopen}A\ {\isacharequal}{\kern0pt}\ {\isacharparenleft}{\kern0pt}{\isasymlambda}{\isacharparenleft}{\kern0pt}u{\isacharcomma}{\kern0pt}v{\isacharparenright}{\kern0pt}{\isachardot}{\kern0pt}\ u\ {\isacharhash}{\kern0pt}\ v{\isacharparenright}{\kern0pt}\ {\isacharbackquote}{\kern0pt}\ {\isacharparenleft}{\kern0pt}{\isacharparenleft}{\kern0pt}carrier\ F\ {\isacharminus}{\kern0pt}\ \ {\isacharbraceleft}{\kern0pt}{\isasymzero}\isactrlbsub F\isactrlesub {\isacharbraceright}{\kern0pt}{\isacharparenright}{\kern0pt}\ {\isasymtimes}\ B{\isacharparenright}{\kern0pt}{\isachardoublequoteclose}\isanewline
\ \ \ \ \isacommand{using}\isamarkupfalse%
\ d\ B{\isacharunderscore}{\kern0pt}def\ \isacommand{by}\isamarkupfalse%
\ auto\isanewline
\ \ \isacommand{moreover}\isamarkupfalse%
\ \isacommand{have}\isamarkupfalse%
\ {\isachardoublequoteopen}inj{\isacharunderscore}{\kern0pt}on\ {\isacharparenleft}{\kern0pt}{\isasymlambda}{\isacharparenleft}{\kern0pt}u{\isacharcomma}{\kern0pt}v{\isacharparenright}{\kern0pt}{\isachardot}{\kern0pt}\ u\ {\isacharhash}{\kern0pt}\ v{\isacharparenright}{\kern0pt}\ {\isacharparenleft}{\kern0pt}{\isacharparenleft}{\kern0pt}carrier\ F\ {\isacharminus}{\kern0pt}\ \ {\isacharbraceleft}{\kern0pt}{\isasymzero}\isactrlbsub F\isactrlesub {\isacharbraceright}{\kern0pt}{\isacharparenright}{\kern0pt}\ {\isasymtimes}\ B{\isacharparenright}{\kern0pt}{\isachardoublequoteclose}\isanewline
\ \ \ \ \isacommand{by}\isamarkupfalse%
\ {\isacharparenleft}{\kern0pt}auto\ intro{\isacharbang}{\kern0pt}{\isacharcolon}{\kern0pt}\ inj{\isacharunderscore}{\kern0pt}onI{\isacharparenright}{\kern0pt}\ \isanewline
\ \ \isacommand{ultimately}\isamarkupfalse%
\ \isacommand{have}\isamarkupfalse%
\ {\isachardoublequoteopen}card\ A\ {\isacharequal}{\kern0pt}\ card\ {\isacharparenleft}{\kern0pt}{\isacharparenleft}{\kern0pt}carrier\ F\ {\isacharminus}{\kern0pt}\ {\isacharbraceleft}{\kern0pt}{\isasymzero}\isactrlbsub F\isactrlesub {\isacharbraceright}{\kern0pt}{\isacharparenright}{\kern0pt}\ {\isasymtimes}\ B{\isacharparenright}{\kern0pt}{\isachardoublequoteclose}\isanewline
\ \ \ \ \isacommand{using}\isamarkupfalse%
\ card{\isacharunderscore}{\kern0pt}image\ \isacommand{by}\isamarkupfalse%
\ meson\isanewline
\ \ \isacommand{moreover}\isamarkupfalse%
\ \isacommand{have}\isamarkupfalse%
\ {\isachardoublequoteopen}card\ B\ {\isacharequal}{\kern0pt}\ {\isacharparenleft}{\kern0pt}card\ {\isacharparenleft}{\kern0pt}carrier\ F{\isacharparenright}{\kern0pt}\ {\isacharcircum}{\kern0pt}\ n{\isacharparenright}{\kern0pt}{\isachardoublequoteclose}\ \isacommand{using}\isamarkupfalse%
\ B{\isacharunderscore}{\kern0pt}def\isanewline
\ \ \ \ \isacommand{using}\isamarkupfalse%
\ card{\isacharunderscore}{\kern0pt}lists{\isacharunderscore}{\kern0pt}length{\isacharunderscore}{\kern0pt}eq\ assms{\isacharparenleft}{\kern0pt}{\isadigit{2}}{\isacharparenright}{\kern0pt}\ \isacommand{by}\isamarkupfalse%
\ blast\isanewline
\ \ \isacommand{ultimately}\isamarkupfalse%
\ \isacommand{have}\isamarkupfalse%
\ {\isachardoublequoteopen}card\ A\ {\isacharequal}{\kern0pt}\ card\ {\isacharparenleft}{\kern0pt}carrier\ F\ {\isacharminus}{\kern0pt}\ {\isacharbraceleft}{\kern0pt}{\isasymzero}\isactrlbsub F\isactrlesub {\isacharbraceright}{\kern0pt}{\isacharparenright}{\kern0pt}\ {\isacharasterisk}{\kern0pt}\ {\isacharparenleft}{\kern0pt}card\ {\isacharparenleft}{\kern0pt}carrier\ F{\isacharparenright}{\kern0pt}\ {\isacharcircum}{\kern0pt}\ n{\isacharparenright}{\kern0pt}{\isachardoublequoteclose}\isanewline
\ \ \ \ \isacommand{by}\isamarkupfalse%
\ {\isacharparenleft}{\kern0pt}simp\ add{\isacharcolon}{\kern0pt}\ card{\isacharunderscore}{\kern0pt}cartesian{\isacharunderscore}{\kern0pt}product{\isacharparenright}{\kern0pt}\isanewline
\ \ \isacommand{moreover}\isamarkupfalse%
\ \isacommand{have}\isamarkupfalse%
\ {\isachardoublequoteopen}card\ {\isacharparenleft}{\kern0pt}carrier\ F\ {\isacharminus}{\kern0pt}\ {\isacharbraceleft}{\kern0pt}{\isasymzero}\isactrlbsub F\isactrlesub {\isacharbraceright}{\kern0pt}{\isacharparenright}{\kern0pt}\ {\isacharequal}{\kern0pt}\ card\ {\isacharparenleft}{\kern0pt}carrier\ F{\isacharparenright}{\kern0pt}\ {\isacharminus}{\kern0pt}\ {\isadigit{1}}{\isachardoublequoteclose}\ \isanewline
\ \ \ \ \isacommand{by}\isamarkupfalse%
\ {\isacharparenleft}{\kern0pt}meson\ assms{\isacharparenleft}{\kern0pt}{\isadigit{1}}{\isacharparenright}{\kern0pt}\ assms{\isacharparenleft}{\kern0pt}{\isadigit{2}}{\isacharparenright}{\kern0pt}\ card{\isacharunderscore}{\kern0pt}Diff{\isacharunderscore}{\kern0pt}singleton\ ring{\isachardot}{\kern0pt}ring{\isacharunderscore}{\kern0pt}simprules{\isacharparenleft}{\kern0pt}{\isadigit{2}}{\isacharparenright}{\kern0pt}{\isacharparenright}{\kern0pt}\isanewline
\ \ \isacommand{ultimately}\isamarkupfalse%
\ \isacommand{show}\isamarkupfalse%
\ {\isachardoublequoteopen}card\ {\isacharparenleft}{\kern0pt}{\isacharbraceleft}{\kern0pt}p{\isachardot}{\kern0pt}\ p\ {\isasymin}\ carrier\ {\isacharparenleft}{\kern0pt}poly{\isacharunderscore}{\kern0pt}ring\ F{\isacharparenright}{\kern0pt}\ {\isasymand}\ length\ p\ {\isacharequal}{\kern0pt}\ Suc\ n{\isacharbraceright}{\kern0pt}{\isacharparenright}{\kern0pt}\ {\isacharequal}{\kern0pt}\ \isanewline
\ \ \ \ \ \ \ \ \ \ {\isacharparenleft}{\kern0pt}card\ {\isacharparenleft}{\kern0pt}carrier\ F{\isacharparenright}{\kern0pt}\ {\isacharminus}{\kern0pt}\ {\isadigit{1}}{\isacharparenright}{\kern0pt}\ {\isacharasterisk}{\kern0pt}\ {\isacharparenleft}{\kern0pt}card\ {\isacharparenleft}{\kern0pt}carrier\ F{\isacharparenright}{\kern0pt}\ {\isacharcircum}{\kern0pt}\ n{\isacharparenright}{\kern0pt}{\isachardoublequoteclose}\ \isacommand{using}\isamarkupfalse%
\ A{\isacharunderscore}{\kern0pt}def\ \isacommand{by}\isamarkupfalse%
\ simp\isanewline
\isacommand{qed}\isamarkupfalse%
%
\endisatagproof
{\isafoldproof}%
%
\isadelimproof
\isanewline
%
\endisadelimproof
\isanewline
\isacommand{lemma}\isamarkupfalse%
\ fixed{\isacharunderscore}{\kern0pt}degree{\isacharunderscore}{\kern0pt}polynomials{\isacharunderscore}{\kern0pt}count{\isacharcolon}{\kern0pt}\isanewline
\ \ \isakeyword{assumes}\ {\isachardoublequoteopen}ring\ F{\isachardoublequoteclose}\isanewline
\ \ \isakeyword{assumes}\ {\isachardoublequoteopen}finite\ {\isacharparenleft}{\kern0pt}carrier\ F{\isacharparenright}{\kern0pt}{\isachardoublequoteclose}\isanewline
\ \ \isakeyword{shows}\ {\isachardoublequoteopen}card\ {\isacharparenleft}{\kern0pt}{\isacharbraceleft}{\kern0pt}p{\isachardot}{\kern0pt}\ p\ {\isasymin}\ carrier\ {\isacharparenleft}{\kern0pt}poly{\isacharunderscore}{\kern0pt}ring\ F{\isacharparenright}{\kern0pt}\ {\isasymand}\ length\ p\ {\isacharequal}{\kern0pt}\ n{\isacharbraceright}{\kern0pt}{\isacharparenright}{\kern0pt}\ {\isacharequal}{\kern0pt}\ \isanewline
\ \ \ \ {\isacharparenleft}{\kern0pt}if\ n\ {\isasymge}\ {\isadigit{1}}\ then\ {\isacharparenleft}{\kern0pt}card\ {\isacharparenleft}{\kern0pt}carrier\ F{\isacharparenright}{\kern0pt}\ {\isacharminus}{\kern0pt}\ {\isadigit{1}}{\isacharparenright}{\kern0pt}\ {\isacharasterisk}{\kern0pt}\ {\isacharparenleft}{\kern0pt}card\ {\isacharparenleft}{\kern0pt}carrier\ F{\isacharparenright}{\kern0pt}\ {\isacharcircum}{\kern0pt}\ {\isacharparenleft}{\kern0pt}n{\isacharminus}{\kern0pt}{\isadigit{1}}{\isacharparenright}{\kern0pt}{\isacharparenright}{\kern0pt}\ else\ {\isadigit{1}}{\isacharparenright}{\kern0pt}{\isachardoublequoteclose}\isanewline
%
\isadelimproof
%
\endisadelimproof
%
\isatagproof
\isacommand{proof}\isamarkupfalse%
\ {\isacharminus}{\kern0pt}\isanewline
\ \ \isacommand{have}\isamarkupfalse%
\ a{\isacharcolon}{\kern0pt}\ {\isachardoublequoteopen}\ {\isacharbrackleft}{\kern0pt}{\isacharbrackright}{\kern0pt}\ {\isasymin}\ carrier\ {\isacharparenleft}{\kern0pt}poly{\isacharunderscore}{\kern0pt}ring\ F{\isacharparenright}{\kern0pt}{\isachardoublequoteclose}\ \isanewline
\ \ \ \ \isacommand{by}\isamarkupfalse%
\ {\isacharparenleft}{\kern0pt}simp\ add{\isacharcolon}{\kern0pt}\ univ{\isacharunderscore}{\kern0pt}poly{\isacharunderscore}{\kern0pt}zero{\isacharunderscore}{\kern0pt}closed{\isacharparenright}{\kern0pt}\isanewline
\ \ \isacommand{show}\isamarkupfalse%
\ {\isacharquery}{\kern0pt}thesis\ \isanewline
\ \ \ \ \isacommand{apply}\isamarkupfalse%
\ {\isacharparenleft}{\kern0pt}cases\ {\isachardoublequoteopen}n{\isachardoublequoteclose}{\isacharparenright}{\kern0pt}\isanewline
\ \ \ \ \isacommand{using}\isamarkupfalse%
\ assms\ a\ \isacommand{apply}\isamarkupfalse%
\ {\isacharparenleft}{\kern0pt}simp{\isacharparenright}{\kern0pt}\ \isanewline
\ \ \ \ \ \isacommand{apply}\isamarkupfalse%
\ {\isacharparenleft}{\kern0pt}metis\ {\isacharparenleft}{\kern0pt}mono{\isacharunderscore}{\kern0pt}tags{\isacharcomma}{\kern0pt}\ lifting{\isacharparenright}{\kern0pt}\ One{\isacharunderscore}{\kern0pt}nat{\isacharunderscore}{\kern0pt}def\ empty{\isacharunderscore}{\kern0pt}Collect{\isacharunderscore}{\kern0pt}eq\ is{\isacharunderscore}{\kern0pt}singletonI{\isacharprime}{\kern0pt}\ \isanewline
\ \ \ \ \ \ \ \ \ \ \ \ is{\isacharunderscore}{\kern0pt}singleton{\isacharunderscore}{\kern0pt}altdef\ mem{\isacharunderscore}{\kern0pt}Collect{\isacharunderscore}{\kern0pt}eq{\isacharparenright}{\kern0pt}\ \isanewline
\ \ \ \ \isacommand{using}\isamarkupfalse%
\ assms\ \isacommand{by}\isamarkupfalse%
\ {\isacharparenleft}{\kern0pt}simp\ add{\isacharcolon}{\kern0pt}nonzero{\isacharunderscore}{\kern0pt}length{\isacharunderscore}{\kern0pt}polynomials{\isacharunderscore}{\kern0pt}count{\isacharparenright}{\kern0pt}\isanewline
\isacommand{qed}\isamarkupfalse%
%
\endisatagproof
{\isafoldproof}%
%
\isadelimproof
\isanewline
%
\endisadelimproof
\isanewline
\isacommand{lemma}\isamarkupfalse%
\ bounded{\isacharunderscore}{\kern0pt}degree{\isacharunderscore}{\kern0pt}polynomials{\isacharunderscore}{\kern0pt}count{\isacharcolon}{\kern0pt}\isanewline
\ \ \isakeyword{assumes}\ {\isachardoublequoteopen}ring\ F{\isachardoublequoteclose}\isanewline
\ \ \isakeyword{assumes}\ {\isachardoublequoteopen}finite\ {\isacharparenleft}{\kern0pt}carrier\ F{\isacharparenright}{\kern0pt}{\isachardoublequoteclose}\isanewline
\ \ \isakeyword{shows}\ {\isachardoublequoteopen}card\ {\isacharparenleft}{\kern0pt}bounded{\isacharunderscore}{\kern0pt}degree{\isacharunderscore}{\kern0pt}polynomials\ F\ n{\isacharparenright}{\kern0pt}\ {\isacharequal}{\kern0pt}\ card\ {\isacharparenleft}{\kern0pt}carrier\ F{\isacharparenright}{\kern0pt}\ {\isacharcircum}{\kern0pt}\ n{\isachardoublequoteclose}\isanewline
%
\isadelimproof
%
\endisadelimproof
%
\isatagproof
\isacommand{proof}\isamarkupfalse%
\ {\isacharminus}{\kern0pt}\isanewline
\ \ \isacommand{have}\isamarkupfalse%
\ {\isachardoublequoteopen}{\isasymzero}\isactrlbsub F\isactrlesub \ {\isasymin}\ carrier\ F{\isachardoublequoteclose}\ \isacommand{using}\isamarkupfalse%
\ assms{\isacharparenleft}{\kern0pt}{\isadigit{1}}{\isacharparenright}{\kern0pt}\ \isacommand{by}\isamarkupfalse%
\ {\isacharparenleft}{\kern0pt}simp\ add{\isacharcolon}{\kern0pt}\ ring{\isachardot}{\kern0pt}ring{\isacharunderscore}{\kern0pt}simprules{\isacharparenleft}{\kern0pt}{\isadigit{2}}{\isacharparenright}{\kern0pt}{\isacharparenright}{\kern0pt}\isanewline
\ \ \isacommand{hence}\isamarkupfalse%
\ b{\isacharcolon}{\kern0pt}\ {\isachardoublequoteopen}card\ {\isacharparenleft}{\kern0pt}carrier\ F{\isacharparenright}{\kern0pt}\ {\isachargreater}{\kern0pt}\ {\isadigit{0}}{\isachardoublequoteclose}\ \isanewline
\ \ \ \ \isacommand{using}\isamarkupfalse%
\ assms{\isacharparenleft}{\kern0pt}{\isadigit{2}}{\isacharparenright}{\kern0pt}\ card{\isacharunderscore}{\kern0pt}gt{\isacharunderscore}{\kern0pt}{\isadigit{0}}{\isacharunderscore}{\kern0pt}iff\ \isacommand{by}\isamarkupfalse%
\ blast\isanewline
\ \ \isacommand{have}\isamarkupfalse%
\ a{\isacharcolon}{\kern0pt}\ {\isachardoublequoteopen}bounded{\isacharunderscore}{\kern0pt}degree{\isacharunderscore}{\kern0pt}polynomials\ F\ n\ {\isacharequal}{\kern0pt}\ {\isacharparenleft}{\kern0pt}{\isasymUnion}\ m\ {\isasymle}\ n{\isachardot}{\kern0pt}\ {\isacharbraceleft}{\kern0pt}p{\isachardot}{\kern0pt}\ \ p\ {\isasymin}\ carrier\ {\isacharparenleft}{\kern0pt}poly{\isacharunderscore}{\kern0pt}ring\ F{\isacharparenright}{\kern0pt}\ {\isasymand}\ length\ p\ {\isacharequal}{\kern0pt}\ m{\isacharbraceright}{\kern0pt}{\isacharparenright}{\kern0pt}{\isachardoublequoteclose}\isanewline
\ \ \ \ \isacommand{apply}\isamarkupfalse%
\ {\isacharparenleft}{\kern0pt}simp\ add{\isacharcolon}{\kern0pt}\ bounded{\isacharunderscore}{\kern0pt}degree{\isacharunderscore}{\kern0pt}polynomials{\isacharunderscore}{\kern0pt}length{\isacharcomma}{\kern0pt}rule\ order{\isacharunderscore}{\kern0pt}antisym{\isacharparenright}{\kern0pt}\isanewline
\ \ \ \ \isacommand{by}\isamarkupfalse%
\ {\isacharparenleft}{\kern0pt}rule\ subsetI{\isacharcomma}{\kern0pt}\ simp{\isacharparenright}{\kern0pt}{\isacharplus}{\kern0pt}\isanewline
\ \ \isacommand{have}\isamarkupfalse%
\ {\isachardoublequoteopen}card\ {\isacharparenleft}{\kern0pt}bounded{\isacharunderscore}{\kern0pt}degree{\isacharunderscore}{\kern0pt}polynomials\ F\ n{\isacharparenright}{\kern0pt}\ {\isacharequal}{\kern0pt}\ {\isacharparenleft}{\kern0pt}{\isasymSum}\ m\ {\isasymle}\ n{\isachardot}{\kern0pt}\ card\ {\isacharbraceleft}{\kern0pt}p{\isachardot}{\kern0pt}\ \ p\ {\isasymin}\ carrier\ {\isacharparenleft}{\kern0pt}poly{\isacharunderscore}{\kern0pt}ring\ F{\isacharparenright}{\kern0pt}\ {\isasymand}\ length\ p\ {\isacharequal}{\kern0pt}\ m{\isacharbraceright}{\kern0pt}{\isacharparenright}{\kern0pt}{\isachardoublequoteclose}\isanewline
\ \ \ \ \isacommand{apply}\isamarkupfalse%
\ {\isacharparenleft}{\kern0pt}simp\ only{\isacharcolon}{\kern0pt}a{\isacharparenright}{\kern0pt}\isanewline
\ \ \ \ \isacommand{apply}\isamarkupfalse%
\ {\isacharparenleft}{\kern0pt}rule\ card{\isacharunderscore}{\kern0pt}UN{\isacharunderscore}{\kern0pt}disjoint{\isacharcomma}{\kern0pt}\ blast{\isacharparenright}{\kern0pt}\isanewline
\ \ \ \ \isacommand{using}\isamarkupfalse%
\ fin{\isacharunderscore}{\kern0pt}fixed{\isacharunderscore}{\kern0pt}degree\ assms\ \isacommand{apply}\isamarkupfalse%
\ blast\isanewline
\ \ \ \ \isacommand{by}\isamarkupfalse%
\ blast\isanewline
\ \ \isacommand{hence}\isamarkupfalse%
\ {\isachardoublequoteopen}card\ {\isacharparenleft}{\kern0pt}bounded{\isacharunderscore}{\kern0pt}degree{\isacharunderscore}{\kern0pt}polynomials\ F\ n{\isacharparenright}{\kern0pt}\ {\isacharequal}{\kern0pt}\ {\isacharparenleft}{\kern0pt}{\isasymSum}\ m\ {\isasymle}\ n{\isachardot}{\kern0pt}\ if\ m\ {\isasymge}\ {\isadigit{1}}\ then\ {\isacharparenleft}{\kern0pt}card\ {\isacharparenleft}{\kern0pt}carrier\ F{\isacharparenright}{\kern0pt}\ {\isacharminus}{\kern0pt}\ {\isadigit{1}}{\isacharparenright}{\kern0pt}\ {\isacharasterisk}{\kern0pt}\ card\ {\isacharparenleft}{\kern0pt}carrier\ F{\isacharparenright}{\kern0pt}\ {\isacharcircum}{\kern0pt}\ {\isacharparenleft}{\kern0pt}m{\isacharminus}{\kern0pt}{\isadigit{1}}{\isacharparenright}{\kern0pt}\ else\ {\isadigit{1}}{\isacharparenright}{\kern0pt}{\isachardoublequoteclose}\isanewline
\ \ \ \ \isacommand{using}\isamarkupfalse%
\ fixed{\isacharunderscore}{\kern0pt}degree{\isacharunderscore}{\kern0pt}polynomials{\isacharunderscore}{\kern0pt}count\ assms\ \isacommand{by}\isamarkupfalse%
\ fastforce\isanewline
\ \ \isacommand{moreover}\isamarkupfalse%
\ \isacommand{have}\isamarkupfalse%
\ {\isachardoublequoteopen}{\isacharparenleft}{\kern0pt}{\isasymSum}\ m\ {\isasymle}\ n{\isachardot}{\kern0pt}\ if\ m\ {\isasymge}\ {\isadigit{1}}\ then\ {\isacharparenleft}{\kern0pt}card\ {\isacharparenleft}{\kern0pt}carrier\ F{\isacharparenright}{\kern0pt}\ {\isacharminus}{\kern0pt}\ {\isadigit{1}}{\isacharparenright}{\kern0pt}\ {\isacharasterisk}{\kern0pt}\ {\isacharparenleft}{\kern0pt}card\ {\isacharparenleft}{\kern0pt}carrier\ F{\isacharparenright}{\kern0pt}\ {\isacharcircum}{\kern0pt}\ {\isacharparenleft}{\kern0pt}m{\isacharminus}{\kern0pt}{\isadigit{1}}{\isacharparenright}{\kern0pt}{\isacharparenright}{\kern0pt}\ else\ {\isadigit{1}}{\isacharparenright}{\kern0pt}\ {\isacharequal}{\kern0pt}\ card\ {\isacharparenleft}{\kern0pt}carrier\ F{\isacharparenright}{\kern0pt}\ {\isacharcircum}{\kern0pt}\ n{\isachardoublequoteclose}\isanewline
\ \ \ \ \isacommand{apply}\isamarkupfalse%
\ {\isacharparenleft}{\kern0pt}induction\ n{\isacharcomma}{\kern0pt}\ simp{\isacharcomma}{\kern0pt}\ simp\ add{\isacharcolon}{\kern0pt}algebra{\isacharunderscore}{\kern0pt}simps{\isacharparenright}{\kern0pt}\ \isacommand{using}\isamarkupfalse%
\ b\ \isacommand{by}\isamarkupfalse%
\ force\isanewline
\ \ \isacommand{ultimately}\isamarkupfalse%
\ \isacommand{show}\isamarkupfalse%
\ {\isacharquery}{\kern0pt}thesis\ \isacommand{by}\isamarkupfalse%
\ auto\isanewline
\isacommand{qed}\isamarkupfalse%
%
\endisatagproof
{\isafoldproof}%
%
\isadelimproof
\isanewline
%
\endisadelimproof
\isanewline
\isacommand{lemma}\isamarkupfalse%
\ non{\isacharunderscore}{\kern0pt}empty{\isacharunderscore}{\kern0pt}bounded{\isacharunderscore}{\kern0pt}degree{\isacharunderscore}{\kern0pt}polynomials{\isacharcolon}{\kern0pt}\isanewline
\ \ \isakeyword{assumes}\ {\isachardoublequoteopen}ring\ F{\isachardoublequoteclose}\isanewline
\ \ \isakeyword{shows}\ {\isachardoublequoteopen}bounded{\isacharunderscore}{\kern0pt}degree{\isacharunderscore}{\kern0pt}polynomials\ F\ k\ {\isasymnoteq}\ {\isacharbraceleft}{\kern0pt}{\isacharbraceright}{\kern0pt}{\isachardoublequoteclose}\isanewline
%
\isadelimproof
%
\endisadelimproof
%
\isatagproof
\isacommand{proof}\isamarkupfalse%
\ {\isacharminus}{\kern0pt}\isanewline
\ \ \isacommand{have}\isamarkupfalse%
\ {\isachardoublequoteopen}{\isasymzero}\isactrlbsub poly{\isacharunderscore}{\kern0pt}ring\ F\isactrlesub \ {\isasymin}\ bounded{\isacharunderscore}{\kern0pt}degree{\isacharunderscore}{\kern0pt}polynomials\ F\ k{\isachardoublequoteclose}\isanewline
\ \ \ \ \isacommand{using}\isamarkupfalse%
\ assms\isanewline
\ \ \ \ \isacommand{by}\isamarkupfalse%
\ {\isacharparenleft}{\kern0pt}simp\ add{\isacharcolon}{\kern0pt}\ bounded{\isacharunderscore}{\kern0pt}degree{\isacharunderscore}{\kern0pt}polynomials{\isacharunderscore}{\kern0pt}def\ univ{\isacharunderscore}{\kern0pt}poly{\isacharunderscore}{\kern0pt}zero\ univ{\isacharunderscore}{\kern0pt}poly{\isacharunderscore}{\kern0pt}zero{\isacharunderscore}{\kern0pt}closed{\isacharparenright}{\kern0pt}\isanewline
\ \ \isacommand{thus}\isamarkupfalse%
\ {\isacharquery}{\kern0pt}thesis\ \isacommand{by}\isamarkupfalse%
\ auto\isanewline
\isacommand{qed}\isamarkupfalse%
%
\endisatagproof
{\isafoldproof}%
%
\isadelimproof
%
\endisadelimproof
%
\isadelimdocument
%
\endisadelimdocument
%
\isatagdocument
%
\isamarkupsubsection{Interpolation Polynomials%
}
\isamarkuptrue%
%
\endisatagdocument
{\isafolddocument}%
%
\isadelimdocument
%
\endisadelimdocument
%
\begin{isamarkuptext}%
It is well known that over any field there is exactly one polynomial with degree at most
\isa{k\ {\isacharminus}{\kern0pt}\ {\isadigit{1}}} interpolating  \isa{k} points. That there is never more that one such
polynomial follow from the fact that a polynomial of degree \isa{k\ {\isacharminus}{\kern0pt}\ {\isadigit{1}}} cannot have more
than \isa{k\ {\isacharminus}{\kern0pt}\ {\isadigit{1}}} roots. This is already shown in HOL-Algebra in
\isa{field{\isachardot}{\kern0pt}size{\isacharunderscore}{\kern0pt}roots{\isacharunderscore}{\kern0pt}le{\isacharunderscore}{\kern0pt}degree}. Existence is usually shown using Lagrange interpolation.

In the case of finite fields it is actually only necessary to show either that there is at most one
such polynomial or at least one - because a function whose domain and co-domain has the same finite
cardinality is injective if and only if it is surjective.

In the following a more generic result (over finite fields) is shown, counting the 
number of polynomials of degree \isa{k\ {\isacharplus}{\kern0pt}\ n\ {\isacharminus}{\kern0pt}\ {\isadigit{1}}} interpolating \isa{k} points for
non-negative \isa{n}. As it turns out there are \isa{{\isacharparenleft}{\kern0pt}card\ {\isacharparenleft}{\kern0pt}carrier\ F{\isacharparenright}{\kern0pt}{\isacharparenright}{\kern0pt}\isactrlbsup n\isactrlesup } such polynomials.
The trick is to observe that, for a given fix on the coefficients of order \isa{k} to 
\isa{k\ {\isacharplus}{\kern0pt}\ n\ {\isacharminus}{\kern0pt}\ {\isadigit{1}}} and the values at \isa{k} points there is at most one fitting polynomial.

An alternative way of stating the above result is that there is bijection between the polynomials
of degree \isa{n\ {\isacharplus}{\kern0pt}\ k\ {\isacharminus}{\kern0pt}\ {\isadigit{1}}} and the product space $F^k \times F^n$ where the first component is
the evaluation of the polynomials at \isa{k} distinct points and the second component are the
coefficients of order at least \isa{k}.%
\end{isamarkuptext}\isamarkuptrue%
\isacommand{definition}\isamarkupfalse%
\ split{\isacharunderscore}{\kern0pt}poly\ \isakeyword{where}\ {\isachardoublequoteopen}split{\isacharunderscore}{\kern0pt}poly\ F\ K\ p\ {\isacharequal}{\kern0pt}\ \isanewline
\ \ {\isacharparenleft}{\kern0pt}restrict\ {\isacharparenleft}{\kern0pt}ring{\isachardot}{\kern0pt}eval\ F\ p{\isacharparenright}{\kern0pt}\ K{\isacharcomma}{\kern0pt}\ {\isasymlambda}k{\isachardot}{\kern0pt}\ ring{\isachardot}{\kern0pt}coeff\ F\ p\ {\isacharparenleft}{\kern0pt}k{\isacharplus}{\kern0pt}card\ K{\isacharparenright}{\kern0pt}{\isacharparenright}{\kern0pt}{\isachardoublequoteclose}%
\begin{isamarkuptext}%
The bijection \isa{split{\isacharunderscore}{\kern0pt}poly} returns the evaluation of the polynomial
at the points in \isa{K} and the coefficients of order at least \isa{card\ K}.

In the following it is shown that its image is a subset of the product space mentioned above, and
that \isa{split{\isacharunderscore}{\kern0pt}poly} is injective and finally that its image is exactly that product space
using cardinalities.%
\end{isamarkuptext}\isamarkuptrue%
\isacommand{lemma}\isamarkupfalse%
\ split{\isacharunderscore}{\kern0pt}poly{\isacharunderscore}{\kern0pt}image{\isacharcolon}{\kern0pt}\isanewline
\ \ \isakeyword{assumes}\ {\isachardoublequoteopen}field\ F{\isachardoublequoteclose}\isanewline
\ \ \isakeyword{assumes}\ {\isachardoublequoteopen}K\ {\isasymsubseteq}\ carrier\ F{\isachardoublequoteclose}\isanewline
\ \ \isakeyword{shows}\ {\isachardoublequoteopen}split{\isacharunderscore}{\kern0pt}poly\ F\ K\ {\isacharbackquote}{\kern0pt}\ bounded{\isacharunderscore}{\kern0pt}degree{\isacharunderscore}{\kern0pt}polynomials\ F\ {\isacharparenleft}{\kern0pt}card\ K\ {\isacharplus}{\kern0pt}\ n{\isacharparenright}{\kern0pt}\ {\isasymsubseteq}\isanewline
\ \ \ \ \ \ \ \ {\isacharparenleft}{\kern0pt}K\ {\isasymrightarrow}\isactrlsub E\ carrier\ F{\isacharparenright}{\kern0pt}\ {\isasymtimes}\ {\isacharbraceleft}{\kern0pt}f{\isachardot}{\kern0pt}\ range\ f\ {\isasymsubseteq}\ carrier\ F\ {\isasymand}\ {\isacharparenleft}{\kern0pt}{\isasymforall}k\ {\isasymge}\ n{\isachardot}{\kern0pt}\ f\ k\ {\isacharequal}{\kern0pt}\ {\isasymzero}\isactrlbsub F\isactrlesub {\isacharparenright}{\kern0pt}{\isacharbraceright}{\kern0pt}{\isachardoublequoteclose}\ \isanewline
%
\isadelimproof
\ \ %
\endisadelimproof
%
\isatagproof
\isacommand{apply}\isamarkupfalse%
\ {\isacharparenleft}{\kern0pt}rule\ image{\isacharunderscore}{\kern0pt}subsetI{\isacharparenright}{\kern0pt}\isanewline
\ \ \isacommand{apply}\isamarkupfalse%
\ {\isacharparenleft}{\kern0pt}simp\ add{\isacharcolon}{\kern0pt}split{\isacharunderscore}{\kern0pt}poly{\isacharunderscore}{\kern0pt}def\ Pi{\isacharunderscore}{\kern0pt}def\ bounded{\isacharunderscore}{\kern0pt}degree{\isacharunderscore}{\kern0pt}polynomials{\isacharunderscore}{\kern0pt}length{\isacharparenright}{\kern0pt}\isanewline
\ \ \isacommand{apply}\isamarkupfalse%
\ {\isacharparenleft}{\kern0pt}rule\ conjI{\isacharcomma}{\kern0pt}\ rule\ allI{\isacharcomma}{\kern0pt}\ rule\ impI{\isacharparenright}{\kern0pt}\ \ \isanewline
\ \ \ \isacommand{apply}\isamarkupfalse%
\ {\isacharparenleft}{\kern0pt}metis\ assms{\isacharparenleft}{\kern0pt}{\isadigit{1}}{\isacharparenright}{\kern0pt}\ assms{\isacharparenleft}{\kern0pt}{\isadigit{2}}{\isacharparenright}{\kern0pt}\ field{\isachardot}{\kern0pt}is{\isacharunderscore}{\kern0pt}ring\ mem{\isacharunderscore}{\kern0pt}Collect{\isacharunderscore}{\kern0pt}eq\ partial{\isacharunderscore}{\kern0pt}object{\isachardot}{\kern0pt}select{\isacharunderscore}{\kern0pt}convs{\isacharparenleft}{\kern0pt}{\isadigit{1}}{\isacharparenright}{\kern0pt}\ \isanewline
\ \ \ \ \ \ \ \ \ \ ring{\isachardot}{\kern0pt}carrier{\isacharunderscore}{\kern0pt}is{\isacharunderscore}{\kern0pt}subring\ ring{\isachardot}{\kern0pt}eval{\isacharunderscore}{\kern0pt}in{\isacharunderscore}{\kern0pt}carrier\ ring{\isachardot}{\kern0pt}polynomial{\isacharunderscore}{\kern0pt}in{\isacharunderscore}{\kern0pt}carrier\ subset{\isacharunderscore}{\kern0pt}iff\ \isanewline
\ \ \ \ \ \ \ \ \ \ univ{\isacharunderscore}{\kern0pt}poly{\isacharunderscore}{\kern0pt}def{\isacharparenright}{\kern0pt}\ \isanewline
\ \ \isacommand{apply}\isamarkupfalse%
\ {\isacharparenleft}{\kern0pt}rule\ conjI{\isacharcomma}{\kern0pt}\ rule\ subsetI{\isacharparenright}{\kern0pt}\ \isanewline
\ \ \ \isacommand{apply}\isamarkupfalse%
\ {\isacharparenleft}{\kern0pt}metis\ {\isacharparenleft}{\kern0pt}no{\isacharunderscore}{\kern0pt}types{\isacharcomma}{\kern0pt}\ lifting{\isacharparenright}{\kern0pt}\ assms{\isacharparenleft}{\kern0pt}{\isadigit{1}}{\isacharparenright}{\kern0pt}\ field{\isachardot}{\kern0pt}is{\isacharunderscore}{\kern0pt}ring\ imageE\ mem{\isacharunderscore}{\kern0pt}Collect{\isacharunderscore}{\kern0pt}eq\ \isanewline
\ \ \ \ \ \ \ \ partial{\isacharunderscore}{\kern0pt}object{\isachardot}{\kern0pt}select{\isacharunderscore}{\kern0pt}convs{\isacharparenleft}{\kern0pt}{\isadigit{1}}{\isacharparenright}{\kern0pt}\ ring{\isachardot}{\kern0pt}carrier{\isacharunderscore}{\kern0pt}is{\isacharunderscore}{\kern0pt}subring\ ring{\isachardot}{\kern0pt}coeff{\isacharunderscore}{\kern0pt}in{\isacharunderscore}{\kern0pt}carrier\isanewline
\ \ \ \ \ \ \ \ ring{\isachardot}{\kern0pt}polynomial{\isacharunderscore}{\kern0pt}in{\isacharunderscore}{\kern0pt}carrier\ univ{\isacharunderscore}{\kern0pt}poly{\isacharunderscore}{\kern0pt}def{\isacharparenright}{\kern0pt}\isanewline
\ \ \isacommand{by}\isamarkupfalse%
\ {\isacharparenleft}{\kern0pt}simp\ add{\isacharcolon}{\kern0pt}\ assms{\isacharparenleft}{\kern0pt}{\isadigit{1}}{\isacharparenright}{\kern0pt}\ field{\isachardot}{\kern0pt}is{\isacharunderscore}{\kern0pt}ring\ ring{\isachardot}{\kern0pt}coeff{\isacharunderscore}{\kern0pt}length{\isacharparenright}{\kern0pt}%
\endisatagproof
{\isafoldproof}%
%
\isadelimproof
\isanewline
%
\endisadelimproof
\isanewline
\isacommand{lemma}\isamarkupfalse%
\ poly{\isacharunderscore}{\kern0pt}neg{\isacharunderscore}{\kern0pt}coeff{\isacharcolon}{\kern0pt}\isanewline
\ \ \isakeyword{assumes}\ {\isachardoublequoteopen}domain\ F{\isachardoublequoteclose}\isanewline
\ \ \isakeyword{assumes}\ {\isachardoublequoteopen}x\ {\isasymin}\ carrier\ {\isacharparenleft}{\kern0pt}poly{\isacharunderscore}{\kern0pt}ring\ F{\isacharparenright}{\kern0pt}{\isachardoublequoteclose}\isanewline
\ \ \isakeyword{shows}\ {\isachardoublequoteopen}ring{\isachardot}{\kern0pt}coeff\ F\ {\isacharparenleft}{\kern0pt}{\isasymominus}\isactrlbsub poly{\isacharunderscore}{\kern0pt}ring\ F\isactrlesub \ x{\isacharparenright}{\kern0pt}\ k\ {\isacharequal}{\kern0pt}\ {\isasymominus}\isactrlbsub F\isactrlesub \ ring{\isachardot}{\kern0pt}coeff\ F\ x\ k{\isachardoublequoteclose}\isanewline
%
\isadelimproof
%
\endisadelimproof
%
\isatagproof
\isacommand{proof}\isamarkupfalse%
\ {\isacharminus}{\kern0pt}\isanewline
\ \ \isacommand{interpret}\isamarkupfalse%
\ ring\ {\isachardoublequoteopen}poly{\isacharunderscore}{\kern0pt}ring\ F{\isachardoublequoteclose}\isanewline
\ \ \ \ \isacommand{using}\isamarkupfalse%
\ assms\ cring{\isacharunderscore}{\kern0pt}def\ domain{\isachardot}{\kern0pt}univ{\isacharunderscore}{\kern0pt}poly{\isacharunderscore}{\kern0pt}is{\isacharunderscore}{\kern0pt}ring\ domain{\isacharunderscore}{\kern0pt}def\ ring{\isachardot}{\kern0pt}carrier{\isacharunderscore}{\kern0pt}is{\isacharunderscore}{\kern0pt}subring\ \isacommand{by}\isamarkupfalse%
\ blast\isanewline
\ \ \isacommand{have}\isamarkupfalse%
\ {\isachardoublequoteopen}{\isasymzero}\isactrlbsub poly{\isacharunderscore}{\kern0pt}ring\ F\isactrlesub \ {\isacharequal}{\kern0pt}\ x\ {\isasymominus}\isactrlbsub poly{\isacharunderscore}{\kern0pt}ring\ F\isactrlesub \ x{\isachardoublequoteclose}\ \isacommand{by}\isamarkupfalse%
\ {\isacharparenleft}{\kern0pt}metis\ assms{\isacharparenleft}{\kern0pt}{\isadigit{2}}{\isacharparenright}{\kern0pt}\ r{\isacharunderscore}{\kern0pt}right{\isacharunderscore}{\kern0pt}minus{\isacharunderscore}{\kern0pt}eq{\isacharparenright}{\kern0pt}\isanewline
\ \ \isacommand{hence}\isamarkupfalse%
\ {\isachardoublequoteopen}ring{\isachardot}{\kern0pt}coeff\ F\ {\isacharparenleft}{\kern0pt}{\isasymzero}\isactrlbsub poly{\isacharunderscore}{\kern0pt}ring\ F\isactrlesub {\isacharparenright}{\kern0pt}\ k\ {\isacharequal}{\kern0pt}\ ring{\isachardot}{\kern0pt}coeff\ F\ x\ k\ {\isasymoplus}\isactrlbsub F\isactrlesub \ ring{\isachardot}{\kern0pt}coeff\ F\ {\isacharparenleft}{\kern0pt}{\isasymominus}\isactrlbsub poly{\isacharunderscore}{\kern0pt}ring\ F\isactrlesub \ x{\isacharparenright}{\kern0pt}\ k{\isachardoublequoteclose}\isanewline
\ \ \ \ \isacommand{by}\isamarkupfalse%
\ {\isacharparenleft}{\kern0pt}metis\ assms\ cring{\isacharunderscore}{\kern0pt}def\ domain{\isachardot}{\kern0pt}univ{\isacharunderscore}{\kern0pt}poly{\isacharunderscore}{\kern0pt}a{\isacharunderscore}{\kern0pt}inv{\isacharunderscore}{\kern0pt}length\ domain{\isacharunderscore}{\kern0pt}def\ dual{\isacharunderscore}{\kern0pt}order{\isachardot}{\kern0pt}refl\ minus{\isacharunderscore}{\kern0pt}eq\ \isanewline
\ \ \ \ \ \ \ \ ring{\isachardot}{\kern0pt}carrier{\isacharunderscore}{\kern0pt}is{\isacharunderscore}{\kern0pt}subring\ ring{\isachardot}{\kern0pt}poly{\isacharunderscore}{\kern0pt}add{\isacharunderscore}{\kern0pt}coeff{\isacharunderscore}{\kern0pt}aux\ univ{\isacharunderscore}{\kern0pt}poly{\isacharunderscore}{\kern0pt}add{\isacharparenright}{\kern0pt}\isanewline
\ \ \isacommand{thus}\isamarkupfalse%
\ {\isacharquery}{\kern0pt}thesis\ \isanewline
\ \ \ \ \isacommand{by}\isamarkupfalse%
\ {\isacharparenleft}{\kern0pt}metis\ abelian{\isacharunderscore}{\kern0pt}group{\isachardot}{\kern0pt}minus{\isacharunderscore}{\kern0pt}equality\ add{\isachardot}{\kern0pt}l{\isacharunderscore}{\kern0pt}inv{\isacharunderscore}{\kern0pt}ex\ assms{\isacharparenleft}{\kern0pt}{\isadigit{1}}{\isacharparenright}{\kern0pt}\ assms{\isacharparenleft}{\kern0pt}{\isadigit{2}}{\isacharparenright}{\kern0pt}\ cring{\isacharunderscore}{\kern0pt}def\ \isanewline
\ \ \ \ \ \ \ \ domain{\isachardot}{\kern0pt}axioms{\isacharparenleft}{\kern0pt}{\isadigit{1}}{\isacharparenright}{\kern0pt}\ is{\isacharunderscore}{\kern0pt}abelian{\isacharunderscore}{\kern0pt}group\ mem{\isacharunderscore}{\kern0pt}Collect{\isacharunderscore}{\kern0pt}eq\ partial{\isacharunderscore}{\kern0pt}object{\isachardot}{\kern0pt}select{\isacharunderscore}{\kern0pt}convs{\isacharparenleft}{\kern0pt}{\isadigit{1}}{\isacharparenright}{\kern0pt}\ \isanewline
\ \ \ \ \ \ \ \ ring{\isachardot}{\kern0pt}carrier{\isacharunderscore}{\kern0pt}is{\isacharunderscore}{\kern0pt}subring\ ring{\isachardot}{\kern0pt}coeff{\isachardot}{\kern0pt}simps{\isacharparenleft}{\kern0pt}{\isadigit{1}}{\isacharparenright}{\kern0pt}\ ring{\isachardot}{\kern0pt}coeff{\isacharunderscore}{\kern0pt}in{\isacharunderscore}{\kern0pt}carrier\ ring{\isachardot}{\kern0pt}polynomial{\isacharunderscore}{\kern0pt}in{\isacharunderscore}{\kern0pt}carrier\isanewline
\ \ \ \ \ \ \ \ ring{\isachardot}{\kern0pt}ring{\isacharunderscore}{\kern0pt}simprules{\isacharparenleft}{\kern0pt}{\isadigit{2}}{\isadigit{0}}{\isacharparenright}{\kern0pt}\ ring{\isacharunderscore}{\kern0pt}def\ univ{\isacharunderscore}{\kern0pt}poly{\isacharunderscore}{\kern0pt}def\ univ{\isacharunderscore}{\kern0pt}poly{\isacharunderscore}{\kern0pt}zero{\isacharparenright}{\kern0pt}\isanewline
\isacommand{qed}\isamarkupfalse%
%
\endisatagproof
{\isafoldproof}%
%
\isadelimproof
\isanewline
%
\endisadelimproof
\isanewline
\isacommand{lemma}\isamarkupfalse%
\ poly{\isacharunderscore}{\kern0pt}substract{\isacharunderscore}{\kern0pt}coeff{\isacharcolon}{\kern0pt}\isanewline
\ \ \isakeyword{assumes}\ {\isachardoublequoteopen}domain\ F{\isachardoublequoteclose}\isanewline
\ \ \isakeyword{assumes}\ {\isachardoublequoteopen}x\ {\isasymin}\ carrier\ {\isacharparenleft}{\kern0pt}poly{\isacharunderscore}{\kern0pt}ring\ F{\isacharparenright}{\kern0pt}{\isachardoublequoteclose}\isanewline
\ \ \isakeyword{assumes}\ {\isachardoublequoteopen}y\ {\isasymin}\ carrier\ {\isacharparenleft}{\kern0pt}poly{\isacharunderscore}{\kern0pt}ring\ F{\isacharparenright}{\kern0pt}{\isachardoublequoteclose}\isanewline
\ \ \isakeyword{shows}\ {\isachardoublequoteopen}ring{\isachardot}{\kern0pt}coeff\ F\ {\isacharparenleft}{\kern0pt}x\ {\isasymominus}\isactrlbsub poly{\isacharunderscore}{\kern0pt}ring\ F\isactrlesub \ y{\isacharparenright}{\kern0pt}\ k\ {\isacharequal}{\kern0pt}\ ring{\isachardot}{\kern0pt}coeff\ F\ x\ k\ {\isasymominus}\isactrlbsub F\isactrlesub \ ring{\isachardot}{\kern0pt}coeff\ F\ y\ k{\isachardoublequoteclose}\isanewline
%
\isadelimproof
\ \ %
\endisadelimproof
%
\isatagproof
\isacommand{apply}\isamarkupfalse%
\ {\isacharparenleft}{\kern0pt}simp\ add{\isacharcolon}{\kern0pt}a{\isacharunderscore}{\kern0pt}minus{\isacharunderscore}{\kern0pt}def\ poly{\isacharunderscore}{\kern0pt}neg{\isacharunderscore}{\kern0pt}coeff{\isacharbrackleft}{\kern0pt}symmetric{\isacharbrackright}{\kern0pt}{\isacharparenright}{\kern0pt}\isanewline
\ \ \isacommand{using}\isamarkupfalse%
\ assms\ ring{\isachardot}{\kern0pt}poly{\isacharunderscore}{\kern0pt}add{\isacharunderscore}{\kern0pt}coeff\ \isanewline
\ \ \isacommand{by}\isamarkupfalse%
\ {\isacharparenleft}{\kern0pt}metis\ abelian{\isacharunderscore}{\kern0pt}group{\isachardot}{\kern0pt}a{\isacharunderscore}{\kern0pt}inv{\isacharunderscore}{\kern0pt}closed\ cring{\isacharunderscore}{\kern0pt}def\ domain{\isachardot}{\kern0pt}univ{\isacharunderscore}{\kern0pt}poly{\isacharunderscore}{\kern0pt}is{\isacharunderscore}{\kern0pt}abelian{\isacharunderscore}{\kern0pt}group\ domain{\isacharunderscore}{\kern0pt}def\ \isanewline
\ \ \ \ \ \ poly{\isacharunderscore}{\kern0pt}neg{\isacharunderscore}{\kern0pt}coeff\ ring{\isachardot}{\kern0pt}carrier{\isacharunderscore}{\kern0pt}is{\isacharunderscore}{\kern0pt}subring\ ring{\isachardot}{\kern0pt}polynomial{\isacharunderscore}{\kern0pt}incl\ univ{\isacharunderscore}{\kern0pt}poly{\isacharunderscore}{\kern0pt}add\ univ{\isacharunderscore}{\kern0pt}poly{\isacharunderscore}{\kern0pt}carrier{\isacharparenright}{\kern0pt}%
\endisatagproof
{\isafoldproof}%
%
\isadelimproof
\isanewline
%
\endisadelimproof
\isanewline
\isacommand{lemma}\isamarkupfalse%
\ poly{\isacharunderscore}{\kern0pt}substract{\isacharunderscore}{\kern0pt}eval{\isacharcolon}{\kern0pt}\isanewline
\ \ \isakeyword{assumes}\ {\isachardoublequoteopen}domain\ F{\isachardoublequoteclose}\isanewline
\ \ \isakeyword{assumes}\ {\isachardoublequoteopen}i\ {\isasymin}\ carrier\ F{\isachardoublequoteclose}\isanewline
\ \ \isakeyword{assumes}\ {\isachardoublequoteopen}x\ {\isasymin}\ carrier\ {\isacharparenleft}{\kern0pt}poly{\isacharunderscore}{\kern0pt}ring\ F{\isacharparenright}{\kern0pt}{\isachardoublequoteclose}\isanewline
\ \ \isakeyword{assumes}\ {\isachardoublequoteopen}y\ {\isasymin}\ carrier\ {\isacharparenleft}{\kern0pt}poly{\isacharunderscore}{\kern0pt}ring\ F{\isacharparenright}{\kern0pt}{\isachardoublequoteclose}\isanewline
\ \ \isakeyword{shows}\ {\isachardoublequoteopen}ring{\isachardot}{\kern0pt}eval\ F\ {\isacharparenleft}{\kern0pt}x\ {\isasymominus}\isactrlbsub poly{\isacharunderscore}{\kern0pt}ring\ F\isactrlesub \ y{\isacharparenright}{\kern0pt}\ i\ {\isacharequal}{\kern0pt}\ ring{\isachardot}{\kern0pt}eval\ F\ x\ i\ {\isasymominus}\isactrlbsub F\isactrlesub \ ring{\isachardot}{\kern0pt}eval\ F\ y\ i{\isachardoublequoteclose}\isanewline
%
\isadelimproof
%
\endisadelimproof
%
\isatagproof
\isacommand{proof}\isamarkupfalse%
\ {\isacharminus}{\kern0pt}\isanewline
\ \ \isacommand{have}\isamarkupfalse%
\ {\isachardoublequoteopen}subring\ {\isacharparenleft}{\kern0pt}carrier\ F{\isacharparenright}{\kern0pt}\ F{\isachardoublequoteclose}\ \isanewline
\ \ \ \ \isacommand{using}\isamarkupfalse%
\ assms{\isacharparenleft}{\kern0pt}{\isadigit{1}}{\isacharparenright}{\kern0pt}\ cring{\isacharunderscore}{\kern0pt}def\ domain{\isacharunderscore}{\kern0pt}def\ ring{\isachardot}{\kern0pt}carrier{\isacharunderscore}{\kern0pt}is{\isacharunderscore}{\kern0pt}subring\ \isacommand{by}\isamarkupfalse%
\ blast\isanewline
\ \ \isacommand{hence}\isamarkupfalse%
\ {\isachardoublequoteopen}ring{\isacharunderscore}{\kern0pt}hom{\isacharunderscore}{\kern0pt}cring\ {\isacharparenleft}{\kern0pt}poly{\isacharunderscore}{\kern0pt}ring\ F{\isacharparenright}{\kern0pt}\ F\ {\isacharparenleft}{\kern0pt}{\isasymlambda}p{\isachardot}{\kern0pt}\ {\isacharparenleft}{\kern0pt}ring{\isachardot}{\kern0pt}eval\ F\ p{\isacharparenright}{\kern0pt}\ i{\isacharparenright}{\kern0pt}{\isachardoublequoteclose}\isanewline
\ \ \ \ \isacommand{by}\isamarkupfalse%
\ {\isacharparenleft}{\kern0pt}simp\ add{\isacharcolon}{\kern0pt}\ assms{\isacharparenleft}{\kern0pt}{\isadigit{1}}{\isacharparenright}{\kern0pt}\ assms{\isacharparenleft}{\kern0pt}{\isadigit{2}}{\isacharparenright}{\kern0pt}\ domain{\isachardot}{\kern0pt}eval{\isacharunderscore}{\kern0pt}cring{\isacharunderscore}{\kern0pt}hom{\isacharparenright}{\kern0pt}\isanewline
\ \ \isacommand{then}\isamarkupfalse%
\ \isacommand{show}\isamarkupfalse%
\ {\isacharquery}{\kern0pt}thesis\ \isacommand{by}\isamarkupfalse%
\ {\isacharparenleft}{\kern0pt}meson\ \ ring{\isacharunderscore}{\kern0pt}hom{\isacharunderscore}{\kern0pt}cring{\isachardot}{\kern0pt}hom{\isacharunderscore}{\kern0pt}sub\ assms{\isacharparenleft}{\kern0pt}{\isadigit{3}}{\isacharparenright}{\kern0pt}\ assms{\isacharparenleft}{\kern0pt}{\isadigit{4}}{\isacharparenright}{\kern0pt}{\isacharparenright}{\kern0pt}\isanewline
\isacommand{qed}\isamarkupfalse%
%
\endisatagproof
{\isafoldproof}%
%
\isadelimproof
\isanewline
%
\endisadelimproof
\isanewline
\isacommand{lemma}\isamarkupfalse%
\ poly{\isacharunderscore}{\kern0pt}degree{\isacharunderscore}{\kern0pt}bound{\isacharunderscore}{\kern0pt}from{\isacharunderscore}{\kern0pt}coeff{\isacharcolon}{\kern0pt}\isanewline
\ \ \isakeyword{assumes}\ {\isachardoublequoteopen}ring\ F{\isachardoublequoteclose}\isanewline
\ \ \isakeyword{assumes}\ {\isachardoublequoteopen}x\ {\isasymin}\ carrier\ {\isacharparenleft}{\kern0pt}poly{\isacharunderscore}{\kern0pt}ring\ F{\isacharparenright}{\kern0pt}{\isachardoublequoteclose}\isanewline
\ \ \isakeyword{assumes}\ {\isachardoublequoteopen}{\isasymAnd}k{\isachardot}{\kern0pt}\ k\ {\isasymge}\ n\ {\isasymLongrightarrow}\ ring{\isachardot}{\kern0pt}coeff\ F\ x\ k\ {\isacharequal}{\kern0pt}\ {\isasymzero}\isactrlbsub F\isactrlesub {\isachardoublequoteclose}\isanewline
\ \ \isakeyword{shows}\ {\isachardoublequoteopen}degree\ x\ {\isacharless}{\kern0pt}\ n\ {\isasymor}\ x\ {\isacharequal}{\kern0pt}\ {\isasymzero}\isactrlbsub poly{\isacharunderscore}{\kern0pt}ring\ F\isactrlesub {\isachardoublequoteclose}\isanewline
%
\isadelimproof
%
\endisadelimproof
%
\isatagproof
\isacommand{proof}\isamarkupfalse%
\ {\isacharparenleft}{\kern0pt}rule\ ccontr{\isacharparenright}{\kern0pt}\isanewline
\ \ \isacommand{assume}\isamarkupfalse%
\ a{\isacharcolon}{\kern0pt}{\isachardoublequoteopen}{\isasymnot}{\isacharparenleft}{\kern0pt}degree\ x\ {\isacharless}{\kern0pt}\ n\ {\isasymor}\ x\ {\isacharequal}{\kern0pt}\ {\isasymzero}\isactrlbsub poly{\isacharunderscore}{\kern0pt}ring\ F\isactrlesub {\isacharparenright}{\kern0pt}{\isachardoublequoteclose}\isanewline
\ \ \isacommand{hence}\isamarkupfalse%
\ b{\isacharcolon}{\kern0pt}{\isachardoublequoteopen}lead{\isacharunderscore}{\kern0pt}coeff\ x\ {\isasymnoteq}\ {\isasymzero}\isactrlbsub F\isactrlesub {\isachardoublequoteclose}\ \isanewline
\ \ \ \ \isacommand{by}\isamarkupfalse%
\ {\isacharparenleft}{\kern0pt}metis\ assms{\isacharparenleft}{\kern0pt}{\isadigit{2}}{\isacharparenright}{\kern0pt}\ polynomial{\isacharunderscore}{\kern0pt}def\ univ{\isacharunderscore}{\kern0pt}poly{\isacharunderscore}{\kern0pt}carrier\ univ{\isacharunderscore}{\kern0pt}poly{\isacharunderscore}{\kern0pt}zero{\isacharparenright}{\kern0pt}\isanewline
\ \ \isacommand{hence}\isamarkupfalse%
\ {\isachardoublequoteopen}ring{\isachardot}{\kern0pt}coeff\ F\ x\ {\isacharparenleft}{\kern0pt}degree\ x{\isacharparenright}{\kern0pt}\ {\isasymnoteq}\ {\isasymzero}\isactrlbsub F\isactrlesub {\isachardoublequoteclose}\ \isanewline
\ \ \ \ \isacommand{by}\isamarkupfalse%
\ {\isacharparenleft}{\kern0pt}metis\ a\ assms{\isacharparenleft}{\kern0pt}{\isadigit{1}}{\isacharparenright}{\kern0pt}\ ring{\isachardot}{\kern0pt}lead{\isacharunderscore}{\kern0pt}coeff{\isacharunderscore}{\kern0pt}simp\ univ{\isacharunderscore}{\kern0pt}poly{\isacharunderscore}{\kern0pt}zero{\isacharparenright}{\kern0pt}\isanewline
\ \ \isacommand{moreover}\isamarkupfalse%
\ \isacommand{have}\isamarkupfalse%
\ {\isachardoublequoteopen}degree\ x\ {\isasymge}\ n{\isachardoublequoteclose}\ \isacommand{by}\isamarkupfalse%
\ {\isacharparenleft}{\kern0pt}meson\ a\ not{\isacharunderscore}{\kern0pt}le{\isacharparenright}{\kern0pt}\isanewline
\ \ \isacommand{ultimately}\isamarkupfalse%
\ \isacommand{show}\isamarkupfalse%
\ {\isachardoublequoteopen}False{\isachardoublequoteclose}\ \isacommand{using}\isamarkupfalse%
\ assms{\isacharparenleft}{\kern0pt}{\isadigit{3}}{\isacharparenright}{\kern0pt}\ \isacommand{by}\isamarkupfalse%
\ blast\isanewline
\isacommand{qed}\isamarkupfalse%
%
\endisatagproof
{\isafoldproof}%
%
\isadelimproof
\isanewline
%
\endisadelimproof
\isanewline
\isacommand{lemma}\isamarkupfalse%
\ max{\isacharunderscore}{\kern0pt}roots{\isacharcolon}{\kern0pt}\isanewline
\ \ \isakeyword{assumes}\ {\isachardoublequoteopen}field\ R{\isachardoublequoteclose}\isanewline
\ \ \isakeyword{assumes}\ {\isachardoublequoteopen}p\ {\isasymin}\ carrier\ {\isacharparenleft}{\kern0pt}poly{\isacharunderscore}{\kern0pt}ring\ R{\isacharparenright}{\kern0pt}{\isachardoublequoteclose}\isanewline
\ \ \isakeyword{assumes}\ {\isachardoublequoteopen}K\ {\isasymsubseteq}\ carrier\ R{\isachardoublequoteclose}\isanewline
\ \ \isakeyword{assumes}\ {\isachardoublequoteopen}finite\ K{\isachardoublequoteclose}\isanewline
\ \ \isakeyword{assumes}\ {\isachardoublequoteopen}degree\ p\ {\isacharless}{\kern0pt}\ card\ K{\isachardoublequoteclose}\isanewline
\ \ \isakeyword{assumes}\ {\isachardoublequoteopen}{\isasymAnd}x{\isachardot}{\kern0pt}\ x\ {\isasymin}\ K\ {\isasymLongrightarrow}\ ring{\isachardot}{\kern0pt}eval\ R\ p\ x\ {\isacharequal}{\kern0pt}\ {\isasymzero}\isactrlbsub R\isactrlesub {\isachardoublequoteclose}\isanewline
\ \ \isakeyword{shows}\ {\isachardoublequoteopen}p\ {\isacharequal}{\kern0pt}\ {\isasymzero}\isactrlbsub poly{\isacharunderscore}{\kern0pt}ring\ R\isactrlesub {\isachardoublequoteclose}\isanewline
%
\isadelimproof
%
\endisadelimproof
%
\isatagproof
\isacommand{proof}\isamarkupfalse%
\ {\isacharparenleft}{\kern0pt}rule\ ccontr{\isacharparenright}{\kern0pt}\isanewline
\ \ \isacommand{assume}\isamarkupfalse%
\ {\isachardoublequoteopen}p\ {\isasymnoteq}\ {\isasymzero}\isactrlbsub poly{\isacharunderscore}{\kern0pt}ring\ R\isactrlesub {\isachardoublequoteclose}\isanewline
\ \ \isacommand{hence}\isamarkupfalse%
\ a{\isacharcolon}{\kern0pt}{\isachardoublequoteopen}p\ {\isasymnoteq}\ {\isacharbrackleft}{\kern0pt}{\isacharbrackright}{\kern0pt}{\isachardoublequoteclose}\ \isacommand{by}\isamarkupfalse%
\ {\isacharparenleft}{\kern0pt}simp\ add{\isacharcolon}{\kern0pt}\ univ{\isacharunderscore}{\kern0pt}poly{\isacharunderscore}{\kern0pt}zero{\isacharparenright}{\kern0pt}\isanewline
\ \ \isacommand{have}\isamarkupfalse%
\ {\isachardoublequoteopen}{\isasymAnd}x{\isachardot}{\kern0pt}\ count\ {\isacharparenleft}{\kern0pt}mset{\isacharunderscore}{\kern0pt}set\ K{\isacharparenright}{\kern0pt}\ x\ {\isasymle}\ count\ {\isacharparenleft}{\kern0pt}ring{\isachardot}{\kern0pt}roots\ R\ p{\isacharparenright}{\kern0pt}\ x{\isachardoublequoteclose}\isanewline
\ \ \isacommand{proof}\isamarkupfalse%
\ {\isacharminus}{\kern0pt}\isanewline
\ \ \ \ \isacommand{fix}\isamarkupfalse%
\ x\isanewline
\ \ \ \ \isacommand{show}\isamarkupfalse%
\ {\isachardoublequoteopen}count\ {\isacharparenleft}{\kern0pt}mset{\isacharunderscore}{\kern0pt}set\ K{\isacharparenright}{\kern0pt}\ x\ {\isasymle}\ count\ {\isacharparenleft}{\kern0pt}ring{\isachardot}{\kern0pt}roots\ R\ p{\isacharparenright}{\kern0pt}\ x{\isachardoublequoteclose}\isanewline
\ \ \ \ \isacommand{proof}\isamarkupfalse%
\ {\isacharparenleft}{\kern0pt}cases\ {\isachardoublequoteopen}x\ {\isasymin}\ K{\isachardoublequoteclose}{\isacharparenright}{\kern0pt}\isanewline
\ \ \ \ \ \ \isacommand{case}\isamarkupfalse%
\ True\isanewline
\ \ \ \ \ \ \isacommand{hence}\isamarkupfalse%
\ {\isachardoublequoteopen}ring{\isachardot}{\kern0pt}is{\isacharunderscore}{\kern0pt}root\ R\ p\ x{\isachardoublequoteclose}\ \isacommand{using}\isamarkupfalse%
\ assms{\isacharparenleft}{\kern0pt}{\isadigit{3}}{\isacharparenright}{\kern0pt}\ assms{\isacharparenleft}{\kern0pt}{\isadigit{6}}{\isacharparenright}{\kern0pt}\ \isanewline
\ \ \ \ \ \ \ \ \isacommand{by}\isamarkupfalse%
\ {\isacharparenleft}{\kern0pt}meson\ a\ assms{\isacharparenleft}{\kern0pt}{\isadigit{1}}{\isacharparenright}{\kern0pt}\ field{\isachardot}{\kern0pt}is{\isacharunderscore}{\kern0pt}ring\ ring{\isachardot}{\kern0pt}is{\isacharunderscore}{\kern0pt}root{\isacharunderscore}{\kern0pt}def\ subsetD{\isacharparenright}{\kern0pt}\isanewline
\ \ \ \ \ \ \isacommand{hence}\isamarkupfalse%
\ {\isachardoublequoteopen}x\ {\isasymin}\ set{\isacharunderscore}{\kern0pt}mset\ {\isacharparenleft}{\kern0pt}ring{\isachardot}{\kern0pt}roots\ R\ p{\isacharparenright}{\kern0pt}{\isachardoublequoteclose}\isanewline
\ \ \ \ \ \ \ \ \isacommand{using}\isamarkupfalse%
\ assms{\isacharparenleft}{\kern0pt}{\isadigit{2}}{\isacharparenright}{\kern0pt}\ assms{\isacharparenleft}{\kern0pt}{\isadigit{1}}{\isacharparenright}{\kern0pt}\ domain{\isachardot}{\kern0pt}roots{\isacharunderscore}{\kern0pt}mem{\isacharunderscore}{\kern0pt}iff{\isacharunderscore}{\kern0pt}is{\isacharunderscore}{\kern0pt}root\ field{\isacharunderscore}{\kern0pt}def\ \isacommand{by}\isamarkupfalse%
\ force\isanewline
\ \ \ \ \ \ \isacommand{hence}\isamarkupfalse%
\ {\isachardoublequoteopen}{\isadigit{1}}\ {\isasymle}\ count\ {\isacharparenleft}{\kern0pt}ring{\isachardot}{\kern0pt}roots\ R\ p{\isacharparenright}{\kern0pt}\ x{\isachardoublequoteclose}\ \isacommand{by}\isamarkupfalse%
\ simp\isanewline
\ \ \ \ \ \ \isacommand{moreover}\isamarkupfalse%
\ \isacommand{have}\isamarkupfalse%
\ {\isachardoublequoteopen}count\ {\isacharparenleft}{\kern0pt}mset{\isacharunderscore}{\kern0pt}set\ K{\isacharparenright}{\kern0pt}\ x\ {\isacharequal}{\kern0pt}\ {\isadigit{1}}{\isachardoublequoteclose}\ \isacommand{using}\isamarkupfalse%
\ True\ assms{\isacharparenleft}{\kern0pt}{\isadigit{4}}{\isacharparenright}{\kern0pt}\ \isacommand{by}\isamarkupfalse%
\ simp\isanewline
\ \ \ \ \ \ \isacommand{ultimately}\isamarkupfalse%
\ \isacommand{show}\isamarkupfalse%
\ {\isacharquery}{\kern0pt}thesis\ \isacommand{by}\isamarkupfalse%
\ presburger\isanewline
\ \ \ \ \isacommand{next}\isamarkupfalse%
\isanewline
\ \ \ \ \ \ \isacommand{case}\isamarkupfalse%
\ False\isanewline
\ \ \ \ \ \ \isacommand{hence}\isamarkupfalse%
\ {\isachardoublequoteopen}count\ {\isacharparenleft}{\kern0pt}mset{\isacharunderscore}{\kern0pt}set\ K{\isacharparenright}{\kern0pt}\ x\ {\isacharequal}{\kern0pt}\ {\isadigit{0}}{\isachardoublequoteclose}\ \isacommand{by}\isamarkupfalse%
\ simp\isanewline
\ \ \ \ \ \ \isacommand{then}\isamarkupfalse%
\ \isacommand{show}\isamarkupfalse%
\ {\isacharquery}{\kern0pt}thesis\ \isacommand{by}\isamarkupfalse%
\ presburger\isanewline
\ \ \ \ \isacommand{qed}\isamarkupfalse%
\isanewline
\ \ \isacommand{qed}\isamarkupfalse%
\isanewline
\ \ \isacommand{hence}\isamarkupfalse%
\ {\isachardoublequoteopen}mset{\isacharunderscore}{\kern0pt}set\ K\ {\isasymsubseteq}{\isacharhash}{\kern0pt}\ ring{\isachardot}{\kern0pt}roots\ R\ p{\isachardoublequoteclose}\isanewline
\ \ \ \ \isacommand{by}\isamarkupfalse%
\ {\isacharparenleft}{\kern0pt}simp\ add{\isacharcolon}{\kern0pt}\ subseteq{\isacharunderscore}{\kern0pt}mset{\isacharunderscore}{\kern0pt}def{\isacharparenright}{\kern0pt}\isanewline
\ \ \isacommand{hence}\isamarkupfalse%
\ {\isachardoublequoteopen}card\ K\ {\isasymle}\ size\ {\isacharparenleft}{\kern0pt}ring{\isachardot}{\kern0pt}roots\ R\ p{\isacharparenright}{\kern0pt}{\isachardoublequoteclose}\ \isanewline
\ \ \ \ \isacommand{by}\isamarkupfalse%
\ {\isacharparenleft}{\kern0pt}metis\ size{\isacharunderscore}{\kern0pt}mset{\isacharunderscore}{\kern0pt}mono\ size{\isacharunderscore}{\kern0pt}mset{\isacharunderscore}{\kern0pt}set{\isacharparenright}{\kern0pt}\isanewline
\ \ \isacommand{moreover}\isamarkupfalse%
\ \isacommand{have}\isamarkupfalse%
\ {\isachardoublequoteopen}size\ {\isacharparenleft}{\kern0pt}ring{\isachardot}{\kern0pt}roots\ R\ p{\isacharparenright}{\kern0pt}\ {\isasymle}\ degree\ p{\isachardoublequoteclose}\isanewline
\ \ \ \ \isacommand{using}\isamarkupfalse%
\ a\ field{\isachardot}{\kern0pt}size{\isacharunderscore}{\kern0pt}roots{\isacharunderscore}{\kern0pt}le{\isacharunderscore}{\kern0pt}degree\ assms\ \isacommand{by}\isamarkupfalse%
\ auto\isanewline
\ \ \isacommand{ultimately}\isamarkupfalse%
\ \isacommand{show}\isamarkupfalse%
\ {\isachardoublequoteopen}False{\isachardoublequoteclose}\ \isacommand{using}\isamarkupfalse%
\ assms{\isacharparenleft}{\kern0pt}{\isadigit{5}}{\isacharparenright}{\kern0pt}\ \isanewline
\ \ \ \ \isacommand{by}\isamarkupfalse%
\ {\isacharparenleft}{\kern0pt}meson\ leD\ less{\isacharunderscore}{\kern0pt}le{\isacharunderscore}{\kern0pt}trans{\isacharparenright}{\kern0pt}\isanewline
\isacommand{qed}\isamarkupfalse%
%
\endisatagproof
{\isafoldproof}%
%
\isadelimproof
\isanewline
%
\endisadelimproof
\isanewline
\isacommand{lemma}\isamarkupfalse%
\ split{\isacharunderscore}{\kern0pt}poly{\isacharunderscore}{\kern0pt}inj{\isacharcolon}{\kern0pt}\isanewline
\ \ \isakeyword{assumes}\ {\isachardoublequoteopen}field\ F{\isachardoublequoteclose}\isanewline
\ \ \isakeyword{assumes}\ {\isachardoublequoteopen}finite\ K{\isachardoublequoteclose}\isanewline
\ \ \isakeyword{assumes}\ {\isachardoublequoteopen}K\ {\isasymsubseteq}\ carrier\ F{\isachardoublequoteclose}\isanewline
\ \ \isakeyword{shows}\ {\isachardoublequoteopen}inj{\isacharunderscore}{\kern0pt}on\ {\isacharparenleft}{\kern0pt}split{\isacharunderscore}{\kern0pt}poly\ F\ K{\isacharparenright}{\kern0pt}\ {\isacharparenleft}{\kern0pt}carrier\ {\isacharparenleft}{\kern0pt}poly{\isacharunderscore}{\kern0pt}ring\ F{\isacharparenright}{\kern0pt}{\isacharparenright}{\kern0pt}{\isachardoublequoteclose}\isanewline
%
\isadelimproof
%
\endisadelimproof
%
\isatagproof
\isacommand{proof}\isamarkupfalse%
\isanewline
\ \ \isacommand{have}\isamarkupfalse%
\ ring{\isacharunderscore}{\kern0pt}F{\isacharcolon}{\kern0pt}\ {\isachardoublequoteopen}ring\ F{\isachardoublequoteclose}\ \isacommand{using}\isamarkupfalse%
\ assms{\isacharparenleft}{\kern0pt}{\isadigit{1}}{\isacharparenright}{\kern0pt}\ field{\isachardot}{\kern0pt}is{\isacharunderscore}{\kern0pt}ring\ \isacommand{by}\isamarkupfalse%
\ blast\isanewline
\ \ \isacommand{have}\isamarkupfalse%
\ domain{\isacharunderscore}{\kern0pt}F{\isacharcolon}{\kern0pt}\ {\isachardoublequoteopen}domain\ F{\isachardoublequoteclose}\ \isacommand{using}\isamarkupfalse%
\ assms{\isacharparenleft}{\kern0pt}{\isadigit{1}}{\isacharparenright}{\kern0pt}\ field{\isacharunderscore}{\kern0pt}def\ \isacommand{by}\isamarkupfalse%
\ blast\isanewline
\ \ \isacommand{fix}\isamarkupfalse%
\ x\isanewline
\ \ \isacommand{fix}\isamarkupfalse%
\ y\isanewline
\ \ \isacommand{assume}\isamarkupfalse%
\ a{\isadigit{1}}{\isacharcolon}{\kern0pt}{\isachardoublequoteopen}x\ {\isasymin}\ carrier\ {\isacharparenleft}{\kern0pt}poly{\isacharunderscore}{\kern0pt}ring\ F{\isacharparenright}{\kern0pt}{\isachardoublequoteclose}\isanewline
\ \ \isacommand{assume}\isamarkupfalse%
\ a{\isadigit{2}}{\isacharcolon}{\kern0pt}{\isachardoublequoteopen}y\ {\isasymin}\ carrier\ {\isacharparenleft}{\kern0pt}poly{\isacharunderscore}{\kern0pt}ring\ F{\isacharparenright}{\kern0pt}{\isachardoublequoteclose}\isanewline
\ \ \isacommand{assume}\isamarkupfalse%
\ a{\isadigit{3}}{\isacharcolon}{\kern0pt}{\isachardoublequoteopen}split{\isacharunderscore}{\kern0pt}poly\ F\ K\ x\ {\isacharequal}{\kern0pt}\ split{\isacharunderscore}{\kern0pt}poly\ F\ K\ y{\isachardoublequoteclose}\isanewline
\isanewline
\ \ \isacommand{have}\isamarkupfalse%
\ x{\isacharunderscore}{\kern0pt}y{\isacharunderscore}{\kern0pt}carrier{\isacharcolon}{\kern0pt}\ {\isachardoublequoteopen}x\ {\isasymominus}\isactrlbsub poly{\isacharunderscore}{\kern0pt}ring\ F\isactrlesub \ y\ {\isasymin}\ carrier\ {\isacharparenleft}{\kern0pt}poly{\isacharunderscore}{\kern0pt}ring\ F{\isacharparenright}{\kern0pt}{\isachardoublequoteclose}\ \isacommand{using}\isamarkupfalse%
\ a{\isadigit{1}}\ a{\isadigit{2}}\isanewline
\ \ \ \ \isacommand{by}\isamarkupfalse%
\ {\isacharparenleft}{\kern0pt}simp\ add{\isacharcolon}{\kern0pt}\ assms{\isacharparenleft}{\kern0pt}{\isadigit{1}}{\isacharparenright}{\kern0pt}\ domain{\isachardot}{\kern0pt}univ{\isacharunderscore}{\kern0pt}poly{\isacharunderscore}{\kern0pt}is{\isacharunderscore}{\kern0pt}ring\ field{\isachardot}{\kern0pt}axioms{\isacharparenleft}{\kern0pt}{\isadigit{1}}{\isacharparenright}{\kern0pt}\ ring{\isachardot}{\kern0pt}carrier{\isacharunderscore}{\kern0pt}is{\isacharunderscore}{\kern0pt}subring\ \isanewline
\ \ \ \ \ \ \ \ ring{\isachardot}{\kern0pt}ring{\isacharunderscore}{\kern0pt}simprules{\isacharparenleft}{\kern0pt}{\isadigit{4}}{\isacharparenright}{\kern0pt}\ ring{\isacharunderscore}{\kern0pt}F{\isacharparenright}{\kern0pt}\isanewline
\ \ \isacommand{have}\isamarkupfalse%
\ {\isachardoublequoteopen}{\isasymAnd}k{\isachardot}{\kern0pt}\ ring{\isachardot}{\kern0pt}coeff\ F\ x\ {\isacharparenleft}{\kern0pt}k{\isacharplus}{\kern0pt}card\ K{\isacharparenright}{\kern0pt}\ {\isacharequal}{\kern0pt}\ ring{\isachardot}{\kern0pt}coeff\ F\ y\ {\isacharparenleft}{\kern0pt}k{\isacharplus}{\kern0pt}card\ K{\isacharparenright}{\kern0pt}{\isachardoublequoteclose}\isanewline
\ \ \ \ \isacommand{using}\isamarkupfalse%
\ a{\isadigit{3}}\ \isacommand{apply}\isamarkupfalse%
\ {\isacharparenleft}{\kern0pt}simp\ add{\isacharcolon}{\kern0pt}split{\isacharunderscore}{\kern0pt}poly{\isacharunderscore}{\kern0pt}def{\isacharparenright}{\kern0pt}\ \isacommand{by}\isamarkupfalse%
\ meson\isanewline
\ \ \isacommand{hence}\isamarkupfalse%
\ {\isachardoublequoteopen}{\isasymAnd}k{\isachardot}{\kern0pt}\ ring{\isachardot}{\kern0pt}coeff\ F\ {\isacharparenleft}{\kern0pt}x\ {\isasymominus}\isactrlbsub poly{\isacharunderscore}{\kern0pt}ring\ F\isactrlesub \ y{\isacharparenright}{\kern0pt}\ {\isacharparenleft}{\kern0pt}k{\isacharplus}{\kern0pt}card\ K{\isacharparenright}{\kern0pt}\ {\isacharequal}{\kern0pt}\ {\isasymzero}\isactrlbsub F\isactrlesub {\isachardoublequoteclose}\isanewline
\ \ \ \ \isacommand{apply}\isamarkupfalse%
\ {\isacharparenleft}{\kern0pt}simp\ add{\isacharcolon}{\kern0pt}domain{\isacharunderscore}{\kern0pt}F\ a{\isadigit{1}}\ a{\isadigit{2}}\ poly{\isacharunderscore}{\kern0pt}substract{\isacharunderscore}{\kern0pt}coeff{\isacharparenright}{\kern0pt}\isanewline
\ \ \ \ \isacommand{by}\isamarkupfalse%
\ {\isacharparenleft}{\kern0pt}meson\ a{\isadigit{2}}\ ring{\isachardot}{\kern0pt}carrier{\isacharunderscore}{\kern0pt}is{\isacharunderscore}{\kern0pt}subring\ ring{\isachardot}{\kern0pt}coeff{\isacharunderscore}{\kern0pt}in{\isacharunderscore}{\kern0pt}carrier\ \isanewline
\ \ \ \ \ \ \ ring{\isachardot}{\kern0pt}polynomial{\isacharunderscore}{\kern0pt}in{\isacharunderscore}{\kern0pt}carrier\ ring{\isachardot}{\kern0pt}r{\isacharunderscore}{\kern0pt}right{\isacharunderscore}{\kern0pt}minus{\isacharunderscore}{\kern0pt}eq\ ring{\isacharunderscore}{\kern0pt}F\ univ{\isacharunderscore}{\kern0pt}poly{\isacharunderscore}{\kern0pt}carrier{\isacharparenright}{\kern0pt}\isanewline
\ \ \isacommand{hence}\isamarkupfalse%
\ {\isachardoublequoteopen}degree\ {\isacharparenleft}{\kern0pt}x\ {\isasymominus}\isactrlbsub poly{\isacharunderscore}{\kern0pt}ring\ F\isactrlesub \ y{\isacharparenright}{\kern0pt}\ {\isacharless}{\kern0pt}\ card\ K\ {\isasymor}\ {\isacharparenleft}{\kern0pt}x\ {\isasymominus}\isactrlbsub poly{\isacharunderscore}{\kern0pt}ring\ F\isactrlesub \ y{\isacharparenright}{\kern0pt}\ {\isacharequal}{\kern0pt}\ {\isasymzero}\isactrlbsub poly{\isacharunderscore}{\kern0pt}ring\ F\isactrlesub {\isachardoublequoteclose}\isanewline
\ \ \ \ \isacommand{by}\isamarkupfalse%
\ {\isacharparenleft}{\kern0pt}metis\ add{\isachardot}{\kern0pt}commute\ le{\isacharunderscore}{\kern0pt}Suc{\isacharunderscore}{\kern0pt}ex\ poly{\isacharunderscore}{\kern0pt}degree{\isacharunderscore}{\kern0pt}bound{\isacharunderscore}{\kern0pt}from{\isacharunderscore}{\kern0pt}coeff\ x{\isacharunderscore}{\kern0pt}y{\isacharunderscore}{\kern0pt}carrier\ ring{\isacharunderscore}{\kern0pt}F{\isacharparenright}{\kern0pt}\isanewline
\ \ \isacommand{moreover}\isamarkupfalse%
\ \isacommand{have}\isamarkupfalse%
\ {\isachardoublequoteopen}{\isasymAnd}k{\isachardot}{\kern0pt}\ k\ {\isasymin}\ K\ {\isasymLongrightarrow}\ ring{\isachardot}{\kern0pt}eval\ F\ x\ k\ {\isacharequal}{\kern0pt}\ ring{\isachardot}{\kern0pt}eval\ F\ y\ k{\isachardoublequoteclose}\isanewline
\ \ \ \ \isacommand{using}\isamarkupfalse%
\ a{\isadigit{3}}\ \isacommand{apply}\isamarkupfalse%
\ {\isacharparenleft}{\kern0pt}simp\ add{\isacharcolon}{\kern0pt}split{\isacharunderscore}{\kern0pt}poly{\isacharunderscore}{\kern0pt}def\ restrict{\isacharunderscore}{\kern0pt}def{\isacharparenright}{\kern0pt}\ \isacommand{by}\isamarkupfalse%
\ meson\ \isanewline
\ \ \isacommand{hence}\isamarkupfalse%
\ {\isachardoublequoteopen}{\isasymAnd}k{\isachardot}{\kern0pt}\ k\ {\isasymin}\ K\ {\isasymLongrightarrow}\ ring{\isachardot}{\kern0pt}eval\ F\ x\ k\ {\isasymominus}\isactrlbsub F\isactrlesub \ ring{\isachardot}{\kern0pt}eval\ F\ y\ k\ {\isacharequal}{\kern0pt}\ {\isasymzero}\isactrlbsub F\isactrlesub {\isachardoublequoteclose}\isanewline
\ \ \ \ \isacommand{by}\isamarkupfalse%
\ {\isacharparenleft}{\kern0pt}metis\ {\isacharparenleft}{\kern0pt}no{\isacharunderscore}{\kern0pt}types{\isacharcomma}{\kern0pt}\ opaque{\isacharunderscore}{\kern0pt}lifting{\isacharparenright}{\kern0pt}\ a{\isadigit{2}}\ assms{\isacharparenleft}{\kern0pt}{\isadigit{3}}{\isacharparenright}{\kern0pt}\ ring{\isachardot}{\kern0pt}eval{\isacharunderscore}{\kern0pt}in{\isacharunderscore}{\kern0pt}carrier\ ring{\isachardot}{\kern0pt}polynomial{\isacharunderscore}{\kern0pt}incl\ \isanewline
\ \ \ \ \ \ \ \ ring{\isachardot}{\kern0pt}r{\isacharunderscore}{\kern0pt}right{\isacharunderscore}{\kern0pt}minus{\isacharunderscore}{\kern0pt}eq\ ring{\isacharunderscore}{\kern0pt}F\ subsetD\ univ{\isacharunderscore}{\kern0pt}poly{\isacharunderscore}{\kern0pt}carrier{\isacharparenright}{\kern0pt}\isanewline
\ \ \isacommand{hence}\isamarkupfalse%
\ {\isachardoublequoteopen}{\isasymAnd}k{\isachardot}{\kern0pt}\ k\ {\isasymin}\ K\ {\isasymLongrightarrow}\ ring{\isachardot}{\kern0pt}eval\ F\ {\isacharparenleft}{\kern0pt}x\ {\isasymominus}\isactrlbsub poly{\isacharunderscore}{\kern0pt}ring\ F\isactrlesub \ y{\isacharparenright}{\kern0pt}\ k\ {\isacharequal}{\kern0pt}\ \ {\isasymzero}\isactrlbsub F\isactrlesub {\isachardoublequoteclose}\isanewline
\ \ \ \ \isacommand{using}\isamarkupfalse%
\ domain{\isacharunderscore}{\kern0pt}F\ a{\isadigit{1}}\ a{\isadigit{2}}\ assms{\isacharparenleft}{\kern0pt}{\isadigit{3}}{\isacharparenright}{\kern0pt}\ poly{\isacharunderscore}{\kern0pt}substract{\isacharunderscore}{\kern0pt}eval\ \isacommand{by}\isamarkupfalse%
\ {\isacharparenleft}{\kern0pt}metis\ {\isacharparenleft}{\kern0pt}no{\isacharunderscore}{\kern0pt}types{\isacharcomma}{\kern0pt}\ opaque{\isacharunderscore}{\kern0pt}lifting{\isacharparenright}{\kern0pt}\ subsetD{\isacharparenright}{\kern0pt}\isanewline
\ \ \isacommand{ultimately}\isamarkupfalse%
\ \isacommand{have}\isamarkupfalse%
\ {\isachardoublequoteopen}x\ {\isasymominus}\isactrlbsub poly{\isacharunderscore}{\kern0pt}ring\ F\isactrlesub \ y\ {\isacharequal}{\kern0pt}\ {\isasymzero}\isactrlbsub poly{\isacharunderscore}{\kern0pt}ring\ F\isactrlesub {\isachardoublequoteclose}\isanewline
\ \ \ \ \isacommand{using}\isamarkupfalse%
\ max{\isacharunderscore}{\kern0pt}roots\ x{\isacharunderscore}{\kern0pt}y{\isacharunderscore}{\kern0pt}carrier\ assms\ \isacommand{by}\isamarkupfalse%
\ blast\isanewline
\ \ \isacommand{then}\isamarkupfalse%
\ \isacommand{show}\isamarkupfalse%
\ {\isachardoublequoteopen}x\ {\isacharequal}{\kern0pt}\ y{\isachardoublequoteclose}\isanewline
\ \ \ \ \isacommand{by}\isamarkupfalse%
\ {\isacharparenleft}{\kern0pt}meson\ assms{\isacharparenleft}{\kern0pt}{\isadigit{1}}{\isacharparenright}{\kern0pt}\ a{\isadigit{1}}\ a{\isadigit{2}}\ domain{\isachardot}{\kern0pt}univ{\isacharunderscore}{\kern0pt}poly{\isacharunderscore}{\kern0pt}is{\isacharunderscore}{\kern0pt}ring\ field{\isacharunderscore}{\kern0pt}def\ ring{\isachardot}{\kern0pt}carrier{\isacharunderscore}{\kern0pt}is{\isacharunderscore}{\kern0pt}subring\ \isanewline
\ \ \ \ \ \ \ \ ring{\isachardot}{\kern0pt}r{\isacharunderscore}{\kern0pt}right{\isacharunderscore}{\kern0pt}minus{\isacharunderscore}{\kern0pt}eq\ ring{\isacharunderscore}{\kern0pt}F{\isacharparenright}{\kern0pt}\isanewline
\isacommand{qed}\isamarkupfalse%
%
\endisatagproof
{\isafoldproof}%
%
\isadelimproof
\isanewline
%
\endisadelimproof
\isanewline
\isacommand{lemma}\isamarkupfalse%
\isanewline
\ \ \isakeyword{assumes}\ {\isachardoublequoteopen}field\ F\ {\isasymand}\ finite\ {\isacharparenleft}{\kern0pt}carrier\ F{\isacharparenright}{\kern0pt}{\isachardoublequoteclose}\isanewline
\ \ \isakeyword{shows}\ \isanewline
\ \ \ \ poly{\isacharunderscore}{\kern0pt}count{\isacharcolon}{\kern0pt}{\isachardoublequoteopen}card\ {\isacharparenleft}{\kern0pt}bounded{\isacharunderscore}{\kern0pt}degree{\isacharunderscore}{\kern0pt}polynomials\ F\ n{\isacharparenright}{\kern0pt}\ {\isacharequal}{\kern0pt}\ card\ {\isacharparenleft}{\kern0pt}carrier\ F{\isacharparenright}{\kern0pt}{\isacharcircum}{\kern0pt}n{\isachardoublequoteclose}\ {\isacharparenleft}{\kern0pt}\isakeyword{is}\ {\isacharquery}{\kern0pt}A{\isacharparenright}{\kern0pt}\ \isakeyword{and}\isanewline
\ \ \ \ finite{\isacharunderscore}{\kern0pt}poly{\isacharunderscore}{\kern0pt}count{\isacharcolon}{\kern0pt}\ {\isachardoublequoteopen}finite\ {\isacharparenleft}{\kern0pt}bounded{\isacharunderscore}{\kern0pt}degree{\isacharunderscore}{\kern0pt}polynomials\ F\ n{\isacharparenright}{\kern0pt}{\isachardoublequoteclose}\ {\isacharparenleft}{\kern0pt}\isakeyword{is}\ {\isacharquery}{\kern0pt}B{\isacharparenright}{\kern0pt}\isanewline
%
\isadelimproof
%
\endisadelimproof
%
\isatagproof
\isacommand{proof}\isamarkupfalse%
\ {\isacharminus}{\kern0pt}\isanewline
\ \ \isacommand{have}\isamarkupfalse%
\ a{\isacharcolon}{\kern0pt}{\isachardoublequoteopen}ring\ F{\isachardoublequoteclose}\ \isacommand{using}\isamarkupfalse%
\ assms{\isacharparenleft}{\kern0pt}{\isadigit{1}}{\isacharparenright}{\kern0pt}\ \isacommand{by}\isamarkupfalse%
\ {\isacharparenleft}{\kern0pt}simp\ add{\isacharcolon}{\kern0pt}\ field{\isachardot}{\kern0pt}is{\isacharunderscore}{\kern0pt}ring{\isacharparenright}{\kern0pt}\isanewline
\ \ \isacommand{show}\isamarkupfalse%
\ {\isacharquery}{\kern0pt}A\ \isacommand{using}\isamarkupfalse%
\ a\ bounded{\isacharunderscore}{\kern0pt}degree{\isacharunderscore}{\kern0pt}polynomials{\isacharunderscore}{\kern0pt}count\ assms\ \isacommand{by}\isamarkupfalse%
\ blast\isanewline
\ \ \isacommand{show}\isamarkupfalse%
\ {\isacharquery}{\kern0pt}B\ \isacommand{using}\isamarkupfalse%
\ a\ fin{\isacharunderscore}{\kern0pt}degree{\isacharunderscore}{\kern0pt}bounded\ assms\ \isacommand{by}\isamarkupfalse%
\ blast\isanewline
\isacommand{qed}\isamarkupfalse%
%
\endisatagproof
{\isafoldproof}%
%
\isadelimproof
\isanewline
%
\endisadelimproof
\isanewline
\isacommand{lemma}\isamarkupfalse%
\isanewline
\ \ \isakeyword{assumes}\ {\isachardoublequoteopen}finite\ {\isacharparenleft}{\kern0pt}B\ {\isacharcolon}{\kern0pt}{\isacharcolon}{\kern0pt}\ {\isacharprime}{\kern0pt}b\ set{\isacharparenright}{\kern0pt}{\isachardoublequoteclose}\isanewline
\ \ \isakeyword{assumes}\ {\isachardoublequoteopen}y\ {\isasymin}\ B{\isachardoublequoteclose}\isanewline
\ \ \isakeyword{shows}\ \isanewline
\ \ \ \ card{\isacharunderscore}{\kern0pt}mostly{\isacharunderscore}{\kern0pt}constant{\isacharunderscore}{\kern0pt}maps{\isacharcolon}{\kern0pt}\ \isanewline
\ \ \ \ {\isachardoublequoteopen}card\ {\isacharbraceleft}{\kern0pt}f{\isachardot}{\kern0pt}\ range\ f\ {\isasymsubseteq}\ B\ {\isasymand}\ {\isacharparenleft}{\kern0pt}{\isasymforall}x{\isachardot}{\kern0pt}\ x\ {\isasymge}\ n\ {\isasymlongrightarrow}\ f\ x\ {\isacharequal}{\kern0pt}\ y{\isacharparenright}{\kern0pt}{\isacharbraceright}{\kern0pt}\ {\isacharequal}{\kern0pt}\ card\ B\ {\isacharcircum}{\kern0pt}\ n{\isachardoublequoteclose}\ {\isacharparenleft}{\kern0pt}\isakeyword{is}\ {\isachardoublequoteopen}card\ {\isacharquery}{\kern0pt}A\ {\isacharequal}{\kern0pt}\ {\isacharquery}{\kern0pt}B{\isachardoublequoteclose}{\isacharparenright}{\kern0pt}\ \isakeyword{and}\isanewline
\ \ \ \ finite{\isacharunderscore}{\kern0pt}mostly{\isacharunderscore}{\kern0pt}constant{\isacharunderscore}{\kern0pt}maps{\isacharcolon}{\kern0pt}\isanewline
\ \ \ \ {\isachardoublequoteopen}finite\ {\isacharbraceleft}{\kern0pt}f{\isachardot}{\kern0pt}\ range\ f\ {\isasymsubseteq}\ B\ {\isasymand}\ {\isacharparenleft}{\kern0pt}{\isasymforall}x{\isachardot}{\kern0pt}\ x\ {\isasymge}\ n\ {\isasymlongrightarrow}\ f\ x\ {\isacharequal}{\kern0pt}\ y{\isacharparenright}{\kern0pt}{\isacharbraceright}{\kern0pt}{\isachardoublequoteclose}\isanewline
%
\isadelimproof
%
\endisadelimproof
%
\isatagproof
\isacommand{proof}\isamarkupfalse%
\ {\isacharminus}{\kern0pt}\isanewline
\ \ \isacommand{define}\isamarkupfalse%
\ C\ \isakeyword{where}\ {\isachardoublequoteopen}C\ {\isacharequal}{\kern0pt}\ {\isacharbraceleft}{\kern0pt}k{\isachardot}{\kern0pt}\ k\ {\isacharless}{\kern0pt}\ n{\isacharbraceright}{\kern0pt}\ {\isasymrightarrow}\isactrlsub E\ B{\isachardoublequoteclose}\isanewline
\ \ \isacommand{define}\isamarkupfalse%
\ forward\ \isakeyword{where}\ {\isachardoublequoteopen}forward\ {\isacharequal}{\kern0pt}\ {\isacharparenleft}{\kern0pt}{\isasymlambda}{\isacharparenleft}{\kern0pt}f\ {\isacharcolon}{\kern0pt}{\isacharcolon}{\kern0pt}\ nat\ {\isasymRightarrow}\ {\isacharprime}{\kern0pt}b{\isacharparenright}{\kern0pt}{\isachardot}{\kern0pt}\ restrict\ f\ {\isacharbraceleft}{\kern0pt}k{\isachardot}{\kern0pt}\ k{\isacharless}{\kern0pt}\ n{\isacharbraceright}{\kern0pt}{\isacharparenright}{\kern0pt}{\isachardoublequoteclose}\isanewline
\ \ \isacommand{define}\isamarkupfalse%
\ backward\ \isakeyword{where}\ {\isachardoublequoteopen}backward\ {\isacharequal}{\kern0pt}\ {\isacharparenleft}{\kern0pt}{\isasymlambda}f\ k{\isachardot}{\kern0pt}\ if\ k\ {\isacharless}{\kern0pt}\ n\ then\ f\ k\ else\ y{\isacharparenright}{\kern0pt}{\isachardoublequoteclose}\isanewline
\isanewline
\ \ \isacommand{have}\isamarkupfalse%
\ forward{\isacharunderscore}{\kern0pt}inject{\isacharcolon}{\kern0pt}{\isachardoublequoteopen}inj{\isacharunderscore}{\kern0pt}on\ forward\ {\isacharquery}{\kern0pt}A{\isachardoublequoteclose}\isanewline
\ \ \ \ \isacommand{apply}\isamarkupfalse%
\ {\isacharparenleft}{\kern0pt}rule\ inj{\isacharunderscore}{\kern0pt}onI{\isacharcomma}{\kern0pt}\ rule\ ext{\isacharcomma}{\kern0pt}\ simp\ add{\isacharcolon}{\kern0pt}forward{\isacharunderscore}{\kern0pt}def\ restrict{\isacharunderscore}{\kern0pt}def{\isacharparenright}{\kern0pt}\isanewline
\ \ \ \ \isacommand{by}\isamarkupfalse%
\ {\isacharparenleft}{\kern0pt}metis\ not{\isacharunderscore}{\kern0pt}le{\isacharparenright}{\kern0pt}\isanewline
\isanewline
\ \ \isacommand{have}\isamarkupfalse%
\ forward{\isacharunderscore}{\kern0pt}image{\isacharcolon}{\kern0pt}{\isachardoublequoteopen}forward\ {\isacharbackquote}{\kern0pt}\ {\isacharquery}{\kern0pt}A\ {\isasymsubseteq}\ C{\isachardoublequoteclose}\isanewline
\ \ \ \ \isacommand{apply}\isamarkupfalse%
\ {\isacharparenleft}{\kern0pt}rule\ image{\isacharunderscore}{\kern0pt}subsetI{\isacharcomma}{\kern0pt}\ simp\ add{\isacharcolon}{\kern0pt}forward{\isacharunderscore}{\kern0pt}def\ C{\isacharunderscore}{\kern0pt}def{\isacharparenright}{\kern0pt}\ \isacommand{by}\isamarkupfalse%
\ blast\isanewline
\ \ \isacommand{have}\isamarkupfalse%
\ finite{\isacharunderscore}{\kern0pt}C{\isacharcolon}{\kern0pt}{\isachardoublequoteopen}finite\ C{\isachardoublequoteclose}\isanewline
\ \ \ \ \isacommand{by}\isamarkupfalse%
\ {\isacharparenleft}{\kern0pt}simp\ add{\isacharcolon}{\kern0pt}C{\isacharunderscore}{\kern0pt}def\ finite{\isacharunderscore}{\kern0pt}PiE\ assms{\isacharparenleft}{\kern0pt}{\isadigit{1}}{\isacharparenright}{\kern0pt}{\isacharparenright}{\kern0pt}\ \isanewline
\isanewline
\ \ \isacommand{have}\isamarkupfalse%
\ card{\isacharunderscore}{\kern0pt}ineq{\isacharunderscore}{\kern0pt}{\isadigit{1}}{\isacharcolon}{\kern0pt}\ {\isachardoublequoteopen}card\ {\isacharquery}{\kern0pt}A\ {\isasymle}\ card\ C{\isachardoublequoteclose}\isanewline
\ \ \ \ \isacommand{using}\isamarkupfalse%
\ card{\isacharunderscore}{\kern0pt}image\ card{\isacharunderscore}{\kern0pt}mono\ forward{\isacharunderscore}{\kern0pt}inject\ forward{\isacharunderscore}{\kern0pt}image\ finite{\isacharunderscore}{\kern0pt}C\ \isacommand{by}\isamarkupfalse%
\ {\isacharparenleft}{\kern0pt}metis\ {\isacharparenleft}{\kern0pt}no{\isacharunderscore}{\kern0pt}types{\isacharcomma}{\kern0pt}\ lifting{\isacharparenright}{\kern0pt}{\isacharparenright}{\kern0pt}\isanewline
\isanewline
\ \ \isacommand{show}\isamarkupfalse%
\ {\isachardoublequoteopen}finite\ {\isacharquery}{\kern0pt}A{\isachardoublequoteclose}\isanewline
\ \ \ \ \isacommand{using}\isamarkupfalse%
\ inj{\isacharunderscore}{\kern0pt}on{\isacharunderscore}{\kern0pt}finite\ forward{\isacharunderscore}{\kern0pt}inject\ forward{\isacharunderscore}{\kern0pt}image\ finite{\isacharunderscore}{\kern0pt}C\ \isacommand{by}\isamarkupfalse%
\ blast\isanewline
\ \ \isacommand{moreover}\isamarkupfalse%
\ \isacommand{have}\isamarkupfalse%
\ {\isachardoublequoteopen}inj{\isacharunderscore}{\kern0pt}on\ backward\ C{\isachardoublequoteclose}\isanewline
\ \ \ \ \isacommand{apply}\isamarkupfalse%
\ {\isacharparenleft}{\kern0pt}rule\ inj{\isacharunderscore}{\kern0pt}onI{\isacharcomma}{\kern0pt}\ rule\ ext{\isacharcomma}{\kern0pt}\ simp\ add{\isacharcolon}{\kern0pt}backward{\isacharunderscore}{\kern0pt}def\ C{\isacharunderscore}{\kern0pt}def{\isacharparenright}{\kern0pt}\ \isanewline
\ \ \ \ \isacommand{by}\isamarkupfalse%
\ {\isacharparenleft}{\kern0pt}metis\ {\isacharparenleft}{\kern0pt}no{\isacharunderscore}{\kern0pt}types{\isacharcomma}{\kern0pt}\ lifting{\isacharparenright}{\kern0pt}\ PiE{\isacharunderscore}{\kern0pt}ext\ mem{\isacharunderscore}{\kern0pt}Collect{\isacharunderscore}{\kern0pt}eq{\isacharparenright}{\kern0pt}\isanewline
\ \ \isacommand{moreover}\isamarkupfalse%
\ \isacommand{have}\isamarkupfalse%
\ {\isachardoublequoteopen}backward\ {\isacharbackquote}{\kern0pt}\ C\ {\isasymsubseteq}\ {\isacharquery}{\kern0pt}A{\isachardoublequoteclose}\isanewline
\ \ \ \ \isacommand{apply}\isamarkupfalse%
\ {\isacharparenleft}{\kern0pt}rule\ image{\isacharunderscore}{\kern0pt}subsetI{\isacharcomma}{\kern0pt}\ simp\ add{\isacharcolon}{\kern0pt}backward{\isacharunderscore}{\kern0pt}def\ C{\isacharunderscore}{\kern0pt}def{\isacharparenright}{\kern0pt}\isanewline
\ \ \ \ \isacommand{apply}\isamarkupfalse%
\ {\isacharparenleft}{\kern0pt}rule\ conjI{\isacharcomma}{\kern0pt}\ rule\ image{\isacharunderscore}{\kern0pt}subsetI{\isacharparenright}{\kern0pt}\ \isacommand{apply}\isamarkupfalse%
\ blast\isanewline
\ \ \ \ \isacommand{by}\isamarkupfalse%
\ {\isacharparenleft}{\kern0pt}rule\ image{\isacharunderscore}{\kern0pt}subsetI{\isacharcomma}{\kern0pt}\ simp\ add{\isacharcolon}{\kern0pt}assms{\isacharparenright}{\kern0pt}\isanewline
\ \ \isacommand{ultimately}\isamarkupfalse%
\ \isacommand{have}\isamarkupfalse%
\ \ card{\isacharunderscore}{\kern0pt}ineq{\isacharunderscore}{\kern0pt}{\isadigit{2}}{\isacharcolon}{\kern0pt}\ {\isachardoublequoteopen}card\ C\ {\isasymle}\ card\ {\isacharquery}{\kern0pt}A{\isachardoublequoteclose}\ \isacommand{by}\isamarkupfalse%
\ {\isacharparenleft}{\kern0pt}metis\ {\isacharparenleft}{\kern0pt}no{\isacharunderscore}{\kern0pt}types{\isacharcomma}{\kern0pt}\ lifting{\isacharparenright}{\kern0pt}\ card{\isacharunderscore}{\kern0pt}image\ card{\isacharunderscore}{\kern0pt}mono{\isacharparenright}{\kern0pt}\isanewline
\isanewline
\ \ \isacommand{have}\isamarkupfalse%
\ {\isachardoublequoteopen}card\ {\isacharquery}{\kern0pt}A\ {\isacharequal}{\kern0pt}\ card\ C{\isachardoublequoteclose}\ \isacommand{using}\isamarkupfalse%
\ card{\isacharunderscore}{\kern0pt}ineq{\isacharunderscore}{\kern0pt}{\isadigit{1}}\ card{\isacharunderscore}{\kern0pt}ineq{\isacharunderscore}{\kern0pt}{\isadigit{2}}\ \isacommand{by}\isamarkupfalse%
\ auto\isanewline
\ \ \isacommand{moreover}\isamarkupfalse%
\ \isacommand{have}\isamarkupfalse%
\ {\isachardoublequoteopen}card\ C\ {\isacharequal}{\kern0pt}\ card\ B\ {\isacharcircum}{\kern0pt}\ n{\isachardoublequoteclose}\ \isacommand{using}\isamarkupfalse%
\ C{\isacharunderscore}{\kern0pt}def\ assms{\isacharparenleft}{\kern0pt}{\isadigit{1}}{\isacharparenright}{\kern0pt}\ \isacommand{by}\isamarkupfalse%
\ {\isacharparenleft}{\kern0pt}simp\ add{\isacharcolon}{\kern0pt}\ card{\isacharunderscore}{\kern0pt}PiE{\isacharparenright}{\kern0pt}\isanewline
\ \ \isacommand{ultimately}\isamarkupfalse%
\ \isacommand{show}\isamarkupfalse%
\ {\isachardoublequoteopen}card\ {\isacharquery}{\kern0pt}A\ {\isacharequal}{\kern0pt}\ {\isacharquery}{\kern0pt}B{\isachardoublequoteclose}\ \isacommand{by}\isamarkupfalse%
\ auto\isanewline
\isacommand{qed}\isamarkupfalse%
%
\endisatagproof
{\isafoldproof}%
%
\isadelimproof
\isanewline
%
\endisadelimproof
\isanewline
\isacommand{lemma}\isamarkupfalse%
\ split{\isacharunderscore}{\kern0pt}poly{\isacharunderscore}{\kern0pt}surj{\isacharcolon}{\kern0pt}\isanewline
\ \ \isakeyword{assumes}\ {\isachardoublequoteopen}field\ F{\isachardoublequoteclose}\isanewline
\ \ \isakeyword{assumes}\ {\isachardoublequoteopen}finite\ {\isacharparenleft}{\kern0pt}carrier\ F{\isacharparenright}{\kern0pt}{\isachardoublequoteclose}\isanewline
\ \ \isakeyword{assumes}\ {\isachardoublequoteopen}K\ {\isasymsubseteq}\ carrier\ F{\isachardoublequoteclose}\isanewline
\ \ \isakeyword{shows}\ {\isachardoublequoteopen}split{\isacharunderscore}{\kern0pt}poly\ F\ K\ {\isacharbackquote}{\kern0pt}\ bounded{\isacharunderscore}{\kern0pt}degree{\isacharunderscore}{\kern0pt}polynomials\ F\ {\isacharparenleft}{\kern0pt}card\ K\ {\isacharplus}{\kern0pt}\ n{\isacharparenright}{\kern0pt}\ {\isacharequal}{\kern0pt}\isanewline
\ \ \ \ \ \ \ \ {\isacharparenleft}{\kern0pt}K\ {\isasymrightarrow}\isactrlsub E\ carrier\ F{\isacharparenright}{\kern0pt}\ {\isasymtimes}\ {\isacharbraceleft}{\kern0pt}f{\isachardot}{\kern0pt}\ range\ f\ {\isasymsubseteq}\ carrier\ F\ {\isasymand}\ {\isacharparenleft}{\kern0pt}{\isasymforall}k\ {\isasymge}\ n{\isachardot}{\kern0pt}\ f\ k\ {\isacharequal}{\kern0pt}\ {\isasymzero}\isactrlbsub F\isactrlesub {\isacharparenright}{\kern0pt}{\isacharbraceright}{\kern0pt}{\isachardoublequoteclose}\ \isanewline
\ \ \ \ \ \ {\isacharparenleft}{\kern0pt}\isakeyword{is}\ {\isachardoublequoteopen}split{\isacharunderscore}{\kern0pt}poly\ F\ K\ {\isacharbackquote}{\kern0pt}\ {\isacharquery}{\kern0pt}A\ {\isacharequal}{\kern0pt}\ {\isacharquery}{\kern0pt}B{\isachardoublequoteclose}{\isacharparenright}{\kern0pt}\isanewline
%
\isadelimproof
%
\endisadelimproof
%
\isatagproof
\isacommand{proof}\isamarkupfalse%
\ {\isacharminus}{\kern0pt}\isanewline
\ \ \isacommand{define}\isamarkupfalse%
\ M\ \isakeyword{where}\ {\isachardoublequoteopen}M\ {\isacharequal}{\kern0pt}\ split{\isacharunderscore}{\kern0pt}poly\ F\ K\ {\isacharbackquote}{\kern0pt}\ {\isacharquery}{\kern0pt}A{\isachardoublequoteclose}\isanewline
\ \ \isacommand{have}\isamarkupfalse%
\ a{\isacharcolon}{\kern0pt}{\isachardoublequoteopen}{\isasymzero}\isactrlbsub F\isactrlesub \ {\isasymin}\ carrier\ F{\isachardoublequoteclose}\ \isacommand{using}\isamarkupfalse%
\ assms{\isacharparenleft}{\kern0pt}{\isadigit{1}}{\isacharparenright}{\kern0pt}\ \isanewline
\ \ \ \ \isacommand{by}\isamarkupfalse%
\ {\isacharparenleft}{\kern0pt}simp\ add{\isacharcolon}{\kern0pt}\ field{\isachardot}{\kern0pt}is{\isacharunderscore}{\kern0pt}ring\ ring{\isachardot}{\kern0pt}ring{\isacharunderscore}{\kern0pt}simprules{\isacharparenleft}{\kern0pt}{\isadigit{2}}{\isacharparenright}{\kern0pt}{\isacharparenright}{\kern0pt}\isanewline
\ \ \isacommand{have}\isamarkupfalse%
\ b{\isacharcolon}{\kern0pt}{\isachardoublequoteopen}finite\ K{\isachardoublequoteclose}\ \isacommand{using}\isamarkupfalse%
\ assms{\isacharparenleft}{\kern0pt}{\isadigit{2}}{\isacharparenright}{\kern0pt}\ assms{\isacharparenleft}{\kern0pt}{\isadigit{3}}{\isacharparenright}{\kern0pt}\ finite{\isacharunderscore}{\kern0pt}subset\ \isacommand{by}\isamarkupfalse%
\ blast\isanewline
\ \ \isacommand{moreover}\isamarkupfalse%
\ \isacommand{have}\isamarkupfalse%
\ {\isachardoublequoteopen}{\isacharquery}{\kern0pt}A\ {\isasymsubseteq}\ carrier\ {\isacharparenleft}{\kern0pt}poly{\isacharunderscore}{\kern0pt}ring\ F{\isacharparenright}{\kern0pt}{\isachardoublequoteclose}\isanewline
\ \ \ \ \isacommand{by}\isamarkupfalse%
\ {\isacharparenleft}{\kern0pt}simp\ add{\isacharcolon}{\kern0pt}\ Collect{\isacharunderscore}{\kern0pt}mono{\isacharunderscore}{\kern0pt}iff\ bounded{\isacharunderscore}{\kern0pt}degree{\isacharunderscore}{\kern0pt}polynomials{\isacharunderscore}{\kern0pt}def{\isacharparenright}{\kern0pt}\isanewline
\ \ \isacommand{ultimately}\isamarkupfalse%
\ \isacommand{have}\isamarkupfalse%
\ {\isachardoublequoteopen}inj{\isacharunderscore}{\kern0pt}on\ {\isacharparenleft}{\kern0pt}split{\isacharunderscore}{\kern0pt}poly\ F\ K{\isacharparenright}{\kern0pt}\ {\isacharquery}{\kern0pt}A{\isachardoublequoteclose}\ \isanewline
\ \ \ \ \isacommand{by}\isamarkupfalse%
\ {\isacharparenleft}{\kern0pt}meson\ split{\isacharunderscore}{\kern0pt}poly{\isacharunderscore}{\kern0pt}inj\ assms{\isacharparenleft}{\kern0pt}{\isadigit{1}}{\isacharparenright}{\kern0pt}\ assms{\isacharparenleft}{\kern0pt}{\isadigit{3}}{\isacharparenright}{\kern0pt}\ inj{\isacharunderscore}{\kern0pt}on{\isacharunderscore}{\kern0pt}subset{\isacharparenright}{\kern0pt}\isanewline
\ \ \isacommand{moreover}\isamarkupfalse%
\ \isacommand{have}\isamarkupfalse%
\ {\isachardoublequoteopen}finite\ {\isacharquery}{\kern0pt}A{\isachardoublequoteclose}\ \isacommand{using}\isamarkupfalse%
\ finite{\isacharunderscore}{\kern0pt}poly{\isacharunderscore}{\kern0pt}count\ assms{\isacharparenleft}{\kern0pt}{\isadigit{2}}{\isacharparenright}{\kern0pt}\ assms{\isacharparenleft}{\kern0pt}{\isadigit{1}}{\isacharparenright}{\kern0pt}\ \isacommand{by}\isamarkupfalse%
\ blast\isanewline
\ \ \isacommand{ultimately}\isamarkupfalse%
\ \isacommand{have}\isamarkupfalse%
\ {\isachardoublequoteopen}card\ {\isacharquery}{\kern0pt}A\ {\isacharequal}{\kern0pt}\ card\ M{\isachardoublequoteclose}\ \isacommand{by}\isamarkupfalse%
\ {\isacharparenleft}{\kern0pt}simp\ add{\isacharcolon}{\kern0pt}\ M{\isacharunderscore}{\kern0pt}def\ card{\isacharunderscore}{\kern0pt}image{\isacharparenright}{\kern0pt}\isanewline
\ \ \isacommand{hence}\isamarkupfalse%
\ {\isachardoublequoteopen}card\ M\ {\isacharequal}{\kern0pt}\ card\ {\isacharparenleft}{\kern0pt}carrier\ F{\isacharparenright}{\kern0pt}{\isacharcircum}{\kern0pt}{\isacharparenleft}{\kern0pt}card\ K\ {\isacharplus}{\kern0pt}\ n{\isacharparenright}{\kern0pt}{\isachardoublequoteclose}\isanewline
\ \ \ \ \isacommand{using}\isamarkupfalse%
\ poly{\isacharunderscore}{\kern0pt}count\ assms{\isacharparenleft}{\kern0pt}{\isadigit{2}}{\isacharparenright}{\kern0pt}\ assms{\isacharparenleft}{\kern0pt}{\isadigit{1}}{\isacharparenright}{\kern0pt}\ \isacommand{by}\isamarkupfalse%
\ metis\isanewline
\ \ \isacommand{moreover}\isamarkupfalse%
\ \isacommand{have}\isamarkupfalse%
\ {\isachardoublequoteopen}M\ {\isasymsubseteq}\ {\isacharquery}{\kern0pt}B{\isachardoublequoteclose}\ \isacommand{using}\isamarkupfalse%
\ split{\isacharunderscore}{\kern0pt}poly{\isacharunderscore}{\kern0pt}image\ M{\isacharunderscore}{\kern0pt}def\ assms\ \isacommand{by}\isamarkupfalse%
\ blast\isanewline
\ \ \isacommand{moreover}\isamarkupfalse%
\ \isacommand{have}\isamarkupfalse%
\ {\isachardoublequoteopen}card\ {\isacharquery}{\kern0pt}B\ {\isacharequal}{\kern0pt}\ card\ {\isacharparenleft}{\kern0pt}carrier\ F{\isacharparenright}{\kern0pt}{\isacharcircum}{\kern0pt}{\isacharparenleft}{\kern0pt}card\ K\ {\isacharplus}{\kern0pt}\ n{\isacharparenright}{\kern0pt}{\isachardoublequoteclose}\ \isanewline
\ \ \ \ \isacommand{by}\isamarkupfalse%
\ {\isacharparenleft}{\kern0pt}simp\ add{\isacharcolon}{\kern0pt}\ a\ assms\ b\ card{\isacharunderscore}{\kern0pt}mostly{\isacharunderscore}{\kern0pt}constant{\isacharunderscore}{\kern0pt}maps\ card{\isacharunderscore}{\kern0pt}PiE\ power{\isacharunderscore}{\kern0pt}add\ card{\isacharunderscore}{\kern0pt}cartesian{\isacharunderscore}{\kern0pt}product{\isacharparenright}{\kern0pt}\ \isanewline
\ \ \isacommand{moreover}\isamarkupfalse%
\ \isacommand{have}\isamarkupfalse%
\ {\isachardoublequoteopen}finite\ {\isacharquery}{\kern0pt}B{\isachardoublequoteclose}\ \isacommand{using}\isamarkupfalse%
\ assms{\isacharparenleft}{\kern0pt}{\isadigit{2}}{\isacharparenright}{\kern0pt}\ a\ b\ \ \isanewline
\ \ \ \ \isacommand{by}\isamarkupfalse%
\ {\isacharparenleft}{\kern0pt}simp\ add{\isacharcolon}{\kern0pt}\ finite{\isacharunderscore}{\kern0pt}mostly{\isacharunderscore}{\kern0pt}constant{\isacharunderscore}{\kern0pt}maps\ finite{\isacharunderscore}{\kern0pt}PiE{\isacharparenright}{\kern0pt}\isanewline
\ \ \isacommand{ultimately}\isamarkupfalse%
\ \isacommand{have}\isamarkupfalse%
\ {\isachardoublequoteopen}M\ {\isacharequal}{\kern0pt}\ {\isacharquery}{\kern0pt}B{\isachardoublequoteclose}\ \isacommand{by}\isamarkupfalse%
\ {\isacharparenleft}{\kern0pt}simp\ add{\isacharcolon}{\kern0pt}\ card{\isacharunderscore}{\kern0pt}seteq{\isacharparenright}{\kern0pt}\isanewline
\ \ \isacommand{thus}\isamarkupfalse%
\ {\isacharquery}{\kern0pt}thesis\ \isacommand{using}\isamarkupfalse%
\ M{\isacharunderscore}{\kern0pt}def\ \isacommand{by}\isamarkupfalse%
\ auto\isanewline
\isacommand{qed}\isamarkupfalse%
%
\endisatagproof
{\isafoldproof}%
%
\isadelimproof
\isanewline
%
\endisadelimproof
\isanewline
\isacommand{lemma}\isamarkupfalse%
\ inv{\isacharunderscore}{\kern0pt}subsetI{\isacharcolon}{\kern0pt}\isanewline
\ \ \isakeyword{assumes}\ {\isachardoublequoteopen}{\isasymAnd}x{\isachardot}{\kern0pt}\ x\ {\isasymin}\ A\ {\isasymLongrightarrow}\ f\ x\ {\isasymin}\ B\ {\isasymLongrightarrow}\ x\ {\isasymin}\ C{\isachardoublequoteclose}\isanewline
\ \ \isakeyword{shows}\ {\isachardoublequoteopen}f\ {\isacharminus}{\kern0pt}{\isacharbackquote}{\kern0pt}\ B\ {\isasyminter}\ A\ {\isasymsubseteq}\ C{\isachardoublequoteclose}\isanewline
%
\isadelimproof
\ \ %
\endisadelimproof
%
\isatagproof
\isacommand{using}\isamarkupfalse%
\ assms\ \isacommand{by}\isamarkupfalse%
\ force%
\endisatagproof
{\isafoldproof}%
%
\isadelimproof
\isanewline
%
\endisadelimproof
\isanewline
\isacommand{lemma}\isamarkupfalse%
\ interpolating{\isacharunderscore}{\kern0pt}polynomials{\isacharunderscore}{\kern0pt}count{\isacharcolon}{\kern0pt}\isanewline
\ \ \isakeyword{assumes}\ {\isachardoublequoteopen}field\ F{\isachardoublequoteclose}\isanewline
\ \ \isakeyword{assumes}\ {\isachardoublequoteopen}finite\ {\isacharparenleft}{\kern0pt}carrier\ F{\isacharparenright}{\kern0pt}{\isachardoublequoteclose}\isanewline
\ \ \isakeyword{assumes}\ {\isachardoublequoteopen}K\ {\isasymsubseteq}\ carrier\ F{\isachardoublequoteclose}\isanewline
\ \ \isakeyword{assumes}\ {\isachardoublequoteopen}f\ {\isacharbackquote}{\kern0pt}\ K\ {\isasymsubseteq}\ carrier\ F{\isachardoublequoteclose}\isanewline
\ \ \isakeyword{shows}\ {\isachardoublequoteopen}card\ {\isacharbraceleft}{\kern0pt}{\isasymomega}\ {\isasymin}\ bounded{\isacharunderscore}{\kern0pt}degree{\isacharunderscore}{\kern0pt}polynomials\ F\ {\isacharparenleft}{\kern0pt}card\ K\ {\isacharplus}{\kern0pt}\ n{\isacharparenright}{\kern0pt}{\isachardot}{\kern0pt}\ {\isacharparenleft}{\kern0pt}{\isasymforall}k\ {\isasymin}\ K{\isachardot}{\kern0pt}\ ring{\isachardot}{\kern0pt}eval\ F\ {\isasymomega}\ k\ {\isacharequal}{\kern0pt}\ f\ k{\isacharparenright}{\kern0pt}{\isacharbraceright}{\kern0pt}\ {\isacharequal}{\kern0pt}\ \isanewline
\ \ \ \ card\ {\isacharparenleft}{\kern0pt}carrier\ F{\isacharparenright}{\kern0pt}{\isacharcircum}{\kern0pt}n{\isachardoublequoteclose}\ \isanewline
\ \ \ \ {\isacharparenleft}{\kern0pt}\isakeyword{is}\ {\isachardoublequoteopen}card\ {\isacharquery}{\kern0pt}A\ {\isacharequal}{\kern0pt}\ {\isacharquery}{\kern0pt}B{\isachardoublequoteclose}{\isacharparenright}{\kern0pt}\isanewline
%
\isadelimproof
%
\endisadelimproof
%
\isatagproof
\isacommand{proof}\isamarkupfalse%
\ {\isacharminus}{\kern0pt}\isanewline
\ \ \isacommand{define}\isamarkupfalse%
\ z\ \isakeyword{where}\ {\isachardoublequoteopen}z\ {\isacharequal}{\kern0pt}\ restrict\ f\ K{\isachardoublequoteclose}\isanewline
\ \ \isacommand{define}\isamarkupfalse%
\ M\ \isakeyword{where}\ {\isachardoublequoteopen}M\ {\isacharequal}{\kern0pt}\ {\isacharbraceleft}{\kern0pt}f{\isachardot}{\kern0pt}\ range\ f\ {\isasymsubseteq}\ carrier\ F\ {\isasymand}\ {\isacharparenleft}{\kern0pt}{\isasymforall}k\ {\isasymge}\ n{\isachardot}{\kern0pt}\ f\ k\ {\isacharequal}{\kern0pt}\ {\isasymzero}\isactrlbsub F\isactrlesub {\isacharparenright}{\kern0pt}{\isacharbraceright}{\kern0pt}{\isachardoublequoteclose}\isanewline
\isanewline
\ \ \isacommand{have}\isamarkupfalse%
\ a{\isacharcolon}{\kern0pt}{\isachardoublequoteopen}{\isasymzero}\isactrlbsub F\isactrlesub \ {\isasymin}\ carrier\ F{\isachardoublequoteclose}\ \isacommand{using}\isamarkupfalse%
\ assms{\isacharparenleft}{\kern0pt}{\isadigit{1}}{\isacharparenright}{\kern0pt}\isanewline
\ \ \ \ \isacommand{by}\isamarkupfalse%
\ {\isacharparenleft}{\kern0pt}simp\ add{\isacharcolon}{\kern0pt}\ field{\isachardot}{\kern0pt}is{\isacharunderscore}{\kern0pt}ring\ ring{\isachardot}{\kern0pt}ring{\isacharunderscore}{\kern0pt}simprules{\isacharparenleft}{\kern0pt}{\isadigit{2}}{\isacharparenright}{\kern0pt}{\isacharparenright}{\kern0pt}\isanewline
\isanewline
\ \ \isacommand{have}\isamarkupfalse%
\ {\isachardoublequoteopen}finite\ K{\isachardoublequoteclose}\ \isacommand{using}\isamarkupfalse%
\ assms{\isacharparenleft}{\kern0pt}{\isadigit{2}}{\isacharparenright}{\kern0pt}\ assms{\isacharparenleft}{\kern0pt}{\isadigit{3}}{\isacharparenright}{\kern0pt}\ finite{\isacharunderscore}{\kern0pt}subset\ \isacommand{by}\isamarkupfalse%
\ blast\isanewline
\ \ \isacommand{hence}\isamarkupfalse%
\ inj{\isacharunderscore}{\kern0pt}on{\isacharunderscore}{\kern0pt}bounded{\isacharcolon}{\kern0pt}\ {\isachardoublequoteopen}inj{\isacharunderscore}{\kern0pt}on\ {\isacharparenleft}{\kern0pt}split{\isacharunderscore}{\kern0pt}poly\ F\ K{\isacharparenright}{\kern0pt}\ {\isacharparenleft}{\kern0pt}bounded{\isacharunderscore}{\kern0pt}degree{\isacharunderscore}{\kern0pt}polynomials\ F\ {\isacharparenleft}{\kern0pt}card\ K\ {\isacharplus}{\kern0pt}\ n{\isacharparenright}{\kern0pt}{\isacharparenright}{\kern0pt}{\isachardoublequoteclose}\isanewline
\ \ \ \ \isacommand{using}\isamarkupfalse%
\ split{\isacharunderscore}{\kern0pt}poly{\isacharunderscore}{\kern0pt}inj\ assms{\isacharparenleft}{\kern0pt}{\isadigit{1}}{\isacharparenright}{\kern0pt}\ assms{\isacharparenleft}{\kern0pt}{\isadigit{3}}{\isacharparenright}{\kern0pt}\ inj{\isacharunderscore}{\kern0pt}on{\isacharunderscore}{\kern0pt}subset\ bounded{\isacharunderscore}{\kern0pt}degree{\isacharunderscore}{\kern0pt}polynomials{\isacharunderscore}{\kern0pt}length\ \isanewline
\ \ \ \ \isacommand{by}\isamarkupfalse%
\ {\isacharparenleft}{\kern0pt}metis\ {\isacharparenleft}{\kern0pt}mono{\isacharunderscore}{\kern0pt}tags{\isacharparenright}{\kern0pt}\ Collect{\isacharunderscore}{\kern0pt}subset{\isacharparenright}{\kern0pt}\isanewline
\ \ \isacommand{moreover}\isamarkupfalse%
\ \isacommand{have}\isamarkupfalse%
\ {\isachardoublequoteopen}z\ {\isasymin}\ {\isacharparenleft}{\kern0pt}K\ {\isasymrightarrow}\isactrlsub E\ carrier\ F{\isacharparenright}{\kern0pt}{\isachardoublequoteclose}\ \isacommand{apply}\isamarkupfalse%
\ {\isacharparenleft}{\kern0pt}simp\ add{\isacharcolon}{\kern0pt}\ z{\isacharunderscore}{\kern0pt}def{\isacharparenright}{\kern0pt}\isanewline
\ \ \ \ \isacommand{using}\isamarkupfalse%
\ assms\ \isacommand{by}\isamarkupfalse%
\ blast\isanewline
\ \ \isacommand{hence}\isamarkupfalse%
\ {\isachardoublequoteopen}{\isacharbraceleft}{\kern0pt}z{\isacharbraceright}{\kern0pt}\ {\isasymtimes}\ M\ {\isasymsubseteq}\ split{\isacharunderscore}{\kern0pt}poly\ F\ K\ {\isacharbackquote}{\kern0pt}\ {\isacharparenleft}{\kern0pt}bounded{\isacharunderscore}{\kern0pt}degree{\isacharunderscore}{\kern0pt}polynomials\ F\ {\isacharparenleft}{\kern0pt}card\ K\ {\isacharplus}{\kern0pt}\ n{\isacharparenright}{\kern0pt}{\isacharparenright}{\kern0pt}{\isachardoublequoteclose}\isanewline
\ \ \ \ \isacommand{apply}\isamarkupfalse%
\ {\isacharparenleft}{\kern0pt}simp\ add{\isacharcolon}{\kern0pt}\ split{\isacharunderscore}{\kern0pt}poly{\isacharunderscore}{\kern0pt}surj\ assms\ M{\isacharunderscore}{\kern0pt}def\ z{\isacharunderscore}{\kern0pt}def{\isacharparenright}{\kern0pt}\ \isanewline
\ \ \ \ \isacommand{by}\isamarkupfalse%
\ fastforce\isanewline
\ \ \isacommand{ultimately}\isamarkupfalse%
\ \isacommand{have}\isamarkupfalse%
\ {\isachardoublequoteopen}card\ {\isacharparenleft}{\kern0pt}{\isacharparenleft}{\kern0pt}split{\isacharunderscore}{\kern0pt}poly\ F\ K\ {\isacharminus}{\kern0pt}{\isacharbackquote}{\kern0pt}\ {\isacharparenleft}{\kern0pt}{\isacharbraceleft}{\kern0pt}z{\isacharbraceright}{\kern0pt}\ {\isasymtimes}\ M{\isacharparenright}{\kern0pt}{\isacharparenright}{\kern0pt}\ {\isasyminter}\ bounded{\isacharunderscore}{\kern0pt}degree{\isacharunderscore}{\kern0pt}polynomials\ F\ {\isacharparenleft}{\kern0pt}card\ K\ {\isacharplus}{\kern0pt}\ n{\isacharparenright}{\kern0pt}{\isacharparenright}{\kern0pt}\isanewline
\ \ \ \ {\isacharequal}{\kern0pt}\ card\ {\isacharparenleft}{\kern0pt}{\isacharbraceleft}{\kern0pt}z{\isacharbraceright}{\kern0pt}\ {\isasymtimes}\ M{\isacharparenright}{\kern0pt}{\isachardoublequoteclose}\ \isacommand{by}\isamarkupfalse%
\ {\isacharparenleft}{\kern0pt}meson\ card{\isacharunderscore}{\kern0pt}vimage{\isacharunderscore}{\kern0pt}inj{\isacharunderscore}{\kern0pt}on{\isacharparenright}{\kern0pt}\isanewline
\ \ \isacommand{moreover}\isamarkupfalse%
\ \isacommand{have}\isamarkupfalse%
\ {\isachardoublequoteopen}{\isacharparenleft}{\kern0pt}split{\isacharunderscore}{\kern0pt}poly\ F\ K\ {\isacharminus}{\kern0pt}{\isacharbackquote}{\kern0pt}\ {\isacharparenleft}{\kern0pt}{\isacharbraceleft}{\kern0pt}z{\isacharbraceright}{\kern0pt}\ {\isasymtimes}\ M{\isacharparenright}{\kern0pt}{\isacharparenright}{\kern0pt}\ {\isasyminter}\ bounded{\isacharunderscore}{\kern0pt}degree{\isacharunderscore}{\kern0pt}polynomials\ F\ {\isacharparenleft}{\kern0pt}card\ K\ {\isacharplus}{\kern0pt}\ n{\isacharparenright}{\kern0pt}\ {\isasymsubseteq}\ {\isacharquery}{\kern0pt}A{\isachardoublequoteclose}\isanewline
\ \ \ \ \isacommand{apply}\isamarkupfalse%
\ {\isacharparenleft}{\kern0pt}rule\ inv{\isacharunderscore}{\kern0pt}subsetI{\isacharparenright}{\kern0pt}\isanewline
\ \ \ \ \isacommand{apply}\isamarkupfalse%
\ {\isacharparenleft}{\kern0pt}simp\ add{\isacharcolon}{\kern0pt}split{\isacharunderscore}{\kern0pt}poly{\isacharunderscore}{\kern0pt}def\ z{\isacharunderscore}{\kern0pt}def\ restrict{\isacharunderscore}{\kern0pt}def{\isacharparenright}{\kern0pt}\isanewline
\ \ \ \ \isacommand{by}\isamarkupfalse%
\ {\isacharparenleft}{\kern0pt}meson{\isacharparenright}{\kern0pt}\isanewline
\ \ \isacommand{moreover}\isamarkupfalse%
\ \isacommand{have}\isamarkupfalse%
\ {\isachardoublequoteopen}finite\ {\isacharquery}{\kern0pt}A{\isachardoublequoteclose}\ \isacommand{by}\isamarkupfalse%
\ {\isacharparenleft}{\kern0pt}simp\ add{\isacharcolon}{\kern0pt}\ finite{\isacharunderscore}{\kern0pt}poly{\isacharunderscore}{\kern0pt}count\ assms{\isacharparenright}{\kern0pt}\isanewline
\ \ \isacommand{ultimately}\isamarkupfalse%
\ \isacommand{have}\isamarkupfalse%
\ card{\isacharunderscore}{\kern0pt}ineq{\isacharunderscore}{\kern0pt}{\isadigit{1}}{\isacharcolon}{\kern0pt}\ {\isachardoublequoteopen}card\ {\isacharparenleft}{\kern0pt}{\isacharbraceleft}{\kern0pt}z{\isacharbraceright}{\kern0pt}\ {\isasymtimes}\ M{\isacharparenright}{\kern0pt}\ {\isasymle}\ card\ {\isacharquery}{\kern0pt}A{\isachardoublequoteclose}\ \isanewline
\ \ \ \ \isacommand{by}\isamarkupfalse%
\ {\isacharparenleft}{\kern0pt}metis\ {\isacharparenleft}{\kern0pt}mono{\isacharunderscore}{\kern0pt}tags{\isacharcomma}{\kern0pt}\ lifting{\isacharparenright}{\kern0pt}\ card{\isacharunderscore}{\kern0pt}mono{\isacharparenright}{\kern0pt}\isanewline
\isanewline
\ \ \isacommand{have}\isamarkupfalse%
\ {\isachardoublequoteopen}split{\isacharunderscore}{\kern0pt}poly\ F\ K\ {\isacharbackquote}{\kern0pt}\ {\isacharquery}{\kern0pt}A\ {\isasymsubseteq}\ {\isacharbraceleft}{\kern0pt}z{\isacharbraceright}{\kern0pt}\ {\isasymtimes}\ M{\isachardoublequoteclose}\isanewline
\ \ \ \ \isacommand{apply}\isamarkupfalse%
\ {\isacharparenleft}{\kern0pt}rule\ image{\isacharunderscore}{\kern0pt}subsetI{\isacharparenright}{\kern0pt}\isanewline
\ \ \ \ \isacommand{apply}\isamarkupfalse%
\ {\isacharparenleft}{\kern0pt}simp\ add{\isacharcolon}{\kern0pt}split{\isacharunderscore}{\kern0pt}poly{\isacharunderscore}{\kern0pt}def\ z{\isacharunderscore}{\kern0pt}def\ M{\isacharunderscore}{\kern0pt}def{\isacharparenright}{\kern0pt}\isanewline
\ \ \ \ \isacommand{apply}\isamarkupfalse%
\ {\isacharparenleft}{\kern0pt}rule\ conjI{\isacharcomma}{\kern0pt}\ fastforce{\isacharparenright}{\kern0pt}\isanewline
\ \ \ \ \isacommand{apply}\isamarkupfalse%
\ {\isacharparenleft}{\kern0pt}simp\ add{\isacharcolon}{\kern0pt}bounded{\isacharunderscore}{\kern0pt}degree{\isacharunderscore}{\kern0pt}polynomials{\isacharunderscore}{\kern0pt}length{\isacharparenright}{\kern0pt}\isanewline
\ \ \ \ \isacommand{apply}\isamarkupfalse%
\ {\isacharparenleft}{\kern0pt}rule\ conjI{\isacharparenright}{\kern0pt}\ \isanewline
\ \ \ \ \ \isacommand{apply}\isamarkupfalse%
\ {\isacharparenleft}{\kern0pt}meson\ assms{\isacharparenleft}{\kern0pt}{\isadigit{1}}{\isacharparenright}{\kern0pt}\ field{\isachardot}{\kern0pt}is{\isacharunderscore}{\kern0pt}ring\ image{\isacharunderscore}{\kern0pt}subsetI\ ring{\isachardot}{\kern0pt}coeff{\isacharunderscore}{\kern0pt}in{\isacharunderscore}{\kern0pt}carrier\ ring{\isachardot}{\kern0pt}polynomial{\isacharunderscore}{\kern0pt}incl\ \isanewline
\ \ \ \ \ \ \ \ \ \ \ \ univ{\isacharunderscore}{\kern0pt}poly{\isacharunderscore}{\kern0pt}carrier{\isacharparenright}{\kern0pt}\isanewline
\ \ \ \ \isacommand{by}\isamarkupfalse%
\ {\isacharparenleft}{\kern0pt}simp\ add{\isacharcolon}{\kern0pt}\ assms{\isacharparenleft}{\kern0pt}{\isadigit{1}}{\isacharparenright}{\kern0pt}\ field{\isachardot}{\kern0pt}is{\isacharunderscore}{\kern0pt}ring\ ring{\isachardot}{\kern0pt}coeff{\isacharunderscore}{\kern0pt}length{\isacharparenright}{\kern0pt}\isanewline
\ \ \isacommand{moreover}\isamarkupfalse%
\ \isacommand{have}\isamarkupfalse%
\ {\isachardoublequoteopen}inj{\isacharunderscore}{\kern0pt}on\ {\isacharparenleft}{\kern0pt}split{\isacharunderscore}{\kern0pt}poly\ F\ K{\isacharparenright}{\kern0pt}\ {\isacharquery}{\kern0pt}A{\isachardoublequoteclose}\ \isacommand{using}\isamarkupfalse%
\ inj{\isacharunderscore}{\kern0pt}on{\isacharunderscore}{\kern0pt}subset\ inj{\isacharunderscore}{\kern0pt}on{\isacharunderscore}{\kern0pt}bounded\ \isacommand{by}\isamarkupfalse%
\ fastforce\isanewline
\ \ \isacommand{moreover}\isamarkupfalse%
\ \isacommand{have}\isamarkupfalse%
\ {\isachardoublequoteopen}finite\ {\isacharparenleft}{\kern0pt}{\isacharbraceleft}{\kern0pt}z{\isacharbraceright}{\kern0pt}\ {\isasymtimes}\ M{\isacharparenright}{\kern0pt}{\isachardoublequoteclose}\ \isacommand{by}\isamarkupfalse%
\ {\isacharparenleft}{\kern0pt}simp\ add{\isacharcolon}{\kern0pt}M{\isacharunderscore}{\kern0pt}def\ finite{\isacharunderscore}{\kern0pt}mostly{\isacharunderscore}{\kern0pt}constant{\isacharunderscore}{\kern0pt}maps\ assms{\isacharparenleft}{\kern0pt}{\isadigit{2}}{\isacharparenright}{\kern0pt}\ a{\isacharparenright}{\kern0pt}\isanewline
\ \ \isacommand{ultimately}\isamarkupfalse%
\ \isacommand{have}\isamarkupfalse%
\ card{\isacharunderscore}{\kern0pt}ineq{\isacharunderscore}{\kern0pt}{\isadigit{2}}{\isacharcolon}{\kern0pt}{\isachardoublequoteopen}card\ {\isacharquery}{\kern0pt}A\ {\isasymle}\ card\ {\isacharparenleft}{\kern0pt}{\isacharbraceleft}{\kern0pt}z{\isacharbraceright}{\kern0pt}\ {\isasymtimes}\ M{\isacharparenright}{\kern0pt}{\isachardoublequoteclose}\ \isacommand{by}\isamarkupfalse%
\ {\isacharparenleft}{\kern0pt}meson\ card{\isacharunderscore}{\kern0pt}inj{\isacharunderscore}{\kern0pt}on{\isacharunderscore}{\kern0pt}le{\isacharparenright}{\kern0pt}\isanewline
\isanewline
\ \ \isacommand{have}\isamarkupfalse%
\ {\isachardoublequoteopen}card\ {\isacharquery}{\kern0pt}A\ {\isacharequal}{\kern0pt}\ card\ {\isacharparenleft}{\kern0pt}{\isacharbraceleft}{\kern0pt}z{\isacharbraceright}{\kern0pt}\ {\isasymtimes}\ M{\isacharparenright}{\kern0pt}{\isachardoublequoteclose}\ \isacommand{using}\isamarkupfalse%
\ card{\isacharunderscore}{\kern0pt}ineq{\isacharunderscore}{\kern0pt}{\isadigit{1}}\ card{\isacharunderscore}{\kern0pt}ineq{\isacharunderscore}{\kern0pt}{\isadigit{2}}\ \isacommand{by}\isamarkupfalse%
\ auto\isanewline
\ \ \isacommand{moreover}\isamarkupfalse%
\ \isacommand{have}\isamarkupfalse%
\ {\isachardoublequoteopen}card\ {\isacharparenleft}{\kern0pt}{\isacharbraceleft}{\kern0pt}z{\isacharbraceright}{\kern0pt}\ {\isasymtimes}\ M{\isacharparenright}{\kern0pt}\ {\isacharequal}{\kern0pt}\ \ card\ {\isacharparenleft}{\kern0pt}carrier\ F{\isacharparenright}{\kern0pt}{\isacharcircum}{\kern0pt}n{\isachardoublequoteclose}\isanewline
\ \ \ \ \isacommand{by}\isamarkupfalse%
\ {\isacharparenleft}{\kern0pt}simp\ add{\isacharcolon}{\kern0pt}card{\isacharunderscore}{\kern0pt}cartesian{\isacharunderscore}{\kern0pt}product\ M{\isacharunderscore}{\kern0pt}def\ a\ card{\isacharunderscore}{\kern0pt}mostly{\isacharunderscore}{\kern0pt}constant{\isacharunderscore}{\kern0pt}maps\ assms{\isacharparenleft}{\kern0pt}{\isadigit{2}}{\isacharparenright}{\kern0pt}\ {\isacharparenright}{\kern0pt}\isanewline
\ \ \isacommand{ultimately}\isamarkupfalse%
\ \isacommand{show}\isamarkupfalse%
\ {\isacharquery}{\kern0pt}thesis\ \isacommand{by}\isamarkupfalse%
\ presburger\isanewline
\isacommand{qed}\isamarkupfalse%
%
\endisatagproof
{\isafoldproof}%
%
\isadelimproof
\isanewline
%
\endisadelimproof
%
\isadelimtheory
\isanewline
%
\endisadelimtheory
%
\isatagtheory
\isacommand{end}\isamarkupfalse%
%
\endisatagtheory
{\isafoldtheory}%
%
\isadelimtheory
%
\endisadelimtheory
%
\end{isabellebody}%
\endinput
%:%file=PolynomialCounting.tex%:%
%:%11=1%:%
%:%27=3%:%
%:%28=3%:%
%:%29=4%:%
%:%30=5%:%
%:%31=6%:%
%:%40=8%:%
%:%41=9%:%
%:%43=11%:%
%:%44=11%:%
%:%45=12%:%
%:%46=13%:%
%:%47=14%:%
%:%48=14%:%
%:%49=15%:%
%:%52=16%:%
%:%56=16%:%
%:%57=16%:%
%:%58=17%:%
%:%59=17%:%
%:%60=18%:%
%:%61=18%:%
%:%62=19%:%
%:%63=19%:%
%:%64=20%:%
%:%65=20%:%
%:%70=20%:%
%:%73=21%:%
%:%74=22%:%
%:%75=22%:%
%:%76=23%:%
%:%77=24%:%
%:%78=25%:%
%:%85=26%:%
%:%86=26%:%
%:%87=27%:%
%:%88=27%:%
%:%89=28%:%
%:%90=28%:%
%:%91=29%:%
%:%92=29%:%
%:%93=29%:%
%:%94=30%:%
%:%95=30%:%
%:%96=31%:%
%:%97=31%:%
%:%98=31%:%
%:%99=32%:%
%:%100=32%:%
%:%101=32%:%
%:%102=33%:%
%:%108=33%:%
%:%111=34%:%
%:%112=35%:%
%:%113=35%:%
%:%114=36%:%
%:%115=37%:%
%:%116=38%:%
%:%123=39%:%
%:%124=39%:%
%:%125=40%:%
%:%126=40%:%
%:%127=41%:%
%:%128=41%:%
%:%129=42%:%
%:%130=42%:%
%:%131=42%:%
%:%132=43%:%
%:%133=43%:%
%:%134=43%:%
%:%135=44%:%
%:%141=44%:%
%:%144=45%:%
%:%145=46%:%
%:%146=46%:%
%:%147=47%:%
%:%148=48%:%
%:%149=49%:%
%:%150=50%:%
%:%157=51%:%
%:%158=51%:%
%:%159=52%:%
%:%160=52%:%
%:%161=53%:%
%:%162=53%:%
%:%163=54%:%
%:%164=54%:%
%:%165=55%:%
%:%166=55%:%
%:%167=55%:%
%:%168=56%:%
%:%169=56%:%
%:%170=57%:%
%:%171=57%:%
%:%172=58%:%
%:%173=58%:%
%:%174=59%:%
%:%175=59%:%
%:%176=60%:%
%:%177=60%:%
%:%178=61%:%
%:%179=61%:%
%:%180=62%:%
%:%181=62%:%
%:%182=63%:%
%:%183=63%:%
%:%184=64%:%
%:%185=65%:%
%:%186=65%:%
%:%187=65%:%
%:%188=65%:%
%:%189=66%:%
%:%190=66%:%
%:%191=67%:%
%:%192=67%:%
%:%193=68%:%
%:%194=68%:%
%:%195=68%:%
%:%196=69%:%
%:%197=69%:%
%:%198=69%:%
%:%199=70%:%
%:%200=70%:%
%:%201=71%:%
%:%202=71%:%
%:%203=71%:%
%:%204=72%:%
%:%205=72%:%
%:%206=72%:%
%:%207=73%:%
%:%208=73%:%
%:%209=73%:%
%:%210=73%:%
%:%211=74%:%
%:%212=74%:%
%:%213=74%:%
%:%214=75%:%
%:%215=75%:%
%:%216=75%:%
%:%217=76%:%
%:%218=76%:%
%:%219=77%:%
%:%220=77%:%
%:%221=77%:%
%:%222=78%:%
%:%223=78%:%
%:%224=79%:%
%:%225=79%:%
%:%226=79%:%
%:%227=80%:%
%:%228=80%:%
%:%229=80%:%
%:%230=81%:%
%:%236=81%:%
%:%239=82%:%
%:%240=83%:%
%:%241=83%:%
%:%242=84%:%
%:%243=85%:%
%:%244=86%:%
%:%245=87%:%
%:%252=88%:%
%:%253=88%:%
%:%254=89%:%
%:%255=89%:%
%:%256=90%:%
%:%257=90%:%
%:%258=91%:%
%:%259=91%:%
%:%260=92%:%
%:%261=92%:%
%:%262=93%:%
%:%263=93%:%
%:%264=93%:%
%:%265=94%:%
%:%266=94%:%
%:%267=95%:%
%:%268=96%:%
%:%269=96%:%
%:%270=96%:%
%:%271=97%:%
%:%277=97%:%
%:%280=98%:%
%:%281=99%:%
%:%282=99%:%
%:%283=100%:%
%:%284=101%:%
%:%285=102%:%
%:%292=103%:%
%:%293=103%:%
%:%294=104%:%
%:%295=104%:%
%:%296=104%:%
%:%297=104%:%
%:%298=105%:%
%:%299=105%:%
%:%300=106%:%
%:%301=106%:%
%:%302=106%:%
%:%303=107%:%
%:%304=107%:%
%:%305=108%:%
%:%306=108%:%
%:%307=109%:%
%:%308=109%:%
%:%309=110%:%
%:%310=110%:%
%:%311=111%:%
%:%312=111%:%
%:%313=112%:%
%:%314=112%:%
%:%315=113%:%
%:%316=113%:%
%:%317=113%:%
%:%318=114%:%
%:%319=114%:%
%:%320=115%:%
%:%321=115%:%
%:%322=116%:%
%:%323=116%:%
%:%324=116%:%
%:%325=117%:%
%:%326=117%:%
%:%327=117%:%
%:%328=118%:%
%:%329=118%:%
%:%330=118%:%
%:%331=118%:%
%:%332=119%:%
%:%333=119%:%
%:%334=119%:%
%:%335=119%:%
%:%336=120%:%
%:%342=120%:%
%:%345=121%:%
%:%346=122%:%
%:%347=122%:%
%:%348=123%:%
%:%349=124%:%
%:%356=125%:%
%:%357=125%:%
%:%358=126%:%
%:%359=126%:%
%:%360=127%:%
%:%361=127%:%
%:%362=128%:%
%:%363=128%:%
%:%364=129%:%
%:%365=129%:%
%:%366=129%:%
%:%367=130%:%
%:%382=132%:%
%:%394=134%:%
%:%395=135%:%
%:%396=136%:%
%:%397=137%:%
%:%398=138%:%
%:%399=139%:%
%:%400=140%:%
%:%401=141%:%
%:%402=142%:%
%:%403=143%:%
%:%404=144%:%
%:%405=145%:%
%:%406=146%:%
%:%407=147%:%
%:%408=148%:%
%:%409=149%:%
%:%410=150%:%
%:%411=151%:%
%:%412=152%:%
%:%413=153%:%
%:%415=155%:%
%:%416=155%:%
%:%419=158%:%
%:%420=159%:%
%:%421=160%:%
%:%422=161%:%
%:%423=162%:%
%:%424=163%:%
%:%426=165%:%
%:%427=165%:%
%:%428=166%:%
%:%429=167%:%
%:%430=168%:%
%:%431=169%:%
%:%434=170%:%
%:%438=170%:%
%:%439=170%:%
%:%440=171%:%
%:%441=171%:%
%:%442=172%:%
%:%443=172%:%
%:%444=173%:%
%:%445=173%:%
%:%446=174%:%
%:%447=175%:%
%:%448=176%:%
%:%449=176%:%
%:%450=177%:%
%:%451=177%:%
%:%452=178%:%
%:%453=179%:%
%:%454=180%:%
%:%455=180%:%
%:%460=180%:%
%:%463=181%:%
%:%464=182%:%
%:%465=182%:%
%:%466=183%:%
%:%467=184%:%
%:%468=185%:%
%:%475=186%:%
%:%476=186%:%
%:%477=187%:%
%:%478=187%:%
%:%479=188%:%
%:%480=188%:%
%:%481=188%:%
%:%482=189%:%
%:%483=189%:%
%:%484=189%:%
%:%485=190%:%
%:%486=190%:%
%:%487=191%:%
%:%488=191%:%
%:%489=192%:%
%:%490=193%:%
%:%491=193%:%
%:%492=194%:%
%:%493=194%:%
%:%494=195%:%
%:%495=196%:%
%:%496=197%:%
%:%497=198%:%
%:%503=198%:%
%:%506=199%:%
%:%507=200%:%
%:%508=200%:%
%:%509=201%:%
%:%510=202%:%
%:%511=203%:%
%:%512=204%:%
%:%515=205%:%
%:%519=205%:%
%:%520=205%:%
%:%521=206%:%
%:%522=206%:%
%:%523=207%:%
%:%524=207%:%
%:%525=208%:%
%:%530=208%:%
%:%533=209%:%
%:%534=210%:%
%:%535=210%:%
%:%536=211%:%
%:%537=212%:%
%:%538=213%:%
%:%539=214%:%
%:%540=215%:%
%:%547=216%:%
%:%548=216%:%
%:%549=217%:%
%:%550=217%:%
%:%551=218%:%
%:%552=218%:%
%:%553=218%:%
%:%554=219%:%
%:%555=219%:%
%:%556=220%:%
%:%557=220%:%
%:%558=221%:%
%:%559=221%:%
%:%560=221%:%
%:%561=221%:%
%:%562=222%:%
%:%568=222%:%
%:%571=223%:%
%:%572=224%:%
%:%573=224%:%
%:%574=225%:%
%:%575=226%:%
%:%576=227%:%
%:%577=228%:%
%:%584=229%:%
%:%585=229%:%
%:%586=230%:%
%:%587=230%:%
%:%588=231%:%
%:%589=231%:%
%:%590=232%:%
%:%591=232%:%
%:%592=233%:%
%:%593=233%:%
%:%594=234%:%
%:%595=234%:%
%:%596=235%:%
%:%597=235%:%
%:%598=235%:%
%:%599=235%:%
%:%600=236%:%
%:%601=236%:%
%:%602=236%:%
%:%603=236%:%
%:%604=236%:%
%:%605=237%:%
%:%611=237%:%
%:%614=238%:%
%:%615=239%:%
%:%616=239%:%
%:%617=240%:%
%:%618=241%:%
%:%619=242%:%
%:%620=243%:%
%:%621=244%:%
%:%622=245%:%
%:%623=246%:%
%:%630=247%:%
%:%631=247%:%
%:%632=248%:%
%:%633=248%:%
%:%634=249%:%
%:%635=249%:%
%:%636=249%:%
%:%637=250%:%
%:%638=250%:%
%:%639=251%:%
%:%640=251%:%
%:%641=252%:%
%:%642=252%:%
%:%643=253%:%
%:%644=253%:%
%:%645=254%:%
%:%646=254%:%
%:%647=255%:%
%:%648=255%:%
%:%649=256%:%
%:%650=256%:%
%:%651=256%:%
%:%652=257%:%
%:%653=257%:%
%:%654=258%:%
%:%655=258%:%
%:%656=259%:%
%:%657=259%:%
%:%658=259%:%
%:%659=260%:%
%:%660=260%:%
%:%661=260%:%
%:%662=261%:%
%:%663=261%:%
%:%664=261%:%
%:%665=261%:%
%:%666=261%:%
%:%667=262%:%
%:%668=262%:%
%:%669=262%:%
%:%670=262%:%
%:%671=263%:%
%:%672=263%:%
%:%673=264%:%
%:%674=264%:%
%:%675=265%:%
%:%676=265%:%
%:%677=265%:%
%:%678=266%:%
%:%679=266%:%
%:%680=266%:%
%:%681=266%:%
%:%682=267%:%
%:%683=267%:%
%:%684=268%:%
%:%685=268%:%
%:%686=269%:%
%:%687=269%:%
%:%688=270%:%
%:%689=270%:%
%:%690=271%:%
%:%691=271%:%
%:%692=272%:%
%:%693=272%:%
%:%694=273%:%
%:%695=273%:%
%:%696=273%:%
%:%697=274%:%
%:%698=274%:%
%:%699=274%:%
%:%700=275%:%
%:%701=275%:%
%:%702=275%:%
%:%703=275%:%
%:%704=276%:%
%:%705=276%:%
%:%706=277%:%
%:%712=277%:%
%:%715=278%:%
%:%716=279%:%
%:%717=279%:%
%:%718=280%:%
%:%719=281%:%
%:%720=282%:%
%:%721=283%:%
%:%728=284%:%
%:%729=284%:%
%:%730=285%:%
%:%731=285%:%
%:%732=285%:%
%:%733=285%:%
%:%734=286%:%
%:%735=286%:%
%:%736=286%:%
%:%737=286%:%
%:%738=287%:%
%:%739=287%:%
%:%740=288%:%
%:%741=288%:%
%:%742=289%:%
%:%743=289%:%
%:%744=290%:%
%:%745=290%:%
%:%746=291%:%
%:%747=291%:%
%:%748=292%:%
%:%749=293%:%
%:%750=293%:%
%:%751=293%:%
%:%752=294%:%
%:%753=294%:%
%:%754=295%:%
%:%755=296%:%
%:%756=296%:%
%:%757=297%:%
%:%758=297%:%
%:%759=297%:%
%:%760=297%:%
%:%761=298%:%
%:%762=298%:%
%:%763=299%:%
%:%764=299%:%
%:%765=300%:%
%:%766=300%:%
%:%767=301%:%
%:%768=302%:%
%:%769=302%:%
%:%770=303%:%
%:%771=303%:%
%:%772=304%:%
%:%773=304%:%
%:%774=304%:%
%:%775=305%:%
%:%776=305%:%
%:%777=305%:%
%:%778=305%:%
%:%779=306%:%
%:%780=306%:%
%:%781=307%:%
%:%782=307%:%
%:%783=308%:%
%:%784=309%:%
%:%785=309%:%
%:%786=310%:%
%:%787=310%:%
%:%788=310%:%
%:%789=311%:%
%:%790=311%:%
%:%791=311%:%
%:%792=312%:%
%:%793=312%:%
%:%794=312%:%
%:%795=313%:%
%:%796=313%:%
%:%797=313%:%
%:%798=314%:%
%:%799=314%:%
%:%800=315%:%
%:%801=316%:%
%:%807=316%:%
%:%810=317%:%
%:%811=318%:%
%:%812=318%:%
%:%813=319%:%
%:%814=320%:%
%:%815=321%:%
%:%816=322%:%
%:%823=323%:%
%:%824=323%:%
%:%825=324%:%
%:%826=324%:%
%:%827=324%:%
%:%828=324%:%
%:%829=325%:%
%:%830=325%:%
%:%831=325%:%
%:%832=325%:%
%:%833=326%:%
%:%834=326%:%
%:%835=326%:%
%:%836=326%:%
%:%837=327%:%
%:%843=327%:%
%:%846=328%:%
%:%847=329%:%
%:%848=329%:%
%:%849=330%:%
%:%850=331%:%
%:%851=332%:%
%:%852=333%:%
%:%853=334%:%
%:%854=335%:%
%:%855=336%:%
%:%862=337%:%
%:%863=337%:%
%:%864=338%:%
%:%865=338%:%
%:%866=339%:%
%:%867=339%:%
%:%868=340%:%
%:%869=340%:%
%:%870=341%:%
%:%871=342%:%
%:%872=342%:%
%:%873=343%:%
%:%874=343%:%
%:%875=344%:%
%:%876=344%:%
%:%877=345%:%
%:%878=346%:%
%:%879=346%:%
%:%880=347%:%
%:%881=347%:%
%:%882=347%:%
%:%883=348%:%
%:%884=348%:%
%:%885=349%:%
%:%886=349%:%
%:%887=350%:%
%:%888=351%:%
%:%889=351%:%
%:%890=352%:%
%:%891=352%:%
%:%892=352%:%
%:%893=353%:%
%:%894=354%:%
%:%895=354%:%
%:%896=355%:%
%:%897=355%:%
%:%898=355%:%
%:%899=356%:%
%:%900=356%:%
%:%901=356%:%
%:%902=357%:%
%:%903=357%:%
%:%904=358%:%
%:%905=358%:%
%:%906=359%:%
%:%907=359%:%
%:%908=359%:%
%:%909=360%:%
%:%910=360%:%
%:%911=361%:%
%:%912=361%:%
%:%913=361%:%
%:%914=362%:%
%:%915=362%:%
%:%916=363%:%
%:%917=363%:%
%:%918=363%:%
%:%919=363%:%
%:%920=364%:%
%:%921=365%:%
%:%922=365%:%
%:%923=365%:%
%:%924=365%:%
%:%925=366%:%
%:%926=366%:%
%:%927=366%:%
%:%928=366%:%
%:%929=366%:%
%:%930=367%:%
%:%931=367%:%
%:%932=367%:%
%:%933=367%:%
%:%934=368%:%
%:%940=368%:%
%:%943=369%:%
%:%944=370%:%
%:%945=370%:%
%:%946=371%:%
%:%947=372%:%
%:%948=373%:%
%:%949=374%:%
%:%950=375%:%
%:%951=376%:%
%:%958=377%:%
%:%959=377%:%
%:%960=378%:%
%:%961=378%:%
%:%962=379%:%
%:%963=379%:%
%:%964=379%:%
%:%965=380%:%
%:%966=380%:%
%:%967=381%:%
%:%968=381%:%
%:%969=381%:%
%:%970=381%:%
%:%971=382%:%
%:%972=382%:%
%:%973=382%:%
%:%974=383%:%
%:%975=383%:%
%:%976=384%:%
%:%977=384%:%
%:%978=384%:%
%:%979=385%:%
%:%980=385%:%
%:%981=386%:%
%:%982=386%:%
%:%983=386%:%
%:%984=386%:%
%:%985=386%:%
%:%986=387%:%
%:%987=387%:%
%:%988=387%:%
%:%989=387%:%
%:%990=388%:%
%:%991=388%:%
%:%992=389%:%
%:%993=389%:%
%:%994=389%:%
%:%995=390%:%
%:%996=390%:%
%:%997=390%:%
%:%998=390%:%
%:%999=390%:%
%:%1000=391%:%
%:%1001=391%:%
%:%1002=391%:%
%:%1003=392%:%
%:%1004=392%:%
%:%1005=393%:%
%:%1006=393%:%
%:%1007=393%:%
%:%1008=393%:%
%:%1009=394%:%
%:%1010=394%:%
%:%1011=395%:%
%:%1012=395%:%
%:%1013=395%:%
%:%1014=395%:%
%:%1015=396%:%
%:%1016=396%:%
%:%1017=396%:%
%:%1018=396%:%
%:%1019=397%:%
%:%1025=397%:%
%:%1028=398%:%
%:%1029=399%:%
%:%1030=399%:%
%:%1031=400%:%
%:%1032=401%:%
%:%1035=402%:%
%:%1039=402%:%
%:%1040=402%:%
%:%1041=402%:%
%:%1046=402%:%
%:%1049=403%:%
%:%1050=404%:%
%:%1051=404%:%
%:%1052=405%:%
%:%1053=406%:%
%:%1054=407%:%
%:%1055=408%:%
%:%1056=409%:%
%:%1057=410%:%
%:%1058=411%:%
%:%1065=412%:%
%:%1066=412%:%
%:%1067=413%:%
%:%1068=413%:%
%:%1069=414%:%
%:%1070=414%:%
%:%1071=415%:%
%:%1072=416%:%
%:%1073=416%:%
%:%1074=416%:%
%:%1075=417%:%
%:%1076=417%:%
%:%1077=418%:%
%:%1078=419%:%
%:%1079=419%:%
%:%1080=419%:%
%:%1081=419%:%
%:%1082=420%:%
%:%1083=420%:%
%:%1084=421%:%
%:%1085=421%:%
%:%1086=422%:%
%:%1087=422%:%
%:%1088=423%:%
%:%1089=423%:%
%:%1090=423%:%
%:%1091=423%:%
%:%1092=424%:%
%:%1093=424%:%
%:%1094=424%:%
%:%1095=425%:%
%:%1096=425%:%
%:%1097=426%:%
%:%1098=426%:%
%:%1099=427%:%
%:%1100=427%:%
%:%1101=428%:%
%:%1102=428%:%
%:%1103=428%:%
%:%1104=429%:%
%:%1105=429%:%
%:%1106=430%:%
%:%1107=430%:%
%:%1108=430%:%
%:%1109=431%:%
%:%1110=431%:%
%:%1111=432%:%
%:%1112=432%:%
%:%1113=433%:%
%:%1114=433%:%
%:%1115=434%:%
%:%1116=434%:%
%:%1117=434%:%
%:%1118=434%:%
%:%1119=435%:%
%:%1120=435%:%
%:%1121=435%:%
%:%1122=436%:%
%:%1123=436%:%
%:%1124=437%:%
%:%1125=438%:%
%:%1126=438%:%
%:%1127=439%:%
%:%1128=439%:%
%:%1129=440%:%
%:%1130=440%:%
%:%1131=441%:%
%:%1132=441%:%
%:%1133=442%:%
%:%1134=442%:%
%:%1135=443%:%
%:%1136=443%:%
%:%1137=444%:%
%:%1138=444%:%
%:%1139=445%:%
%:%1140=446%:%
%:%1141=446%:%
%:%1142=447%:%
%:%1143=447%:%
%:%1144=447%:%
%:%1145=447%:%
%:%1146=447%:%
%:%1147=448%:%
%:%1148=448%:%
%:%1149=448%:%
%:%1150=448%:%
%:%1151=449%:%
%:%1152=449%:%
%:%1153=449%:%
%:%1154=449%:%
%:%1155=450%:%
%:%1156=451%:%
%:%1157=451%:%
%:%1158=451%:%
%:%1159=451%:%
%:%1160=452%:%
%:%1161=452%:%
%:%1162=452%:%
%:%1163=453%:%
%:%1164=453%:%
%:%1165=454%:%
%:%1166=454%:%
%:%1167=454%:%
%:%1168=454%:%
%:%1169=455%:%
%:%1175=455%:%
%:%1180=456%:%
%:%1185=457%:%

%
\begin{isabellebody}%
\setisabellecontext{Product{\isacharunderscore}{\kern0pt}PMF{\isacharunderscore}{\kern0pt}Ext}%
%
\isadelimdocument
%
\endisadelimdocument
%
\isatagdocument
%
\isamarkupsection{Indexed Products of Probability Mass Functions%
}
\isamarkuptrue%
%
\endisatagdocument
{\isafolddocument}%
%
\isadelimdocument
%
\endisadelimdocument
%
\begin{isamarkuptext}%
This section introduces a restricted version of \isa{Pi{\isacharunderscore}{\kern0pt}pmf} where the default value is undefined
and contains some additional results about that case in addition to \verb|HOL-Probability.Product_PMF|%
\end{isamarkuptext}\isamarkuptrue%
%
\isadelimtheory
%
\endisadelimtheory
%
\isatagtheory
\isacommand{theory}\isamarkupfalse%
\ Product{\isacharunderscore}{\kern0pt}PMF{\isacharunderscore}{\kern0pt}Ext\isanewline
\ \ \isakeyword{imports}\ Main\ Probability{\isacharunderscore}{\kern0pt}Ext\ {\isachardoublequoteopen}HOL{\isacharminus}{\kern0pt}Probability{\isachardot}{\kern0pt}Product{\isacharunderscore}{\kern0pt}PMF{\isachardoublequoteclose}\isanewline
\isakeyword{begin}%
\endisatagtheory
{\isafoldtheory}%
%
\isadelimtheory
%
\endisadelimtheory
\isanewline
\isanewline
\isacommand{definition}\isamarkupfalse%
\ prod{\isacharunderscore}{\kern0pt}pmf\ \isakeyword{where}\ {\isachardoublequoteopen}prod{\isacharunderscore}{\kern0pt}pmf\ I\ M\ {\isacharequal}{\kern0pt}\ Pi{\isacharunderscore}{\kern0pt}pmf\ I\ undefined\ M{\isachardoublequoteclose}\isanewline
\isanewline
\isacommand{lemma}\isamarkupfalse%
\ pmf{\isacharunderscore}{\kern0pt}prod{\isacharunderscore}{\kern0pt}pmf{\isacharcolon}{\kern0pt}\ \isanewline
\ \ \isakeyword{assumes}\ {\isachardoublequoteopen}finite\ I{\isachardoublequoteclose}\isanewline
\ \ \isakeyword{shows}\ {\isachardoublequoteopen}pmf\ {\isacharparenleft}{\kern0pt}prod{\isacharunderscore}{\kern0pt}pmf\ I\ M{\isacharparenright}{\kern0pt}\ x\ {\isacharequal}{\kern0pt}\ {\isacharparenleft}{\kern0pt}if\ x\ {\isasymin}\ extensional\ I\ then\ {\isasymProd}i\ {\isasymin}\ I{\isachardot}{\kern0pt}\ {\isacharparenleft}{\kern0pt}pmf\ {\isacharparenleft}{\kern0pt}M\ i{\isacharparenright}{\kern0pt}{\isacharparenright}{\kern0pt}\ {\isacharparenleft}{\kern0pt}x\ i{\isacharparenright}{\kern0pt}\ else\ {\isadigit{0}}{\isacharparenright}{\kern0pt}{\isachardoublequoteclose}\isanewline
%
\isadelimproof
\ \ %
\endisadelimproof
%
\isatagproof
\isacommand{by}\isamarkupfalse%
\ {\isacharparenleft}{\kern0pt}simp\ add{\isacharcolon}{\kern0pt}prod{\isacharunderscore}{\kern0pt}pmf{\isacharunderscore}{\kern0pt}def\ \ pmf{\isacharunderscore}{\kern0pt}Pi{\isacharbrackleft}{\kern0pt}OF\ assms{\isacharparenleft}{\kern0pt}{\isadigit{1}}{\isacharparenright}{\kern0pt}{\isacharbrackright}{\kern0pt}\ extensional{\isacharunderscore}{\kern0pt}def{\isacharparenright}{\kern0pt}%
\endisatagproof
{\isafoldproof}%
%
\isadelimproof
\isanewline
%
\endisadelimproof
\isanewline
\isacommand{lemma}\isamarkupfalse%
\ set{\isacharunderscore}{\kern0pt}prod{\isacharunderscore}{\kern0pt}pmf{\isacharcolon}{\kern0pt}\isanewline
\ \ \isakeyword{assumes}\ {\isachardoublequoteopen}finite\ I{\isachardoublequoteclose}\isanewline
\ \ \isakeyword{shows}\ {\isachardoublequoteopen}set{\isacharunderscore}{\kern0pt}pmf\ {\isacharparenleft}{\kern0pt}prod{\isacharunderscore}{\kern0pt}pmf\ I\ M{\isacharparenright}{\kern0pt}\ {\isacharequal}{\kern0pt}\ PiE\ I\ {\isacharparenleft}{\kern0pt}set{\isacharunderscore}{\kern0pt}pmf\ {\isasymcirc}\ M{\isacharparenright}{\kern0pt}{\isachardoublequoteclose}\isanewline
%
\isadelimproof
\ \ %
\endisadelimproof
%
\isatagproof
\isacommand{apply}\isamarkupfalse%
\ {\isacharparenleft}{\kern0pt}simp\ add{\isacharcolon}{\kern0pt}set{\isacharunderscore}{\kern0pt}pmf{\isacharunderscore}{\kern0pt}eq\ pmf{\isacharunderscore}{\kern0pt}prod{\isacharunderscore}{\kern0pt}pmf{\isacharbrackleft}{\kern0pt}OF\ assms{\isacharparenleft}{\kern0pt}{\isadigit{1}}{\isacharparenright}{\kern0pt}{\isacharbrackright}{\kern0pt}\ prod{\isacharunderscore}{\kern0pt}zero{\isacharunderscore}{\kern0pt}iff{\isacharbrackleft}{\kern0pt}OF\ assms{\isacharparenleft}{\kern0pt}{\isadigit{1}}{\isacharparenright}{\kern0pt}{\isacharbrackright}{\kern0pt}{\isacharparenright}{\kern0pt}\isanewline
\ \ \isacommand{apply}\isamarkupfalse%
\ {\isacharparenleft}{\kern0pt}simp\ add{\isacharcolon}{\kern0pt}set{\isacharunderscore}{\kern0pt}pmf{\isacharunderscore}{\kern0pt}iff{\isacharbrackleft}{\kern0pt}symmetric{\isacharbrackright}{\kern0pt}\ PiE{\isacharunderscore}{\kern0pt}def\ Pi{\isacharunderscore}{\kern0pt}def{\isacharparenright}{\kern0pt}\isanewline
\ \ \isacommand{by}\isamarkupfalse%
\ blast%
\endisatagproof
{\isafoldproof}%
%
\isadelimproof
\isanewline
%
\endisadelimproof
\isanewline
\isacommand{lemma}\isamarkupfalse%
\ set{\isacharunderscore}{\kern0pt}pmf{\isacharunderscore}{\kern0pt}iff{\isacharprime}{\kern0pt}{\isacharcolon}{\kern0pt}\ {\isachardoublequoteopen}x\ {\isasymnotin}\ set{\isacharunderscore}{\kern0pt}pmf\ M\ {\isasymlongleftrightarrow}\ pmf\ M\ x\ {\isacharequal}{\kern0pt}\ {\isadigit{0}}{\isachardoublequoteclose}\isanewline
%
\isadelimproof
\ \ %
\endisadelimproof
%
\isatagproof
\isacommand{using}\isamarkupfalse%
\ set{\isacharunderscore}{\kern0pt}pmf{\isacharunderscore}{\kern0pt}iff\ \isacommand{by}\isamarkupfalse%
\ metis%
\endisatagproof
{\isafoldproof}%
%
\isadelimproof
\isanewline
%
\endisadelimproof
\isanewline
\isacommand{lemma}\isamarkupfalse%
\ prob{\isacharunderscore}{\kern0pt}prod{\isacharunderscore}{\kern0pt}pmf{\isacharcolon}{\kern0pt}\isanewline
\ \ \isakeyword{assumes}\ {\isachardoublequoteopen}finite\ I{\isachardoublequoteclose}\isanewline
\ \ \isakeyword{shows}\ {\isachardoublequoteopen}measure\ {\isacharparenleft}{\kern0pt}measure{\isacharunderscore}{\kern0pt}pmf\ {\isacharparenleft}{\kern0pt}prod{\isacharunderscore}{\kern0pt}pmf\ I\ M{\isacharparenright}{\kern0pt}{\isacharparenright}{\kern0pt}\ {\isacharparenleft}{\kern0pt}Pi\ I\ A{\isacharparenright}{\kern0pt}\ {\isacharequal}{\kern0pt}\ {\isacharparenleft}{\kern0pt}{\isasymProd}\ i\ {\isasymin}\ I{\isachardot}{\kern0pt}\ measure\ {\isacharparenleft}{\kern0pt}M\ i{\isacharparenright}{\kern0pt}\ {\isacharparenleft}{\kern0pt}A\ i{\isacharparenright}{\kern0pt}{\isacharparenright}{\kern0pt}{\isachardoublequoteclose}\isanewline
%
\isadelimproof
\ \ %
\endisadelimproof
%
\isatagproof
\isacommand{apply}\isamarkupfalse%
\ {\isacharparenleft}{\kern0pt}simp\ add{\isacharcolon}{\kern0pt}prod{\isacharunderscore}{\kern0pt}pmf{\isacharunderscore}{\kern0pt}def{\isacharparenright}{\kern0pt}\isanewline
\ \ \isacommand{by}\isamarkupfalse%
\ {\isacharparenleft}{\kern0pt}subst\ measure{\isacharunderscore}{\kern0pt}Pi{\isacharunderscore}{\kern0pt}pmf{\isacharunderscore}{\kern0pt}Pi{\isacharbrackleft}{\kern0pt}OF\ assms{\isacharparenleft}{\kern0pt}{\isadigit{1}}{\isacharparenright}{\kern0pt}{\isacharbrackright}{\kern0pt}{\isacharcomma}{\kern0pt}\ simp{\isacharparenright}{\kern0pt}%
\endisatagproof
{\isafoldproof}%
%
\isadelimproof
\isanewline
%
\endisadelimproof
\isanewline
\isacommand{lemma}\isamarkupfalse%
\ prob{\isacharunderscore}{\kern0pt}prod{\isacharunderscore}{\kern0pt}pmf{\isacharprime}{\kern0pt}{\isacharcolon}{\kern0pt}\ \isanewline
\ \ \isakeyword{assumes}\ {\isachardoublequoteopen}finite\ I{\isachardoublequoteclose}\isanewline
\ \ \isakeyword{assumes}\ {\isachardoublequoteopen}J\ {\isasymsubseteq}\ I{\isachardoublequoteclose}\isanewline
\ \ \isakeyword{shows}\ {\isachardoublequoteopen}measure\ {\isacharparenleft}{\kern0pt}measure{\isacharunderscore}{\kern0pt}pmf\ {\isacharparenleft}{\kern0pt}prod{\isacharunderscore}{\kern0pt}pmf\ I\ M{\isacharparenright}{\kern0pt}{\isacharparenright}{\kern0pt}\ {\isacharparenleft}{\kern0pt}Pi\ J\ A{\isacharparenright}{\kern0pt}\ {\isacharequal}{\kern0pt}\ {\isacharparenleft}{\kern0pt}{\isasymProd}\ i\ {\isasymin}\ J{\isachardot}{\kern0pt}\ measure\ {\isacharparenleft}{\kern0pt}M\ i{\isacharparenright}{\kern0pt}\ {\isacharparenleft}{\kern0pt}A\ i{\isacharparenright}{\kern0pt}{\isacharparenright}{\kern0pt}{\isachardoublequoteclose}\isanewline
%
\isadelimproof
%
\endisadelimproof
%
\isatagproof
\isacommand{proof}\isamarkupfalse%
\ {\isacharminus}{\kern0pt}\isanewline
\ \ \isacommand{have}\isamarkupfalse%
\ a{\isacharcolon}{\kern0pt}{\isachardoublequoteopen}Pi\ J\ A\ {\isacharequal}{\kern0pt}\ Pi\ I\ {\isacharparenleft}{\kern0pt}{\isasymlambda}i{\isachardot}{\kern0pt}\ if\ i\ {\isasymin}\ J\ then\ A\ i\ else\ UNIV{\isacharparenright}{\kern0pt}{\isachardoublequoteclose}\isanewline
\ \ \ \ \isacommand{apply}\isamarkupfalse%
\ {\isacharparenleft}{\kern0pt}simp\ add{\isacharcolon}{\kern0pt}Pi{\isacharunderscore}{\kern0pt}def{\isacharparenright}{\kern0pt}\isanewline
\ \ \ \ \isacommand{apply}\isamarkupfalse%
\ {\isacharparenleft}{\kern0pt}rule\ Collect{\isacharunderscore}{\kern0pt}cong{\isacharparenright}{\kern0pt}\isanewline
\ \ \ \ \isacommand{using}\isamarkupfalse%
\ assms{\isacharparenleft}{\kern0pt}{\isadigit{2}}{\isacharparenright}{\kern0pt}\ \isacommand{by}\isamarkupfalse%
\ blast\isanewline
\ \ \isacommand{show}\isamarkupfalse%
\ {\isacharquery}{\kern0pt}thesis\isanewline
\ \ \ \ \isacommand{apply}\isamarkupfalse%
\ {\isacharparenleft}{\kern0pt}simp\ add{\isacharcolon}{\kern0pt}if{\isacharunderscore}{\kern0pt}distrib\ \ a\ prob{\isacharunderscore}{\kern0pt}prod{\isacharunderscore}{\kern0pt}pmf{\isacharbrackleft}{\kern0pt}OF\ assms{\isacharparenleft}{\kern0pt}{\isadigit{1}}{\isacharparenright}{\kern0pt}{\isacharbrackright}{\kern0pt}\ prod{\isachardot}{\kern0pt}If{\isacharunderscore}{\kern0pt}cases{\isacharbrackleft}{\kern0pt}OF\ assms{\isacharparenleft}{\kern0pt}{\isadigit{1}}{\isacharparenright}{\kern0pt}{\isacharbrackright}{\kern0pt}{\isacharparenright}{\kern0pt}\isanewline
\ \ \ \ \isacommand{apply}\isamarkupfalse%
\ {\isacharparenleft}{\kern0pt}rule\ arg{\isacharunderscore}{\kern0pt}cong{\isadigit{2}}{\isacharbrackleft}{\kern0pt}\isakeyword{where}\ f{\isacharequal}{\kern0pt}{\isachardoublequoteopen}prod{\isachardoublequoteclose}{\isacharbrackright}{\kern0pt}{\isacharcomma}{\kern0pt}\ simp{\isacharparenright}{\kern0pt}\isanewline
\ \ \ \ \isacommand{using}\isamarkupfalse%
\ assms{\isacharparenleft}{\kern0pt}{\isadigit{2}}{\isacharparenright}{\kern0pt}\ \isacommand{by}\isamarkupfalse%
\ blast\isanewline
\isacommand{qed}\isamarkupfalse%
%
\endisatagproof
{\isafoldproof}%
%
\isadelimproof
\isanewline
%
\endisadelimproof
\isanewline
\isacommand{lemma}\isamarkupfalse%
\ prob{\isacharunderscore}{\kern0pt}prod{\isacharunderscore}{\kern0pt}pmf{\isacharunderscore}{\kern0pt}slice{\isacharcolon}{\kern0pt}\isanewline
\ \ \isakeyword{assumes}\ {\isachardoublequoteopen}finite\ I{\isachardoublequoteclose}\isanewline
\ \ \isakeyword{assumes}\ {\isachardoublequoteopen}i\ {\isasymin}\ I{\isachardoublequoteclose}\isanewline
\ \ \isakeyword{shows}\ {\isachardoublequoteopen}measure\ {\isacharparenleft}{\kern0pt}measure{\isacharunderscore}{\kern0pt}pmf\ {\isacharparenleft}{\kern0pt}prod{\isacharunderscore}{\kern0pt}pmf\ I\ M{\isacharparenright}{\kern0pt}{\isacharparenright}{\kern0pt}\ {\isacharbraceleft}{\kern0pt}{\isasymomega}{\isachardot}{\kern0pt}\ P\ {\isacharparenleft}{\kern0pt}{\isasymomega}\ i{\isacharparenright}{\kern0pt}{\isacharbraceright}{\kern0pt}\ {\isacharequal}{\kern0pt}\ measure\ {\isacharparenleft}{\kern0pt}M\ i{\isacharparenright}{\kern0pt}\ {\isacharbraceleft}{\kern0pt}{\isasymomega}{\isachardot}{\kern0pt}\ P\ {\isasymomega}{\isacharbraceright}{\kern0pt}{\isachardoublequoteclose}\isanewline
%
\isadelimproof
\ \ %
\endisadelimproof
%
\isatagproof
\isacommand{using}\isamarkupfalse%
\ prob{\isacharunderscore}{\kern0pt}prod{\isacharunderscore}{\kern0pt}pmf{\isacharprime}{\kern0pt}{\isacharbrackleft}{\kern0pt}OF\ assms{\isacharparenleft}{\kern0pt}{\isadigit{1}}{\isacharparenright}{\kern0pt}{\isacharcomma}{\kern0pt}\ \isakeyword{where}\ J{\isacharequal}{\kern0pt}{\isachardoublequoteopen}{\isacharbraceleft}{\kern0pt}i{\isacharbraceright}{\kern0pt}{\isachardoublequoteclose}\ \isakeyword{and}\ M{\isacharequal}{\kern0pt}{\isachardoublequoteopen}M{\isachardoublequoteclose}\ \isakeyword{and}\ A{\isacharequal}{\kern0pt}{\isachardoublequoteopen}{\isasymlambda}{\isacharunderscore}{\kern0pt}{\isachardot}{\kern0pt}\ Collect\ P{\isachardoublequoteclose}{\isacharbrackright}{\kern0pt}\isanewline
\ \ \isacommand{by}\isamarkupfalse%
\ {\isacharparenleft}{\kern0pt}simp\ add{\isacharcolon}{\kern0pt}assms\ Pi{\isacharunderscore}{\kern0pt}def{\isacharparenright}{\kern0pt}%
\endisatagproof
{\isafoldproof}%
%
\isadelimproof
\isanewline
%
\endisadelimproof
\isanewline
\isacommand{lemma}\isamarkupfalse%
\ range{\isacharunderscore}{\kern0pt}inter{\isacharcolon}{\kern0pt}\ {\isachardoublequoteopen}range\ {\isacharparenleft}{\kern0pt}{\isacharparenleft}{\kern0pt}{\isasyminter}{\isacharparenright}{\kern0pt}\ F{\isacharparenright}{\kern0pt}\ {\isacharequal}{\kern0pt}\ Pow\ F{\isachardoublequoteclose}\isanewline
%
\isadelimproof
\ \ %
\endisadelimproof
%
\isatagproof
\isacommand{apply}\isamarkupfalse%
\ {\isacharparenleft}{\kern0pt}rule\ order{\isacharunderscore}{\kern0pt}antisym{\isacharcomma}{\kern0pt}\ rule\ subsetI{\isacharcomma}{\kern0pt}\ simp\ add{\isacharcolon}{\kern0pt}image{\isacharunderscore}{\kern0pt}def{\isacharcomma}{\kern0pt}\ blast{\isacharparenright}{\kern0pt}\isanewline
\ \ \isacommand{by}\isamarkupfalse%
\ {\isacharparenleft}{\kern0pt}rule\ subsetI{\isacharcomma}{\kern0pt}\ simp\ add{\isacharcolon}{\kern0pt}image{\isacharunderscore}{\kern0pt}def{\isacharcomma}{\kern0pt}\ blast{\isacharparenright}{\kern0pt}%
\endisatagproof
{\isafoldproof}%
%
\isadelimproof
%
\endisadelimproof
%
\begin{isamarkuptext}%
On a finite set $M$ the $\sigma$-Algebra generated by singletons and the empty set
is already the power set of $M$.%
\end{isamarkuptext}\isamarkuptrue%
\isacommand{lemma}\isamarkupfalse%
\ sigma{\isacharunderscore}{\kern0pt}sets{\isacharunderscore}{\kern0pt}singletons{\isacharunderscore}{\kern0pt}and{\isacharunderscore}{\kern0pt}empty{\isacharcolon}{\kern0pt}\isanewline
\ \ \isakeyword{assumes}\ {\isachardoublequoteopen}countable\ M{\isachardoublequoteclose}\isanewline
\ \ \isakeyword{shows}\ {\isachardoublequoteopen}sigma{\isacharunderscore}{\kern0pt}sets\ M\ {\isacharparenleft}{\kern0pt}insert\ {\isacharbraceleft}{\kern0pt}{\isacharbraceright}{\kern0pt}\ {\isacharparenleft}{\kern0pt}{\isacharparenleft}{\kern0pt}{\isasymlambda}k{\isachardot}{\kern0pt}\ {\isacharbraceleft}{\kern0pt}k{\isacharbraceright}{\kern0pt}{\isacharparenright}{\kern0pt}\ {\isacharbackquote}{\kern0pt}\ M{\isacharparenright}{\kern0pt}{\isacharparenright}{\kern0pt}\ {\isacharequal}{\kern0pt}\ Pow\ M{\isachardoublequoteclose}\isanewline
%
\isadelimproof
%
\endisadelimproof
%
\isatagproof
\isacommand{proof}\isamarkupfalse%
\ {\isacharminus}{\kern0pt}\isanewline
\ \ \isacommand{have}\isamarkupfalse%
\ {\isachardoublequoteopen}sigma{\isacharunderscore}{\kern0pt}sets\ M\ {\isacharparenleft}{\kern0pt}{\isacharparenleft}{\kern0pt}{\isasymlambda}k{\isachardot}{\kern0pt}\ {\isacharbraceleft}{\kern0pt}k{\isacharbraceright}{\kern0pt}{\isacharparenright}{\kern0pt}\ {\isacharbackquote}{\kern0pt}\ M{\isacharparenright}{\kern0pt}\ {\isacharequal}{\kern0pt}\ Pow\ M{\isachardoublequoteclose}\isanewline
\ \ \ \ \isacommand{using}\isamarkupfalse%
\ assms\ sigma{\isacharunderscore}{\kern0pt}sets{\isacharunderscore}{\kern0pt}singletons\ \isacommand{by}\isamarkupfalse%
\ auto\isanewline
\ \ \isacommand{hence}\isamarkupfalse%
\ {\isachardoublequoteopen}Pow\ M\ {\isasymsubseteq}\ sigma{\isacharunderscore}{\kern0pt}sets\ M\ {\isacharparenleft}{\kern0pt}insert\ {\isacharbraceleft}{\kern0pt}{\isacharbraceright}{\kern0pt}\ {\isacharparenleft}{\kern0pt}{\isacharparenleft}{\kern0pt}{\isasymlambda}k{\isachardot}{\kern0pt}\ {\isacharbraceleft}{\kern0pt}k{\isacharbraceright}{\kern0pt}{\isacharparenright}{\kern0pt}\ {\isacharbackquote}{\kern0pt}\ M{\isacharparenright}{\kern0pt}{\isacharparenright}{\kern0pt}{\isachardoublequoteclose}\isanewline
\ \ \ \ \isacommand{by}\isamarkupfalse%
\ {\isacharparenleft}{\kern0pt}metis\ sigma{\isacharunderscore}{\kern0pt}sets{\isacharunderscore}{\kern0pt}subseteq\ subset{\isacharunderscore}{\kern0pt}insertI{\isacharparenright}{\kern0pt}\isanewline
\ \ \isacommand{moreover}\isamarkupfalse%
\ \isacommand{have}\isamarkupfalse%
\ {\isachardoublequoteopen}{\isacharparenleft}{\kern0pt}insert\ {\isacharbraceleft}{\kern0pt}{\isacharbraceright}{\kern0pt}\ {\isacharparenleft}{\kern0pt}{\isacharparenleft}{\kern0pt}{\isasymlambda}k{\isachardot}{\kern0pt}\ {\isacharbraceleft}{\kern0pt}k{\isacharbraceright}{\kern0pt}{\isacharparenright}{\kern0pt}\ {\isacharbackquote}{\kern0pt}\ M{\isacharparenright}{\kern0pt}{\isacharparenright}{\kern0pt}\ {\isasymsubseteq}\ Pow\ M{\isachardoublequoteclose}\ \isacommand{by}\isamarkupfalse%
\ blast\isanewline
\ \ \isacommand{hence}\isamarkupfalse%
\ {\isachardoublequoteopen}sigma{\isacharunderscore}{\kern0pt}sets\ M\ {\isacharparenleft}{\kern0pt}insert\ {\isacharbraceleft}{\kern0pt}{\isacharbraceright}{\kern0pt}\ {\isacharparenleft}{\kern0pt}{\isacharparenleft}{\kern0pt}{\isasymlambda}k{\isachardot}{\kern0pt}\ {\isacharbraceleft}{\kern0pt}k{\isacharbraceright}{\kern0pt}{\isacharparenright}{\kern0pt}\ {\isacharbackquote}{\kern0pt}\ M{\isacharparenright}{\kern0pt}{\isacharparenright}{\kern0pt}\ {\isasymsubseteq}\ Pow\ M{\isachardoublequoteclose}\isanewline
\ \ \ \ \isacommand{by}\isamarkupfalse%
\ {\isacharparenleft}{\kern0pt}meson\ sigma{\isacharunderscore}{\kern0pt}algebra{\isachardot}{\kern0pt}sigma{\isacharunderscore}{\kern0pt}sets{\isacharunderscore}{\kern0pt}subset\ sigma{\isacharunderscore}{\kern0pt}algebra{\isacharunderscore}{\kern0pt}Pow{\isacharparenright}{\kern0pt}\isanewline
\ \ \isacommand{ultimately}\isamarkupfalse%
\ \isacommand{show}\isamarkupfalse%
\ {\isacharquery}{\kern0pt}thesis\ \isacommand{by}\isamarkupfalse%
\ force\isanewline
\isacommand{qed}\isamarkupfalse%
%
\endisatagproof
{\isafoldproof}%
%
\isadelimproof
\isanewline
%
\endisadelimproof
\isanewline
\isacommand{lemma}\isamarkupfalse%
\ indep{\isacharunderscore}{\kern0pt}vars{\isacharunderscore}{\kern0pt}pmf{\isacharcolon}{\kern0pt}\isanewline
\ \ \isakeyword{assumes}\ {\isachardoublequoteopen}{\isasymAnd}a\ J{\isachardot}{\kern0pt}\ J\ {\isasymsubseteq}\ I\ {\isasymLongrightarrow}\ finite\ J\ {\isasymLongrightarrow}\ \isanewline
\ \ \ \ {\isasymP}{\isacharparenleft}{\kern0pt}{\isasymomega}\ in\ measure{\isacharunderscore}{\kern0pt}pmf\ M{\isachardot}{\kern0pt}\ {\isasymforall}i\ {\isasymin}\ J{\isachardot}{\kern0pt}\ X\ i\ {\isasymomega}\ {\isacharequal}{\kern0pt}\ a\ i{\isacharparenright}{\kern0pt}\ {\isacharequal}{\kern0pt}\ {\isacharparenleft}{\kern0pt}{\isasymProd}i\ {\isasymin}\ J{\isachardot}{\kern0pt}\ {\isasymP}{\isacharparenleft}{\kern0pt}{\isasymomega}\ in\ measure{\isacharunderscore}{\kern0pt}pmf\ M{\isachardot}{\kern0pt}\ X\ i\ {\isasymomega}\ {\isacharequal}{\kern0pt}\ a\ i{\isacharparenright}{\kern0pt}{\isacharparenright}{\kern0pt}{\isachardoublequoteclose}\isanewline
\ \ \isakeyword{shows}\ {\isachardoublequoteopen}prob{\isacharunderscore}{\kern0pt}space{\isachardot}{\kern0pt}indep{\isacharunderscore}{\kern0pt}vars\ {\isacharparenleft}{\kern0pt}measure{\isacharunderscore}{\kern0pt}pmf\ M{\isacharparenright}{\kern0pt}\ {\isacharparenleft}{\kern0pt}{\isasymlambda}i{\isachardot}{\kern0pt}\ measure{\isacharunderscore}{\kern0pt}pmf\ {\isacharparenleft}{\kern0pt}\ M{\isacharprime}{\kern0pt}\ i{\isacharparenright}{\kern0pt}{\isacharparenright}{\kern0pt}\ X\ I{\isachardoublequoteclose}\isanewline
%
\isadelimproof
%
\endisadelimproof
%
\isatagproof
\isacommand{proof}\isamarkupfalse%
\ {\isacharminus}{\kern0pt}\isanewline
\ \ \isacommand{define}\isamarkupfalse%
\ G\ \isakeyword{where}\ {\isachardoublequoteopen}G\ {\isacharequal}{\kern0pt}\ {\isacharparenleft}{\kern0pt}{\isasymlambda}i{\isachardot}{\kern0pt}\ {\isacharbraceleft}{\kern0pt}{\isacharbraceleft}{\kern0pt}{\isacharbraceright}{\kern0pt}{\isacharbraceright}{\kern0pt}\ {\isasymunion}\ {\isacharparenleft}{\kern0pt}{\isasymlambda}x{\isachardot}{\kern0pt}\ {\isacharbraceleft}{\kern0pt}x{\isacharbraceright}{\kern0pt}{\isacharparenright}{\kern0pt}\ {\isacharbackquote}{\kern0pt}\ {\isacharparenleft}{\kern0pt}X\ i\ {\isacharbackquote}{\kern0pt}\ set{\isacharunderscore}{\kern0pt}pmf\ M{\isacharparenright}{\kern0pt}{\isacharparenright}{\kern0pt}{\isachardoublequoteclose}\isanewline
\ \ \isacommand{define}\isamarkupfalse%
\ F\ \isakeyword{where}\ {\isachardoublequoteopen}F\ {\isacharequal}{\kern0pt}\ {\isacharparenleft}{\kern0pt}{\isasymlambda}i{\isachardot}{\kern0pt}\ {\isacharbraceleft}{\kern0pt}X\ i\ {\isacharminus}{\kern0pt}{\isacharbackquote}{\kern0pt}\ a\ {\isasyminter}\ set{\isacharunderscore}{\kern0pt}pmf\ M{\isacharbar}{\kern0pt}a{\isachardot}{\kern0pt}\ a\ {\isasymin}\ G\ i{\isacharbraceright}{\kern0pt}{\isacharparenright}{\kern0pt}{\isachardoublequoteclose}\isanewline
\isanewline
\ \ \isacommand{have}\isamarkupfalse%
\ g{\isacharcolon}{\kern0pt}\ {\isachardoublequoteopen}{\isasymAnd}i{\isachardot}{\kern0pt}\ i\ {\isasymin}\ I\ {\isasymLongrightarrow}\ sigma{\isacharunderscore}{\kern0pt}sets\ {\isacharparenleft}{\kern0pt}X\ i\ {\isacharbackquote}{\kern0pt}\ set{\isacharunderscore}{\kern0pt}pmf\ M{\isacharparenright}{\kern0pt}\ {\isacharparenleft}{\kern0pt}G\ i{\isacharparenright}{\kern0pt}\ {\isacharequal}{\kern0pt}\ Pow\ {\isacharparenleft}{\kern0pt}X\ i\ {\isacharbackquote}{\kern0pt}\ set{\isacharunderscore}{\kern0pt}pmf\ M{\isacharparenright}{\kern0pt}{\isachardoublequoteclose}\isanewline
\ \ \ \ \isacommand{by}\isamarkupfalse%
\ {\isacharparenleft}{\kern0pt}simp\ add{\isacharcolon}{\kern0pt}G{\isacharunderscore}{\kern0pt}def{\isacharcomma}{\kern0pt}\ metis\ countable{\isacharunderscore}{\kern0pt}image\ countable{\isacharunderscore}{\kern0pt}set{\isacharunderscore}{\kern0pt}pmf\ sigma{\isacharunderscore}{\kern0pt}sets{\isacharunderscore}{\kern0pt}singletons{\isacharunderscore}{\kern0pt}and{\isacharunderscore}{\kern0pt}empty{\isacharparenright}{\kern0pt}\isanewline
\isanewline
\ \ \isacommand{have}\isamarkupfalse%
\ e{\isacharcolon}{\kern0pt}\ {\isachardoublequoteopen}{\isasymAnd}i{\isachardot}{\kern0pt}\ i\ {\isasymin}\ I\ {\isasymLongrightarrow}\ F\ i\ {\isasymsubseteq}\ Pow\ {\isacharparenleft}{\kern0pt}set{\isacharunderscore}{\kern0pt}pmf\ M{\isacharparenright}{\kern0pt}{\isachardoublequoteclose}\isanewline
\ \ \ \ \isacommand{by}\isamarkupfalse%
\ {\isacharparenleft}{\kern0pt}simp\ add{\isacharcolon}{\kern0pt}F{\isacharunderscore}{\kern0pt}def{\isacharcomma}{\kern0pt}\ rule\ subsetI{\isacharcomma}{\kern0pt}\ simp{\isacharcomma}{\kern0pt}\ blast{\isacharparenright}{\kern0pt}\isanewline
\isanewline
\ \ \isacommand{have}\isamarkupfalse%
\ a{\isacharcolon}{\kern0pt}{\isachardoublequoteopen}distr\ {\isacharparenleft}{\kern0pt}restrict{\isacharunderscore}{\kern0pt}space\ {\isacharparenleft}{\kern0pt}measure{\isacharunderscore}{\kern0pt}pmf\ M{\isacharparenright}{\kern0pt}\ {\isacharparenleft}{\kern0pt}set{\isacharunderscore}{\kern0pt}pmf\ M{\isacharparenright}{\kern0pt}{\isacharparenright}{\kern0pt}\ {\isacharparenleft}{\kern0pt}measure{\isacharunderscore}{\kern0pt}pmf\ M{\isacharparenright}{\kern0pt}\ id\ {\isacharequal}{\kern0pt}\ measure{\isacharunderscore}{\kern0pt}pmf\ M{\isachardoublequoteclose}\isanewline
\ \ \ \ \isacommand{apply}\isamarkupfalse%
\ {\isacharparenleft}{\kern0pt}rule\ measure{\isacharunderscore}{\kern0pt}eqI{\isacharcomma}{\kern0pt}\ simp{\isacharcomma}{\kern0pt}\ simp{\isacharparenright}{\kern0pt}\isanewline
\ \ \ \ \isacommand{apply}\isamarkupfalse%
\ {\isacharparenleft}{\kern0pt}subst\ emeasure{\isacharunderscore}{\kern0pt}distr{\isacharparenright}{\kern0pt}\isanewline
\ \ \ \ \isacommand{apply}\isamarkupfalse%
\ {\isacharparenleft}{\kern0pt}simp\ add{\isacharcolon}{\kern0pt}measurable{\isacharunderscore}{\kern0pt}def\ sets{\isacharunderscore}{\kern0pt}restrict{\isacharunderscore}{\kern0pt}space{\isacharparenright}{\kern0pt}\ \isanewline
\ \ \ \ \ \ \isacommand{apply}\isamarkupfalse%
\ blast\isanewline
\ \ \ \ \ \isacommand{apply}\isamarkupfalse%
\ simp\isanewline
\ \ \ \ \isacommand{apply}\isamarkupfalse%
\ {\isacharparenleft}{\kern0pt}simp\ add{\isacharcolon}{\kern0pt}emeasure{\isacharunderscore}{\kern0pt}restrict{\isacharunderscore}{\kern0pt}space{\isacharparenright}{\kern0pt}\isanewline
\ \ \ \ \isacommand{by}\isamarkupfalse%
\ {\isacharparenleft}{\kern0pt}metis\ emeasure{\isacharunderscore}{\kern0pt}Int{\isacharunderscore}{\kern0pt}set{\isacharunderscore}{\kern0pt}pmf{\isacharparenright}{\kern0pt}\isanewline
\isanewline
\ \ \isacommand{have}\isamarkupfalse%
\ b{\isacharcolon}{\kern0pt}\ {\isachardoublequoteopen}prob{\isacharunderscore}{\kern0pt}space\ {\isacharparenleft}{\kern0pt}restrict{\isacharunderscore}{\kern0pt}space\ {\isacharparenleft}{\kern0pt}measure{\isacharunderscore}{\kern0pt}pmf\ M{\isacharparenright}{\kern0pt}\ {\isacharparenleft}{\kern0pt}set{\isacharunderscore}{\kern0pt}pmf\ M{\isacharparenright}{\kern0pt}{\isacharparenright}{\kern0pt}{\isachardoublequoteclose}\isanewline
\ \ \ \ \isacommand{apply}\isamarkupfalse%
\ {\isacharparenleft}{\kern0pt}rule\ prob{\isacharunderscore}{\kern0pt}spaceI{\isacharparenright}{\kern0pt}\isanewline
\ \ \ \ \isacommand{apply}\isamarkupfalse%
\ simp\isanewline
\ \ \ \ \isacommand{apply}\isamarkupfalse%
\ {\isacharparenleft}{\kern0pt}subst\ emeasure{\isacharunderscore}{\kern0pt}restrict{\isacharunderscore}{\kern0pt}space{\isacharcomma}{\kern0pt}\ simp{\isacharcomma}{\kern0pt}\ simp{\isacharparenright}{\kern0pt}\isanewline
\ \ \ \ \isacommand{using}\isamarkupfalse%
\ emeasure{\isacharunderscore}{\kern0pt}pmf\ \isacommand{by}\isamarkupfalse%
\ blast\isanewline
\isanewline
\ \ \isacommand{have}\isamarkupfalse%
\ d{\isacharcolon}{\kern0pt}{\isachardoublequoteopen}{\isasymAnd}i{\isachardot}{\kern0pt}\ i\ {\isasymin}\ I\ {\isasymLongrightarrow}\ {\isacharbraceleft}{\kern0pt}u{\isachardot}{\kern0pt}\ {\isasymexists}A{\isachardot}{\kern0pt}\ u\ {\isacharequal}{\kern0pt}\ X\ i\ {\isacharminus}{\kern0pt}{\isacharbackquote}{\kern0pt}\ A\ {\isasyminter}\ set{\isacharunderscore}{\kern0pt}pmf\ M{\isacharbraceright}{\kern0pt}\ {\isacharequal}{\kern0pt}\ sigma{\isacharunderscore}{\kern0pt}sets\ {\isacharparenleft}{\kern0pt}set{\isacharunderscore}{\kern0pt}pmf\ M{\isacharparenright}{\kern0pt}\ {\isacharparenleft}{\kern0pt}F\ i{\isacharparenright}{\kern0pt}{\isachardoublequoteclose}\isanewline
\ \ \isacommand{proof}\isamarkupfalse%
\ {\isacharminus}{\kern0pt}\isanewline
\ \ \ \ \isacommand{fix}\isamarkupfalse%
\ i\isanewline
\ \ \ \ \isacommand{assume}\isamarkupfalse%
\ d{\isadigit{1}}{\isacharcolon}{\kern0pt}{\isachardoublequoteopen}i\ {\isasymin}\ I{\isachardoublequoteclose}\isanewline
\ \ \ \ \isacommand{have}\isamarkupfalse%
\ d{\isadigit{2}}{\isacharcolon}{\kern0pt}\ {\isachardoublequoteopen}{\isasymAnd}A{\isachardot}{\kern0pt}\ X\ i\ {\isacharminus}{\kern0pt}{\isacharbackquote}{\kern0pt}\ A\ {\isasyminter}\ set{\isacharunderscore}{\kern0pt}pmf\ M\ {\isacharequal}{\kern0pt}\ X\ i\ {\isacharminus}{\kern0pt}{\isacharbackquote}{\kern0pt}\ {\isacharparenleft}{\kern0pt}A\ {\isasyminter}\ X\ i\ {\isacharbackquote}{\kern0pt}\ set{\isacharunderscore}{\kern0pt}pmf\ M{\isacharparenright}{\kern0pt}\ {\isasyminter}\ set{\isacharunderscore}{\kern0pt}pmf\ M{\isachardoublequoteclose}\isanewline
\ \ \ \ \ \ \isacommand{apply}\isamarkupfalse%
\ {\isacharparenleft}{\kern0pt}rule\ order{\isacharunderscore}{\kern0pt}antisym{\isacharparenright}{\kern0pt}\isanewline
\ \ \ \ \ \ \isacommand{by}\isamarkupfalse%
\ {\isacharparenleft}{\kern0pt}rule\ subsetI{\isacharcomma}{\kern0pt}\ simp{\isacharparenright}{\kern0pt}{\isacharplus}{\kern0pt}\isanewline
\ \ \ \ \isacommand{show}\isamarkupfalse%
\ {\isachardoublequoteopen}\ {\isacharbraceleft}{\kern0pt}u{\isachardot}{\kern0pt}\ {\isasymexists}A{\isachardot}{\kern0pt}\ u\ {\isacharequal}{\kern0pt}\ X\ i\ {\isacharminus}{\kern0pt}{\isacharbackquote}{\kern0pt}\ A\ {\isasyminter}\ set{\isacharunderscore}{\kern0pt}pmf\ M{\isacharbraceright}{\kern0pt}\ {\isacharequal}{\kern0pt}\ sigma{\isacharunderscore}{\kern0pt}sets\ {\isacharparenleft}{\kern0pt}set{\isacharunderscore}{\kern0pt}pmf\ M{\isacharparenright}{\kern0pt}\ {\isacharparenleft}{\kern0pt}F\ i{\isacharparenright}{\kern0pt}{\isachardoublequoteclose}\isanewline
\ \ \ \ \ \ \isacommand{apply}\isamarkupfalse%
\ {\isacharparenleft}{\kern0pt}simp\ add{\isacharcolon}{\kern0pt}F{\isacharunderscore}{\kern0pt}def{\isacharparenright}{\kern0pt}\isanewline
\ \ \ \ \ \ \isacommand{apply}\isamarkupfalse%
\ {\isacharparenleft}{\kern0pt}subst\ sigma{\isacharunderscore}{\kern0pt}sets{\isacharunderscore}{\kern0pt}vimage{\isacharunderscore}{\kern0pt}commute{\isacharbrackleft}{\kern0pt}symmetric{\isacharcomma}{\kern0pt}\ \isakeyword{where}\ {\isasymOmega}{\isacharprime}{\kern0pt}\ {\isacharequal}{\kern0pt}\ {\isachardoublequoteopen}X\ i\ {\isacharbackquote}{\kern0pt}\ set{\isacharunderscore}{\kern0pt}pmf\ M{\isachardoublequoteclose}{\isacharbrackright}{\kern0pt}{\isacharcomma}{\kern0pt}\ blast{\isacharparenright}{\kern0pt}\isanewline
\ \ \ \ \ \ \isacommand{using}\isamarkupfalse%
\ d{\isadigit{1}}\ \isacommand{apply}\isamarkupfalse%
\ {\isacharparenleft}{\kern0pt}simp\ add{\isacharcolon}{\kern0pt}g{\isacharparenright}{\kern0pt}\isanewline
\ \ \ \ \ \ \isacommand{apply}\isamarkupfalse%
\ {\isacharparenleft}{\kern0pt}rule\ order{\isacharunderscore}{\kern0pt}antisym{\isacharparenright}{\kern0pt}\isanewline
\ \ \ \ \ \ \ \isacommand{apply}\isamarkupfalse%
\ {\isacharparenleft}{\kern0pt}rule\ subsetI{\isacharcomma}{\kern0pt}\ simp{\isacharcomma}{\kern0pt}\ meson\ inf{\isacharunderscore}{\kern0pt}le{\isadigit{2}}\ d{\isadigit{2}}{\isacharparenright}{\kern0pt}\isanewline
\ \ \ \ \ \ \isacommand{by}\isamarkupfalse%
\ {\isacharparenleft}{\kern0pt}rule\ subsetI{\isacharcomma}{\kern0pt}\ simp{\isacharcomma}{\kern0pt}\ blast{\isacharparenright}{\kern0pt}\isanewline
\ \ \isacommand{qed}\isamarkupfalse%
\isanewline
\isanewline
\ \ \isacommand{have}\isamarkupfalse%
\ h{\isacharcolon}{\kern0pt}{\isachardoublequoteopen}{\isasymAnd}J\ A{\isachardot}{\kern0pt}\ J\ {\isasymsubseteq}\ I\ {\isasymLongrightarrow}\ J\ {\isasymnoteq}\ {\isacharbraceleft}{\kern0pt}{\isacharbraceright}{\kern0pt}\ {\isasymLongrightarrow}\ finite\ J\ {\isasymLongrightarrow}\ A{\isasymin}Pi\ J\ F\ {\isasymLongrightarrow}\isanewline
\ \ \ \ \ \ \ \ \ \ \ \ \ \ \ Sigma{\isacharunderscore}{\kern0pt}Algebra{\isachardot}{\kern0pt}measure\ {\isacharparenleft}{\kern0pt}restrict{\isacharunderscore}{\kern0pt}space\ {\isacharparenleft}{\kern0pt}measure{\isacharunderscore}{\kern0pt}pmf\ M{\isacharparenright}{\kern0pt}\ {\isacharparenleft}{\kern0pt}set{\isacharunderscore}{\kern0pt}pmf\ M{\isacharparenright}{\kern0pt}{\isacharparenright}{\kern0pt}\ {\isacharparenleft}{\kern0pt}{\isasymInter}\ {\isacharparenleft}{\kern0pt}A\ {\isacharbackquote}{\kern0pt}\ J{\isacharparenright}{\kern0pt}{\isacharparenright}{\kern0pt}\ {\isacharequal}{\kern0pt}\isanewline
\ \ \ \ \ \ \ \ \ \ \ \ \ \ \ {\isacharparenleft}{\kern0pt}{\isasymProd}j{\isasymin}J{\isachardot}{\kern0pt}\ Sigma{\isacharunderscore}{\kern0pt}Algebra{\isachardot}{\kern0pt}measure\ {\isacharparenleft}{\kern0pt}restrict{\isacharunderscore}{\kern0pt}space\ {\isacharparenleft}{\kern0pt}measure{\isacharunderscore}{\kern0pt}pmf\ M{\isacharparenright}{\kern0pt}\ {\isacharparenleft}{\kern0pt}set{\isacharunderscore}{\kern0pt}pmf\ M{\isacharparenright}{\kern0pt}{\isacharparenright}{\kern0pt}\ {\isacharparenleft}{\kern0pt}A\ j{\isacharparenright}{\kern0pt}{\isacharparenright}{\kern0pt}{\isachardoublequoteclose}\isanewline
\ \ \isacommand{proof}\isamarkupfalse%
\ {\isacharminus}{\kern0pt}\isanewline
\ \ \ \ \isacommand{fix}\isamarkupfalse%
\ J\ A\isanewline
\ \ \ \ \isacommand{assume}\isamarkupfalse%
\ h{\isadigit{1}}{\isacharcolon}{\kern0pt}\ {\isachardoublequoteopen}J\ {\isasymsubseteq}\ I{\isachardoublequoteclose}\isanewline
\ \ \ \ \isacommand{assume}\isamarkupfalse%
\ h{\isadigit{2}}{\isacharcolon}{\kern0pt}\ {\isachardoublequoteopen}J\ {\isasymnoteq}\ {\isacharbraceleft}{\kern0pt}{\isacharbraceright}{\kern0pt}{\isachardoublequoteclose}\isanewline
\ \ \ \ \isacommand{assume}\isamarkupfalse%
\ h{\isadigit{3}}{\isacharcolon}{\kern0pt}{\isachardoublequoteopen}finite\ J{\isachardoublequoteclose}\isanewline
\ \ \ \ \isacommand{assume}\isamarkupfalse%
\ h{\isadigit{4}}{\isacharcolon}{\kern0pt}\ {\isachardoublequoteopen}A\ {\isasymin}\ Pi\ J\ F{\isachardoublequoteclose}\isanewline
\isanewline
\ \ \ \ \isacommand{have}\isamarkupfalse%
\ h{\isadigit{5}}{\isacharcolon}{\kern0pt}\ {\isachardoublequoteopen}{\isasymAnd}j{\isachardot}{\kern0pt}\ j\ {\isasymin}\ J\ {\isasymLongrightarrow}\ A\ j\ {\isasymsubseteq}\ set{\isacharunderscore}{\kern0pt}pmf\ M{\isachardoublequoteclose}\isanewline
\ \ \ \ \ \ \isacommand{by}\isamarkupfalse%
\ {\isacharparenleft}{\kern0pt}metis\ PiE\ PowD\ h{\isadigit{1}}\ subsetD\ e\ h{\isadigit{4}}{\isacharparenright}{\kern0pt}\isanewline
\ \ \ \ \isacommand{obtain}\isamarkupfalse%
\ a\ \isakeyword{where}\ h{\isadigit{6}}{\isacharcolon}{\kern0pt}{\isachardoublequoteopen}{\isasymAnd}j{\isachardot}{\kern0pt}\ j\ {\isasymin}\ J\ \ {\isasymLongrightarrow}\ A\ j\ {\isacharequal}{\kern0pt}\ X\ j\ {\isacharminus}{\kern0pt}{\isacharbackquote}{\kern0pt}\ a\ j\ {\isasyminter}\ set{\isacharunderscore}{\kern0pt}pmf\ M\ {\isasymand}\ a\ j\ {\isasymin}\ G\ j{\isachardoublequoteclose}\isanewline
\ \ \ \ \ \ \isacommand{using}\isamarkupfalse%
\ h{\isadigit{4}}\ \isacommand{by}\isamarkupfalse%
\ {\isacharparenleft}{\kern0pt}simp\ add{\isacharcolon}{\kern0pt}Pi{\isacharunderscore}{\kern0pt}def\ F{\isacharunderscore}{\kern0pt}def{\isacharcomma}{\kern0pt}\ metis{\isacharparenright}{\kern0pt}\isanewline
\isanewline
\ \ \ \ \isacommand{show}\isamarkupfalse%
\ {\isachardoublequoteopen}Sigma{\isacharunderscore}{\kern0pt}Algebra{\isachardot}{\kern0pt}measure\ {\isacharparenleft}{\kern0pt}restrict{\isacharunderscore}{\kern0pt}space\ {\isacharparenleft}{\kern0pt}measure{\isacharunderscore}{\kern0pt}pmf\ M{\isacharparenright}{\kern0pt}\ {\isacharparenleft}{\kern0pt}set{\isacharunderscore}{\kern0pt}pmf\ M{\isacharparenright}{\kern0pt}{\isacharparenright}{\kern0pt}\ {\isacharparenleft}{\kern0pt}{\isasymInter}\ {\isacharparenleft}{\kern0pt}A\ {\isacharbackquote}{\kern0pt}\ J{\isacharparenright}{\kern0pt}{\isacharparenright}{\kern0pt}\ {\isacharequal}{\kern0pt}\isanewline
\ \ \ \ \ \ \ \ \ \ \ \ \ \ \ {\isacharparenleft}{\kern0pt}{\isasymProd}j{\isasymin}J{\isachardot}{\kern0pt}\ Sigma{\isacharunderscore}{\kern0pt}Algebra{\isachardot}{\kern0pt}measure\ {\isacharparenleft}{\kern0pt}restrict{\isacharunderscore}{\kern0pt}space\ {\isacharparenleft}{\kern0pt}measure{\isacharunderscore}{\kern0pt}pmf\ M{\isacharparenright}{\kern0pt}\ {\isacharparenleft}{\kern0pt}set{\isacharunderscore}{\kern0pt}pmf\ M{\isacharparenright}{\kern0pt}{\isacharparenright}{\kern0pt}\ {\isacharparenleft}{\kern0pt}A\ j{\isacharparenright}{\kern0pt}{\isacharparenright}{\kern0pt}{\isachardoublequoteclose}\isanewline
\ \ \ \ \isacommand{proof}\isamarkupfalse%
\ {\isacharparenleft}{\kern0pt}cases\ {\isachardoublequoteopen}{\isasymexists}j\ {\isasymin}\ J{\isachardot}{\kern0pt}\ A\ j\ {\isacharequal}{\kern0pt}\ {\isacharbraceleft}{\kern0pt}{\isacharbraceright}{\kern0pt}{\isachardoublequoteclose}{\isacharparenright}{\kern0pt}\isanewline
\ \ \ \ \ \ \isacommand{case}\isamarkupfalse%
\ True\isanewline
\ \ \ \ \ \ \isacommand{hence}\isamarkupfalse%
\ {\isachardoublequoteopen}{\isasymInter}\ {\isacharparenleft}{\kern0pt}A\ {\isacharbackquote}{\kern0pt}\ J{\isacharparenright}{\kern0pt}\ {\isacharequal}{\kern0pt}\ {\isacharbraceleft}{\kern0pt}{\isacharbraceright}{\kern0pt}{\isachardoublequoteclose}\ \isacommand{by}\isamarkupfalse%
\ blast\isanewline
\ \ \ \ \ \ \isacommand{then}\isamarkupfalse%
\ \isacommand{show}\isamarkupfalse%
\ {\isacharquery}{\kern0pt}thesis\isanewline
\ \ \ \ \ \ \ \ \isacommand{using}\isamarkupfalse%
\ h{\isadigit{3}}\ True\ \isacommand{apply}\isamarkupfalse%
\ simp\ \isanewline
\ \ \ \ \ \ \ \ \isacommand{by}\isamarkupfalse%
\ {\isacharparenleft}{\kern0pt}metis\ measure{\isacharunderscore}{\kern0pt}empty{\isacharparenright}{\kern0pt}\isanewline
\ \ \ \ \isacommand{next}\isamarkupfalse%
\isanewline
\ \ \ \ \ \ \isacommand{case}\isamarkupfalse%
\ False\isanewline
\ \ \ \ \ \ \isacommand{then}\isamarkupfalse%
\ \isacommand{have}\isamarkupfalse%
\ {\isachardoublequoteopen}{\isasymAnd}j{\isachardot}{\kern0pt}\ j\ {\isasymin}\ J\ {\isasymLongrightarrow}\ a\ j\ {\isasymnoteq}\ {\isacharbraceleft}{\kern0pt}{\isacharbraceright}{\kern0pt}{\isachardoublequoteclose}\ \isacommand{using}\isamarkupfalse%
\ h{\isadigit{6}}\ \isacommand{by}\isamarkupfalse%
\ auto\isanewline
\ \ \ \ \ \ \isacommand{hence}\isamarkupfalse%
\ {\isachardoublequoteopen}{\isasymAnd}j{\isachardot}{\kern0pt}\ j\ {\isasymin}\ J\ {\isasymLongrightarrow}\ a\ j\ {\isasymin}\ {\isacharparenleft}{\kern0pt}{\isasymlambda}x{\isachardot}{\kern0pt}\ {\isacharbraceleft}{\kern0pt}x{\isacharbraceright}{\kern0pt}{\isacharparenright}{\kern0pt}\ {\isacharbackquote}{\kern0pt}\ X\ j\ {\isacharbackquote}{\kern0pt}\ set{\isacharunderscore}{\kern0pt}pmf\ M{\isachardoublequoteclose}\ \isacommand{using}\isamarkupfalse%
\ h{\isadigit{6}}\ \isacommand{by}\isamarkupfalse%
\ {\isacharparenleft}{\kern0pt}simp\ add{\isacharcolon}{\kern0pt}G{\isacharunderscore}{\kern0pt}def{\isacharparenright}{\kern0pt}\ \isanewline
\ \ \ \ \ \ \isacommand{then}\isamarkupfalse%
\ \isacommand{obtain}\isamarkupfalse%
\ b\ \isakeyword{where}\ h{\isadigit{7}}{\isacharcolon}{\kern0pt}\ {\isachardoublequoteopen}{\isasymAnd}j{\isachardot}{\kern0pt}\ j\ {\isasymin}\ J\ {\isasymLongrightarrow}\ a\ j\ {\isacharequal}{\kern0pt}\ {\isacharbraceleft}{\kern0pt}b\ j{\isacharbraceright}{\kern0pt}{\isachardoublequoteclose}\ \isacommand{by}\isamarkupfalse%
\ {\isacharparenleft}{\kern0pt}simp\ add{\isacharcolon}{\kern0pt}image{\isacharunderscore}{\kern0pt}def{\isacharcomma}{\kern0pt}\ metis{\isacharparenright}{\kern0pt}\ \isanewline
\isanewline
\ \ \ \ \ \ \isacommand{have}\isamarkupfalse%
\ {\isachardoublequoteopen}Sigma{\isacharunderscore}{\kern0pt}Algebra{\isachardot}{\kern0pt}measure\ {\isacharparenleft}{\kern0pt}restrict{\isacharunderscore}{\kern0pt}space\ {\isacharparenleft}{\kern0pt}measure{\isacharunderscore}{\kern0pt}pmf\ M{\isacharparenright}{\kern0pt}\ {\isacharparenleft}{\kern0pt}set{\isacharunderscore}{\kern0pt}pmf\ M{\isacharparenright}{\kern0pt}{\isacharparenright}{\kern0pt}\ {\isacharparenleft}{\kern0pt}{\isasymInter}\ {\isacharparenleft}{\kern0pt}A\ {\isacharbackquote}{\kern0pt}\ J{\isacharparenright}{\kern0pt}{\isacharparenright}{\kern0pt}\ {\isacharequal}{\kern0pt}\isanewline
\ \ \ \ \ \ \ \ Sigma{\isacharunderscore}{\kern0pt}Algebra{\isachardot}{\kern0pt}measure\ {\isacharparenleft}{\kern0pt}measure{\isacharunderscore}{\kern0pt}pmf\ M{\isacharparenright}{\kern0pt}\ {\isacharparenleft}{\kern0pt}{\isasymInter}\ j\ {\isasymin}\ J{\isachardot}{\kern0pt}\ A\ j{\isacharparenright}{\kern0pt}{\isachardoublequoteclose}\isanewline
\ \ \ \ \ \ \ \ \isacommand{apply}\isamarkupfalse%
\ {\isacharparenleft}{\kern0pt}subst\ measure{\isacharunderscore}{\kern0pt}restrict{\isacharunderscore}{\kern0pt}space{\isacharcomma}{\kern0pt}\ simp{\isacharparenright}{\kern0pt}\isanewline
\ \ \ \ \ \ \ \ \isacommand{using}\isamarkupfalse%
\ h{\isadigit{5}}\ h{\isadigit{2}}\ \isacommand{apply}\isamarkupfalse%
\ blast\isanewline
\ \ \ \ \ \ \ \ \isacommand{by}\isamarkupfalse%
\ simp\isanewline
\ \ \ \ \ \ \isacommand{also}\isamarkupfalse%
\ \isacommand{have}\isamarkupfalse%
\ {\isachardoublequoteopen}{\isachardot}{\kern0pt}{\isachardot}{\kern0pt}{\isachardot}{\kern0pt}\ {\isacharequal}{\kern0pt}\ Sigma{\isacharunderscore}{\kern0pt}Algebra{\isachardot}{\kern0pt}measure\ {\isacharparenleft}{\kern0pt}measure{\isacharunderscore}{\kern0pt}pmf\ M{\isacharparenright}{\kern0pt}\ {\isacharparenleft}{\kern0pt}{\isacharbraceleft}{\kern0pt}{\isasymomega}{\isachardot}{\kern0pt}\ {\isasymforall}j\ {\isasymin}\ J{\isachardot}{\kern0pt}\ X\ j\ {\isasymomega}\ {\isacharequal}{\kern0pt}\ b\ j{\isacharbraceright}{\kern0pt}{\isacharparenright}{\kern0pt}{\isachardoublequoteclose}\isanewline
\ \ \ \ \ \ \ \ \isacommand{using}\isamarkupfalse%
\ h{\isadigit{2}}\ h{\isadigit{6}}\ h{\isadigit{7}}\ \isacommand{apply}\isamarkupfalse%
\ {\isacharparenleft}{\kern0pt}simp\ add{\isacharcolon}{\kern0pt}vimage{\isacharunderscore}{\kern0pt}def\ measure{\isacharunderscore}{\kern0pt}Int{\isacharunderscore}{\kern0pt}set{\isacharunderscore}{\kern0pt}pmf{\isacharparenright}{\kern0pt}\isanewline
\ \ \ \ \ \ \ \ \isacommand{by}\isamarkupfalse%
\ {\isacharparenleft}{\kern0pt}rule\ arg{\isacharunderscore}{\kern0pt}cong{\isadigit{2}}\ {\isacharbrackleft}{\kern0pt}\isakeyword{where}\ f{\isacharequal}{\kern0pt}{\isachardoublequoteopen}measure{\isachardoublequoteclose}{\isacharbrackright}{\kern0pt}{\isacharcomma}{\kern0pt}\ simp{\isacharcomma}{\kern0pt}\ blast{\isacharparenright}{\kern0pt}\isanewline
\ \ \ \ \ \ \isacommand{also}\isamarkupfalse%
\ \isacommand{have}\isamarkupfalse%
\ {\isachardoublequoteopen}{\isachardot}{\kern0pt}{\isachardot}{\kern0pt}{\isachardot}{\kern0pt}\ {\isacharequal}{\kern0pt}\ {\isacharparenleft}{\kern0pt}{\isasymProd}\ j{\isasymin}\ J{\isachardot}{\kern0pt}\ Sigma{\isacharunderscore}{\kern0pt}Algebra{\isachardot}{\kern0pt}measure\ {\isacharparenleft}{\kern0pt}measure{\isacharunderscore}{\kern0pt}pmf\ M{\isacharparenright}{\kern0pt}\ {\isacharparenleft}{\kern0pt}A\ j{\isacharparenright}{\kern0pt}{\isacharparenright}{\kern0pt}{\isachardoublequoteclose}\isanewline
\ \ \ \ \ \ \ \ \isacommand{using}\isamarkupfalse%
\ h{\isadigit{6}}\ h{\isadigit{7}}\ h{\isadigit{2}}\ assms{\isacharparenleft}{\kern0pt}{\isadigit{1}}{\isacharparenright}{\kern0pt}{\isacharbrackleft}{\kern0pt}OF\ h{\isadigit{1}}\ h{\isadigit{3}}{\isacharbrackright}{\kern0pt}\ \isacommand{by}\isamarkupfalse%
\ {\isacharparenleft}{\kern0pt}simp\ add{\isacharcolon}{\kern0pt}vimage{\isacharunderscore}{\kern0pt}def\ measure{\isacharunderscore}{\kern0pt}Int{\isacharunderscore}{\kern0pt}set{\isacharunderscore}{\kern0pt}pmf{\isacharparenright}{\kern0pt}\isanewline
\ \ \ \ \ \ \isacommand{also}\isamarkupfalse%
\ \isacommand{have}\isamarkupfalse%
\ {\isachardoublequoteopen}{\isachardot}{\kern0pt}{\isachardot}{\kern0pt}{\isachardot}{\kern0pt}\ {\isacharequal}{\kern0pt}\ {\isacharparenleft}{\kern0pt}{\isasymProd}j{\isasymin}J{\isachardot}{\kern0pt}\ Sigma{\isacharunderscore}{\kern0pt}Algebra{\isachardot}{\kern0pt}measure\ {\isacharparenleft}{\kern0pt}restrict{\isacharunderscore}{\kern0pt}space\ {\isacharparenleft}{\kern0pt}measure{\isacharunderscore}{\kern0pt}pmf\ M{\isacharparenright}{\kern0pt}\ {\isacharparenleft}{\kern0pt}set{\isacharunderscore}{\kern0pt}pmf\ M{\isacharparenright}{\kern0pt}{\isacharparenright}{\kern0pt}\ {\isacharparenleft}{\kern0pt}A\ j{\isacharparenright}{\kern0pt}{\isacharparenright}{\kern0pt}{\isachardoublequoteclose}\isanewline
\ \ \ \ \ \ \ \ \isacommand{by}\isamarkupfalse%
\ {\isacharparenleft}{\kern0pt}rule\ prod{\isachardot}{\kern0pt}cong{\isacharcomma}{\kern0pt}\ simp{\isacharcomma}{\kern0pt}\ subst\ measure{\isacharunderscore}{\kern0pt}restrict{\isacharunderscore}{\kern0pt}space{\isacharcomma}{\kern0pt}\ simp{\isacharcomma}{\kern0pt}\ metis\ h{\isadigit{5}}{\isacharcomma}{\kern0pt}\ simp{\isacharparenright}{\kern0pt}\isanewline
\ \ \ \ \ \ \isacommand{finally}\isamarkupfalse%
\ \isacommand{show}\isamarkupfalse%
\ {\isacharquery}{\kern0pt}thesis\ \isacommand{by}\isamarkupfalse%
\ blast\isanewline
\ \ \ \ \isacommand{qed}\isamarkupfalse%
\isanewline
\ \ \isacommand{qed}\isamarkupfalse%
\isanewline
\isanewline
\ \ \isacommand{have}\isamarkupfalse%
\ i{\isacharcolon}{\kern0pt}\ {\isachardoublequoteopen}{\isasymAnd}i{\isachardot}{\kern0pt}\ i\ {\isasymin}\ I\ {\isasymLongrightarrow}\ Int{\isacharunderscore}{\kern0pt}stable\ {\isacharparenleft}{\kern0pt}F\ i{\isacharparenright}{\kern0pt}{\isachardoublequoteclose}\isanewline
\ \ \isacommand{proof}\isamarkupfalse%
\ {\isacharparenleft}{\kern0pt}rule\ Int{\isacharunderscore}{\kern0pt}stableI{\isacharparenright}{\kern0pt}\isanewline
\ \ \ \ \isacommand{fix}\isamarkupfalse%
\ i\ a\ b\isanewline
\ \ \ \ \isacommand{assume}\isamarkupfalse%
\ {\isachardoublequoteopen}i\ {\isasymin}\ I{\isachardoublequoteclose}\isanewline
\ \ \ \ \isacommand{assume}\isamarkupfalse%
\ {\isachardoublequoteopen}a\ {\isasymin}\ F\ i{\isachardoublequoteclose}\isanewline
\ \ \ \ \isacommand{moreover}\isamarkupfalse%
\ \isacommand{assume}\isamarkupfalse%
\ {\isachardoublequoteopen}b\ {\isasymin}\ F\ i{\isachardoublequoteclose}\isanewline
\ \ \ \ \isacommand{ultimately}\isamarkupfalse%
\ \isacommand{show}\isamarkupfalse%
\ {\isachardoublequoteopen}a\ {\isasyminter}\ b\ {\isasymin}\ {\isacharparenleft}{\kern0pt}F\ i{\isacharparenright}{\kern0pt}{\isachardoublequoteclose}\isanewline
\ \ \ \ \ \ \isacommand{apply}\isamarkupfalse%
\ {\isacharparenleft}{\kern0pt}cases\ {\isachardoublequoteopen}a\ {\isasyminter}\ b\ {\isacharequal}{\kern0pt}\ {\isacharbraceleft}{\kern0pt}{\isacharbraceright}{\kern0pt}{\isachardoublequoteclose}{\isacharcomma}{\kern0pt}\ simp\ add{\isacharcolon}{\kern0pt}F{\isacharunderscore}{\kern0pt}def\ G{\isacharunderscore}{\kern0pt}def{\isacharcomma}{\kern0pt}\ blast{\isacharparenright}{\kern0pt}\isanewline
\ \ \ \ \ \ \isacommand{by}\isamarkupfalse%
\ {\isacharparenleft}{\kern0pt}simp\ add{\isacharcolon}{\kern0pt}F{\isacharunderscore}{\kern0pt}def\ G{\isacharunderscore}{\kern0pt}def{\isacharcomma}{\kern0pt}\ blast{\isacharparenright}{\kern0pt}\isanewline
\ \ \isacommand{qed}\isamarkupfalse%
\isanewline
\isanewline
\ \ \isacommand{have}\isamarkupfalse%
\ c{\isacharcolon}{\kern0pt}\ {\isachardoublequoteopen}prob{\isacharunderscore}{\kern0pt}space{\isachardot}{\kern0pt}indep{\isacharunderscore}{\kern0pt}sets\ {\isacharparenleft}{\kern0pt}restrict{\isacharunderscore}{\kern0pt}space\ {\isacharparenleft}{\kern0pt}measure{\isacharunderscore}{\kern0pt}pmf\ M{\isacharparenright}{\kern0pt}\ {\isacharparenleft}{\kern0pt}set{\isacharunderscore}{\kern0pt}pmf\ M{\isacharparenright}{\kern0pt}{\isacharparenright}{\kern0pt}\ {\isacharparenleft}{\kern0pt}{\isasymlambda}i{\isachardot}{\kern0pt}\ {\isacharbraceleft}{\kern0pt}u{\isachardot}{\kern0pt}\ {\isasymexists}A{\isachardot}{\kern0pt}\ u\ {\isacharequal}{\kern0pt}\ X\ i\ {\isacharminus}{\kern0pt}{\isacharbackquote}{\kern0pt}\ A\ {\isasyminter}\ set{\isacharunderscore}{\kern0pt}pmf\ M{\isacharbraceright}{\kern0pt}{\isacharparenright}{\kern0pt}\ I{\isachardoublequoteclose}\isanewline
\ \ \ \ \isacommand{apply}\isamarkupfalse%
\ {\isacharparenleft}{\kern0pt}simp\ add{\isacharcolon}{\kern0pt}\ d\ cong{\isacharcolon}{\kern0pt}prob{\isacharunderscore}{\kern0pt}space{\isachardot}{\kern0pt}indep{\isacharunderscore}{\kern0pt}sets{\isacharunderscore}{\kern0pt}cong{\isacharbrackleft}{\kern0pt}OF\ b{\isacharbrackright}{\kern0pt}{\isacharparenright}{\kern0pt}\isanewline
\ \ \ \ \isacommand{apply}\isamarkupfalse%
\ {\isacharparenleft}{\kern0pt}rule\ prob{\isacharunderscore}{\kern0pt}space{\isachardot}{\kern0pt}indep{\isacharunderscore}{\kern0pt}sets{\isacharunderscore}{\kern0pt}sigma{\isacharbrackleft}{\kern0pt}\isakeyword{where}\ M{\isacharequal}{\kern0pt}{\isachardoublequoteopen}restrict{\isacharunderscore}{\kern0pt}space\ {\isacharparenleft}{\kern0pt}measure{\isacharunderscore}{\kern0pt}pmf\ M{\isacharparenright}{\kern0pt}\ {\isacharparenleft}{\kern0pt}set{\isacharunderscore}{\kern0pt}pmf\ M{\isacharparenright}{\kern0pt}{\isachardoublequoteclose}{\isacharcomma}{\kern0pt}\ simplified{\isacharbrackright}{\kern0pt}{\isacharparenright}{\kern0pt}\isanewline
\ \ \ \ \ \ \isacommand{apply}\isamarkupfalse%
\ {\isacharparenleft}{\kern0pt}metis\ b{\isacharparenright}{\kern0pt}\isanewline
\ \ \ \ \ \isacommand{apply}\isamarkupfalse%
\ {\isacharparenleft}{\kern0pt}subst\ prob{\isacharunderscore}{\kern0pt}space{\isachardot}{\kern0pt}indep{\isacharunderscore}{\kern0pt}sets{\isacharunderscore}{\kern0pt}def{\isacharcomma}{\kern0pt}\ metis\ b{\isacharcomma}{\kern0pt}\ simp\ add{\isacharcolon}{\kern0pt}sets{\isacharunderscore}{\kern0pt}restrict{\isacharunderscore}{\kern0pt}space\ range{\isacharunderscore}{\kern0pt}inter\ e{\isacharparenright}{\kern0pt}\isanewline
\ \ \ \ \ \isacommand{apply}\isamarkupfalse%
\ {\isacharparenleft}{\kern0pt}metis\ h{\isacharparenright}{\kern0pt}\isanewline
\ \ \ \ \isacommand{by}\isamarkupfalse%
\ {\isacharparenleft}{\kern0pt}metis\ i{\isacharparenright}{\kern0pt}\isanewline
\ \ \isanewline
\ \ \isacommand{show}\isamarkupfalse%
\ {\isacharquery}{\kern0pt}thesis\isanewline
\ \ \ \ \isacommand{apply}\isamarkupfalse%
\ {\isacharparenleft}{\kern0pt}subst\ a\ {\isacharbrackleft}{\kern0pt}symmetric{\isacharbrackright}{\kern0pt}{\isacharparenright}{\kern0pt}\isanewline
\ \ \ \ \isacommand{apply}\isamarkupfalse%
\ {\isacharparenleft}{\kern0pt}rule\ indep{\isacharunderscore}{\kern0pt}vars{\isacharunderscore}{\kern0pt}distr{\isacharparenright}{\kern0pt}\isanewline
\ \ \ \ \isacommand{apply}\isamarkupfalse%
\ {\isacharparenleft}{\kern0pt}simp\ add{\isacharcolon}{\kern0pt}measurable{\isacharunderscore}{\kern0pt}def\ sets{\isacharunderscore}{\kern0pt}restrict{\isacharunderscore}{\kern0pt}space{\isacharparenright}{\kern0pt}\ \isanewline
\ \ \ \ \ \ \ \isacommand{apply}\isamarkupfalse%
\ blast\isanewline
\ \ \ \ \ \ \isacommand{apply}\isamarkupfalse%
\ simp\isanewline
\ \ \ \ \isacommand{apply}\isamarkupfalse%
\ simp\isanewline
\ \ \ \ \isacommand{apply}\isamarkupfalse%
\ {\isacharparenleft}{\kern0pt}subst\ prob{\isacharunderscore}{\kern0pt}space{\isachardot}{\kern0pt}indep{\isacharunderscore}{\kern0pt}vars{\isacharunderscore}{\kern0pt}def{\isadigit{2}}{\isacharparenright}{\kern0pt}\isanewline
\ \ \ \ \ \ \isacommand{apply}\isamarkupfalse%
\ {\isacharparenleft}{\kern0pt}metis\ b{\isacharparenright}{\kern0pt}\isanewline
\ \ \ \ \ \isacommand{apply}\isamarkupfalse%
\ {\isacharparenleft}{\kern0pt}simp\ add{\isacharcolon}{\kern0pt}measurable{\isacharunderscore}{\kern0pt}def\ sets{\isacharunderscore}{\kern0pt}restrict{\isacharunderscore}{\kern0pt}space\ range{\isacharunderscore}{\kern0pt}inter{\isacharparenright}{\kern0pt}\isanewline
\ \ \ \ \isacommand{by}\isamarkupfalse%
\ {\isacharparenleft}{\kern0pt}metis\ c{\isacharcomma}{\kern0pt}\ metis\ b{\isacharparenright}{\kern0pt}\isanewline
\isacommand{qed}\isamarkupfalse%
%
\endisatagproof
{\isafoldproof}%
%
\isadelimproof
\isanewline
%
\endisadelimproof
\isanewline
\isacommand{lemma}\isamarkupfalse%
\ indep{\isacharunderscore}{\kern0pt}vars{\isacharunderscore}{\kern0pt}restrict{\isacharcolon}{\kern0pt}\isanewline
\ \ \isakeyword{fixes}\ M\ {\isacharcolon}{\kern0pt}{\isacharcolon}{\kern0pt}\ {\isachardoublequoteopen}{\isacharprime}{\kern0pt}a\ {\isasymRightarrow}\ {\isacharprime}{\kern0pt}b\ pmf{\isachardoublequoteclose}\isanewline
\ \ \isakeyword{fixes}\ J\ {\isacharcolon}{\kern0pt}{\isacharcolon}{\kern0pt}\ {\isachardoublequoteopen}{\isacharprime}{\kern0pt}c\ set{\isachardoublequoteclose}\isanewline
\ \ \isakeyword{assumes}\ {\isachardoublequoteopen}disjoint{\isacharunderscore}{\kern0pt}family{\isacharunderscore}{\kern0pt}on\ f\ J{\isachardoublequoteclose}\isanewline
\ \ \isakeyword{assumes}\ {\isachardoublequoteopen}J\ {\isasymnoteq}\ {\isacharbraceleft}{\kern0pt}{\isacharbraceright}{\kern0pt}{\isachardoublequoteclose}\isanewline
\ \ \isakeyword{assumes}\ {\isachardoublequoteopen}{\isasymAnd}i{\isachardot}{\kern0pt}\ i\ {\isasymin}\ J\ {\isasymLongrightarrow}\ f\ i\ {\isasymsubseteq}\ I{\isachardoublequoteclose}\isanewline
\ \ \isakeyword{assumes}\ {\isachardoublequoteopen}finite\ I{\isachardoublequoteclose}\isanewline
\ \ \isakeyword{shows}\ {\isachardoublequoteopen}prob{\isacharunderscore}{\kern0pt}space{\isachardot}{\kern0pt}indep{\isacharunderscore}{\kern0pt}vars\ {\isacharparenleft}{\kern0pt}measure{\isacharunderscore}{\kern0pt}pmf\ {\isacharparenleft}{\kern0pt}prod{\isacharunderscore}{\kern0pt}pmf\ I\ M{\isacharparenright}{\kern0pt}{\isacharparenright}{\kern0pt}\ {\isacharparenleft}{\kern0pt}{\isasymlambda}i{\isachardot}{\kern0pt}\ measure{\isacharunderscore}{\kern0pt}pmf\ {\isacharparenleft}{\kern0pt}prod{\isacharunderscore}{\kern0pt}pmf\ {\isacharparenleft}{\kern0pt}f\ i{\isacharparenright}{\kern0pt}\ M{\isacharparenright}{\kern0pt}{\isacharparenright}{\kern0pt}\ {\isacharparenleft}{\kern0pt}{\isasymlambda}i\ {\isasymomega}{\isachardot}{\kern0pt}\ restrict\ {\isasymomega}\ {\isacharparenleft}{\kern0pt}f\ i{\isacharparenright}{\kern0pt}{\isacharparenright}{\kern0pt}\ J{\isachardoublequoteclose}\isanewline
%
\isadelimproof
%
\endisadelimproof
%
\isatagproof
\isacommand{proof}\isamarkupfalse%
\ {\isacharparenleft}{\kern0pt}rule\ indep{\isacharunderscore}{\kern0pt}vars{\isacharunderscore}{\kern0pt}pmf{\isacharbrackleft}{\kern0pt}simplified{\isacharbrackright}{\kern0pt}{\isacharparenright}{\kern0pt}\isanewline
\ \ \isacommand{fix}\isamarkupfalse%
\ a\ {\isacharcolon}{\kern0pt}{\isacharcolon}{\kern0pt}\ {\isachardoublequoteopen}{\isacharprime}{\kern0pt}c\ {\isasymRightarrow}\ {\isacharprime}{\kern0pt}a\ {\isasymRightarrow}\ {\isacharprime}{\kern0pt}b{\isachardoublequoteclose}\isanewline
\ \ \isacommand{fix}\isamarkupfalse%
\ J{\isacharprime}{\kern0pt}\isanewline
\ \ \isacommand{assume}\isamarkupfalse%
\ e{\isacharcolon}{\kern0pt}{\isachardoublequoteopen}J{\isacharprime}{\kern0pt}\ {\isasymsubseteq}\ J{\isachardoublequoteclose}\isanewline
\ \ \isacommand{assume}\isamarkupfalse%
\ c{\isacharcolon}{\kern0pt}{\isachardoublequoteopen}finite\ J{\isacharprime}{\kern0pt}{\isachardoublequoteclose}\isanewline
\ \ \isacommand{show}\isamarkupfalse%
\ {\isachardoublequoteopen}measure{\isacharunderscore}{\kern0pt}pmf{\isachardot}{\kern0pt}prob\ {\isacharparenleft}{\kern0pt}prod{\isacharunderscore}{\kern0pt}pmf\ I\ M{\isacharparenright}{\kern0pt}\ {\isacharbraceleft}{\kern0pt}{\isasymomega}{\isachardot}{\kern0pt}\ {\isasymforall}i{\isasymin}J{\isacharprime}{\kern0pt}{\isachardot}{\kern0pt}\ restrict\ {\isasymomega}\ {\isacharparenleft}{\kern0pt}f\ i{\isacharparenright}{\kern0pt}\ {\isacharequal}{\kern0pt}\ a\ i{\isacharbraceright}{\kern0pt}\ {\isacharequal}{\kern0pt}\isanewline
\ \ \ \ \ \ \ {\isacharparenleft}{\kern0pt}{\isasymProd}i{\isasymin}J{\isacharprime}{\kern0pt}{\isachardot}{\kern0pt}\ measure{\isacharunderscore}{\kern0pt}pmf{\isachardot}{\kern0pt}prob\ {\isacharparenleft}{\kern0pt}prod{\isacharunderscore}{\kern0pt}pmf\ I\ M{\isacharparenright}{\kern0pt}\ {\isacharbraceleft}{\kern0pt}{\isasymomega}{\isachardot}{\kern0pt}\ restrict\ {\isasymomega}\ {\isacharparenleft}{\kern0pt}f\ i{\isacharparenright}{\kern0pt}\ {\isacharequal}{\kern0pt}\ a\ i{\isacharbraceright}{\kern0pt}{\isacharparenright}{\kern0pt}{\isachardoublequoteclose}\isanewline
\ \ \isacommand{proof}\isamarkupfalse%
\ {\isacharparenleft}{\kern0pt}cases\ {\isachardoublequoteopen}{\isasymforall}j\ {\isasymin}\ J{\isacharprime}{\kern0pt}{\isachardot}{\kern0pt}\ a\ j\ {\isasymin}\ extensional\ {\isacharparenleft}{\kern0pt}f\ j{\isacharparenright}{\kern0pt}{\isachardoublequoteclose}{\isacharparenright}{\kern0pt}\isanewline
\ \ \ \ \isacommand{case}\isamarkupfalse%
\ True\isanewline
\ \ \ \ \isacommand{define}\isamarkupfalse%
\ b\ \isakeyword{where}\ {\isachardoublequoteopen}b\ {\isacharequal}{\kern0pt}\ {\isacharparenleft}{\kern0pt}{\isasymlambda}i{\isachardot}{\kern0pt}\ if\ i\ {\isasymin}\ {\isacharparenleft}{\kern0pt}{\isasymUnion}\ {\isacharparenleft}{\kern0pt}f\ {\isacharbackquote}{\kern0pt}\ J{\isacharprime}{\kern0pt}{\isacharparenright}{\kern0pt}{\isacharparenright}{\kern0pt}\ then\ a\ {\isacharparenleft}{\kern0pt}THE\ j{\isachardot}{\kern0pt}\ i\ {\isasymin}\ f\ j\ {\isasymand}\ j\ {\isasymin}\ J{\isacharprime}{\kern0pt}{\isacharparenright}{\kern0pt}\ i\ else\ undefined{\isacharparenright}{\kern0pt}{\isachardoublequoteclose}\ \isanewline
\ \ \ \ \isacommand{have}\isamarkupfalse%
\ b{\isacharunderscore}{\kern0pt}def{\isacharcolon}{\kern0pt}{\isachardoublequoteopen}{\isasymAnd}i{\isachardot}{\kern0pt}\ i\ {\isasymin}\ J{\isacharprime}{\kern0pt}\ {\isasymLongrightarrow}\ a\ i\ {\isacharequal}{\kern0pt}\ restrict\ b\ {\isacharparenleft}{\kern0pt}f\ i{\isacharparenright}{\kern0pt}{\isachardoublequoteclose}\isanewline
\ \ \ \ \isacommand{proof}\isamarkupfalse%
\ {\isacharminus}{\kern0pt}\isanewline
\ \ \ \ \ \ \isacommand{fix}\isamarkupfalse%
\ i\ \isanewline
\ \ \ \ \ \ \isacommand{assume}\isamarkupfalse%
\ b{\isacharunderscore}{\kern0pt}def{\isacharunderscore}{\kern0pt}{\isadigit{1}}{\isacharcolon}{\kern0pt}{\isachardoublequoteopen}i\ {\isasymin}\ J{\isacharprime}{\kern0pt}{\isachardoublequoteclose}\isanewline
\ \ \ \ \ \ \isacommand{have}\isamarkupfalse%
\ b{\isacharunderscore}{\kern0pt}def{\isacharunderscore}{\kern0pt}{\isadigit{2}}{\isacharcolon}{\kern0pt}\ {\isachardoublequoteopen}{\isasymAnd}x{\isachardot}{\kern0pt}\ x\ {\isasymin}\ f\ i\ {\isasymLongrightarrow}\ i\ {\isacharequal}{\kern0pt}\ {\isacharparenleft}{\kern0pt}THE\ j{\isachardot}{\kern0pt}\ x\ {\isasymin}\ f\ j\ {\isasymand}\ j\ {\isasymin}\ J{\isacharprime}{\kern0pt}{\isacharparenright}{\kern0pt}{\isachardoublequoteclose}\isanewline
\ \ \ \ \ \ \ \ \isacommand{using}\isamarkupfalse%
\ disjoint{\isacharunderscore}{\kern0pt}family{\isacharunderscore}{\kern0pt}on{\isacharunderscore}{\kern0pt}mono{\isacharbrackleft}{\kern0pt}OF\ e\ assms{\isacharparenleft}{\kern0pt}{\isadigit{1}}{\isacharparenright}{\kern0pt}{\isacharbrackright}{\kern0pt}\ b{\isacharunderscore}{\kern0pt}def{\isacharunderscore}{\kern0pt}{\isadigit{1}}\ \isanewline
\ \ \ \ \ \ \ \ \isacommand{apply}\isamarkupfalse%
\ {\isacharparenleft}{\kern0pt}simp\ add{\isacharcolon}{\kern0pt}disjoint{\isacharunderscore}{\kern0pt}family{\isacharunderscore}{\kern0pt}on{\isacharunderscore}{\kern0pt}def{\isacharparenright}{\kern0pt}\ \isanewline
\ \ \ \ \ \ \ \ \isacommand{by}\isamarkupfalse%
\ {\isacharparenleft}{\kern0pt}metis\ {\isacharparenleft}{\kern0pt}mono{\isacharunderscore}{\kern0pt}tags{\isacharcomma}{\kern0pt}\ lifting{\isacharparenright}{\kern0pt}\ IntI\ empty{\isacharunderscore}{\kern0pt}iff\ the{\isacharunderscore}{\kern0pt}equality{\isacharparenright}{\kern0pt}\isanewline
\ \ \ \ \ \ \isacommand{show}\isamarkupfalse%
\ {\isachardoublequoteopen}a\ i\ {\isacharequal}{\kern0pt}\ restrict\ b\ {\isacharparenleft}{\kern0pt}f\ i{\isacharparenright}{\kern0pt}{\isachardoublequoteclose}\isanewline
\ \ \ \ \ \ \ \ \isacommand{apply}\isamarkupfalse%
\ {\isacharparenleft}{\kern0pt}rule\ extensionalityI{\isacharbrackleft}{\kern0pt}\isakeyword{where}\ A\ {\isacharequal}{\kern0pt}{\isachardoublequoteopen}f\ i{\isachardoublequoteclose}{\isacharbrackright}{\kern0pt}{\isacharparenright}{\kern0pt}\ \isacommand{using}\isamarkupfalse%
\ b{\isacharunderscore}{\kern0pt}def{\isacharunderscore}{\kern0pt}{\isadigit{1}}\ True\ \isacommand{apply}\isamarkupfalse%
\ blast\isanewline
\ \ \ \ \ \ \ \ \ \isacommand{apply}\isamarkupfalse%
\ {\isacharparenleft}{\kern0pt}rule\ restrict{\isacharunderscore}{\kern0pt}extensional{\isacharparenright}{\kern0pt}\isanewline
\ \ \ \ \ \ \ \ \isacommand{apply}\isamarkupfalse%
\ {\isacharparenleft}{\kern0pt}simp\ add{\isacharcolon}{\kern0pt}restrict{\isacharunderscore}{\kern0pt}apply{\isacharprime}{\kern0pt}\ b{\isacharunderscore}{\kern0pt}def\ b{\isacharunderscore}{\kern0pt}def{\isacharunderscore}{\kern0pt}{\isadigit{2}}{\isacharbrackleft}{\kern0pt}symmetric{\isacharbrackright}{\kern0pt}{\isacharparenright}{\kern0pt}\isanewline
\ \ \ \ \ \ \ \ \isacommand{using}\isamarkupfalse%
\ b{\isacharunderscore}{\kern0pt}def{\isacharunderscore}{\kern0pt}{\isadigit{1}}\ \isacommand{by}\isamarkupfalse%
\ force\isanewline
\ \ \ \ \isacommand{qed}\isamarkupfalse%
\ \ \ \ \isanewline
\ \ \ \ \isacommand{have}\isamarkupfalse%
\ a{\isacharcolon}{\kern0pt}{\isachardoublequoteopen}{\isacharbraceleft}{\kern0pt}{\isasymomega}{\isachardot}{\kern0pt}\ {\isasymforall}i{\isasymin}J{\isacharprime}{\kern0pt}{\isachardot}{\kern0pt}\ restrict\ {\isasymomega}\ {\isacharparenleft}{\kern0pt}f\ i{\isacharparenright}{\kern0pt}\ {\isacharequal}{\kern0pt}\ a\ i{\isacharbraceright}{\kern0pt}\ {\isacharequal}{\kern0pt}\ Pi\ {\isacharparenleft}{\kern0pt}{\isasymUnion}\ {\isacharparenleft}{\kern0pt}f\ {\isacharbackquote}{\kern0pt}\ J{\isacharprime}{\kern0pt}{\isacharparenright}{\kern0pt}{\isacharparenright}{\kern0pt}\ {\isacharparenleft}{\kern0pt}{\isasymlambda}i{\isachardot}{\kern0pt}\ {\isacharbraceleft}{\kern0pt}b\ i{\isacharbraceright}{\kern0pt}{\isacharparenright}{\kern0pt}{\isachardoublequoteclose}\isanewline
\ \ \ \ \ \ \isacommand{apply}\isamarkupfalse%
\ {\isacharparenleft}{\kern0pt}simp\ add{\isacharcolon}{\kern0pt}b{\isacharunderscore}{\kern0pt}def{\isacharparenright}{\kern0pt}\isanewline
\ \ \ \ \ \ \isacommand{apply}\isamarkupfalse%
\ {\isacharparenleft}{\kern0pt}rule\ order{\isacharunderscore}{\kern0pt}antisym{\isacharparenright}{\kern0pt}\isanewline
\ \ \ \ \ \ \ \isacommand{apply}\isamarkupfalse%
\ {\isacharparenleft}{\kern0pt}rule\ subsetI{\isacharcomma}{\kern0pt}\ simp\ add{\isacharcolon}{\kern0pt}Pi{\isacharunderscore}{\kern0pt}def{\isacharcomma}{\kern0pt}\ metis\ restrict{\isacharunderscore}{\kern0pt}apply{\isacharprime}{\kern0pt}{\isacharparenright}{\kern0pt}\isanewline
\ \ \ \ \ \ \isacommand{by}\isamarkupfalse%
\ {\isacharparenleft}{\kern0pt}rule\ subsetI{\isacharcomma}{\kern0pt}\ simp\ add{\isacharcolon}{\kern0pt}Pi{\isacharunderscore}{\kern0pt}def{\isacharcomma}{\kern0pt}\ meson\ assms{\isacharparenleft}{\kern0pt}{\isadigit{3}}{\isacharparenright}{\kern0pt}\ e\ restrict{\isacharunderscore}{\kern0pt}ext\ singletonD\ subsetD{\isacharparenright}{\kern0pt}\isanewline
\ \ \ \ \isacommand{have}\isamarkupfalse%
\ b{\isacharcolon}{\kern0pt}{\isachardoublequoteopen}{\isasymAnd}i{\isachardot}{\kern0pt}\ i\ {\isasymin}\ J{\isacharprime}{\kern0pt}\ {\isasymLongrightarrow}\ {\isacharbraceleft}{\kern0pt}{\isasymomega}{\isachardot}{\kern0pt}\ restrict\ {\isasymomega}\ {\isacharparenleft}{\kern0pt}f\ i{\isacharparenright}{\kern0pt}\ {\isacharequal}{\kern0pt}\ a\ i{\isacharbraceright}{\kern0pt}\ {\isacharequal}{\kern0pt}\ Pi\ {\isacharparenleft}{\kern0pt}f\ i{\isacharparenright}{\kern0pt}\ {\isacharparenleft}{\kern0pt}{\isasymlambda}i{\isachardot}{\kern0pt}\ {\isacharbraceleft}{\kern0pt}b\ i{\isacharbraceright}{\kern0pt}{\isacharparenright}{\kern0pt}{\isachardoublequoteclose}\isanewline
\ \ \ \ \ \ \isacommand{apply}\isamarkupfalse%
\ {\isacharparenleft}{\kern0pt}simp\ add{\isacharcolon}{\kern0pt}b{\isacharunderscore}{\kern0pt}def{\isacharparenright}{\kern0pt}\isanewline
\ \ \ \ \ \ \isacommand{apply}\isamarkupfalse%
\ {\isacharparenleft}{\kern0pt}rule\ order{\isacharunderscore}{\kern0pt}antisym{\isacharparenright}{\kern0pt}\isanewline
\ \ \ \ \ \ \ \isacommand{apply}\isamarkupfalse%
\ {\isacharparenleft}{\kern0pt}rule\ subsetI{\isacharcomma}{\kern0pt}\ simp\ add{\isacharcolon}{\kern0pt}Pi{\isacharunderscore}{\kern0pt}def{\isacharcomma}{\kern0pt}\ metis\ restrict{\isacharunderscore}{\kern0pt}apply{\isacharprime}{\kern0pt}{\isacharparenright}{\kern0pt}\isanewline
\ \ \ \ \ \ \isacommand{by}\isamarkupfalse%
\ {\isacharparenleft}{\kern0pt}rule\ subsetI{\isacharcomma}{\kern0pt}\ simp\ add{\isacharcolon}{\kern0pt}Pi{\isacharunderscore}{\kern0pt}def{\isacharcomma}{\kern0pt}\ meson\ assms{\isacharparenleft}{\kern0pt}{\isadigit{3}}{\isacharparenright}{\kern0pt}\ e\ restrict{\isacharunderscore}{\kern0pt}ext\ singletonD\ subsetD{\isacharparenright}{\kern0pt}\isanewline
\ \ \ \ \isacommand{show}\isamarkupfalse%
\ {\isacharquery}{\kern0pt}thesis\isanewline
\ \ \ \ \ \ \isacommand{apply}\isamarkupfalse%
\ {\isacharparenleft}{\kern0pt}simp\ add{\isacharcolon}{\kern0pt}\ a\ b{\isacharparenright}{\kern0pt}\isanewline
\ \ \ \ \ \ \isacommand{apply}\isamarkupfalse%
\ {\isacharparenleft}{\kern0pt}subst\ prob{\isacharunderscore}{\kern0pt}prod{\isacharunderscore}{\kern0pt}pmf{\isacharprime}{\kern0pt}{\isacharbrackleft}{\kern0pt}OF\ assms{\isacharparenleft}{\kern0pt}{\isadigit{4}}{\isacharparenright}{\kern0pt}{\isacharbrackright}{\kern0pt}{\isacharcomma}{\kern0pt}\ meson\ UN{\isacharunderscore}{\kern0pt}least\ e\ in{\isacharunderscore}{\kern0pt}mono\ assms{\isacharparenleft}{\kern0pt}{\isadigit{3}}{\isacharparenright}{\kern0pt}{\isacharparenright}{\kern0pt}\isanewline
\ \ \ \ \ \ \isacommand{apply}\isamarkupfalse%
\ {\isacharparenleft}{\kern0pt}subst\ prod{\isachardot}{\kern0pt}UNION{\isacharunderscore}{\kern0pt}disjoint{\isacharcomma}{\kern0pt}\ metis\ c{\isacharparenright}{\kern0pt}\isanewline
\ \ \ \ \ \ \ \ \isacommand{apply}\isamarkupfalse%
\ {\isacharparenleft}{\kern0pt}metis\ in{\isacharunderscore}{\kern0pt}mono\ \ e\ assms{\isacharparenleft}{\kern0pt}{\isadigit{3}}{\isacharparenright}{\kern0pt}\ assms{\isacharparenleft}{\kern0pt}{\isadigit{4}}{\isacharparenright}{\kern0pt}\ finite{\isacharunderscore}{\kern0pt}subset{\isacharparenright}{\kern0pt}\isanewline
\ \ \ \ \ \ \ \isacommand{apply}\isamarkupfalse%
\ {\isacharparenleft}{\kern0pt}metis\ e\ \ disjoint{\isacharunderscore}{\kern0pt}family{\isacharunderscore}{\kern0pt}on{\isacharunderscore}{\kern0pt}def\ assms{\isacharparenleft}{\kern0pt}{\isadigit{1}}{\isacharparenright}{\kern0pt}\ subset{\isacharunderscore}{\kern0pt}eq{\isacharparenright}{\kern0pt}\isanewline
\ \ \ \ \ \ \isacommand{apply}\isamarkupfalse%
\ {\isacharparenleft}{\kern0pt}rule\ prod{\isachardot}{\kern0pt}cong{\isacharcomma}{\kern0pt}\ simp{\isacharparenright}{\kern0pt}\isanewline
\ \ \ \ \ \ \isacommand{apply}\isamarkupfalse%
\ {\isacharparenleft}{\kern0pt}subst\ prob{\isacharunderscore}{\kern0pt}prod{\isacharunderscore}{\kern0pt}pmf{\isacharprime}{\kern0pt}{\isacharbrackleft}{\kern0pt}OF\ assms{\isacharparenleft}{\kern0pt}{\isadigit{4}}{\isacharparenright}{\kern0pt}{\isacharbrackright}{\kern0pt}{\isacharparenright}{\kern0pt}\ \isacommand{using}\isamarkupfalse%
\ e\ assms{\isacharparenleft}{\kern0pt}{\isadigit{3}}{\isacharparenright}{\kern0pt}\ \isacommand{apply}\isamarkupfalse%
\ blast\isanewline
\ \ \ \ \ \ \isacommand{by}\isamarkupfalse%
\ simp\isanewline
\ \ \isacommand{next}\isamarkupfalse%
\isanewline
\ \ \ \ \isacommand{case}\isamarkupfalse%
\ False\isanewline
\ \ \ \ \isacommand{then}\isamarkupfalse%
\ \isacommand{obtain}\isamarkupfalse%
\ j\ \isakeyword{where}\ j{\isacharunderscore}{\kern0pt}def{\isacharcolon}{\kern0pt}\ {\isachardoublequoteopen}j\ {\isasymin}\ J{\isacharprime}{\kern0pt}{\isachardoublequoteclose}\ \isakeyword{and}\ {\isachardoublequoteopen}a\ j\ {\isasymnotin}\ extensional\ {\isacharparenleft}{\kern0pt}f\ j{\isacharparenright}{\kern0pt}{\isachardoublequoteclose}\ \isacommand{by}\isamarkupfalse%
\ blast\isanewline
\ \ \ \ \isacommand{hence}\isamarkupfalse%
\ {\isachardoublequoteopen}{\isasymAnd}{\isasymomega}{\isachardot}{\kern0pt}\ restrict\ {\isasymomega}\ {\isacharparenleft}{\kern0pt}f\ j{\isacharparenright}{\kern0pt}\ {\isasymnoteq}\ a\ j{\isachardoublequoteclose}\ \isacommand{by}\isamarkupfalse%
\ {\isacharparenleft}{\kern0pt}metis\ restrict{\isacharunderscore}{\kern0pt}extensional{\isacharparenright}{\kern0pt}\isanewline
\ \ \ \ \isacommand{then}\isamarkupfalse%
\ \isacommand{show}\isamarkupfalse%
\ {\isacharquery}{\kern0pt}thesis\ \isanewline
\ \ \ \ \ \ \isacommand{by}\isamarkupfalse%
\ {\isacharparenleft}{\kern0pt}metis\ {\isacharparenleft}{\kern0pt}mono{\isacharunderscore}{\kern0pt}tags{\isacharcomma}{\kern0pt}\ lifting{\isacharparenright}{\kern0pt}\ Collect{\isacharunderscore}{\kern0pt}empty{\isacharunderscore}{\kern0pt}eq\ j{\isacharunderscore}{\kern0pt}def\ c\ measure{\isacharunderscore}{\kern0pt}empty\ prod{\isacharunderscore}{\kern0pt}zero{\isacharunderscore}{\kern0pt}iff{\isacharparenright}{\kern0pt}\isanewline
\ \ \isacommand{qed}\isamarkupfalse%
\isanewline
\isacommand{qed}\isamarkupfalse%
%
\endisatagproof
{\isafoldproof}%
%
\isadelimproof
\ \ \isanewline
%
\endisadelimproof
\isanewline
\isacommand{lemma}\isamarkupfalse%
\ indep{\isacharunderscore}{\kern0pt}vars{\isacharunderscore}{\kern0pt}restrict{\isacharunderscore}{\kern0pt}intro{\isacharcolon}{\kern0pt}\isanewline
\ \ \isakeyword{fixes}\ M\ {\isacharcolon}{\kern0pt}{\isacharcolon}{\kern0pt}\ {\isachardoublequoteopen}{\isacharprime}{\kern0pt}a\ {\isasymRightarrow}\ {\isacharprime}{\kern0pt}b\ pmf{\isachardoublequoteclose}\isanewline
\ \ \isakeyword{fixes}\ J\ {\isacharcolon}{\kern0pt}{\isacharcolon}{\kern0pt}\ {\isachardoublequoteopen}{\isacharprime}{\kern0pt}c\ set{\isachardoublequoteclose}\isanewline
\ \ \isakeyword{assumes}\ {\isachardoublequoteopen}{\isasymAnd}{\isasymomega}\ i{\isachardot}{\kern0pt}\ i\ {\isasymin}\ J\ {\isasymLongrightarrow}\ \ X\ i\ {\isasymomega}\ {\isacharequal}{\kern0pt}\ X\ i\ {\isacharparenleft}{\kern0pt}restrict\ {\isasymomega}\ {\isacharparenleft}{\kern0pt}f\ i{\isacharparenright}{\kern0pt}{\isacharparenright}{\kern0pt}{\isachardoublequoteclose}\isanewline
\ \ \isakeyword{assumes}\ {\isachardoublequoteopen}disjoint{\isacharunderscore}{\kern0pt}family{\isacharunderscore}{\kern0pt}on\ f\ J{\isachardoublequoteclose}\isanewline
\ \ \isakeyword{assumes}\ {\isachardoublequoteopen}J\ {\isasymnoteq}\ {\isacharbraceleft}{\kern0pt}{\isacharbraceright}{\kern0pt}{\isachardoublequoteclose}\isanewline
\ \ \isakeyword{assumes}\ {\isachardoublequoteopen}{\isasymAnd}i{\isachardot}{\kern0pt}\ i\ {\isasymin}\ J\ {\isasymLongrightarrow}\ f\ i\ {\isasymsubseteq}\ I{\isachardoublequoteclose}\isanewline
\ \ \isakeyword{assumes}\ {\isachardoublequoteopen}finite\ I{\isachardoublequoteclose}\isanewline
\ \ \isakeyword{assumes}\ {\isachardoublequoteopen}{\isasymAnd}{\isasymomega}\ i{\isachardot}{\kern0pt}\ i\ {\isasymin}\ J\ {\isasymLongrightarrow}\ \ X\ i\ {\isasymomega}\ {\isasymin}\ space\ {\isacharparenleft}{\kern0pt}M{\isacharprime}{\kern0pt}\ i{\isacharparenright}{\kern0pt}{\isachardoublequoteclose}\isanewline
\ \ \isakeyword{shows}\ {\isachardoublequoteopen}prob{\isacharunderscore}{\kern0pt}space{\isachardot}{\kern0pt}indep{\isacharunderscore}{\kern0pt}vars\ {\isacharparenleft}{\kern0pt}measure{\isacharunderscore}{\kern0pt}pmf\ {\isacharparenleft}{\kern0pt}prod{\isacharunderscore}{\kern0pt}pmf\ I\ M{\isacharparenright}{\kern0pt}{\isacharparenright}{\kern0pt}\ M{\isacharprime}{\kern0pt}\ {\isacharparenleft}{\kern0pt}{\isasymlambda}i\ {\isasymomega}{\isachardot}{\kern0pt}\ X\ i\ {\isasymomega}{\isacharparenright}{\kern0pt}\ J{\isachardoublequoteclose}\isanewline
%
\isadelimproof
%
\endisadelimproof
%
\isatagproof
\isacommand{proof}\isamarkupfalse%
\ {\isacharminus}{\kern0pt}\isanewline
\ \ \isacommand{have}\isamarkupfalse%
\ {\isachardoublequoteopen}prob{\isacharunderscore}{\kern0pt}space{\isachardot}{\kern0pt}indep{\isacharunderscore}{\kern0pt}vars\ {\isacharparenleft}{\kern0pt}measure{\isacharunderscore}{\kern0pt}pmf\ {\isacharparenleft}{\kern0pt}prod{\isacharunderscore}{\kern0pt}pmf\ I\ M{\isacharparenright}{\kern0pt}{\isacharparenright}{\kern0pt}\ M{\isacharprime}{\kern0pt}\ {\isacharparenleft}{\kern0pt}{\isasymlambda}i\ {\isasymomega}{\isachardot}{\kern0pt}\ X\ i\ {\isacharparenleft}{\kern0pt}restrict\ {\isasymomega}\ {\isacharparenleft}{\kern0pt}f\ i{\isacharparenright}{\kern0pt}{\isacharparenright}{\kern0pt}{\isacharparenright}{\kern0pt}\ J{\isachardoublequoteclose}\ {\isacharparenleft}{\kern0pt}\isakeyword{is}\ {\isacharquery}{\kern0pt}A{\isacharparenright}{\kern0pt}\isanewline
\ \ \ \ \isacommand{apply}\isamarkupfalse%
\ {\isacharparenleft}{\kern0pt}rule\ prob{\isacharunderscore}{\kern0pt}space{\isachardot}{\kern0pt}indep{\isacharunderscore}{\kern0pt}vars{\isacharunderscore}{\kern0pt}compose{\isadigit{2}}{\isacharbrackleft}{\kern0pt}\isakeyword{where}\ X{\isacharequal}{\kern0pt}{\isachardoublequoteopen}{\isasymlambda}i\ {\isasymomega}{\isachardot}{\kern0pt}\ restrict\ {\isasymomega}\ {\isacharparenleft}{\kern0pt}f\ i{\isacharparenright}{\kern0pt}{\isachardoublequoteclose}{\isacharbrackright}{\kern0pt}{\isacharparenright}{\kern0pt}\isanewline
\ \ \ \ \ \ \isacommand{apply}\isamarkupfalse%
\ {\isacharparenleft}{\kern0pt}metis\ prob{\isacharunderscore}{\kern0pt}space{\isacharunderscore}{\kern0pt}measure{\isacharunderscore}{\kern0pt}pmf{\isacharparenright}{\kern0pt}\isanewline
\ \ \ \ \ \isacommand{apply}\isamarkupfalse%
\ {\isacharparenleft}{\kern0pt}rule\ indep{\isacharunderscore}{\kern0pt}vars{\isacharunderscore}{\kern0pt}restrict{\isacharcomma}{\kern0pt}\ metis\ assms{\isacharparenleft}{\kern0pt}{\isadigit{2}}{\isacharparenright}{\kern0pt}{\isacharcomma}{\kern0pt}\ metis\ assms{\isacharparenleft}{\kern0pt}{\isadigit{3}}{\isacharparenright}{\kern0pt}{\isacharcomma}{\kern0pt}\ metis\ assms{\isacharparenleft}{\kern0pt}{\isadigit{4}}{\isacharparenright}{\kern0pt}{\isacharcomma}{\kern0pt}\ metis\ assms{\isacharparenleft}{\kern0pt}{\isadigit{5}}{\isacharparenright}{\kern0pt}{\isacharparenright}{\kern0pt}\isanewline
\ \ \ \ \isacommand{apply}\isamarkupfalse%
\ simp\ \isacommand{using}\isamarkupfalse%
\ assms{\isacharparenleft}{\kern0pt}{\isadigit{6}}{\isacharparenright}{\kern0pt}\ \isacommand{by}\isamarkupfalse%
\ blast\isanewline
\ \ \isacommand{moreover}\isamarkupfalse%
\ \isacommand{have}\isamarkupfalse%
\ {\isachardoublequoteopen}{\isacharquery}{\kern0pt}A\ {\isacharequal}{\kern0pt}\ {\isacharquery}{\kern0pt}thesis{\isachardoublequoteclose}\isanewline
\ \ \ \ \isacommand{apply}\isamarkupfalse%
\ {\isacharparenleft}{\kern0pt}rule\ prob{\isacharunderscore}{\kern0pt}space{\isachardot}{\kern0pt}indep{\isacharunderscore}{\kern0pt}vars{\isacharunderscore}{\kern0pt}cong{\isacharcomma}{\kern0pt}\ metis\ prob{\isacharunderscore}{\kern0pt}space{\isacharunderscore}{\kern0pt}measure{\isacharunderscore}{\kern0pt}pmf{\isacharcomma}{\kern0pt}\ simp{\isacharparenright}{\kern0pt}\isanewline
\ \ \ \ \isacommand{by}\isamarkupfalse%
\ {\isacharparenleft}{\kern0pt}rule\ ext{\isacharcomma}{\kern0pt}\ metis\ assms{\isacharparenleft}{\kern0pt}{\isadigit{1}}{\isacharparenright}{\kern0pt}{\isacharcomma}{\kern0pt}\ simp{\isacharparenright}{\kern0pt}\isanewline
\ \ \isacommand{ultimately}\isamarkupfalse%
\ \isacommand{show}\isamarkupfalse%
\ {\isacharquery}{\kern0pt}thesis\ \isacommand{by}\isamarkupfalse%
\ blast\isanewline
\isacommand{qed}\isamarkupfalse%
%
\endisatagproof
{\isafoldproof}%
%
\isadelimproof
\isanewline
%
\endisadelimproof
\isanewline
\isacommand{lemma}\isamarkupfalse%
\ has{\isacharunderscore}{\kern0pt}bochner{\isacharunderscore}{\kern0pt}integral{\isacharunderscore}{\kern0pt}prod{\isacharunderscore}{\kern0pt}pmfI{\isacharcolon}{\kern0pt}\isanewline
\ \ \isakeyword{fixes}\ f\ {\isacharcolon}{\kern0pt}{\isacharcolon}{\kern0pt}\ {\isachardoublequoteopen}{\isacharprime}{\kern0pt}a\ {\isasymRightarrow}\ {\isacharprime}{\kern0pt}b\ {\isasymRightarrow}\ {\isacharparenleft}{\kern0pt}{\isacharprime}{\kern0pt}c\ {\isacharcolon}{\kern0pt}{\isacharcolon}{\kern0pt}\ {\isacharbraceleft}{\kern0pt}second{\isacharunderscore}{\kern0pt}countable{\isacharunderscore}{\kern0pt}topology{\isacharcomma}{\kern0pt}banach{\isacharcomma}{\kern0pt}real{\isacharunderscore}{\kern0pt}normed{\isacharunderscore}{\kern0pt}field{\isacharbraceright}{\kern0pt}{\isacharparenright}{\kern0pt}{\isachardoublequoteclose}\isanewline
\ \ \isakeyword{assumes}\ {\isachardoublequoteopen}finite\ I{\isachardoublequoteclose}\isanewline
\ \ \isakeyword{assumes}\ {\isachardoublequoteopen}{\isasymAnd}i{\isachardot}{\kern0pt}\ i\ {\isasymin}\ I\ {\isasymLongrightarrow}\ has{\isacharunderscore}{\kern0pt}bochner{\isacharunderscore}{\kern0pt}integral\ {\isacharparenleft}{\kern0pt}measure{\isacharunderscore}{\kern0pt}pmf\ {\isacharparenleft}{\kern0pt}M\ i{\isacharparenright}{\kern0pt}{\isacharparenright}{\kern0pt}\ {\isacharparenleft}{\kern0pt}f\ i{\isacharparenright}{\kern0pt}\ {\isacharparenleft}{\kern0pt}r\ i{\isacharparenright}{\kern0pt}{\isachardoublequoteclose}\isanewline
\ \ \isakeyword{shows}\ {\isachardoublequoteopen}has{\isacharunderscore}{\kern0pt}bochner{\isacharunderscore}{\kern0pt}integral\ {\isacharparenleft}{\kern0pt}prod{\isacharunderscore}{\kern0pt}pmf\ I\ M{\isacharparenright}{\kern0pt}\ {\isacharparenleft}{\kern0pt}{\isasymlambda}x{\isachardot}{\kern0pt}\ {\isacharparenleft}{\kern0pt}{\isasymProd}i\ {\isasymin}\ I{\isachardot}{\kern0pt}\ f\ i\ {\isacharparenleft}{\kern0pt}x\ i{\isacharparenright}{\kern0pt}{\isacharparenright}{\kern0pt}{\isacharparenright}{\kern0pt}\ {\isacharparenleft}{\kern0pt}{\isasymProd}i\ {\isasymin}\ I{\isachardot}{\kern0pt}\ r\ i{\isacharparenright}{\kern0pt}{\isachardoublequoteclose}\isanewline
%
\isadelimproof
%
\endisadelimproof
%
\isatagproof
\isacommand{proof}\isamarkupfalse%
\ {\isacharminus}{\kern0pt}\isanewline
\ \ \isacommand{define}\isamarkupfalse%
\ M{\isacharprime}{\kern0pt}\ \isakeyword{where}\ {\isachardoublequoteopen}M{\isacharprime}{\kern0pt}\ {\isacharequal}{\kern0pt}\ {\isacharparenleft}{\kern0pt}{\isasymlambda}i{\isachardot}{\kern0pt}\ if\ i\ {\isasymin}\ I\ then\ restrict{\isacharunderscore}{\kern0pt}space\ {\isacharparenleft}{\kern0pt}measure{\isacharunderscore}{\kern0pt}pmf\ {\isacharparenleft}{\kern0pt}M\ i{\isacharparenright}{\kern0pt}{\isacharparenright}{\kern0pt}\ {\isacharparenleft}{\kern0pt}set{\isacharunderscore}{\kern0pt}pmf\ {\isacharparenleft}{\kern0pt}M\ i{\isacharparenright}{\kern0pt}{\isacharparenright}{\kern0pt}\ else\ count{\isacharunderscore}{\kern0pt}space\ {\isacharbraceleft}{\kern0pt}undefined{\isacharbraceright}{\kern0pt}{\isacharparenright}{\kern0pt}{\isachardoublequoteclose}\isanewline
\isanewline
\ \ \isacommand{have}\isamarkupfalse%
\ a{\isacharcolon}{\kern0pt}{\isachardoublequoteopen}{\isasymAnd}i{\isachardot}{\kern0pt}\ i\ {\isasymin}\ I\ {\isasymLongrightarrow}\ finite{\isacharunderscore}{\kern0pt}measure\ {\isacharparenleft}{\kern0pt}restrict{\isacharunderscore}{\kern0pt}space\ {\isacharparenleft}{\kern0pt}measure{\isacharunderscore}{\kern0pt}pmf\ {\isacharparenleft}{\kern0pt}M\ i{\isacharparenright}{\kern0pt}{\isacharparenright}{\kern0pt}\ {\isacharparenleft}{\kern0pt}set{\isacharunderscore}{\kern0pt}pmf\ {\isacharparenleft}{\kern0pt}M\ i{\isacharparenright}{\kern0pt}{\isacharparenright}{\kern0pt}{\isacharparenright}{\kern0pt}{\isachardoublequoteclose}\isanewline
\ \ \ \ \isacommand{apply}\isamarkupfalse%
\ {\isacharparenleft}{\kern0pt}rule\ finite{\isacharunderscore}{\kern0pt}measureI{\isacharparenright}{\kern0pt}\isanewline
\ \ \ \ \isacommand{by}\isamarkupfalse%
\ {\isacharparenleft}{\kern0pt}simp\ add{\isacharcolon}{\kern0pt}emeasure{\isacharunderscore}{\kern0pt}restrict{\isacharunderscore}{\kern0pt}space{\isacharparenright}{\kern0pt}\isanewline
\isanewline
\ \ \isacommand{interpret}\isamarkupfalse%
\ product{\isacharunderscore}{\kern0pt}sigma{\isacharunderscore}{\kern0pt}finite\ M{\isacharprime}{\kern0pt}\isanewline
\ \ \ \ \isacommand{apply}\isamarkupfalse%
\ {\isacharparenleft}{\kern0pt}simp\ add{\isacharcolon}{\kern0pt}product{\isacharunderscore}{\kern0pt}sigma{\isacharunderscore}{\kern0pt}finite{\isacharunderscore}{\kern0pt}def\ M{\isacharprime}{\kern0pt}{\isacharunderscore}{\kern0pt}def{\isacharparenright}{\kern0pt}\isanewline
\ \ \ \ \isacommand{by}\isamarkupfalse%
\ {\isacharparenleft}{\kern0pt}metis\ a\ finite{\isacharunderscore}{\kern0pt}measure{\isachardot}{\kern0pt}axioms{\isacharparenleft}{\kern0pt}{\isadigit{1}}{\isacharparenright}{\kern0pt}\ finite{\isachardot}{\kern0pt}emptyI\ finite{\isacharunderscore}{\kern0pt}insert\ sigma{\isacharunderscore}{\kern0pt}finite{\isacharunderscore}{\kern0pt}measure{\isacharunderscore}{\kern0pt}count{\isacharunderscore}{\kern0pt}space{\isacharunderscore}{\kern0pt}finite{\isacharparenright}{\kern0pt}\isanewline
\isanewline
\ \ \isacommand{have}\isamarkupfalse%
\ {\isachardoublequoteopen}{\isasymAnd}i{\isachardot}{\kern0pt}\ i\ {\isasymin}\ I\ {\isasymLongrightarrow}\ has{\isacharunderscore}{\kern0pt}bochner{\isacharunderscore}{\kern0pt}integral\ {\isacharparenleft}{\kern0pt}M{\isacharprime}{\kern0pt}\ i{\isacharparenright}{\kern0pt}\ {\isacharparenleft}{\kern0pt}f\ i{\isacharparenright}{\kern0pt}\ {\isacharparenleft}{\kern0pt}r\ i{\isacharparenright}{\kern0pt}{\isachardoublequoteclose}\isanewline
\ \ \ \ \isacommand{apply}\isamarkupfalse%
\ {\isacharparenleft}{\kern0pt}simp\ add{\isacharcolon}{\kern0pt}M{\isacharprime}{\kern0pt}{\isacharunderscore}{\kern0pt}def\ has{\isacharunderscore}{\kern0pt}bochner{\isacharunderscore}{\kern0pt}integral{\isacharunderscore}{\kern0pt}restrict{\isacharunderscore}{\kern0pt}space{\isacharparenright}{\kern0pt}\isanewline
\ \ \ \ \isacommand{apply}\isamarkupfalse%
\ {\isacharparenleft}{\kern0pt}rule\ has{\isacharunderscore}{\kern0pt}bochner{\isacharunderscore}{\kern0pt}integralI{\isacharunderscore}{\kern0pt}AE{\isacharbrackleft}{\kern0pt}OF\ assms{\isacharparenleft}{\kern0pt}{\isadigit{2}}{\isacharparenright}{\kern0pt}{\isacharbrackright}{\kern0pt}{\isacharcomma}{\kern0pt}\ simp{\isacharcomma}{\kern0pt}\ simp{\isacharparenright}{\kern0pt}\isanewline
\ \ \ \ \isacommand{by}\isamarkupfalse%
\ {\isacharparenleft}{\kern0pt}subst\ AE{\isacharunderscore}{\kern0pt}measure{\isacharunderscore}{\kern0pt}pmf{\isacharunderscore}{\kern0pt}iff{\isacharcomma}{\kern0pt}\ simp{\isacharparenright}{\kern0pt}\isanewline
\isanewline
\ \ \isacommand{hence}\isamarkupfalse%
\ b{\isacharcolon}{\kern0pt}{\isachardoublequoteopen}has{\isacharunderscore}{\kern0pt}bochner{\isacharunderscore}{\kern0pt}integral\ {\isacharparenleft}{\kern0pt}PiM\ I\ M{\isacharprime}{\kern0pt}{\isacharparenright}{\kern0pt}\ {\isacharparenleft}{\kern0pt}{\isasymlambda}x{\isachardot}{\kern0pt}\ {\isacharparenleft}{\kern0pt}{\isasymProd}i\ {\isasymin}\ I{\isachardot}{\kern0pt}\ f\ i\ {\isacharparenleft}{\kern0pt}x\ i{\isacharparenright}{\kern0pt}{\isacharparenright}{\kern0pt}{\isacharparenright}{\kern0pt}\ {\isacharparenleft}{\kern0pt}{\isasymProd}i\ {\isasymin}\ I{\isachardot}{\kern0pt}\ r\ i{\isacharparenright}{\kern0pt}{\isachardoublequoteclose}\isanewline
\ \ \ \ \isacommand{apply}\isamarkupfalse%
\ {\isacharparenleft}{\kern0pt}subst\ has{\isacharunderscore}{\kern0pt}bochner{\isacharunderscore}{\kern0pt}integral{\isacharunderscore}{\kern0pt}iff{\isacharparenright}{\kern0pt}\isanewline
\ \ \ \ \isacommand{apply}\isamarkupfalse%
\ {\isacharparenleft}{\kern0pt}rule\ conjI{\isacharparenright}{\kern0pt}\isanewline
\ \ \ \ \ \isacommand{apply}\isamarkupfalse%
\ {\isacharparenleft}{\kern0pt}rule\ product{\isacharunderscore}{\kern0pt}integrable{\isacharunderscore}{\kern0pt}prod{\isacharbrackleft}{\kern0pt}OF\ assms{\isacharparenleft}{\kern0pt}{\isadigit{1}}{\isacharparenright}{\kern0pt}{\isacharbrackright}{\kern0pt}{\isacharparenright}{\kern0pt}\isanewline
\ \ \ \ \ \isacommand{apply}\isamarkupfalse%
\ {\isacharparenleft}{\kern0pt}simp\ add{\isacharcolon}{\kern0pt}\ has{\isacharunderscore}{\kern0pt}bochner{\isacharunderscore}{\kern0pt}integral{\isacharunderscore}{\kern0pt}iff{\isacharparenright}{\kern0pt}\isanewline
\ \ \ \ \isacommand{apply}\isamarkupfalse%
\ {\isacharparenleft}{\kern0pt}subst\ \ product{\isacharunderscore}{\kern0pt}integral{\isacharunderscore}{\kern0pt}prod{\isacharbrackleft}{\kern0pt}OF\ assms{\isacharparenleft}{\kern0pt}{\isadigit{1}}{\isacharparenright}{\kern0pt}{\isacharbrackright}{\kern0pt}{\isacharparenright}{\kern0pt}\isanewline
\ \ \ \ \isacommand{apply}\isamarkupfalse%
\ {\isacharparenleft}{\kern0pt}simp\ add{\isacharcolon}{\kern0pt}\ has{\isacharunderscore}{\kern0pt}bochner{\isacharunderscore}{\kern0pt}integral{\isacharunderscore}{\kern0pt}iff{\isacharparenright}{\kern0pt}\isanewline
\ \ \ \ \isacommand{apply}\isamarkupfalse%
\ {\isacharparenleft}{\kern0pt}rule\ prod{\isachardot}{\kern0pt}cong{\isacharcomma}{\kern0pt}\ simp{\isacharparenright}{\kern0pt}\isanewline
\ \ \ \ \isacommand{by}\isamarkupfalse%
\ {\isacharparenleft}{\kern0pt}simp\ add{\isacharcolon}{\kern0pt}\ has{\isacharunderscore}{\kern0pt}bochner{\isacharunderscore}{\kern0pt}integral{\isacharunderscore}{\kern0pt}iff{\isacharparenright}{\kern0pt}\isanewline
\isanewline
\ \ \isacommand{have}\isamarkupfalse%
\ d{\isacharcolon}{\kern0pt}{\isachardoublequoteopen}sets\ {\isacharparenleft}{\kern0pt}Pi\isactrlsub M\ I\ M{\isacharprime}{\kern0pt}{\isacharparenright}{\kern0pt}\ {\isacharequal}{\kern0pt}\ Pow\ {\isacharparenleft}{\kern0pt}Pi\isactrlsub E\ I\ {\isacharparenleft}{\kern0pt}set{\isacharunderscore}{\kern0pt}pmf\ {\isasymcirc}\ M{\isacharparenright}{\kern0pt}{\isacharparenright}{\kern0pt}{\isachardoublequoteclose}\isanewline
\ \ \ \ \isacommand{apply}\isamarkupfalse%
\ {\isacharparenleft}{\kern0pt}simp\ add{\isacharcolon}{\kern0pt}sets{\isacharunderscore}{\kern0pt}PiM\ M{\isacharprime}{\kern0pt}{\isacharunderscore}{\kern0pt}def\ comp{\isacharunderscore}{\kern0pt}def\ cong{\isacharcolon}{\kern0pt}PiM{\isacharunderscore}{\kern0pt}cong{\isacharparenright}{\kern0pt}\isanewline
\ \ \ \ \isacommand{apply}\isamarkupfalse%
\ {\isacharparenleft}{\kern0pt}rule\ order{\isacharunderscore}{\kern0pt}antisym{\isacharparenright}{\kern0pt}\isanewline
\ \ \ \ \ \isacommand{apply}\isamarkupfalse%
\ {\isacharparenleft}{\kern0pt}rule\ subsetI{\isacharparenright}{\kern0pt}\isanewline
\ \ \ \ \ \isacommand{apply}\isamarkupfalse%
\ {\isacharparenleft}{\kern0pt}simp{\isacharparenright}{\kern0pt}\isanewline
\ \ \ \ \ \isacommand{apply}\isamarkupfalse%
\ {\isacharparenleft}{\kern0pt}rule\ sigma{\isacharunderscore}{\kern0pt}sets{\isacharunderscore}{\kern0pt}into{\isacharunderscore}{\kern0pt}sp\ {\isacharbrackleft}{\kern0pt}\isakeyword{where}\ A{\isacharequal}{\kern0pt}{\isachardoublequoteopen}prod{\isacharunderscore}{\kern0pt}algebra\ I\ {\isacharparenleft}{\kern0pt}{\isasymlambda}x{\isachardot}{\kern0pt}\ restrict{\isacharunderscore}{\kern0pt}space\ {\isacharparenleft}{\kern0pt}measure{\isacharunderscore}{\kern0pt}pmf\ {\isacharparenleft}{\kern0pt}M\ x{\isacharparenright}{\kern0pt}{\isacharparenright}{\kern0pt}\ {\isacharparenleft}{\kern0pt}set{\isacharunderscore}{\kern0pt}pmf\ {\isacharparenleft}{\kern0pt}M\ x{\isacharparenright}{\kern0pt}{\isacharparenright}{\kern0pt}{\isacharparenright}{\kern0pt}{\isachardoublequoteclose}{\isacharbrackright}{\kern0pt}{\isacharparenright}{\kern0pt}\isanewline
\ \ \ \ \ \ \isacommand{apply}\isamarkupfalse%
\ {\isacharparenleft}{\kern0pt}metis\ {\isacharparenleft}{\kern0pt}mono{\isacharunderscore}{\kern0pt}tags{\isacharcomma}{\kern0pt}\ lifting{\isacharparenright}{\kern0pt}\ \ prod{\isacharunderscore}{\kern0pt}algebra{\isacharunderscore}{\kern0pt}sets{\isacharunderscore}{\kern0pt}into{\isacharunderscore}{\kern0pt}space\ space{\isacharunderscore}{\kern0pt}restrict{\isacharunderscore}{\kern0pt}space\ PiE{\isacharunderscore}{\kern0pt}cong\ UNIV{\isacharunderscore}{\kern0pt}I\ sets{\isacharunderscore}{\kern0pt}measure{\isacharunderscore}{\kern0pt}pmf\ space{\isacharunderscore}{\kern0pt}restrict{\isacharunderscore}{\kern0pt}space{\isadigit{2}}{\isacharparenright}{\kern0pt}\isanewline
\ \ \ \ \ \isacommand{apply}\isamarkupfalse%
\ simp\isanewline
\ \ \ \ \isacommand{apply}\isamarkupfalse%
\ {\isacharparenleft}{\kern0pt}subst\ sigma{\isacharunderscore}{\kern0pt}sets{\isacharunderscore}{\kern0pt}singletons{\isacharbrackleft}{\kern0pt}symmetric{\isacharbrackright}{\kern0pt}{\isacharparenright}{\kern0pt}\isanewline
\ \ \ \ \ \isacommand{apply}\isamarkupfalse%
\ {\isacharparenleft}{\kern0pt}rule\ countable{\isacharunderscore}{\kern0pt}PiE{\isacharcomma}{\kern0pt}\ metis\ assms{\isacharparenleft}{\kern0pt}{\isadigit{1}}{\isacharparenright}{\kern0pt}{\isacharcomma}{\kern0pt}\ metis\ countable{\isacharunderscore}{\kern0pt}set{\isacharunderscore}{\kern0pt}pmf{\isacharparenright}{\kern0pt}\isanewline
\ \ \ \ \isacommand{apply}\isamarkupfalse%
\ {\isacharparenleft}{\kern0pt}rule\ sigma{\isacharunderscore}{\kern0pt}sets{\isacharunderscore}{\kern0pt}subseteq{\isacharparenright}{\kern0pt}\isanewline
\ \ \ \ \isacommand{apply}\isamarkupfalse%
\ {\isacharparenleft}{\kern0pt}rule\ image{\isacharunderscore}{\kern0pt}subsetI{\isacharparenright}{\kern0pt}\isanewline
\ \ \ \ \isacommand{apply}\isamarkupfalse%
\ {\isacharparenleft}{\kern0pt}subst\ PiE{\isacharunderscore}{\kern0pt}singleton{\isacharbrackleft}{\kern0pt}symmetric{\isacharcomma}{\kern0pt}\ \isakeyword{where}\ A{\isacharequal}{\kern0pt}{\isachardoublequoteopen}I{\isachardoublequoteclose}{\isacharbrackright}{\kern0pt}{\isacharcomma}{\kern0pt}\ simp\ add{\isacharcolon}{\kern0pt}PiE{\isacharunderscore}{\kern0pt}def{\isacharparenright}{\kern0pt}\isanewline
\ \ \ \ \isacommand{apply}\isamarkupfalse%
\ {\isacharparenleft}{\kern0pt}rule\ prod{\isacharunderscore}{\kern0pt}algebraI{\isacharunderscore}{\kern0pt}finite{\isacharcomma}{\kern0pt}\ metis\ assms{\isacharparenleft}{\kern0pt}{\isadigit{1}}{\isacharparenright}{\kern0pt}{\isacharparenright}{\kern0pt}\isanewline
\ \ \ \ \isacommand{apply}\isamarkupfalse%
\ {\isacharparenleft}{\kern0pt}simp\ add{\isacharcolon}{\kern0pt}sets{\isacharunderscore}{\kern0pt}restrict{\isacharunderscore}{\kern0pt}space\ PiE{\isacharunderscore}{\kern0pt}iff\ image{\isacharunderscore}{\kern0pt}def{\isacharparenright}{\kern0pt}\isanewline
\ \ \ \ \isacommand{by}\isamarkupfalse%
\ blast\isanewline
\isanewline
\ \ \isacommand{have}\isamarkupfalse%
\ c{\isacharcolon}{\kern0pt}{\isachardoublequoteopen}PiM\ I\ M{\isacharprime}{\kern0pt}\ {\isacharequal}{\kern0pt}\ restrict{\isacharunderscore}{\kern0pt}space\ {\isacharparenleft}{\kern0pt}measure{\isacharunderscore}{\kern0pt}pmf\ {\isacharparenleft}{\kern0pt}prod{\isacharunderscore}{\kern0pt}pmf\ I\ M{\isacharparenright}{\kern0pt}{\isacharparenright}{\kern0pt}\ {\isacharparenleft}{\kern0pt}PiE\ I\ {\isacharparenleft}{\kern0pt}set{\isacharunderscore}{\kern0pt}pmf\ {\isasymcirc}\ M{\isacharparenright}{\kern0pt}{\isacharparenright}{\kern0pt}{\isachardoublequoteclose}\isanewline
\ \ \ \ \isacommand{apply}\isamarkupfalse%
\ {\isacharparenleft}{\kern0pt}rule\ measure{\isacharunderscore}{\kern0pt}eqI{\isacharunderscore}{\kern0pt}countable{\isacharbrackleft}{\kern0pt}\isakeyword{where}\ A{\isacharequal}{\kern0pt}{\isachardoublequoteopen}PiE\ I\ {\isacharparenleft}{\kern0pt}set{\isacharunderscore}{\kern0pt}pmf\ {\isasymcirc}\ M{\isacharparenright}{\kern0pt}{\isachardoublequoteclose}{\isacharbrackright}{\kern0pt}{\isacharparenright}{\kern0pt}\isanewline
\ \ \ \ \ \ \ \isacommand{apply}\isamarkupfalse%
\ {\isacharparenleft}{\kern0pt}metis\ d{\isacharparenright}{\kern0pt}\isanewline
\ \ \ \ \ \ \isacommand{apply}\isamarkupfalse%
\ {\isacharparenleft}{\kern0pt}simp\ add{\isacharcolon}{\kern0pt}sets{\isacharunderscore}{\kern0pt}restrict{\isacharunderscore}{\kern0pt}space\ image{\isacharunderscore}{\kern0pt}def{\isacharcomma}{\kern0pt}\ fastforce{\isacharparenright}{\kern0pt}\isanewline
\ \ \ \ \ \isacommand{apply}\isamarkupfalse%
\ {\isacharparenleft}{\kern0pt}rule\ countable{\isacharunderscore}{\kern0pt}PiE{\isacharcomma}{\kern0pt}\ metis\ assms{\isacharparenleft}{\kern0pt}{\isadigit{1}}{\isacharparenright}{\kern0pt}{\isacharcomma}{\kern0pt}\ simp\ add{\isacharcolon}{\kern0pt}comp{\isacharunderscore}{\kern0pt}def{\isacharparenright}{\kern0pt}\isanewline
\ \ \ \ \isacommand{apply}\isamarkupfalse%
\ {\isacharparenleft}{\kern0pt}subst\ PiE{\isacharunderscore}{\kern0pt}singleton{\isacharbrackleft}{\kern0pt}symmetric{\isacharcomma}{\kern0pt}\ \isakeyword{where}\ A{\isacharequal}{\kern0pt}{\isachardoublequoteopen}I{\isachardoublequoteclose}{\isacharbrackright}{\kern0pt}{\isacharcomma}{\kern0pt}\ simp\ add{\isacharcolon}{\kern0pt}PiE{\isacharunderscore}{\kern0pt}def{\isacharparenright}{\kern0pt}\isanewline
\ \ \ \ \isacommand{apply}\isamarkupfalse%
\ {\isacharparenleft}{\kern0pt}subst\ emeasure{\isacharunderscore}{\kern0pt}PiM{\isacharcomma}{\kern0pt}\ metis\ assms{\isacharparenleft}{\kern0pt}{\isadigit{1}}{\isacharparenright}{\kern0pt}{\isacharcomma}{\kern0pt}\ simp\ add{\isacharcolon}{\kern0pt}M{\isacharprime}{\kern0pt}{\isacharunderscore}{\kern0pt}def\ sets{\isacharunderscore}{\kern0pt}restrict{\isacharunderscore}{\kern0pt}space{\isacharcomma}{\kern0pt}\ fastforce{\isacharparenright}{\kern0pt}\isanewline
\ \ \ \ \isacommand{apply}\isamarkupfalse%
\ {\isacharparenleft}{\kern0pt}subst\ emeasure{\isacharunderscore}{\kern0pt}restrict{\isacharunderscore}{\kern0pt}space{\isacharcomma}{\kern0pt}\ simp{\isacharcomma}{\kern0pt}\ simp{\isacharparenright}{\kern0pt}\isanewline
\ \ \ \ \isacommand{apply}\isamarkupfalse%
\ {\isacharparenleft}{\kern0pt}simp\ add{\isacharcolon}{\kern0pt}emeasure{\isacharunderscore}{\kern0pt}pmf{\isacharunderscore}{\kern0pt}single\ pmf{\isacharunderscore}{\kern0pt}prod{\isacharunderscore}{\kern0pt}pmf{\isacharbrackleft}{\kern0pt}OF\ assms{\isacharparenleft}{\kern0pt}{\isadigit{1}}{\isacharparenright}{\kern0pt}{\isacharbrackright}{\kern0pt}\ PiE{\isacharunderscore}{\kern0pt}def\ prod{\isacharunderscore}{\kern0pt}ennreal{\isacharbrackleft}{\kern0pt}symmetric{\isacharbrackright}{\kern0pt}\ M{\isacharprime}{\kern0pt}{\isacharunderscore}{\kern0pt}def{\isacharparenright}{\kern0pt}\isanewline
\ \ \ \ \isacommand{apply}\isamarkupfalse%
\ {\isacharparenleft}{\kern0pt}rule\ prod{\isachardot}{\kern0pt}cong{\isacharcomma}{\kern0pt}\ simp{\isacharparenright}{\kern0pt}\isanewline
\ \ \ \ \isacommand{apply}\isamarkupfalse%
\ {\isacharparenleft}{\kern0pt}subst\ emeasure{\isacharunderscore}{\kern0pt}restrict{\isacharunderscore}{\kern0pt}space{\isacharcomma}{\kern0pt}\ simp{\isacharcomma}{\kern0pt}\ simp\ add{\isacharcolon}{\kern0pt}Pi{\isacharunderscore}{\kern0pt}iff{\isacharparenright}{\kern0pt}\isanewline
\ \ \ \ \isacommand{by}\isamarkupfalse%
\ {\isacharparenleft}{\kern0pt}simp\ add{\isacharcolon}{\kern0pt}emeasure{\isacharunderscore}{\kern0pt}pmf{\isacharunderscore}{\kern0pt}single{\isacharparenright}{\kern0pt}\isanewline
\isanewline
\ \ \isacommand{have}\isamarkupfalse%
\ a{\isacharcolon}{\kern0pt}{\isachardoublequoteopen}has{\isacharunderscore}{\kern0pt}bochner{\isacharunderscore}{\kern0pt}integral\ {\isacharparenleft}{\kern0pt}prod{\isacharunderscore}{\kern0pt}pmf\ I\ M{\isacharparenright}{\kern0pt}\ {\isacharparenleft}{\kern0pt}{\isasymlambda}x{\isachardot}{\kern0pt}\ indicator\ {\isacharparenleft}{\kern0pt}PiE\ I\ {\isacharparenleft}{\kern0pt}set{\isacharunderscore}{\kern0pt}pmf\ {\isasymcirc}\ M{\isacharparenright}{\kern0pt}{\isacharparenright}{\kern0pt}\ x\ {\isacharasterisk}{\kern0pt}\isactrlsub R\ {\isacharparenleft}{\kern0pt}{\isasymProd}i\ {\isasymin}\ I{\isachardot}{\kern0pt}\ f\ i\ {\isacharparenleft}{\kern0pt}x\ i{\isacharparenright}{\kern0pt}{\isacharparenright}{\kern0pt}{\isacharparenright}{\kern0pt}\ {\isacharparenleft}{\kern0pt}{\isasymProd}i\ {\isasymin}\ I{\isachardot}{\kern0pt}\ r\ i{\isacharparenright}{\kern0pt}{\isachardoublequoteclose}\isanewline
\ \ \ \ \isacommand{apply}\isamarkupfalse%
\ {\isacharparenleft}{\kern0pt}subst\ Lebesgue{\isacharunderscore}{\kern0pt}Measure{\isachardot}{\kern0pt}has{\isacharunderscore}{\kern0pt}bochner{\isacharunderscore}{\kern0pt}integral{\isacharunderscore}{\kern0pt}restrict{\isacharunderscore}{\kern0pt}space{\isacharbrackleft}{\kern0pt}symmetric{\isacharbrackright}{\kern0pt}{\isacharcomma}{\kern0pt}\ simp{\isacharparenright}{\kern0pt}\isanewline
\ \ \ \ \isacommand{by}\isamarkupfalse%
\ {\isacharparenleft}{\kern0pt}subst\ c{\isacharbrackleft}{\kern0pt}symmetric{\isacharbrackright}{\kern0pt}{\isacharcomma}{\kern0pt}\ metis\ b{\isacharparenright}{\kern0pt}\isanewline
\isanewline
\ \ \isacommand{have}\isamarkupfalse%
\ {\isachardoublequoteopen}{\isacharparenleft}{\kern0pt}{\isasymlambda}x{\isachardot}{\kern0pt}\ {\isasymProd}i\ {\isasymin}\ I{\isachardot}{\kern0pt}\ f\ i\ {\isacharparenleft}{\kern0pt}x\ i{\isacharparenright}{\kern0pt}{\isacharparenright}{\kern0pt}\ {\isasymin}\ borel{\isacharunderscore}{\kern0pt}measurable\ {\isacharparenleft}{\kern0pt}prod{\isacharunderscore}{\kern0pt}pmf\ I\ M{\isacharparenright}{\kern0pt}{\isachardoublequoteclose}\isanewline
\ \ \ \ \isacommand{by}\isamarkupfalse%
\ simp\isanewline
\ \ \isacommand{show}\isamarkupfalse%
\ {\isachardoublequoteopen}has{\isacharunderscore}{\kern0pt}bochner{\isacharunderscore}{\kern0pt}integral\ {\isacharparenleft}{\kern0pt}prod{\isacharunderscore}{\kern0pt}pmf\ I\ M{\isacharparenright}{\kern0pt}\ {\isacharparenleft}{\kern0pt}{\isasymlambda}x{\isachardot}{\kern0pt}\ {\isacharparenleft}{\kern0pt}{\isasymProd}i\ {\isasymin}\ I{\isachardot}{\kern0pt}\ f\ i\ {\isacharparenleft}{\kern0pt}x\ i{\isacharparenright}{\kern0pt}{\isacharparenright}{\kern0pt}{\isacharparenright}{\kern0pt}\ {\isacharparenleft}{\kern0pt}{\isasymProd}i\ {\isasymin}\ I{\isachardot}{\kern0pt}\ r\ i{\isacharparenright}{\kern0pt}{\isachardoublequoteclose}\isanewline
\ \ \ \ \isacommand{apply}\isamarkupfalse%
\ {\isacharparenleft}{\kern0pt}rule\ has{\isacharunderscore}{\kern0pt}bochner{\isacharunderscore}{\kern0pt}integralI{\isacharunderscore}{\kern0pt}AE{\isacharbrackleft}{\kern0pt}OF\ a{\isacharbrackright}{\kern0pt}{\isacharcomma}{\kern0pt}\ simp{\isacharparenright}{\kern0pt}\isanewline
\ \ \ \ \isacommand{apply}\isamarkupfalse%
\ {\isacharparenleft}{\kern0pt}subst\ AE{\isacharunderscore}{\kern0pt}measure{\isacharunderscore}{\kern0pt}pmf{\isacharunderscore}{\kern0pt}iff{\isacharparenright}{\kern0pt}\isanewline
\ \ \ \ \isacommand{using}\isamarkupfalse%
\ assms\ \isacommand{by}\isamarkupfalse%
\ {\isacharparenleft}{\kern0pt}simp\ add{\isacharcolon}{\kern0pt}set{\isacharunderscore}{\kern0pt}prod{\isacharunderscore}{\kern0pt}pmf{\isacharparenright}{\kern0pt}\isanewline
\isacommand{qed}\isamarkupfalse%
%
\endisatagproof
{\isafoldproof}%
%
\isadelimproof
\isanewline
%
\endisadelimproof
\isanewline
\isacommand{lemma}\isamarkupfalse%
\isanewline
\ \ \isakeyword{fixes}\ f\ {\isacharcolon}{\kern0pt}{\isacharcolon}{\kern0pt}\ {\isachardoublequoteopen}{\isacharprime}{\kern0pt}a\ {\isasymRightarrow}\ {\isacharprime}{\kern0pt}b\ {\isasymRightarrow}\ {\isacharparenleft}{\kern0pt}{\isacharprime}{\kern0pt}c\ {\isacharcolon}{\kern0pt}{\isacharcolon}{\kern0pt}\ {\isacharbraceleft}{\kern0pt}second{\isacharunderscore}{\kern0pt}countable{\isacharunderscore}{\kern0pt}topology{\isacharcomma}{\kern0pt}banach{\isacharcomma}{\kern0pt}real{\isacharunderscore}{\kern0pt}normed{\isacharunderscore}{\kern0pt}field{\isacharbraceright}{\kern0pt}{\isacharparenright}{\kern0pt}{\isachardoublequoteclose}\isanewline
\ \ \isakeyword{assumes}\ {\isachardoublequoteopen}finite\ I{\isachardoublequoteclose}\isanewline
\ \ \isakeyword{assumes}\ {\isachardoublequoteopen}{\isasymAnd}i{\isachardot}{\kern0pt}\ i\ {\isasymin}\ I\ {\isasymLongrightarrow}\ integrable\ {\isacharparenleft}{\kern0pt}measure{\isacharunderscore}{\kern0pt}pmf\ {\isacharparenleft}{\kern0pt}M\ i{\isacharparenright}{\kern0pt}{\isacharparenright}{\kern0pt}\ {\isacharparenleft}{\kern0pt}f\ i{\isacharparenright}{\kern0pt}{\isachardoublequoteclose}\isanewline
\ \ \isakeyword{shows}\ prod{\isacharunderscore}{\kern0pt}pmf{\isacharunderscore}{\kern0pt}integrable{\isacharcolon}{\kern0pt}\ {\isachardoublequoteopen}integrable\ {\isacharparenleft}{\kern0pt}prod{\isacharunderscore}{\kern0pt}pmf\ I\ M{\isacharparenright}{\kern0pt}\ {\isacharparenleft}{\kern0pt}{\isasymlambda}x{\isachardot}{\kern0pt}\ {\isacharparenleft}{\kern0pt}{\isasymProd}i\ {\isasymin}\ I{\isachardot}{\kern0pt}\ f\ i\ {\isacharparenleft}{\kern0pt}x\ i{\isacharparenright}{\kern0pt}{\isacharparenright}{\kern0pt}{\isacharparenright}{\kern0pt}{\isachardoublequoteclose}\ {\isacharparenleft}{\kern0pt}\isakeyword{is}\ {\isacharquery}{\kern0pt}A{\isacharparenright}{\kern0pt}\ \isakeyword{and}\isanewline
\ \ \ prod{\isacharunderscore}{\kern0pt}pmf{\isacharunderscore}{\kern0pt}integral{\isacharcolon}{\kern0pt}\ {\isachardoublequoteopen}integral\isactrlsup L\ {\isacharparenleft}{\kern0pt}prod{\isacharunderscore}{\kern0pt}pmf\ I\ M{\isacharparenright}{\kern0pt}\ {\isacharparenleft}{\kern0pt}{\isasymlambda}x{\isachardot}{\kern0pt}\ {\isacharparenleft}{\kern0pt}{\isasymProd}i\ {\isasymin}\ I{\isachardot}{\kern0pt}\ f\ i\ {\isacharparenleft}{\kern0pt}x\ i{\isacharparenright}{\kern0pt}{\isacharparenright}{\kern0pt}{\isacharparenright}{\kern0pt}\ {\isacharequal}{\kern0pt}\ \isanewline
\ \ \ \ {\isacharparenleft}{\kern0pt}{\isasymProd}\ i\ {\isasymin}\ I{\isachardot}{\kern0pt}\ integral\isactrlsup L\ {\isacharparenleft}{\kern0pt}M\ i{\isacharparenright}{\kern0pt}\ {\isacharparenleft}{\kern0pt}f\ i{\isacharparenright}{\kern0pt}{\isacharparenright}{\kern0pt}{\isachardoublequoteclose}\ {\isacharparenleft}{\kern0pt}\isakeyword{is}\ {\isacharquery}{\kern0pt}B{\isacharparenright}{\kern0pt}\isanewline
%
\isadelimproof
%
\endisadelimproof
%
\isatagproof
\isacommand{proof}\isamarkupfalse%
\ {\isacharminus}{\kern0pt}\isanewline
\ \ \isacommand{have}\isamarkupfalse%
\ a{\isacharcolon}{\kern0pt}{\isachardoublequoteopen}has{\isacharunderscore}{\kern0pt}bochner{\isacharunderscore}{\kern0pt}integral\ {\isacharparenleft}{\kern0pt}prod{\isacharunderscore}{\kern0pt}pmf\ I\ M{\isacharparenright}{\kern0pt}\ {\isacharparenleft}{\kern0pt}{\isasymlambda}x{\isachardot}{\kern0pt}\ {\isacharparenleft}{\kern0pt}{\isasymProd}i\ {\isasymin}\ I{\isachardot}{\kern0pt}\ f\ i\ {\isacharparenleft}{\kern0pt}x\ i{\isacharparenright}{\kern0pt}{\isacharparenright}{\kern0pt}{\isacharparenright}{\kern0pt}\ \ {\isacharparenleft}{\kern0pt}{\isasymProd}\ i\ {\isasymin}\ I{\isachardot}{\kern0pt}\ integral\isactrlsup L\ {\isacharparenleft}{\kern0pt}M\ i{\isacharparenright}{\kern0pt}\ {\isacharparenleft}{\kern0pt}f\ i{\isacharparenright}{\kern0pt}{\isacharparenright}{\kern0pt}{\isachardoublequoteclose}\isanewline
\ \ \ \ \isacommand{apply}\isamarkupfalse%
\ {\isacharparenleft}{\kern0pt}rule\ has{\isacharunderscore}{\kern0pt}bochner{\isacharunderscore}{\kern0pt}integral{\isacharunderscore}{\kern0pt}prod{\isacharunderscore}{\kern0pt}pmfI{\isacharbrackleft}{\kern0pt}OF\ assms{\isacharparenleft}{\kern0pt}{\isadigit{1}}{\isacharparenright}{\kern0pt}{\isacharbrackright}{\kern0pt}{\isacharparenright}{\kern0pt}\isanewline
\ \ \ \ \isacommand{by}\isamarkupfalse%
\ {\isacharparenleft}{\kern0pt}rule\ has{\isacharunderscore}{\kern0pt}bochner{\isacharunderscore}{\kern0pt}integral{\isacharunderscore}{\kern0pt}integrable{\isacharbrackleft}{\kern0pt}OF\ assms{\isacharparenleft}{\kern0pt}{\isadigit{2}}{\isacharparenright}{\kern0pt}{\isacharbrackright}{\kern0pt}{\isacharcomma}{\kern0pt}\ simp{\isacharparenright}{\kern0pt}\isanewline
\ \ \isacommand{show}\isamarkupfalse%
\ {\isacharquery}{\kern0pt}A\ \isacommand{using}\isamarkupfalse%
\ a\ has{\isacharunderscore}{\kern0pt}bochner{\isacharunderscore}{\kern0pt}integral{\isacharunderscore}{\kern0pt}iff\ \isacommand{by}\isamarkupfalse%
\ blast\isanewline
\ \ \isacommand{show}\isamarkupfalse%
\ {\isacharquery}{\kern0pt}B\ \isacommand{using}\isamarkupfalse%
\ a\ has{\isacharunderscore}{\kern0pt}bochner{\isacharunderscore}{\kern0pt}integral{\isacharunderscore}{\kern0pt}iff\ \isacommand{by}\isamarkupfalse%
\ blast\isanewline
\isacommand{qed}\isamarkupfalse%
%
\endisatagproof
{\isafoldproof}%
%
\isadelimproof
\isanewline
%
\endisadelimproof
\isanewline
\isacommand{lemma}\isamarkupfalse%
\ has{\isacharunderscore}{\kern0pt}bochner{\isacharunderscore}{\kern0pt}integral{\isacharunderscore}{\kern0pt}prod{\isacharunderscore}{\kern0pt}pmf{\isacharunderscore}{\kern0pt}sliceI{\isacharcolon}{\kern0pt}\isanewline
\ \ \isakeyword{fixes}\ f\ {\isacharcolon}{\kern0pt}{\isacharcolon}{\kern0pt}\ {\isachardoublequoteopen}{\isacharprime}{\kern0pt}a\ {\isasymRightarrow}\ {\isacharparenleft}{\kern0pt}{\isacharprime}{\kern0pt}b\ {\isacharcolon}{\kern0pt}{\isacharcolon}{\kern0pt}\ {\isacharbraceleft}{\kern0pt}second{\isacharunderscore}{\kern0pt}countable{\isacharunderscore}{\kern0pt}topology{\isacharcomma}{\kern0pt}banach{\isacharcomma}{\kern0pt}real{\isacharunderscore}{\kern0pt}normed{\isacharunderscore}{\kern0pt}field{\isacharbraceright}{\kern0pt}{\isacharparenright}{\kern0pt}{\isachardoublequoteclose}\isanewline
\ \ \isakeyword{assumes}\ {\isachardoublequoteopen}finite\ I{\isachardoublequoteclose}\isanewline
\ \ \isakeyword{assumes}\ {\isachardoublequoteopen}i\ {\isasymin}\ I{\isachardoublequoteclose}\isanewline
\ \ \isakeyword{assumes}\ {\isachardoublequoteopen}has{\isacharunderscore}{\kern0pt}bochner{\isacharunderscore}{\kern0pt}integral\ {\isacharparenleft}{\kern0pt}measure{\isacharunderscore}{\kern0pt}pmf\ {\isacharparenleft}{\kern0pt}M\ i{\isacharparenright}{\kern0pt}{\isacharparenright}{\kern0pt}\ {\isacharparenleft}{\kern0pt}f{\isacharparenright}{\kern0pt}\ r{\isachardoublequoteclose}\isanewline
\ \ \isakeyword{shows}\ {\isachardoublequoteopen}has{\isacharunderscore}{\kern0pt}bochner{\isacharunderscore}{\kern0pt}integral\ {\isacharparenleft}{\kern0pt}prod{\isacharunderscore}{\kern0pt}pmf\ I\ M{\isacharparenright}{\kern0pt}\ {\isacharparenleft}{\kern0pt}{\isasymlambda}x{\isachardot}{\kern0pt}\ {\isacharparenleft}{\kern0pt}f\ {\isacharparenleft}{\kern0pt}x\ i{\isacharparenright}{\kern0pt}{\isacharparenright}{\kern0pt}{\isacharparenright}{\kern0pt}\ r{\isachardoublequoteclose}\isanewline
%
\isadelimproof
%
\endisadelimproof
%
\isatagproof
\isacommand{proof}\isamarkupfalse%
\ {\isacharminus}{\kern0pt}\isanewline
\ \ \isacommand{define}\isamarkupfalse%
\ g\ \isakeyword{where}\ {\isachardoublequoteopen}g\ {\isacharequal}{\kern0pt}\ {\isacharparenleft}{\kern0pt}{\isasymlambda}j\ {\isasymomega}{\isachardot}{\kern0pt}\ if\ j\ {\isacharequal}{\kern0pt}\ i\ then\ f\ {\isasymomega}\ else\ {\isadigit{1}}{\isacharparenright}{\kern0pt}{\isachardoublequoteclose}\isanewline
\isanewline
\ \ \isacommand{have}\isamarkupfalse%
\ b{\isacharcolon}{\kern0pt}\ {\isachardoublequoteopen}{\isasymAnd}M{\isachardot}{\kern0pt}\ has{\isacharunderscore}{\kern0pt}bochner{\isacharunderscore}{\kern0pt}integral\ {\isacharparenleft}{\kern0pt}measure{\isacharunderscore}{\kern0pt}pmf\ M{\isacharparenright}{\kern0pt}\ {\isacharparenleft}{\kern0pt}{\isasymlambda}{\isasymomega}{\isachardot}{\kern0pt}\ {\isadigit{1}}{\isacharcolon}{\kern0pt}{\isacharcolon}{\kern0pt}{\isacharprime}{\kern0pt}b{\isacharparenright}{\kern0pt}\ {\isadigit{1}}{\isachardoublequoteclose}\isanewline
\ \ \ \ \isacommand{apply}\isamarkupfalse%
\ {\isacharparenleft}{\kern0pt}subst\ has{\isacharunderscore}{\kern0pt}bochner{\isacharunderscore}{\kern0pt}integral{\isacharunderscore}{\kern0pt}iff{\isacharcomma}{\kern0pt}\ rule\ conjI{\isacharcomma}{\kern0pt}\ simp{\isacharparenright}{\kern0pt}\isanewline
\ \ \ \ \isacommand{by}\isamarkupfalse%
\ {\isacharparenleft}{\kern0pt}subst\ lebesgue{\isacharunderscore}{\kern0pt}integral{\isacharunderscore}{\kern0pt}const{\isacharcomma}{\kern0pt}\ simp{\isacharparenright}{\kern0pt}\isanewline
\ \ \isanewline
\ \ \isacommand{have}\isamarkupfalse%
\ a{\isacharcolon}{\kern0pt}{\isachardoublequoteopen}{\isasymAnd}j{\isachardot}{\kern0pt}\ j\ {\isasymin}\ I\ {\isasymLongrightarrow}\ has{\isacharunderscore}{\kern0pt}bochner{\isacharunderscore}{\kern0pt}integral\ {\isacharparenleft}{\kern0pt}measure{\isacharunderscore}{\kern0pt}pmf\ {\isacharparenleft}{\kern0pt}M\ j{\isacharparenright}{\kern0pt}{\isacharparenright}{\kern0pt}\ {\isacharparenleft}{\kern0pt}g\ j{\isacharparenright}{\kern0pt}\ {\isacharparenleft}{\kern0pt}if\ j\ {\isacharequal}{\kern0pt}\ i\ then\ r\ else\ {\isadigit{1}}{\isacharparenright}{\kern0pt}{\isachardoublequoteclose}\isanewline
\ \ \ \ \isacommand{using}\isamarkupfalse%
\ assms{\isacharparenleft}{\kern0pt}{\isadigit{3}}{\isacharparenright}{\kern0pt}\ \isacommand{by}\isamarkupfalse%
\ {\isacharparenleft}{\kern0pt}simp\ add{\isacharcolon}{\kern0pt}g{\isacharunderscore}{\kern0pt}def\ b{\isacharparenright}{\kern0pt}\isanewline
\ \ \isacommand{have}\isamarkupfalse%
\ {\isachardoublequoteopen}has{\isacharunderscore}{\kern0pt}bochner{\isacharunderscore}{\kern0pt}integral\ {\isacharparenleft}{\kern0pt}prod{\isacharunderscore}{\kern0pt}pmf\ I\ M{\isacharparenright}{\kern0pt}\ {\isacharparenleft}{\kern0pt}{\isasymlambda}x{\isachardot}{\kern0pt}\ {\isacharparenleft}{\kern0pt}{\isasymProd}j\ {\isasymin}\ I{\isachardot}{\kern0pt}\ g\ j\ {\isacharparenleft}{\kern0pt}x\ j{\isacharparenright}{\kern0pt}{\isacharparenright}{\kern0pt}{\isacharparenright}{\kern0pt}\ {\isacharparenleft}{\kern0pt}{\isasymProd}j\ {\isasymin}\ I{\isachardot}{\kern0pt}\ if\ j\ {\isacharequal}{\kern0pt}\ i\ then\ r\ else\ {\isadigit{1}}{\isacharparenright}{\kern0pt}{\isachardoublequoteclose}\isanewline
\ \ \ \ \isacommand{by}\isamarkupfalse%
\ {\isacharparenleft}{\kern0pt}rule\ has{\isacharunderscore}{\kern0pt}bochner{\isacharunderscore}{\kern0pt}integral{\isacharunderscore}{\kern0pt}prod{\isacharunderscore}{\kern0pt}pmfI{\isacharbrackleft}{\kern0pt}OF\ assms{\isacharparenleft}{\kern0pt}{\isadigit{1}}{\isacharparenright}{\kern0pt}{\isacharbrackright}{\kern0pt}{\isacharcomma}{\kern0pt}\ metis\ a{\isacharparenright}{\kern0pt}\isanewline
\ \ \isacommand{thus}\isamarkupfalse%
\ {\isacharquery}{\kern0pt}thesis\isanewline
\ \ \ \ \isacommand{using}\isamarkupfalse%
\ assms{\isacharparenleft}{\kern0pt}{\isadigit{2}}{\isacharparenright}{\kern0pt}\ \isacommand{by}\isamarkupfalse%
\ {\isacharparenleft}{\kern0pt}simp\ add{\isacharcolon}{\kern0pt}g{\isacharunderscore}{\kern0pt}def\ prod{\isachardot}{\kern0pt}If{\isacharunderscore}{\kern0pt}cases{\isacharbrackleft}{\kern0pt}OF\ assms{\isacharparenleft}{\kern0pt}{\isadigit{1}}{\isacharparenright}{\kern0pt}{\isacharbrackright}{\kern0pt}{\isacharparenright}{\kern0pt}\isanewline
\isacommand{qed}\isamarkupfalse%
%
\endisatagproof
{\isafoldproof}%
%
\isadelimproof
\isanewline
%
\endisadelimproof
\isanewline
\isacommand{lemma}\isamarkupfalse%
\isanewline
\ \ \isakeyword{fixes}\ f\ {\isacharcolon}{\kern0pt}{\isacharcolon}{\kern0pt}\ {\isachardoublequoteopen}{\isacharprime}{\kern0pt}a\ {\isasymRightarrow}\ {\isacharparenleft}{\kern0pt}{\isacharprime}{\kern0pt}b\ {\isacharcolon}{\kern0pt}{\isacharcolon}{\kern0pt}\ {\isacharbraceleft}{\kern0pt}second{\isacharunderscore}{\kern0pt}countable{\isacharunderscore}{\kern0pt}topology{\isacharcomma}{\kern0pt}banach{\isacharcomma}{\kern0pt}real{\isacharunderscore}{\kern0pt}normed{\isacharunderscore}{\kern0pt}field{\isacharbraceright}{\kern0pt}{\isacharparenright}{\kern0pt}{\isachardoublequoteclose}\isanewline
\ \ \isakeyword{assumes}\ {\isachardoublequoteopen}finite\ I{\isachardoublequoteclose}\isanewline
\ \ \isakeyword{assumes}\ {\isachardoublequoteopen}i\ {\isasymin}\ I{\isachardoublequoteclose}\isanewline
\ \ \isakeyword{assumes}\ {\isachardoublequoteopen}integrable\ {\isacharparenleft}{\kern0pt}measure{\isacharunderscore}{\kern0pt}pmf\ {\isacharparenleft}{\kern0pt}M\ i{\isacharparenright}{\kern0pt}{\isacharparenright}{\kern0pt}\ f{\isachardoublequoteclose}\isanewline
\ \ \isakeyword{shows}\ integrable{\isacharunderscore}{\kern0pt}prod{\isacharunderscore}{\kern0pt}pmf{\isacharunderscore}{\kern0pt}slice{\isacharcolon}{\kern0pt}\ {\isachardoublequoteopen}integrable\ {\isacharparenleft}{\kern0pt}prod{\isacharunderscore}{\kern0pt}pmf\ I\ M{\isacharparenright}{\kern0pt}\ {\isacharparenleft}{\kern0pt}{\isasymlambda}x{\isachardot}{\kern0pt}\ {\isacharparenleft}{\kern0pt}f\ {\isacharparenleft}{\kern0pt}x\ i{\isacharparenright}{\kern0pt}{\isacharparenright}{\kern0pt}{\isacharparenright}{\kern0pt}{\isachardoublequoteclose}\ {\isacharparenleft}{\kern0pt}\isakeyword{is}\ {\isacharquery}{\kern0pt}A{\isacharparenright}{\kern0pt}\ \isakeyword{and}\isanewline
\ \ \ integral{\isacharunderscore}{\kern0pt}prod{\isacharunderscore}{\kern0pt}pmf{\isacharunderscore}{\kern0pt}slice{\isacharcolon}{\kern0pt}\ {\isachardoublequoteopen}integral\isactrlsup L\ {\isacharparenleft}{\kern0pt}prod{\isacharunderscore}{\kern0pt}pmf\ I\ M{\isacharparenright}{\kern0pt}\ {\isacharparenleft}{\kern0pt}{\isasymlambda}x{\isachardot}{\kern0pt}\ {\isacharparenleft}{\kern0pt}f\ {\isacharparenleft}{\kern0pt}x\ i{\isacharparenright}{\kern0pt}{\isacharparenright}{\kern0pt}{\isacharparenright}{\kern0pt}\ {\isacharequal}{\kern0pt}\ integral\isactrlsup L\ {\isacharparenleft}{\kern0pt}M\ i{\isacharparenright}{\kern0pt}\ f{\isachardoublequoteclose}\ {\isacharparenleft}{\kern0pt}\isakeyword{is}\ {\isacharquery}{\kern0pt}B{\isacharparenright}{\kern0pt}\isanewline
%
\isadelimproof
%
\endisadelimproof
%
\isatagproof
\isacommand{proof}\isamarkupfalse%
\ {\isacharminus}{\kern0pt}\isanewline
\ \ \isacommand{have}\isamarkupfalse%
\ a{\isacharcolon}{\kern0pt}{\isachardoublequoteopen}has{\isacharunderscore}{\kern0pt}bochner{\isacharunderscore}{\kern0pt}integral\ {\isacharparenleft}{\kern0pt}prod{\isacharunderscore}{\kern0pt}pmf\ I\ M{\isacharparenright}{\kern0pt}\ {\isacharparenleft}{\kern0pt}{\isasymlambda}x{\isachardot}{\kern0pt}\ {\isacharparenleft}{\kern0pt}f\ {\isacharparenleft}{\kern0pt}x\ i{\isacharparenright}{\kern0pt}{\isacharparenright}{\kern0pt}{\isacharparenright}{\kern0pt}\ {\isacharparenleft}{\kern0pt}integral\isactrlsup L\ {\isacharparenleft}{\kern0pt}M\ i{\isacharparenright}{\kern0pt}\ f{\isacharparenright}{\kern0pt}{\isachardoublequoteclose}\isanewline
\ \ \ \ \isacommand{apply}\isamarkupfalse%
\ {\isacharparenleft}{\kern0pt}rule\ has{\isacharunderscore}{\kern0pt}bochner{\isacharunderscore}{\kern0pt}integral{\isacharunderscore}{\kern0pt}prod{\isacharunderscore}{\kern0pt}pmf{\isacharunderscore}{\kern0pt}sliceI{\isacharbrackleft}{\kern0pt}OF\ assms{\isacharparenleft}{\kern0pt}{\isadigit{1}}{\isacharparenright}{\kern0pt}\ assms{\isacharparenleft}{\kern0pt}{\isadigit{2}}{\isacharparenright}{\kern0pt}{\isacharbrackright}{\kern0pt}{\isacharparenright}{\kern0pt}\isanewline
\ \ \ \ \isacommand{using}\isamarkupfalse%
\ assms{\isacharparenleft}{\kern0pt}{\isadigit{3}}{\isacharparenright}{\kern0pt}\ \isacommand{by}\isamarkupfalse%
\ {\isacharparenleft}{\kern0pt}simp\ add{\isacharcolon}{\kern0pt}\ has{\isacharunderscore}{\kern0pt}bochner{\isacharunderscore}{\kern0pt}integral{\isacharunderscore}{\kern0pt}iff{\isacharparenright}{\kern0pt}\isanewline
\ \ \isacommand{show}\isamarkupfalse%
\ {\isacharquery}{\kern0pt}A\ \isacommand{using}\isamarkupfalse%
\ a\ has{\isacharunderscore}{\kern0pt}bochner{\isacharunderscore}{\kern0pt}integral{\isacharunderscore}{\kern0pt}iff\ \isacommand{by}\isamarkupfalse%
\ blast\isanewline
\ \ \isacommand{show}\isamarkupfalse%
\ {\isacharquery}{\kern0pt}B\ \isacommand{using}\isamarkupfalse%
\ a\ has{\isacharunderscore}{\kern0pt}bochner{\isacharunderscore}{\kern0pt}integral{\isacharunderscore}{\kern0pt}iff\ \isacommand{by}\isamarkupfalse%
\ blast\isanewline
\isacommand{qed}\isamarkupfalse%
%
\endisatagproof
{\isafoldproof}%
%
\isadelimproof
\isanewline
%
\endisadelimproof
\isanewline
\isanewline
\isacommand{lemma}\isamarkupfalse%
\ variance{\isacharunderscore}{\kern0pt}prod{\isacharunderscore}{\kern0pt}pmf{\isacharunderscore}{\kern0pt}slice{\isacharcolon}{\kern0pt}\isanewline
\ \ \isakeyword{fixes}\ f\ {\isacharcolon}{\kern0pt}{\isacharcolon}{\kern0pt}\ {\isachardoublequoteopen}{\isacharprime}{\kern0pt}a\ {\isasymRightarrow}\ real{\isachardoublequoteclose}\isanewline
\ \ \isakeyword{assumes}\ {\isachardoublequoteopen}i\ {\isasymin}\ I{\isachardoublequoteclose}\ {\isachardoublequoteopen}finite\ I{\isachardoublequoteclose}\isanewline
\ \ \isakeyword{assumes}\ {\isachardoublequoteopen}integrable\ {\isacharparenleft}{\kern0pt}measure{\isacharunderscore}{\kern0pt}pmf\ {\isacharparenleft}{\kern0pt}M\ i{\isacharparenright}{\kern0pt}{\isacharparenright}{\kern0pt}\ {\isacharparenleft}{\kern0pt}{\isasymlambda}{\isasymomega}{\isachardot}{\kern0pt}\ f\ {\isasymomega}{\isacharcircum}{\kern0pt}{\isadigit{2}}{\isacharparenright}{\kern0pt}{\isachardoublequoteclose}\isanewline
\ \ \isakeyword{shows}\ {\isachardoublequoteopen}prob{\isacharunderscore}{\kern0pt}space{\isachardot}{\kern0pt}variance\ {\isacharparenleft}{\kern0pt}prod{\isacharunderscore}{\kern0pt}pmf\ I\ M{\isacharparenright}{\kern0pt}\ {\isacharparenleft}{\kern0pt}{\isasymlambda}{\isasymomega}{\isachardot}{\kern0pt}\ f\ {\isacharparenleft}{\kern0pt}{\isasymomega}\ i{\isacharparenright}{\kern0pt}{\isacharparenright}{\kern0pt}\ {\isacharequal}{\kern0pt}\ prob{\isacharunderscore}{\kern0pt}space{\isachardot}{\kern0pt}variance\ {\isacharparenleft}{\kern0pt}M\ i{\isacharparenright}{\kern0pt}\ f{\isachardoublequoteclose}\isanewline
%
\isadelimproof
%
\endisadelimproof
%
\isatagproof
\isacommand{proof}\isamarkupfalse%
\ {\isacharminus}{\kern0pt}\isanewline
\ \ \isacommand{have}\isamarkupfalse%
\ a{\isacharcolon}{\kern0pt}{\isachardoublequoteopen}integrable\ {\isacharparenleft}{\kern0pt}measure{\isacharunderscore}{\kern0pt}pmf\ {\isacharparenleft}{\kern0pt}M\ i{\isacharparenright}{\kern0pt}{\isacharparenright}{\kern0pt}\ f{\isachardoublequoteclose}\isanewline
\ \ \ \ \isacommand{apply}\isamarkupfalse%
\ {\isacharparenleft}{\kern0pt}rule\ measure{\isacharunderscore}{\kern0pt}pmf{\isachardot}{\kern0pt}square{\isacharunderscore}{\kern0pt}integrable{\isacharunderscore}{\kern0pt}imp{\isacharunderscore}{\kern0pt}integrable{\isacharparenright}{\kern0pt}\isanewline
\ \ \ \ \isacommand{using}\isamarkupfalse%
\ assms{\isacharparenleft}{\kern0pt}{\isadigit{3}}{\isacharparenright}{\kern0pt}\ \isacommand{by}\isamarkupfalse%
\ auto\isanewline
\isanewline
\ \ \isacommand{show}\isamarkupfalse%
\ {\isacharquery}{\kern0pt}thesis\isanewline
\ \ \ \ \isacommand{apply}\isamarkupfalse%
\ {\isacharparenleft}{\kern0pt}subst\ measure{\isacharunderscore}{\kern0pt}pmf{\isachardot}{\kern0pt}variance{\isacharunderscore}{\kern0pt}eq{\isacharparenright}{\kern0pt}\isanewline
\ \ \ \ \ \ \isacommand{apply}\isamarkupfalse%
\ {\isacharparenleft}{\kern0pt}rule\ integrable{\isacharunderscore}{\kern0pt}prod{\isacharunderscore}{\kern0pt}pmf{\isacharunderscore}{\kern0pt}slice{\isacharbrackleft}{\kern0pt}OF\ assms{\isacharparenleft}{\kern0pt}{\isadigit{2}}{\isacharparenright}{\kern0pt}\ assms{\isacharparenleft}{\kern0pt}{\isadigit{1}}{\isacharparenright}{\kern0pt}{\isacharbrackright}{\kern0pt}{\isacharcomma}{\kern0pt}\ metis\ a{\isacharparenright}{\kern0pt}\isanewline
\ \ \ \ \ \isacommand{apply}\isamarkupfalse%
\ {\isacharparenleft}{\kern0pt}rule\ integrable{\isacharunderscore}{\kern0pt}prod{\isacharunderscore}{\kern0pt}pmf{\isacharunderscore}{\kern0pt}slice{\isacharbrackleft}{\kern0pt}OF\ assms{\isacharparenleft}{\kern0pt}{\isadigit{2}}{\isacharparenright}{\kern0pt}\ assms{\isacharparenleft}{\kern0pt}{\isadigit{1}}{\isacharparenright}{\kern0pt}{\isacharbrackright}{\kern0pt}{\isacharcomma}{\kern0pt}\ metis\ assms{\isacharparenleft}{\kern0pt}{\isadigit{3}}{\isacharparenright}{\kern0pt}{\isacharparenright}{\kern0pt}\isanewline
\ \ \ \ \isacommand{apply}\isamarkupfalse%
\ {\isacharparenleft}{\kern0pt}subst\ measure{\isacharunderscore}{\kern0pt}pmf{\isachardot}{\kern0pt}variance{\isacharunderscore}{\kern0pt}eq{\isacharbrackleft}{\kern0pt}OF\ a\ assms{\isacharparenleft}{\kern0pt}{\isadigit{3}}{\isacharparenright}{\kern0pt}{\isacharbrackright}{\kern0pt}{\isacharparenright}{\kern0pt}\isanewline
\ \ \ \ \isacommand{apply}\isamarkupfalse%
\ {\isacharparenleft}{\kern0pt}subst\ integral{\isacharunderscore}{\kern0pt}prod{\isacharunderscore}{\kern0pt}pmf{\isacharunderscore}{\kern0pt}slice{\isacharbrackleft}{\kern0pt}OF\ assms{\isacharparenleft}{\kern0pt}{\isadigit{2}}{\isacharparenright}{\kern0pt}\ assms{\isacharparenleft}{\kern0pt}{\isadigit{1}}{\isacharparenright}{\kern0pt}{\isacharbrackright}{\kern0pt}{\isacharcomma}{\kern0pt}\ metis\ assms{\isacharparenleft}{\kern0pt}{\isadigit{3}}{\isacharparenright}{\kern0pt}{\isacharparenright}{\kern0pt}\isanewline
\ \ \ \ \isacommand{apply}\isamarkupfalse%
\ {\isacharparenleft}{\kern0pt}subst\ integral{\isacharunderscore}{\kern0pt}prod{\isacharunderscore}{\kern0pt}pmf{\isacharunderscore}{\kern0pt}slice{\isacharbrackleft}{\kern0pt}OF\ assms{\isacharparenleft}{\kern0pt}{\isadigit{2}}{\isacharparenright}{\kern0pt}\ assms{\isacharparenleft}{\kern0pt}{\isadigit{1}}{\isacharparenright}{\kern0pt}{\isacharbrackright}{\kern0pt}{\isacharcomma}{\kern0pt}\ metis\ a{\isacharparenright}{\kern0pt}\isanewline
\ \ \ \ \isacommand{by}\isamarkupfalse%
\ simp\isanewline
\isacommand{qed}\isamarkupfalse%
%
\endisatagproof
{\isafoldproof}%
%
\isadelimproof
\isanewline
%
\endisadelimproof
\isanewline
\isacommand{lemma}\isamarkupfalse%
\ PiE{\isacharunderscore}{\kern0pt}defaut{\isacharunderscore}{\kern0pt}undefined{\isacharunderscore}{\kern0pt}eq{\isacharcolon}{\kern0pt}\ {\isachardoublequoteopen}PiE{\isacharunderscore}{\kern0pt}dflt\ I\ undefined\ M\ {\isacharequal}{\kern0pt}\ PiE\ I\ M{\isachardoublequoteclose}\ \isanewline
%
\isadelimproof
\ \ %
\endisadelimproof
%
\isatagproof
\isacommand{apply}\isamarkupfalse%
\ {\isacharparenleft}{\kern0pt}rule\ set{\isacharunderscore}{\kern0pt}eqI{\isacharparenright}{\kern0pt}\isanewline
\ \ \isacommand{apply}\isamarkupfalse%
\ {\isacharparenleft}{\kern0pt}simp\ add{\isacharcolon}{\kern0pt}PiE{\isacharunderscore}{\kern0pt}dflt{\isacharunderscore}{\kern0pt}def\ PiE{\isacharunderscore}{\kern0pt}def\ extensional{\isacharunderscore}{\kern0pt}def\ Pi{\isacharunderscore}{\kern0pt}def{\isacharparenright}{\kern0pt}\ \isacommand{by}\isamarkupfalse%
\ blast%
\endisatagproof
{\isafoldproof}%
%
\isadelimproof
\isanewline
%
\endisadelimproof
\isanewline
\isanewline
\isacommand{lemma}\isamarkupfalse%
\ pmf{\isacharunderscore}{\kern0pt}of{\isacharunderscore}{\kern0pt}set{\isacharunderscore}{\kern0pt}prod{\isacharcolon}{\kern0pt}\isanewline
\ \ \isakeyword{assumes}\ {\isachardoublequoteopen}finite\ I{\isachardoublequoteclose}\isanewline
\ \ \isakeyword{assumes}\ {\isachardoublequoteopen}{\isasymAnd}x{\isachardot}{\kern0pt}\ x\ {\isasymin}\ I\ {\isasymLongrightarrow}\ finite\ {\isacharparenleft}{\kern0pt}M\ x{\isacharparenright}{\kern0pt}{\isachardoublequoteclose}\isanewline
\ \ \isakeyword{assumes}\ {\isachardoublequoteopen}{\isasymAnd}x{\isachardot}{\kern0pt}\ x\ {\isasymin}\ I\ {\isasymLongrightarrow}\ M\ x\ {\isasymnoteq}\ {\isacharbraceleft}{\kern0pt}{\isacharbraceright}{\kern0pt}{\isachardoublequoteclose}\isanewline
\ \ \isakeyword{shows}\ {\isachardoublequoteopen}pmf{\isacharunderscore}{\kern0pt}of{\isacharunderscore}{\kern0pt}set\ {\isacharparenleft}{\kern0pt}PiE\ I\ M{\isacharparenright}{\kern0pt}\ {\isacharequal}{\kern0pt}\ prod{\isacharunderscore}{\kern0pt}pmf\ I\ {\isacharparenleft}{\kern0pt}{\isasymlambda}i{\isachardot}{\kern0pt}\ pmf{\isacharunderscore}{\kern0pt}of{\isacharunderscore}{\kern0pt}set\ {\isacharparenleft}{\kern0pt}M\ i{\isacharparenright}{\kern0pt}{\isacharparenright}{\kern0pt}{\isachardoublequoteclose}\isanewline
%
\isadelimproof
\ \ %
\endisadelimproof
%
\isatagproof
\isacommand{by}\isamarkupfalse%
\ {\isacharparenleft}{\kern0pt}simp\ add{\isacharcolon}{\kern0pt}prod{\isacharunderscore}{\kern0pt}pmf{\isacharunderscore}{\kern0pt}def\ PiE{\isacharunderscore}{\kern0pt}defaut{\isacharunderscore}{\kern0pt}undefined{\isacharunderscore}{\kern0pt}eq\ Pi{\isacharunderscore}{\kern0pt}pmf{\isacharunderscore}{\kern0pt}of{\isacharunderscore}{\kern0pt}set{\isacharbrackleft}{\kern0pt}OF\ assms{\isacharparenleft}{\kern0pt}{\isadigit{1}}{\isacharparenright}{\kern0pt}\ assms{\isacharparenleft}{\kern0pt}{\isadigit{2}}{\isacharparenright}{\kern0pt}\ assms{\isacharparenleft}{\kern0pt}{\isadigit{3}}{\isacharparenright}{\kern0pt}{\isacharbrackright}{\kern0pt}{\isacharparenright}{\kern0pt}%
\endisatagproof
{\isafoldproof}%
%
\isadelimproof
\isanewline
%
\endisadelimproof
\isanewline
\isanewline
\isacommand{lemma}\isamarkupfalse%
\ extensionality{\isacharunderscore}{\kern0pt}iff{\isacharcolon}{\kern0pt}\isanewline
\ \ \isakeyword{assumes}\ {\isachardoublequoteopen}f\ {\isasymin}\ extensional\ I{\isachardoublequoteclose}\isanewline
\ \ \isakeyword{shows}\ {\isachardoublequoteopen}{\isacharparenleft}{\kern0pt}{\isacharparenleft}{\kern0pt}{\isasymlambda}i\ {\isasymin}\ I{\isachardot}{\kern0pt}\ g\ i{\isacharparenright}{\kern0pt}\ {\isacharequal}{\kern0pt}\ f{\isacharparenright}{\kern0pt}\ {\isacharequal}{\kern0pt}\ {\isacharparenleft}{\kern0pt}{\isasymforall}i\ {\isasymin}\ I{\isachardot}{\kern0pt}\ g\ i\ {\isacharequal}{\kern0pt}\ f\ i{\isacharparenright}{\kern0pt}{\isachardoublequoteclose}\isanewline
%
\isadelimproof
\ \ %
\endisadelimproof
%
\isatagproof
\isacommand{using}\isamarkupfalse%
\ assms\ \isacommand{apply}\isamarkupfalse%
\ {\isacharparenleft}{\kern0pt}simp\ add{\isacharcolon}{\kern0pt}extensional{\isacharunderscore}{\kern0pt}def\ restrict{\isacharunderscore}{\kern0pt}def{\isacharparenright}{\kern0pt}\ \isacommand{by}\isamarkupfalse%
\ auto%
\endisatagproof
{\isafoldproof}%
%
\isadelimproof
\isanewline
%
\endisadelimproof
\isanewline
\isacommand{lemma}\isamarkupfalse%
\ of{\isacharunderscore}{\kern0pt}bool{\isacharunderscore}{\kern0pt}prod{\isacharcolon}{\kern0pt}\isanewline
\ \ \isakeyword{assumes}\ {\isachardoublequoteopen}finite\ I{\isachardoublequoteclose}\isanewline
\ \ \isakeyword{shows}\ {\isachardoublequoteopen}of{\isacharunderscore}{\kern0pt}bool\ {\isacharparenleft}{\kern0pt}{\isasymforall}i\ {\isasymin}\ I{\isachardot}{\kern0pt}\ P\ i{\isacharparenright}{\kern0pt}\ {\isacharequal}{\kern0pt}\ {\isacharparenleft}{\kern0pt}{\isasymProd}i\ {\isasymin}\ I{\isachardot}{\kern0pt}\ {\isacharparenleft}{\kern0pt}of{\isacharunderscore}{\kern0pt}bool\ {\isacharparenleft}{\kern0pt}P\ i{\isacharparenright}{\kern0pt}\ {\isacharcolon}{\kern0pt}{\isacharcolon}{\kern0pt}\ {\isacharprime}{\kern0pt}a\ {\isacharcolon}{\kern0pt}{\isacharcolon}{\kern0pt}\ field{\isacharparenright}{\kern0pt}{\isacharparenright}{\kern0pt}{\isachardoublequoteclose}\isanewline
%
\isadelimproof
\ \ %
\endisadelimproof
%
\isatagproof
\isacommand{using}\isamarkupfalse%
\ assms\ \isacommand{by}\isamarkupfalse%
\ {\isacharparenleft}{\kern0pt}induction\ I\ rule{\isacharcolon}{\kern0pt}finite{\isacharunderscore}{\kern0pt}induct{\isacharcomma}{\kern0pt}\ simp{\isacharcomma}{\kern0pt}\ simp{\isacharparenright}{\kern0pt}%
\endisatagproof
{\isafoldproof}%
%
\isadelimproof
\isanewline
%
\endisadelimproof
\isanewline
\isacommand{lemma}\isamarkupfalse%
\ map{\isacharunderscore}{\kern0pt}ptw{\isacharcolon}{\kern0pt}\isanewline
\ \ \isakeyword{fixes}\ I\ {\isacharcolon}{\kern0pt}{\isacharcolon}{\kern0pt}\ {\isachardoublequoteopen}{\isacharprime}{\kern0pt}a\ set{\isachardoublequoteclose}\isanewline
\ \ \isakeyword{fixes}\ M\ {\isacharcolon}{\kern0pt}{\isacharcolon}{\kern0pt}\ {\isachardoublequoteopen}{\isacharprime}{\kern0pt}a\ {\isasymRightarrow}\ {\isacharprime}{\kern0pt}b\ pmf{\isachardoublequoteclose}\ \isanewline
\ \ \isakeyword{fixes}\ f\ {\isacharcolon}{\kern0pt}{\isacharcolon}{\kern0pt}\ {\isachardoublequoteopen}{\isacharprime}{\kern0pt}b\ {\isasymRightarrow}\ {\isacharprime}{\kern0pt}c{\isachardoublequoteclose}\isanewline
\ \ \isakeyword{assumes}\ {\isachardoublequoteopen}finite\ I{\isachardoublequoteclose}\isanewline
\ \ \isakeyword{shows}\ {\isachardoublequoteopen}prod{\isacharunderscore}{\kern0pt}pmf\ I\ M\ {\isasymbind}\ {\isacharparenleft}{\kern0pt}{\isasymlambda}x{\isachardot}{\kern0pt}\ return{\isacharunderscore}{\kern0pt}pmf\ {\isacharparenleft}{\kern0pt}{\isasymlambda}i\ {\isasymin}\ I{\isachardot}{\kern0pt}\ f\ {\isacharparenleft}{\kern0pt}x\ i{\isacharparenright}{\kern0pt}{\isacharparenright}{\kern0pt}{\isacharparenright}{\kern0pt}\ {\isacharequal}{\kern0pt}\ prod{\isacharunderscore}{\kern0pt}pmf\ I\ {\isacharparenleft}{\kern0pt}{\isasymlambda}i{\isachardot}{\kern0pt}\ {\isacharparenleft}{\kern0pt}M\ i\ {\isasymbind}\ {\isacharparenleft}{\kern0pt}{\isasymlambda}x{\isachardot}{\kern0pt}\ return{\isacharunderscore}{\kern0pt}pmf\ {\isacharparenleft}{\kern0pt}f\ x{\isacharparenright}{\kern0pt}{\isacharparenright}{\kern0pt}{\isacharparenright}{\kern0pt}{\isacharparenright}{\kern0pt}{\isachardoublequoteclose}\isanewline
%
\isadelimproof
%
\endisadelimproof
%
\isatagproof
\isacommand{proof}\isamarkupfalse%
\ {\isacharparenleft}{\kern0pt}rule\ pmf{\isacharunderscore}{\kern0pt}eqI{\isacharparenright}{\kern0pt}\isanewline
\ \ \isacommand{fix}\isamarkupfalse%
\ i\ {\isacharcolon}{\kern0pt}{\isacharcolon}{\kern0pt}\ {\isachardoublequoteopen}{\isacharprime}{\kern0pt}a\ {\isasymRightarrow}\ {\isacharprime}{\kern0pt}c{\isachardoublequoteclose}\isanewline
\isanewline
\ \ \isacommand{have}\isamarkupfalse%
\ a{\isacharcolon}{\kern0pt}{\isachardoublequoteopen}{\isasymAnd}x{\isachardot}{\kern0pt}\ i\ {\isasymin}\ extensional\ I\ {\isasymLongrightarrow}\ {\isacharparenleft}{\kern0pt}of{\isacharunderscore}{\kern0pt}bool\ {\isacharparenleft}{\kern0pt}{\isacharparenleft}{\kern0pt}{\isasymlambda}j{\isasymin}I{\isachardot}{\kern0pt}\ f\ {\isacharparenleft}{\kern0pt}x\ j{\isacharparenright}{\kern0pt}{\isacharparenright}{\kern0pt}\ {\isacharequal}{\kern0pt}\ i{\isacharparenright}{\kern0pt}\ {\isacharcolon}{\kern0pt}{\isacharcolon}{\kern0pt}\ real{\isacharparenright}{\kern0pt}\ {\isacharequal}{\kern0pt}\ {\isacharparenleft}{\kern0pt}{\isasymProd}j\ {\isasymin}\ I{\isachardot}{\kern0pt}\ of{\isacharunderscore}{\kern0pt}bool\ {\isacharparenleft}{\kern0pt}f\ {\isacharparenleft}{\kern0pt}x\ j{\isacharparenright}{\kern0pt}\ {\isacharequal}{\kern0pt}\ i\ j{\isacharparenright}{\kern0pt}{\isacharparenright}{\kern0pt}{\isachardoublequoteclose}\isanewline
\ \ \ \ \isacommand{apply}\isamarkupfalse%
\ {\isacharparenleft}{\kern0pt}subst\ extensionality{\isacharunderscore}{\kern0pt}iff{\isacharcomma}{\kern0pt}\ simp{\isacharparenright}{\kern0pt}\isanewline
\ \ \ \ \isacommand{by}\isamarkupfalse%
\ {\isacharparenleft}{\kern0pt}rule\ of{\isacharunderscore}{\kern0pt}bool{\isacharunderscore}{\kern0pt}prod{\isacharbrackleft}{\kern0pt}OF\ assms{\isacharparenleft}{\kern0pt}{\isadigit{1}}{\isacharparenright}{\kern0pt}{\isacharbrackright}{\kern0pt}{\isacharparenright}{\kern0pt}\isanewline
\isanewline
\ \ \isacommand{have}\isamarkupfalse%
\ b{\isacharcolon}{\kern0pt}{\isachardoublequoteopen}{\isasymAnd}x{\isachardot}{\kern0pt}\ i\ {\isasymnotin}\ extensional\ I\ {\isasymLongrightarrow}\ of{\isacharunderscore}{\kern0pt}bool\ {\isacharparenleft}{\kern0pt}{\isacharparenleft}{\kern0pt}{\isasymlambda}j{\isasymin}I{\isachardot}{\kern0pt}\ f\ {\isacharparenleft}{\kern0pt}x\ j{\isacharparenright}{\kern0pt}{\isacharparenright}{\kern0pt}\ {\isacharequal}{\kern0pt}\ i{\isacharparenright}{\kern0pt}\ {\isacharequal}{\kern0pt}\ {\isadigit{0}}{\isachardoublequoteclose}\isanewline
\ \ \ \ \isacommand{by}\isamarkupfalse%
\ auto\isanewline
\isanewline
\ \ \isacommand{show}\isamarkupfalse%
\ {\isachardoublequoteopen}pmf\ {\isacharparenleft}{\kern0pt}prod{\isacharunderscore}{\kern0pt}pmf\ I\ M\ {\isasymbind}\ {\isacharparenleft}{\kern0pt}{\isasymlambda}x{\isachardot}{\kern0pt}\ return{\isacharunderscore}{\kern0pt}pmf\ {\isacharparenleft}{\kern0pt}{\isasymlambda}i{\isasymin}I{\isachardot}{\kern0pt}\ f\ {\isacharparenleft}{\kern0pt}x\ i{\isacharparenright}{\kern0pt}{\isacharparenright}{\kern0pt}{\isacharparenright}{\kern0pt}{\isacharparenright}{\kern0pt}\ i\ {\isacharequal}{\kern0pt}\ pmf\ {\isacharparenleft}{\kern0pt}prod{\isacharunderscore}{\kern0pt}pmf\ I\ {\isacharparenleft}{\kern0pt}{\isasymlambda}i{\isachardot}{\kern0pt}\ M\ i\ {\isasymbind}\ {\isacharparenleft}{\kern0pt}{\isasymlambda}x{\isachardot}{\kern0pt}\ return{\isacharunderscore}{\kern0pt}pmf\ {\isacharparenleft}{\kern0pt}f\ x{\isacharparenright}{\kern0pt}{\isacharparenright}{\kern0pt}{\isacharparenright}{\kern0pt}{\isacharparenright}{\kern0pt}\ i{\isachardoublequoteclose}\isanewline
\ \ \isacommand{apply}\isamarkupfalse%
\ {\isacharparenleft}{\kern0pt}subst\ pmf{\isacharunderscore}{\kern0pt}bind{\isacharparenright}{\kern0pt}\isanewline
\ \ \isacommand{apply}\isamarkupfalse%
\ {\isacharparenleft}{\kern0pt}subst\ pmf{\isacharunderscore}{\kern0pt}prod{\isacharunderscore}{\kern0pt}pmf{\isacharparenright}{\kern0pt}\ \isacommand{defer}\isamarkupfalse%
\isanewline
\ \ \isacommand{apply}\isamarkupfalse%
\ {\isacharparenleft}{\kern0pt}subst\ pmf{\isacharunderscore}{\kern0pt}bind{\isacharparenright}{\kern0pt}\isanewline
\ \ \ \isacommand{apply}\isamarkupfalse%
\ {\isacharparenleft}{\kern0pt}simp\ add{\isacharcolon}{\kern0pt}indicator{\isacharunderscore}{\kern0pt}def{\isacharparenright}{\kern0pt}\isanewline
\ \ \ \isacommand{apply}\isamarkupfalse%
\ {\isacharparenleft}{\kern0pt}rule\ conjI{\isacharcomma}{\kern0pt}\ rule\ impI{\isacharparenright}{\kern0pt}\isanewline
\ \ \ \ \ \ \isacommand{apply}\isamarkupfalse%
\ {\isacharparenleft}{\kern0pt}subst\ a{\isacharcomma}{\kern0pt}\ simp{\isacharparenright}{\kern0pt}\isanewline
\ \ \ \ \ \ \isacommand{apply}\isamarkupfalse%
\ {\isacharparenleft}{\kern0pt}subst\ prod{\isacharunderscore}{\kern0pt}pmf{\isacharunderscore}{\kern0pt}integral{\isacharbrackleft}{\kern0pt}OF\ assms{\isacharparenleft}{\kern0pt}{\isadigit{1}}{\isacharparenright}{\kern0pt}{\isacharbrackright}{\kern0pt}{\isacharparenright}{\kern0pt}\isanewline
\ \ \ \ \ \ \ \isacommand{apply}\isamarkupfalse%
\ {\isacharparenleft}{\kern0pt}rule\ finite{\isacharunderscore}{\kern0pt}measure{\isachardot}{\kern0pt}integrable{\isacharunderscore}{\kern0pt}const{\isacharunderscore}{\kern0pt}bound{\isacharbrackleft}{\kern0pt}\isakeyword{where}\ B{\isacharequal}{\kern0pt}{\isachardoublequoteopen}{\isadigit{1}}{\isachardoublequoteclose}{\isacharbrackright}{\kern0pt}{\isacharcomma}{\kern0pt}\ simp{\isacharcomma}{\kern0pt}\ simp{\isacharcomma}{\kern0pt}\ simp{\isacharcomma}{\kern0pt}\ simp{\isacharparenright}{\kern0pt}\isanewline
\ \ \ \ \isacommand{by}\isamarkupfalse%
\ {\isacharparenleft}{\kern0pt}simp\ add{\isacharcolon}{\kern0pt}b{\isacharcomma}{\kern0pt}\ metis\ assms{\isacharparenleft}{\kern0pt}{\isadigit{1}}{\isacharparenright}{\kern0pt}{\isacharparenright}{\kern0pt}\isanewline
\isacommand{qed}\isamarkupfalse%
%
\endisatagproof
{\isafoldproof}%
%
\isadelimproof
\isanewline
%
\endisadelimproof
\isanewline
\isacommand{lemma}\isamarkupfalse%
\ pair{\isacharunderscore}{\kern0pt}pmfI{\isacharcolon}{\kern0pt}\isanewline
\ \ {\isachardoublequoteopen}A\ {\isasymbind}\ {\isacharparenleft}{\kern0pt}{\isasymlambda}a{\isachardot}{\kern0pt}\ B\ {\isasymbind}\ {\isacharparenleft}{\kern0pt}{\isasymlambda}b{\isachardot}{\kern0pt}\ return{\isacharunderscore}{\kern0pt}pmf\ {\isacharparenleft}{\kern0pt}f\ a\ b{\isacharparenright}{\kern0pt}{\isacharparenright}{\kern0pt}{\isacharparenright}{\kern0pt}\ {\isacharequal}{\kern0pt}\ pair{\isacharunderscore}{\kern0pt}pmf\ A\ B\ {\isasymbind}\ {\isacharparenleft}{\kern0pt}{\isasymlambda}{\isacharparenleft}{\kern0pt}a{\isacharcomma}{\kern0pt}b{\isacharparenright}{\kern0pt}{\isachardot}{\kern0pt}\ return{\isacharunderscore}{\kern0pt}pmf\ {\isacharparenleft}{\kern0pt}f\ a\ b{\isacharparenright}{\kern0pt}{\isacharparenright}{\kern0pt}{\isachardoublequoteclose}\isanewline
%
\isadelimproof
\ \ %
\endisadelimproof
%
\isatagproof
\isacommand{apply}\isamarkupfalse%
\ {\isacharparenleft}{\kern0pt}simp\ add{\isacharcolon}{\kern0pt}pair{\isacharunderscore}{\kern0pt}pmf{\isacharunderscore}{\kern0pt}def{\isacharparenright}{\kern0pt}\isanewline
\ \ \isacommand{apply}\isamarkupfalse%
\ {\isacharparenleft}{\kern0pt}subst\ bind{\isacharunderscore}{\kern0pt}assoc{\isacharunderscore}{\kern0pt}pmf{\isacharparenright}{\kern0pt}\isanewline
\ \ \isacommand{apply}\isamarkupfalse%
\ {\isacharparenleft}{\kern0pt}subst\ bind{\isacharunderscore}{\kern0pt}assoc{\isacharunderscore}{\kern0pt}pmf{\isacharparenright}{\kern0pt}\isanewline
\ \ \isacommand{by}\isamarkupfalse%
\ {\isacharparenleft}{\kern0pt}simp\ add{\isacharcolon}{\kern0pt}bind{\isacharunderscore}{\kern0pt}return{\isacharunderscore}{\kern0pt}pmf{\isacharparenright}{\kern0pt}%
\endisatagproof
{\isafoldproof}%
%
\isadelimproof
\isanewline
%
\endisadelimproof
\isanewline
\isacommand{lemma}\isamarkupfalse%
\ pmf{\isacharunderscore}{\kern0pt}pair{\isacharprime}{\kern0pt}{\isacharcolon}{\kern0pt}\isanewline
\ \ {\isachardoublequoteopen}pmf\ {\isacharparenleft}{\kern0pt}pair{\isacharunderscore}{\kern0pt}pmf\ M\ N{\isacharparenright}{\kern0pt}\ x\ {\isacharequal}{\kern0pt}\ pmf\ M\ {\isacharparenleft}{\kern0pt}fst\ x{\isacharparenright}{\kern0pt}\ {\isacharasterisk}{\kern0pt}\ pmf\ N\ {\isacharparenleft}{\kern0pt}snd\ x{\isacharparenright}{\kern0pt}{\isachardoublequoteclose}\isanewline
%
\isadelimproof
\ \ %
\endisadelimproof
%
\isatagproof
\isacommand{by}\isamarkupfalse%
\ {\isacharparenleft}{\kern0pt}cases\ x{\isacharcomma}{\kern0pt}simp\ add{\isacharcolon}{\kern0pt}pmf{\isacharunderscore}{\kern0pt}pair{\isacharparenright}{\kern0pt}%
\endisatagproof
{\isafoldproof}%
%
\isadelimproof
\isanewline
%
\endisadelimproof
\isanewline
\isacommand{lemma}\isamarkupfalse%
\ pair{\isacharunderscore}{\kern0pt}pmf{\isacharunderscore}{\kern0pt}ptw{\isacharcolon}{\kern0pt}\isanewline
\ \ \isakeyword{assumes}\ {\isachardoublequoteopen}finite\ I{\isachardoublequoteclose}\isanewline
\ \ \isakeyword{shows}\ {\isachardoublequoteopen}pair{\isacharunderscore}{\kern0pt}pmf\ {\isacharparenleft}{\kern0pt}prod{\isacharunderscore}{\kern0pt}pmf\ I\ A\ {\isacharcolon}{\kern0pt}{\isacharcolon}{\kern0pt}\ {\isacharparenleft}{\kern0pt}{\isacharparenleft}{\kern0pt}{\isacharprime}{\kern0pt}i\ {\isasymRightarrow}\ {\isacharprime}{\kern0pt}a{\isacharparenright}{\kern0pt}\ pmf{\isacharparenright}{\kern0pt}{\isacharparenright}{\kern0pt}\ {\isacharparenleft}{\kern0pt}prod{\isacharunderscore}{\kern0pt}pmf\ I\ B\ {\isacharcolon}{\kern0pt}{\isacharcolon}{\kern0pt}\ {\isacharparenleft}{\kern0pt}{\isacharparenleft}{\kern0pt}{\isacharprime}{\kern0pt}i\ {\isasymRightarrow}\ {\isacharprime}{\kern0pt}b{\isacharparenright}{\kern0pt}\ pmf{\isacharparenright}{\kern0pt}{\isacharparenright}{\kern0pt}\ {\isacharequal}{\kern0pt}\ \isanewline
\ \ \ \ prod{\isacharunderscore}{\kern0pt}pmf\ I\ {\isacharparenleft}{\kern0pt}{\isasymlambda}i{\isachardot}{\kern0pt}\ pair{\isacharunderscore}{\kern0pt}pmf\ {\isacharparenleft}{\kern0pt}A\ i{\isacharparenright}{\kern0pt}\ {\isacharparenleft}{\kern0pt}B\ i{\isacharparenright}{\kern0pt}{\isacharparenright}{\kern0pt}\ {\isasymbind}\ \isanewline
\ \ \ \ \ \ {\isacharparenleft}{\kern0pt}{\isasymlambda}f{\isachardot}{\kern0pt}\ return{\isacharunderscore}{\kern0pt}pmf\ {\isacharparenleft}{\kern0pt}restrict\ {\isacharparenleft}{\kern0pt}fst\ {\isasymcirc}\ f{\isacharparenright}{\kern0pt}\ I{\isacharcomma}{\kern0pt}\ restrict\ {\isacharparenleft}{\kern0pt}snd\ {\isasymcirc}\ f{\isacharparenright}{\kern0pt}\ I{\isacharparenright}{\kern0pt}{\isacharparenright}{\kern0pt}{\isachardoublequoteclose}\isanewline
\ \ \ \ {\isacharparenleft}{\kern0pt}\isakeyword{is}\ {\isachardoublequoteopen}{\isacharquery}{\kern0pt}lhs\ {\isacharequal}{\kern0pt}\ {\isacharquery}{\kern0pt}rhs{\isachardoublequoteclose}{\isacharparenright}{\kern0pt}\isanewline
%
\isadelimproof
%
\endisadelimproof
%
\isatagproof
\isacommand{proof}\isamarkupfalse%
\ {\isacharminus}{\kern0pt}\isanewline
\ \ \isacommand{define}\isamarkupfalse%
\ h\ \isakeyword{where}\ {\isachardoublequoteopen}h\ {\isacharequal}{\kern0pt}\ {\isacharparenleft}{\kern0pt}{\isasymlambda}f\ x{\isachardot}{\kern0pt}\ \isanewline
\ \ \ \ if\ x\ {\isasymin}\ I\ then\ \isanewline
\ \ \ \ \ \ f\ x\ \isanewline
\ \ \ \ else\ {\isacharparenleft}{\kern0pt}\isanewline
\ \ \ \ \ \ if\ {\isacharparenleft}{\kern0pt}f\ x{\isacharparenright}{\kern0pt}\ {\isacharequal}{\kern0pt}\ undefined\ then\ \isanewline
\ \ \ \ \ \ \ \ {\isacharparenleft}{\kern0pt}undefined\ {\isacharcolon}{\kern0pt}{\isacharcolon}{\kern0pt}\ {\isacharprime}{\kern0pt}a{\isacharcomma}{\kern0pt}undefined\ {\isacharcolon}{\kern0pt}{\isacharcolon}{\kern0pt}\ {\isacharprime}{\kern0pt}b{\isacharparenright}{\kern0pt}\isanewline
\ \ \ \ \ \ else\ {\isacharparenleft}{\kern0pt}\isanewline
\ \ \ \ \ \ \ \ if\ {\isacharparenleft}{\kern0pt}f\ x{\isacharparenright}{\kern0pt}\ {\isacharequal}{\kern0pt}\ {\isacharparenleft}{\kern0pt}undefined{\isacharcomma}{\kern0pt}undefined{\isacharparenright}{\kern0pt}\ then\ \isanewline
\ \ \ \ \ \ \ \ \ \ undefined\ \isanewline
\ \ \ \ \ \ \ \ else\isanewline
\ \ \ \ \ \ \ \ \ \ f\ x{\isacharparenright}{\kern0pt}{\isacharparenright}{\kern0pt}{\isacharparenright}{\kern0pt}{\isachardoublequoteclose}\isanewline
\isanewline
\ \ \isacommand{have}\isamarkupfalse%
\ h{\isacharunderscore}{\kern0pt}h{\isacharunderscore}{\kern0pt}id{\isacharcolon}{\kern0pt}\ {\isachardoublequoteopen}{\isasymAnd}f{\isachardot}{\kern0pt}\ h\ {\isacharparenleft}{\kern0pt}h\ f{\isacharparenright}{\kern0pt}\ {\isacharequal}{\kern0pt}\ f{\isachardoublequoteclose}\isanewline
\ \ \ \ \isacommand{apply}\isamarkupfalse%
\ {\isacharparenleft}{\kern0pt}rule\ ext{\isacharparenright}{\kern0pt}\isanewline
\ \ \ \ \isacommand{by}\isamarkupfalse%
\ {\isacharparenleft}{\kern0pt}simp\ add{\isacharcolon}{\kern0pt}h{\isacharunderscore}{\kern0pt}def{\isacharparenright}{\kern0pt}\isanewline
\ \ \isanewline
\ \ \isacommand{have}\isamarkupfalse%
\ b{\isacharcolon}{\kern0pt}{\isachardoublequoteopen}{\isasymAnd}i\ g{\isachardot}{\kern0pt}\ i\ {\isasymin}\ I\ {\isasymLongrightarrow}\ h\ g\ i\ {\isacharequal}{\kern0pt}\ g\ i{\isachardoublequoteclose}\ \isanewline
\ \ \ \ \isacommand{by}\isamarkupfalse%
\ {\isacharparenleft}{\kern0pt}simp\ add{\isacharcolon}{\kern0pt}h{\isacharunderscore}{\kern0pt}def{\isacharparenright}{\kern0pt}\isanewline
\isanewline
\ \ \isacommand{have}\isamarkupfalse%
\ a{\isacharcolon}{\kern0pt}{\isachardoublequoteopen}inj\ {\isacharparenleft}{\kern0pt}{\isasymlambda}f{\isachardot}{\kern0pt}\ {\isacharparenleft}{\kern0pt}fst\ {\isasymcirc}\ h\ f{\isacharcomma}{\kern0pt}\ snd\ {\isasymcirc}\ h\ f{\isacharparenright}{\kern0pt}{\isacharparenright}{\kern0pt}{\isachardoublequoteclose}\isanewline
\ \ \isacommand{proof}\isamarkupfalse%
\ {\isacharparenleft}{\kern0pt}rule\ injI{\isacharparenright}{\kern0pt}\isanewline
\ \ \ \ \isacommand{fix}\isamarkupfalse%
\ x\ y\isanewline
\ \ \ \ \isacommand{assume}\isamarkupfalse%
\ {\isachardoublequoteopen}{\isacharparenleft}{\kern0pt}fst\ {\isasymcirc}\ h\ x{\isacharcomma}{\kern0pt}\ snd\ {\isasymcirc}\ h\ x{\isacharparenright}{\kern0pt}\ {\isacharequal}{\kern0pt}\ {\isacharparenleft}{\kern0pt}fst\ {\isasymcirc}\ h\ y{\isacharcomma}{\kern0pt}\ snd\ {\isasymcirc}\ h\ y{\isacharparenright}{\kern0pt}{\isachardoublequoteclose}\isanewline
\ \ \ \ \isacommand{hence}\isamarkupfalse%
\ a{\isadigit{1}}{\isacharcolon}{\kern0pt}{\isachardoublequoteopen}h\ x\ {\isacharequal}{\kern0pt}\ h\ y{\isachardoublequoteclose}\isanewline
\ \ \ \ \ \ \isacommand{by}\isamarkupfalse%
\ {\isacharparenleft}{\kern0pt}simp{\isacharcomma}{\kern0pt}\ metis\ convol{\isacharunderscore}{\kern0pt}expand{\isacharunderscore}{\kern0pt}snd{\isacharparenright}{\kern0pt}\isanewline
\ \ \ \ \isacommand{show}\isamarkupfalse%
\ {\isachardoublequoteopen}x\ {\isacharequal}{\kern0pt}\ y{\isachardoublequoteclose}\isanewline
\ \ \ \ \ \ \isacommand{apply}\isamarkupfalse%
\ {\isacharparenleft}{\kern0pt}rule\ ext{\isacharparenright}{\kern0pt}\isanewline
\ \ \ \ \ \ \isacommand{using}\isamarkupfalse%
\ a{\isadigit{1}}\ \isacommand{apply}\isamarkupfalse%
\ {\isacharparenleft}{\kern0pt}simp\ add{\isacharcolon}{\kern0pt}h{\isacharunderscore}{\kern0pt}def{\isacharparenright}{\kern0pt}\ \isanewline
\ \ \ \ \ \ \isacommand{by}\isamarkupfalse%
\ {\isacharparenleft}{\kern0pt}metis\ {\isacharparenleft}{\kern0pt}no{\isacharunderscore}{\kern0pt}types{\isacharcomma}{\kern0pt}\ opaque{\isacharunderscore}{\kern0pt}lifting{\isacharparenright}{\kern0pt}{\isacharparenright}{\kern0pt}\isanewline
\ \ \isacommand{qed}\isamarkupfalse%
\isanewline
\isanewline
\ \ \isacommand{have}\isamarkupfalse%
\ c{\isacharcolon}{\kern0pt}{\isachardoublequoteopen}{\isasymAnd}g{\isachardot}{\kern0pt}\ {\isacharparenleft}{\kern0pt}fst\ {\isasymcirc}\ h\ g\ {\isasymin}\ extensional\ I\ {\isasymand}\ snd\ {\isasymcirc}\ h\ g\ {\isasymin}\ extensional\ I{\isacharparenright}{\kern0pt}\ {\isacharequal}{\kern0pt}\ {\isacharparenleft}{\kern0pt}g\ {\isasymin}\ extensional\ I{\isacharparenright}{\kern0pt}{\isachardoublequoteclose}\ \isanewline
\ \ \ \ \isacommand{apply}\isamarkupfalse%
\ {\isacharparenleft}{\kern0pt}rule\ order{\isacharunderscore}{\kern0pt}antisym{\isacharparenright}{\kern0pt}\isanewline
\ \ \ \ \isacommand{apply}\isamarkupfalse%
\ {\isacharparenleft}{\kern0pt}simp\ add{\isacharcolon}{\kern0pt}h{\isacharunderscore}{\kern0pt}def\ extensional{\isacharunderscore}{\kern0pt}def{\isacharparenright}{\kern0pt}\ \isanewline
\ \ \ \ \ \isacommand{apply}\isamarkupfalse%
\ {\isacharparenleft}{\kern0pt}metis\ prod{\isachardot}{\kern0pt}collapse{\isacharparenright}{\kern0pt}\isanewline
\ \ \ \ \isacommand{by}\isamarkupfalse%
\ {\isacharparenleft}{\kern0pt}simp\ add{\isacharcolon}{\kern0pt}h{\isacharunderscore}{\kern0pt}def\ extensional{\isacharunderscore}{\kern0pt}def{\isacharparenright}{\kern0pt}\ \isanewline
\isanewline
\ \ \isacommand{have}\isamarkupfalse%
\ {\isachardoublequoteopen}pair{\isacharunderscore}{\kern0pt}pmf\ {\isacharparenleft}{\kern0pt}prod{\isacharunderscore}{\kern0pt}pmf\ I\ A\ {\isacharcolon}{\kern0pt}{\isacharcolon}{\kern0pt}\ {\isacharparenleft}{\kern0pt}{\isacharparenleft}{\kern0pt}{\isacharprime}{\kern0pt}i\ {\isasymRightarrow}\ {\isacharprime}{\kern0pt}a{\isacharparenright}{\kern0pt}\ pmf{\isacharparenright}{\kern0pt}{\isacharparenright}{\kern0pt}\ {\isacharparenleft}{\kern0pt}prod{\isacharunderscore}{\kern0pt}pmf\ I\ B\ {\isacharcolon}{\kern0pt}{\isacharcolon}{\kern0pt}\ {\isacharparenleft}{\kern0pt}{\isacharparenleft}{\kern0pt}{\isacharprime}{\kern0pt}i\ {\isasymRightarrow}\ {\isacharprime}{\kern0pt}b{\isacharparenright}{\kern0pt}\ pmf{\isacharparenright}{\kern0pt}{\isacharparenright}{\kern0pt}\ {\isacharequal}{\kern0pt}\ prod{\isacharunderscore}{\kern0pt}pmf\ I\ {\isacharparenleft}{\kern0pt}{\isasymlambda}i{\isachardot}{\kern0pt}\ pair{\isacharunderscore}{\kern0pt}pmf\ {\isacharparenleft}{\kern0pt}A\ i{\isacharparenright}{\kern0pt}\ {\isacharparenleft}{\kern0pt}B\ i{\isacharparenright}{\kern0pt}{\isacharparenright}{\kern0pt}\ {\isasymbind}\isanewline
\ \ \ \ \ \ {\isacharparenleft}{\kern0pt}{\isasymlambda}f{\isachardot}{\kern0pt}\ return{\isacharunderscore}{\kern0pt}pmf\ {\isacharparenleft}{\kern0pt}fst\ {\isasymcirc}\ h\ f{\isacharcomma}{\kern0pt}\ snd\ {\isasymcirc}\ h\ f{\isacharparenright}{\kern0pt}{\isacharparenright}{\kern0pt}{\isachardoublequoteclose}\isanewline
\ \ \isacommand{proof}\isamarkupfalse%
\ {\isacharparenleft}{\kern0pt}rule\ pmf{\isacharunderscore}{\kern0pt}eqI{\isacharparenright}{\kern0pt}\isanewline
\ \ \ \ \isacommand{fix}\isamarkupfalse%
\ f\isanewline
\ \ \ \ \isacommand{define}\isamarkupfalse%
\ g\ \isakeyword{where}\ {\isachardoublequoteopen}g\ {\isacharequal}{\kern0pt}\ h\ {\isacharparenleft}{\kern0pt}{\isasymlambda}i{\isachardot}{\kern0pt}\ {\isacharparenleft}{\kern0pt}fst\ f\ i{\isacharcomma}{\kern0pt}\ snd\ f\ i{\isacharparenright}{\kern0pt}{\isacharparenright}{\kern0pt}{\isachardoublequoteclose}\isanewline
\ \ \ \ \isacommand{hence}\isamarkupfalse%
\ \ g{\isacharunderscore}{\kern0pt}rev{\isacharcolon}{\kern0pt}\ {\isachardoublequoteopen}f\ {\isacharequal}{\kern0pt}\ {\isacharparenleft}{\kern0pt}{\isasymlambda}f{\isachardot}{\kern0pt}\ {\isacharparenleft}{\kern0pt}fst\ {\isasymcirc}\ h\ f{\isacharcomma}{\kern0pt}\ snd\ {\isasymcirc}\ h\ f{\isacharparenright}{\kern0pt}{\isacharparenright}{\kern0pt}\ g{\isachardoublequoteclose}\ \isanewline
\ \ \ \ \ \ \isacommand{by}\isamarkupfalse%
\ {\isacharparenleft}{\kern0pt}simp\ add{\isacharcolon}{\kern0pt}comp{\isacharunderscore}{\kern0pt}def\ h{\isacharunderscore}{\kern0pt}h{\isacharunderscore}{\kern0pt}id{\isacharparenright}{\kern0pt}\ \isanewline
\ \ \ \ \isacommand{show}\isamarkupfalse%
\ {\isachardoublequoteopen}\ pmf\ {\isacharparenleft}{\kern0pt}pair{\isacharunderscore}{\kern0pt}pmf\ {\isacharparenleft}{\kern0pt}prod{\isacharunderscore}{\kern0pt}pmf\ I\ A{\isacharparenright}{\kern0pt}\ {\isacharparenleft}{\kern0pt}prod{\isacharunderscore}{\kern0pt}pmf\ I\ B{\isacharparenright}{\kern0pt}{\isacharparenright}{\kern0pt}\ f\ {\isacharequal}{\kern0pt}\isanewline
\ \ \ \ \ \ \ \ \ pmf\ {\isacharparenleft}{\kern0pt}prod{\isacharunderscore}{\kern0pt}pmf\ I\ {\isacharparenleft}{\kern0pt}{\isasymlambda}i{\isachardot}{\kern0pt}\ pair{\isacharunderscore}{\kern0pt}pmf\ {\isacharparenleft}{\kern0pt}A\ i{\isacharparenright}{\kern0pt}\ {\isacharparenleft}{\kern0pt}B\ i{\isacharparenright}{\kern0pt}{\isacharparenright}{\kern0pt}\ {\isasymbind}\ {\isacharparenleft}{\kern0pt}{\isasymlambda}f{\isachardot}{\kern0pt}\ return{\isacharunderscore}{\kern0pt}pmf\ {\isacharparenleft}{\kern0pt}fst\ {\isasymcirc}\ h\ f{\isacharcomma}{\kern0pt}\ snd\ {\isasymcirc}\ h\ f{\isacharparenright}{\kern0pt}{\isacharparenright}{\kern0pt}{\isacharparenright}{\kern0pt}\ f{\isachardoublequoteclose}\isanewline
\ \ \ \ \ \ \isacommand{apply}\isamarkupfalse%
\ {\isacharparenleft}{\kern0pt}subst\ map{\isacharunderscore}{\kern0pt}pmf{\isacharunderscore}{\kern0pt}def{\isacharbrackleft}{\kern0pt}symmetric{\isacharbrackright}{\kern0pt}{\isacharcomma}{\kern0pt}\ simp\ add{\isacharcolon}{\kern0pt}\ g{\isacharunderscore}{\kern0pt}rev{\isacharcomma}{\kern0pt}\ subst\ pmf{\isacharunderscore}{\kern0pt}map{\isacharunderscore}{\kern0pt}inj{\isacharprime}{\kern0pt}{\isacharcomma}{\kern0pt}\ metis\ a{\isacharparenright}{\kern0pt}\isanewline
\ \ \ \ \ \ \isacommand{apply}\isamarkupfalse%
\ {\isacharparenleft}{\kern0pt}simp\ add{\isacharcolon}{\kern0pt}pmf{\isacharunderscore}{\kern0pt}pair{\isacharprime}{\kern0pt}\ pmf{\isacharunderscore}{\kern0pt}prod{\isacharunderscore}{\kern0pt}pmf{\isacharbrackleft}{\kern0pt}OF\ assms{\isacharparenleft}{\kern0pt}{\isadigit{1}}{\isacharparenright}{\kern0pt}{\isacharbrackright}{\kern0pt}\ b\ prod{\isachardot}{\kern0pt}distrib{\isacharparenright}{\kern0pt}\isanewline
\ \ \ \ \ \ \isacommand{using}\isamarkupfalse%
\ c\ \isacommand{by}\isamarkupfalse%
\ blast\isanewline
\ \ \isacommand{qed}\isamarkupfalse%
\isanewline
\ \ \isacommand{also}\isamarkupfalse%
\ \isacommand{have}\isamarkupfalse%
\ {\isachardoublequoteopen}{\isachardot}{\kern0pt}{\isachardot}{\kern0pt}{\isachardot}{\kern0pt}\ {\isacharequal}{\kern0pt}\ {\isacharquery}{\kern0pt}rhs{\isachardoublequoteclose}\isanewline
\ \ \ \ \isacommand{apply}\isamarkupfalse%
\ {\isacharparenleft}{\kern0pt}rule\ bind{\isacharunderscore}{\kern0pt}pmf{\isacharunderscore}{\kern0pt}cong\ {\isacharcomma}{\kern0pt}simp{\isacharparenright}{\kern0pt}\isanewline
\ \ \ \ \isacommand{apply}\isamarkupfalse%
\ {\isacharparenleft}{\kern0pt}simp\ add{\isacharcolon}{\kern0pt}\ h{\isacharunderscore}{\kern0pt}def\ comp{\isacharunderscore}{\kern0pt}def\ set{\isacharunderscore}{\kern0pt}prod{\isacharunderscore}{\kern0pt}pmf{\isacharbrackleft}{\kern0pt}OF\ assms{\isacharparenleft}{\kern0pt}{\isadigit{1}}{\isacharparenright}{\kern0pt}{\isacharbrackright}{\kern0pt}\ PiE{\isacharunderscore}{\kern0pt}iff\ extensional{\isacharunderscore}{\kern0pt}def\ restrict{\isacharunderscore}{\kern0pt}def{\isacharparenright}{\kern0pt}\isanewline
\ \ \ \ \isacommand{apply}\isamarkupfalse%
\ {\isacharparenleft}{\kern0pt}rule\ conjI{\isacharparenright}{\kern0pt}\isanewline
\ \ \ \ \isacommand{by}\isamarkupfalse%
{\isacharparenleft}{\kern0pt}rule\ ext{\isacharcomma}{\kern0pt}\ simp{\isacharparenright}{\kern0pt}{\isacharplus}{\kern0pt}\isanewline
\ \ \isacommand{finally}\isamarkupfalse%
\ \isacommand{show}\isamarkupfalse%
\ {\isacharquery}{\kern0pt}thesis\ \isanewline
\ \ \ \ \isacommand{by}\isamarkupfalse%
\ blast\isanewline
\isacommand{qed}\isamarkupfalse%
%
\endisatagproof
{\isafoldproof}%
%
\isadelimproof
\isanewline
%
\endisadelimproof
%
\isadelimtheory
\isanewline
%
\endisadelimtheory
%
\isatagtheory
\isacommand{end}\isamarkupfalse%
%
\endisatagtheory
{\isafoldtheory}%
%
\isadelimtheory
%
\endisadelimtheory
%
\end{isabellebody}%
\endinput
%:%file=Product_PMF_Ext.tex%:%
%:%11=1%:%
%:%23=3%:%
%:%24=4%:%
%:%32=6%:%
%:%33=6%:%
%:%34=7%:%
%:%35=8%:%
%:%42=8%:%
%:%43=9%:%
%:%44=10%:%
%:%45=10%:%
%:%46=11%:%
%:%47=12%:%
%:%48=12%:%
%:%49=13%:%
%:%50=14%:%
%:%53=15%:%
%:%57=15%:%
%:%58=15%:%
%:%63=15%:%
%:%66=16%:%
%:%67=17%:%
%:%68=17%:%
%:%69=18%:%
%:%70=19%:%
%:%73=20%:%
%:%77=20%:%
%:%78=20%:%
%:%79=21%:%
%:%80=21%:%
%:%81=22%:%
%:%82=22%:%
%:%87=22%:%
%:%90=23%:%
%:%91=24%:%
%:%92=24%:%
%:%95=25%:%
%:%99=25%:%
%:%100=25%:%
%:%101=25%:%
%:%106=25%:%
%:%109=26%:%
%:%110=27%:%
%:%111=27%:%
%:%112=28%:%
%:%113=29%:%
%:%116=30%:%
%:%120=30%:%
%:%121=30%:%
%:%122=31%:%
%:%123=31%:%
%:%128=31%:%
%:%131=32%:%
%:%132=33%:%
%:%133=33%:%
%:%134=34%:%
%:%135=35%:%
%:%136=36%:%
%:%143=37%:%
%:%144=37%:%
%:%145=38%:%
%:%146=38%:%
%:%147=39%:%
%:%148=39%:%
%:%149=40%:%
%:%150=40%:%
%:%151=41%:%
%:%152=41%:%
%:%153=41%:%
%:%154=42%:%
%:%155=42%:%
%:%156=43%:%
%:%157=43%:%
%:%158=44%:%
%:%159=44%:%
%:%160=45%:%
%:%161=45%:%
%:%162=45%:%
%:%163=46%:%
%:%169=46%:%
%:%172=47%:%
%:%173=48%:%
%:%174=48%:%
%:%175=49%:%
%:%176=50%:%
%:%177=51%:%
%:%180=52%:%
%:%184=52%:%
%:%185=52%:%
%:%186=53%:%
%:%187=53%:%
%:%192=53%:%
%:%195=54%:%
%:%196=55%:%
%:%197=55%:%
%:%200=56%:%
%:%204=56%:%
%:%205=56%:%
%:%206=57%:%
%:%207=57%:%
%:%216=59%:%
%:%217=60%:%
%:%219=61%:%
%:%220=61%:%
%:%221=62%:%
%:%222=63%:%
%:%229=64%:%
%:%230=64%:%
%:%231=65%:%
%:%232=65%:%
%:%233=66%:%
%:%234=66%:%
%:%235=66%:%
%:%236=67%:%
%:%237=67%:%
%:%238=68%:%
%:%239=68%:%
%:%240=69%:%
%:%241=69%:%
%:%242=69%:%
%:%243=69%:%
%:%244=70%:%
%:%245=70%:%
%:%246=71%:%
%:%247=71%:%
%:%248=72%:%
%:%249=72%:%
%:%250=72%:%
%:%251=72%:%
%:%252=73%:%
%:%258=73%:%
%:%261=74%:%
%:%262=75%:%
%:%263=75%:%
%:%264=76%:%
%:%265=77%:%
%:%266=78%:%
%:%273=79%:%
%:%274=79%:%
%:%275=80%:%
%:%276=80%:%
%:%277=81%:%
%:%278=81%:%
%:%279=82%:%
%:%280=83%:%
%:%281=83%:%
%:%282=84%:%
%:%283=84%:%
%:%284=85%:%
%:%285=86%:%
%:%286=86%:%
%:%287=87%:%
%:%288=87%:%
%:%289=88%:%
%:%290=89%:%
%:%291=89%:%
%:%292=90%:%
%:%293=90%:%
%:%294=91%:%
%:%295=91%:%
%:%296=92%:%
%:%297=92%:%
%:%298=93%:%
%:%299=93%:%
%:%300=94%:%
%:%301=94%:%
%:%302=95%:%
%:%303=95%:%
%:%304=96%:%
%:%305=96%:%
%:%306=97%:%
%:%307=98%:%
%:%308=98%:%
%:%309=99%:%
%:%310=99%:%
%:%311=100%:%
%:%312=100%:%
%:%313=101%:%
%:%314=101%:%
%:%315=102%:%
%:%316=102%:%
%:%317=102%:%
%:%318=103%:%
%:%319=104%:%
%:%320=104%:%
%:%321=105%:%
%:%322=105%:%
%:%323=106%:%
%:%324=106%:%
%:%325=107%:%
%:%326=107%:%
%:%327=108%:%
%:%328=108%:%
%:%329=109%:%
%:%330=109%:%
%:%331=110%:%
%:%332=110%:%
%:%333=111%:%
%:%334=111%:%
%:%335=112%:%
%:%336=112%:%
%:%337=113%:%
%:%338=113%:%
%:%339=114%:%
%:%340=114%:%
%:%341=114%:%
%:%342=115%:%
%:%343=115%:%
%:%344=116%:%
%:%345=116%:%
%:%346=117%:%
%:%347=117%:%
%:%348=118%:%
%:%349=118%:%
%:%350=119%:%
%:%351=120%:%
%:%352=120%:%
%:%354=122%:%
%:%355=123%:%
%:%356=123%:%
%:%357=124%:%
%:%358=124%:%
%:%359=125%:%
%:%360=125%:%
%:%361=126%:%
%:%362=126%:%
%:%363=127%:%
%:%364=127%:%
%:%365=128%:%
%:%366=128%:%
%:%367=129%:%
%:%368=130%:%
%:%369=130%:%
%:%370=131%:%
%:%371=131%:%
%:%372=132%:%
%:%373=132%:%
%:%374=133%:%
%:%375=133%:%
%:%376=133%:%
%:%377=134%:%
%:%378=135%:%
%:%379=135%:%
%:%380=136%:%
%:%381=137%:%
%:%382=137%:%
%:%383=138%:%
%:%384=138%:%
%:%385=139%:%
%:%386=139%:%
%:%387=139%:%
%:%388=140%:%
%:%389=140%:%
%:%390=140%:%
%:%391=141%:%
%:%392=141%:%
%:%393=141%:%
%:%394=142%:%
%:%395=142%:%
%:%396=143%:%
%:%397=143%:%
%:%398=144%:%
%:%399=144%:%
%:%400=145%:%
%:%401=145%:%
%:%402=145%:%
%:%403=145%:%
%:%404=145%:%
%:%405=146%:%
%:%406=146%:%
%:%407=146%:%
%:%408=146%:%
%:%409=147%:%
%:%410=147%:%
%:%411=147%:%
%:%412=147%:%
%:%413=148%:%
%:%414=149%:%
%:%415=149%:%
%:%416=150%:%
%:%417=151%:%
%:%418=151%:%
%:%419=152%:%
%:%420=152%:%
%:%421=152%:%
%:%422=153%:%
%:%423=153%:%
%:%424=154%:%
%:%425=154%:%
%:%426=154%:%
%:%427=155%:%
%:%428=155%:%
%:%429=155%:%
%:%430=156%:%
%:%431=156%:%
%:%432=157%:%
%:%433=157%:%
%:%434=157%:%
%:%435=158%:%
%:%436=158%:%
%:%437=158%:%
%:%438=159%:%
%:%439=159%:%
%:%440=159%:%
%:%441=160%:%
%:%442=160%:%
%:%443=161%:%
%:%444=161%:%
%:%445=161%:%
%:%446=161%:%
%:%447=162%:%
%:%448=162%:%
%:%449=163%:%
%:%450=163%:%
%:%451=164%:%
%:%452=165%:%
%:%453=165%:%
%:%454=166%:%
%:%455=166%:%
%:%456=167%:%
%:%457=167%:%
%:%458=168%:%
%:%459=168%:%
%:%460=169%:%
%:%461=169%:%
%:%462=170%:%
%:%463=170%:%
%:%464=170%:%
%:%465=171%:%
%:%466=171%:%
%:%467=171%:%
%:%468=172%:%
%:%469=172%:%
%:%470=173%:%
%:%471=173%:%
%:%472=174%:%
%:%473=174%:%
%:%474=175%:%
%:%475=176%:%
%:%476=176%:%
%:%477=177%:%
%:%478=177%:%
%:%479=178%:%
%:%480=178%:%
%:%481=179%:%
%:%482=179%:%
%:%483=180%:%
%:%484=180%:%
%:%485=181%:%
%:%486=181%:%
%:%487=182%:%
%:%488=182%:%
%:%489=183%:%
%:%490=184%:%
%:%491=184%:%
%:%492=185%:%
%:%493=185%:%
%:%494=186%:%
%:%495=186%:%
%:%496=187%:%
%:%497=187%:%
%:%498=188%:%
%:%499=188%:%
%:%500=189%:%
%:%501=189%:%
%:%502=190%:%
%:%503=190%:%
%:%504=191%:%
%:%505=191%:%
%:%506=192%:%
%:%507=192%:%
%:%508=193%:%
%:%509=193%:%
%:%510=194%:%
%:%511=194%:%
%:%512=195%:%
%:%518=195%:%
%:%521=196%:%
%:%522=197%:%
%:%523=197%:%
%:%524=198%:%
%:%525=199%:%
%:%526=200%:%
%:%527=201%:%
%:%528=202%:%
%:%529=203%:%
%:%530=204%:%
%:%537=205%:%
%:%538=205%:%
%:%539=206%:%
%:%540=206%:%
%:%541=207%:%
%:%542=207%:%
%:%543=208%:%
%:%544=208%:%
%:%545=209%:%
%:%546=209%:%
%:%547=210%:%
%:%548=210%:%
%:%549=211%:%
%:%550=212%:%
%:%551=212%:%
%:%552=213%:%
%:%553=213%:%
%:%554=214%:%
%:%555=214%:%
%:%556=215%:%
%:%557=215%:%
%:%558=216%:%
%:%559=216%:%
%:%560=217%:%
%:%561=217%:%
%:%562=218%:%
%:%563=218%:%
%:%564=219%:%
%:%565=219%:%
%:%566=220%:%
%:%567=220%:%
%:%568=221%:%
%:%569=221%:%
%:%570=222%:%
%:%571=222%:%
%:%572=223%:%
%:%573=223%:%
%:%574=224%:%
%:%575=224%:%
%:%576=224%:%
%:%577=224%:%
%:%578=225%:%
%:%579=225%:%
%:%580=226%:%
%:%581=226%:%
%:%582=227%:%
%:%583=227%:%
%:%584=227%:%
%:%585=228%:%
%:%586=228%:%
%:%587=229%:%
%:%588=229%:%
%:%589=230%:%
%:%590=230%:%
%:%591=231%:%
%:%592=231%:%
%:%593=232%:%
%:%594=232%:%
%:%595=233%:%
%:%596=233%:%
%:%597=234%:%
%:%598=234%:%
%:%599=235%:%
%:%600=235%:%
%:%601=236%:%
%:%602=236%:%
%:%603=237%:%
%:%604=237%:%
%:%605=238%:%
%:%606=238%:%
%:%607=239%:%
%:%608=239%:%
%:%609=240%:%
%:%610=240%:%
%:%611=241%:%
%:%612=241%:%
%:%613=242%:%
%:%614=242%:%
%:%615=243%:%
%:%616=243%:%
%:%617=244%:%
%:%618=244%:%
%:%619=245%:%
%:%620=245%:%
%:%621=246%:%
%:%622=246%:%
%:%623=246%:%
%:%624=246%:%
%:%625=247%:%
%:%626=247%:%
%:%627=248%:%
%:%628=248%:%
%:%629=249%:%
%:%630=249%:%
%:%631=250%:%
%:%632=250%:%
%:%633=250%:%
%:%634=250%:%
%:%635=251%:%
%:%636=251%:%
%:%637=251%:%
%:%638=252%:%
%:%639=252%:%
%:%640=252%:%
%:%641=253%:%
%:%642=253%:%
%:%643=254%:%
%:%644=254%:%
%:%645=255%:%
%:%651=255%:%
%:%654=256%:%
%:%655=257%:%
%:%656=257%:%
%:%657=258%:%
%:%658=259%:%
%:%659=260%:%
%:%660=261%:%
%:%661=262%:%
%:%662=263%:%
%:%663=264%:%
%:%664=265%:%
%:%665=266%:%
%:%672=267%:%
%:%673=267%:%
%:%674=268%:%
%:%675=268%:%
%:%676=269%:%
%:%677=269%:%
%:%678=270%:%
%:%679=270%:%
%:%680=271%:%
%:%681=271%:%
%:%682=272%:%
%:%683=272%:%
%:%684=272%:%
%:%685=272%:%
%:%686=273%:%
%:%687=273%:%
%:%688=273%:%
%:%689=274%:%
%:%690=274%:%
%:%691=275%:%
%:%692=275%:%
%:%693=276%:%
%:%694=276%:%
%:%695=276%:%
%:%696=276%:%
%:%697=277%:%
%:%703=277%:%
%:%706=278%:%
%:%707=279%:%
%:%708=279%:%
%:%709=280%:%
%:%710=281%:%
%:%711=282%:%
%:%712=283%:%
%:%719=284%:%
%:%720=284%:%
%:%721=285%:%
%:%722=285%:%
%:%723=286%:%
%:%724=287%:%
%:%725=287%:%
%:%726=288%:%
%:%727=288%:%
%:%728=289%:%
%:%729=289%:%
%:%730=290%:%
%:%731=291%:%
%:%732=291%:%
%:%733=292%:%
%:%734=292%:%
%:%735=293%:%
%:%736=293%:%
%:%737=294%:%
%:%738=295%:%
%:%739=295%:%
%:%740=296%:%
%:%741=296%:%
%:%742=297%:%
%:%743=297%:%
%:%744=298%:%
%:%745=298%:%
%:%746=299%:%
%:%747=300%:%
%:%748=300%:%
%:%749=301%:%
%:%750=301%:%
%:%751=302%:%
%:%752=302%:%
%:%753=303%:%
%:%754=303%:%
%:%755=304%:%
%:%756=304%:%
%:%757=305%:%
%:%758=305%:%
%:%759=306%:%
%:%760=306%:%
%:%761=307%:%
%:%762=307%:%
%:%763=308%:%
%:%764=308%:%
%:%765=309%:%
%:%766=310%:%
%:%767=310%:%
%:%768=311%:%
%:%769=311%:%
%:%770=312%:%
%:%771=312%:%
%:%772=313%:%
%:%773=313%:%
%:%774=314%:%
%:%775=314%:%
%:%776=315%:%
%:%777=315%:%
%:%778=316%:%
%:%779=316%:%
%:%780=317%:%
%:%781=317%:%
%:%782=318%:%
%:%783=318%:%
%:%784=319%:%
%:%785=319%:%
%:%786=320%:%
%:%787=320%:%
%:%788=321%:%
%:%789=321%:%
%:%790=322%:%
%:%791=322%:%
%:%792=323%:%
%:%793=323%:%
%:%794=324%:%
%:%795=324%:%
%:%796=325%:%
%:%797=325%:%
%:%798=326%:%
%:%799=327%:%
%:%800=327%:%
%:%801=328%:%
%:%802=328%:%
%:%803=329%:%
%:%804=329%:%
%:%805=330%:%
%:%806=330%:%
%:%807=331%:%
%:%808=331%:%
%:%809=332%:%
%:%810=332%:%
%:%811=333%:%
%:%812=333%:%
%:%813=334%:%
%:%814=334%:%
%:%815=335%:%
%:%816=335%:%
%:%817=336%:%
%:%818=336%:%
%:%819=337%:%
%:%820=337%:%
%:%821=338%:%
%:%822=338%:%
%:%823=339%:%
%:%824=340%:%
%:%825=340%:%
%:%826=341%:%
%:%827=341%:%
%:%828=342%:%
%:%829=342%:%
%:%830=343%:%
%:%831=344%:%
%:%832=344%:%
%:%833=345%:%
%:%834=345%:%
%:%835=346%:%
%:%836=346%:%
%:%837=347%:%
%:%838=347%:%
%:%839=348%:%
%:%840=348%:%
%:%841=349%:%
%:%842=349%:%
%:%843=349%:%
%:%844=350%:%
%:%850=350%:%
%:%853=351%:%
%:%854=352%:%
%:%855=352%:%
%:%856=353%:%
%:%857=354%:%
%:%858=355%:%
%:%859=356%:%
%:%860=357%:%
%:%861=358%:%
%:%868=359%:%
%:%869=359%:%
%:%870=360%:%
%:%871=360%:%
%:%872=361%:%
%:%873=361%:%
%:%874=362%:%
%:%875=362%:%
%:%876=363%:%
%:%877=363%:%
%:%878=363%:%
%:%879=363%:%
%:%880=364%:%
%:%881=364%:%
%:%882=364%:%
%:%883=364%:%
%:%884=365%:%
%:%890=365%:%
%:%893=366%:%
%:%894=367%:%
%:%895=367%:%
%:%896=368%:%
%:%897=369%:%
%:%898=370%:%
%:%899=371%:%
%:%900=372%:%
%:%907=373%:%
%:%908=373%:%
%:%909=374%:%
%:%910=374%:%
%:%911=375%:%
%:%912=376%:%
%:%913=376%:%
%:%914=377%:%
%:%915=377%:%
%:%916=378%:%
%:%917=378%:%
%:%918=379%:%
%:%919=380%:%
%:%920=380%:%
%:%921=381%:%
%:%922=381%:%
%:%923=381%:%
%:%924=382%:%
%:%925=382%:%
%:%926=383%:%
%:%927=383%:%
%:%928=384%:%
%:%929=384%:%
%:%930=385%:%
%:%931=385%:%
%:%932=385%:%
%:%933=386%:%
%:%939=386%:%
%:%942=387%:%
%:%943=388%:%
%:%944=388%:%
%:%945=389%:%
%:%946=390%:%
%:%947=391%:%
%:%948=392%:%
%:%949=393%:%
%:%950=394%:%
%:%957=395%:%
%:%958=395%:%
%:%959=396%:%
%:%960=396%:%
%:%961=397%:%
%:%962=397%:%
%:%963=398%:%
%:%964=398%:%
%:%965=398%:%
%:%966=399%:%
%:%967=399%:%
%:%968=399%:%
%:%969=399%:%
%:%970=400%:%
%:%971=400%:%
%:%972=400%:%
%:%973=400%:%
%:%974=401%:%
%:%980=401%:%
%:%983=402%:%
%:%984=403%:%
%:%985=404%:%
%:%986=404%:%
%:%987=405%:%
%:%988=406%:%
%:%989=407%:%
%:%990=408%:%
%:%997=409%:%
%:%998=409%:%
%:%999=410%:%
%:%1000=410%:%
%:%1001=411%:%
%:%1002=411%:%
%:%1003=412%:%
%:%1004=412%:%
%:%1005=412%:%
%:%1006=413%:%
%:%1007=414%:%
%:%1008=414%:%
%:%1009=415%:%
%:%1010=415%:%
%:%1011=416%:%
%:%1012=416%:%
%:%1013=417%:%
%:%1014=417%:%
%:%1015=418%:%
%:%1016=418%:%
%:%1017=419%:%
%:%1018=419%:%
%:%1019=420%:%
%:%1020=420%:%
%:%1021=421%:%
%:%1022=421%:%
%:%1023=422%:%
%:%1029=422%:%
%:%1032=423%:%
%:%1033=424%:%
%:%1034=424%:%
%:%1037=425%:%
%:%1041=425%:%
%:%1042=425%:%
%:%1043=426%:%
%:%1044=426%:%
%:%1045=426%:%
%:%1050=426%:%
%:%1053=427%:%
%:%1054=428%:%
%:%1055=429%:%
%:%1056=429%:%
%:%1057=430%:%
%:%1058=431%:%
%:%1059=432%:%
%:%1060=433%:%
%:%1063=434%:%
%:%1067=434%:%
%:%1068=434%:%
%:%1073=434%:%
%:%1076=435%:%
%:%1077=436%:%
%:%1078=437%:%
%:%1079=437%:%
%:%1080=438%:%
%:%1081=439%:%
%:%1084=440%:%
%:%1088=440%:%
%:%1089=440%:%
%:%1090=440%:%
%:%1091=440%:%
%:%1096=440%:%
%:%1099=441%:%
%:%1100=442%:%
%:%1101=442%:%
%:%1102=443%:%
%:%1103=444%:%
%:%1106=445%:%
%:%1110=445%:%
%:%1111=445%:%
%:%1112=445%:%
%:%1117=445%:%
%:%1120=446%:%
%:%1121=447%:%
%:%1122=447%:%
%:%1123=448%:%
%:%1124=449%:%
%:%1125=450%:%
%:%1126=451%:%
%:%1127=452%:%
%:%1134=453%:%
%:%1135=453%:%
%:%1136=454%:%
%:%1137=454%:%
%:%1138=455%:%
%:%1139=456%:%
%:%1140=456%:%
%:%1141=457%:%
%:%1142=457%:%
%:%1143=458%:%
%:%1144=458%:%
%:%1145=459%:%
%:%1146=460%:%
%:%1147=460%:%
%:%1148=461%:%
%:%1149=461%:%
%:%1150=462%:%
%:%1151=463%:%
%:%1152=463%:%
%:%1153=464%:%
%:%1154=464%:%
%:%1155=465%:%
%:%1156=465%:%
%:%1157=465%:%
%:%1158=466%:%
%:%1159=466%:%
%:%1160=467%:%
%:%1161=467%:%
%:%1162=468%:%
%:%1163=468%:%
%:%1164=469%:%
%:%1165=469%:%
%:%1166=470%:%
%:%1167=470%:%
%:%1168=471%:%
%:%1169=471%:%
%:%1170=472%:%
%:%1171=472%:%
%:%1172=473%:%
%:%1178=473%:%
%:%1181=474%:%
%:%1182=475%:%
%:%1183=475%:%
%:%1184=476%:%
%:%1187=477%:%
%:%1191=477%:%
%:%1192=477%:%
%:%1193=478%:%
%:%1194=478%:%
%:%1195=479%:%
%:%1196=479%:%
%:%1197=480%:%
%:%1198=480%:%
%:%1203=480%:%
%:%1206=481%:%
%:%1207=482%:%
%:%1208=482%:%
%:%1209=483%:%
%:%1212=484%:%
%:%1216=484%:%
%:%1217=484%:%
%:%1222=484%:%
%:%1225=485%:%
%:%1226=486%:%
%:%1227=486%:%
%:%1228=487%:%
%:%1229=488%:%
%:%1231=490%:%
%:%1232=491%:%
%:%1239=492%:%
%:%1240=492%:%
%:%1241=493%:%
%:%1242=493%:%
%:%1252=503%:%
%:%1253=504%:%
%:%1254=505%:%
%:%1255=505%:%
%:%1256=506%:%
%:%1257=506%:%
%:%1258=507%:%
%:%1259=507%:%
%:%1260=508%:%
%:%1261=509%:%
%:%1262=509%:%
%:%1263=510%:%
%:%1264=510%:%
%:%1265=511%:%
%:%1266=512%:%
%:%1267=512%:%
%:%1268=513%:%
%:%1269=513%:%
%:%1270=514%:%
%:%1271=514%:%
%:%1272=515%:%
%:%1273=515%:%
%:%1274=516%:%
%:%1275=516%:%
%:%1276=517%:%
%:%1277=517%:%
%:%1278=518%:%
%:%1279=518%:%
%:%1280=519%:%
%:%1281=519%:%
%:%1282=520%:%
%:%1283=520%:%
%:%1284=520%:%
%:%1285=521%:%
%:%1286=521%:%
%:%1287=522%:%
%:%1288=522%:%
%:%1289=523%:%
%:%1290=524%:%
%:%1291=524%:%
%:%1292=525%:%
%:%1293=525%:%
%:%1294=526%:%
%:%1295=526%:%
%:%1296=527%:%
%:%1297=527%:%
%:%1298=528%:%
%:%1299=528%:%
%:%1300=529%:%
%:%1301=530%:%
%:%1302=530%:%
%:%1303=531%:%
%:%1304=532%:%
%:%1305=532%:%
%:%1306=533%:%
%:%1307=533%:%
%:%1308=534%:%
%:%1309=534%:%
%:%1310=535%:%
%:%1311=535%:%
%:%1312=536%:%
%:%1313=536%:%
%:%1314=537%:%
%:%1315=537%:%
%:%1316=538%:%
%:%1317=539%:%
%:%1318=539%:%
%:%1319=540%:%
%:%1320=540%:%
%:%1321=541%:%
%:%1322=541%:%
%:%1323=541%:%
%:%1324=542%:%
%:%1325=542%:%
%:%1326=543%:%
%:%1327=543%:%
%:%1328=543%:%
%:%1329=544%:%
%:%1330=544%:%
%:%1331=545%:%
%:%1332=545%:%
%:%1333=546%:%
%:%1334=546%:%
%:%1335=547%:%
%:%1336=547%:%
%:%1337=548%:%
%:%1338=548%:%
%:%1339=548%:%
%:%1340=549%:%
%:%1341=549%:%
%:%1342=550%:%
%:%1348=550%:%
%:%1353=551%:%
%:%1358=552%:%

%
\begin{isabellebody}%
\setisabellecontext{UniversalHashFamily}%
%
\isadelimdocument
%
\endisadelimdocument
%
\isatagdocument
%
\isamarkupsection{Universal Hash Families%
}
\isamarkuptrue%
%
\endisatagdocument
{\isafolddocument}%
%
\isadelimdocument
%
\endisadelimdocument
%
\isadelimtheory
%
\endisadelimtheory
%
\isatagtheory
\isacommand{theory}\isamarkupfalse%
\ UniversalHashFamily\isanewline
\ \ \isakeyword{imports}\ Main\ PolynomialCounting\ Product{\isacharunderscore}{\kern0pt}PMF{\isacharunderscore}{\kern0pt}Ext\isanewline
\isakeyword{begin}%
\endisatagtheory
{\isafoldtheory}%
%
\isadelimtheory
%
\endisadelimtheory
%
\begin{isamarkuptext}%
A k-universal hash family $\mathcal H$ is probability space, whose elements are hash functions 
with domain $U$ and range ${i. i < m}$ such that:

\begin{itemize}
\item For every fixed $x \in U$ and value $y < m$ exactly $\frac{1}{m}$ of the hash functions map
  $x$ to $y$: $P_{h \in \mathcal H} \left(h(x) = y\right) = \frac{1}{m}$.
\item For at most $k$ universe elements: $x_1,\cdots,x_m$ the functions $h(x_1), \cdots, h(x_m)$ 
  are independent random variables.
\end{itemize}

In this section, we construct $k$-universal hash families following the approach outlined
by Wegman and Carter using the polynomials of degree less than $k$ over a finite field.%
\end{isamarkuptext}\isamarkuptrue%
%
\begin{isamarkuptext}%
A hash function is just polynomial evaluation.%
\end{isamarkuptext}\isamarkuptrue%
\isacommand{definition}\isamarkupfalse%
\ hash\ {\isacharcolon}{\kern0pt}{\isacharcolon}{\kern0pt}\ {\isachardoublequoteopen}{\isacharparenleft}{\kern0pt}{\isacharprime}{\kern0pt}a{\isacharcomma}{\kern0pt}\ {\isacharprime}{\kern0pt}b{\isacharparenright}{\kern0pt}\ ring{\isacharunderscore}{\kern0pt}scheme\ {\isasymRightarrow}\ {\isacharprime}{\kern0pt}a\ {\isasymRightarrow}\ {\isacharprime}{\kern0pt}a\ list\ {\isasymRightarrow}\ {\isacharprime}{\kern0pt}a{\isachardoublequoteclose}\ \isanewline
\ \ \isakeyword{where}\ {\isachardoublequoteopen}hash\ F\ x\ {\isasymomega}\ {\isacharequal}{\kern0pt}\ ring{\isachardot}{\kern0pt}eval\ F\ {\isasymomega}\ x{\isachardoublequoteclose}\isanewline
\isanewline
\isacommand{lemma}\isamarkupfalse%
\ hash{\isacharunderscore}{\kern0pt}range{\isacharcolon}{\kern0pt}\isanewline
\ \ \isakeyword{assumes}\ {\isachardoublequoteopen}ring\ F{\isachardoublequoteclose}\isanewline
\ \ \isakeyword{assumes}\ {\isachardoublequoteopen}{\isasymomega}\ {\isasymin}\ bounded{\isacharunderscore}{\kern0pt}degree{\isacharunderscore}{\kern0pt}polynomials\ F\ n{\isachardoublequoteclose}\isanewline
\ \ \isakeyword{assumes}\ {\isachardoublequoteopen}x\ {\isasymin}\ carrier\ F{\isachardoublequoteclose}\isanewline
\ \ \isakeyword{shows}\ {\isachardoublequoteopen}hash\ F\ x\ {\isasymomega}\ {\isasymin}\ carrier\ F{\isachardoublequoteclose}\isanewline
%
\isadelimproof
\ \ %
\endisadelimproof
%
\isatagproof
\isacommand{using}\isamarkupfalse%
\ assms\ \isanewline
\ \ \isacommand{apply}\isamarkupfalse%
\ {\isacharparenleft}{\kern0pt}simp\ add{\isacharcolon}{\kern0pt}hash{\isacharunderscore}{\kern0pt}def\ bounded{\isacharunderscore}{\kern0pt}degree{\isacharunderscore}{\kern0pt}polynomials{\isacharunderscore}{\kern0pt}def{\isacharparenright}{\kern0pt}\isanewline
\ \ \isacommand{by}\isamarkupfalse%
\ {\isacharparenleft}{\kern0pt}metis\ ring{\isachardot}{\kern0pt}eval{\isacharunderscore}{\kern0pt}in{\isacharunderscore}{\kern0pt}carrier\ ring{\isachardot}{\kern0pt}polynomial{\isacharunderscore}{\kern0pt}incl\ univ{\isacharunderscore}{\kern0pt}poly{\isacharunderscore}{\kern0pt}carrier{\isacharparenright}{\kern0pt}%
\endisatagproof
{\isafoldproof}%
%
\isadelimproof
\isanewline
%
\endisadelimproof
\isanewline
\isacommand{lemma}\isamarkupfalse%
\ hash{\isacharunderscore}{\kern0pt}range{\isacharunderscore}{\kern0pt}{\isadigit{2}}{\isacharcolon}{\kern0pt}\isanewline
\ \ \isakeyword{assumes}\ {\isachardoublequoteopen}ring\ F{\isachardoublequoteclose}\isanewline
\ \ \isakeyword{assumes}\ {\isachardoublequoteopen}{\isasymomega}\ {\isasymin}\ bounded{\isacharunderscore}{\kern0pt}degree{\isacharunderscore}{\kern0pt}polynomials\ F\ n{\isachardoublequoteclose}\isanewline
\ \ \isakeyword{shows}\ {\isachardoublequoteopen}{\isacharparenleft}{\kern0pt}{\isasymlambda}x{\isachardot}{\kern0pt}\ hash\ F\ x\ {\isasymomega}{\isacharparenright}{\kern0pt}\ {\isacharbackquote}{\kern0pt}\ carrier\ F\ {\isasymsubseteq}\ carrier\ F{\isachardoublequoteclose}\isanewline
%
\isadelimproof
\ \ %
\endisadelimproof
%
\isatagproof
\isacommand{apply}\isamarkupfalse%
\ {\isacharparenleft}{\kern0pt}rule\ image{\isacharunderscore}{\kern0pt}subsetI{\isacharparenright}{\kern0pt}\isanewline
\ \ \isacommand{by}\isamarkupfalse%
\ {\isacharparenleft}{\kern0pt}metis\ hash{\isacharunderscore}{\kern0pt}range\ assms{\isacharparenright}{\kern0pt}%
\endisatagproof
{\isafoldproof}%
%
\isadelimproof
\isanewline
%
\endisadelimproof
\isanewline
\isacommand{lemma}\isamarkupfalse%
\ poly{\isacharunderscore}{\kern0pt}cards{\isacharcolon}{\kern0pt}\isanewline
\ \ \isakeyword{assumes}\ {\isachardoublequoteopen}field\ F\ {\isasymand}\ finite\ {\isacharparenleft}{\kern0pt}carrier\ F{\isacharparenright}{\kern0pt}{\isachardoublequoteclose}\isanewline
\ \ \isakeyword{assumes}\ {\isachardoublequoteopen}K\ {\isasymsubseteq}\ carrier\ F{\isachardoublequoteclose}\isanewline
\ \ \isakeyword{assumes}\ {\isachardoublequoteopen}card\ K\ {\isasymle}\ n{\isachardoublequoteclose}\isanewline
\ \ \isakeyword{assumes}\ {\isachardoublequoteopen}y\ {\isacharbackquote}{\kern0pt}\ K\ {\isasymsubseteq}\ {\isacharparenleft}{\kern0pt}carrier\ F{\isacharparenright}{\kern0pt}{\isachardoublequoteclose}\isanewline
\ \ \isakeyword{shows}\ {\isachardoublequoteopen}card\ {\isacharbraceleft}{\kern0pt}{\isasymomega}\ {\isasymin}\ bounded{\isacharunderscore}{\kern0pt}degree{\isacharunderscore}{\kern0pt}polynomials\ F\ n{\isachardot}{\kern0pt}\ {\isacharparenleft}{\kern0pt}{\isasymforall}k\ {\isasymin}\ K{\isachardot}{\kern0pt}\ ring{\isachardot}{\kern0pt}eval\ F\ {\isasymomega}\ k\ {\isacharequal}{\kern0pt}\ y\ k{\isacharparenright}{\kern0pt}{\isacharbraceright}{\kern0pt}\ {\isacharequal}{\kern0pt}\ \isanewline
\ \ \ \ \ \ \ \ \ card\ {\isacharparenleft}{\kern0pt}carrier\ F{\isacharparenright}{\kern0pt}{\isacharcircum}{\kern0pt}{\isacharparenleft}{\kern0pt}n{\isacharminus}{\kern0pt}card\ K{\isacharparenright}{\kern0pt}{\isachardoublequoteclose}\isanewline
%
\isadelimproof
\ \ %
\endisadelimproof
%
\isatagproof
\isacommand{using}\isamarkupfalse%
\ interpolating{\isacharunderscore}{\kern0pt}polynomials{\isacharunderscore}{\kern0pt}count{\isacharbrackleft}{\kern0pt}\isakeyword{where}\ n{\isacharequal}{\kern0pt}{\isachardoublequoteopen}n{\isacharminus}{\kern0pt}card\ K{\isachardoublequoteclose}\ \isakeyword{and}\ f{\isacharequal}{\kern0pt}{\isachardoublequoteopen}y{\isachardoublequoteclose}\ \isakeyword{and}\ F{\isacharequal}{\kern0pt}{\isachardoublequoteopen}F{\isachardoublequoteclose}\ \isakeyword{and}\ K{\isacharequal}{\kern0pt}{\isachardoublequoteopen}K{\isachardoublequoteclose}{\isacharbrackright}{\kern0pt}\ \ assms\ \isanewline
\ \ \isacommand{by}\isamarkupfalse%
\ fastforce%
\endisatagproof
{\isafoldproof}%
%
\isadelimproof
\isanewline
%
\endisadelimproof
\isanewline
\isacommand{lemma}\isamarkupfalse%
\ poly{\isacharunderscore}{\kern0pt}cards{\isacharunderscore}{\kern0pt}single{\isacharcolon}{\kern0pt}\isanewline
\ \ \isakeyword{assumes}\ {\isachardoublequoteopen}field\ F\ {\isasymand}\ finite\ {\isacharparenleft}{\kern0pt}carrier\ F{\isacharparenright}{\kern0pt}{\isachardoublequoteclose}\isanewline
\ \ \isakeyword{assumes}\ {\isachardoublequoteopen}k\ {\isasymin}\ carrier\ F{\isachardoublequoteclose}\isanewline
\ \ \isakeyword{assumes}\ {\isachardoublequoteopen}{\isadigit{1}}\ {\isasymle}\ n{\isachardoublequoteclose}\isanewline
\ \ \isakeyword{assumes}\ {\isachardoublequoteopen}y\ {\isasymin}\ carrier\ F{\isachardoublequoteclose}\isanewline
\ \ \isakeyword{shows}\ {\isachardoublequoteopen}card\ {\isacharbraceleft}{\kern0pt}{\isasymomega}\ {\isasymin}\ bounded{\isacharunderscore}{\kern0pt}degree{\isacharunderscore}{\kern0pt}polynomials\ F\ n{\isachardot}{\kern0pt}\ ring{\isachardot}{\kern0pt}eval\ F\ {\isasymomega}\ k\ {\isacharequal}{\kern0pt}\ y{\isacharbraceright}{\kern0pt}\ {\isacharequal}{\kern0pt}\ \isanewline
\ \ \ \ \ \ \ \ \ card\ {\isacharparenleft}{\kern0pt}carrier\ F{\isacharparenright}{\kern0pt}{\isacharcircum}{\kern0pt}{\isacharparenleft}{\kern0pt}n{\isacharminus}{\kern0pt}{\isadigit{1}}{\isacharparenright}{\kern0pt}{\isachardoublequoteclose}\isanewline
%
\isadelimproof
\ \ %
\endisadelimproof
%
\isatagproof
\isacommand{using}\isamarkupfalse%
\ poly{\isacharunderscore}{\kern0pt}cards{\isacharbrackleft}{\kern0pt}OF\ assms{\isacharparenleft}{\kern0pt}{\isadigit{1}}{\isacharparenright}{\kern0pt}{\isacharcomma}{\kern0pt}\ \isakeyword{where}\ K{\isacharequal}{\kern0pt}{\isachardoublequoteopen}{\isacharbraceleft}{\kern0pt}k{\isacharbraceright}{\kern0pt}{\isachardoublequoteclose}\ \isakeyword{and}\ y{\isacharequal}{\kern0pt}{\isachardoublequoteopen}{\isasymlambda}{\isacharunderscore}{\kern0pt}{\isachardot}{\kern0pt}\ y{\isachardoublequoteclose}{\isacharcomma}{\kern0pt}\ simplified{\isacharbrackright}{\kern0pt}\ assms{\isacharparenleft}{\kern0pt}{\isadigit{3}}{\isacharparenright}{\kern0pt}\ assms{\isacharparenleft}{\kern0pt}{\isadigit{4}}{\isacharparenright}{\kern0pt}{\isacharbrackleft}{\kern0pt}simplified{\isacharbrackright}{\kern0pt}\isanewline
\ \ \isacommand{by}\isamarkupfalse%
\ {\isacharparenleft}{\kern0pt}simp\ add{\isacharcolon}{\kern0pt}assms{\isacharparenright}{\kern0pt}%
\endisatagproof
{\isafoldproof}%
%
\isadelimproof
\isanewline
%
\endisadelimproof
\isanewline
\isacommand{lemma}\isamarkupfalse%
\ expand{\isacharunderscore}{\kern0pt}subset{\isacharunderscore}{\kern0pt}filter{\isacharcolon}{\kern0pt}\ {\isachardoublequoteopen}{\isacharbraceleft}{\kern0pt}x\ {\isasymin}\ A{\isachardot}{\kern0pt}\ P\ x{\isacharbraceright}{\kern0pt}\ {\isacharequal}{\kern0pt}\ A\ {\isasyminter}\ {\isacharbraceleft}{\kern0pt}x{\isachardot}{\kern0pt}\ P\ x{\isacharbraceright}{\kern0pt}{\isachardoublequoteclose}\isanewline
%
\isadelimproof
\ \ %
\endisadelimproof
%
\isatagproof
\isacommand{by}\isamarkupfalse%
\ force%
\endisatagproof
{\isafoldproof}%
%
\isadelimproof
\isanewline
%
\endisadelimproof
\isanewline
\isacommand{lemma}\isamarkupfalse%
\ hash{\isacharunderscore}{\kern0pt}prob{\isacharcolon}{\kern0pt}\isanewline
\ \ \isakeyword{assumes}\ {\isachardoublequoteopen}field\ F\ {\isasymand}\ finite\ {\isacharparenleft}{\kern0pt}carrier\ F{\isacharparenright}{\kern0pt}{\isachardoublequoteclose}\isanewline
\ \ \isakeyword{assumes}\ {\isachardoublequoteopen}K\ {\isasymsubseteq}\ carrier\ F{\isachardoublequoteclose}\isanewline
\ \ \isakeyword{assumes}\ {\isachardoublequoteopen}card\ K\ {\isasymle}\ n{\isachardoublequoteclose}\isanewline
\ \ \isakeyword{assumes}\ {\isachardoublequoteopen}y\ {\isacharbackquote}{\kern0pt}\ K\ {\isasymsubseteq}\ carrier\ F{\isachardoublequoteclose}\isanewline
\ \ \isakeyword{shows}\ {\isachardoublequoteopen}{\isasymP}{\isacharparenleft}{\kern0pt}{\isasymomega}\ in\ pmf{\isacharunderscore}{\kern0pt}of{\isacharunderscore}{\kern0pt}set\ {\isacharparenleft}{\kern0pt}bounded{\isacharunderscore}{\kern0pt}degree{\isacharunderscore}{\kern0pt}polynomials\ F\ n{\isacharparenright}{\kern0pt}{\isachardot}{\kern0pt}\ {\isacharparenleft}{\kern0pt}{\isasymforall}x\ {\isasymin}\ K{\isachardot}{\kern0pt}\ hash\ F\ x\ {\isasymomega}\ {\isacharequal}{\kern0pt}\ y\ x{\isacharparenright}{\kern0pt}{\isacharparenright}{\kern0pt}\ {\isacharequal}{\kern0pt}\ {\isadigit{1}}{\isacharslash}{\kern0pt}{\isacharparenleft}{\kern0pt}real\ {\isacharparenleft}{\kern0pt}card\ {\isacharparenleft}{\kern0pt}carrier\ F{\isacharparenright}{\kern0pt}{\isacharparenright}{\kern0pt}{\isacharparenright}{\kern0pt}{\isacharcircum}{\kern0pt}card\ K{\isachardoublequoteclose}\ \isanewline
%
\isadelimproof
%
\endisadelimproof
%
\isatagproof
\isacommand{proof}\isamarkupfalse%
\ {\isacharminus}{\kern0pt}\isanewline
\ \ \isacommand{have}\isamarkupfalse%
\ {\isachardoublequoteopen}{\isasymzero}\isactrlbsub F\isactrlesub \ {\isasymin}\ carrier\ F{\isachardoublequoteclose}\isanewline
\ \ \ \ \isacommand{using}\isamarkupfalse%
\ assms{\isacharparenleft}{\kern0pt}{\isadigit{1}}{\isacharparenright}{\kern0pt}\ field{\isachardot}{\kern0pt}is{\isacharunderscore}{\kern0pt}ring\ ring{\isachardot}{\kern0pt}ring{\isacharunderscore}{\kern0pt}simprules{\isacharparenleft}{\kern0pt}{\isadigit{2}}{\isacharparenright}{\kern0pt}\ \isacommand{by}\isamarkupfalse%
\ blast\isanewline
\isanewline
\ \ \isacommand{hence}\isamarkupfalse%
\ a{\isacharcolon}{\kern0pt}{\isachardoublequoteopen}card\ {\isacharparenleft}{\kern0pt}carrier\ F{\isacharparenright}{\kern0pt}\ {\isachargreater}{\kern0pt}\ {\isadigit{0}}{\isachardoublequoteclose}\isanewline
\ \ \ \ \isacommand{apply}\isamarkupfalse%
\ {\isacharparenleft}{\kern0pt}subst\ card{\isacharunderscore}{\kern0pt}gt{\isacharunderscore}{\kern0pt}{\isadigit{0}}{\isacharunderscore}{\kern0pt}iff{\isacharparenright}{\kern0pt}\ \isanewline
\ \ \ \ \isacommand{using}\isamarkupfalse%
\ assms{\isacharparenleft}{\kern0pt}{\isadigit{1}}{\isacharparenright}{\kern0pt}\ \isacommand{by}\isamarkupfalse%
\ blast\isanewline
\isanewline
\ \ \isacommand{show}\isamarkupfalse%
\ {\isacharquery}{\kern0pt}thesis\isanewline
\ \ \ \ \isacommand{apply}\isamarkupfalse%
\ {\isacharparenleft}{\kern0pt}subst\ measure{\isacharunderscore}{\kern0pt}pmf{\isacharunderscore}{\kern0pt}of{\isacharunderscore}{\kern0pt}set{\isacharparenright}{\kern0pt}\isanewline
\ \ \ \ \ \ \isacommand{apply}\isamarkupfalse%
\ {\isacharparenleft}{\kern0pt}metis\ non{\isacharunderscore}{\kern0pt}empty{\isacharunderscore}{\kern0pt}bounded{\isacharunderscore}{\kern0pt}degree{\isacharunderscore}{\kern0pt}polynomials\ field{\isachardot}{\kern0pt}is{\isacharunderscore}{\kern0pt}ring\ \ assms{\isacharparenleft}{\kern0pt}{\isadigit{1}}{\isacharparenright}{\kern0pt}{\isacharparenright}{\kern0pt}\isanewline
\ \ \ \ \ \isacommand{apply}\isamarkupfalse%
\ {\isacharparenleft}{\kern0pt}metis\ fin{\isacharunderscore}{\kern0pt}degree{\isacharunderscore}{\kern0pt}bounded\ field{\isachardot}{\kern0pt}is{\isacharunderscore}{\kern0pt}ring\ assms{\isacharparenleft}{\kern0pt}{\isadigit{1}}{\isacharparenright}{\kern0pt}{\isacharparenright}{\kern0pt}\isanewline
\ \ \ \ \isacommand{apply}\isamarkupfalse%
\ {\isacharparenleft}{\kern0pt}simp\ add{\isacharcolon}{\kern0pt}hash{\isacharunderscore}{\kern0pt}def\ expand{\isacharunderscore}{\kern0pt}subset{\isacharunderscore}{\kern0pt}filter{\isacharbrackleft}{\kern0pt}symmetric{\isacharbrackright}{\kern0pt}{\isacharparenright}{\kern0pt}\isanewline
\ \ \ \ \isacommand{apply}\isamarkupfalse%
\ {\isacharparenleft}{\kern0pt}subst\ poly{\isacharunderscore}{\kern0pt}cards{\isacharbrackleft}{\kern0pt}OF\ assms{\isacharparenleft}{\kern0pt}{\isadigit{1}}{\isacharparenright}{\kern0pt}\ assms{\isacharparenleft}{\kern0pt}{\isadigit{2}}{\isacharparenright}{\kern0pt}\ assms{\isacharparenleft}{\kern0pt}{\isadigit{3}}{\isacharparenright}{\kern0pt}\ assms{\isacharparenleft}{\kern0pt}{\isadigit{4}}{\isacharparenright}{\kern0pt}{\isacharbrackright}{\kern0pt}{\isacharparenright}{\kern0pt}\isanewline
\ \ \ \ \isacommand{apply}\isamarkupfalse%
\ {\isacharparenleft}{\kern0pt}subst\ bounded{\isacharunderscore}{\kern0pt}degree{\isacharunderscore}{\kern0pt}polynomials{\isacharunderscore}{\kern0pt}count{\isacharcomma}{\kern0pt}\ metis\ field{\isachardot}{\kern0pt}is{\isacharunderscore}{\kern0pt}ring\ assms{\isacharparenleft}{\kern0pt}{\isadigit{1}}{\isacharparenright}{\kern0pt}{\isacharcomma}{\kern0pt}\ metis\ assms{\isacharparenleft}{\kern0pt}{\isadigit{1}}{\isacharparenright}{\kern0pt}{\isacharparenright}{\kern0pt}\isanewline
\ \ \ \ \isacommand{apply}\isamarkupfalse%
\ {\isacharparenleft}{\kern0pt}subst\ frac{\isacharunderscore}{\kern0pt}eq{\isacharunderscore}{\kern0pt}eq{\isacharparenright}{\kern0pt}\isanewline
\ \ \ \ \isacommand{apply}\isamarkupfalse%
\ {\isacharparenleft}{\kern0pt}simp\ add{\isacharcolon}{\kern0pt}a{\isacharcomma}{\kern0pt}\ simp\ add{\isacharcolon}{\kern0pt}a{\isacharcomma}{\kern0pt}\ simp{\isacharparenright}{\kern0pt}\isanewline
\ \ \ \ \isacommand{by}\isamarkupfalse%
\ {\isacharparenleft}{\kern0pt}metis\ assms{\isacharparenleft}{\kern0pt}{\isadigit{3}}{\isacharparenright}{\kern0pt}\ le{\isacharunderscore}{\kern0pt}add{\isacharunderscore}{\kern0pt}diff{\isacharunderscore}{\kern0pt}inverse{\isadigit{2}}\ power{\isacharunderscore}{\kern0pt}add{\isacharparenright}{\kern0pt}\isanewline
\isacommand{qed}\isamarkupfalse%
%
\endisatagproof
{\isafoldproof}%
%
\isadelimproof
\isanewline
%
\endisadelimproof
\isanewline
\isacommand{lemma}\isamarkupfalse%
\ hash{\isacharunderscore}{\kern0pt}prob{\isacharunderscore}{\kern0pt}single{\isacharcolon}{\kern0pt}\isanewline
\ \ \isakeyword{assumes}\ {\isachardoublequoteopen}field\ F\ {\isasymand}\ finite\ {\isacharparenleft}{\kern0pt}carrier\ F{\isacharparenright}{\kern0pt}{\isachardoublequoteclose}\isanewline
\ \ \isakeyword{assumes}\ {\isachardoublequoteopen}x\ {\isasymin}\ carrier\ F{\isachardoublequoteclose}\isanewline
\ \ \isakeyword{assumes}\ {\isachardoublequoteopen}{\isadigit{1}}\ {\isasymle}\ n{\isachardoublequoteclose}\isanewline
\ \ \isakeyword{assumes}\ {\isachardoublequoteopen}y\ {\isasymin}\ carrier\ F{\isachardoublequoteclose}\isanewline
\ \ \isakeyword{shows}\ {\isachardoublequoteopen}{\isasymP}{\isacharparenleft}{\kern0pt}{\isasymomega}\ in\ pmf{\isacharunderscore}{\kern0pt}of{\isacharunderscore}{\kern0pt}set\ {\isacharparenleft}{\kern0pt}bounded{\isacharunderscore}{\kern0pt}degree{\isacharunderscore}{\kern0pt}polynomials\ F\ n{\isacharparenright}{\kern0pt}{\isachardot}{\kern0pt}\ hash\ F\ x\ {\isasymomega}\ {\isacharequal}{\kern0pt}\ y{\isacharparenright}{\kern0pt}\ {\isacharequal}{\kern0pt}\ {\isadigit{1}}{\isacharslash}{\kern0pt}{\isacharparenleft}{\kern0pt}real\ {\isacharparenleft}{\kern0pt}card\ {\isacharparenleft}{\kern0pt}carrier\ F{\isacharparenright}{\kern0pt}{\isacharparenright}{\kern0pt}{\isacharparenright}{\kern0pt}{\isachardoublequoteclose}\ \isanewline
%
\isadelimproof
\ \ %
\endisadelimproof
%
\isatagproof
\isacommand{using}\isamarkupfalse%
\ hash{\isacharunderscore}{\kern0pt}prob{\isacharbrackleft}{\kern0pt}OF\ assms{\isacharparenleft}{\kern0pt}{\isadigit{1}}{\isacharparenright}{\kern0pt}{\isacharcomma}{\kern0pt}\ \isakeyword{where}\ K{\isacharequal}{\kern0pt}{\isachardoublequoteopen}{\isacharbraceleft}{\kern0pt}x{\isacharbraceright}{\kern0pt}{\isachardoublequoteclose}\ \isakeyword{and}\ y{\isacharequal}{\kern0pt}{\isachardoublequoteopen}{\isasymlambda}{\isacharunderscore}{\kern0pt}{\isachardot}{\kern0pt}\ y{\isachardoublequoteclose}{\isacharcomma}{\kern0pt}\ simplified{\isacharbrackright}{\kern0pt}\ assms\ \isanewline
\ \ \isacommand{by}\isamarkupfalse%
\ {\isacharparenleft}{\kern0pt}metis\ {\isacharparenleft}{\kern0pt}no{\isacharunderscore}{\kern0pt}types{\isacharcomma}{\kern0pt}\ lifting{\isacharparenright}{\kern0pt}\ Collect{\isacharunderscore}{\kern0pt}cong\ One{\isacharunderscore}{\kern0pt}nat{\isacharunderscore}{\kern0pt}def\ UNIV{\isacharunderscore}{\kern0pt}I\ space{\isacharunderscore}{\kern0pt}measure{\isacharunderscore}{\kern0pt}pmf{\isacharparenright}{\kern0pt}%
\endisatagproof
{\isafoldproof}%
%
\isadelimproof
\isanewline
%
\endisadelimproof
\isanewline
\isacommand{lemma}\isamarkupfalse%
\ hash{\isacharunderscore}{\kern0pt}indep{\isacharunderscore}{\kern0pt}pmf{\isacharcolon}{\kern0pt}\isanewline
\ \ \isakeyword{assumes}\ {\isachardoublequoteopen}field\ F\ {\isasymand}\ finite\ {\isacharparenleft}{\kern0pt}carrier\ F{\isacharparenright}{\kern0pt}{\isachardoublequoteclose}\isanewline
\ \ \isakeyword{assumes}\ {\isachardoublequoteopen}J{\isasymsubseteq}carrier\ F{\isachardoublequoteclose}\isanewline
\ \ \isakeyword{assumes}\ {\isachardoublequoteopen}finite\ J{\isachardoublequoteclose}\ \isanewline
\ \ \isakeyword{assumes}\ {\isachardoublequoteopen}card\ J\ {\isasymle}\ n{\isachardoublequoteclose}\isanewline
\ \ \isakeyword{assumes}\ {\isachardoublequoteopen}{\isadigit{1}}\ {\isasymle}\ n{\isachardoublequoteclose}\isanewline
\ \ \isakeyword{shows}\ {\isachardoublequoteopen}prob{\isacharunderscore}{\kern0pt}space{\isachardot}{\kern0pt}indep{\isacharunderscore}{\kern0pt}vars\ {\isacharparenleft}{\kern0pt}pmf{\isacharunderscore}{\kern0pt}of{\isacharunderscore}{\kern0pt}set\ {\isacharparenleft}{\kern0pt}bounded{\isacharunderscore}{\kern0pt}degree{\isacharunderscore}{\kern0pt}polynomials\ F\ n{\isacharparenright}{\kern0pt}{\isacharparenright}{\kern0pt}\ \isanewline
\ \ \ \ {\isacharparenleft}{\kern0pt}{\isasymlambda}{\isacharunderscore}{\kern0pt}{\isachardot}{\kern0pt}\ pmf{\isacharunderscore}{\kern0pt}of{\isacharunderscore}{\kern0pt}set\ {\isacharparenleft}{\kern0pt}carrier\ F{\isacharparenright}{\kern0pt}{\isacharparenright}{\kern0pt}\ {\isacharparenleft}{\kern0pt}hash\ F{\isacharparenright}{\kern0pt}\ J{\isachardoublequoteclose}\isanewline
%
\isadelimproof
%
\endisadelimproof
%
\isatagproof
\isacommand{proof}\isamarkupfalse%
\ {\isacharminus}{\kern0pt}\isanewline
\ \ \isacommand{have}\isamarkupfalse%
\ {\isachardoublequoteopen}{\isasymzero}\isactrlbsub poly{\isacharunderscore}{\kern0pt}ring\ F\isactrlesub \ {\isasymin}\ bounded{\isacharunderscore}{\kern0pt}degree{\isacharunderscore}{\kern0pt}polynomials\ F\ n{\isachardoublequoteclose}\isanewline
\ \ \ \ \isacommand{apply}\isamarkupfalse%
\ {\isacharparenleft}{\kern0pt}simp\ add{\isacharcolon}{\kern0pt}bounded{\isacharunderscore}{\kern0pt}degree{\isacharunderscore}{\kern0pt}polynomials{\isacharunderscore}{\kern0pt}def{\isacharparenright}{\kern0pt}\isanewline
\ \ \ \ \isacommand{apply}\isamarkupfalse%
\ {\isacharparenleft}{\kern0pt}rule\ conjI{\isacharparenright}{\kern0pt}\isanewline
\ \ \ \ \ \isacommand{apply}\isamarkupfalse%
\ {\isacharparenleft}{\kern0pt}simp\ add{\isacharcolon}{\kern0pt}\ univ{\isacharunderscore}{\kern0pt}poly{\isacharunderscore}{\kern0pt}zero\ univ{\isacharunderscore}{\kern0pt}poly{\isacharunderscore}{\kern0pt}zero{\isacharunderscore}{\kern0pt}closed{\isacharparenright}{\kern0pt}\isanewline
\ \ \ \ \isacommand{using}\isamarkupfalse%
\ univ{\isacharunderscore}{\kern0pt}poly{\isacharunderscore}{\kern0pt}zero\ \isacommand{by}\isamarkupfalse%
\ blast\isanewline
\ \ \isacommand{hence}\isamarkupfalse%
\ b{\isacharcolon}{\kern0pt}\ {\isachardoublequoteopen}bounded{\isacharunderscore}{\kern0pt}degree{\isacharunderscore}{\kern0pt}polynomials\ F\ n\ {\isasymnoteq}\ {\isacharbraceleft}{\kern0pt}{\isacharbraceright}{\kern0pt}{\isachardoublequoteclose}\isanewline
\ \ \ \ \isacommand{by}\isamarkupfalse%
\ blast\isanewline
\ \ \isacommand{have}\isamarkupfalse%
\ c{\isacharcolon}{\kern0pt}\ {\isachardoublequoteopen}finite\ {\isacharparenleft}{\kern0pt}bounded{\isacharunderscore}{\kern0pt}degree{\isacharunderscore}{\kern0pt}polynomials\ F\ n{\isacharparenright}{\kern0pt}{\isachardoublequoteclose}\isanewline
\ \ \ \ \isacommand{by}\isamarkupfalse%
\ {\isacharparenleft}{\kern0pt}metis\ finite{\isacharunderscore}{\kern0pt}poly{\isacharunderscore}{\kern0pt}count\ assms{\isacharparenleft}{\kern0pt}{\isadigit{1}}{\isacharparenright}{\kern0pt}{\isacharparenright}{\kern0pt}\isanewline
\ \ \isacommand{have}\isamarkupfalse%
\ d{\isacharcolon}{\kern0pt}\ {\isachardoublequoteopen}{\isasymAnd}\ A\ P{\isachardot}{\kern0pt}\ A\ {\isasyminter}\ {\isacharbraceleft}{\kern0pt}{\isasymomega}{\isachardot}{\kern0pt}\ P\ {\isasymomega}{\isacharbraceright}{\kern0pt}\ {\isacharequal}{\kern0pt}\ {\isacharbraceleft}{\kern0pt}{\isasymomega}\ {\isasymin}\ A{\isachardot}{\kern0pt}\ P\ {\isasymomega}{\isacharbraceright}{\kern0pt}{\isachardoublequoteclose}\isanewline
\ \ \ \ \isacommand{by}\isamarkupfalse%
\ blast\isanewline
\isanewline
\ \ \isacommand{have}\isamarkupfalse%
\ fin{\isacharunderscore}{\kern0pt}carr{\isacharcolon}{\kern0pt}\ {\isachardoublequoteopen}finite\ {\isacharparenleft}{\kern0pt}carrier\ F{\isacharparenright}{\kern0pt}{\isachardoublequoteclose}\ \isacommand{using}\isamarkupfalse%
\ assms{\isacharparenleft}{\kern0pt}{\isadigit{1}}{\isacharparenright}{\kern0pt}\ \isacommand{by}\isamarkupfalse%
\ blast\isanewline
\ \ \isacommand{have}\isamarkupfalse%
\ e{\isacharcolon}{\kern0pt}{\isachardoublequoteopen}ring\ F{\isachardoublequoteclose}\ \isacommand{using}\isamarkupfalse%
\ assms{\isacharparenleft}{\kern0pt}{\isadigit{1}}{\isacharparenright}{\kern0pt}\ field{\isachardot}{\kern0pt}is{\isacharunderscore}{\kern0pt}ring\ \isacommand{by}\isamarkupfalse%
\ blast\isanewline
\ \ \isacommand{have}\isamarkupfalse%
\ f{\isacharcolon}{\kern0pt}\ {\isachardoublequoteopen}{\isadigit{0}}\ {\isacharless}{\kern0pt}\ card\ {\isacharparenleft}{\kern0pt}carrier\ F{\isacharparenright}{\kern0pt}{\isachardoublequoteclose}\ \isanewline
\ \ \ \ \isacommand{by}\isamarkupfalse%
\ {\isacharparenleft}{\kern0pt}metis\ assms{\isacharparenleft}{\kern0pt}{\isadigit{1}}{\isacharparenright}{\kern0pt}\ card{\isacharunderscore}{\kern0pt}{\isadigit{0}}{\isacharunderscore}{\kern0pt}eq\ e\ empty{\isacharunderscore}{\kern0pt}iff\ gr{\isadigit{0}}I\ ring{\isachardot}{\kern0pt}ring{\isacharunderscore}{\kern0pt}simprules{\isacharparenleft}{\kern0pt}{\isadigit{2}}{\isacharparenright}{\kern0pt}{\isacharparenright}{\kern0pt}\isanewline
\isanewline
\ \ \isacommand{define}\isamarkupfalse%
\ {\isasymOmega}\ \isakeyword{where}\ {\isachardoublequoteopen}{\isasymOmega}\ {\isacharequal}{\kern0pt}\ {\isacharparenleft}{\kern0pt}pmf{\isacharunderscore}{\kern0pt}of{\isacharunderscore}{\kern0pt}set\ {\isacharparenleft}{\kern0pt}bounded{\isacharunderscore}{\kern0pt}degree{\isacharunderscore}{\kern0pt}polynomials\ F\ n{\isacharparenright}{\kern0pt}{\isacharparenright}{\kern0pt}{\isachardoublequoteclose}\isanewline
\ \ \isacommand{have}\isamarkupfalse%
\ a{\isacharcolon}{\kern0pt}{\isachardoublequoteopen}{\isasymAnd}a\ J{\isacharprime}{\kern0pt}{\isachardot}{\kern0pt}\isanewline
\ \ \ \ \ \ \ J{\isacharprime}{\kern0pt}\ {\isasymsubseteq}\ J\ {\isasymLongrightarrow}\isanewline
\ \ \ \ \ \ \ finite\ J{\isacharprime}{\kern0pt}\ {\isasymLongrightarrow}\isanewline
\ \ \ \ \ \ \ measure\ {\isasymOmega}\ {\isacharbraceleft}{\kern0pt}{\isasymomega}{\isachardot}{\kern0pt}\ {\isasymforall}x{\isasymin}J{\isacharprime}{\kern0pt}{\isachardot}{\kern0pt}\ hash\ F\ x\ {\isasymomega}\ {\isacharequal}{\kern0pt}\ a\ x{\isacharbraceright}{\kern0pt}\ {\isacharequal}{\kern0pt}\isanewline
\ \ \ \ \ \ \ {\isacharparenleft}{\kern0pt}{\isasymProd}x{\isasymin}J{\isacharprime}{\kern0pt}{\isachardot}{\kern0pt}\ measure\ {\isasymOmega}\ {\isacharbraceleft}{\kern0pt}{\isasymomega}{\isachardot}{\kern0pt}\ hash\ F\ x\ {\isasymomega}\ {\isacharequal}{\kern0pt}\ a\ x{\isacharbraceright}{\kern0pt}{\isacharparenright}{\kern0pt}{\isachardoublequoteclose}\isanewline
\ \ \isacommand{proof}\isamarkupfalse%
\ {\isacharminus}{\kern0pt}\isanewline
\ \ \ \ \isacommand{fix}\isamarkupfalse%
\ a\isanewline
\ \ \ \ \isacommand{fix}\isamarkupfalse%
\ J{\isacharprime}{\kern0pt}\isanewline
\ \ \ \ \isacommand{assume}\isamarkupfalse%
\ a{\isacharunderscore}{\kern0pt}{\isadigit{1}}{\isacharcolon}{\kern0pt}\ {\isachardoublequoteopen}J{\isacharprime}{\kern0pt}\ {\isasymsubseteq}\ J{\isachardoublequoteclose}\isanewline
\ \ \ \ \isacommand{assume}\isamarkupfalse%
\ a{\isacharunderscore}{\kern0pt}{\isadigit{1}}{\isadigit{1}}{\isacharcolon}{\kern0pt}\ {\isachardoublequoteopen}finite\ J{\isacharprime}{\kern0pt}{\isachardoublequoteclose}\isanewline
\ \ \ \ \isacommand{have}\isamarkupfalse%
\ a{\isacharunderscore}{\kern0pt}{\isadigit{2}}{\isacharcolon}{\kern0pt}\ {\isachardoublequoteopen}card\ J{\isacharprime}{\kern0pt}\ {\isasymle}\ n{\isachardoublequoteclose}\ \isacommand{by}\isamarkupfalse%
\ {\isacharparenleft}{\kern0pt}metis\ card{\isacharunderscore}{\kern0pt}mono\ order{\isacharunderscore}{\kern0pt}trans\ a{\isacharunderscore}{\kern0pt}{\isadigit{1}}\ assms{\isacharparenleft}{\kern0pt}{\isadigit{3}}{\isacharparenright}{\kern0pt}\ assms{\isacharparenleft}{\kern0pt}{\isadigit{4}}{\isacharparenright}{\kern0pt}{\isacharparenright}{\kern0pt}\isanewline
\ \ \ \ \isacommand{have}\isamarkupfalse%
\ a{\isacharunderscore}{\kern0pt}{\isadigit{3}}{\isacharcolon}{\kern0pt}\ {\isachardoublequoteopen}J{\isacharprime}{\kern0pt}\ {\isasymsubseteq}\ carrier\ F{\isachardoublequoteclose}\ \isacommand{by}\isamarkupfalse%
\ {\isacharparenleft}{\kern0pt}metis\ order{\isacharunderscore}{\kern0pt}trans\ a{\isacharunderscore}{\kern0pt}{\isadigit{1}}\ assms{\isacharparenleft}{\kern0pt}{\isadigit{2}}{\isacharparenright}{\kern0pt}{\isacharparenright}{\kern0pt}\isanewline
\ \ \ \ \isacommand{have}\isamarkupfalse%
\ a{\isacharunderscore}{\kern0pt}{\isadigit{4}}{\isacharcolon}{\kern0pt}\ {\isachardoublequoteopen}{\isadigit{1}}\ {\isasymle}n\ {\isachardoublequoteclose}\ \isacommand{using}\isamarkupfalse%
\ assms\ \isacommand{by}\isamarkupfalse%
\ blast\isanewline
\isanewline
\ \ \ \ \isacommand{show}\isamarkupfalse%
\ {\isachardoublequoteopen}measure{\isacharunderscore}{\kern0pt}pmf{\isachardot}{\kern0pt}prob\ {\isasymOmega}\ {\isacharbraceleft}{\kern0pt}{\isasymomega}{\isachardot}{\kern0pt}\ {\isasymforall}x{\isasymin}J{\isacharprime}{\kern0pt}{\isachardot}{\kern0pt}\ hash\ F\ x\ {\isasymomega}\ {\isacharequal}{\kern0pt}\ a\ x{\isacharbraceright}{\kern0pt}\ {\isacharequal}{\kern0pt}\isanewline
\ \ \ \ \ \ \ {\isacharparenleft}{\kern0pt}{\isasymProd}x{\isasymin}J{\isacharprime}{\kern0pt}{\isachardot}{\kern0pt}\ measure{\isacharunderscore}{\kern0pt}pmf{\isachardot}{\kern0pt}prob\ {\isasymOmega}\ {\isacharbraceleft}{\kern0pt}{\isasymomega}{\isachardot}{\kern0pt}\ hash\ F\ x\ {\isasymomega}\ {\isacharequal}{\kern0pt}\ a\ x{\isacharbraceright}{\kern0pt}{\isacharparenright}{\kern0pt}{\isachardoublequoteclose}\isanewline
\ \ \ \ \isacommand{proof}\isamarkupfalse%
\ {\isacharparenleft}{\kern0pt}cases\ {\isachardoublequoteopen}a\ {\isacharbackquote}{\kern0pt}\ J{\isacharprime}{\kern0pt}\ {\isasymsubseteq}\ carrier\ F{\isachardoublequoteclose}{\isacharparenright}{\kern0pt}\isanewline
\ \ \ \ \ \ \isacommand{case}\isamarkupfalse%
\ True\isanewline
\ \ \ \ \ \ \isacommand{have}\isamarkupfalse%
\ a{\isacharunderscore}{\kern0pt}{\isadigit{5}}{\isacharcolon}{\kern0pt}\ {\isachardoublequoteopen}{\isasymAnd}x{\isachardot}{\kern0pt}\ x\ {\isasymin}\ J{\isacharprime}{\kern0pt}\ {\isasymLongrightarrow}\ x\ {\isasymin}\ carrier\ F{\isachardoublequoteclose}\ \ \isacommand{using}\isamarkupfalse%
\ a{\isacharunderscore}{\kern0pt}{\isadigit{1}}\ assms{\isacharparenleft}{\kern0pt}{\isadigit{2}}{\isacharparenright}{\kern0pt}\ order{\isacharunderscore}{\kern0pt}trans\ \isacommand{by}\isamarkupfalse%
\ force\isanewline
\ \ \ \ \ \ \isacommand{have}\isamarkupfalse%
\ a{\isacharunderscore}{\kern0pt}{\isadigit{6}}{\isacharcolon}{\kern0pt}\ {\isachardoublequoteopen}{\isasymAnd}x{\isachardot}{\kern0pt}\ x\ {\isasymin}\ J{\isacharprime}{\kern0pt}\ {\isasymLongrightarrow}\ a\ x\ {\isasymin}\ carrier\ F{\isachardoublequoteclose}\ \ \isacommand{using}\isamarkupfalse%
\ True\ \isacommand{by}\isamarkupfalse%
\ force\isanewline
\ \ \ \ \ \ \isacommand{show}\isamarkupfalse%
\ {\isacharquery}{\kern0pt}thesis\ \isanewline
\ \ \ \ \ \ \ \isacommand{apply}\isamarkupfalse%
\ {\isacharparenleft}{\kern0pt}simp\ add{\isacharcolon}{\kern0pt}{\isasymOmega}{\isacharunderscore}{\kern0pt}def\ measure{\isacharunderscore}{\kern0pt}pmf{\isacharunderscore}{\kern0pt}of{\isacharunderscore}{\kern0pt}set{\isacharbrackleft}{\kern0pt}OF\ b\ c{\isacharbrackright}{\kern0pt}\ d\ hash{\isacharunderscore}{\kern0pt}def{\isacharparenright}{\kern0pt}\isanewline
\ \ \ \ \ \ \ \isacommand{apply}\isamarkupfalse%
\ {\isacharparenleft}{\kern0pt}subst\ poly{\isacharunderscore}{\kern0pt}cards{\isacharbrackleft}{\kern0pt}OF\ assms{\isacharparenleft}{\kern0pt}{\isadigit{1}}{\isacharparenright}{\kern0pt}\ a{\isacharunderscore}{\kern0pt}{\isadigit{3}}\ a{\isacharunderscore}{\kern0pt}{\isadigit{2}}{\isacharbrackright}{\kern0pt}{\isacharcomma}{\kern0pt}\ metis\ True{\isacharparenright}{\kern0pt}\isanewline
\ \ \ \ \ \ \ \ \isacommand{apply}\isamarkupfalse%
\ {\isacharparenleft}{\kern0pt}simp\ add{\isacharcolon}{\kern0pt}bounded{\isacharunderscore}{\kern0pt}degree{\isacharunderscore}{\kern0pt}polynomials{\isacharunderscore}{\kern0pt}count{\isacharbrackleft}{\kern0pt}OF\ e\ fin{\isacharunderscore}{\kern0pt}carr{\isacharbrackright}{\kern0pt}\ poly{\isacharunderscore}{\kern0pt}cards{\isacharunderscore}{\kern0pt}single{\isacharbrackleft}{\kern0pt}OF\ assms{\isacharparenleft}{\kern0pt}{\isadigit{1}}{\isacharparenright}{\kern0pt}\ a{\isacharunderscore}{\kern0pt}{\isadigit{5}}\ a{\isacharunderscore}{\kern0pt}{\isadigit{4}}\ a{\isacharunderscore}{\kern0pt}{\isadigit{6}}{\isacharbrackright}{\kern0pt}\ power{\isacharunderscore}{\kern0pt}divide{\isacharparenright}{\kern0pt}\isanewline
\ \ \ \ \ \ \ \ \isacommand{apply}\isamarkupfalse%
\ {\isacharparenleft}{\kern0pt}subst\ frac{\isacharunderscore}{\kern0pt}eq{\isacharunderscore}{\kern0pt}eq{\isacharcomma}{\kern0pt}\ simp\ add{\isacharcolon}{\kern0pt}f{\isacharcomma}{\kern0pt}\ simp\ add{\isacharcolon}{\kern0pt}f{\isacharparenright}{\kern0pt}\ \isanewline
\ \ \ \ \ \ \ \ \ \ \isacommand{apply}\isamarkupfalse%
\ {\isacharparenleft}{\kern0pt}simp\ add{\isacharcolon}{\kern0pt}power{\isacharunderscore}{\kern0pt}add{\isacharbrackleft}{\kern0pt}symmetric{\isacharbrackright}{\kern0pt}\ power{\isacharunderscore}{\kern0pt}mult{\isacharbrackleft}{\kern0pt}symmetric{\isacharbrackright}{\kern0pt}{\isacharparenright}{\kern0pt}\isanewline
\ \ \ \ \ \ \ \ \ \ \isacommand{apply}\isamarkupfalse%
\ {\isacharparenleft}{\kern0pt}rule\ arg{\isacharunderscore}{\kern0pt}cong{\isadigit{2}}{\isacharbrackleft}{\kern0pt}\isakeyword{where}\ f{\isacharequal}{\kern0pt}{\isachardoublequoteopen}{\isasymlambda}x\ y{\isachardot}{\kern0pt}\ x\ {\isacharcircum}{\kern0pt}\ y{\isachardoublequoteclose}{\isacharbrackright}{\kern0pt}{\isacharcomma}{\kern0pt}\ simp{\isacharparenright}{\kern0pt}\isanewline
\ \ \ \ \ \ \ \ \isacommand{using}\isamarkupfalse%
\ a{\isacharunderscore}{\kern0pt}{\isadigit{2}}\ a{\isacharunderscore}{\kern0pt}{\isadigit{4}}\ mult{\isacharunderscore}{\kern0pt}eq{\isacharunderscore}{\kern0pt}if\ \isacommand{by}\isamarkupfalse%
\ force\isanewline
\ \ \ \ \isacommand{next}\isamarkupfalse%
\isanewline
\ \ \ \ \ \ \isacommand{case}\isamarkupfalse%
\ False\isanewline
\ \ \ \ \ \ \isacommand{then}\isamarkupfalse%
\ \isacommand{obtain}\isamarkupfalse%
\ j\ \isakeyword{where}\ a{\isacharunderscore}{\kern0pt}{\isadigit{8}}{\isacharcolon}{\kern0pt}\ {\isachardoublequoteopen}j\ {\isasymin}\ J{\isacharprime}{\kern0pt}{\isachardoublequoteclose}\ \isakeyword{and}\ a{\isacharunderscore}{\kern0pt}{\isadigit{9}}{\isacharcolon}{\kern0pt}\ {\isachardoublequoteopen}a\ j\ {\isasymnotin}\ carrier\ F{\isachardoublequoteclose}\ \isacommand{by}\isamarkupfalse%
\ blast\isanewline
\ \ \ \ \ \ \isacommand{have}\isamarkupfalse%
\ a{\isacharunderscore}{\kern0pt}{\isadigit{7}}{\isacharcolon}{\kern0pt}\ {\isachardoublequoteopen}{\isasymAnd}x\ {\isasymomega}{\isachardot}{\kern0pt}\ {\isasymomega}\ {\isasymin}\ bounded{\isacharunderscore}{\kern0pt}degree{\isacharunderscore}{\kern0pt}polynomials\ F\ n\ {\isasymLongrightarrow}\ x\ {\isasymin}\ carrier\ F\ {\isasymLongrightarrow}\ hash\ F\ x\ {\isasymomega}\ {\isasymin}\ carrier\ F{\isachardoublequoteclose}\isanewline
\ \ \ \ \ \ \ \ \isacommand{apply}\isamarkupfalse%
\ {\isacharparenleft}{\kern0pt}simp\ add{\isacharcolon}{\kern0pt}bounded{\isacharunderscore}{\kern0pt}degree{\isacharunderscore}{\kern0pt}polynomials{\isacharunderscore}{\kern0pt}def\ hash{\isacharunderscore}{\kern0pt}def{\isacharparenright}{\kern0pt}\isanewline
\ \ \ \ \ \ \ \ \isacommand{by}\isamarkupfalse%
\ {\isacharparenleft}{\kern0pt}metis\ e\ ring{\isachardot}{\kern0pt}eval{\isacharunderscore}{\kern0pt}in{\isacharunderscore}{\kern0pt}carrier\ \ ring{\isachardot}{\kern0pt}polynomial{\isacharunderscore}{\kern0pt}incl\ univ{\isacharunderscore}{\kern0pt}poly{\isacharunderscore}{\kern0pt}carrier{\isacharparenright}{\kern0pt}\isanewline
\ \ \ \ \ \ \isacommand{have}\isamarkupfalse%
\ a{\isacharunderscore}{\kern0pt}{\isadigit{1}}{\isadigit{0}}{\isacharcolon}{\kern0pt}\ {\isachardoublequoteopen}{\isacharbraceleft}{\kern0pt}{\isasymomega}\ {\isasymin}\ bounded{\isacharunderscore}{\kern0pt}degree{\isacharunderscore}{\kern0pt}polynomials\ F\ n{\isachardot}{\kern0pt}\ {\isasymforall}x{\isasymin}J{\isacharprime}{\kern0pt}{\isachardot}{\kern0pt}\ hash\ F\ x\ {\isasymomega}\ {\isacharequal}{\kern0pt}\ a\ x{\isacharbraceright}{\kern0pt}\ {\isacharequal}{\kern0pt}\ {\isacharbraceleft}{\kern0pt}{\isacharbraceright}{\kern0pt}{\isachardoublequoteclose}\isanewline
\ \ \ \ \ \ \ \ \isacommand{apply}\isamarkupfalse%
\ {\isacharparenleft}{\kern0pt}rule\ order{\isacharunderscore}{\kern0pt}antisym{\isacharparenright}{\kern0pt}\isanewline
\ \ \ \ \ \ \ \ \isacommand{apply}\isamarkupfalse%
\ {\isacharparenleft}{\kern0pt}rule\ subsetI{\isacharcomma}{\kern0pt}\ simp{\isacharcomma}{\kern0pt}\ metis\ a{\isacharunderscore}{\kern0pt}{\isadigit{7}}\ a{\isacharunderscore}{\kern0pt}{\isadigit{8}}\ a{\isacharunderscore}{\kern0pt}{\isadigit{9}}\ \ a{\isacharunderscore}{\kern0pt}{\isadigit{3}}\ in{\isacharunderscore}{\kern0pt}mono{\isacharparenright}{\kern0pt}\isanewline
\ \ \ \ \ \ \ \ \isacommand{by}\isamarkupfalse%
\ {\isacharparenleft}{\kern0pt}rule\ subsetI{\isacharcomma}{\kern0pt}\ simp{\isacharparenright}{\kern0pt}\ \isanewline
\ \ \ \ \ \ \isacommand{have}\isamarkupfalse%
\ a{\isacharunderscore}{\kern0pt}{\isadigit{1}}{\isadigit{2}}{\isacharcolon}{\kern0pt}\ {\isachardoublequoteopen}{\isacharbraceleft}{\kern0pt}{\isasymomega}\ {\isasymin}\ bounded{\isacharunderscore}{\kern0pt}degree{\isacharunderscore}{\kern0pt}polynomials\ F\ n{\isachardot}{\kern0pt}\ hash\ F\ j\ {\isasymomega}\ {\isacharequal}{\kern0pt}\ a\ j{\isacharbraceright}{\kern0pt}\ {\isacharequal}{\kern0pt}\ {\isacharbraceleft}{\kern0pt}{\isacharbraceright}{\kern0pt}{\isachardoublequoteclose}\isanewline
\ \ \ \ \ \ \ \ \isacommand{apply}\isamarkupfalse%
\ {\isacharparenleft}{\kern0pt}rule\ order{\isacharunderscore}{\kern0pt}antisym{\isacharparenright}{\kern0pt}\isanewline
\ \ \ \ \ \ \ \ \isacommand{apply}\isamarkupfalse%
\ {\isacharparenleft}{\kern0pt}rule\ subsetI{\isacharcomma}{\kern0pt}\ simp{\isacharcomma}{\kern0pt}\ metis\ a{\isacharunderscore}{\kern0pt}{\isadigit{7}}\ a{\isacharunderscore}{\kern0pt}{\isadigit{8}}\ a{\isacharunderscore}{\kern0pt}{\isadigit{9}}\ \ a{\isacharunderscore}{\kern0pt}{\isadigit{3}}\ in{\isacharunderscore}{\kern0pt}mono{\isacharparenright}{\kern0pt}\isanewline
\ \ \ \ \ \ \ \ \isacommand{by}\isamarkupfalse%
\ {\isacharparenleft}{\kern0pt}rule\ subsetI{\isacharcomma}{\kern0pt}\ simp{\isacharparenright}{\kern0pt}\ \isanewline
\ \ \ \ \ \ \isacommand{then}\isamarkupfalse%
\ \isacommand{show}\isamarkupfalse%
\ {\isacharquery}{\kern0pt}thesis\isanewline
\ \ \ \ \ \ \ \ \isacommand{apply}\isamarkupfalse%
\ {\isacharparenleft}{\kern0pt}simp\ add{\isacharcolon}{\kern0pt}{\isasymOmega}{\isacharunderscore}{\kern0pt}def\ measure{\isacharunderscore}{\kern0pt}pmf{\isacharunderscore}{\kern0pt}of{\isacharunderscore}{\kern0pt}set{\isacharbrackleft}{\kern0pt}OF\ b\ c{\isacharbrackright}{\kern0pt}\ d\ a{\isacharunderscore}{\kern0pt}{\isadigit{1}}{\isadigit{0}}{\isacharparenright}{\kern0pt}\isanewline
\ \ \ \ \ \ \ \ \isacommand{apply}\isamarkupfalse%
\ {\isacharparenleft}{\kern0pt}rule\ prod{\isacharunderscore}{\kern0pt}zero{\isacharcomma}{\kern0pt}\ metis\ a{\isacharunderscore}{\kern0pt}{\isadigit{1}}{\isadigit{1}}{\isacharparenright}{\kern0pt}\isanewline
\ \ \ \ \ \ \ \ \isacommand{apply}\isamarkupfalse%
\ {\isacharparenleft}{\kern0pt}rule\ bexI{\isacharbrackleft}{\kern0pt}\isakeyword{where}\ x{\isacharequal}{\kern0pt}{\isachardoublequoteopen}j{\isachardoublequoteclose}{\isacharbrackright}{\kern0pt}{\isacharparenright}{\kern0pt}\isanewline
\ \ \ \ \ \ \ \ \isacommand{by}\isamarkupfalse%
\ {\isacharparenleft}{\kern0pt}simp\ add{\isacharcolon}{\kern0pt}a{\isacharunderscore}{\kern0pt}{\isadigit{1}}{\isadigit{2}}\ a{\isacharunderscore}{\kern0pt}{\isadigit{8}}{\isacharparenright}{\kern0pt}{\isacharplus}{\kern0pt}\isanewline
\ \ \ \ \isacommand{qed}\isamarkupfalse%
\isanewline
\ \ \isacommand{qed}\isamarkupfalse%
\isanewline
\ \ \isacommand{show}\isamarkupfalse%
\ {\isacharquery}{\kern0pt}thesis\isanewline
\ \ \ \ \isacommand{apply}\isamarkupfalse%
\ {\isacharparenleft}{\kern0pt}rule\ indep{\isacharunderscore}{\kern0pt}vars{\isacharunderscore}{\kern0pt}pmf{\isacharparenright}{\kern0pt}\isanewline
\ \ \ \ \isacommand{using}\isamarkupfalse%
\ a\ \isacommand{by}\isamarkupfalse%
\ {\isacharparenleft}{\kern0pt}simp\ add{\isacharcolon}{\kern0pt}{\isasymOmega}{\isacharunderscore}{\kern0pt}def{\isacharparenright}{\kern0pt}\isanewline
\isacommand{qed}\isamarkupfalse%
%
\endisatagproof
{\isafoldproof}%
%
\isadelimproof
%
\endisadelimproof
%
\begin{isamarkuptext}%
We introduce k-wise independent random variables using the existing definition of
independent random variables.%
\end{isamarkuptext}\isamarkuptrue%
\isacommand{definition}\isamarkupfalse%
\ {\isacharparenleft}{\kern0pt}\isakeyword{in}\ prob{\isacharunderscore}{\kern0pt}space{\isacharparenright}{\kern0pt}\ k{\isacharunderscore}{\kern0pt}wise{\isacharunderscore}{\kern0pt}indep{\isacharunderscore}{\kern0pt}vars\ {\isacharcolon}{\kern0pt}{\isacharcolon}{\kern0pt}\ \isanewline
\ \ {\isachardoublequoteopen}nat\ {\isasymRightarrow}\ {\isacharparenleft}{\kern0pt}{\isacharprime}{\kern0pt}b\ {\isasymRightarrow}\ {\isacharprime}{\kern0pt}c\ measure{\isacharparenright}{\kern0pt}\ {\isasymRightarrow}\ {\isacharparenleft}{\kern0pt}{\isacharprime}{\kern0pt}b\ {\isasymRightarrow}\ {\isacharprime}{\kern0pt}a\ {\isasymRightarrow}\ {\isacharprime}{\kern0pt}c{\isacharparenright}{\kern0pt}\ {\isasymRightarrow}\ {\isacharprime}{\kern0pt}b\ set\ {\isasymRightarrow}\ bool{\isachardoublequoteclose}\ \isakeyword{where}\isanewline
\ \ {\isachardoublequoteopen}k{\isacharunderscore}{\kern0pt}wise{\isacharunderscore}{\kern0pt}indep{\isacharunderscore}{\kern0pt}vars\ k\ M{\isacharprime}{\kern0pt}\ X{\isacharprime}{\kern0pt}\ I\ {\isacharequal}{\kern0pt}\ {\isacharparenleft}{\kern0pt}{\isasymforall}J\ {\isasymsubseteq}\ I{\isachardot}{\kern0pt}\ card\ J\ {\isasymle}\ k\ {\isasymlongrightarrow}\ finite\ J\ {\isasymlongrightarrow}\ indep{\isacharunderscore}{\kern0pt}vars\ M{\isacharprime}{\kern0pt}\ X{\isacharprime}{\kern0pt}\ J{\isacharparenright}{\kern0pt}{\isachardoublequoteclose}\ \isanewline
\isanewline
\isacommand{lemma}\isamarkupfalse%
\ hash{\isacharunderscore}{\kern0pt}k{\isacharunderscore}{\kern0pt}wise{\isacharunderscore}{\kern0pt}indep{\isacharcolon}{\kern0pt}\isanewline
\ \ \isakeyword{assumes}\ {\isachardoublequoteopen}field\ F\ {\isasymand}\ finite\ {\isacharparenleft}{\kern0pt}carrier\ F{\isacharparenright}{\kern0pt}{\isachardoublequoteclose}\isanewline
\ \ \isakeyword{assumes}\ {\isachardoublequoteopen}{\isadigit{1}}\ {\isasymle}\ n{\isachardoublequoteclose}\isanewline
\ \ \isakeyword{shows}\ {\isachardoublequoteopen}prob{\isacharunderscore}{\kern0pt}space{\isachardot}{\kern0pt}k{\isacharunderscore}{\kern0pt}wise{\isacharunderscore}{\kern0pt}indep{\isacharunderscore}{\kern0pt}vars\ {\isacharparenleft}{\kern0pt}pmf{\isacharunderscore}{\kern0pt}of{\isacharunderscore}{\kern0pt}set\ {\isacharparenleft}{\kern0pt}bounded{\isacharunderscore}{\kern0pt}degree{\isacharunderscore}{\kern0pt}polynomials\ F\ n{\isacharparenright}{\kern0pt}{\isacharparenright}{\kern0pt}\ n\isanewline
\ \ \ \ {\isacharparenleft}{\kern0pt}{\isasymlambda}{\isacharunderscore}{\kern0pt}{\isachardot}{\kern0pt}\ pmf{\isacharunderscore}{\kern0pt}of{\isacharunderscore}{\kern0pt}set\ {\isacharparenleft}{\kern0pt}carrier\ F{\isacharparenright}{\kern0pt}{\isacharparenright}{\kern0pt}\ {\isacharparenleft}{\kern0pt}hash\ F{\isacharparenright}{\kern0pt}\ {\isacharparenleft}{\kern0pt}carrier\ F{\isacharparenright}{\kern0pt}{\isachardoublequoteclose}\isanewline
%
\isadelimproof
\ \ %
\endisadelimproof
%
\isatagproof
\isacommand{apply}\isamarkupfalse%
\ {\isacharparenleft}{\kern0pt}simp\ add{\isacharcolon}{\kern0pt}measure{\isacharunderscore}{\kern0pt}pmf{\isachardot}{\kern0pt}k{\isacharunderscore}{\kern0pt}wise{\isacharunderscore}{\kern0pt}indep{\isacharunderscore}{\kern0pt}vars{\isacharunderscore}{\kern0pt}def{\isacharparenright}{\kern0pt}\isanewline
\ \ \isacommand{using}\isamarkupfalse%
\ hash{\isacharunderscore}{\kern0pt}indep{\isacharunderscore}{\kern0pt}pmf{\isacharbrackleft}{\kern0pt}OF\ assms{\isacharparenleft}{\kern0pt}{\isadigit{1}}{\isacharparenright}{\kern0pt}\ {\isacharunderscore}{\kern0pt}\ {\isacharunderscore}{\kern0pt}\ {\isacharunderscore}{\kern0pt}\ assms{\isacharparenleft}{\kern0pt}{\isadigit{2}}{\isacharparenright}{\kern0pt}{\isacharbrackright}{\kern0pt}\ \isacommand{by}\isamarkupfalse%
\ blast%
\endisatagproof
{\isafoldproof}%
%
\isadelimproof
\isanewline
%
\endisadelimproof
\isanewline
\isacommand{lemma}\isamarkupfalse%
\ hash{\isacharunderscore}{\kern0pt}inj{\isacharunderscore}{\kern0pt}if{\isacharunderscore}{\kern0pt}degree{\isacharunderscore}{\kern0pt}{\isadigit{1}}{\isacharcolon}{\kern0pt}\isanewline
\ \ \isakeyword{assumes}\ {\isachardoublequoteopen}field\ F\ {\isasymand}\ finite\ {\isacharparenleft}{\kern0pt}carrier\ F{\isacharparenright}{\kern0pt}{\isachardoublequoteclose}\isanewline
\ \ \isakeyword{assumes}\ {\isachardoublequoteopen}{\isasymomega}\ {\isasymin}\ bounded{\isacharunderscore}{\kern0pt}degree{\isacharunderscore}{\kern0pt}polynomials\ F\ n{\isachardoublequoteclose}\isanewline
\ \ \isakeyword{assumes}\ {\isachardoublequoteopen}degree\ {\isasymomega}\ {\isacharequal}{\kern0pt}\ {\isadigit{1}}{\isachardoublequoteclose}\isanewline
\ \ \isakeyword{shows}\ {\isachardoublequoteopen}inj{\isacharunderscore}{\kern0pt}on\ {\isacharparenleft}{\kern0pt}{\isasymlambda}x{\isachardot}{\kern0pt}\ hash\ F\ x\ {\isasymomega}{\isacharparenright}{\kern0pt}\ {\isacharparenleft}{\kern0pt}carrier\ F{\isacharparenright}{\kern0pt}{\isachardoublequoteclose}\isanewline
%
\isadelimproof
%
\endisadelimproof
%
\isatagproof
\isacommand{proof}\isamarkupfalse%
\ {\isacharparenleft}{\kern0pt}rule\ inj{\isacharunderscore}{\kern0pt}onI{\isacharparenright}{\kern0pt}\isanewline
\ \ \isacommand{fix}\isamarkupfalse%
\ x\ y\isanewline
\ \ \isacommand{assume}\isamarkupfalse%
\ a{\isadigit{1}}{\isacharcolon}{\kern0pt}\ {\isachardoublequoteopen}x\ {\isasymin}\ carrier\ F{\isachardoublequoteclose}\isanewline
\ \ \isacommand{assume}\isamarkupfalse%
\ a{\isadigit{2}}{\isacharcolon}{\kern0pt}\ {\isachardoublequoteopen}y\ {\isasymin}\ carrier\ F{\isachardoublequoteclose}\isanewline
\ \ \isacommand{assume}\isamarkupfalse%
\ a{\isadigit{3}}{\isacharcolon}{\kern0pt}\ {\isachardoublequoteopen}hash\ F\ x\ {\isasymomega}\ {\isacharequal}{\kern0pt}\ hash\ F\ y\ {\isasymomega}{\isachardoublequoteclose}\isanewline
\isanewline
\ \ \isacommand{interpret}\isamarkupfalse%
\ field\ F\isanewline
\ \ \ \ \isacommand{by}\isamarkupfalse%
\ {\isacharparenleft}{\kern0pt}metis\ assms{\isacharparenleft}{\kern0pt}{\isadigit{1}}{\isacharparenright}{\kern0pt}{\isacharparenright}{\kern0pt}\isanewline
\isanewline
\ \ \isacommand{obtain}\isamarkupfalse%
\ u\ v\ \isakeyword{where}\ {\isasymomega}{\isacharunderscore}{\kern0pt}def{\isacharcolon}{\kern0pt}\ {\isachardoublequoteopen}{\isasymomega}\ {\isacharequal}{\kern0pt}\ {\isacharbrackleft}{\kern0pt}u{\isacharcomma}{\kern0pt}v{\isacharbrackright}{\kern0pt}{\isachardoublequoteclose}\ \isacommand{using}\isamarkupfalse%
\ assms{\isacharparenleft}{\kern0pt}{\isadigit{3}}{\isacharparenright}{\kern0pt}\isanewline
\ \ \ \ \isacommand{apply}\isamarkupfalse%
\ {\isacharparenleft}{\kern0pt}cases\ {\isasymomega}{\isacharcomma}{\kern0pt}\ simp{\isacharparenright}{\kern0pt}\isanewline
\ \ \ \ \isacommand{by}\isamarkupfalse%
\ {\isacharparenleft}{\kern0pt}cases\ {\isachardoublequoteopen}{\isacharparenleft}{\kern0pt}tl\ {\isasymomega}{\isacharparenright}{\kern0pt}{\isachardoublequoteclose}{\isacharcomma}{\kern0pt}\ simp{\isacharcomma}{\kern0pt}\ simp{\isacharparenright}{\kern0pt}\isanewline
\isanewline
\ \ \isacommand{have}\isamarkupfalse%
\ u{\isacharunderscore}{\kern0pt}carr{\isacharcolon}{\kern0pt}\ {\isachardoublequoteopen}u\ {\isasymin}\ carrier\ F\ {\isacharminus}{\kern0pt}\ {\isacharbraceleft}{\kern0pt}{\isasymzero}\isactrlbsub F\isactrlesub {\isacharbraceright}{\kern0pt}{\isachardoublequoteclose}\isanewline
\ \ \ \ \isacommand{using}\isamarkupfalse%
\ {\isasymomega}{\isacharunderscore}{\kern0pt}def\ assms\ \isacommand{apply}\isamarkupfalse%
\ {\isacharparenleft}{\kern0pt}simp\ add{\isacharcolon}{\kern0pt}bounded{\isacharunderscore}{\kern0pt}degree{\isacharunderscore}{\kern0pt}polynomials{\isacharunderscore}{\kern0pt}def{\isacharparenright}{\kern0pt}\isanewline
\ \ \ \ \isacommand{by}\isamarkupfalse%
\ {\isacharparenleft}{\kern0pt}metis\ field{\isachardot}{\kern0pt}is{\isacharunderscore}{\kern0pt}ring\ list{\isachardot}{\kern0pt}sel{\isacharparenleft}{\kern0pt}{\isadigit{1}}{\isacharparenright}{\kern0pt}\ ring{\isachardot}{\kern0pt}degree{\isacharunderscore}{\kern0pt}oneE\ assms{\isacharparenleft}{\kern0pt}{\isadigit{1}}{\isacharparenright}{\kern0pt}\ assms{\isacharparenleft}{\kern0pt}{\isadigit{3}}{\isacharparenright}{\kern0pt}{\isacharparenright}{\kern0pt}\isanewline
\isanewline
\ \ \isacommand{have}\isamarkupfalse%
\ v{\isacharunderscore}{\kern0pt}carr{\isacharcolon}{\kern0pt}\ {\isachardoublequoteopen}v\ {\isasymin}\ carrier\ F{\isachardoublequoteclose}\ \isanewline
\ \ \ \ \isacommand{using}\isamarkupfalse%
\ {\isasymomega}{\isacharunderscore}{\kern0pt}def\ assms{\isacharparenleft}{\kern0pt}{\isadigit{2}}{\isacharparenright}{\kern0pt}\ \isacommand{apply}\isamarkupfalse%
\ {\isacharparenleft}{\kern0pt}simp\ add{\isacharcolon}{\kern0pt}bounded{\isacharunderscore}{\kern0pt}degree{\isacharunderscore}{\kern0pt}polynomials{\isacharunderscore}{\kern0pt}def{\isacharparenright}{\kern0pt}\isanewline
\ \ \ \ \isacommand{by}\isamarkupfalse%
\ {\isacharparenleft}{\kern0pt}metis\ assms{\isacharparenleft}{\kern0pt}{\isadigit{1}}{\isacharparenright}{\kern0pt}\ assms{\isacharparenleft}{\kern0pt}{\isadigit{3}}{\isacharparenright}{\kern0pt}\ field{\isachardot}{\kern0pt}is{\isacharunderscore}{\kern0pt}ring\ list{\isachardot}{\kern0pt}inject\ ring{\isachardot}{\kern0pt}degree{\isacharunderscore}{\kern0pt}oneE{\isacharparenright}{\kern0pt}\isanewline
\isanewline
\ \ \isacommand{have}\isamarkupfalse%
\ {\isachardoublequoteopen}u\ {\isasymotimes}\isactrlbsub F\isactrlesub \ x\ {\isasymoplus}\isactrlbsub F\isactrlesub \ v\ {\isacharequal}{\kern0pt}\ u\ {\isasymotimes}\isactrlbsub F\isactrlesub \ y\ \ {\isasymoplus}\isactrlbsub F\isactrlesub \ v{\isachardoublequoteclose}\isanewline
\ \ \ \ \isacommand{using}\isamarkupfalse%
\ a{\isadigit{1}}\ a{\isadigit{2}}\ a{\isadigit{3}}\ u{\isacharunderscore}{\kern0pt}carr\ v{\isacharunderscore}{\kern0pt}carr\ \isacommand{by}\isamarkupfalse%
\ {\isacharparenleft}{\kern0pt}simp\ add{\isacharcolon}{\kern0pt}hash{\isacharunderscore}{\kern0pt}def\ {\isasymomega}{\isacharunderscore}{\kern0pt}def{\isacharparenright}{\kern0pt}\isanewline
\isanewline
\ \ \isacommand{thus}\isamarkupfalse%
\ {\isachardoublequoteopen}x\ {\isacharequal}{\kern0pt}\ y{\isachardoublequoteclose}\isanewline
\ \ \ \ \isacommand{using}\isamarkupfalse%
\ u{\isacharunderscore}{\kern0pt}carr\ a{\isadigit{1}}\ a{\isadigit{2}}\ v{\isacharunderscore}{\kern0pt}carr\isanewline
\ \ \ \ \isacommand{by}\isamarkupfalse%
\ {\isacharparenleft}{\kern0pt}simp\ add{\isacharcolon}{\kern0pt}\ local{\isachardot}{\kern0pt}field{\isacharunderscore}{\kern0pt}Units{\isacharparenright}{\kern0pt}\isanewline
\isacommand{qed}\isamarkupfalse%
%
\endisatagproof
{\isafoldproof}%
%
\isadelimproof
\isanewline
%
\endisadelimproof
\isanewline
\isacommand{lemma}\isamarkupfalse%
\ {\isacharparenleft}{\kern0pt}\isakeyword{in}\ prob{\isacharunderscore}{\kern0pt}space{\isacharparenright}{\kern0pt}\ k{\isacharunderscore}{\kern0pt}wise{\isacharunderscore}{\kern0pt}subset{\isacharcolon}{\kern0pt}\isanewline
\ \ \isakeyword{assumes}\ {\isachardoublequoteopen}k{\isacharunderscore}{\kern0pt}wise{\isacharunderscore}{\kern0pt}indep{\isacharunderscore}{\kern0pt}vars\ k\ M{\isacharprime}{\kern0pt}\ X{\isacharprime}{\kern0pt}\ I{\isachardoublequoteclose}\isanewline
\ \ \isakeyword{assumes}\ {\isachardoublequoteopen}J\ {\isasymsubseteq}\ I{\isachardoublequoteclose}\isanewline
\ \ \isakeyword{shows}\ {\isachardoublequoteopen}k{\isacharunderscore}{\kern0pt}wise{\isacharunderscore}{\kern0pt}indep{\isacharunderscore}{\kern0pt}vars\ k\ M{\isacharprime}{\kern0pt}\ X{\isacharprime}{\kern0pt}\ J{\isachardoublequoteclose}\isanewline
%
\isadelimproof
\ \ %
\endisadelimproof
%
\isatagproof
\isacommand{using}\isamarkupfalse%
\ assms\ \isacommand{by}\isamarkupfalse%
\ {\isacharparenleft}{\kern0pt}simp\ add{\isacharcolon}{\kern0pt}k{\isacharunderscore}{\kern0pt}wise{\isacharunderscore}{\kern0pt}indep{\isacharunderscore}{\kern0pt}vars{\isacharunderscore}{\kern0pt}def{\isacharparenright}{\kern0pt}%
\endisatagproof
{\isafoldproof}%
%
\isadelimproof
\isanewline
%
\endisadelimproof
%
\isadelimtheory
\isanewline
%
\endisadelimtheory
%
\isatagtheory
\isacommand{end}\isamarkupfalse%
%
\endisatagtheory
{\isafoldtheory}%
%
\isadelimtheory
%
\endisadelimtheory
%
\end{isabellebody}%
\endinput
%:%file=UniversalHashFamily.tex%:%
%:%11=1%:%
%:%27=3%:%
%:%28=3%:%
%:%29=4%:%
%:%30=5%:%
%:%39=7%:%
%:%40=8%:%
%:%41=9%:%
%:%42=10%:%
%:%43=11%:%
%:%44=12%:%
%:%45=13%:%
%:%46=14%:%
%:%47=15%:%
%:%48=16%:%
%:%49=17%:%
%:%50=18%:%
%:%54=20%:%
%:%56=22%:%
%:%57=22%:%
%:%58=23%:%
%:%59=24%:%
%:%60=25%:%
%:%61=25%:%
%:%62=26%:%
%:%63=27%:%
%:%64=28%:%
%:%65=29%:%
%:%68=30%:%
%:%72=30%:%
%:%73=30%:%
%:%74=31%:%
%:%75=31%:%
%:%76=32%:%
%:%77=32%:%
%:%82=32%:%
%:%85=33%:%
%:%86=34%:%
%:%87=34%:%
%:%88=35%:%
%:%89=36%:%
%:%90=37%:%
%:%93=38%:%
%:%97=38%:%
%:%98=38%:%
%:%99=39%:%
%:%100=39%:%
%:%105=39%:%
%:%108=40%:%
%:%109=41%:%
%:%110=41%:%
%:%111=42%:%
%:%112=43%:%
%:%113=44%:%
%:%114=45%:%
%:%115=46%:%
%:%116=47%:%
%:%119=48%:%
%:%123=48%:%
%:%124=48%:%
%:%125=49%:%
%:%126=49%:%
%:%131=49%:%
%:%134=50%:%
%:%135=51%:%
%:%136=51%:%
%:%137=52%:%
%:%138=53%:%
%:%139=54%:%
%:%140=55%:%
%:%141=56%:%
%:%142=57%:%
%:%145=58%:%
%:%149=58%:%
%:%150=58%:%
%:%151=59%:%
%:%152=59%:%
%:%157=59%:%
%:%160=60%:%
%:%161=61%:%
%:%162=61%:%
%:%165=62%:%
%:%169=62%:%
%:%170=62%:%
%:%175=62%:%
%:%178=63%:%
%:%179=64%:%
%:%180=64%:%
%:%181=65%:%
%:%182=66%:%
%:%183=67%:%
%:%184=68%:%
%:%185=69%:%
%:%192=70%:%
%:%193=70%:%
%:%194=71%:%
%:%195=71%:%
%:%196=72%:%
%:%197=72%:%
%:%198=72%:%
%:%199=73%:%
%:%200=74%:%
%:%201=74%:%
%:%202=75%:%
%:%203=75%:%
%:%204=76%:%
%:%205=76%:%
%:%206=76%:%
%:%207=77%:%
%:%208=78%:%
%:%209=78%:%
%:%210=79%:%
%:%211=79%:%
%:%212=80%:%
%:%213=80%:%
%:%214=81%:%
%:%215=81%:%
%:%216=82%:%
%:%217=82%:%
%:%218=83%:%
%:%219=83%:%
%:%220=84%:%
%:%221=84%:%
%:%222=85%:%
%:%223=85%:%
%:%224=86%:%
%:%225=86%:%
%:%226=87%:%
%:%227=87%:%
%:%228=88%:%
%:%234=88%:%
%:%237=89%:%
%:%238=90%:%
%:%239=90%:%
%:%240=91%:%
%:%241=92%:%
%:%242=93%:%
%:%243=94%:%
%:%244=95%:%
%:%247=96%:%
%:%251=96%:%
%:%252=96%:%
%:%253=97%:%
%:%254=97%:%
%:%259=97%:%
%:%262=98%:%
%:%263=99%:%
%:%264=99%:%
%:%265=100%:%
%:%266=101%:%
%:%267=102%:%
%:%268=103%:%
%:%269=104%:%
%:%270=105%:%
%:%271=106%:%
%:%278=107%:%
%:%279=107%:%
%:%280=108%:%
%:%281=108%:%
%:%282=109%:%
%:%283=109%:%
%:%284=110%:%
%:%285=110%:%
%:%286=111%:%
%:%287=111%:%
%:%288=112%:%
%:%289=112%:%
%:%290=112%:%
%:%291=113%:%
%:%292=113%:%
%:%293=114%:%
%:%294=114%:%
%:%295=115%:%
%:%296=115%:%
%:%297=116%:%
%:%298=116%:%
%:%299=117%:%
%:%300=117%:%
%:%301=118%:%
%:%302=118%:%
%:%303=119%:%
%:%304=120%:%
%:%305=120%:%
%:%306=120%:%
%:%307=120%:%
%:%308=121%:%
%:%309=121%:%
%:%310=121%:%
%:%311=121%:%
%:%312=122%:%
%:%313=122%:%
%:%314=123%:%
%:%315=123%:%
%:%316=124%:%
%:%317=125%:%
%:%318=125%:%
%:%319=126%:%
%:%320=126%:%
%:%324=130%:%
%:%325=131%:%
%:%326=131%:%
%:%327=132%:%
%:%328=132%:%
%:%329=133%:%
%:%330=133%:%
%:%331=134%:%
%:%332=134%:%
%:%333=135%:%
%:%334=135%:%
%:%335=136%:%
%:%336=136%:%
%:%337=136%:%
%:%338=137%:%
%:%339=137%:%
%:%340=137%:%
%:%341=138%:%
%:%342=138%:%
%:%343=138%:%
%:%344=138%:%
%:%345=139%:%
%:%346=140%:%
%:%347=140%:%
%:%348=141%:%
%:%349=142%:%
%:%350=142%:%
%:%351=143%:%
%:%352=143%:%
%:%353=144%:%
%:%354=144%:%
%:%355=144%:%
%:%356=144%:%
%:%357=145%:%
%:%358=145%:%
%:%359=145%:%
%:%360=145%:%
%:%361=146%:%
%:%362=146%:%
%:%363=147%:%
%:%364=147%:%
%:%365=148%:%
%:%366=148%:%
%:%367=149%:%
%:%368=149%:%
%:%369=150%:%
%:%370=150%:%
%:%371=151%:%
%:%372=151%:%
%:%373=152%:%
%:%374=152%:%
%:%375=153%:%
%:%376=153%:%
%:%377=153%:%
%:%378=154%:%
%:%379=154%:%
%:%380=155%:%
%:%381=155%:%
%:%382=156%:%
%:%383=156%:%
%:%384=156%:%
%:%385=156%:%
%:%386=157%:%
%:%387=157%:%
%:%388=158%:%
%:%389=158%:%
%:%390=159%:%
%:%391=159%:%
%:%392=160%:%
%:%393=160%:%
%:%394=161%:%
%:%395=161%:%
%:%396=162%:%
%:%397=162%:%
%:%398=163%:%
%:%399=163%:%
%:%400=164%:%
%:%401=164%:%
%:%402=165%:%
%:%403=165%:%
%:%404=166%:%
%:%405=166%:%
%:%406=167%:%
%:%407=167%:%
%:%408=168%:%
%:%409=168%:%
%:%410=168%:%
%:%411=169%:%
%:%412=169%:%
%:%413=170%:%
%:%414=170%:%
%:%415=171%:%
%:%416=171%:%
%:%417=172%:%
%:%418=172%:%
%:%419=173%:%
%:%420=173%:%
%:%421=174%:%
%:%422=174%:%
%:%423=175%:%
%:%424=175%:%
%:%425=176%:%
%:%426=176%:%
%:%427=177%:%
%:%428=177%:%
%:%429=177%:%
%:%430=178%:%
%:%440=180%:%
%:%441=181%:%
%:%443=183%:%
%:%444=183%:%
%:%445=184%:%
%:%446=185%:%
%:%447=186%:%
%:%448=187%:%
%:%449=187%:%
%:%450=188%:%
%:%451=189%:%
%:%452=190%:%
%:%453=191%:%
%:%456=192%:%
%:%460=192%:%
%:%461=192%:%
%:%462=193%:%
%:%463=193%:%
%:%464=193%:%
%:%469=193%:%
%:%472=194%:%
%:%473=195%:%
%:%474=195%:%
%:%475=196%:%
%:%476=197%:%
%:%477=198%:%
%:%478=199%:%
%:%485=200%:%
%:%486=200%:%
%:%487=201%:%
%:%488=201%:%
%:%489=202%:%
%:%490=202%:%
%:%491=203%:%
%:%492=203%:%
%:%493=204%:%
%:%494=204%:%
%:%495=205%:%
%:%496=206%:%
%:%497=206%:%
%:%498=207%:%
%:%499=207%:%
%:%500=208%:%
%:%501=209%:%
%:%502=209%:%
%:%503=209%:%
%:%504=210%:%
%:%505=210%:%
%:%506=211%:%
%:%507=211%:%
%:%508=212%:%
%:%509=213%:%
%:%510=213%:%
%:%511=214%:%
%:%512=214%:%
%:%513=214%:%
%:%514=215%:%
%:%515=215%:%
%:%516=216%:%
%:%517=217%:%
%:%518=217%:%
%:%519=218%:%
%:%520=218%:%
%:%521=218%:%
%:%522=219%:%
%:%523=219%:%
%:%524=220%:%
%:%525=221%:%
%:%526=221%:%
%:%527=222%:%
%:%528=222%:%
%:%529=222%:%
%:%530=223%:%
%:%531=224%:%
%:%532=224%:%
%:%533=225%:%
%:%534=225%:%
%:%535=226%:%
%:%536=226%:%
%:%537=227%:%
%:%543=227%:%
%:%546=228%:%
%:%547=229%:%
%:%548=229%:%
%:%549=230%:%
%:%550=231%:%
%:%551=232%:%
%:%554=233%:%
%:%558=233%:%
%:%559=233%:%
%:%560=233%:%
%:%565=233%:%
%:%570=234%:%
%:%575=235%:%

%
\begin{isabellebody}%
\setisabellecontext{UniversalHashFamilyOfPrime}%
%
\isadelimdocument
%
\endisadelimdocument
%
\isatagdocument
%
\isamarkupsection{Universal Hash Family for $\{0..<p\}$%
}
\isamarkuptrue%
%
\endisatagdocument
{\isafolddocument}%
%
\isadelimdocument
%
\endisadelimdocument
%
\begin{isamarkuptext}%
Specialization of universal hash families from arbitrary finite 
  fields to $\{0..<p\}$.%
\end{isamarkuptext}\isamarkuptrue%
%
\isadelimtheory
%
\endisadelimtheory
%
\isatagtheory
\isacommand{theory}\isamarkupfalse%
\ UniversalHashFamilyOfPrime\isanewline
\ \ \isakeyword{imports}\ Field\ UniversalHashFamily\ Probability{\isacharunderscore}{\kern0pt}Ext\ Encoding\isanewline
\isakeyword{begin}%
\endisatagtheory
{\isafoldtheory}%
%
\isadelimtheory
%
\endisadelimtheory
\isanewline
\isanewline
\isacommand{lemma}\isamarkupfalse%
\ fin{\isacharunderscore}{\kern0pt}bounded{\isacharunderscore}{\kern0pt}degree{\isacharunderscore}{\kern0pt}polynomials{\isacharcolon}{\kern0pt}\isanewline
\ \ \isakeyword{assumes}\ {\isachardoublequoteopen}p\ {\isachargreater}{\kern0pt}\ {\isadigit{0}}{\isachardoublequoteclose}\isanewline
\ \ \isakeyword{shows}\ {\isachardoublequoteopen}finite\ {\isacharparenleft}{\kern0pt}bounded{\isacharunderscore}{\kern0pt}degree{\isacharunderscore}{\kern0pt}polynomials\ {\isacharparenleft}{\kern0pt}ZFact\ {\isacharparenleft}{\kern0pt}int\ p{\isacharparenright}{\kern0pt}{\isacharparenright}{\kern0pt}\ n{\isacharparenright}{\kern0pt}{\isachardoublequoteclose}\isanewline
%
\isadelimproof
\ \ %
\endisadelimproof
%
\isatagproof
\isacommand{apply}\isamarkupfalse%
\ {\isacharparenleft}{\kern0pt}rule\ fin{\isacharunderscore}{\kern0pt}degree{\isacharunderscore}{\kern0pt}bounded{\isacharparenright}{\kern0pt}\isanewline
\ \ \ \isacommand{apply}\isamarkupfalse%
\ {\isacharparenleft}{\kern0pt}metis\ ZFact{\isacharunderscore}{\kern0pt}is{\isacharunderscore}{\kern0pt}cring\ cring{\isacharunderscore}{\kern0pt}def{\isacharparenright}{\kern0pt}\isanewline
\ \ \isacommand{by}\isamarkupfalse%
\ {\isacharparenleft}{\kern0pt}rule\ zfact{\isacharunderscore}{\kern0pt}finite{\isacharbrackleft}{\kern0pt}OF\ assms{\isacharbrackright}{\kern0pt}{\isacharparenright}{\kern0pt}%
\endisatagproof
{\isafoldproof}%
%
\isadelimproof
\isanewline
%
\endisadelimproof
\isanewline
\isacommand{lemma}\isamarkupfalse%
\ ne{\isacharunderscore}{\kern0pt}bounded{\isacharunderscore}{\kern0pt}degree{\isacharunderscore}{\kern0pt}polynomials{\isacharcolon}{\kern0pt}\isanewline
\ \ \isakeyword{shows}\ {\isachardoublequoteopen}bounded{\isacharunderscore}{\kern0pt}degree{\isacharunderscore}{\kern0pt}polynomials\ {\isacharparenleft}{\kern0pt}ZFact\ {\isacharparenleft}{\kern0pt}int\ p{\isacharparenright}{\kern0pt}{\isacharparenright}{\kern0pt}\ n\ {\isasymnoteq}\ {\isacharbraceleft}{\kern0pt}{\isacharbraceright}{\kern0pt}{\isachardoublequoteclose}\isanewline
%
\isadelimproof
\ \ %
\endisadelimproof
%
\isatagproof
\isacommand{apply}\isamarkupfalse%
\ {\isacharparenleft}{\kern0pt}rule\ non{\isacharunderscore}{\kern0pt}empty{\isacharunderscore}{\kern0pt}bounded{\isacharunderscore}{\kern0pt}degree{\isacharunderscore}{\kern0pt}polynomials{\isacharparenright}{\kern0pt}\isanewline
\ \ \isacommand{by}\isamarkupfalse%
\ {\isacharparenleft}{\kern0pt}metis\ ZFact{\isacharunderscore}{\kern0pt}is{\isacharunderscore}{\kern0pt}cring\ cring{\isacharunderscore}{\kern0pt}def{\isacharparenright}{\kern0pt}%
\endisatagproof
{\isafoldproof}%
%
\isadelimproof
\isanewline
%
\endisadelimproof
\isanewline
\isacommand{lemma}\isamarkupfalse%
\ card{\isacharunderscore}{\kern0pt}bounded{\isacharunderscore}{\kern0pt}degree{\isacharunderscore}{\kern0pt}polynomials{\isacharcolon}{\kern0pt}\isanewline
\ \ \isakeyword{assumes}\ {\isachardoublequoteopen}p\ {\isachargreater}{\kern0pt}\ {\isadigit{0}}{\isachardoublequoteclose}\isanewline
\ \ \isakeyword{shows}\ {\isachardoublequoteopen}card\ {\isacharparenleft}{\kern0pt}bounded{\isacharunderscore}{\kern0pt}degree{\isacharunderscore}{\kern0pt}polynomials\ {\isacharparenleft}{\kern0pt}ZFact\ {\isacharparenleft}{\kern0pt}int\ p{\isacharparenright}{\kern0pt}{\isacharparenright}{\kern0pt}\ n{\isacharparenright}{\kern0pt}\ {\isacharequal}{\kern0pt}\ p{\isacharcircum}{\kern0pt}n{\isachardoublequoteclose}\isanewline
%
\isadelimproof
\ \ %
\endisadelimproof
%
\isatagproof
\isacommand{apply}\isamarkupfalse%
\ {\isacharparenleft}{\kern0pt}subst\ bounded{\isacharunderscore}{\kern0pt}degree{\isacharunderscore}{\kern0pt}polynomials{\isacharunderscore}{\kern0pt}count{\isacharparenright}{\kern0pt}\isanewline
\ \ \ \ \isacommand{apply}\isamarkupfalse%
\ {\isacharparenleft}{\kern0pt}metis\ ZFact{\isacharunderscore}{\kern0pt}is{\isacharunderscore}{\kern0pt}cring\ cring{\isacharunderscore}{\kern0pt}def{\isacharparenright}{\kern0pt}\isanewline
\ \ \ \isacommand{apply}\isamarkupfalse%
\ {\isacharparenleft}{\kern0pt}rule\ zfact{\isacharunderscore}{\kern0pt}finite{\isacharbrackleft}{\kern0pt}OF\ assms{\isacharbrackright}{\kern0pt}{\isacharparenright}{\kern0pt}\isanewline
\ \ \isacommand{by}\isamarkupfalse%
\ {\isacharparenleft}{\kern0pt}subst\ zfact{\isacharunderscore}{\kern0pt}card{\isacharcomma}{\kern0pt}\ metis\ assms{\isacharcomma}{\kern0pt}\ simp{\isacharparenright}{\kern0pt}%
\endisatagproof
{\isafoldproof}%
%
\isadelimproof
\isanewline
%
\endisadelimproof
\isanewline
\isacommand{fun}\isamarkupfalse%
\ hash\ {\isacharcolon}{\kern0pt}{\isacharcolon}{\kern0pt}\ {\isachardoublequoteopen}nat\ {\isasymRightarrow}\ nat\ {\isasymRightarrow}\ int\ set\ list\ {\isasymRightarrow}\ nat{\isachardoublequoteclose}\isanewline
\ \ \isakeyword{where}\ {\isachardoublequoteopen}hash\ p\ x\ f\ {\isacharequal}{\kern0pt}\ the{\isacharunderscore}{\kern0pt}inv{\isacharunderscore}{\kern0pt}into\ {\isacharbraceleft}{\kern0pt}{\isadigit{0}}{\isachardot}{\kern0pt}{\isachardot}{\kern0pt}{\isacharless}{\kern0pt}p{\isacharbraceright}{\kern0pt}\ {\isacharparenleft}{\kern0pt}zfact{\isacharunderscore}{\kern0pt}embed\ p{\isacharparenright}{\kern0pt}\ {\isacharparenleft}{\kern0pt}UniversalHashFamily{\isachardot}{\kern0pt}hash\ {\isacharparenleft}{\kern0pt}ZFact\ p{\isacharparenright}{\kern0pt}\ {\isacharparenleft}{\kern0pt}zfact{\isacharunderscore}{\kern0pt}embed\ p\ x{\isacharparenright}{\kern0pt}\ f{\isacharparenright}{\kern0pt}{\isachardoublequoteclose}\isanewline
\isanewline
\isacommand{declare}\isamarkupfalse%
\ hash{\isachardot}{\kern0pt}simps\ {\isacharbrackleft}{\kern0pt}simp\ del{\isacharbrackright}{\kern0pt}\isanewline
\isanewline
\isacommand{lemma}\isamarkupfalse%
\ hash{\isacharunderscore}{\kern0pt}range{\isacharcolon}{\kern0pt}\isanewline
\ \ \isakeyword{assumes}\ {\isachardoublequoteopen}p\ {\isachargreater}{\kern0pt}\ {\isadigit{0}}{\isachardoublequoteclose}\isanewline
\ \ \isakeyword{assumes}\ {\isachardoublequoteopen}{\isasymomega}\ {\isasymin}\ bounded{\isacharunderscore}{\kern0pt}degree{\isacharunderscore}{\kern0pt}polynomials\ {\isacharparenleft}{\kern0pt}ZFact\ {\isacharparenleft}{\kern0pt}int\ p{\isacharparenright}{\kern0pt}{\isacharparenright}{\kern0pt}\ n{\isachardoublequoteclose}\isanewline
\ \ \isakeyword{assumes}\ {\isachardoublequoteopen}x\ {\isacharless}{\kern0pt}\ p{\isachardoublequoteclose}\isanewline
\ \ \isakeyword{shows}\ {\isachardoublequoteopen}hash\ p\ x\ {\isasymomega}\ {\isacharless}{\kern0pt}\ p{\isachardoublequoteclose}\isanewline
%
\isadelimproof
%
\endisadelimproof
%
\isatagproof
\isacommand{proof}\isamarkupfalse%
\ {\isacharminus}{\kern0pt}\isanewline
\ \ \isacommand{have}\isamarkupfalse%
\ {\isachardoublequoteopen}UniversalHashFamily{\isachardot}{\kern0pt}hash\ {\isacharparenleft}{\kern0pt}ZFact\ {\isacharparenleft}{\kern0pt}int\ p{\isacharparenright}{\kern0pt}{\isacharparenright}{\kern0pt}\ {\isacharparenleft}{\kern0pt}zfact{\isacharunderscore}{\kern0pt}embed\ p\ x{\isacharparenright}{\kern0pt}\ {\isasymomega}\ {\isasymin}\ carrier\ {\isacharparenleft}{\kern0pt}ZFact\ {\isacharparenleft}{\kern0pt}int\ p{\isacharparenright}{\kern0pt}{\isacharparenright}{\kern0pt}{\isachardoublequoteclose}\isanewline
\ \ \ \ \isacommand{apply}\isamarkupfalse%
\ {\isacharparenleft}{\kern0pt}rule\ UniversalHashFamily{\isachardot}{\kern0pt}hash{\isacharunderscore}{\kern0pt}range{\isacharbrackleft}{\kern0pt}OF\ {\isacharunderscore}{\kern0pt}\ assms{\isacharparenleft}{\kern0pt}{\isadigit{2}}{\isacharparenright}{\kern0pt}{\isacharbrackright}{\kern0pt}{\isacharparenright}{\kern0pt}\isanewline
\ \ \ \ \ \isacommand{apply}\isamarkupfalse%
\ {\isacharparenleft}{\kern0pt}metis\ ZFact{\isacharunderscore}{\kern0pt}is{\isacharunderscore}{\kern0pt}cring\ cring{\isacharunderscore}{\kern0pt}def{\isacharparenright}{\kern0pt}\isanewline
\ \ \ \ \isacommand{by}\isamarkupfalse%
\ {\isacharparenleft}{\kern0pt}metis\ zfact{\isacharunderscore}{\kern0pt}embed{\isacharunderscore}{\kern0pt}ran{\isacharbrackleft}{\kern0pt}OF\ assms{\isacharparenleft}{\kern0pt}{\isadigit{1}}{\isacharparenright}{\kern0pt}{\isacharbrackright}{\kern0pt}\ assms{\isacharparenleft}{\kern0pt}{\isadigit{3}}{\isacharparenright}{\kern0pt}\ atLeast{\isadigit{0}}LessThan\ image{\isacharunderscore}{\kern0pt}eqI\ lessThan{\isacharunderscore}{\kern0pt}iff{\isacharparenright}{\kern0pt}\isanewline
\ \ \isacommand{thus}\isamarkupfalse%
\ {\isacharquery}{\kern0pt}thesis\isanewline
\ \ \ \ \isacommand{using}\isamarkupfalse%
\ the{\isacharunderscore}{\kern0pt}inv{\isacharunderscore}{\kern0pt}into{\isacharunderscore}{\kern0pt}into{\isacharbrackleft}{\kern0pt}OF\ zfact{\isacharunderscore}{\kern0pt}embed{\isacharunderscore}{\kern0pt}inj{\isacharbrackleft}{\kern0pt}OF\ assms{\isacharparenleft}{\kern0pt}{\isadigit{1}}{\isacharparenright}{\kern0pt}{\isacharbrackright}{\kern0pt}{\isacharcomma}{\kern0pt}\ \isakeyword{where}\ B{\isacharequal}{\kern0pt}{\isachardoublequoteopen}{\isacharbraceleft}{\kern0pt}{\isadigit{0}}{\isachardot}{\kern0pt}{\isachardot}{\kern0pt}{\isacharless}{\kern0pt}p{\isacharbraceright}{\kern0pt}{\isachardoublequoteclose}{\isacharbrackright}{\kern0pt}\isanewline
\ \ \ \ \ \ zfact{\isacharunderscore}{\kern0pt}embed{\isacharunderscore}{\kern0pt}ran{\isacharbrackleft}{\kern0pt}OF\ assms{\isacharparenleft}{\kern0pt}{\isadigit{1}}{\isacharparenright}{\kern0pt}{\isacharbrackright}{\kern0pt}\isanewline
\ \ \ \ \isacommand{by}\isamarkupfalse%
\ {\isacharparenleft}{\kern0pt}simp\ add{\isacharcolon}{\kern0pt}hash{\isachardot}{\kern0pt}simps{\isacharparenright}{\kern0pt}\isanewline
\isacommand{qed}\isamarkupfalse%
%
\endisatagproof
{\isafoldproof}%
%
\isadelimproof
\isanewline
%
\endisadelimproof
\isanewline
\isacommand{lemma}\isamarkupfalse%
\ hash{\isacharunderscore}{\kern0pt}inj{\isacharunderscore}{\kern0pt}if{\isacharunderscore}{\kern0pt}degree{\isacharunderscore}{\kern0pt}{\isadigit{1}}{\isacharcolon}{\kern0pt}\isanewline
\ \ \isakeyword{assumes}\ {\isachardoublequoteopen}prime\ p{\isachardoublequoteclose}\isanewline
\ \ \isakeyword{assumes}\ {\isachardoublequoteopen}{\isasymomega}\ {\isasymin}\ bounded{\isacharunderscore}{\kern0pt}degree{\isacharunderscore}{\kern0pt}polynomials\ {\isacharparenleft}{\kern0pt}ZFact\ {\isacharparenleft}{\kern0pt}int\ p{\isacharparenright}{\kern0pt}{\isacharparenright}{\kern0pt}\ n{\isachardoublequoteclose}\isanewline
\ \ \isakeyword{assumes}\ {\isachardoublequoteopen}degree\ {\isasymomega}\ {\isacharequal}{\kern0pt}\ {\isadigit{1}}{\isachardoublequoteclose}\isanewline
\ \ \isakeyword{shows}\ {\isachardoublequoteopen}inj{\isacharunderscore}{\kern0pt}on\ {\isacharparenleft}{\kern0pt}{\isasymlambda}x{\isachardot}{\kern0pt}\ hash\ p\ x\ {\isasymomega}{\isacharparenright}{\kern0pt}\ {\isacharbraceleft}{\kern0pt}{\isadigit{0}}{\isachardot}{\kern0pt}{\isachardot}{\kern0pt}{\isacharless}{\kern0pt}p{\isacharbraceright}{\kern0pt}{\isachardoublequoteclose}\isanewline
%
\isadelimproof
%
\endisadelimproof
%
\isatagproof
\isacommand{proof}\isamarkupfalse%
\ {\isacharminus}{\kern0pt}\isanewline
\ \ \isacommand{have}\isamarkupfalse%
\ p{\isacharunderscore}{\kern0pt}ge{\isacharunderscore}{\kern0pt}{\isadigit{0}}{\isacharcolon}{\kern0pt}\ {\isachardoublequoteopen}p\ {\isachargreater}{\kern0pt}\ {\isadigit{0}}{\isachardoublequoteclose}\ \isacommand{using}\isamarkupfalse%
\ assms{\isacharparenleft}{\kern0pt}{\isadigit{1}}{\isacharparenright}{\kern0pt}\ \ \isanewline
\ \ \ \ \isacommand{by}\isamarkupfalse%
\ {\isacharparenleft}{\kern0pt}simp\ add{\isacharcolon}{\kern0pt}\ prime{\isacharunderscore}{\kern0pt}gt{\isacharunderscore}{\kern0pt}{\isadigit{0}}{\isacharunderscore}{\kern0pt}nat{\isacharparenright}{\kern0pt}\isanewline
\isanewline
\ \ \isacommand{have}\isamarkupfalse%
\ ring{\isacharunderscore}{\kern0pt}p{\isacharcolon}{\kern0pt}\ {\isachardoublequoteopen}ring\ {\isacharparenleft}{\kern0pt}ZFact\ {\isacharparenleft}{\kern0pt}int\ p{\isacharparenright}{\kern0pt}{\isacharparenright}{\kern0pt}{\isachardoublequoteclose}\isanewline
\ \ \ \ \isacommand{by}\isamarkupfalse%
\ {\isacharparenleft}{\kern0pt}metis\ ZFact{\isacharunderscore}{\kern0pt}is{\isacharunderscore}{\kern0pt}cring\ cring{\isacharunderscore}{\kern0pt}def{\isacharparenright}{\kern0pt}\isanewline
\isanewline
\ \ \isacommand{have}\isamarkupfalse%
\ {\isachardoublequoteopen}inj{\isacharunderscore}{\kern0pt}on\ {\isacharparenleft}{\kern0pt}the{\isacharunderscore}{\kern0pt}inv{\isacharunderscore}{\kern0pt}into\ {\isacharbraceleft}{\kern0pt}{\isadigit{0}}{\isachardot}{\kern0pt}{\isachardot}{\kern0pt}{\isacharless}{\kern0pt}p{\isacharbraceright}{\kern0pt}\ {\isacharparenleft}{\kern0pt}zfact{\isacharunderscore}{\kern0pt}embed\ p{\isacharparenright}{\kern0pt}\ {\isasymcirc}\ {\isacharparenleft}{\kern0pt}{\isasymlambda}x{\isachardot}{\kern0pt}\ \ {\isacharparenleft}{\kern0pt}UniversalHashFamily{\isachardot}{\kern0pt}hash\ {\isacharparenleft}{\kern0pt}ZFact\ {\isacharparenleft}{\kern0pt}int\ p{\isacharparenright}{\kern0pt}{\isacharparenright}{\kern0pt}\ x\ {\isasymomega}{\isacharparenright}{\kern0pt}{\isacharparenright}{\kern0pt}\ {\isasymcirc}\ {\isacharparenleft}{\kern0pt}zfact{\isacharunderscore}{\kern0pt}embed\ p{\isacharparenright}{\kern0pt}{\isacharparenright}{\kern0pt}\ {\isacharbraceleft}{\kern0pt}{\isadigit{0}}{\isachardot}{\kern0pt}{\isachardot}{\kern0pt}{\isacharless}{\kern0pt}p{\isacharbraceright}{\kern0pt}{\isachardoublequoteclose}\isanewline
\ \ \ \ \isacommand{apply}\isamarkupfalse%
\ {\isacharparenleft}{\kern0pt}rule\ comp{\isacharunderscore}{\kern0pt}inj{\isacharunderscore}{\kern0pt}on{\isacharbrackleft}{\kern0pt}OF\ zfact{\isacharunderscore}{\kern0pt}embed{\isacharunderscore}{\kern0pt}inj{\isacharbrackleft}{\kern0pt}OF\ p{\isacharunderscore}{\kern0pt}ge{\isacharunderscore}{\kern0pt}{\isadigit{0}}{\isacharbrackright}{\kern0pt}{\isacharbrackright}{\kern0pt}{\isacharparenright}{\kern0pt}\isanewline
\ \ \ \ \isacommand{apply}\isamarkupfalse%
\ {\isacharparenleft}{\kern0pt}subst\ zfact{\isacharunderscore}{\kern0pt}embed{\isacharunderscore}{\kern0pt}ran{\isacharbrackleft}{\kern0pt}OF\ p{\isacharunderscore}{\kern0pt}ge{\isacharunderscore}{\kern0pt}{\isadigit{0}}{\isacharbrackright}{\kern0pt}{\isacharparenright}{\kern0pt}\isanewline
\ \ \ \ \isacommand{apply}\isamarkupfalse%
\ {\isacharparenleft}{\kern0pt}rule\ comp{\isacharunderscore}{\kern0pt}inj{\isacharunderscore}{\kern0pt}on{\isacharparenright}{\kern0pt}\isanewline
\ \ \ \ \ \isacommand{apply}\isamarkupfalse%
\ {\isacharparenleft}{\kern0pt}rule\ UniversalHashFamily{\isachardot}{\kern0pt}hash{\isacharunderscore}{\kern0pt}inj{\isacharunderscore}{\kern0pt}if{\isacharunderscore}{\kern0pt}degree{\isacharunderscore}{\kern0pt}{\isadigit{1}}{\isacharbrackleft}{\kern0pt}OF\ {\isacharunderscore}{\kern0pt}\ assms{\isacharparenleft}{\kern0pt}{\isadigit{2}}{\isacharparenright}{\kern0pt}\ assms{\isacharparenleft}{\kern0pt}{\isadigit{3}}{\isacharparenright}{\kern0pt}{\isacharbrackright}{\kern0pt}{\isacharparenright}{\kern0pt}\isanewline
\ \ \ \ \ \isacommand{apply}\isamarkupfalse%
\ {\isacharparenleft}{\kern0pt}metis\ zfact{\isacharunderscore}{\kern0pt}prime{\isacharunderscore}{\kern0pt}is{\isacharunderscore}{\kern0pt}field{\isacharbrackleft}{\kern0pt}OF\ assms{\isacharparenleft}{\kern0pt}{\isadigit{1}}{\isacharparenright}{\kern0pt}{\isacharbrackright}{\kern0pt}\ zfact{\isacharunderscore}{\kern0pt}finite{\isacharbrackleft}{\kern0pt}OF\ p{\isacharunderscore}{\kern0pt}ge{\isacharunderscore}{\kern0pt}{\isadigit{0}}{\isacharbrackright}{\kern0pt}{\isacharparenright}{\kern0pt}\isanewline
\ \ \ \ \isacommand{apply}\isamarkupfalse%
\ {\isacharparenleft}{\kern0pt}rule\ inj{\isacharunderscore}{\kern0pt}on{\isacharunderscore}{\kern0pt}subset{\isacharbrackleft}{\kern0pt}OF\ {\isacharunderscore}{\kern0pt}\ UniversalHashFamily{\isachardot}{\kern0pt}hash{\isacharunderscore}{\kern0pt}range{\isacharunderscore}{\kern0pt}{\isadigit{2}}{\isacharbrackleft}{\kern0pt}OF\ ring{\isacharunderscore}{\kern0pt}p\ assms{\isacharparenleft}{\kern0pt}{\isadigit{2}}{\isacharparenright}{\kern0pt}{\isacharbrackright}{\kern0pt}{\isacharbrackright}{\kern0pt}{\isacharparenright}{\kern0pt}\isanewline
\ \ \ \ \isacommand{apply}\isamarkupfalse%
\ {\isacharparenleft}{\kern0pt}subst\ zfact{\isacharunderscore}{\kern0pt}embed{\isacharunderscore}{\kern0pt}ran{\isacharbrackleft}{\kern0pt}OF\ p{\isacharunderscore}{\kern0pt}ge{\isacharunderscore}{\kern0pt}{\isadigit{0}}{\isacharcomma}{\kern0pt}\ symmetric{\isacharbrackright}{\kern0pt}{\isacharparenright}{\kern0pt}\isanewline
\ \ \ \ \isacommand{by}\isamarkupfalse%
\ {\isacharparenleft}{\kern0pt}rule\ inj{\isacharunderscore}{\kern0pt}on{\isacharunderscore}{\kern0pt}the{\isacharunderscore}{\kern0pt}inv{\isacharunderscore}{\kern0pt}into{\isacharbrackleft}{\kern0pt}OF\ zfact{\isacharunderscore}{\kern0pt}embed{\isacharunderscore}{\kern0pt}inj{\isacharbrackleft}{\kern0pt}OF\ p{\isacharunderscore}{\kern0pt}ge{\isacharunderscore}{\kern0pt}{\isadigit{0}}{\isacharbrackright}{\kern0pt}{\isacharbrackright}{\kern0pt}{\isacharparenright}{\kern0pt}\isanewline
\isanewline
\ \ \isacommand{thus}\isamarkupfalse%
\ {\isacharquery}{\kern0pt}thesis\isanewline
\ \ \ \ \isacommand{by}\isamarkupfalse%
\ {\isacharparenleft}{\kern0pt}simp\ add{\isacharcolon}{\kern0pt}hash{\isachardot}{\kern0pt}simps\ comp{\isacharunderscore}{\kern0pt}def{\isacharparenright}{\kern0pt}\isanewline
\isacommand{qed}\isamarkupfalse%
%
\endisatagproof
{\isafoldproof}%
%
\isadelimproof
\isanewline
%
\endisadelimproof
\isanewline
\isacommand{lemma}\isamarkupfalse%
\ hash{\isacharunderscore}{\kern0pt}prob{\isacharcolon}{\kern0pt}\isanewline
\ \ \isakeyword{assumes}\ {\isachardoublequoteopen}prime\ p{\isachardoublequoteclose}\isanewline
\ \ \isakeyword{assumes}\ {\isachardoublequoteopen}K\ {\isasymsubseteq}\ {\isacharbraceleft}{\kern0pt}{\isadigit{0}}{\isachardot}{\kern0pt}{\isachardot}{\kern0pt}{\isacharless}{\kern0pt}p{\isacharbraceright}{\kern0pt}{\isachardoublequoteclose}\isanewline
\ \ \isakeyword{assumes}\ {\isachardoublequoteopen}y\ {\isacharbackquote}{\kern0pt}\ K\ {\isasymsubseteq}\ {\isacharbraceleft}{\kern0pt}{\isadigit{0}}{\isachardot}{\kern0pt}{\isachardot}{\kern0pt}{\isacharless}{\kern0pt}p{\isacharbraceright}{\kern0pt}{\isachardoublequoteclose}\isanewline
\ \ \isakeyword{assumes}\ {\isachardoublequoteopen}card\ K\ {\isasymle}\ n{\isachardoublequoteclose}\isanewline
\ \ \isakeyword{shows}\ {\isachardoublequoteopen}{\isasymP}{\isacharparenleft}{\kern0pt}{\isasymomega}\ in\ measure{\isacharunderscore}{\kern0pt}pmf\ {\isacharparenleft}{\kern0pt}pmf{\isacharunderscore}{\kern0pt}of{\isacharunderscore}{\kern0pt}set\ {\isacharparenleft}{\kern0pt}bounded{\isacharunderscore}{\kern0pt}degree{\isacharunderscore}{\kern0pt}polynomials\ {\isacharparenleft}{\kern0pt}ZFact\ {\isacharparenleft}{\kern0pt}int\ p{\isacharparenright}{\kern0pt}{\isacharparenright}{\kern0pt}\ n{\isacharparenright}{\kern0pt}{\isacharparenright}{\kern0pt}{\isachardot}{\kern0pt}\isanewline
\ \ \ \ {\isacharparenleft}{\kern0pt}{\isasymforall}x\ {\isasymin}\ K{\isachardot}{\kern0pt}\ hash\ p\ x\ {\isasymomega}\ {\isacharequal}{\kern0pt}\ {\isacharparenleft}{\kern0pt}y\ x{\isacharparenright}{\kern0pt}{\isacharparenright}{\kern0pt}{\isacharparenright}{\kern0pt}\ {\isacharequal}{\kern0pt}\ {\isadigit{1}}\ {\isacharslash}{\kern0pt}\ real\ p{\isacharcircum}{\kern0pt}card\ K{\isachardoublequoteclose}\isanewline
%
\isadelimproof
%
\endisadelimproof
%
\isatagproof
\isacommand{proof}\isamarkupfalse%
\ {\isacharminus}{\kern0pt}\isanewline
\ \ \isacommand{define}\isamarkupfalse%
\ y{\isacharprime}{\kern0pt}\ \isakeyword{where}\ {\isachardoublequoteopen}y{\isacharprime}{\kern0pt}\ {\isacharequal}{\kern0pt}\ zfact{\isacharunderscore}{\kern0pt}embed\ p\ {\isasymcirc}\ y\ {\isasymcirc}\ {\isacharparenleft}{\kern0pt}the{\isacharunderscore}{\kern0pt}inv{\isacharunderscore}{\kern0pt}into\ K\ {\isacharparenleft}{\kern0pt}zfact{\isacharunderscore}{\kern0pt}embed\ p{\isacharparenright}{\kern0pt}{\isacharparenright}{\kern0pt}{\isachardoublequoteclose}\isanewline
\ \ \isacommand{define}\isamarkupfalse%
\ {\isasymOmega}\ \isakeyword{where}\ {\isachardoublequoteopen}{\isasymOmega}\ {\isacharequal}{\kern0pt}\ pmf{\isacharunderscore}{\kern0pt}of{\isacharunderscore}{\kern0pt}set\ {\isacharparenleft}{\kern0pt}bounded{\isacharunderscore}{\kern0pt}degree{\isacharunderscore}{\kern0pt}polynomials\ {\isacharparenleft}{\kern0pt}ZFact\ {\isacharparenleft}{\kern0pt}int\ p{\isacharparenright}{\kern0pt}{\isacharparenright}{\kern0pt}\ n{\isacharparenright}{\kern0pt}{\isachardoublequoteclose}\isanewline
\isanewline
\ \ \isacommand{have}\isamarkupfalse%
\ p{\isacharunderscore}{\kern0pt}ge{\isacharunderscore}{\kern0pt}{\isadigit{0}}{\isacharcolon}{\kern0pt}\ {\isachardoublequoteopen}p\ {\isachargreater}{\kern0pt}\ {\isadigit{0}}{\isachardoublequoteclose}\ \isacommand{using}\isamarkupfalse%
\ prime{\isacharunderscore}{\kern0pt}gt{\isacharunderscore}{\kern0pt}{\isadigit{0}}{\isacharunderscore}{\kern0pt}nat{\isacharbrackleft}{\kern0pt}OF\ assms{\isacharparenleft}{\kern0pt}{\isadigit{1}}{\isacharparenright}{\kern0pt}{\isacharbrackright}{\kern0pt}\ \isacommand{by}\isamarkupfalse%
\ simp\isanewline
\isanewline
\ \ \isacommand{have}\isamarkupfalse%
\ {\isachardoublequoteopen}{\isasymAnd}x{\isachardot}{\kern0pt}\ x\ {\isasymin}\ zfact{\isacharunderscore}{\kern0pt}embed\ p\ {\isacharbackquote}{\kern0pt}\ K\ {\isasymLongrightarrow}\ the{\isacharunderscore}{\kern0pt}inv{\isacharunderscore}{\kern0pt}into\ K\ {\isacharparenleft}{\kern0pt}zfact{\isacharunderscore}{\kern0pt}embed\ p{\isacharparenright}{\kern0pt}\ x\ {\isasymin}\ K{\isachardoublequoteclose}\isanewline
\ \ \ \ \isacommand{apply}\isamarkupfalse%
\ {\isacharparenleft}{\kern0pt}rule\ the{\isacharunderscore}{\kern0pt}inv{\isacharunderscore}{\kern0pt}into{\isacharunderscore}{\kern0pt}into{\isacharparenright}{\kern0pt}\isanewline
\ \ \ \ \ \ \isacommand{apply}\isamarkupfalse%
\ {\isacharparenleft}{\kern0pt}metis\ zfact{\isacharunderscore}{\kern0pt}embed{\isacharunderscore}{\kern0pt}inj{\isacharbrackleft}{\kern0pt}OF\ p{\isacharunderscore}{\kern0pt}ge{\isacharunderscore}{\kern0pt}{\isadigit{0}}{\isacharbrackright}{\kern0pt}\ assms{\isacharparenleft}{\kern0pt}{\isadigit{2}}{\isacharparenright}{\kern0pt}\ inj{\isacharunderscore}{\kern0pt}on{\isacharunderscore}{\kern0pt}subset{\isacharparenright}{\kern0pt}\isanewline
\ \ \ \ \isacommand{by}\isamarkupfalse%
\ auto\isanewline
\isanewline
\ \ \isacommand{hence}\isamarkupfalse%
\ ran{\isacharunderscore}{\kern0pt}y{\isacharcolon}{\kern0pt}\ {\isachardoublequoteopen}{\isasymAnd}x{\isachardot}{\kern0pt}\ x\ {\isasymin}\ zfact{\isacharunderscore}{\kern0pt}embed\ p\ {\isacharbackquote}{\kern0pt}\ K\ {\isasymLongrightarrow}\ y\ {\isacharparenleft}{\kern0pt}the{\isacharunderscore}{\kern0pt}inv{\isacharunderscore}{\kern0pt}into\ K\ {\isacharparenleft}{\kern0pt}zfact{\isacharunderscore}{\kern0pt}embed\ p{\isacharparenright}{\kern0pt}\ x{\isacharparenright}{\kern0pt}\ {\isasymin}\ {\isacharbraceleft}{\kern0pt}{\isadigit{0}}{\isachardot}{\kern0pt}{\isachardot}{\kern0pt}{\isacharless}{\kern0pt}p{\isacharbraceright}{\kern0pt}{\isachardoublequoteclose}\isanewline
\ \ \ \ \isacommand{using}\isamarkupfalse%
\ assms{\isacharparenleft}{\kern0pt}{\isadigit{3}}{\isacharparenright}{\kern0pt}\ \isacommand{by}\isamarkupfalse%
\ blast\isanewline
\isanewline
\ \ \isacommand{have}\isamarkupfalse%
\ ran{\isacharunderscore}{\kern0pt}y{\isacharprime}{\kern0pt}{\isacharcolon}{\kern0pt}\ {\isachardoublequoteopen}y{\isacharprime}{\kern0pt}\ {\isacharbackquote}{\kern0pt}\ {\isacharparenleft}{\kern0pt}zfact{\isacharunderscore}{\kern0pt}embed\ p\ {\isacharbackquote}{\kern0pt}\ K{\isacharparenright}{\kern0pt}\ {\isasymsubseteq}\ carrier\ {\isacharparenleft}{\kern0pt}ZFact\ {\isacharparenleft}{\kern0pt}int\ p{\isacharparenright}{\kern0pt}{\isacharparenright}{\kern0pt}{\isachardoublequoteclose}\isanewline
\ \ \ \ \isacommand{apply}\isamarkupfalse%
\ {\isacharparenleft}{\kern0pt}rule\ image{\isacharunderscore}{\kern0pt}subsetI{\isacharparenright}{\kern0pt}\isanewline
\ \ \ \ \isacommand{apply}\isamarkupfalse%
\ {\isacharparenleft}{\kern0pt}simp\ add{\isacharcolon}{\kern0pt}y{\isacharprime}{\kern0pt}{\isacharunderscore}{\kern0pt}def{\isacharparenright}{\kern0pt}\isanewline
\ \ \ \ \isacommand{by}\isamarkupfalse%
\ {\isacharparenleft}{\kern0pt}metis\ zfact{\isacharunderscore}{\kern0pt}embed{\isacharunderscore}{\kern0pt}ran{\isacharbrackleft}{\kern0pt}OF\ p{\isacharunderscore}{\kern0pt}ge{\isacharunderscore}{\kern0pt}{\isadigit{0}}{\isacharbrackright}{\kern0pt}\ imageI\ ran{\isacharunderscore}{\kern0pt}y{\isacharparenright}{\kern0pt}\isanewline
\isanewline
\ \ \isacommand{have}\isamarkupfalse%
\ K{\isacharunderscore}{\kern0pt}embed{\isacharcolon}{\kern0pt}\ {\isachardoublequoteopen}zfact{\isacharunderscore}{\kern0pt}embed\ p\ {\isacharbackquote}{\kern0pt}\ K\ {\isasymsubseteq}\ carrier\ {\isacharparenleft}{\kern0pt}ZFact\ {\isacharparenleft}{\kern0pt}int\ p{\isacharparenright}{\kern0pt}{\isacharparenright}{\kern0pt}{\isachardoublequoteclose}\isanewline
\ \ \ \ \isacommand{using}\isamarkupfalse%
\ zfact{\isacharunderscore}{\kern0pt}embed{\isacharunderscore}{\kern0pt}ran{\isacharbrackleft}{\kern0pt}OF\ p{\isacharunderscore}{\kern0pt}ge{\isacharunderscore}{\kern0pt}{\isadigit{0}}{\isacharbrackright}{\kern0pt}\ assms{\isacharparenleft}{\kern0pt}{\isadigit{2}}{\isacharparenright}{\kern0pt}\ \isacommand{by}\isamarkupfalse%
\ auto\isanewline
\isanewline
\ \ \isacommand{have}\isamarkupfalse%
\ ring{\isacharunderscore}{\kern0pt}zfact{\isacharcolon}{\kern0pt}\ {\isachardoublequoteopen}ring\ {\isacharparenleft}{\kern0pt}ZFact\ {\isacharparenleft}{\kern0pt}int\ p{\isacharparenright}{\kern0pt}{\isacharparenright}{\kern0pt}{\isachardoublequoteclose}\ \isanewline
\ \ \ \ \isacommand{using}\isamarkupfalse%
\ ZFact{\isacharunderscore}{\kern0pt}is{\isacharunderscore}{\kern0pt}cring\ cring{\isacharunderscore}{\kern0pt}def\ \isacommand{by}\isamarkupfalse%
\ blast\isanewline
\isanewline
\ \ \isacommand{have}\isamarkupfalse%
\ {\isachardoublequoteopen}{\isasymP}{\isacharparenleft}{\kern0pt}{\isasymomega}\ in\ measure{\isacharunderscore}{\kern0pt}pmf\ {\isacharparenleft}{\kern0pt}pmf{\isacharunderscore}{\kern0pt}of{\isacharunderscore}{\kern0pt}set\ {\isacharparenleft}{\kern0pt}bounded{\isacharunderscore}{\kern0pt}degree{\isacharunderscore}{\kern0pt}polynomials\ {\isacharparenleft}{\kern0pt}ZFact\ {\isacharparenleft}{\kern0pt}int\ p{\isacharparenright}{\kern0pt}{\isacharparenright}{\kern0pt}\ n{\isacharparenright}{\kern0pt}{\isacharparenright}{\kern0pt}{\isachardot}{\kern0pt}\ \isanewline
\ \ \ \ {\isacharparenleft}{\kern0pt}{\isasymforall}x\ {\isasymin}\ K{\isachardot}{\kern0pt}\ hash\ p\ x\ {\isasymomega}\ {\isacharequal}{\kern0pt}\ {\isacharparenleft}{\kern0pt}y\ x{\isacharparenright}{\kern0pt}{\isacharparenright}{\kern0pt}{\isacharparenright}{\kern0pt}\ {\isacharequal}{\kern0pt}\ {\isasymP}{\isacharparenleft}{\kern0pt}{\isasymomega}\ in\ measure{\isacharunderscore}{\kern0pt}pmf\ {\isasymOmega}{\isachardot}{\kern0pt}\ {\isacharparenleft}{\kern0pt}{\isasymforall}x\ {\isasymin}\ K{\isachardot}{\kern0pt}\ hash\ p\ x\ {\isasymomega}\ {\isacharequal}{\kern0pt}\ {\isacharparenleft}{\kern0pt}y\ x{\isacharparenright}{\kern0pt}{\isacharparenright}{\kern0pt}{\isacharparenright}{\kern0pt}{\isachardoublequoteclose}\isanewline
\ \ \ \ \isacommand{by}\isamarkupfalse%
\ {\isacharparenleft}{\kern0pt}simp\ add{\isacharcolon}{\kern0pt}\ {\isasymOmega}{\isacharunderscore}{\kern0pt}def{\isacharparenright}{\kern0pt}\isanewline
\ \ \isacommand{also}\isamarkupfalse%
\ \isacommand{have}\isamarkupfalse%
\ {\isachardoublequoteopen}{\isachardot}{\kern0pt}{\isachardot}{\kern0pt}{\isachardot}{\kern0pt}\ {\isacharequal}{\kern0pt}\isanewline
\ \ \ \ {\isasymP}{\isacharparenleft}{\kern0pt}{\isasymomega}\ in\ measure{\isacharunderscore}{\kern0pt}pmf\ {\isasymOmega}{\isachardot}{\kern0pt}\ {\isacharparenleft}{\kern0pt}{\isasymforall}x\ {\isasymin}\ zfact{\isacharunderscore}{\kern0pt}embed\ p\ {\isacharbackquote}{\kern0pt}\ K{\isachardot}{\kern0pt}\ UniversalHashFamily{\isachardot}{\kern0pt}hash\ {\isacharparenleft}{\kern0pt}ZFact\ {\isacharparenleft}{\kern0pt}int\ p{\isacharparenright}{\kern0pt}{\isacharparenright}{\kern0pt}\ x\ {\isasymomega}\ {\isacharequal}{\kern0pt}\ y{\isacharprime}{\kern0pt}\ x{\isacharparenright}{\kern0pt}{\isacharparenright}{\kern0pt}{\isachardoublequoteclose}\isanewline
\ \ \ \ \isacommand{apply}\isamarkupfalse%
\ {\isacharparenleft}{\kern0pt}rule\ pmf{\isacharunderscore}{\kern0pt}eq{\isacharparenright}{\kern0pt}\isanewline
\ \ \ \ \isacommand{apply}\isamarkupfalse%
\ {\isacharparenleft}{\kern0pt}simp\ add{\isacharcolon}{\kern0pt}y{\isacharprime}{\kern0pt}{\isacharunderscore}{\kern0pt}def\ hash{\isachardot}{\kern0pt}simps\ {\isasymOmega}{\isacharunderscore}{\kern0pt}def{\isacharparenright}{\kern0pt}\isanewline
\ \ \ \ \isacommand{apply}\isamarkupfalse%
\ {\isacharparenleft}{\kern0pt}subst\ {\isacharparenleft}{\kern0pt}asm{\isacharparenright}{\kern0pt}\ set{\isacharunderscore}{\kern0pt}pmf{\isacharunderscore}{\kern0pt}of{\isacharunderscore}{\kern0pt}set{\isacharcomma}{\kern0pt}\ metis\ ne{\isacharunderscore}{\kern0pt}bounded{\isacharunderscore}{\kern0pt}degree{\isacharunderscore}{\kern0pt}polynomials{\isacharcomma}{\kern0pt}\ \isanewline
\ \ \ \ \ \ \ \ \ \ \ \ metis\ fin{\isacharunderscore}{\kern0pt}bounded{\isacharunderscore}{\kern0pt}degree{\isacharunderscore}{\kern0pt}polynomials{\isacharbrackleft}{\kern0pt}OF\ p{\isacharunderscore}{\kern0pt}ge{\isacharunderscore}{\kern0pt}{\isadigit{0}}{\isacharbrackright}{\kern0pt}{\isacharparenright}{\kern0pt}\isanewline
\ \ \ \ \isacommand{apply}\isamarkupfalse%
\ {\isacharparenleft}{\kern0pt}rule\ ball{\isacharunderscore}{\kern0pt}cong{\isacharcomma}{\kern0pt}\ simp{\isacharparenright}{\kern0pt}\isanewline
\ \ \ \ \isacommand{apply}\isamarkupfalse%
\ {\isacharparenleft}{\kern0pt}subst\ the{\isacharunderscore}{\kern0pt}inv{\isacharunderscore}{\kern0pt}into{\isacharunderscore}{\kern0pt}f{\isacharunderscore}{\kern0pt}f{\isacharparenright}{\kern0pt}\isanewline
\ \ \ \ \ \ \isacommand{apply}\isamarkupfalse%
\ {\isacharparenleft}{\kern0pt}metis\ zfact{\isacharunderscore}{\kern0pt}embed{\isacharunderscore}{\kern0pt}inj{\isacharbrackleft}{\kern0pt}OF\ p{\isacharunderscore}{\kern0pt}ge{\isacharunderscore}{\kern0pt}{\isadigit{0}}{\isacharbrackright}{\kern0pt}\ assms{\isacharparenleft}{\kern0pt}{\isadigit{2}}{\isacharparenright}{\kern0pt}\ inj{\isacharunderscore}{\kern0pt}on{\isacharunderscore}{\kern0pt}subset{\isacharparenright}{\kern0pt}\isanewline
\ \ \ \ \ \isacommand{apply}\isamarkupfalse%
\ {\isacharparenleft}{\kern0pt}simp{\isacharparenright}{\kern0pt}\isanewline
\ \ \ \ \isacommand{apply}\isamarkupfalse%
\ {\isacharparenleft}{\kern0pt}subst\ eq{\isacharunderscore}{\kern0pt}commute{\isacharparenright}{\kern0pt}\isanewline
\ \ \ \ \isacommand{apply}\isamarkupfalse%
\ {\isacharparenleft}{\kern0pt}rule\ order{\isacharunderscore}{\kern0pt}antisym{\isacharparenright}{\kern0pt}\isanewline
\ \ \ \ \ \isacommand{apply}\isamarkupfalse%
\ {\isacharparenleft}{\kern0pt}simp{\isacharcomma}{\kern0pt}\ rule\ impI{\isacharparenright}{\kern0pt}\isanewline
\ \ \ \ \ \isacommand{apply}\isamarkupfalse%
\ {\isacharparenleft}{\kern0pt}subst\ f{\isacharunderscore}{\kern0pt}the{\isacharunderscore}{\kern0pt}inv{\isacharunderscore}{\kern0pt}into{\isacharunderscore}{\kern0pt}f{\isacharbrackleft}{\kern0pt}OF\ zfact{\isacharunderscore}{\kern0pt}embed{\isacharunderscore}{\kern0pt}inj{\isacharbrackleft}{\kern0pt}OF\ p{\isacharunderscore}{\kern0pt}ge{\isacharunderscore}{\kern0pt}{\isadigit{0}}{\isacharbrackright}{\kern0pt}{\isacharbrackright}{\kern0pt}{\isacharparenright}{\kern0pt}\isanewline
\ \ \ \ \ \ \isacommand{apply}\isamarkupfalse%
\ {\isacharparenleft}{\kern0pt}subst\ zfact{\isacharunderscore}{\kern0pt}embed{\isacharunderscore}{\kern0pt}ran{\isacharbrackleft}{\kern0pt}OF\ p{\isacharunderscore}{\kern0pt}ge{\isacharunderscore}{\kern0pt}{\isadigit{0}}{\isacharbrackright}{\kern0pt}{\isacharparenright}{\kern0pt}\isanewline
\ \ \ \ \ \ \isacommand{apply}\isamarkupfalse%
\ {\isacharparenleft}{\kern0pt}rule\ UniversalHashFamily{\isachardot}{\kern0pt}hash{\isacharunderscore}{\kern0pt}range{\isacharbrackleft}{\kern0pt}OF\ ring{\isacharunderscore}{\kern0pt}zfact{\isacharcomma}{\kern0pt}\ \isakeyword{where}\ n{\isacharequal}{\kern0pt}{\isachardoublequoteopen}n{\isachardoublequoteclose}{\isacharbrackright}{\kern0pt}{\isacharcomma}{\kern0pt}\ simp{\isacharparenright}{\kern0pt}\isanewline
\ \ \ \ \ \ \isacommand{apply}\isamarkupfalse%
\ {\isacharparenleft}{\kern0pt}meson\ K{\isacharunderscore}{\kern0pt}embed\ image{\isacharunderscore}{\kern0pt}subset{\isacharunderscore}{\kern0pt}iff{\isacharparenright}{\kern0pt}\isanewline
\ \ \ \ \ \isacommand{apply}\isamarkupfalse%
\ simp\isanewline
\ \ \ \ \isacommand{apply}\isamarkupfalse%
\ {\isacharparenleft}{\kern0pt}simp{\isacharcomma}{\kern0pt}\ rule\ impI{\isacharparenright}{\kern0pt}\isanewline
\ \ \ \ \isacommand{apply}\isamarkupfalse%
\ {\isacharparenleft}{\kern0pt}subst\ the{\isacharunderscore}{\kern0pt}inv{\isacharunderscore}{\kern0pt}into{\isacharunderscore}{\kern0pt}f{\isacharunderscore}{\kern0pt}f{\isacharbrackleft}{\kern0pt}OF\ zfact{\isacharunderscore}{\kern0pt}embed{\isacharunderscore}{\kern0pt}inj{\isacharbrackleft}{\kern0pt}OF\ p{\isacharunderscore}{\kern0pt}ge{\isacharunderscore}{\kern0pt}{\isadigit{0}}{\isacharbrackright}{\kern0pt}{\isacharbrackright}{\kern0pt}{\isacharparenright}{\kern0pt}\isanewline
\ \ \ \ \ \isacommand{apply}\isamarkupfalse%
\ {\isacharparenleft}{\kern0pt}metis\ assms{\isacharparenleft}{\kern0pt}{\isadigit{3}}{\isacharparenright}{\kern0pt}\ image{\isacharunderscore}{\kern0pt}subset{\isacharunderscore}{\kern0pt}iff{\isacharparenright}{\kern0pt}\ \isanewline
\ \ \ \ \isacommand{by}\isamarkupfalse%
\ simp\isanewline
\ \ \isacommand{also}\isamarkupfalse%
\ \isacommand{have}\isamarkupfalse%
\ {\isachardoublequoteopen}{\isachardot}{\kern0pt}{\isachardot}{\kern0pt}{\isachardot}{\kern0pt}\ {\isacharequal}{\kern0pt}\isanewline
\ \ \ \ {\isadigit{1}}\ {\isacharslash}{\kern0pt}\ real\ {\isacharparenleft}{\kern0pt}card\ {\isacharparenleft}{\kern0pt}carrier\ {\isacharparenleft}{\kern0pt}ZFact\ {\isacharparenleft}{\kern0pt}int\ p{\isacharparenright}{\kern0pt}{\isacharparenright}{\kern0pt}{\isacharparenright}{\kern0pt}{\isacharparenright}{\kern0pt}{\isacharcircum}{\kern0pt}{\isacharparenleft}{\kern0pt}card\ {\isacharparenleft}{\kern0pt}zfact{\isacharunderscore}{\kern0pt}embed\ p\ {\isacharbackquote}{\kern0pt}\ K{\isacharparenright}{\kern0pt}{\isacharparenright}{\kern0pt}{\isachardoublequoteclose}\isanewline
\ \ \ \ \isacommand{apply}\isamarkupfalse%
\ {\isacharparenleft}{\kern0pt}simp\ only{\isacharcolon}{\kern0pt}\ {\isasymOmega}{\isacharunderscore}{\kern0pt}def{\isacharparenright}{\kern0pt}\isanewline
\ \ \ \ \isacommand{apply}\isamarkupfalse%
\ {\isacharparenleft}{\kern0pt}rule\ UniversalHashFamily{\isachardot}{\kern0pt}hash{\isacharunderscore}{\kern0pt}prob{\isacharbrackleft}{\kern0pt}\isakeyword{where}\ K{\isacharequal}{\kern0pt}{\isachardoublequoteopen}zfact{\isacharunderscore}{\kern0pt}embed\ p\ {\isacharbackquote}{\kern0pt}\ K{\isachardoublequoteclose}\ \isakeyword{and}\ F{\isacharequal}{\kern0pt}{\isachardoublequoteopen}ZFact\ {\isacharparenleft}{\kern0pt}int\ p{\isacharparenright}{\kern0pt}{\isachardoublequoteclose}\ \isakeyword{and}\ n{\isacharequal}{\kern0pt}{\isachardoublequoteopen}n{\isachardoublequoteclose}\ \isakeyword{and}\ y{\isacharequal}{\kern0pt}{\isachardoublequoteopen}y{\isacharprime}{\kern0pt}{\isachardoublequoteclose}{\isacharbrackright}{\kern0pt}{\isacharparenright}{\kern0pt}\isanewline
\ \ \ \ \ \ \ \isacommand{apply}\isamarkupfalse%
\ {\isacharparenleft}{\kern0pt}metis\ zfact{\isacharunderscore}{\kern0pt}prime{\isacharunderscore}{\kern0pt}is{\isacharunderscore}{\kern0pt}field{\isacharbrackleft}{\kern0pt}OF\ assms{\isacharparenleft}{\kern0pt}{\isadigit{1}}{\isacharparenright}{\kern0pt}{\isacharbrackright}{\kern0pt}\ zfact{\isacharunderscore}{\kern0pt}finite{\isacharbrackleft}{\kern0pt}OF\ p{\isacharunderscore}{\kern0pt}ge{\isacharunderscore}{\kern0pt}{\isadigit{0}}{\isacharbrackright}{\kern0pt}{\isacharparenright}{\kern0pt}\isanewline
\ \ \ \ \ \ \isacommand{apply}\isamarkupfalse%
\ {\isacharparenleft}{\kern0pt}metis\ zfact{\isacharunderscore}{\kern0pt}embed{\isacharunderscore}{\kern0pt}ran{\isacharbrackleft}{\kern0pt}OF\ p{\isacharunderscore}{\kern0pt}ge{\isacharunderscore}{\kern0pt}{\isadigit{0}}{\isacharbrackright}{\kern0pt}\ assms{\isacharparenleft}{\kern0pt}{\isadigit{2}}{\isacharparenright}{\kern0pt}\ image{\isacharunderscore}{\kern0pt}mono{\isacharparenright}{\kern0pt}\isanewline
\ \ \ \ \ \isacommand{apply}\isamarkupfalse%
\ {\isacharparenleft}{\kern0pt}rule\ order{\isacharunderscore}{\kern0pt}trans{\isacharbrackleft}{\kern0pt}OF\ card{\isacharunderscore}{\kern0pt}image{\isacharunderscore}{\kern0pt}le{\isacharbrackright}{\kern0pt}{\isacharcomma}{\kern0pt}\ rule\ finite{\isacharunderscore}{\kern0pt}subset{\isacharbrackleft}{\kern0pt}OF\ assms{\isacharparenleft}{\kern0pt}{\isadigit{2}}{\isacharparenright}{\kern0pt}{\isacharbrackright}{\kern0pt}{\isacharcomma}{\kern0pt}\ simp{\isacharcomma}{\kern0pt}\ metis\ assms{\isacharparenleft}{\kern0pt}{\isadigit{4}}{\isacharparenright}{\kern0pt}{\isacharparenright}{\kern0pt}\isanewline
\ \ \ \ \isacommand{using}\isamarkupfalse%
\ K{\isacharunderscore}{\kern0pt}embed\ ran{\isacharunderscore}{\kern0pt}y{\isacharprime}{\kern0pt}\ \isacommand{by}\isamarkupfalse%
\ blast\isanewline
\ \ \isacommand{also}\isamarkupfalse%
\ \isacommand{have}\isamarkupfalse%
\ {\isachardoublequoteopen}{\isachardot}{\kern0pt}{\isachardot}{\kern0pt}{\isachardot}{\kern0pt}\ {\isacharequal}{\kern0pt}\ {\isadigit{1}}{\isacharslash}{\kern0pt}real\ p{\isacharcircum}{\kern0pt}{\isacharparenleft}{\kern0pt}card\ K{\isacharparenright}{\kern0pt}{\isachardoublequoteclose}\ \isanewline
\ \ \ \ \isacommand{apply}\isamarkupfalse%
\ {\isacharparenleft}{\kern0pt}subst\ card{\isacharunderscore}{\kern0pt}image{\isacharcomma}{\kern0pt}\ meson\ inj{\isacharunderscore}{\kern0pt}on{\isacharunderscore}{\kern0pt}subset\ zfact{\isacharunderscore}{\kern0pt}embed{\isacharunderscore}{\kern0pt}inj{\isacharbrackleft}{\kern0pt}OF\ p{\isacharunderscore}{\kern0pt}ge{\isacharunderscore}{\kern0pt}{\isadigit{0}}{\isacharbrackright}{\kern0pt}\ assms{\isacharparenleft}{\kern0pt}{\isadigit{2}}{\isacharparenright}{\kern0pt}{\isacharparenright}{\kern0pt}\isanewline
\ \ \ \ \isacommand{apply}\isamarkupfalse%
\ {\isacharparenleft}{\kern0pt}subst\ zfact{\isacharunderscore}{\kern0pt}card{\isacharbrackleft}{\kern0pt}OF\ p{\isacharunderscore}{\kern0pt}ge{\isacharunderscore}{\kern0pt}{\isadigit{0}}{\isacharbrackright}{\kern0pt}{\isacharparenright}{\kern0pt}\isanewline
\ \ \ \ \isacommand{by}\isamarkupfalse%
\ simp\isanewline
\ \ \isacommand{finally}\isamarkupfalse%
\ \isacommand{show}\isamarkupfalse%
\ {\isacharquery}{\kern0pt}thesis\ \isacommand{by}\isamarkupfalse%
\ simp\isanewline
\isacommand{qed}\isamarkupfalse%
%
\endisatagproof
{\isafoldproof}%
%
\isadelimproof
\isanewline
%
\endisadelimproof
\isanewline
\isacommand{lemma}\isamarkupfalse%
\ hash{\isacharunderscore}{\kern0pt}prob{\isacharunderscore}{\kern0pt}{\isadigit{2}}{\isacharcolon}{\kern0pt}\isanewline
\ \ \isakeyword{assumes}\ {\isachardoublequoteopen}prime\ p{\isachardoublequoteclose}\isanewline
\ \ \isakeyword{assumes}\ {\isachardoublequoteopen}inj{\isacharunderscore}{\kern0pt}on\ x\ K{\isachardoublequoteclose}\isanewline
\ \ \isakeyword{assumes}\ {\isachardoublequoteopen}x\ {\isacharbackquote}{\kern0pt}\ K\ {\isasymsubseteq}\ {\isacharbraceleft}{\kern0pt}{\isadigit{0}}{\isachardot}{\kern0pt}{\isachardot}{\kern0pt}{\isacharless}{\kern0pt}p{\isacharbraceright}{\kern0pt}{\isachardoublequoteclose}\isanewline
\ \ \isakeyword{assumes}\ {\isachardoublequoteopen}y\ {\isacharbackquote}{\kern0pt}\ K\ {\isasymsubseteq}\ {\isacharbraceleft}{\kern0pt}{\isadigit{0}}{\isachardot}{\kern0pt}{\isachardot}{\kern0pt}{\isacharless}{\kern0pt}p{\isacharbraceright}{\kern0pt}{\isachardoublequoteclose}\isanewline
\ \ \isakeyword{assumes}\ {\isachardoublequoteopen}card\ K\ {\isasymle}\ n{\isachardoublequoteclose}\isanewline
\ \ \isakeyword{shows}\ {\isachardoublequoteopen}{\isasymP}{\isacharparenleft}{\kern0pt}{\isasymomega}\ in\ measure{\isacharunderscore}{\kern0pt}pmf\ {\isacharparenleft}{\kern0pt}pmf{\isacharunderscore}{\kern0pt}of{\isacharunderscore}{\kern0pt}set\ {\isacharparenleft}{\kern0pt}bounded{\isacharunderscore}{\kern0pt}degree{\isacharunderscore}{\kern0pt}polynomials\ {\isacharparenleft}{\kern0pt}ZFact\ {\isacharparenleft}{\kern0pt}int\ p{\isacharparenright}{\kern0pt}{\isacharparenright}{\kern0pt}\ n{\isacharparenright}{\kern0pt}{\isacharparenright}{\kern0pt}{\isachardot}{\kern0pt}\isanewline
\ \ \ \ {\isacharparenleft}{\kern0pt}{\isasymforall}k\ {\isasymin}\ K{\isachardot}{\kern0pt}\ hash\ p\ {\isacharparenleft}{\kern0pt}x\ k{\isacharparenright}{\kern0pt}\ {\isasymomega}\ {\isacharequal}{\kern0pt}\ {\isacharparenleft}{\kern0pt}y\ k{\isacharparenright}{\kern0pt}{\isacharparenright}{\kern0pt}{\isacharparenright}{\kern0pt}\ {\isacharequal}{\kern0pt}\ {\isadigit{1}}\ {\isacharslash}{\kern0pt}\ real\ p{\isacharcircum}{\kern0pt}card\ K{\isachardoublequoteclose}\ {\isacharparenleft}{\kern0pt}\isakeyword{is}\ {\isachardoublequoteopen}{\isacharquery}{\kern0pt}lhs\ {\isacharequal}{\kern0pt}\ {\isacharquery}{\kern0pt}rhs{\isachardoublequoteclose}{\isacharparenright}{\kern0pt}\isanewline
%
\isadelimproof
%
\endisadelimproof
%
\isatagproof
\isacommand{proof}\isamarkupfalse%
\ {\isacharminus}{\kern0pt}\isanewline
\ \ \isacommand{define}\isamarkupfalse%
\ y{\isacharprime}{\kern0pt}\ \isakeyword{where}\ {\isachardoublequoteopen}y{\isacharprime}{\kern0pt}\ {\isacharequal}{\kern0pt}\ y\ {\isasymcirc}\ {\isacharparenleft}{\kern0pt}the{\isacharunderscore}{\kern0pt}inv{\isacharunderscore}{\kern0pt}into\ K\ x{\isacharparenright}{\kern0pt}{\isachardoublequoteclose}\isanewline
\ \ \isacommand{have}\isamarkupfalse%
\ {\isachardoublequoteopen}{\isacharquery}{\kern0pt}lhs\ {\isacharequal}{\kern0pt}\ {\isasymP}{\isacharparenleft}{\kern0pt}{\isasymomega}\ in\ measure{\isacharunderscore}{\kern0pt}pmf\ {\isacharparenleft}{\kern0pt}pmf{\isacharunderscore}{\kern0pt}of{\isacharunderscore}{\kern0pt}set\ {\isacharparenleft}{\kern0pt}bounded{\isacharunderscore}{\kern0pt}degree{\isacharunderscore}{\kern0pt}polynomials\ {\isacharparenleft}{\kern0pt}ZFact\ {\isacharparenleft}{\kern0pt}int\ p{\isacharparenright}{\kern0pt}{\isacharparenright}{\kern0pt}\ n{\isacharparenright}{\kern0pt}{\isacharparenright}{\kern0pt}{\isachardot}{\kern0pt}\isanewline
\ \ \ \ {\isacharparenleft}{\kern0pt}{\isasymforall}k\ {\isasymin}\ x\ {\isacharbackquote}{\kern0pt}\ K{\isachardot}{\kern0pt}\ hash\ p\ k\ {\isasymomega}\ {\isacharequal}{\kern0pt}\ y{\isacharprime}{\kern0pt}\ k{\isacharparenright}{\kern0pt}{\isacharparenright}{\kern0pt}{\isachardoublequoteclose}\isanewline
\ \ \ \ \isacommand{apply}\isamarkupfalse%
\ {\isacharparenleft}{\kern0pt}rule\ pmf{\isacharunderscore}{\kern0pt}eq{\isacharparenright}{\kern0pt}\isanewline
\ \ \ \ \isacommand{apply}\isamarkupfalse%
\ {\isacharparenleft}{\kern0pt}simp\ add{\isacharcolon}{\kern0pt}y{\isacharprime}{\kern0pt}{\isacharunderscore}{\kern0pt}def{\isacharparenright}{\kern0pt}\isanewline
\ \ \ \ \isacommand{apply}\isamarkupfalse%
\ {\isacharparenleft}{\kern0pt}rule\ ball{\isacharunderscore}{\kern0pt}cong{\isacharcomma}{\kern0pt}\ simp{\isacharparenright}{\kern0pt}\isanewline
\ \ \ \ \isacommand{by}\isamarkupfalse%
\ {\isacharparenleft}{\kern0pt}subst\ the{\isacharunderscore}{\kern0pt}inv{\isacharunderscore}{\kern0pt}into{\isacharunderscore}{\kern0pt}f{\isacharunderscore}{\kern0pt}f{\isacharbrackleft}{\kern0pt}OF\ assms{\isacharparenleft}{\kern0pt}{\isadigit{2}}{\isacharparenright}{\kern0pt}{\isacharbrackright}{\kern0pt}{\isacharcomma}{\kern0pt}\ simp{\isacharcomma}{\kern0pt}\ simp{\isacharparenright}{\kern0pt}\isanewline
\ \ \isacommand{also}\isamarkupfalse%
\ \isacommand{have}\isamarkupfalse%
\ {\isachardoublequoteopen}{\isachardot}{\kern0pt}{\isachardot}{\kern0pt}{\isachardot}{\kern0pt}\ {\isacharequal}{\kern0pt}\ {\isadigit{1}}\ {\isacharslash}{\kern0pt}\ real\ p\ {\isacharcircum}{\kern0pt}\ card\ {\isacharparenleft}{\kern0pt}x\ {\isacharbackquote}{\kern0pt}\ K{\isacharparenright}{\kern0pt}{\isachardoublequoteclose}\ \isanewline
\ \ \ \ \isacommand{apply}\isamarkupfalse%
\ {\isacharparenleft}{\kern0pt}rule\ hash{\isacharunderscore}{\kern0pt}prob{\isacharbrackleft}{\kern0pt}OF\ assms{\isacharparenleft}{\kern0pt}{\isadigit{1}}{\isacharparenright}{\kern0pt}\ assms{\isacharparenleft}{\kern0pt}{\isadigit{3}}{\isacharparenright}{\kern0pt}{\isacharbrackright}{\kern0pt}{\isacharparenright}{\kern0pt}\ \isanewline
\ \ \ \ \ \isacommand{using}\isamarkupfalse%
\ assms\ \isacommand{apply}\isamarkupfalse%
\ {\isacharparenleft}{\kern0pt}simp\ add{\isacharcolon}{\kern0pt}\ y{\isacharprime}{\kern0pt}{\isacharunderscore}{\kern0pt}def\ subset{\isacharunderscore}{\kern0pt}eq\ the{\isacharunderscore}{\kern0pt}inv{\isacharunderscore}{\kern0pt}into{\isacharunderscore}{\kern0pt}f{\isacharunderscore}{\kern0pt}f{\isacharparenright}{\kern0pt}\isanewline
\ \ \ \ \ \isacommand{by}\isamarkupfalse%
\ {\isacharparenleft}{\kern0pt}metis\ card{\isacharunderscore}{\kern0pt}image\ assms{\isacharparenleft}{\kern0pt}{\isadigit{2}}{\isacharparenright}{\kern0pt}\ assms{\isacharparenleft}{\kern0pt}{\isadigit{5}}{\isacharparenright}{\kern0pt}{\isacharparenright}{\kern0pt}\isanewline
\ \ \isacommand{also}\isamarkupfalse%
\ \isacommand{have}\isamarkupfalse%
\ {\isachardoublequoteopen}{\isachardot}{\kern0pt}{\isachardot}{\kern0pt}{\isachardot}{\kern0pt}\ {\isacharequal}{\kern0pt}\ {\isacharquery}{\kern0pt}rhs{\isachardoublequoteclose}\isanewline
\ \ \ \ \ \isacommand{by}\isamarkupfalse%
\ {\isacharparenleft}{\kern0pt}subst\ card{\isacharunderscore}{\kern0pt}image{\isacharbrackleft}{\kern0pt}OF\ assms{\isacharparenleft}{\kern0pt}{\isadigit{2}}{\isacharparenright}{\kern0pt}{\isacharbrackright}{\kern0pt}{\isacharcomma}{\kern0pt}\ simp{\isacharparenright}{\kern0pt}\isanewline
\ \ \ \isacommand{finally}\isamarkupfalse%
\ \isacommand{show}\isamarkupfalse%
\ {\isacharquery}{\kern0pt}thesis\ \isacommand{by}\isamarkupfalse%
\ simp\isanewline
\isacommand{qed}\isamarkupfalse%
%
\endisatagproof
{\isafoldproof}%
%
\isadelimproof
\isanewline
%
\endisadelimproof
\isanewline
\isacommand{lemma}\isamarkupfalse%
\ hash{\isacharunderscore}{\kern0pt}prob{\isacharunderscore}{\kern0pt}range{\isacharcolon}{\kern0pt}\isanewline
\ \ \isakeyword{assumes}\ {\isachardoublequoteopen}prime\ p{\isachardoublequoteclose}\isanewline
\ \ \isakeyword{assumes}\ {\isachardoublequoteopen}x\ {\isacharless}{\kern0pt}\ p{\isachardoublequoteclose}\isanewline
\ \ \isakeyword{assumes}\ {\isachardoublequoteopen}n\ {\isachargreater}{\kern0pt}\ {\isadigit{0}}{\isachardoublequoteclose}\isanewline
\ \ \isakeyword{shows}\ {\isachardoublequoteopen}{\isasymP}{\isacharparenleft}{\kern0pt}{\isasymomega}\ in\ measure{\isacharunderscore}{\kern0pt}pmf\ {\isacharparenleft}{\kern0pt}pmf{\isacharunderscore}{\kern0pt}of{\isacharunderscore}{\kern0pt}set\ {\isacharparenleft}{\kern0pt}bounded{\isacharunderscore}{\kern0pt}degree{\isacharunderscore}{\kern0pt}polynomials\ {\isacharparenleft}{\kern0pt}ZFact\ {\isacharparenleft}{\kern0pt}int\ p{\isacharparenright}{\kern0pt}{\isacharparenright}{\kern0pt}\ n{\isacharparenright}{\kern0pt}{\isacharparenright}{\kern0pt}{\isachardot}{\kern0pt}\isanewline
\ \ \ \ hash\ p\ x\ {\isasymomega}\ {\isasymin}\ A{\isacharparenright}{\kern0pt}\ {\isacharequal}{\kern0pt}\ card\ {\isacharparenleft}{\kern0pt}A\ {\isasyminter}\ {\isacharbraceleft}{\kern0pt}{\isadigit{0}}{\isachardot}{\kern0pt}{\isachardot}{\kern0pt}{\isacharless}{\kern0pt}p{\isacharbraceright}{\kern0pt}{\isacharparenright}{\kern0pt}\ {\isacharslash}{\kern0pt}\ p{\isachardoublequoteclose}\isanewline
%
\isadelimproof
%
\endisadelimproof
%
\isatagproof
\isacommand{proof}\isamarkupfalse%
\ {\isacharminus}{\kern0pt}\isanewline
\ \ \isacommand{define}\isamarkupfalse%
\ {\isasymOmega}\ \isakeyword{where}\ {\isachardoublequoteopen}{\isasymOmega}\ {\isacharequal}{\kern0pt}\ measure{\isacharunderscore}{\kern0pt}pmf\ {\isacharparenleft}{\kern0pt}pmf{\isacharunderscore}{\kern0pt}of{\isacharunderscore}{\kern0pt}set\ {\isacharparenleft}{\kern0pt}bounded{\isacharunderscore}{\kern0pt}degree{\isacharunderscore}{\kern0pt}polynomials\ {\isacharparenleft}{\kern0pt}ZFact\ {\isacharparenleft}{\kern0pt}int\ p{\isacharparenright}{\kern0pt}{\isacharparenright}{\kern0pt}\ n{\isacharparenright}{\kern0pt}{\isacharparenright}{\kern0pt}{\isachardoublequoteclose}\isanewline
\ \ \isacommand{have}\isamarkupfalse%
\ p{\isacharunderscore}{\kern0pt}ge{\isacharunderscore}{\kern0pt}{\isadigit{0}}{\isacharcolon}{\kern0pt}\ {\isachardoublequoteopen}p\ {\isachargreater}{\kern0pt}\ {\isadigit{0}}{\isachardoublequoteclose}\ \isacommand{using}\isamarkupfalse%
\ assms{\isacharparenleft}{\kern0pt}{\isadigit{1}}{\isacharparenright}{\kern0pt}\ \isacommand{by}\isamarkupfalse%
\ {\isacharparenleft}{\kern0pt}simp\ add{\isacharcolon}{\kern0pt}\ prime{\isacharunderscore}{\kern0pt}gt{\isacharunderscore}{\kern0pt}{\isadigit{0}}{\isacharunderscore}{\kern0pt}nat{\isacharparenright}{\kern0pt}\isanewline
\isanewline
\ \ \isacommand{have}\isamarkupfalse%
\ {\isachardoublequoteopen}{\isasymP}{\isacharparenleft}{\kern0pt}{\isasymomega}\ in\ {\isasymOmega}{\isachardot}{\kern0pt}\ hash\ p\ x\ {\isasymomega}\ {\isasymin}\ A{\isacharparenright}{\kern0pt}\ {\isacharequal}{\kern0pt}\ measure\ {\isasymOmega}\ {\isacharparenleft}{\kern0pt}{\isasymUnion}\ k\ {\isasymin}\ A\ {\isasyminter}\ {\isacharbraceleft}{\kern0pt}{\isadigit{0}}{\isachardot}{\kern0pt}{\isachardot}{\kern0pt}{\isacharless}{\kern0pt}p{\isacharbraceright}{\kern0pt}{\isachardot}{\kern0pt}\ {\isacharbraceleft}{\kern0pt}{\isasymomega}{\isachardot}{\kern0pt}\ hash\ p\ x\ {\isasymomega}\ {\isacharequal}{\kern0pt}\ k{\isacharbraceright}{\kern0pt}{\isacharparenright}{\kern0pt}{\isachardoublequoteclose}\isanewline
\ \ \ \ \isacommand{apply}\isamarkupfalse%
\ {\isacharparenleft}{\kern0pt}simp\ only{\isacharcolon}{\kern0pt}{\isasymOmega}{\isacharunderscore}{\kern0pt}def{\isacharparenright}{\kern0pt}\isanewline
\ \ \ \ \isacommand{apply}\isamarkupfalse%
\ {\isacharparenleft}{\kern0pt}rule\ pmf{\isacharunderscore}{\kern0pt}eq{\isacharcomma}{\kern0pt}\ simp{\isacharparenright}{\kern0pt}\isanewline
\ \ \ \ \isacommand{apply}\isamarkupfalse%
\ {\isacharparenleft}{\kern0pt}subst\ {\isacharparenleft}{\kern0pt}asm{\isacharparenright}{\kern0pt}\ set{\isacharunderscore}{\kern0pt}pmf{\isacharunderscore}{\kern0pt}of{\isacharunderscore}{\kern0pt}set{\isacharbrackleft}{\kern0pt}OF\ ne{\isacharunderscore}{\kern0pt}bounded{\isacharunderscore}{\kern0pt}degree{\isacharunderscore}{\kern0pt}polynomials\ fin{\isacharunderscore}{\kern0pt}bounded{\isacharunderscore}{\kern0pt}degree{\isacharunderscore}{\kern0pt}polynomials{\isacharbrackleft}{\kern0pt}OF\ p{\isacharunderscore}{\kern0pt}ge{\isacharunderscore}{\kern0pt}{\isadigit{0}}{\isacharbrackright}{\kern0pt}{\isacharbrackright}{\kern0pt}{\isacharparenright}{\kern0pt}\isanewline
\ \ \ \ \isacommand{using}\isamarkupfalse%
\ hash{\isacharunderscore}{\kern0pt}range{\isacharbrackleft}{\kern0pt}OF\ p{\isacharunderscore}{\kern0pt}ge{\isacharunderscore}{\kern0pt}{\isadigit{0}}\ {\isacharunderscore}{\kern0pt}\ assms{\isacharparenleft}{\kern0pt}{\isadigit{2}}{\isacharparenright}{\kern0pt}{\isacharbrackright}{\kern0pt}\ \isacommand{by}\isamarkupfalse%
\ simp\isanewline
\ \ \isacommand{also}\isamarkupfalse%
\ \isacommand{have}\isamarkupfalse%
\ {\isachardoublequoteopen}{\isachardot}{\kern0pt}{\isachardot}{\kern0pt}{\isachardot}{\kern0pt}\ {\isacharequal}{\kern0pt}\ {\isacharparenleft}{\kern0pt}{\isasymSum}\ k\ {\isasymin}\ {\isacharparenleft}{\kern0pt}A\ {\isasyminter}\ {\isacharbraceleft}{\kern0pt}{\isadigit{0}}{\isachardot}{\kern0pt}{\isachardot}{\kern0pt}{\isacharless}{\kern0pt}p{\isacharbraceright}{\kern0pt}{\isacharparenright}{\kern0pt}{\isachardot}{\kern0pt}\ measure\ {\isasymOmega}\ {\isacharbraceleft}{\kern0pt}{\isasymomega}{\isachardot}{\kern0pt}\ hash\ p\ x\ {\isasymomega}\ {\isacharequal}{\kern0pt}\ k{\isacharbraceright}{\kern0pt}{\isacharparenright}{\kern0pt}{\isachardoublequoteclose}\isanewline
\ \ \ \ \isacommand{apply}\isamarkupfalse%
\ {\isacharparenleft}{\kern0pt}rule\ measure{\isacharunderscore}{\kern0pt}finite{\isacharunderscore}{\kern0pt}Union{\isacharcomma}{\kern0pt}\ simp{\isacharcomma}{\kern0pt}\ simp\ add{\isacharcolon}{\kern0pt}{\isasymOmega}{\isacharunderscore}{\kern0pt}def{\isacharparenright}{\kern0pt}\isanewline
\ \ \ \ \ \isacommand{apply}\isamarkupfalse%
\ {\isacharparenleft}{\kern0pt}simp\ add{\isacharcolon}{\kern0pt}disjoint{\isacharunderscore}{\kern0pt}family{\isacharunderscore}{\kern0pt}on{\isacharunderscore}{\kern0pt}def{\isacharcomma}{\kern0pt}\ fastforce{\isacharparenright}{\kern0pt}\ \isanewline
\ \ \ \ \isacommand{by}\isamarkupfalse%
\ {\isacharparenleft}{\kern0pt}simp\ add{\isacharcolon}{\kern0pt}{\isasymOmega}{\isacharunderscore}{\kern0pt}def{\isacharparenright}{\kern0pt}\isanewline
\ \ \isacommand{also}\isamarkupfalse%
\ \isacommand{have}\isamarkupfalse%
\ {\isachardoublequoteopen}{\isachardot}{\kern0pt}{\isachardot}{\kern0pt}{\isachardot}{\kern0pt}\ {\isacharequal}{\kern0pt}\ {\isacharparenleft}{\kern0pt}{\isasymSum}\ k\ {\isasymin}\ {\isacharparenleft}{\kern0pt}A\ {\isasyminter}\ {\isacharbraceleft}{\kern0pt}{\isadigit{0}}{\isachardot}{\kern0pt}{\isachardot}{\kern0pt}{\isacharless}{\kern0pt}p{\isacharbraceright}{\kern0pt}{\isacharparenright}{\kern0pt}{\isachardot}{\kern0pt}\ {\isasymP}{\isacharparenleft}{\kern0pt}{\isasymomega}\ in\ {\isasymOmega}{\isachardot}{\kern0pt}\ {\isasymforall}x{\isacharprime}{\kern0pt}\ {\isasymin}\ {\isacharbraceleft}{\kern0pt}x{\isacharbraceright}{\kern0pt}{\isachardot}{\kern0pt}\ hash\ p\ x{\isacharprime}{\kern0pt}\ {\isasymomega}\ {\isacharequal}{\kern0pt}\ k\ {\isacharparenright}{\kern0pt}{\isacharparenright}{\kern0pt}{\isachardoublequoteclose}\isanewline
\ \ \ \ \isacommand{by}\isamarkupfalse%
\ {\isacharparenleft}{\kern0pt}simp\ add{\isacharcolon}{\kern0pt}{\isasymOmega}{\isacharunderscore}{\kern0pt}def{\isacharparenright}{\kern0pt}\isanewline
\ \ \isacommand{also}\isamarkupfalse%
\ \isacommand{have}\isamarkupfalse%
\ {\isachardoublequoteopen}{\isachardot}{\kern0pt}{\isachardot}{\kern0pt}{\isachardot}{\kern0pt}\ {\isacharequal}{\kern0pt}\ {\isacharparenleft}{\kern0pt}{\isasymSum}\ k\ {\isasymin}\ {\isacharparenleft}{\kern0pt}A\ {\isasyminter}\ {\isacharbraceleft}{\kern0pt}{\isadigit{0}}{\isachardot}{\kern0pt}{\isachardot}{\kern0pt}{\isacharless}{\kern0pt}p{\isacharbraceright}{\kern0pt}{\isacharparenright}{\kern0pt}{\isachardot}{\kern0pt}\ {\isadigit{1}}{\isacharslash}{\kern0pt}\ real\ p\ {\isacharcircum}{\kern0pt}\ card\ {\isacharbraceleft}{\kern0pt}x{\isacharbraceright}{\kern0pt}{\isacharparenright}{\kern0pt}{\isachardoublequoteclose}\isanewline
\ \ \ \ \isacommand{apply}\isamarkupfalse%
\ {\isacharparenleft}{\kern0pt}rule\ sum{\isachardot}{\kern0pt}cong{\isacharcomma}{\kern0pt}\ simp{\isacharparenright}{\kern0pt}\isanewline
\ \ \ \ \isacommand{apply}\isamarkupfalse%
\ {\isacharparenleft}{\kern0pt}simp\ only{\isacharcolon}{\kern0pt}{\isasymOmega}{\isacharunderscore}{\kern0pt}def{\isacharparenright}{\kern0pt}\isanewline
\ \ \ \ \isacommand{apply}\isamarkupfalse%
\ {\isacharparenleft}{\kern0pt}rule\ hash{\isacharunderscore}{\kern0pt}prob{\isacharbrackleft}{\kern0pt}OF\ assms{\isacharparenleft}{\kern0pt}{\isadigit{1}}{\isacharparenright}{\kern0pt}{\isacharbrackright}{\kern0pt}{\isacharcomma}{\kern0pt}\ simp\ add{\isacharcolon}{\kern0pt}assms{\isacharcomma}{\kern0pt}\ simp{\isacharparenright}{\kern0pt}\isanewline
\ \ \ \ \isacommand{using}\isamarkupfalse%
\ assms{\isacharparenleft}{\kern0pt}{\isadigit{3}}{\isacharparenright}{\kern0pt}\ \isacommand{by}\isamarkupfalse%
\ simp\isanewline
\ \ \isacommand{also}\isamarkupfalse%
\ \isacommand{have}\isamarkupfalse%
\ {\isachardoublequoteopen}{\isachardot}{\kern0pt}{\isachardot}{\kern0pt}{\isachardot}{\kern0pt}\ {\isacharequal}{\kern0pt}\ card\ {\isacharparenleft}{\kern0pt}A\ {\isasyminter}\ {\isacharbraceleft}{\kern0pt}{\isadigit{0}}{\isachardot}{\kern0pt}{\isachardot}{\kern0pt}{\isacharless}{\kern0pt}p{\isacharbraceright}{\kern0pt}{\isacharparenright}{\kern0pt}\ {\isacharslash}{\kern0pt}\ real\ p{\isachardoublequoteclose}\isanewline
\ \ \ \ \isacommand{by}\isamarkupfalse%
\ simp\isanewline
\ \ \isacommand{finally}\isamarkupfalse%
\ \isacommand{show}\isamarkupfalse%
\ {\isacharquery}{\kern0pt}thesis\isanewline
\ \ \ \ \isacommand{by}\isamarkupfalse%
\ {\isacharparenleft}{\kern0pt}simp\ only{\isacharcolon}{\kern0pt}{\isasymOmega}{\isacharunderscore}{\kern0pt}def{\isacharparenright}{\kern0pt}\isanewline
\isacommand{qed}\isamarkupfalse%
%
\endisatagproof
{\isafoldproof}%
%
\isadelimproof
\isanewline
%
\endisadelimproof
\isanewline
\isacommand{lemma}\isamarkupfalse%
\ hash{\isacharunderscore}{\kern0pt}k{\isacharunderscore}{\kern0pt}wise{\isacharunderscore}{\kern0pt}indep{\isacharcolon}{\kern0pt}\isanewline
\ \ \isakeyword{assumes}\ {\isachardoublequoteopen}prime\ p{\isachardoublequoteclose}\isanewline
\ \ \isakeyword{assumes}\ {\isachardoublequoteopen}{\isadigit{1}}\ {\isasymle}\ n{\isachardoublequoteclose}\isanewline
\ \ \isakeyword{shows}\ {\isachardoublequoteopen}prob{\isacharunderscore}{\kern0pt}space{\isachardot}{\kern0pt}k{\isacharunderscore}{\kern0pt}wise{\isacharunderscore}{\kern0pt}indep{\isacharunderscore}{\kern0pt}vars\ {\isacharparenleft}{\kern0pt}measure{\isacharunderscore}{\kern0pt}pmf\ {\isacharparenleft}{\kern0pt}pmf{\isacharunderscore}{\kern0pt}of{\isacharunderscore}{\kern0pt}set\ {\isacharparenleft}{\kern0pt}bounded{\isacharunderscore}{\kern0pt}degree{\isacharunderscore}{\kern0pt}polynomials\ {\isacharparenleft}{\kern0pt}ZFact\ {\isacharparenleft}{\kern0pt}int\ p{\isacharparenright}{\kern0pt}{\isacharparenright}{\kern0pt}\ n{\isacharparenright}{\kern0pt}{\isacharparenright}{\kern0pt}{\isacharparenright}{\kern0pt}\isanewline
\ \ \ n\ {\isacharparenleft}{\kern0pt}{\isasymlambda}{\isacharunderscore}{\kern0pt}{\isachardot}{\kern0pt}\ pmf{\isacharunderscore}{\kern0pt}of{\isacharunderscore}{\kern0pt}set\ {\isacharbraceleft}{\kern0pt}{\isadigit{0}}{\isachardot}{\kern0pt}{\isachardot}{\kern0pt}{\isacharless}{\kern0pt}p{\isacharbraceright}{\kern0pt}{\isacharparenright}{\kern0pt}\ {\isacharparenleft}{\kern0pt}hash\ p{\isacharparenright}{\kern0pt}\ {\isacharbraceleft}{\kern0pt}{\isadigit{0}}{\isachardot}{\kern0pt}{\isachardot}{\kern0pt}{\isacharless}{\kern0pt}p{\isacharbraceright}{\kern0pt}{\isachardoublequoteclose}\isanewline
%
\isadelimproof
%
\endisadelimproof
%
\isatagproof
\isacommand{proof}\isamarkupfalse%
\ {\isacharminus}{\kern0pt}\isanewline
\ \ \isacommand{have}\isamarkupfalse%
\ p{\isacharunderscore}{\kern0pt}ge{\isacharunderscore}{\kern0pt}{\isadigit{0}}{\isacharcolon}{\kern0pt}\ {\isachardoublequoteopen}p\ {\isachargreater}{\kern0pt}\ {\isadigit{0}}{\isachardoublequoteclose}\isanewline
\ \ \ \ \isacommand{using}\isamarkupfalse%
\ assms{\isacharparenleft}{\kern0pt}{\isadigit{1}}{\isacharparenright}{\kern0pt}\ \isacommand{by}\isamarkupfalse%
\ {\isacharparenleft}{\kern0pt}simp\ add{\isacharcolon}{\kern0pt}\ prime{\isacharunderscore}{\kern0pt}gt{\isacharunderscore}{\kern0pt}{\isadigit{0}}{\isacharunderscore}{\kern0pt}nat{\isacharparenright}{\kern0pt}\isanewline
\isanewline
\ \ \isacommand{have}\isamarkupfalse%
\ a{\isacharcolon}{\kern0pt}{\isachardoublequoteopen}{\isasymAnd}J{\isachardot}{\kern0pt}\ J\ {\isasymsubseteq}\ {\isacharbraceleft}{\kern0pt}{\isadigit{0}}{\isachardot}{\kern0pt}{\isachardot}{\kern0pt}{\isacharless}{\kern0pt}p{\isacharbraceright}{\kern0pt}\ {\isasymLongrightarrow}\ card\ J\ {\isasymle}\ n\ {\isasymLongrightarrow}\ finite\ J\ {\isasymLongrightarrow}\isanewline
\ \ \ \ prob{\isacharunderscore}{\kern0pt}space{\isachardot}{\kern0pt}indep{\isacharunderscore}{\kern0pt}vars\ {\isacharparenleft}{\kern0pt}measure{\isacharunderscore}{\kern0pt}pmf\ {\isacharparenleft}{\kern0pt}pmf{\isacharunderscore}{\kern0pt}of{\isacharunderscore}{\kern0pt}set\ {\isacharparenleft}{\kern0pt}bounded{\isacharunderscore}{\kern0pt}degree{\isacharunderscore}{\kern0pt}polynomials\ {\isacharparenleft}{\kern0pt}ZFact\ {\isacharparenleft}{\kern0pt}int\ p{\isacharparenright}{\kern0pt}{\isacharparenright}{\kern0pt}\ n{\isacharparenright}{\kern0pt}{\isacharparenright}{\kern0pt}{\isacharparenright}{\kern0pt}\isanewline
\ \ \ \ \ \ \ \ {\isacharparenleft}{\kern0pt}{\isacharparenleft}{\kern0pt}{\isasymlambda}x{\isachardot}{\kern0pt}\ measure{\isacharunderscore}{\kern0pt}pmf\ {\isacharparenleft}{\kern0pt}pmf{\isacharunderscore}{\kern0pt}of{\isacharunderscore}{\kern0pt}set\ {\isacharbraceleft}{\kern0pt}{\isadigit{0}}{\isachardot}{\kern0pt}{\isachardot}{\kern0pt}{\isacharless}{\kern0pt}p{\isacharbraceright}{\kern0pt}{\isacharparenright}{\kern0pt}{\isacharparenright}{\kern0pt}\ {\isasymcirc}\ zfact{\isacharunderscore}{\kern0pt}embed\ p{\isacharparenright}{\kern0pt}\ {\isacharparenleft}{\kern0pt}{\isasymlambda}i\ {\isasymomega}{\isachardot}{\kern0pt}\ hash\ p\ i\ {\isasymomega}{\isacharparenright}{\kern0pt}\ J{\isachardoublequoteclose}\isanewline
\ \ \ \ \isacommand{apply}\isamarkupfalse%
\ {\isacharparenleft}{\kern0pt}subst\ hash{\isachardot}{\kern0pt}simps{\isacharparenright}{\kern0pt}\isanewline
\ \ \ \ \isacommand{apply}\isamarkupfalse%
\ {\isacharparenleft}{\kern0pt}rule\ prob{\isacharunderscore}{\kern0pt}space{\isachardot}{\kern0pt}indep{\isacharunderscore}{\kern0pt}vars{\isacharunderscore}{\kern0pt}reindex{\isacharbrackleft}{\kern0pt}OF\ prob{\isacharunderscore}{\kern0pt}space{\isacharunderscore}{\kern0pt}measure{\isacharunderscore}{\kern0pt}pmf{\isacharbrackright}{\kern0pt}{\isacharparenright}{\kern0pt}\isanewline
\ \ \ \ \ \isacommand{apply}\isamarkupfalse%
\ {\isacharparenleft}{\kern0pt}rule\ inj{\isacharunderscore}{\kern0pt}on{\isacharunderscore}{\kern0pt}subset{\isacharbrackleft}{\kern0pt}OF\ zfact{\isacharunderscore}{\kern0pt}embed{\isacharunderscore}{\kern0pt}inj{\isacharbrackleft}{\kern0pt}OF\ p{\isacharunderscore}{\kern0pt}ge{\isacharunderscore}{\kern0pt}{\isadigit{0}}{\isacharbrackright}{\kern0pt}{\isacharbrackright}{\kern0pt}{\isacharcomma}{\kern0pt}\ simp{\isacharparenright}{\kern0pt}\isanewline
\ \ \ \ \isacommand{apply}\isamarkupfalse%
\ {\isacharparenleft}{\kern0pt}rule\ prob{\isacharunderscore}{\kern0pt}space{\isachardot}{\kern0pt}indep{\isacharunderscore}{\kern0pt}vars{\isacharunderscore}{\kern0pt}compose{\isadigit{2}}{\isacharbrackleft}{\kern0pt}\isakeyword{where}\ Y{\isacharequal}{\kern0pt}{\isachardoublequoteopen}{\isasymlambda}{\isacharunderscore}{\kern0pt}{\isachardot}{\kern0pt}\ the{\isacharunderscore}{\kern0pt}inv{\isacharunderscore}{\kern0pt}into\ {\isacharbraceleft}{\kern0pt}{\isadigit{0}}{\isachardot}{\kern0pt}{\isachardot}{\kern0pt}{\isacharless}{\kern0pt}p{\isacharbraceright}{\kern0pt}\ {\isacharparenleft}{\kern0pt}zfact{\isacharunderscore}{\kern0pt}embed\ p{\isacharparenright}{\kern0pt}{\isachardoublequoteclose}\ \isakeyword{and}\ M{\isacharprime}{\kern0pt}{\isacharequal}{\kern0pt}{\isachardoublequoteopen}{\isasymlambda}{\isacharunderscore}{\kern0pt}{\isachardot}{\kern0pt}\ measure{\isacharunderscore}{\kern0pt}pmf\ {\isacharparenleft}{\kern0pt}pmf{\isacharunderscore}{\kern0pt}of{\isacharunderscore}{\kern0pt}set\ {\isacharparenleft}{\kern0pt}carrier\ {\isacharparenleft}{\kern0pt}ZFact\ p{\isacharparenright}{\kern0pt}{\isacharparenright}{\kern0pt}{\isacharparenright}{\kern0pt}{\isachardoublequoteclose}{\isacharbrackright}{\kern0pt}{\isacharparenright}{\kern0pt}\isanewline
\ \ \ \ \ \ \isacommand{apply}\isamarkupfalse%
\ {\isacharparenleft}{\kern0pt}rule\ prob{\isacharunderscore}{\kern0pt}space{\isacharunderscore}{\kern0pt}measure{\isacharunderscore}{\kern0pt}pmf{\isacharparenright}{\kern0pt}\isanewline
\ \ \ \ \ \isacommand{apply}\isamarkupfalse%
\ {\isacharparenleft}{\kern0pt}rule\ hash{\isacharunderscore}{\kern0pt}indep{\isacharunderscore}{\kern0pt}pmf{\isacharcomma}{\kern0pt}\ metis\ zfact{\isacharunderscore}{\kern0pt}prime{\isacharunderscore}{\kern0pt}is{\isacharunderscore}{\kern0pt}field{\isacharbrackleft}{\kern0pt}OF\ assms{\isacharparenleft}{\kern0pt}{\isadigit{1}}{\isacharparenright}{\kern0pt}{\isacharbrackright}{\kern0pt}\ zfact{\isacharunderscore}{\kern0pt}finite{\isacharbrackleft}{\kern0pt}OF\ p{\isacharunderscore}{\kern0pt}ge{\isacharunderscore}{\kern0pt}{\isadigit{0}}{\isacharbrackright}{\kern0pt}{\isacharparenright}{\kern0pt}\isanewline
\ \ \ \ \ \ \ \ \isacommand{using}\isamarkupfalse%
\ zfact{\isacharunderscore}{\kern0pt}embed{\isacharunderscore}{\kern0pt}ran{\isacharbrackleft}{\kern0pt}OF\ p{\isacharunderscore}{\kern0pt}ge{\isacharunderscore}{\kern0pt}{\isadigit{0}}{\isacharbrackright}{\kern0pt}\ \isacommand{apply}\isamarkupfalse%
\ blast\isanewline
\ \ \ \ \ \ \ \isacommand{apply}\isamarkupfalse%
\ simp\isanewline
\ \ \ \ \ \ \isacommand{apply}\isamarkupfalse%
\ {\isacharparenleft}{\kern0pt}subst\ card{\isacharunderscore}{\kern0pt}image{\isacharcomma}{\kern0pt}\ metis\ zfact{\isacharunderscore}{\kern0pt}embed{\isacharunderscore}{\kern0pt}inj{\isacharbrackleft}{\kern0pt}OF\ p{\isacharunderscore}{\kern0pt}ge{\isacharunderscore}{\kern0pt}{\isadigit{0}}{\isacharbrackright}{\kern0pt}\ inj{\isacharunderscore}{\kern0pt}on{\isacharunderscore}{\kern0pt}subset{\isacharcomma}{\kern0pt}\ simp{\isacharparenright}{\kern0pt}\isanewline
\ \ \ \ \ \isacommand{apply}\isamarkupfalse%
\ {\isacharparenleft}{\kern0pt}metis\ assms{\isacharparenleft}{\kern0pt}{\isadigit{2}}{\isacharparenright}{\kern0pt}{\isacharparenright}{\kern0pt}\isanewline
\ \ \ \ \isacommand{by}\isamarkupfalse%
\ simp\isanewline
\isanewline
\ \ \isacommand{show}\isamarkupfalse%
\ {\isacharquery}{\kern0pt}thesis\isanewline
\ \ \ \ \isacommand{using}\isamarkupfalse%
\ a\ \isacommand{by}\isamarkupfalse%
\ {\isacharparenleft}{\kern0pt}simp\ add{\isacharcolon}{\kern0pt}measure{\isacharunderscore}{\kern0pt}pmf{\isachardot}{\kern0pt}k{\isacharunderscore}{\kern0pt}wise{\isacharunderscore}{\kern0pt}indep{\isacharunderscore}{\kern0pt}vars{\isacharunderscore}{\kern0pt}def\ comp{\isacharunderscore}{\kern0pt}def{\isacharparenright}{\kern0pt}\isanewline
\isacommand{qed}\isamarkupfalse%
%
\endisatagproof
{\isafoldproof}%
%
\isadelimproof
%
\endisadelimproof
%
\isadelimdocument
%
\endisadelimdocument
%
\isatagdocument
%
\isamarkupsubsection{Encoding%
}
\isamarkuptrue%
%
\endisatagdocument
{\isafolddocument}%
%
\isadelimdocument
%
\endisadelimdocument
\isacommand{fun}\isamarkupfalse%
\ zfact\isactrlsub S\ \isakeyword{where}\ {\isachardoublequoteopen}zfact\isactrlsub S\ p\ x\ {\isacharequal}{\kern0pt}\ {\isacharparenleft}{\kern0pt}\isanewline
\ \ \ \ if\ x\ {\isasymin}\ zfact{\isacharunderscore}{\kern0pt}embed\ p\ {\isacharbackquote}{\kern0pt}\ {\isacharbraceleft}{\kern0pt}{\isadigit{0}}{\isachardot}{\kern0pt}{\isachardot}{\kern0pt}{\isacharless}{\kern0pt}p{\isacharbraceright}{\kern0pt}\ then\isanewline
\ \ \ \ \ \ N\isactrlsub S\ {\isacharparenleft}{\kern0pt}the{\isacharunderscore}{\kern0pt}inv{\isacharunderscore}{\kern0pt}into\ {\isacharbraceleft}{\kern0pt}{\isadigit{0}}{\isachardot}{\kern0pt}{\isachardot}{\kern0pt}{\isacharless}{\kern0pt}p{\isacharbraceright}{\kern0pt}\ {\isacharparenleft}{\kern0pt}zfact{\isacharunderscore}{\kern0pt}embed\ p{\isacharparenright}{\kern0pt}\ x{\isacharparenright}{\kern0pt}\isanewline
\ \ \ \ else\isanewline
\ \ \ \ \ None\isanewline
\ \ {\isacharparenright}{\kern0pt}{\isachardoublequoteclose}\isanewline
\isanewline
\isacommand{lemma}\isamarkupfalse%
\ zfact{\isacharunderscore}{\kern0pt}encoding\ {\isacharcolon}{\kern0pt}\ \isanewline
\ \ {\isachardoublequoteopen}is{\isacharunderscore}{\kern0pt}encoding\ {\isacharparenleft}{\kern0pt}zfact\isactrlsub S\ p{\isacharparenright}{\kern0pt}{\isachardoublequoteclose}\isanewline
%
\isadelimproof
%
\endisadelimproof
%
\isatagproof
\isacommand{proof}\isamarkupfalse%
\ {\isacharminus}{\kern0pt}\isanewline
\ \ \isacommand{have}\isamarkupfalse%
\ {\isachardoublequoteopen}p\ {\isachargreater}{\kern0pt}\ {\isadigit{0}}\ {\isasymLongrightarrow}\ is{\isacharunderscore}{\kern0pt}encoding\ {\isacharparenleft}{\kern0pt}{\isasymlambda}x{\isachardot}{\kern0pt}\ zfact\isactrlsub S\ p\ x{\isacharparenright}{\kern0pt}{\isachardoublequoteclose}\isanewline
\ \ \ \ \isacommand{apply}\isamarkupfalse%
\ simp\ \isanewline
\ \ \ \ \isacommand{apply}\isamarkupfalse%
\ {\isacharparenleft}{\kern0pt}rule\ encoding{\isacharunderscore}{\kern0pt}compose{\isacharbrackleft}{\kern0pt}\isakeyword{where}\ f{\isacharequal}{\kern0pt}{\isachardoublequoteopen}N\isactrlsub S{\isachardoublequoteclose}{\isacharbrackright}{\kern0pt}{\isacharparenright}{\kern0pt}\isanewline
\ \ \ \ \ \isacommand{apply}\isamarkupfalse%
\ {\isacharparenleft}{\kern0pt}metis\ nat{\isacharunderscore}{\kern0pt}encoding{\isacharcomma}{\kern0pt}\ simp{\isacharparenright}{\kern0pt}\isanewline
\ \ \ \ \isacommand{by}\isamarkupfalse%
\ {\isacharparenleft}{\kern0pt}metis\ inj{\isacharunderscore}{\kern0pt}on{\isacharunderscore}{\kern0pt}the{\isacharunderscore}{\kern0pt}inv{\isacharunderscore}{\kern0pt}into\ zfact{\isacharunderscore}{\kern0pt}embed{\isacharunderscore}{\kern0pt}inj{\isacharparenright}{\kern0pt}\isanewline
\ \ \isacommand{moreover}\isamarkupfalse%
\ \isacommand{have}\isamarkupfalse%
\ {\isachardoublequoteopen}is{\isacharunderscore}{\kern0pt}encoding\ {\isacharparenleft}{\kern0pt}zfact\isactrlsub S\ {\isadigit{0}}{\isacharparenright}{\kern0pt}{\isachardoublequoteclose}\isanewline
\ \ \ \ \isacommand{by}\isamarkupfalse%
\ {\isacharparenleft}{\kern0pt}simp\ add{\isacharcolon}{\kern0pt}is{\isacharunderscore}{\kern0pt}encoding{\isacharunderscore}{\kern0pt}def{\isacharparenright}{\kern0pt}\isanewline
\ \ \isacommand{ultimately}\isamarkupfalse%
\ \isacommand{show}\isamarkupfalse%
\ {\isacharquery}{\kern0pt}thesis\ \isacommand{by}\isamarkupfalse%
\ blast\isanewline
\isacommand{qed}\isamarkupfalse%
%
\endisatagproof
{\isafoldproof}%
%
\isadelimproof
\isanewline
%
\endisadelimproof
\isanewline
\isacommand{lemma}\isamarkupfalse%
\ bounded{\isacharunderscore}{\kern0pt}degree{\isacharunderscore}{\kern0pt}polynomial{\isacharunderscore}{\kern0pt}bit{\isacharunderscore}{\kern0pt}count{\isacharcolon}{\kern0pt}\isanewline
\ \ \isakeyword{assumes}\ {\isachardoublequoteopen}p\ {\isachargreater}{\kern0pt}\ {\isadigit{0}}{\isachardoublequoteclose}\isanewline
\ \ \isakeyword{assumes}\ {\isachardoublequoteopen}x\ {\isasymin}\ bounded{\isacharunderscore}{\kern0pt}degree{\isacharunderscore}{\kern0pt}polynomials\ {\isacharparenleft}{\kern0pt}ZFact\ p{\isacharparenright}{\kern0pt}\ n{\isachardoublequoteclose}\isanewline
\ \ \isakeyword{shows}\ {\isachardoublequoteopen}bit{\isacharunderscore}{\kern0pt}count\ {\isacharparenleft}{\kern0pt}list\isactrlsub S\ {\isacharparenleft}{\kern0pt}zfact\isactrlsub S\ p{\isacharparenright}{\kern0pt}\ x{\isacharparenright}{\kern0pt}\ {\isasymle}\ ereal\ {\isacharparenleft}{\kern0pt}real\ n\ {\isacharasterisk}{\kern0pt}\ {\isacharparenleft}{\kern0pt}{\isadigit{2}}\ {\isacharasterisk}{\kern0pt}\ log\ {\isadigit{2}}\ p\ {\isacharplus}{\kern0pt}\ {\isadigit{2}}{\isacharparenright}{\kern0pt}\ {\isacharplus}{\kern0pt}\ {\isadigit{1}}{\isacharparenright}{\kern0pt}{\isachardoublequoteclose}\isanewline
%
\isadelimproof
%
\endisadelimproof
%
\isatagproof
\isacommand{proof}\isamarkupfalse%
\ {\isacharminus}{\kern0pt}\isanewline
\ \ \isacommand{have}\isamarkupfalse%
\ b{\isacharcolon}{\kern0pt}{\isachardoublequoteopen}real\ {\isacharparenleft}{\kern0pt}length\ x{\isacharparenright}{\kern0pt}\ {\isasymle}\ real\ n{\isachardoublequoteclose}\isanewline
\ \ \ \ \isacommand{using}\isamarkupfalse%
\ assms{\isacharparenleft}{\kern0pt}{\isadigit{2}}{\isacharparenright}{\kern0pt}\ \isanewline
\ \ \ \ \isacommand{apply}\isamarkupfalse%
\ {\isacharparenleft}{\kern0pt}simp\ add{\isacharcolon}{\kern0pt}bounded{\isacharunderscore}{\kern0pt}degree{\isacharunderscore}{\kern0pt}polynomials{\isacharunderscore}{\kern0pt}def{\isacharparenright}{\kern0pt}\isanewline
\ \ \ \ \isacommand{apply}\isamarkupfalse%
\ {\isacharparenleft}{\kern0pt}cases\ {\isachardoublequoteopen}x{\isacharequal}{\kern0pt}{\isacharbrackleft}{\kern0pt}{\isacharbrackright}{\kern0pt}{\isachardoublequoteclose}{\isacharcomma}{\kern0pt}\ simp{\isacharcomma}{\kern0pt}\ simp{\isacharparenright}{\kern0pt}\isanewline
\ \ \ \ \isacommand{by}\isamarkupfalse%
\ linarith\isanewline
\isanewline
\ \ \isacommand{have}\isamarkupfalse%
\ a{\isacharcolon}{\kern0pt}{\isachardoublequoteopen}{\isasymAnd}y{\isachardot}{\kern0pt}\ y\ {\isasymin}\ set\ x\ {\isasymLongrightarrow}\ y\ {\isasymin}\ zfact{\isacharunderscore}{\kern0pt}embed\ p\ {\isacharbackquote}{\kern0pt}\ {\isacharbraceleft}{\kern0pt}{\isadigit{0}}{\isachardot}{\kern0pt}{\isachardot}{\kern0pt}{\isacharless}{\kern0pt}p{\isacharbraceright}{\kern0pt}{\isachardoublequoteclose}\ \isanewline
\ \ \ \ \isacommand{using}\isamarkupfalse%
\ assms{\isacharparenleft}{\kern0pt}{\isadigit{2}}{\isacharparenright}{\kern0pt}\ \isanewline
\ \ \ \ \isacommand{apply}\isamarkupfalse%
\ {\isacharparenleft}{\kern0pt}simp\ add{\isacharcolon}{\kern0pt}bounded{\isacharunderscore}{\kern0pt}degree{\isacharunderscore}{\kern0pt}polynomials{\isacharunderscore}{\kern0pt}def{\isacharparenright}{\kern0pt}\isanewline
\ \ \ \ \isacommand{by}\isamarkupfalse%
\ {\isacharparenleft}{\kern0pt}metis\ length{\isacharunderscore}{\kern0pt}greater{\isacharunderscore}{\kern0pt}{\isadigit{0}}{\isacharunderscore}{\kern0pt}conv\ length{\isacharunderscore}{\kern0pt}pos{\isacharunderscore}{\kern0pt}if{\isacharunderscore}{\kern0pt}in{\isacharunderscore}{\kern0pt}set\ polynomial{\isacharunderscore}{\kern0pt}def\ subsetD\ univ{\isacharunderscore}{\kern0pt}poly{\isacharunderscore}{\kern0pt}carrier\ zfact{\isacharunderscore}{\kern0pt}embed{\isacharunderscore}{\kern0pt}ran{\isacharbrackleft}{\kern0pt}OF\ assms{\isacharparenleft}{\kern0pt}{\isadigit{1}}{\isacharparenright}{\kern0pt}{\isacharbrackright}{\kern0pt}{\isacharparenright}{\kern0pt}\isanewline
\isanewline
\ \ \isacommand{have}\isamarkupfalse%
\ {\isachardoublequoteopen}bit{\isacharunderscore}{\kern0pt}count\ {\isacharparenleft}{\kern0pt}list\isactrlsub S\ {\isacharparenleft}{\kern0pt}zfact\isactrlsub S\ p{\isacharparenright}{\kern0pt}\ x{\isacharparenright}{\kern0pt}\ {\isasymle}\ ereal\ {\isacharparenleft}{\kern0pt}real\ {\isacharparenleft}{\kern0pt}length\ x{\isacharparenright}{\kern0pt}{\isacharparenright}{\kern0pt}\ {\isacharasterisk}{\kern0pt}\ {\isacharparenleft}{\kern0pt}\ ereal\ {\isacharparenleft}{\kern0pt}{\isadigit{2}}\ {\isacharasterisk}{\kern0pt}\ log\ {\isadigit{2}}\ {\isacharparenleft}{\kern0pt}{\isadigit{1}}\ {\isacharplus}{\kern0pt}\ real\ {\isacharparenleft}{\kern0pt}p{\isacharminus}{\kern0pt}{\isadigit{1}}{\isacharparenright}{\kern0pt}{\isacharparenright}{\kern0pt}\ {\isacharplus}{\kern0pt}\ {\isadigit{1}}{\isacharparenright}{\kern0pt}\ {\isacharplus}{\kern0pt}\ {\isadigit{1}}{\isacharparenright}{\kern0pt}\ {\isacharplus}{\kern0pt}\ {\isadigit{1}}{\isachardoublequoteclose}\isanewline
\ \ \ \ \isacommand{apply}\isamarkupfalse%
\ {\isacharparenleft}{\kern0pt}rule\ list{\isacharunderscore}{\kern0pt}bit{\isacharunderscore}{\kern0pt}count{\isacharunderscore}{\kern0pt}est{\isacharparenright}{\kern0pt}\isanewline
\ \ \ \ \isacommand{apply}\isamarkupfalse%
\ {\isacharparenleft}{\kern0pt}simp\ add{\isacharcolon}{\kern0pt}a\ del{\isacharcolon}{\kern0pt}N\isactrlsub S{\isachardot}{\kern0pt}simps{\isacharparenright}{\kern0pt}\isanewline
\ \ \ \ \isacommand{apply}\isamarkupfalse%
\ {\isacharparenleft}{\kern0pt}rule\ nat{\isacharunderscore}{\kern0pt}bit{\isacharunderscore}{\kern0pt}count{\isacharunderscore}{\kern0pt}est{\isacharparenright}{\kern0pt}\isanewline
\ \ \ \ \isacommand{by}\isamarkupfalse%
\ {\isacharparenleft}{\kern0pt}metis\ a\ the{\isacharunderscore}{\kern0pt}inv{\isacharunderscore}{\kern0pt}into{\isacharunderscore}{\kern0pt}into{\isacharbrackleft}{\kern0pt}OF\ zfact{\isacharunderscore}{\kern0pt}embed{\isacharunderscore}{\kern0pt}inj{\isacharbrackleft}{\kern0pt}OF\ assms{\isacharparenleft}{\kern0pt}{\isadigit{1}}{\isacharparenright}{\kern0pt}{\isacharbrackright}{\kern0pt}{\isacharcomma}{\kern0pt}\ \isakeyword{where}\ B{\isacharequal}{\kern0pt}{\isachardoublequoteopen}{\isacharbraceleft}{\kern0pt}{\isadigit{0}}{\isachardot}{\kern0pt}{\isachardot}{\kern0pt}{\isacharless}{\kern0pt}p{\isacharbraceright}{\kern0pt}{\isachardoublequoteclose}{\isacharcomma}{\kern0pt}\ simplified{\isacharbrackright}{\kern0pt}\isanewline
\ \ \ \ \ \ \ \ Suc{\isacharunderscore}{\kern0pt}pred\ assms{\isacharparenleft}{\kern0pt}{\isadigit{1}}{\isacharparenright}{\kern0pt}\ less{\isacharunderscore}{\kern0pt}Suc{\isacharunderscore}{\kern0pt}eq{\isacharunderscore}{\kern0pt}le{\isacharparenright}{\kern0pt}\isanewline
\ \ \isacommand{also}\isamarkupfalse%
\ \isacommand{have}\isamarkupfalse%
\ {\isachardoublequoteopen}{\isachardot}{\kern0pt}{\isachardot}{\kern0pt}{\isachardot}{\kern0pt}\ {\isasymle}\ ereal\ {\isacharparenleft}{\kern0pt}real\ n{\isacharparenright}{\kern0pt}\ {\isacharasterisk}{\kern0pt}\ {\isacharparenleft}{\kern0pt}{\isadigit{2}}\ {\isacharplus}{\kern0pt}\ ereal\ {\isacharparenleft}{\kern0pt}{\isadigit{2}}\ {\isacharasterisk}{\kern0pt}\ log\ {\isadigit{2}}\ p{\isacharparenright}{\kern0pt}\ {\isacharparenright}{\kern0pt}\ {\isacharplus}{\kern0pt}\ {\isadigit{1}}{\isachardoublequoteclose}\isanewline
\ \ \ \ \isacommand{apply}\isamarkupfalse%
\ simp\isanewline
\ \ \ \ \isacommand{apply}\isamarkupfalse%
\ {\isacharparenleft}{\kern0pt}rule\ mult{\isacharunderscore}{\kern0pt}mono{\isacharcomma}{\kern0pt}\ metis\ b{\isacharparenright}{\kern0pt}\isanewline
\ \ \ \ \ \ \isacommand{apply}\isamarkupfalse%
\ {\isacharparenleft}{\kern0pt}rule\ add{\isacharunderscore}{\kern0pt}mono{\isacharparenright}{\kern0pt}\isanewline
\ \ \ \ \isacommand{using}\isamarkupfalse%
\ assms{\isacharparenleft}{\kern0pt}{\isadigit{1}}{\isacharparenright}{\kern0pt}\ \isacommand{by}\isamarkupfalse%
\ simp{\isacharplus}{\kern0pt}\isanewline
\ \ \isacommand{also}\isamarkupfalse%
\ \isacommand{have}\isamarkupfalse%
\ {\isachardoublequoteopen}{\isachardot}{\kern0pt}{\isachardot}{\kern0pt}{\isachardot}{\kern0pt}\ {\isacharequal}{\kern0pt}\ ereal\ {\isacharparenleft}{\kern0pt}real\ n\ {\isacharasterisk}{\kern0pt}\ {\isacharparenleft}{\kern0pt}{\isadigit{2}}\ {\isacharasterisk}{\kern0pt}\ log\ {\isadigit{2}}\ p\ {\isacharplus}{\kern0pt}\ {\isadigit{2}}{\isacharparenright}{\kern0pt}\ {\isacharplus}{\kern0pt}\ {\isadigit{1}}{\isacharparenright}{\kern0pt}{\isachardoublequoteclose}\isanewline
\ \ \ \ \isacommand{by}\isamarkupfalse%
\ simp\isanewline
\ \ \isacommand{finally}\isamarkupfalse%
\ \isacommand{show}\isamarkupfalse%
\ {\isacharquery}{\kern0pt}thesis\ \isacommand{by}\isamarkupfalse%
\ simp\isanewline
\isacommand{qed}\isamarkupfalse%
%
\endisatagproof
{\isafoldproof}%
%
\isadelimproof
\isanewline
%
\endisadelimproof
%
\isadelimtheory
\isanewline
%
\endisadelimtheory
%
\isatagtheory
\isacommand{end}\isamarkupfalse%
%
\endisatagtheory
{\isafoldtheory}%
%
\isadelimtheory
%
\endisadelimtheory
%
\end{isabellebody}%
\endinput
%:%file=UniversalHashFamilyOfPrime.tex%:%
%:%11=1%:%
%:%23=3%:%
%:%24=4%:%
%:%32=6%:%
%:%33=6%:%
%:%34=7%:%
%:%35=8%:%
%:%42=8%:%
%:%43=9%:%
%:%44=10%:%
%:%45=10%:%
%:%46=11%:%
%:%47=12%:%
%:%50=13%:%
%:%54=13%:%
%:%55=13%:%
%:%56=14%:%
%:%57=14%:%
%:%58=15%:%
%:%59=15%:%
%:%64=15%:%
%:%67=16%:%
%:%68=17%:%
%:%69=17%:%
%:%70=18%:%
%:%73=19%:%
%:%77=19%:%
%:%78=19%:%
%:%79=20%:%
%:%80=20%:%
%:%85=20%:%
%:%88=21%:%
%:%89=22%:%
%:%90=22%:%
%:%91=23%:%
%:%92=24%:%
%:%95=25%:%
%:%99=25%:%
%:%100=25%:%
%:%101=26%:%
%:%102=26%:%
%:%103=27%:%
%:%104=27%:%
%:%105=28%:%
%:%106=28%:%
%:%111=28%:%
%:%114=29%:%
%:%115=30%:%
%:%116=30%:%
%:%117=31%:%
%:%118=32%:%
%:%119=33%:%
%:%120=33%:%
%:%121=34%:%
%:%122=35%:%
%:%123=35%:%
%:%124=36%:%
%:%125=37%:%
%:%126=38%:%
%:%127=39%:%
%:%134=40%:%
%:%135=40%:%
%:%136=41%:%
%:%137=41%:%
%:%138=42%:%
%:%139=42%:%
%:%140=43%:%
%:%141=43%:%
%:%142=44%:%
%:%143=44%:%
%:%144=45%:%
%:%145=45%:%
%:%146=46%:%
%:%147=46%:%
%:%148=47%:%
%:%149=48%:%
%:%150=48%:%
%:%151=49%:%
%:%157=49%:%
%:%160=50%:%
%:%161=51%:%
%:%162=51%:%
%:%163=52%:%
%:%164=53%:%
%:%165=54%:%
%:%166=55%:%
%:%173=56%:%
%:%174=56%:%
%:%175=57%:%
%:%176=57%:%
%:%177=57%:%
%:%178=58%:%
%:%179=58%:%
%:%180=59%:%
%:%181=60%:%
%:%182=60%:%
%:%183=61%:%
%:%184=61%:%
%:%185=62%:%
%:%186=63%:%
%:%187=63%:%
%:%188=64%:%
%:%189=64%:%
%:%190=65%:%
%:%191=65%:%
%:%192=66%:%
%:%193=66%:%
%:%194=67%:%
%:%195=67%:%
%:%196=68%:%
%:%197=68%:%
%:%198=69%:%
%:%199=69%:%
%:%200=70%:%
%:%201=70%:%
%:%202=71%:%
%:%203=71%:%
%:%204=72%:%
%:%205=73%:%
%:%206=73%:%
%:%207=74%:%
%:%208=74%:%
%:%209=75%:%
%:%215=75%:%
%:%218=76%:%
%:%219=77%:%
%:%220=77%:%
%:%221=78%:%
%:%222=79%:%
%:%223=80%:%
%:%224=81%:%
%:%225=82%:%
%:%226=83%:%
%:%233=84%:%
%:%234=84%:%
%:%235=85%:%
%:%236=85%:%
%:%237=86%:%
%:%238=86%:%
%:%239=87%:%
%:%240=88%:%
%:%241=88%:%
%:%242=88%:%
%:%243=88%:%
%:%244=89%:%
%:%245=90%:%
%:%246=90%:%
%:%247=91%:%
%:%248=91%:%
%:%249=92%:%
%:%250=92%:%
%:%251=93%:%
%:%252=93%:%
%:%253=94%:%
%:%254=95%:%
%:%255=95%:%
%:%256=96%:%
%:%257=96%:%
%:%258=96%:%
%:%259=97%:%
%:%260=98%:%
%:%261=98%:%
%:%262=99%:%
%:%263=99%:%
%:%264=100%:%
%:%265=100%:%
%:%266=101%:%
%:%267=101%:%
%:%268=102%:%
%:%269=103%:%
%:%270=103%:%
%:%271=104%:%
%:%272=104%:%
%:%273=104%:%
%:%274=105%:%
%:%275=106%:%
%:%276=106%:%
%:%277=107%:%
%:%278=107%:%
%:%279=107%:%
%:%280=108%:%
%:%281=109%:%
%:%282=109%:%
%:%283=110%:%
%:%284=111%:%
%:%285=111%:%
%:%286=112%:%
%:%287=112%:%
%:%288=112%:%
%:%289=113%:%
%:%290=114%:%
%:%291=114%:%
%:%292=115%:%
%:%293=115%:%
%:%294=116%:%
%:%295=116%:%
%:%296=117%:%
%:%297=118%:%
%:%298=118%:%
%:%299=119%:%
%:%300=119%:%
%:%301=120%:%
%:%302=120%:%
%:%303=121%:%
%:%304=121%:%
%:%305=122%:%
%:%306=122%:%
%:%307=123%:%
%:%308=123%:%
%:%309=124%:%
%:%310=124%:%
%:%311=125%:%
%:%312=125%:%
%:%313=126%:%
%:%314=126%:%
%:%315=127%:%
%:%316=127%:%
%:%317=128%:%
%:%318=128%:%
%:%319=129%:%
%:%320=129%:%
%:%321=130%:%
%:%322=130%:%
%:%323=131%:%
%:%324=131%:%
%:%325=132%:%
%:%326=132%:%
%:%327=133%:%
%:%328=133%:%
%:%329=134%:%
%:%330=134%:%
%:%331=134%:%
%:%332=135%:%
%:%333=136%:%
%:%334=136%:%
%:%335=137%:%
%:%336=137%:%
%:%337=138%:%
%:%338=138%:%
%:%339=139%:%
%:%340=139%:%
%:%341=140%:%
%:%342=140%:%
%:%343=141%:%
%:%344=141%:%
%:%345=141%:%
%:%346=142%:%
%:%347=142%:%
%:%348=142%:%
%:%349=143%:%
%:%350=143%:%
%:%351=144%:%
%:%352=144%:%
%:%353=145%:%
%:%354=145%:%
%:%355=146%:%
%:%356=146%:%
%:%357=146%:%
%:%358=146%:%
%:%359=147%:%
%:%365=147%:%
%:%368=148%:%
%:%369=149%:%
%:%370=149%:%
%:%371=150%:%
%:%372=151%:%
%:%373=152%:%
%:%374=153%:%
%:%375=154%:%
%:%376=155%:%
%:%377=156%:%
%:%384=157%:%
%:%385=157%:%
%:%386=158%:%
%:%387=158%:%
%:%388=159%:%
%:%389=159%:%
%:%390=160%:%
%:%391=161%:%
%:%392=161%:%
%:%393=162%:%
%:%394=162%:%
%:%395=163%:%
%:%396=163%:%
%:%397=164%:%
%:%398=164%:%
%:%399=165%:%
%:%400=165%:%
%:%401=165%:%
%:%402=166%:%
%:%403=166%:%
%:%404=167%:%
%:%405=167%:%
%:%406=167%:%
%:%407=168%:%
%:%408=168%:%
%:%409=169%:%
%:%410=169%:%
%:%411=169%:%
%:%412=170%:%
%:%413=170%:%
%:%414=171%:%
%:%415=171%:%
%:%416=171%:%
%:%417=171%:%
%:%418=172%:%
%:%424=172%:%
%:%427=173%:%
%:%428=174%:%
%:%429=174%:%
%:%430=175%:%
%:%431=176%:%
%:%432=177%:%
%:%433=178%:%
%:%434=179%:%
%:%441=180%:%
%:%442=180%:%
%:%443=181%:%
%:%444=181%:%
%:%445=182%:%
%:%446=182%:%
%:%447=182%:%
%:%448=182%:%
%:%449=183%:%
%:%450=184%:%
%:%451=184%:%
%:%452=185%:%
%:%453=185%:%
%:%454=186%:%
%:%455=186%:%
%:%456=187%:%
%:%457=187%:%
%:%458=188%:%
%:%459=188%:%
%:%460=188%:%
%:%461=189%:%
%:%462=189%:%
%:%463=189%:%
%:%464=190%:%
%:%465=190%:%
%:%466=191%:%
%:%467=191%:%
%:%468=192%:%
%:%469=192%:%
%:%470=193%:%
%:%471=193%:%
%:%472=193%:%
%:%473=194%:%
%:%474=194%:%
%:%475=195%:%
%:%476=195%:%
%:%477=195%:%
%:%478=196%:%
%:%479=196%:%
%:%480=197%:%
%:%481=197%:%
%:%482=198%:%
%:%483=198%:%
%:%484=199%:%
%:%485=199%:%
%:%486=199%:%
%:%487=200%:%
%:%488=200%:%
%:%489=200%:%
%:%490=201%:%
%:%491=201%:%
%:%492=202%:%
%:%493=202%:%
%:%494=202%:%
%:%495=203%:%
%:%496=203%:%
%:%497=204%:%
%:%503=204%:%
%:%506=205%:%
%:%507=206%:%
%:%508=206%:%
%:%509=207%:%
%:%510=208%:%
%:%511=209%:%
%:%512=210%:%
%:%519=211%:%
%:%520=211%:%
%:%521=212%:%
%:%522=212%:%
%:%523=213%:%
%:%524=213%:%
%:%525=213%:%
%:%526=214%:%
%:%527=215%:%
%:%528=215%:%
%:%530=217%:%
%:%531=218%:%
%:%532=218%:%
%:%533=219%:%
%:%534=219%:%
%:%535=220%:%
%:%536=220%:%
%:%537=221%:%
%:%538=221%:%
%:%539=222%:%
%:%540=222%:%
%:%541=223%:%
%:%542=223%:%
%:%543=224%:%
%:%544=224%:%
%:%545=224%:%
%:%546=225%:%
%:%547=225%:%
%:%548=226%:%
%:%549=226%:%
%:%550=227%:%
%:%551=227%:%
%:%552=228%:%
%:%553=228%:%
%:%554=229%:%
%:%555=230%:%
%:%556=230%:%
%:%557=231%:%
%:%558=231%:%
%:%559=231%:%
%:%560=232%:%
%:%575=234%:%
%:%585=236%:%
%:%586=236%:%
%:%591=241%:%
%:%592=242%:%
%:%593=243%:%
%:%594=243%:%
%:%595=244%:%
%:%602=245%:%
%:%603=245%:%
%:%604=246%:%
%:%605=246%:%
%:%606=247%:%
%:%607=247%:%
%:%608=248%:%
%:%609=248%:%
%:%610=249%:%
%:%611=249%:%
%:%612=250%:%
%:%613=250%:%
%:%614=251%:%
%:%615=251%:%
%:%616=251%:%
%:%617=252%:%
%:%618=252%:%
%:%619=253%:%
%:%620=253%:%
%:%621=253%:%
%:%622=253%:%
%:%623=254%:%
%:%629=254%:%
%:%632=255%:%
%:%633=256%:%
%:%634=256%:%
%:%635=257%:%
%:%636=258%:%
%:%637=259%:%
%:%644=260%:%
%:%645=260%:%
%:%646=261%:%
%:%647=261%:%
%:%648=262%:%
%:%649=262%:%
%:%650=263%:%
%:%651=263%:%
%:%652=264%:%
%:%653=264%:%
%:%654=265%:%
%:%655=265%:%
%:%656=266%:%
%:%657=267%:%
%:%658=267%:%
%:%659=268%:%
%:%660=268%:%
%:%661=269%:%
%:%662=269%:%
%:%663=270%:%
%:%664=270%:%
%:%665=271%:%
%:%666=272%:%
%:%667=272%:%
%:%668=273%:%
%:%669=273%:%
%:%670=274%:%
%:%671=274%:%
%:%672=275%:%
%:%673=275%:%
%:%674=276%:%
%:%675=276%:%
%:%676=277%:%
%:%677=278%:%
%:%678=278%:%
%:%679=278%:%
%:%680=279%:%
%:%681=279%:%
%:%682=280%:%
%:%683=280%:%
%:%684=281%:%
%:%685=281%:%
%:%686=282%:%
%:%687=282%:%
%:%688=282%:%
%:%689=283%:%
%:%690=283%:%
%:%691=283%:%
%:%692=284%:%
%:%693=284%:%
%:%694=285%:%
%:%695=285%:%
%:%696=285%:%
%:%697=285%:%
%:%698=286%:%
%:%704=286%:%
%:%709=287%:%
%:%714=288%:%

%
\begin{isabellebody}%
\setisabellecontext{Landau{\isacharunderscore}{\kern0pt}Ext}%
%
\isadelimdocument
%
\endisadelimdocument
%
\isatagdocument
%
\isamarkupsection{Landau Symbols%
}
\isamarkuptrue%
%
\endisatagdocument
{\isafolddocument}%
%
\isadelimdocument
%
\endisadelimdocument
%
\isadelimtheory
%
\endisadelimtheory
%
\isatagtheory
\isacommand{theory}\isamarkupfalse%
\ Landau{\isacharunderscore}{\kern0pt}Ext\isanewline
\ \ \isakeyword{imports}\ {\isachardoublequoteopen}HOL{\isacharminus}{\kern0pt}Library{\isachardot}{\kern0pt}Landau{\isacharunderscore}{\kern0pt}Symbols{\isachardoublequoteclose}\ {\isachardoublequoteopen}HOL{\isachardot}{\kern0pt}Topological{\isacharunderscore}{\kern0pt}Spaces{\isachardoublequoteclose}\isanewline
\isakeyword{begin}%
\endisatagtheory
{\isafoldtheory}%
%
\isadelimtheory
%
\endisadelimtheory
%
\begin{isamarkuptext}%
This section contains results about Landau Symbols in addition to "HOL-Library.Landau".%
\end{isamarkuptext}\isamarkuptrue%
%
\begin{isamarkuptext}%
The following lemma is an intentional copy of \isa{sum{\isacharunderscore}{\kern0pt}in{\isacharunderscore}{\kern0pt}bigo} with order of assumptions reversed *)%
\end{isamarkuptext}\isamarkuptrue%
\isacommand{lemma}\isamarkupfalse%
\ sum{\isacharunderscore}{\kern0pt}in{\isacharunderscore}{\kern0pt}bigo{\isacharunderscore}{\kern0pt}r{\isacharcolon}{\kern0pt}\ \isanewline
\ \ \isakeyword{assumes}\ {\isachardoublequoteopen}f{\isadigit{2}}\ {\isasymin}\ O{\isacharbrackleft}{\kern0pt}F{\isacharprime}{\kern0pt}{\isacharbrackright}{\kern0pt}{\isacharparenleft}{\kern0pt}g{\isacharparenright}{\kern0pt}{\isachardoublequoteclose}\isanewline
\ \ \isakeyword{assumes}\ {\isachardoublequoteopen}f{\isadigit{1}}\ {\isasymin}\ O{\isacharbrackleft}{\kern0pt}F{\isacharprime}{\kern0pt}{\isacharbrackright}{\kern0pt}{\isacharparenleft}{\kern0pt}g{\isacharparenright}{\kern0pt}{\isachardoublequoteclose}\isanewline
\ \ \isakeyword{shows}\ {\isachardoublequoteopen}{\isacharparenleft}{\kern0pt}{\isasymlambda}x{\isachardot}{\kern0pt}\ f{\isadigit{1}}\ x\ {\isacharplus}{\kern0pt}\ f{\isadigit{2}}\ x{\isacharparenright}{\kern0pt}\ {\isasymin}\ O{\isacharbrackleft}{\kern0pt}F{\isacharprime}{\kern0pt}{\isacharbrackright}{\kern0pt}{\isacharparenleft}{\kern0pt}g{\isacharparenright}{\kern0pt}{\isachardoublequoteclose}\isanewline
%
\isadelimproof
\ \ %
\endisadelimproof
%
\isatagproof
\isacommand{by}\isamarkupfalse%
\ {\isacharparenleft}{\kern0pt}rule\ sum{\isacharunderscore}{\kern0pt}in{\isacharunderscore}{\kern0pt}bigo{\isacharbrackleft}{\kern0pt}OF\ assms{\isacharparenleft}{\kern0pt}{\isadigit{2}}{\isacharparenright}{\kern0pt}\ assms{\isacharparenleft}{\kern0pt}{\isadigit{1}}{\isacharparenright}{\kern0pt}{\isacharbrackright}{\kern0pt}{\isacharparenright}{\kern0pt}%
\endisatagproof
{\isafoldproof}%
%
\isadelimproof
\ \isanewline
%
\endisadelimproof
\isanewline
\isacommand{lemma}\isamarkupfalse%
\ landau{\isacharunderscore}{\kern0pt}sum{\isacharcolon}{\kern0pt}\isanewline
\ \ \isakeyword{assumes}\ {\isachardoublequoteopen}eventually\ {\isacharparenleft}{\kern0pt}{\isasymlambda}x{\isachardot}{\kern0pt}\ g{\isadigit{1}}\ x\ {\isasymge}\ {\isacharparenleft}{\kern0pt}{\isadigit{0}}{\isacharcolon}{\kern0pt}{\isacharcolon}{\kern0pt}real{\isacharparenright}{\kern0pt}{\isacharparenright}{\kern0pt}\ F{\isacharprime}{\kern0pt}{\isachardoublequoteclose}\ \isanewline
\ \ \isakeyword{assumes}\ {\isachardoublequoteopen}eventually\ {\isacharparenleft}{\kern0pt}{\isasymlambda}x{\isachardot}{\kern0pt}\ g{\isadigit{2}}\ x\ {\isasymge}\ {\isadigit{0}}{\isacharparenright}{\kern0pt}\ F{\isacharprime}{\kern0pt}{\isachardoublequoteclose}\ \isanewline
\ \ \isakeyword{assumes}\ {\isachardoublequoteopen}f{\isadigit{1}}\ {\isasymin}\ O{\isacharbrackleft}{\kern0pt}F{\isacharprime}{\kern0pt}{\isacharbrackright}{\kern0pt}{\isacharparenleft}{\kern0pt}g{\isadigit{1}}{\isacharparenright}{\kern0pt}{\isachardoublequoteclose}\isanewline
\ \ \isakeyword{assumes}\ {\isachardoublequoteopen}f{\isadigit{2}}\ {\isasymin}\ O{\isacharbrackleft}{\kern0pt}F{\isacharprime}{\kern0pt}{\isacharbrackright}{\kern0pt}{\isacharparenleft}{\kern0pt}g{\isadigit{2}}{\isacharparenright}{\kern0pt}{\isachardoublequoteclose}\isanewline
\ \ \isakeyword{shows}\ {\isachardoublequoteopen}{\isacharparenleft}{\kern0pt}{\isasymlambda}x{\isachardot}{\kern0pt}\ f{\isadigit{1}}\ x\ {\isacharplus}{\kern0pt}\ f{\isadigit{2}}\ x{\isacharparenright}{\kern0pt}\ {\isasymin}\ O{\isacharbrackleft}{\kern0pt}F{\isacharprime}{\kern0pt}{\isacharbrackright}{\kern0pt}{\isacharparenleft}{\kern0pt}{\isasymlambda}x{\isachardot}{\kern0pt}\ g{\isadigit{1}}\ x\ {\isacharplus}{\kern0pt}\ g{\isadigit{2}}\ x{\isacharparenright}{\kern0pt}{\isachardoublequoteclose}\isanewline
%
\isadelimproof
%
\endisadelimproof
%
\isatagproof
\isacommand{proof}\isamarkupfalse%
\ {\isacharminus}{\kern0pt}\isanewline
\ \ \isacommand{obtain}\isamarkupfalse%
\ c{\isadigit{1}}\ \isakeyword{where}\ a{\isadigit{1}}{\isacharcolon}{\kern0pt}\ {\isachardoublequoteopen}c{\isadigit{1}}\ {\isachargreater}{\kern0pt}\ {\isadigit{0}}{\isachardoublequoteclose}\ \isakeyword{and}\ b{\isadigit{1}}{\isacharcolon}{\kern0pt}\ {\isachardoublequoteopen}eventually\ {\isacharparenleft}{\kern0pt}{\isasymlambda}x{\isachardot}{\kern0pt}\ abs\ {\isacharparenleft}{\kern0pt}f{\isadigit{1}}\ x{\isacharparenright}{\kern0pt}\ {\isasymle}\ c{\isadigit{1}}\ {\isacharasterisk}{\kern0pt}\ abs\ {\isacharparenleft}{\kern0pt}g{\isadigit{1}}\ x{\isacharparenright}{\kern0pt}{\isacharparenright}{\kern0pt}\ F{\isacharprime}{\kern0pt}{\isachardoublequoteclose}\isanewline
\ \ \ \ \isacommand{using}\isamarkupfalse%
\ assms{\isacharparenleft}{\kern0pt}{\isadigit{3}}{\isacharparenright}{\kern0pt}\ \isacommand{by}\isamarkupfalse%
\ {\isacharparenleft}{\kern0pt}simp\ add{\isacharcolon}{\kern0pt}bigo{\isacharunderscore}{\kern0pt}def{\isacharcomma}{\kern0pt}\ blast{\isacharparenright}{\kern0pt}\isanewline
\ \ \isacommand{obtain}\isamarkupfalse%
\ c{\isadigit{2}}\ \isakeyword{where}\ a{\isadigit{2}}{\isacharcolon}{\kern0pt}\ {\isachardoublequoteopen}c{\isadigit{2}}\ {\isachargreater}{\kern0pt}\ {\isadigit{0}}{\isachardoublequoteclose}\ \isakeyword{and}\ b{\isadigit{2}}{\isacharcolon}{\kern0pt}\ {\isachardoublequoteopen}eventually\ {\isacharparenleft}{\kern0pt}{\isasymlambda}x{\isachardot}{\kern0pt}\ abs\ {\isacharparenleft}{\kern0pt}f{\isadigit{2}}\ x{\isacharparenright}{\kern0pt}\ {\isasymle}\ c{\isadigit{2}}\ {\isacharasterisk}{\kern0pt}\ abs\ {\isacharparenleft}{\kern0pt}g{\isadigit{2}}\ x{\isacharparenright}{\kern0pt}{\isacharparenright}{\kern0pt}\ F{\isacharprime}{\kern0pt}{\isachardoublequoteclose}\isanewline
\ \ \ \ \isacommand{using}\isamarkupfalse%
\ assms{\isacharparenleft}{\kern0pt}{\isadigit{4}}{\isacharparenright}{\kern0pt}\ \isacommand{by}\isamarkupfalse%
\ {\isacharparenleft}{\kern0pt}simp\ add{\isacharcolon}{\kern0pt}bigo{\isacharunderscore}{\kern0pt}def{\isacharcomma}{\kern0pt}\ blast{\isacharparenright}{\kern0pt}\isanewline
\ \ \isacommand{have}\isamarkupfalse%
\ {\isachardoublequoteopen}eventually\ {\isacharparenleft}{\kern0pt}{\isasymlambda}x{\isachardot}{\kern0pt}\ abs\ {\isacharparenleft}{\kern0pt}f{\isadigit{1}}\ x\ {\isacharplus}{\kern0pt}\ f{\isadigit{2}}\ x{\isacharparenright}{\kern0pt}\ {\isasymle}\ {\isacharparenleft}{\kern0pt}max\ c{\isadigit{1}}\ c{\isadigit{2}}{\isacharparenright}{\kern0pt}\ {\isacharasterisk}{\kern0pt}\ abs\ {\isacharparenleft}{\kern0pt}g{\isadigit{1}}\ x\ {\isacharplus}{\kern0pt}\ g{\isadigit{2}}\ x{\isacharparenright}{\kern0pt}{\isacharparenright}{\kern0pt}\ F{\isacharprime}{\kern0pt}{\isachardoublequoteclose}\isanewline
\ \ \isacommand{proof}\isamarkupfalse%
\ {\isacharparenleft}{\kern0pt}rule\ eventually{\isacharunderscore}{\kern0pt}mono{\isacharbrackleft}{\kern0pt}OF\ eventually{\isacharunderscore}{\kern0pt}conj{\isacharbrackleft}{\kern0pt}OF\ b{\isadigit{1}}\ eventually{\isacharunderscore}{\kern0pt}conj{\isacharbrackleft}{\kern0pt}OF\ b{\isadigit{2}}\ eventually{\isacharunderscore}{\kern0pt}conj{\isacharbrackleft}{\kern0pt}OF\ assms{\isacharparenleft}{\kern0pt}{\isadigit{1}}{\isacharparenright}{\kern0pt}\ assms{\isacharparenleft}{\kern0pt}{\isadigit{2}}{\isacharparenright}{\kern0pt}{\isacharbrackright}{\kern0pt}{\isacharbrackright}{\kern0pt}{\isacharbrackright}{\kern0pt}{\isacharbrackright}{\kern0pt}{\isacharparenright}{\kern0pt}\isanewline
\ \ \ \ \isacommand{fix}\isamarkupfalse%
\ x\isanewline
\ \ \ \ \isacommand{assume}\isamarkupfalse%
\ a{\isacharcolon}{\kern0pt}\ {\isachardoublequoteopen}{\isasymbar}f{\isadigit{1}}\ x{\isasymbar}\ {\isasymle}\ c{\isadigit{1}}\ {\isacharasterisk}{\kern0pt}\ {\isasymbar}g{\isadigit{1}}\ x{\isasymbar}\ {\isasymand}\ {\isasymbar}f{\isadigit{2}}\ x{\isasymbar}\ {\isasymle}\ c{\isadigit{2}}\ {\isacharasterisk}{\kern0pt}\ {\isasymbar}g{\isadigit{2}}\ x{\isasymbar}\ {\isasymand}\ {\isadigit{0}}\ {\isasymle}\ g{\isadigit{1}}\ x\ {\isasymand}\ {\isadigit{0}}\ {\isasymle}\ g{\isadigit{2}}\ x{\isachardoublequoteclose}\isanewline
\ \ \ \ \isacommand{have}\isamarkupfalse%
\ {\isachardoublequoteopen}{\isasymbar}f{\isadigit{1}}\ x\ {\isacharplus}{\kern0pt}\ f{\isadigit{2}}\ x{\isasymbar}\ {\isasymle}\ {\isasymbar}f{\isadigit{1}}\ x\ {\isasymbar}\ {\isacharplus}{\kern0pt}\ {\isasymbar}f{\isadigit{2}}\ x{\isasymbar}{\isachardoublequoteclose}\ \isacommand{using}\isamarkupfalse%
\ abs{\isacharunderscore}{\kern0pt}triangle{\isacharunderscore}{\kern0pt}ineq\ \isacommand{by}\isamarkupfalse%
\ blast\isanewline
\ \ \ \ \isacommand{also}\isamarkupfalse%
\ \isacommand{have}\isamarkupfalse%
\ {\isachardoublequoteopen}{\isachardot}{\kern0pt}{\isachardot}{\kern0pt}{\isachardot}{\kern0pt}\ {\isasymle}\ c{\isadigit{1}}\ {\isacharasterisk}{\kern0pt}\ \ {\isasymbar}g{\isadigit{1}}\ x{\isasymbar}\ {\isacharplus}{\kern0pt}\ c{\isadigit{2}}\ {\isacharasterisk}{\kern0pt}\ {\isasymbar}g{\isadigit{2}}\ x{\isasymbar}{\isachardoublequoteclose}\ \isacommand{using}\isamarkupfalse%
\ a\ add{\isacharunderscore}{\kern0pt}mono\ \isacommand{by}\isamarkupfalse%
\ blast\isanewline
\ \ \ \ \isacommand{also}\isamarkupfalse%
\ \isacommand{have}\isamarkupfalse%
\ {\isachardoublequoteopen}{\isachardot}{\kern0pt}{\isachardot}{\kern0pt}{\isachardot}{\kern0pt}\ {\isasymle}\ max\ c{\isadigit{1}}\ c{\isadigit{2}}\ {\isacharasterisk}{\kern0pt}\ {\isasymbar}g{\isadigit{1}}\ x{\isasymbar}\ {\isacharplus}{\kern0pt}\ max\ c{\isadigit{1}}\ c{\isadigit{2}}\ {\isacharasterisk}{\kern0pt}\ {\isasymbar}g{\isadigit{2}}\ x{\isasymbar}{\isachardoublequoteclose}\ \isanewline
\ \ \ \ \ \ \isacommand{apply}\isamarkupfalse%
\ {\isacharparenleft}{\kern0pt}rule\ add{\isacharunderscore}{\kern0pt}mono{\isacharparenright}{\kern0pt}\isanewline
\ \ \ \ \ \ \ \isacommand{apply}\isamarkupfalse%
\ {\isacharparenleft}{\kern0pt}rule\ mult{\isacharunderscore}{\kern0pt}right{\isacharunderscore}{\kern0pt}mono{\isacharcomma}{\kern0pt}\ simp{\isacharparenright}{\kern0pt}\isanewline
\ \ \ \ \ \ \ \isacommand{apply}\isamarkupfalse%
\ {\isacharparenleft}{\kern0pt}metis\ a\ a{\isadigit{1}}\ abs{\isacharunderscore}{\kern0pt}le{\isacharunderscore}{\kern0pt}zero{\isacharunderscore}{\kern0pt}iff\ abs{\isacharunderscore}{\kern0pt}zero\ linorder{\isacharunderscore}{\kern0pt}not{\isacharunderscore}{\kern0pt}less\ order{\isacharunderscore}{\kern0pt}trans\ semiring{\isacharunderscore}{\kern0pt}norm{\isacharparenleft}{\kern0pt}{\isadigit{6}}{\isadigit{3}}{\isacharparenright}{\kern0pt}\ zero{\isacharunderscore}{\kern0pt}le{\isacharunderscore}{\kern0pt}mult{\isacharunderscore}{\kern0pt}iff{\isacharparenright}{\kern0pt}\isanewline
\ \ \ \ \ \ \isacommand{apply}\isamarkupfalse%
\ {\isacharparenleft}{\kern0pt}rule\ mult{\isacharunderscore}{\kern0pt}right{\isacharunderscore}{\kern0pt}mono{\isacharcomma}{\kern0pt}\ simp{\isacharparenright}{\kern0pt}\isanewline
\ \ \ \ \ \ \isacommand{by}\isamarkupfalse%
\ {\isacharparenleft}{\kern0pt}metis\ a\ a{\isadigit{2}}\ abs{\isacharunderscore}{\kern0pt}le{\isacharunderscore}{\kern0pt}zero{\isacharunderscore}{\kern0pt}iff\ abs{\isacharunderscore}{\kern0pt}zero\ linorder{\isacharunderscore}{\kern0pt}not{\isacharunderscore}{\kern0pt}less\ order{\isacharunderscore}{\kern0pt}trans\ semiring{\isacharunderscore}{\kern0pt}norm{\isacharparenleft}{\kern0pt}{\isadigit{6}}{\isadigit{3}}{\isacharparenright}{\kern0pt}\ zero{\isacharunderscore}{\kern0pt}le{\isacharunderscore}{\kern0pt}mult{\isacharunderscore}{\kern0pt}iff{\isacharparenright}{\kern0pt}\isanewline
\ \ \ \ \isacommand{also}\isamarkupfalse%
\ \isacommand{have}\isamarkupfalse%
\ {\isachardoublequoteopen}{\isachardot}{\kern0pt}{\isachardot}{\kern0pt}{\isachardot}{\kern0pt}\ {\isasymle}\ max\ c{\isadigit{1}}\ c{\isadigit{2}}\ {\isacharasterisk}{\kern0pt}\ {\isacharparenleft}{\kern0pt}{\isasymbar}g{\isadigit{1}}\ x\ {\isacharplus}{\kern0pt}\ g{\isadigit{2}}\ x{\isasymbar}{\isacharparenright}{\kern0pt}{\isachardoublequoteclose}\isanewline
\ \ \ \ \ \ \isacommand{apply}\isamarkupfalse%
\ {\isacharparenleft}{\kern0pt}subst\ distrib{\isacharunderscore}{\kern0pt}left{\isacharbrackleft}{\kern0pt}symmetric{\isacharbrackright}{\kern0pt}{\isacharparenright}{\kern0pt}\isanewline
\ \ \ \ \ \ \isacommand{apply}\isamarkupfalse%
\ {\isacharparenleft}{\kern0pt}rule\ mult{\isacharunderscore}{\kern0pt}left{\isacharunderscore}{\kern0pt}mono{\isacharparenright}{\kern0pt}\isanewline
\ \ \ \ \ \ \isacommand{using}\isamarkupfalse%
\ a\ a{\isadigit{1}}\ a{\isadigit{2}}\ \isacommand{by}\isamarkupfalse%
\ auto\isanewline
\ \ \ \ \isacommand{finally}\isamarkupfalse%
\ \isacommand{show}\isamarkupfalse%
\ {\isachardoublequoteopen}{\isasymbar}f{\isadigit{1}}\ x\ {\isacharplus}{\kern0pt}\ f{\isadigit{2}}\ x{\isasymbar}\ {\isasymle}\ max\ c{\isadigit{1}}\ c{\isadigit{2}}\ {\isacharasterisk}{\kern0pt}\ {\isasymbar}g{\isadigit{1}}\ x\ {\isacharplus}{\kern0pt}\ g{\isadigit{2}}\ x{\isasymbar}{\isachardoublequoteclose}\ \isacommand{by}\isamarkupfalse%
\ {\isacharparenleft}{\kern0pt}simp\ add{\isacharcolon}{\kern0pt}algebra{\isacharunderscore}{\kern0pt}simps{\isacharparenright}{\kern0pt}\isanewline
\ \ \isacommand{qed}\isamarkupfalse%
\isanewline
\ \ \isacommand{thus}\isamarkupfalse%
\ {\isacharquery}{\kern0pt}thesis\isanewline
\ \ \ \ \isacommand{apply}\isamarkupfalse%
\ {\isacharparenleft}{\kern0pt}simp\ add{\isacharcolon}{\kern0pt}bigo{\isacharunderscore}{\kern0pt}def{\isacharparenright}{\kern0pt}\isanewline
\ \ \ \ \isacommand{apply}\isamarkupfalse%
\ {\isacharparenleft}{\kern0pt}rule\ exI{\isacharbrackleft}{\kern0pt}\isakeyword{where}\ x{\isacharequal}{\kern0pt}\ {\isachardoublequoteopen}max\ c{\isadigit{1}}\ c{\isadigit{2}}{\isachardoublequoteclose}{\isacharbrackright}{\kern0pt}{\isacharparenright}{\kern0pt}\isanewline
\ \ \ \ \isacommand{using}\isamarkupfalse%
\ a{\isadigit{1}}\ a{\isadigit{2}}\ \isacommand{by}\isamarkupfalse%
\ linarith\isanewline
\isacommand{qed}\isamarkupfalse%
%
\endisatagproof
{\isafoldproof}%
%
\isadelimproof
\isanewline
%
\endisadelimproof
\isanewline
\isacommand{lemma}\isamarkupfalse%
\ landau{\isacharunderscore}{\kern0pt}sum{\isacharunderscore}{\kern0pt}{\isadigit{1}}{\isacharcolon}{\kern0pt}\isanewline
\ \ \isakeyword{assumes}\ {\isachardoublequoteopen}eventually\ {\isacharparenleft}{\kern0pt}{\isasymlambda}x{\isachardot}{\kern0pt}\ g{\isadigit{1}}\ x\ {\isasymge}\ {\isacharparenleft}{\kern0pt}{\isadigit{0}}{\isacharcolon}{\kern0pt}{\isacharcolon}{\kern0pt}real{\isacharparenright}{\kern0pt}{\isacharparenright}{\kern0pt}\ F{\isacharprime}{\kern0pt}{\isachardoublequoteclose}\ \isanewline
\ \ \isakeyword{assumes}\ {\isachardoublequoteopen}eventually\ {\isacharparenleft}{\kern0pt}{\isasymlambda}x{\isachardot}{\kern0pt}\ g{\isadigit{2}}\ x\ {\isasymge}\ {\isadigit{0}}{\isacharparenright}{\kern0pt}\ F{\isacharprime}{\kern0pt}{\isachardoublequoteclose}\ \isanewline
\ \ \isakeyword{assumes}\ {\isachardoublequoteopen}f\ {\isasymin}\ O{\isacharbrackleft}{\kern0pt}F{\isacharprime}{\kern0pt}{\isacharbrackright}{\kern0pt}{\isacharparenleft}{\kern0pt}g{\isadigit{1}}{\isacharparenright}{\kern0pt}{\isachardoublequoteclose}\isanewline
\ \ \isakeyword{shows}\ {\isachardoublequoteopen}f\ {\isasymin}\ O{\isacharbrackleft}{\kern0pt}F{\isacharprime}{\kern0pt}{\isacharbrackright}{\kern0pt}{\isacharparenleft}{\kern0pt}{\isasymlambda}x{\isachardot}{\kern0pt}\ g{\isadigit{1}}\ x\ {\isacharplus}{\kern0pt}\ g{\isadigit{2}}\ x{\isacharparenright}{\kern0pt}{\isachardoublequoteclose}\isanewline
%
\isadelimproof
%
\endisadelimproof
%
\isatagproof
\isacommand{proof}\isamarkupfalse%
\ {\isacharminus}{\kern0pt}\isanewline
\ \ \isacommand{have}\isamarkupfalse%
\ {\isachardoublequoteopen}f\ {\isacharequal}{\kern0pt}\ {\isacharparenleft}{\kern0pt}{\isasymlambda}x{\isachardot}{\kern0pt}\ f\ x\ {\isacharplus}{\kern0pt}\ {\isadigit{0}}{\isacharparenright}{\kern0pt}{\isachardoublequoteclose}\isanewline
\ \ \ \ \isacommand{by}\isamarkupfalse%
\ simp\isanewline
\ \ \isacommand{also}\isamarkupfalse%
\ \isacommand{have}\isamarkupfalse%
\ {\isachardoublequoteopen}{\isachardot}{\kern0pt}{\isachardot}{\kern0pt}{\isachardot}{\kern0pt}\ {\isasymin}\ O{\isacharbrackleft}{\kern0pt}F{\isacharprime}{\kern0pt}{\isacharbrackright}{\kern0pt}{\isacharparenleft}{\kern0pt}{\isasymlambda}x{\isachardot}{\kern0pt}\ g{\isadigit{1}}\ x\ {\isacharplus}{\kern0pt}\ g{\isadigit{2}}\ x{\isacharparenright}{\kern0pt}{\isachardoublequoteclose}\isanewline
\ \ \ \ \isacommand{by}\isamarkupfalse%
\ {\isacharparenleft}{\kern0pt}rule\ landau{\isacharunderscore}{\kern0pt}sum{\isacharbrackleft}{\kern0pt}OF\ assms{\isacharparenleft}{\kern0pt}{\isadigit{1}}{\isacharparenright}{\kern0pt}\ assms{\isacharparenleft}{\kern0pt}{\isadigit{2}}{\isacharparenright}{\kern0pt}\ assms{\isacharparenleft}{\kern0pt}{\isadigit{3}}{\isacharparenright}{\kern0pt}\ zero{\isacharunderscore}{\kern0pt}in{\isacharunderscore}{\kern0pt}bigo{\isacharbrackright}{\kern0pt}{\isacharparenright}{\kern0pt}\isanewline
\ \ \isacommand{finally}\isamarkupfalse%
\ \isacommand{show}\isamarkupfalse%
\ {\isacharquery}{\kern0pt}thesis\ \isacommand{by}\isamarkupfalse%
\ simp\isanewline
\isacommand{qed}\isamarkupfalse%
%
\endisatagproof
{\isafoldproof}%
%
\isadelimproof
\isanewline
%
\endisadelimproof
\isanewline
\isacommand{lemma}\isamarkupfalse%
\ landau{\isacharunderscore}{\kern0pt}sum{\isacharunderscore}{\kern0pt}{\isadigit{2}}{\isacharcolon}{\kern0pt}\isanewline
\ \ \isakeyword{assumes}\ {\isachardoublequoteopen}eventually\ {\isacharparenleft}{\kern0pt}{\isasymlambda}x{\isachardot}{\kern0pt}\ g{\isadigit{1}}\ x\ {\isasymge}\ {\isacharparenleft}{\kern0pt}{\isadigit{0}}{\isacharcolon}{\kern0pt}{\isacharcolon}{\kern0pt}real{\isacharparenright}{\kern0pt}{\isacharparenright}{\kern0pt}\ F{\isacharprime}{\kern0pt}{\isachardoublequoteclose}\ \isanewline
\ \ \isakeyword{assumes}\ {\isachardoublequoteopen}eventually\ {\isacharparenleft}{\kern0pt}{\isasymlambda}x{\isachardot}{\kern0pt}\ g{\isadigit{2}}\ x\ {\isasymge}\ {\isadigit{0}}{\isacharparenright}{\kern0pt}\ F{\isacharprime}{\kern0pt}{\isachardoublequoteclose}\ \isanewline
\ \ \isakeyword{assumes}\ {\isachardoublequoteopen}f\ {\isasymin}\ O{\isacharbrackleft}{\kern0pt}F{\isacharprime}{\kern0pt}{\isacharbrackright}{\kern0pt}{\isacharparenleft}{\kern0pt}g{\isadigit{2}}{\isacharparenright}{\kern0pt}{\isachardoublequoteclose}\isanewline
\ \ \isakeyword{shows}\ {\isachardoublequoteopen}f\ {\isasymin}\ O{\isacharbrackleft}{\kern0pt}F{\isacharprime}{\kern0pt}{\isacharbrackright}{\kern0pt}{\isacharparenleft}{\kern0pt}{\isasymlambda}x{\isachardot}{\kern0pt}\ g{\isadigit{1}}\ x\ {\isacharplus}{\kern0pt}\ g{\isadigit{2}}\ x{\isacharparenright}{\kern0pt}{\isachardoublequoteclose}\isanewline
%
\isadelimproof
%
\endisadelimproof
%
\isatagproof
\isacommand{proof}\isamarkupfalse%
\ {\isacharminus}{\kern0pt}\isanewline
\ \ \isacommand{have}\isamarkupfalse%
\ {\isachardoublequoteopen}f\ {\isacharequal}{\kern0pt}\ {\isacharparenleft}{\kern0pt}{\isasymlambda}x{\isachardot}{\kern0pt}\ {\isadigit{0}}\ {\isacharplus}{\kern0pt}\ f\ x{\isacharparenright}{\kern0pt}{\isachardoublequoteclose}\isanewline
\ \ \ \ \isacommand{by}\isamarkupfalse%
\ simp\isanewline
\ \ \isacommand{also}\isamarkupfalse%
\ \isacommand{have}\isamarkupfalse%
\ {\isachardoublequoteopen}{\isachardot}{\kern0pt}{\isachardot}{\kern0pt}{\isachardot}{\kern0pt}\ {\isasymin}\ O{\isacharbrackleft}{\kern0pt}F{\isacharprime}{\kern0pt}{\isacharbrackright}{\kern0pt}{\isacharparenleft}{\kern0pt}{\isasymlambda}x{\isachardot}{\kern0pt}\ g{\isadigit{1}}\ x\ {\isacharplus}{\kern0pt}\ g{\isadigit{2}}\ x{\isacharparenright}{\kern0pt}{\isachardoublequoteclose}\isanewline
\ \ \ \ \isacommand{by}\isamarkupfalse%
\ {\isacharparenleft}{\kern0pt}rule\ landau{\isacharunderscore}{\kern0pt}sum{\isacharbrackleft}{\kern0pt}OF\ assms{\isacharparenleft}{\kern0pt}{\isadigit{1}}{\isacharparenright}{\kern0pt}\ assms{\isacharparenleft}{\kern0pt}{\isadigit{2}}{\isacharparenright}{\kern0pt}\ zero{\isacharunderscore}{\kern0pt}in{\isacharunderscore}{\kern0pt}bigo\ assms{\isacharparenleft}{\kern0pt}{\isadigit{3}}{\isacharparenright}{\kern0pt}{\isacharbrackright}{\kern0pt}{\isacharparenright}{\kern0pt}\isanewline
\ \ \isacommand{finally}\isamarkupfalse%
\ \isacommand{show}\isamarkupfalse%
\ {\isacharquery}{\kern0pt}thesis\ \isacommand{by}\isamarkupfalse%
\ simp\isanewline
\isacommand{qed}\isamarkupfalse%
%
\endisatagproof
{\isafoldproof}%
%
\isadelimproof
\isanewline
%
\endisadelimproof
\isanewline
\isacommand{lemma}\isamarkupfalse%
\ landau{\isacharunderscore}{\kern0pt}ln{\isacharunderscore}{\kern0pt}{\isadigit{3}}{\isacharcolon}{\kern0pt}\isanewline
\ \ \isakeyword{assumes}\ {\isachardoublequoteopen}eventually\ {\isacharparenleft}{\kern0pt}{\isasymlambda}x{\isachardot}{\kern0pt}\ {\isacharparenleft}{\kern0pt}{\isadigit{1}}{\isacharcolon}{\kern0pt}{\isacharcolon}{\kern0pt}real{\isacharparenright}{\kern0pt}\ {\isasymle}\ f\ x{\isacharparenright}{\kern0pt}\ F{\isacharprime}{\kern0pt}{\isachardoublequoteclose}\ \isanewline
\ \ \isakeyword{assumes}\ {\isachardoublequoteopen}f\ {\isasymin}\ O{\isacharbrackleft}{\kern0pt}F{\isacharprime}{\kern0pt}{\isacharbrackright}{\kern0pt}{\isacharparenleft}{\kern0pt}g{\isacharparenright}{\kern0pt}{\isachardoublequoteclose}\ \isanewline
\ \ \isakeyword{shows}\ {\isachardoublequoteopen}{\isacharparenleft}{\kern0pt}{\isasymlambda}x{\isachardot}{\kern0pt}\ ln\ {\isacharparenleft}{\kern0pt}f\ x{\isacharparenright}{\kern0pt}{\isacharparenright}{\kern0pt}\ {\isasymin}\ O{\isacharbrackleft}{\kern0pt}F{\isacharprime}{\kern0pt}{\isacharbrackright}{\kern0pt}{\isacharparenleft}{\kern0pt}g{\isacharparenright}{\kern0pt}{\isachardoublequoteclose}\ \isanewline
%
\isadelimproof
%
\endisadelimproof
%
\isatagproof
\isacommand{proof}\isamarkupfalse%
\ {\isacharminus}{\kern0pt}\isanewline
\ \ \isacommand{have}\isamarkupfalse%
\ a{\isacharcolon}{\kern0pt}{\isachardoublequoteopen}{\isacharparenleft}{\kern0pt}{\isasymlambda}x{\isachardot}{\kern0pt}\ ln\ {\isacharparenleft}{\kern0pt}f\ x{\isacharparenright}{\kern0pt}{\isacharparenright}{\kern0pt}\ {\isasymin}\ O{\isacharbrackleft}{\kern0pt}F{\isacharprime}{\kern0pt}{\isacharbrackright}{\kern0pt}{\isacharparenleft}{\kern0pt}f{\isacharparenright}{\kern0pt}{\isachardoublequoteclose}\isanewline
\ \ \ \ \isacommand{apply}\isamarkupfalse%
\ {\isacharparenleft}{\kern0pt}rule\ landau{\isacharunderscore}{\kern0pt}o{\isachardot}{\kern0pt}big{\isacharunderscore}{\kern0pt}mono{\isacharcomma}{\kern0pt}\ simp{\isacharparenright}{\kern0pt}\isanewline
\ \ \ \ \isacommand{apply}\isamarkupfalse%
\ {\isacharparenleft}{\kern0pt}rule\ eventually{\isacharunderscore}{\kern0pt}mono{\isacharbrackleft}{\kern0pt}OF\ assms{\isacharparenleft}{\kern0pt}{\isadigit{1}}{\isacharparenright}{\kern0pt}{\isacharbrackright}{\kern0pt}{\isacharparenright}{\kern0pt}\isanewline
\ \ \ \ \isacommand{apply}\isamarkupfalse%
\ {\isacharparenleft}{\kern0pt}subst\ abs{\isacharunderscore}{\kern0pt}of{\isacharunderscore}{\kern0pt}nonneg{\isacharcomma}{\kern0pt}\ subst\ ln{\isacharunderscore}{\kern0pt}ge{\isacharunderscore}{\kern0pt}zero{\isacharunderscore}{\kern0pt}iff{\isacharcomma}{\kern0pt}\ simp{\isacharcomma}{\kern0pt}\ simp{\isacharcomma}{\kern0pt}\ simp{\isacharparenright}{\kern0pt}\isanewline
\ \ \ \ \isacommand{using}\isamarkupfalse%
\ ln{\isacharunderscore}{\kern0pt}less{\isacharunderscore}{\kern0pt}self\ \isanewline
\ \ \ \ \isacommand{by}\isamarkupfalse%
\ {\isacharparenleft}{\kern0pt}meson\ ln{\isacharunderscore}{\kern0pt}bound\ order{\isachardot}{\kern0pt}strict{\isacharunderscore}{\kern0pt}trans{\isadigit{2}}\ zero{\isacharunderscore}{\kern0pt}less{\isacharunderscore}{\kern0pt}one{\isacharparenright}{\kern0pt}\isanewline
\ \ \isacommand{show}\isamarkupfalse%
\ {\isacharquery}{\kern0pt}thesis\isanewline
\ \ \ \ \isacommand{by}\isamarkupfalse%
\ {\isacharparenleft}{\kern0pt}rule\ landau{\isacharunderscore}{\kern0pt}o{\isachardot}{\kern0pt}big{\isacharunderscore}{\kern0pt}trans{\isacharbrackleft}{\kern0pt}OF\ a\ assms{\isacharparenleft}{\kern0pt}{\isadigit{2}}{\isacharparenright}{\kern0pt}{\isacharbrackright}{\kern0pt}{\isacharparenright}{\kern0pt}\isanewline
\isacommand{qed}\isamarkupfalse%
%
\endisatagproof
{\isafoldproof}%
%
\isadelimproof
\isanewline
%
\endisadelimproof
\isanewline
\isacommand{lemma}\isamarkupfalse%
\ landau{\isacharunderscore}{\kern0pt}ln{\isacharunderscore}{\kern0pt}{\isadigit{2}}{\isacharcolon}{\kern0pt}\isanewline
\ \ \isakeyword{assumes}\ {\isachardoublequoteopen}a\ {\isachargreater}{\kern0pt}\ {\isacharparenleft}{\kern0pt}{\isadigit{1}}{\isacharcolon}{\kern0pt}{\isacharcolon}{\kern0pt}real{\isacharparenright}{\kern0pt}{\isachardoublequoteclose}\isanewline
\ \ \isakeyword{assumes}\ {\isachardoublequoteopen}eventually\ {\isacharparenleft}{\kern0pt}{\isasymlambda}x{\isachardot}{\kern0pt}\ {\isadigit{1}}\ {\isasymle}\ f\ x{\isacharparenright}{\kern0pt}\ F{\isacharprime}{\kern0pt}{\isachardoublequoteclose}\ \isanewline
\ \ \isakeyword{assumes}\ {\isachardoublequoteopen}eventually\ {\isacharparenleft}{\kern0pt}{\isasymlambda}x{\isachardot}{\kern0pt}\ a\ {\isasymle}\ g\ x{\isacharparenright}{\kern0pt}\ F{\isacharprime}{\kern0pt}{\isachardoublequoteclose}\ \isanewline
\ \ \isakeyword{assumes}\ {\isachardoublequoteopen}f\ {\isasymin}\ O{\isacharbrackleft}{\kern0pt}F{\isacharprime}{\kern0pt}{\isacharbrackright}{\kern0pt}{\isacharparenleft}{\kern0pt}g{\isacharparenright}{\kern0pt}{\isachardoublequoteclose}\ \isanewline
\ \ \isakeyword{shows}\ {\isachardoublequoteopen}{\isacharparenleft}{\kern0pt}{\isasymlambda}x{\isachardot}{\kern0pt}\ ln\ {\isacharparenleft}{\kern0pt}f\ x{\isacharparenright}{\kern0pt}{\isacharparenright}{\kern0pt}\ {\isasymin}\ O{\isacharbrackleft}{\kern0pt}F{\isacharprime}{\kern0pt}{\isacharbrackright}{\kern0pt}{\isacharparenleft}{\kern0pt}{\isasymlambda}x{\isachardot}{\kern0pt}\ ln\ {\isacharparenleft}{\kern0pt}g\ x{\isacharparenright}{\kern0pt}{\isacharparenright}{\kern0pt}{\isachardoublequoteclose}\ \isanewline
%
\isadelimproof
%
\endisadelimproof
%
\isatagproof
\isacommand{proof}\isamarkupfalse%
\ {\isacharminus}{\kern0pt}\isanewline
\ \ \isacommand{obtain}\isamarkupfalse%
\ c\ \isakeyword{where}\ a{\isacharcolon}{\kern0pt}\ {\isachardoublequoteopen}c\ {\isachargreater}{\kern0pt}\ {\isadigit{0}}{\isachardoublequoteclose}\ \isakeyword{and}\ b{\isacharcolon}{\kern0pt}\ {\isachardoublequoteopen}eventually\ {\isacharparenleft}{\kern0pt}{\isasymlambda}x{\isachardot}{\kern0pt}\ abs\ {\isacharparenleft}{\kern0pt}f\ x{\isacharparenright}{\kern0pt}\ {\isasymle}\ c\ {\isacharasterisk}{\kern0pt}\ abs\ {\isacharparenleft}{\kern0pt}g\ x{\isacharparenright}{\kern0pt}{\isacharparenright}{\kern0pt}\ F{\isacharprime}{\kern0pt}{\isachardoublequoteclose}\isanewline
\ \ \ \ \isacommand{using}\isamarkupfalse%
\ assms{\isacharparenleft}{\kern0pt}{\isadigit{4}}{\isacharparenright}{\kern0pt}\ \isacommand{by}\isamarkupfalse%
\ {\isacharparenleft}{\kern0pt}simp\ add{\isacharcolon}{\kern0pt}bigo{\isacharunderscore}{\kern0pt}def{\isacharcomma}{\kern0pt}\ blast{\isacharparenright}{\kern0pt}\isanewline
\ \ \isacommand{define}\isamarkupfalse%
\ d\ \isakeyword{where}\ {\isachardoublequoteopen}d\ {\isacharequal}{\kern0pt}\ {\isadigit{1}}\ {\isacharplus}{\kern0pt}\ {\isacharparenleft}{\kern0pt}max\ {\isadigit{0}}\ {\isacharparenleft}{\kern0pt}ln\ c{\isacharparenright}{\kern0pt}{\isacharparenright}{\kern0pt}\ {\isacharslash}{\kern0pt}\ ln\ a{\isachardoublequoteclose}\isanewline
\ \ \isacommand{have}\isamarkupfalse%
\ d{\isacharcolon}{\kern0pt}{\isachardoublequoteopen}eventually\ {\isacharparenleft}{\kern0pt}{\isasymlambda}x{\isachardot}{\kern0pt}\ abs\ {\isacharparenleft}{\kern0pt}ln\ {\isacharparenleft}{\kern0pt}f\ x{\isacharparenright}{\kern0pt}{\isacharparenright}{\kern0pt}\ {\isasymle}\ d\ {\isacharasterisk}{\kern0pt}\ abs\ {\isacharparenleft}{\kern0pt}ln\ {\isacharparenleft}{\kern0pt}g\ x{\isacharparenright}{\kern0pt}{\isacharparenright}{\kern0pt}{\isacharparenright}{\kern0pt}\ F{\isacharprime}{\kern0pt}{\isachardoublequoteclose}\isanewline
\ \ \isacommand{proof}\isamarkupfalse%
\ {\isacharparenleft}{\kern0pt}rule\ eventually{\isacharunderscore}{\kern0pt}mono{\isacharbrackleft}{\kern0pt}OF\ eventually{\isacharunderscore}{\kern0pt}conj{\isacharbrackleft}{\kern0pt}OF\ b\ eventually{\isacharunderscore}{\kern0pt}conj{\isacharbrackleft}{\kern0pt}OF\ assms{\isacharparenleft}{\kern0pt}{\isadigit{3}}{\isacharparenright}{\kern0pt}\ assms{\isacharparenleft}{\kern0pt}{\isadigit{2}}{\isacharparenright}{\kern0pt}{\isacharbrackright}{\kern0pt}{\isacharbrackright}{\kern0pt}{\isacharbrackright}{\kern0pt}{\isacharparenright}{\kern0pt}\isanewline
\ \ \ \ \isacommand{fix}\isamarkupfalse%
\ x\isanewline
\ \ \ \ \isacommand{assume}\isamarkupfalse%
\ c{\isacharcolon}{\kern0pt}{\isachardoublequoteopen}{\isasymbar}f\ x{\isasymbar}\ {\isasymle}\ c\ {\isacharasterisk}{\kern0pt}\ {\isasymbar}g\ x{\isasymbar}\ {\isasymand}\ a\ {\isasymle}\ g\ x\ {\isasymand}\ {\isadigit{1}}\ {\isasymle}\ f\ x{\isachardoublequoteclose}\ \isanewline
\ \ \ \ \isacommand{have}\isamarkupfalse%
\ {\isachardoublequoteopen}abs\ {\isacharparenleft}{\kern0pt}ln\ {\isacharparenleft}{\kern0pt}f\ x{\isacharparenright}{\kern0pt}{\isacharparenright}{\kern0pt}\ {\isacharequal}{\kern0pt}\ ln\ {\isacharparenleft}{\kern0pt}f\ x{\isacharparenright}{\kern0pt}{\isachardoublequoteclose}\isanewline
\ \ \ \ \ \ \isacommand{by}\isamarkupfalse%
\ {\isacharparenleft}{\kern0pt}subst\ abs{\isacharunderscore}{\kern0pt}of{\isacharunderscore}{\kern0pt}nonneg{\isacharcomma}{\kern0pt}\ rule\ ln{\isacharunderscore}{\kern0pt}ge{\isacharunderscore}{\kern0pt}zero{\isacharcomma}{\kern0pt}\ metis\ c{\isacharcomma}{\kern0pt}\ simp{\isacharparenright}{\kern0pt}\isanewline
\ \ \ \ \isacommand{also}\isamarkupfalse%
\ \isacommand{have}\isamarkupfalse%
\ {\isachardoublequoteopen}{\isachardot}{\kern0pt}{\isachardot}{\kern0pt}{\isachardot}{\kern0pt}\ {\isasymle}\ ln\ {\isacharparenleft}{\kern0pt}c\ {\isacharasterisk}{\kern0pt}\ abs\ {\isacharparenleft}{\kern0pt}g\ x{\isacharparenright}{\kern0pt}{\isacharparenright}{\kern0pt}{\isachardoublequoteclose}\isanewline
\ \ \ \ \ \ \isacommand{apply}\isamarkupfalse%
\ {\isacharparenleft}{\kern0pt}subst\ ln{\isacharunderscore}{\kern0pt}le{\isacharunderscore}{\kern0pt}cancel{\isacharunderscore}{\kern0pt}iff{\isacharparenright}{\kern0pt}\ \isacommand{using}\isamarkupfalse%
\ c\ \isacommand{apply}\isamarkupfalse%
\ simp\isanewline
\ \ \ \ \ \ \ \isacommand{apply}\isamarkupfalse%
\ {\isacharparenleft}{\kern0pt}rule\ mult{\isacharunderscore}{\kern0pt}pos{\isacharunderscore}{\kern0pt}pos{\isacharbrackleft}{\kern0pt}OF\ a{\isacharbrackright}{\kern0pt}{\isacharparenright}{\kern0pt}\ \isacommand{using}\isamarkupfalse%
\ c\ assms{\isacharparenleft}{\kern0pt}{\isadigit{1}}{\isacharparenright}{\kern0pt}\ \isacommand{apply}\isamarkupfalse%
\ simp\isanewline
\ \ \ \ \ \ \isacommand{using}\isamarkupfalse%
\ c\ \isacommand{by}\isamarkupfalse%
\ linarith\isanewline
\ \ \ \ \isacommand{also}\isamarkupfalse%
\ \isacommand{have}\isamarkupfalse%
\ {\isachardoublequoteopen}{\isachardot}{\kern0pt}{\isachardot}{\kern0pt}{\isachardot}{\kern0pt}\ {\isasymle}\ ln\ c\ {\isacharplus}{\kern0pt}\ ln\ {\isacharparenleft}{\kern0pt}abs\ {\isacharparenleft}{\kern0pt}g\ x{\isacharparenright}{\kern0pt}{\isacharparenright}{\kern0pt}{\isachardoublequoteclose}\isanewline
\ \ \ \ \ \ \isacommand{apply}\isamarkupfalse%
\ {\isacharparenleft}{\kern0pt}subst\ ln{\isacharunderscore}{\kern0pt}mult{\isacharbrackleft}{\kern0pt}OF\ a{\isacharbrackright}{\kern0pt}{\isacharparenright}{\kern0pt}\isanewline
\ \ \ \ \ \ \isacommand{using}\isamarkupfalse%
\ c\ assms{\isacharparenleft}{\kern0pt}{\isadigit{1}}{\isacharparenright}{\kern0pt}\ \isacommand{by}\isamarkupfalse%
\ simp{\isacharplus}{\kern0pt}\isanewline
\ \ \ \ \isacommand{also}\isamarkupfalse%
\ \isacommand{have}\isamarkupfalse%
\ {\isachardoublequoteopen}{\isachardot}{\kern0pt}{\isachardot}{\kern0pt}{\isachardot}{\kern0pt}\ {\isasymle}\ {\isacharparenleft}{\kern0pt}d{\isacharminus}{\kern0pt}{\isadigit{1}}{\isacharparenright}{\kern0pt}{\isacharasterisk}{\kern0pt}ln\ a\ {\isacharplus}{\kern0pt}\ ln\ {\isacharparenleft}{\kern0pt}g\ x{\isacharparenright}{\kern0pt}{\isachardoublequoteclose}\isanewline
\ \ \ \ \ \ \isacommand{apply}\isamarkupfalse%
\ {\isacharparenleft}{\kern0pt}rule\ add{\isacharunderscore}{\kern0pt}mono{\isacharparenright}{\kern0pt}\isanewline
\ \ \ \ \ \ \isacommand{using}\isamarkupfalse%
\ assms{\isacharparenleft}{\kern0pt}{\isadigit{1}}{\isacharparenright}{\kern0pt}\ \isacommand{apply}\isamarkupfalse%
\ {\isacharparenleft}{\kern0pt}simp\ add{\isacharcolon}{\kern0pt}d{\isacharunderscore}{\kern0pt}def{\isacharparenright}{\kern0pt}\isanewline
\ \ \ \ \ \ \isacommand{apply}\isamarkupfalse%
\ {\isacharparenleft}{\kern0pt}subst\ abs{\isacharunderscore}{\kern0pt}of{\isacharunderscore}{\kern0pt}nonneg{\isacharparenright}{\kern0pt}\isanewline
\ \ \ \ \ \ \isacommand{using}\isamarkupfalse%
\ c\ assms{\isacharparenleft}{\kern0pt}{\isadigit{1}}{\isacharparenright}{\kern0pt}\ \isacommand{by}\isamarkupfalse%
\ simp{\isacharplus}{\kern0pt}\isanewline
\ \ \ \ \isacommand{also}\isamarkupfalse%
\ \isacommand{have}\isamarkupfalse%
\ {\isachardoublequoteopen}{\isachardot}{\kern0pt}{\isachardot}{\kern0pt}{\isachardot}{\kern0pt}\ {\isasymle}\ {\isacharparenleft}{\kern0pt}d{\isacharminus}{\kern0pt}{\isadigit{1}}{\isacharparenright}{\kern0pt}{\isacharasterisk}{\kern0pt}\ ln\ {\isacharparenleft}{\kern0pt}g\ x{\isacharparenright}{\kern0pt}\ {\isacharplus}{\kern0pt}\ ln\ {\isacharparenleft}{\kern0pt}g\ x{\isacharparenright}{\kern0pt}{\isachardoublequoteclose}\isanewline
\ \ \ \ \ \ \isacommand{apply}\isamarkupfalse%
\ {\isacharparenleft}{\kern0pt}rule\ add{\isacharunderscore}{\kern0pt}mono{\isacharparenright}{\kern0pt}\isanewline
\ \ \ \ \ \ \ \isacommand{apply}\isamarkupfalse%
\ {\isacharparenleft}{\kern0pt}rule\ mult{\isacharunderscore}{\kern0pt}left{\isacharunderscore}{\kern0pt}mono{\isacharparenright}{\kern0pt}\isanewline
\ \ \ \ \ \ \ \ \isacommand{apply}\isamarkupfalse%
\ {\isacharparenleft}{\kern0pt}subst\ ln{\isacharunderscore}{\kern0pt}le{\isacharunderscore}{\kern0pt}cancel{\isacharunderscore}{\kern0pt}iff{\isacharparenright}{\kern0pt}\isanewline
\ \ \ \ \ \ \isacommand{using}\isamarkupfalse%
\ assms{\isacharparenleft}{\kern0pt}{\isadigit{1}}{\isacharparenright}{\kern0pt}\ \isacommand{apply}\isamarkupfalse%
\ simp\isanewline
\ \ \ \ \ \ \isacommand{using}\isamarkupfalse%
\ c\ assms{\isacharparenleft}{\kern0pt}{\isadigit{1}}{\isacharparenright}{\kern0pt}\ \isacommand{apply}\isamarkupfalse%
\ simp\isanewline
\ \ \ \ \ \ \isacommand{using}\isamarkupfalse%
\ c\ assms{\isacharparenleft}{\kern0pt}{\isadigit{1}}{\isacharparenright}{\kern0pt}\ \isacommand{apply}\isamarkupfalse%
\ simp\isanewline
\ \ \ \ \ \ \ \isacommand{apply}\isamarkupfalse%
\ {\isacharparenleft}{\kern0pt}simp\ add{\isacharcolon}{\kern0pt}d{\isacharunderscore}{\kern0pt}def{\isacharparenright}{\kern0pt}\isanewline
\ \ \ \ \ \ \ \isacommand{apply}\isamarkupfalse%
\ {\isacharparenleft}{\kern0pt}rule\ divide{\isacharunderscore}{\kern0pt}nonneg{\isacharunderscore}{\kern0pt}nonneg{\isacharcomma}{\kern0pt}\ simp{\isacharcomma}{\kern0pt}\ rule\ ln{\isacharunderscore}{\kern0pt}ge{\isacharunderscore}{\kern0pt}zero{\isacharparenright}{\kern0pt}\ \isacommand{using}\isamarkupfalse%
\ assms{\isacharparenleft}{\kern0pt}{\isadigit{1}}{\isacharparenright}{\kern0pt}\ \isacommand{apply}\isamarkupfalse%
\ simp\isanewline
\ \ \ \ \ \ \isacommand{by}\isamarkupfalse%
\ simp\isanewline
\ \ \ \ \isacommand{also}\isamarkupfalse%
\ \isacommand{have}\isamarkupfalse%
\ {\isachardoublequoteopen}{\isachardot}{\kern0pt}{\isachardot}{\kern0pt}{\isachardot}{\kern0pt}\ {\isacharequal}{\kern0pt}\ d\ {\isacharasterisk}{\kern0pt}\ ln\ {\isacharparenleft}{\kern0pt}g\ x{\isacharparenright}{\kern0pt}{\isachardoublequoteclose}\ \isacommand{by}\isamarkupfalse%
\ {\isacharparenleft}{\kern0pt}simp\ add{\isacharcolon}{\kern0pt}algebra{\isacharunderscore}{\kern0pt}simps{\isacharparenright}{\kern0pt}\isanewline
\ \ \ \ \isacommand{also}\isamarkupfalse%
\ \isacommand{have}\isamarkupfalse%
\ {\isachardoublequoteopen}{\isachardot}{\kern0pt}{\isachardot}{\kern0pt}{\isachardot}{\kern0pt}\ {\isacharequal}{\kern0pt}\ d\ {\isacharasterisk}{\kern0pt}\ abs\ {\isacharparenleft}{\kern0pt}ln\ {\isacharparenleft}{\kern0pt}g\ x{\isacharparenright}{\kern0pt}{\isacharparenright}{\kern0pt}{\isachardoublequoteclose}\isanewline
\ \ \ \ \ \ \isacommand{apply}\isamarkupfalse%
\ {\isacharparenleft}{\kern0pt}subst\ abs{\isacharunderscore}{\kern0pt}of{\isacharunderscore}{\kern0pt}nonneg{\isacharparenright}{\kern0pt}\isanewline
\ \ \ \ \ \ \ \isacommand{apply}\isamarkupfalse%
\ {\isacharparenleft}{\kern0pt}rule\ ln{\isacharunderscore}{\kern0pt}ge{\isacharunderscore}{\kern0pt}zero{\isacharparenright}{\kern0pt}\ \isacommand{using}\isamarkupfalse%
\ c\ assms{\isacharparenleft}{\kern0pt}{\isadigit{1}}{\isacharparenright}{\kern0pt}\ \isacommand{by}\isamarkupfalse%
\ simp{\isacharplus}{\kern0pt}\isanewline
\ \ \ \ \isacommand{finally}\isamarkupfalse%
\ \isacommand{show}\isamarkupfalse%
\ {\isachardoublequoteopen}abs\ {\isacharparenleft}{\kern0pt}ln\ {\isacharparenleft}{\kern0pt}f\ x{\isacharparenright}{\kern0pt}{\isacharparenright}{\kern0pt}\ {\isasymle}\ d\ {\isacharasterisk}{\kern0pt}\ abs\ {\isacharparenleft}{\kern0pt}ln\ {\isacharparenleft}{\kern0pt}g\ x{\isacharparenright}{\kern0pt}{\isacharparenright}{\kern0pt}{\isachardoublequoteclose}\ \isacommand{by}\isamarkupfalse%
\ simp\isanewline
\ \ \isacommand{qed}\isamarkupfalse%
\isanewline
\ \ \isacommand{show}\isamarkupfalse%
\ {\isacharquery}{\kern0pt}thesis\isanewline
\ \ \ \ \isacommand{apply}\isamarkupfalse%
\ {\isacharparenleft}{\kern0pt}simp\ add{\isacharcolon}{\kern0pt}bigo{\isacharunderscore}{\kern0pt}def{\isacharparenright}{\kern0pt}\isanewline
\ \ \ \ \isacommand{apply}\isamarkupfalse%
\ {\isacharparenleft}{\kern0pt}rule\ exI{\isacharbrackleft}{\kern0pt}\isakeyword{where}\ x{\isacharequal}{\kern0pt}{\isachardoublequoteopen}d{\isachardoublequoteclose}{\isacharbrackright}{\kern0pt}{\isacharparenright}{\kern0pt}\isanewline
\ \ \ \ \isacommand{apply}\isamarkupfalse%
\ {\isacharparenleft}{\kern0pt}rule\ conjI{\isacharcomma}{\kern0pt}\ simp\ add{\isacharcolon}{\kern0pt}d{\isacharunderscore}{\kern0pt}def{\isacharparenright}{\kern0pt}\isanewline
\ \ \ \ \ \isacommand{apply}\isamarkupfalse%
\ {\isacharparenleft}{\kern0pt}meson\ add{\isacharunderscore}{\kern0pt}pos{\isacharunderscore}{\kern0pt}nonneg\ assms{\isacharparenleft}{\kern0pt}{\isadigit{1}}{\isacharparenright}{\kern0pt}\ less{\isacharunderscore}{\kern0pt}le{\isacharunderscore}{\kern0pt}not{\isacharunderscore}{\kern0pt}le\ less{\isacharunderscore}{\kern0pt}numeral{\isacharunderscore}{\kern0pt}extra{\isacharparenleft}{\kern0pt}{\isadigit{1}}{\isacharparenright}{\kern0pt}\ ln{\isacharunderscore}{\kern0pt}ge{\isacharunderscore}{\kern0pt}zero\ max{\isachardot}{\kern0pt}cobounded{\isadigit{1}}\ zero{\isacharunderscore}{\kern0pt}le{\isacharunderscore}{\kern0pt}divide{\isacharunderscore}{\kern0pt}iff{\isacharparenright}{\kern0pt}\isanewline
\ \ \ \ \isacommand{by}\isamarkupfalse%
\ {\isacharparenleft}{\kern0pt}metis\ d{\isacharparenright}{\kern0pt}\isanewline
\isacommand{qed}\isamarkupfalse%
%
\endisatagproof
{\isafoldproof}%
%
\isadelimproof
\isanewline
%
\endisadelimproof
\isanewline
\isacommand{lemma}\isamarkupfalse%
\ landau{\isacharunderscore}{\kern0pt}real{\isacharunderscore}{\kern0pt}nat{\isacharcolon}{\kern0pt}\isanewline
\ \ \isakeyword{fixes}\ f\ {\isacharcolon}{\kern0pt}{\isacharcolon}{\kern0pt}\ {\isachardoublequoteopen}{\isacharprime}{\kern0pt}a\ {\isasymRightarrow}\ int{\isachardoublequoteclose}\isanewline
\ \ \isakeyword{assumes}\ {\isachardoublequoteopen}{\isacharparenleft}{\kern0pt}{\isasymlambda}x{\isachardot}{\kern0pt}\ of{\isacharunderscore}{\kern0pt}int\ {\isacharparenleft}{\kern0pt}f\ x{\isacharparenright}{\kern0pt}{\isacharparenright}{\kern0pt}\ {\isasymin}\ O{\isacharbrackleft}{\kern0pt}F{\isacharprime}{\kern0pt}{\isacharbrackright}{\kern0pt}{\isacharparenleft}{\kern0pt}g{\isacharparenright}{\kern0pt}{\isachardoublequoteclose}\isanewline
\ \ \isakeyword{shows}\ {\isachardoublequoteopen}{\isacharparenleft}{\kern0pt}{\isasymlambda}x{\isachardot}{\kern0pt}\ real\ {\isacharparenleft}{\kern0pt}nat\ {\isacharparenleft}{\kern0pt}f\ x{\isacharparenright}{\kern0pt}{\isacharparenright}{\kern0pt}{\isacharparenright}{\kern0pt}\ {\isasymin}\ O{\isacharbrackleft}{\kern0pt}F{\isacharprime}{\kern0pt}{\isacharbrackright}{\kern0pt}{\isacharparenleft}{\kern0pt}g{\isacharparenright}{\kern0pt}{\isachardoublequoteclose}\isanewline
%
\isadelimproof
%
\endisadelimproof
%
\isatagproof
\isacommand{proof}\isamarkupfalse%
\ {\isacharminus}{\kern0pt}\isanewline
\ \ \isacommand{obtain}\isamarkupfalse%
\ c\ \isakeyword{where}\ a{\isacharcolon}{\kern0pt}\ {\isachardoublequoteopen}c\ {\isachargreater}{\kern0pt}\ {\isadigit{0}}{\isachardoublequoteclose}\ \isakeyword{and}\ b{\isacharcolon}{\kern0pt}\ {\isachardoublequoteopen}eventually\ {\isacharparenleft}{\kern0pt}{\isasymlambda}x{\isachardot}{\kern0pt}\ abs\ {\isacharparenleft}{\kern0pt}of{\isacharunderscore}{\kern0pt}int\ {\isacharparenleft}{\kern0pt}f\ x{\isacharparenright}{\kern0pt}{\isacharparenright}{\kern0pt}\ {\isasymle}\ c\ {\isacharasterisk}{\kern0pt}\ abs\ {\isacharparenleft}{\kern0pt}g\ x{\isacharparenright}{\kern0pt}{\isacharparenright}{\kern0pt}\ F{\isacharprime}{\kern0pt}{\isachardoublequoteclose}\isanewline
\ \ \ \ \isacommand{using}\isamarkupfalse%
\ assms{\isacharparenleft}{\kern0pt}{\isadigit{1}}{\isacharparenright}{\kern0pt}\ \isacommand{by}\isamarkupfalse%
\ {\isacharparenleft}{\kern0pt}simp\ add{\isacharcolon}{\kern0pt}bigo{\isacharunderscore}{\kern0pt}def{\isacharcomma}{\kern0pt}\ blast{\isacharparenright}{\kern0pt}\isanewline
\isanewline
\ \ \isacommand{show}\isamarkupfalse%
\ {\isacharquery}{\kern0pt}thesis\isanewline
\ \ \ \ \isacommand{apply}\isamarkupfalse%
\ {\isacharparenleft}{\kern0pt}simp\ add{\isacharcolon}{\kern0pt}bigo{\isacharunderscore}{\kern0pt}def{\isacharparenright}{\kern0pt}\isanewline
\ \ \ \ \isacommand{apply}\isamarkupfalse%
\ {\isacharparenleft}{\kern0pt}rule\ exI{\isacharbrackleft}{\kern0pt}\isakeyword{where}\ x{\isacharequal}{\kern0pt}{\isachardoublequoteopen}c{\isachardoublequoteclose}{\isacharbrackright}{\kern0pt}{\isacharparenright}{\kern0pt}\isanewline
\ \ \ \ \isacommand{apply}\isamarkupfalse%
\ {\isacharparenleft}{\kern0pt}rule\ conjI{\isacharbrackleft}{\kern0pt}OF\ a{\isacharbrackright}{\kern0pt}{\isacharparenright}{\kern0pt}\isanewline
\ \ \ \ \isacommand{apply}\isamarkupfalse%
\ {\isacharparenleft}{\kern0pt}rule\ eventually{\isacharunderscore}{\kern0pt}mono{\isacharbrackleft}{\kern0pt}OF\ b{\isacharbrackright}{\kern0pt}{\isacharparenright}{\kern0pt}\isanewline
\ \ \ \ \isacommand{by}\isamarkupfalse%
\ simp\isanewline
\isacommand{qed}\isamarkupfalse%
%
\endisatagproof
{\isafoldproof}%
%
\isadelimproof
\isanewline
%
\endisadelimproof
\isanewline
\isacommand{lemma}\isamarkupfalse%
\ landau{\isacharunderscore}{\kern0pt}ceil{\isacharcolon}{\kern0pt}\isanewline
\ \ \isakeyword{assumes}\ {\isachardoublequoteopen}{\isacharparenleft}{\kern0pt}{\isasymlambda}{\isacharunderscore}{\kern0pt}{\isachardot}{\kern0pt}\ {\isadigit{1}}{\isacharparenright}{\kern0pt}\ {\isasymin}\ O{\isacharbrackleft}{\kern0pt}F{\isacharprime}{\kern0pt}{\isacharbrackright}{\kern0pt}{\isacharparenleft}{\kern0pt}g{\isacharparenright}{\kern0pt}{\isachardoublequoteclose}\isanewline
\ \ \isakeyword{assumes}\ {\isachardoublequoteopen}f\ {\isasymin}\ O{\isacharbrackleft}{\kern0pt}F{\isacharprime}{\kern0pt}{\isacharbrackright}{\kern0pt}{\isacharparenleft}{\kern0pt}g{\isacharparenright}{\kern0pt}{\isachardoublequoteclose}\isanewline
\ \ \isakeyword{shows}\ {\isachardoublequoteopen}{\isacharparenleft}{\kern0pt}{\isasymlambda}x{\isachardot}{\kern0pt}\ real{\isacharunderscore}{\kern0pt}of{\isacharunderscore}{\kern0pt}int\ {\isasymlceil}f\ x{\isasymrceil}{\isacharparenright}{\kern0pt}\ {\isasymin}\ O{\isacharbrackleft}{\kern0pt}F{\isacharprime}{\kern0pt}{\isacharbrackright}{\kern0pt}{\isacharparenleft}{\kern0pt}g{\isacharparenright}{\kern0pt}{\isachardoublequoteclose}\isanewline
%
\isadelimproof
\ \ %
\endisadelimproof
%
\isatagproof
\isacommand{apply}\isamarkupfalse%
\ {\isacharparenleft}{\kern0pt}rule\ landau{\isacharunderscore}{\kern0pt}o{\isachardot}{\kern0pt}big{\isacharunderscore}{\kern0pt}trans{\isacharbrackleft}{\kern0pt}\isakeyword{where}\ g{\isacharequal}{\kern0pt}{\isachardoublequoteopen}{\isasymlambda}x{\isachardot}{\kern0pt}\ {\isadigit{1}}\ {\isacharplus}{\kern0pt}\ abs\ {\isacharparenleft}{\kern0pt}f\ x{\isacharparenright}{\kern0pt}{\isachardoublequoteclose}{\isacharbrackright}{\kern0pt}{\isacharparenright}{\kern0pt}\isanewline
\ \ \ \isacommand{apply}\isamarkupfalse%
\ {\isacharparenleft}{\kern0pt}rule\ landau{\isacharunderscore}{\kern0pt}o{\isachardot}{\kern0pt}big{\isacharunderscore}{\kern0pt}mono{\isacharparenright}{\kern0pt}\isanewline
\ \ \ \isacommand{apply}\isamarkupfalse%
\ {\isacharparenleft}{\kern0pt}rule\ always{\isacharunderscore}{\kern0pt}eventually{\isacharcomma}{\kern0pt}\ rule\ allI{\isacharcomma}{\kern0pt}\ simp{\isacharcomma}{\kern0pt}\ linarith{\isacharparenright}{\kern0pt}\isanewline
\ \ \isacommand{by}\isamarkupfalse%
\ {\isacharparenleft}{\kern0pt}rule\ sum{\isacharunderscore}{\kern0pt}in{\isacharunderscore}{\kern0pt}bigo{\isacharbrackleft}{\kern0pt}OF\ assms{\isacharparenleft}{\kern0pt}{\isadigit{1}}{\isacharparenright}{\kern0pt}{\isacharbrackright}{\kern0pt}{\isacharcomma}{\kern0pt}\ simp\ add{\isacharcolon}{\kern0pt}assms{\isacharparenright}{\kern0pt}%
\endisatagproof
{\isafoldproof}%
%
\isadelimproof
\isanewline
%
\endisadelimproof
\isanewline
\isacommand{lemma}\isamarkupfalse%
\ landau{\isacharunderscore}{\kern0pt}nat{\isacharunderscore}{\kern0pt}ceil{\isacharcolon}{\kern0pt}\isanewline
\ \ \isakeyword{assumes}\ {\isachardoublequoteopen}{\isacharparenleft}{\kern0pt}{\isasymlambda}{\isacharunderscore}{\kern0pt}{\isachardot}{\kern0pt}\ {\isadigit{1}}{\isacharparenright}{\kern0pt}\ {\isasymin}\ O{\isacharbrackleft}{\kern0pt}F{\isacharprime}{\kern0pt}{\isacharbrackright}{\kern0pt}{\isacharparenleft}{\kern0pt}g{\isacharparenright}{\kern0pt}{\isachardoublequoteclose}\isanewline
\ \ \isakeyword{assumes}\ {\isachardoublequoteopen}f\ {\isasymin}\ O{\isacharbrackleft}{\kern0pt}F{\isacharprime}{\kern0pt}{\isacharbrackright}{\kern0pt}{\isacharparenleft}{\kern0pt}g{\isacharparenright}{\kern0pt}{\isachardoublequoteclose}\isanewline
\ \ \isakeyword{shows}\ {\isachardoublequoteopen}{\isacharparenleft}{\kern0pt}{\isasymlambda}x{\isachardot}{\kern0pt}\ real\ {\isacharparenleft}{\kern0pt}nat\ {\isasymlceil}f\ x{\isasymrceil}{\isacharparenright}{\kern0pt}{\isacharparenright}{\kern0pt}\ {\isasymin}\ O{\isacharbrackleft}{\kern0pt}F{\isacharprime}{\kern0pt}{\isacharbrackright}{\kern0pt}{\isacharparenleft}{\kern0pt}g{\isacharparenright}{\kern0pt}{\isachardoublequoteclose}\isanewline
%
\isadelimproof
\ \ %
\endisadelimproof
%
\isatagproof
\isacommand{apply}\isamarkupfalse%
\ {\isacharparenleft}{\kern0pt}rule\ landau{\isacharunderscore}{\kern0pt}real{\isacharunderscore}{\kern0pt}nat{\isacharparenright}{\kern0pt}\isanewline
\ \ \isacommand{by}\isamarkupfalse%
\ {\isacharparenleft}{\kern0pt}rule\ landau{\isacharunderscore}{\kern0pt}ceil{\isacharbrackleft}{\kern0pt}OF\ assms{\isacharparenleft}{\kern0pt}{\isadigit{1}}{\isacharparenright}{\kern0pt}\ assms{\isacharparenleft}{\kern0pt}{\isadigit{2}}{\isacharparenright}{\kern0pt}{\isacharbrackright}{\kern0pt}{\isacharparenright}{\kern0pt}%
\endisatagproof
{\isafoldproof}%
%
\isadelimproof
\isanewline
%
\endisadelimproof
\isanewline
\isacommand{lemma}\isamarkupfalse%
\ landau{\isacharunderscore}{\kern0pt}const{\isacharunderscore}{\kern0pt}inv{\isacharcolon}{\kern0pt}\isanewline
\ \ \isakeyword{assumes}\ {\isachardoublequoteopen}c\ {\isachargreater}{\kern0pt}\ {\isacharparenleft}{\kern0pt}{\isadigit{0}}{\isacharcolon}{\kern0pt}{\isacharcolon}{\kern0pt}real{\isacharparenright}{\kern0pt}{\isachardoublequoteclose}\isanewline
\ \ \isakeyword{assumes}\ {\isachardoublequoteopen}{\isacharparenleft}{\kern0pt}{\isasymlambda}x{\isachardot}{\kern0pt}\ {\isadigit{1}}\ {\isacharslash}{\kern0pt}\ f\ x{\isacharparenright}{\kern0pt}\ {\isasymin}\ O{\isacharbrackleft}{\kern0pt}F{\isacharprime}{\kern0pt}{\isacharbrackright}{\kern0pt}{\isacharparenleft}{\kern0pt}g{\isacharparenright}{\kern0pt}{\isachardoublequoteclose}\isanewline
\ \ \isakeyword{shows}\ {\isachardoublequoteopen}{\isacharparenleft}{\kern0pt}{\isasymlambda}x{\isachardot}{\kern0pt}\ c\ {\isacharslash}{\kern0pt}\ f\ x{\isacharparenright}{\kern0pt}\ {\isasymin}\ O{\isacharbrackleft}{\kern0pt}F{\isacharprime}{\kern0pt}{\isacharbrackright}{\kern0pt}{\isacharparenleft}{\kern0pt}g{\isacharparenright}{\kern0pt}{\isachardoublequoteclose}\isanewline
%
\isadelimproof
%
\endisadelimproof
%
\isatagproof
\isacommand{proof}\isamarkupfalse%
\ {\isacharminus}{\kern0pt}\isanewline
\ \ \isacommand{obtain}\isamarkupfalse%
\ d\ \isakeyword{where}\ a{\isacharcolon}{\kern0pt}\ {\isachardoublequoteopen}d\ {\isachargreater}{\kern0pt}\ {\isadigit{0}}{\isachardoublequoteclose}\ \isakeyword{and}\ b{\isacharcolon}{\kern0pt}\ {\isachardoublequoteopen}eventually\ {\isacharparenleft}{\kern0pt}{\isasymlambda}x{\isachardot}{\kern0pt}\ abs\ {\isacharparenleft}{\kern0pt}{\isadigit{1}}\ {\isacharslash}{\kern0pt}\ f\ x{\isacharparenright}{\kern0pt}\ {\isasymle}\ d\ {\isacharasterisk}{\kern0pt}\ abs\ {\isacharparenleft}{\kern0pt}g\ x{\isacharparenright}{\kern0pt}{\isacharparenright}{\kern0pt}\ F{\isacharprime}{\kern0pt}{\isachardoublequoteclose}\isanewline
\ \ \ \ \isacommand{using}\isamarkupfalse%
\ assms{\isacharparenleft}{\kern0pt}{\isadigit{2}}{\isacharparenright}{\kern0pt}\ \isacommand{by}\isamarkupfalse%
\ {\isacharparenleft}{\kern0pt}simp\ add{\isacharcolon}{\kern0pt}bigo{\isacharunderscore}{\kern0pt}def{\isacharcomma}{\kern0pt}\ blast{\isacharparenright}{\kern0pt}\isanewline
\ \ \isacommand{have}\isamarkupfalse%
\ c{\isacharcolon}{\kern0pt}{\isachardoublequoteopen}eventually\ {\isacharparenleft}{\kern0pt}{\isasymlambda}x{\isachardot}{\kern0pt}\ {\isasymbar}c{\isasymbar}\ {\isacharslash}{\kern0pt}\ {\isasymbar}f\ x{\isasymbar}\ {\isasymle}\ {\isacharparenleft}{\kern0pt}c{\isacharparenright}{\kern0pt}{\isacharasterisk}{\kern0pt}d\ {\isacharasterisk}{\kern0pt}\ abs\ {\isacharparenleft}{\kern0pt}g\ x{\isacharparenright}{\kern0pt}{\isacharparenright}{\kern0pt}\ F{\isacharprime}{\kern0pt}{\isachardoublequoteclose}\isanewline
\ \ \ \ \isacommand{apply}\isamarkupfalse%
\ {\isacharparenleft}{\kern0pt}rule\ eventually{\isacharunderscore}{\kern0pt}mono{\isacharbrackleft}{\kern0pt}OF\ b{\isacharbrackright}{\kern0pt}{\isacharparenright}{\kern0pt}\isanewline
\ \ \ \ \isacommand{using}\isamarkupfalse%
\ assms{\isacharparenleft}{\kern0pt}{\isadigit{1}}{\isacharparenright}{\kern0pt}\isanewline
\ \ \ \ \isacommand{apply}\isamarkupfalse%
\ simp\ \isanewline
\ \ \ \ \isacommand{by}\isamarkupfalse%
\ {\isacharparenleft}{\kern0pt}metis\ Groups{\isachardot}{\kern0pt}mult{\isacharunderscore}{\kern0pt}ac{\isacharparenleft}{\kern0pt}{\isadigit{2}}{\isacharparenright}{\kern0pt}\ Groups{\isachardot}{\kern0pt}mult{\isacharunderscore}{\kern0pt}ac{\isacharparenleft}{\kern0pt}{\isadigit{3}}{\isacharparenright}{\kern0pt}\ divide{\isacharunderscore}{\kern0pt}inverse\ inverse{\isacharunderscore}{\kern0pt}eq{\isacharunderscore}{\kern0pt}divide\ less{\isacharunderscore}{\kern0pt}imp{\isacharunderscore}{\kern0pt}le\ mult{\isacharunderscore}{\kern0pt}le{\isacharunderscore}{\kern0pt}cancel{\isacharunderscore}{\kern0pt}left\ not{\isacharunderscore}{\kern0pt}less{\isacharparenright}{\kern0pt}\isanewline
\ \ \isacommand{show}\isamarkupfalse%
\ {\isacharquery}{\kern0pt}thesis\isanewline
\ \ \ \ \isacommand{apply}\isamarkupfalse%
\ {\isacharparenleft}{\kern0pt}simp\ add{\isacharcolon}{\kern0pt}bigo{\isacharunderscore}{\kern0pt}def{\isacharparenright}{\kern0pt}\isanewline
\ \ \ \ \isacommand{apply}\isamarkupfalse%
\ {\isacharparenleft}{\kern0pt}rule\ exI{\isacharbrackleft}{\kern0pt}\isakeyword{where}\ x{\isacharequal}{\kern0pt}{\isachardoublequoteopen}c{\isacharasterisk}{\kern0pt}d{\isachardoublequoteclose}{\isacharbrackright}{\kern0pt}{\isacharparenright}{\kern0pt}\isanewline
\ \ \ \ \isacommand{apply}\isamarkupfalse%
\ {\isacharparenleft}{\kern0pt}rule\ conjI{\isacharcomma}{\kern0pt}\ rule\ mult{\isacharunderscore}{\kern0pt}pos{\isacharunderscore}{\kern0pt}pos{\isacharbrackleft}{\kern0pt}OF\ assms{\isacharparenleft}{\kern0pt}{\isadigit{1}}{\isacharparenright}{\kern0pt}\ a{\isacharbrackright}{\kern0pt}{\isacharparenright}{\kern0pt}\isanewline
\ \ \ \ \isacommand{by}\isamarkupfalse%
\ {\isacharparenleft}{\kern0pt}rule\ c{\isacharparenright}{\kern0pt}\isanewline
\isacommand{qed}\isamarkupfalse%
%
\endisatagproof
{\isafoldproof}%
%
\isadelimproof
\isanewline
%
\endisadelimproof
\isanewline
\isacommand{lemma}\isamarkupfalse%
\ eventually{\isacharunderscore}{\kern0pt}nonneg{\isacharunderscore}{\kern0pt}div{\isacharcolon}{\kern0pt}\isanewline
\ \ \isakeyword{assumes}\ {\isachardoublequoteopen}eventually\ {\isacharparenleft}{\kern0pt}{\isasymlambda}x{\isachardot}{\kern0pt}\ {\isacharparenleft}{\kern0pt}{\isadigit{0}}{\isacharcolon}{\kern0pt}{\isacharcolon}{\kern0pt}real{\isacharparenright}{\kern0pt}\ {\isasymle}\ f\ x{\isacharparenright}{\kern0pt}\ F{\isacharprime}{\kern0pt}{\isachardoublequoteclose}\isanewline
\ \ \isakeyword{assumes}\ {\isachardoublequoteopen}eventually\ {\isacharparenleft}{\kern0pt}{\isasymlambda}x{\isachardot}{\kern0pt}\ {\isadigit{0}}\ {\isacharless}{\kern0pt}\ g\ x{\isacharparenright}{\kern0pt}\ F{\isacharprime}{\kern0pt}{\isachardoublequoteclose}\isanewline
\ \ \isakeyword{shows}\ {\isachardoublequoteopen}eventually\ {\isacharparenleft}{\kern0pt}{\isasymlambda}x{\isachardot}{\kern0pt}\ {\isadigit{0}}\ {\isasymle}\ f\ x\ {\isacharslash}{\kern0pt}\ g\ x{\isacharparenright}{\kern0pt}\ F{\isacharprime}{\kern0pt}{\isachardoublequoteclose}\isanewline
%
\isadelimproof
\ \ %
\endisadelimproof
%
\isatagproof
\isacommand{apply}\isamarkupfalse%
\ {\isacharparenleft}{\kern0pt}rule\ eventually{\isacharunderscore}{\kern0pt}mono{\isacharbrackleft}{\kern0pt}OF\ eventually{\isacharunderscore}{\kern0pt}conj{\isacharbrackleft}{\kern0pt}OF\ assms{\isacharparenleft}{\kern0pt}{\isadigit{1}}{\isacharparenright}{\kern0pt}\ assms{\isacharparenleft}{\kern0pt}{\isadigit{2}}{\isacharparenright}{\kern0pt}{\isacharbrackright}{\kern0pt}{\isacharbrackright}{\kern0pt}{\isacharparenright}{\kern0pt}\isanewline
\ \ \isacommand{by}\isamarkupfalse%
\ simp%
\endisatagproof
{\isafoldproof}%
%
\isadelimproof
\isanewline
%
\endisadelimproof
\isanewline
\isacommand{lemma}\isamarkupfalse%
\ eventually{\isacharunderscore}{\kern0pt}nonneg{\isacharunderscore}{\kern0pt}add{\isacharcolon}{\kern0pt}\isanewline
\ \ \isakeyword{assumes}\ {\isachardoublequoteopen}eventually\ {\isacharparenleft}{\kern0pt}{\isasymlambda}x{\isachardot}{\kern0pt}\ {\isacharparenleft}{\kern0pt}{\isadigit{0}}{\isacharcolon}{\kern0pt}{\isacharcolon}{\kern0pt}real{\isacharparenright}{\kern0pt}\ {\isasymle}\ f\ x{\isacharparenright}{\kern0pt}\ F{\isacharprime}{\kern0pt}{\isachardoublequoteclose}\isanewline
\ \ \isakeyword{assumes}\ {\isachardoublequoteopen}eventually\ {\isacharparenleft}{\kern0pt}{\isasymlambda}x{\isachardot}{\kern0pt}\ {\isadigit{0}}\ {\isasymle}\ g\ x{\isacharparenright}{\kern0pt}\ F{\isacharprime}{\kern0pt}{\isachardoublequoteclose}\isanewline
\ \ \isakeyword{shows}\ {\isachardoublequoteopen}eventually\ {\isacharparenleft}{\kern0pt}{\isasymlambda}x{\isachardot}{\kern0pt}\ {\isadigit{0}}\ {\isasymle}\ f\ x\ {\isacharplus}{\kern0pt}\ g\ x{\isacharparenright}{\kern0pt}\ F{\isacharprime}{\kern0pt}{\isachardoublequoteclose}\isanewline
%
\isadelimproof
\ \ %
\endisadelimproof
%
\isatagproof
\isacommand{apply}\isamarkupfalse%
\ {\isacharparenleft}{\kern0pt}rule\ eventually{\isacharunderscore}{\kern0pt}mono{\isacharbrackleft}{\kern0pt}OF\ eventually{\isacharunderscore}{\kern0pt}conj{\isacharbrackleft}{\kern0pt}OF\ assms{\isacharparenleft}{\kern0pt}{\isadigit{1}}{\isacharparenright}{\kern0pt}\ assms{\isacharparenleft}{\kern0pt}{\isadigit{2}}{\isacharparenright}{\kern0pt}{\isacharbrackright}{\kern0pt}{\isacharbrackright}{\kern0pt}{\isacharparenright}{\kern0pt}\isanewline
\ \ \isacommand{by}\isamarkupfalse%
\ simp%
\endisatagproof
{\isafoldproof}%
%
\isadelimproof
\isanewline
%
\endisadelimproof
\isanewline
\isacommand{lemma}\isamarkupfalse%
\ eventually{\isacharunderscore}{\kern0pt}ln{\isacharunderscore}{\kern0pt}ge{\isacharunderscore}{\kern0pt}iff{\isacharcolon}{\kern0pt}\isanewline
\ \ \isakeyword{assumes}\ {\isachardoublequoteopen}eventually\ {\isacharparenleft}{\kern0pt}{\isasymlambda}x{\isachardot}{\kern0pt}\ {\isacharparenleft}{\kern0pt}exp\ {\isacharparenleft}{\kern0pt}c{\isacharcolon}{\kern0pt}{\isacharcolon}{\kern0pt}real{\isacharparenright}{\kern0pt}{\isacharparenright}{\kern0pt}\ {\isasymle}\ f\ x{\isacharparenright}{\kern0pt}\ F{\isacharprime}{\kern0pt}{\isachardoublequoteclose}\isanewline
\ \ \isakeyword{shows}\ {\isachardoublequoteopen}eventually\ {\isacharparenleft}{\kern0pt}{\isasymlambda}x{\isachardot}{\kern0pt}\ c\ {\isasymle}\ ln\ {\isacharparenleft}{\kern0pt}f\ x{\isacharparenright}{\kern0pt}{\isacharparenright}{\kern0pt}\ F{\isacharprime}{\kern0pt}{\isachardoublequoteclose}\isanewline
%
\isadelimproof
\ \ %
\endisadelimproof
%
\isatagproof
\isacommand{apply}\isamarkupfalse%
\ {\isacharparenleft}{\kern0pt}rule\ eventually{\isacharunderscore}{\kern0pt}mono{\isacharbrackleft}{\kern0pt}OF\ assms{\isacharparenleft}{\kern0pt}{\isadigit{1}}{\isacharparenright}{\kern0pt}{\isacharbrackright}{\kern0pt}{\isacharparenright}{\kern0pt}\isanewline
\ \ \isacommand{by}\isamarkupfalse%
\ {\isacharparenleft}{\kern0pt}meson\ ln{\isacharunderscore}{\kern0pt}ge{\isacharunderscore}{\kern0pt}iff\ \ exp{\isacharunderscore}{\kern0pt}gt{\isacharunderscore}{\kern0pt}zero\ order{\isacharunderscore}{\kern0pt}less{\isacharunderscore}{\kern0pt}le{\isacharunderscore}{\kern0pt}trans{\isacharparenright}{\kern0pt}%
\endisatagproof
{\isafoldproof}%
%
\isadelimproof
\isanewline
%
\endisadelimproof
\isanewline
\isacommand{lemma}\isamarkupfalse%
\ div{\isacharunderscore}{\kern0pt}commute{\isacharcolon}{\kern0pt}\ {\isachardoublequoteopen}{\isacharparenleft}{\kern0pt}a{\isacharcolon}{\kern0pt}{\isacharcolon}{\kern0pt}real{\isacharparenright}{\kern0pt}\ {\isacharslash}{\kern0pt}\ b\ {\isacharequal}{\kern0pt}\ {\isacharparenleft}{\kern0pt}{\isadigit{1}}{\isacharslash}{\kern0pt}b{\isacharparenright}{\kern0pt}\ {\isacharasterisk}{\kern0pt}\ a{\isachardoublequoteclose}%
\isadelimproof
\ %
\endisadelimproof
%
\isatagproof
\isacommand{by}\isamarkupfalse%
\ simp%
\endisatagproof
{\isafoldproof}%
%
\isadelimproof
%
\endisadelimproof
\isanewline
\isanewline
\isacommand{lemma}\isamarkupfalse%
\ eventually{\isacharunderscore}{\kern0pt}prod{\isadigit{1}}{\isacharprime}{\kern0pt}{\isacharcolon}{\kern0pt}\isanewline
\ \ \isakeyword{assumes}\ {\isachardoublequoteopen}B\ {\isasymnoteq}\ bot{\isachardoublequoteclose}\isanewline
\ \ \isakeyword{shows}\ {\isachardoublequoteopen}{\isacharparenleft}{\kern0pt}{\isasymforall}\isactrlsub F\ x\ in\ A\ {\isasymtimes}\isactrlsub F\ B{\isachardot}{\kern0pt}\ P\ {\isacharparenleft}{\kern0pt}fst\ x{\isacharparenright}{\kern0pt}{\isacharparenright}{\kern0pt}\ {\isasymlongleftrightarrow}\ {\isacharparenleft}{\kern0pt}{\isasymforall}\isactrlsub F\ x\ in\ A{\isachardot}{\kern0pt}\ P\ x{\isacharparenright}{\kern0pt}{\isachardoublequoteclose}\isanewline
%
\isadelimproof
\ \ %
\endisadelimproof
%
\isatagproof
\isacommand{apply}\isamarkupfalse%
\ {\isacharparenleft}{\kern0pt}subst\ {\isacharparenleft}{\kern0pt}{\isadigit{2}}{\isacharparenright}{\kern0pt}\ eventually{\isacharunderscore}{\kern0pt}prod{\isadigit{1}}{\isacharbrackleft}{\kern0pt}OF\ assms{\isacharparenleft}{\kern0pt}{\isadigit{1}}{\isacharparenright}{\kern0pt}{\isacharcomma}{\kern0pt}\ symmetric{\isacharbrackright}{\kern0pt}{\isacharparenright}{\kern0pt}\isanewline
\ \ \isacommand{apply}\isamarkupfalse%
\ {\isacharparenleft}{\kern0pt}rule\ arg{\isacharunderscore}{\kern0pt}cong{\isadigit{2}}{\isacharbrackleft}{\kern0pt}\isakeyword{where}\ f{\isacharequal}{\kern0pt}{\isachardoublequoteopen}eventually{\isachardoublequoteclose}{\isacharbrackright}{\kern0pt}{\isacharparenright}{\kern0pt}\isanewline
\ \ \isacommand{by}\isamarkupfalse%
\ {\isacharparenleft}{\kern0pt}rule\ ext{\isacharcomma}{\kern0pt}\ simp\ add{\isacharcolon}{\kern0pt}case{\isacharunderscore}{\kern0pt}prod{\isacharunderscore}{\kern0pt}beta{\isacharcomma}{\kern0pt}\ simp{\isacharparenright}{\kern0pt}%
\endisatagproof
{\isafoldproof}%
%
\isadelimproof
\isanewline
%
\endisadelimproof
\isanewline
\isacommand{lemma}\isamarkupfalse%
\ eventually{\isacharunderscore}{\kern0pt}prod{\isadigit{2}}{\isacharprime}{\kern0pt}{\isacharcolon}{\kern0pt}\isanewline
\ \ \isakeyword{assumes}\ {\isachardoublequoteopen}A\ {\isasymnoteq}\ bot{\isachardoublequoteclose}\isanewline
\ \ \isakeyword{shows}\ {\isachardoublequoteopen}{\isacharparenleft}{\kern0pt}{\isasymforall}\isactrlsub F\ x\ in\ A\ {\isasymtimes}\isactrlsub F\ B{\isachardot}{\kern0pt}\ P\ {\isacharparenleft}{\kern0pt}snd\ x{\isacharparenright}{\kern0pt}{\isacharparenright}{\kern0pt}\ {\isasymlongleftrightarrow}\ {\isacharparenleft}{\kern0pt}{\isasymforall}\isactrlsub F\ x\ in\ B{\isachardot}{\kern0pt}\ P\ x{\isacharparenright}{\kern0pt}{\isachardoublequoteclose}\isanewline
%
\isadelimproof
\ \ %
\endisadelimproof
%
\isatagproof
\isacommand{apply}\isamarkupfalse%
\ {\isacharparenleft}{\kern0pt}subst\ {\isacharparenleft}{\kern0pt}{\isadigit{2}}{\isacharparenright}{\kern0pt}\ eventually{\isacharunderscore}{\kern0pt}prod{\isadigit{2}}{\isacharbrackleft}{\kern0pt}OF\ assms{\isacharparenleft}{\kern0pt}{\isadigit{1}}{\isacharparenright}{\kern0pt}{\isacharcomma}{\kern0pt}\ symmetric{\isacharbrackright}{\kern0pt}{\isacharparenright}{\kern0pt}\isanewline
\ \ \isacommand{apply}\isamarkupfalse%
\ {\isacharparenleft}{\kern0pt}rule\ arg{\isacharunderscore}{\kern0pt}cong{\isadigit{2}}{\isacharbrackleft}{\kern0pt}\isakeyword{where}\ f{\isacharequal}{\kern0pt}{\isachardoublequoteopen}eventually{\isachardoublequoteclose}{\isacharbrackright}{\kern0pt}{\isacharparenright}{\kern0pt}\isanewline
\ \ \isacommand{by}\isamarkupfalse%
\ {\isacharparenleft}{\kern0pt}rule\ ext{\isacharcomma}{\kern0pt}\ simp\ add{\isacharcolon}{\kern0pt}case{\isacharunderscore}{\kern0pt}prod{\isacharunderscore}{\kern0pt}beta{\isacharcomma}{\kern0pt}\ simp{\isacharparenright}{\kern0pt}%
\endisatagproof
{\isafoldproof}%
%
\isadelimproof
\isanewline
%
\endisadelimproof
\isanewline
\isacommand{instantiation}\isamarkupfalse%
\ rat\ {\isacharcolon}{\kern0pt}{\isacharcolon}{\kern0pt}\ linorder{\isacharunderscore}{\kern0pt}topology\isanewline
\isakeyword{begin}\isanewline
\isanewline
\isacommand{definition}\isamarkupfalse%
\ open{\isacharunderscore}{\kern0pt}rat\ {\isacharcolon}{\kern0pt}{\isacharcolon}{\kern0pt}\ {\isachardoublequoteopen}rat\ set\ {\isasymRightarrow}\ bool{\isachardoublequoteclose}\isanewline
\ \ \isakeyword{where}\ {\isachardoublequoteopen}open{\isacharunderscore}{\kern0pt}rat\ {\isacharequal}{\kern0pt}\ generate{\isacharunderscore}{\kern0pt}topology\ {\isacharparenleft}{\kern0pt}range\ {\isacharparenleft}{\kern0pt}{\isasymlambda}a{\isachardot}{\kern0pt}\ {\isacharbraceleft}{\kern0pt}{\isachardot}{\kern0pt}{\isachardot}{\kern0pt}{\isacharless}{\kern0pt}\ a{\isacharbraceright}{\kern0pt}{\isacharparenright}{\kern0pt}\ {\isasymunion}\ range\ {\isacharparenleft}{\kern0pt}{\isasymlambda}a{\isachardot}{\kern0pt}\ {\isacharbraceleft}{\kern0pt}a\ {\isacharless}{\kern0pt}{\isachardot}{\kern0pt}{\isachardot}{\kern0pt}{\isacharbraceright}{\kern0pt}{\isacharparenright}{\kern0pt}{\isacharparenright}{\kern0pt}{\isachardoublequoteclose}\isanewline
\isanewline
\isacommand{instance}\isamarkupfalse%
\isanewline
%
\isadelimproof
\ \ %
\endisadelimproof
%
\isatagproof
\isacommand{by}\isamarkupfalse%
\ standard\ {\isacharparenleft}{\kern0pt}rule\ open{\isacharunderscore}{\kern0pt}rat{\isacharunderscore}{\kern0pt}def{\isacharparenright}{\kern0pt}%
\endisatagproof
{\isafoldproof}%
%
\isadelimproof
\isanewline
%
\endisadelimproof
\isacommand{end}\isamarkupfalse%
\isanewline
\isanewline
\isacommand{lemma}\isamarkupfalse%
\ inv{\isacharunderscore}{\kern0pt}at{\isacharunderscore}{\kern0pt}right{\isacharunderscore}{\kern0pt}{\isadigit{0}}{\isacharunderscore}{\kern0pt}inf{\isacharcolon}{\kern0pt}\isanewline
\ \ {\isachardoublequoteopen}{\isasymforall}\isactrlsub F\ x\ in\ at{\isacharunderscore}{\kern0pt}right\ {\isadigit{0}}{\isachardot}{\kern0pt}\ c\ {\isasymle}\ {\isadigit{1}}\ {\isacharslash}{\kern0pt}\ real{\isacharunderscore}{\kern0pt}of{\isacharunderscore}{\kern0pt}rat\ x{\isachardoublequoteclose}\isanewline
%
\isadelimproof
\ \ %
\endisadelimproof
%
\isatagproof
\isacommand{apply}\isamarkupfalse%
\ {\isacharparenleft}{\kern0pt}rule\ eventually{\isacharunderscore}{\kern0pt}at{\isacharunderscore}{\kern0pt}rightI{\isacharbrackleft}{\kern0pt}\isakeyword{where}\ b{\isacharequal}{\kern0pt}{\isachardoublequoteopen}{\isadigit{1}}{\isacharslash}{\kern0pt}rat{\isacharunderscore}{\kern0pt}of{\isacharunderscore}{\kern0pt}int\ {\isacharparenleft}{\kern0pt}max\ {\isasymlceil}c{\isasymrceil}\ {\isadigit{1}}{\isacharparenright}{\kern0pt}{\isachardoublequoteclose}{\isacharbrackright}{\kern0pt}{\isacharparenright}{\kern0pt}\isanewline
\ \ \ \isacommand{apply}\isamarkupfalse%
\ {\isacharparenleft}{\kern0pt}rule\ order{\isacharunderscore}{\kern0pt}trans{\isacharbrackleft}{\kern0pt}\isakeyword{where}\ y{\isacharequal}{\kern0pt}{\isachardoublequoteopen}real{\isacharunderscore}{\kern0pt}of{\isacharunderscore}{\kern0pt}int\ {\isacharparenleft}{\kern0pt}max\ {\isasymlceil}c{\isasymrceil}\ {\isadigit{1}}{\isacharparenright}{\kern0pt}{\isachardoublequoteclose}{\isacharbrackright}{\kern0pt}{\isacharcomma}{\kern0pt}\ linarith{\isacharparenright}{\kern0pt}\isanewline
\ \ \ \isacommand{apply}\isamarkupfalse%
\ {\isacharparenleft}{\kern0pt}subst\ pos{\isacharunderscore}{\kern0pt}le{\isacharunderscore}{\kern0pt}divide{\isacharunderscore}{\kern0pt}eq{\isacharcomma}{\kern0pt}\ simp{\isacharparenright}{\kern0pt}\isanewline
\ \ \ \isacommand{apply}\isamarkupfalse%
\ simp\isanewline
\ \ \ \isacommand{apply}\isamarkupfalse%
\ {\isacharparenleft}{\kern0pt}subst\ {\isacharparenleft}{\kern0pt}asm{\isacharparenright}{\kern0pt}\ pos{\isacharunderscore}{\kern0pt}less{\isacharunderscore}{\kern0pt}divide{\isacharunderscore}{\kern0pt}eq{\isacharcomma}{\kern0pt}\ simp{\isacharparenright}{\kern0pt}\isanewline
\ \ \ \isacommand{apply}\isamarkupfalse%
\ {\isacharparenleft}{\kern0pt}metis\ less{\isacharunderscore}{\kern0pt}eq{\isacharunderscore}{\kern0pt}real{\isacharunderscore}{\kern0pt}def\ mult{\isachardot}{\kern0pt}commute\ of{\isacharunderscore}{\kern0pt}rat{\isacharunderscore}{\kern0pt}less{\isacharunderscore}{\kern0pt}{\isadigit{1}}{\isacharunderscore}{\kern0pt}iff\ of{\isacharunderscore}{\kern0pt}rat{\isacharunderscore}{\kern0pt}mult\ of{\isacharunderscore}{\kern0pt}rat{\isacharunderscore}{\kern0pt}of{\isacharunderscore}{\kern0pt}int{\isacharunderscore}{\kern0pt}eq{\isacharparenright}{\kern0pt}\isanewline
\ \ \isacommand{by}\isamarkupfalse%
\ simp%
\endisatagproof
{\isafoldproof}%
%
\isadelimproof
\isanewline
%
\endisadelimproof
%
\isadelimtheory
\isanewline
%
\endisadelimtheory
%
\isatagtheory
\isacommand{end}\isamarkupfalse%
%
\endisatagtheory
{\isafoldtheory}%
%
\isadelimtheory
%
\endisadelimtheory
%
\end{isabellebody}%
\endinput
%:%file=Landau_Ext.tex%:%
%:%11=1%:%
%:%27=3%:%
%:%28=3%:%
%:%29=4%:%
%:%30=5%:%
%:%39=7%:%
%:%43=9%:%
%:%45=11%:%
%:%46=11%:%
%:%47=12%:%
%:%48=13%:%
%:%49=14%:%
%:%52=15%:%
%:%56=15%:%
%:%57=15%:%
%:%62=15%:%
%:%65=16%:%
%:%66=17%:%
%:%67=17%:%
%:%68=18%:%
%:%69=19%:%
%:%70=20%:%
%:%71=21%:%
%:%72=22%:%
%:%79=23%:%
%:%80=23%:%
%:%81=24%:%
%:%82=24%:%
%:%83=25%:%
%:%84=25%:%
%:%85=25%:%
%:%86=26%:%
%:%87=26%:%
%:%88=27%:%
%:%89=27%:%
%:%90=27%:%
%:%91=28%:%
%:%92=28%:%
%:%93=29%:%
%:%94=29%:%
%:%95=30%:%
%:%96=30%:%
%:%97=31%:%
%:%98=31%:%
%:%99=32%:%
%:%100=32%:%
%:%101=32%:%
%:%102=32%:%
%:%103=33%:%
%:%104=33%:%
%:%105=33%:%
%:%106=33%:%
%:%107=33%:%
%:%108=34%:%
%:%109=34%:%
%:%110=34%:%
%:%111=35%:%
%:%112=35%:%
%:%113=36%:%
%:%114=36%:%
%:%115=37%:%
%:%116=37%:%
%:%117=38%:%
%:%118=38%:%
%:%119=39%:%
%:%120=39%:%
%:%121=40%:%
%:%122=40%:%
%:%123=40%:%
%:%124=41%:%
%:%125=41%:%
%:%126=42%:%
%:%127=42%:%
%:%128=43%:%
%:%129=43%:%
%:%130=43%:%
%:%131=44%:%
%:%132=44%:%
%:%133=44%:%
%:%134=44%:%
%:%135=45%:%
%:%136=45%:%
%:%137=46%:%
%:%138=46%:%
%:%139=47%:%
%:%140=47%:%
%:%141=48%:%
%:%142=48%:%
%:%143=49%:%
%:%144=49%:%
%:%145=49%:%
%:%146=50%:%
%:%152=50%:%
%:%155=51%:%
%:%156=52%:%
%:%157=52%:%
%:%158=53%:%
%:%159=54%:%
%:%160=55%:%
%:%161=56%:%
%:%168=57%:%
%:%169=57%:%
%:%170=58%:%
%:%171=58%:%
%:%172=59%:%
%:%173=59%:%
%:%174=60%:%
%:%175=60%:%
%:%176=60%:%
%:%177=61%:%
%:%178=61%:%
%:%179=62%:%
%:%180=62%:%
%:%181=62%:%
%:%182=62%:%
%:%183=63%:%
%:%189=63%:%
%:%192=64%:%
%:%193=65%:%
%:%194=65%:%
%:%195=66%:%
%:%196=67%:%
%:%197=68%:%
%:%198=69%:%
%:%205=70%:%
%:%206=70%:%
%:%207=71%:%
%:%208=71%:%
%:%209=72%:%
%:%210=72%:%
%:%211=73%:%
%:%212=73%:%
%:%213=73%:%
%:%214=74%:%
%:%215=74%:%
%:%216=75%:%
%:%217=75%:%
%:%218=75%:%
%:%219=75%:%
%:%220=76%:%
%:%226=76%:%
%:%229=77%:%
%:%230=78%:%
%:%231=78%:%
%:%232=79%:%
%:%233=80%:%
%:%234=81%:%
%:%241=82%:%
%:%242=82%:%
%:%243=83%:%
%:%244=83%:%
%:%245=84%:%
%:%246=84%:%
%:%247=85%:%
%:%248=85%:%
%:%249=86%:%
%:%250=86%:%
%:%251=87%:%
%:%252=87%:%
%:%253=88%:%
%:%254=88%:%
%:%255=89%:%
%:%256=89%:%
%:%257=90%:%
%:%258=90%:%
%:%259=91%:%
%:%265=91%:%
%:%268=92%:%
%:%269=93%:%
%:%270=93%:%
%:%271=94%:%
%:%272=95%:%
%:%273=96%:%
%:%274=97%:%
%:%275=98%:%
%:%282=99%:%
%:%283=99%:%
%:%284=100%:%
%:%285=100%:%
%:%286=101%:%
%:%287=101%:%
%:%288=101%:%
%:%289=102%:%
%:%290=102%:%
%:%291=103%:%
%:%292=103%:%
%:%293=104%:%
%:%294=104%:%
%:%295=105%:%
%:%296=105%:%
%:%297=106%:%
%:%298=106%:%
%:%299=107%:%
%:%300=107%:%
%:%301=108%:%
%:%302=108%:%
%:%303=109%:%
%:%304=109%:%
%:%305=109%:%
%:%306=110%:%
%:%307=110%:%
%:%308=110%:%
%:%309=110%:%
%:%310=111%:%
%:%311=111%:%
%:%312=111%:%
%:%313=111%:%
%:%314=112%:%
%:%315=112%:%
%:%316=112%:%
%:%317=113%:%
%:%318=113%:%
%:%319=113%:%
%:%320=114%:%
%:%321=114%:%
%:%322=115%:%
%:%323=115%:%
%:%324=115%:%
%:%325=116%:%
%:%326=116%:%
%:%327=116%:%
%:%328=117%:%
%:%329=117%:%
%:%330=118%:%
%:%331=118%:%
%:%332=118%:%
%:%333=119%:%
%:%334=119%:%
%:%335=120%:%
%:%336=120%:%
%:%337=120%:%
%:%338=121%:%
%:%339=121%:%
%:%340=121%:%
%:%341=122%:%
%:%342=122%:%
%:%343=123%:%
%:%344=123%:%
%:%345=124%:%
%:%346=124%:%
%:%347=125%:%
%:%348=125%:%
%:%349=125%:%
%:%350=126%:%
%:%351=126%:%
%:%352=126%:%
%:%353=127%:%
%:%354=127%:%
%:%355=127%:%
%:%356=128%:%
%:%357=128%:%
%:%358=129%:%
%:%359=129%:%
%:%360=129%:%
%:%361=129%:%
%:%362=130%:%
%:%363=130%:%
%:%364=131%:%
%:%365=131%:%
%:%366=131%:%
%:%367=131%:%
%:%368=132%:%
%:%369=132%:%
%:%370=132%:%
%:%371=133%:%
%:%372=133%:%
%:%373=134%:%
%:%374=134%:%
%:%375=134%:%
%:%376=134%:%
%:%377=135%:%
%:%378=135%:%
%:%379=135%:%
%:%380=135%:%
%:%381=136%:%
%:%382=136%:%
%:%383=137%:%
%:%384=137%:%
%:%385=138%:%
%:%386=138%:%
%:%387=139%:%
%:%388=139%:%
%:%389=140%:%
%:%390=140%:%
%:%391=141%:%
%:%392=141%:%
%:%393=142%:%
%:%394=142%:%
%:%395=143%:%
%:%401=143%:%
%:%404=144%:%
%:%405=145%:%
%:%406=145%:%
%:%407=146%:%
%:%408=147%:%
%:%409=148%:%
%:%416=149%:%
%:%417=149%:%
%:%418=150%:%
%:%419=150%:%
%:%420=151%:%
%:%421=151%:%
%:%422=151%:%
%:%423=152%:%
%:%424=153%:%
%:%425=153%:%
%:%426=154%:%
%:%427=154%:%
%:%428=155%:%
%:%429=155%:%
%:%430=156%:%
%:%431=156%:%
%:%432=157%:%
%:%433=157%:%
%:%434=158%:%
%:%435=158%:%
%:%436=159%:%
%:%442=159%:%
%:%445=160%:%
%:%446=161%:%
%:%447=161%:%
%:%448=162%:%
%:%449=163%:%
%:%450=164%:%
%:%453=165%:%
%:%457=165%:%
%:%458=165%:%
%:%459=166%:%
%:%460=166%:%
%:%461=167%:%
%:%462=167%:%
%:%463=168%:%
%:%464=168%:%
%:%469=168%:%
%:%472=169%:%
%:%473=170%:%
%:%474=170%:%
%:%475=171%:%
%:%476=172%:%
%:%477=173%:%
%:%480=174%:%
%:%484=174%:%
%:%485=174%:%
%:%486=175%:%
%:%487=175%:%
%:%492=175%:%
%:%495=176%:%
%:%496=177%:%
%:%497=177%:%
%:%498=178%:%
%:%499=179%:%
%:%500=180%:%
%:%507=181%:%
%:%508=181%:%
%:%509=182%:%
%:%510=182%:%
%:%511=183%:%
%:%512=183%:%
%:%513=183%:%
%:%514=184%:%
%:%515=184%:%
%:%516=185%:%
%:%517=185%:%
%:%518=186%:%
%:%519=186%:%
%:%520=187%:%
%:%521=187%:%
%:%522=188%:%
%:%523=188%:%
%:%524=189%:%
%:%525=189%:%
%:%526=190%:%
%:%527=190%:%
%:%528=191%:%
%:%529=191%:%
%:%530=192%:%
%:%531=192%:%
%:%532=193%:%
%:%533=193%:%
%:%534=194%:%
%:%540=194%:%
%:%543=195%:%
%:%544=196%:%
%:%545=196%:%
%:%546=197%:%
%:%547=198%:%
%:%548=199%:%
%:%551=200%:%
%:%555=200%:%
%:%556=200%:%
%:%557=201%:%
%:%558=201%:%
%:%563=201%:%
%:%566=202%:%
%:%567=203%:%
%:%568=203%:%
%:%569=204%:%
%:%570=205%:%
%:%571=206%:%
%:%574=207%:%
%:%578=207%:%
%:%579=207%:%
%:%580=208%:%
%:%581=208%:%
%:%586=208%:%
%:%589=209%:%
%:%590=210%:%
%:%591=210%:%
%:%592=211%:%
%:%593=212%:%
%:%596=213%:%
%:%600=213%:%
%:%601=213%:%
%:%602=214%:%
%:%603=214%:%
%:%608=214%:%
%:%611=215%:%
%:%612=216%:%
%:%613=216%:%
%:%615=216%:%
%:%619=216%:%
%:%620=216%:%
%:%627=216%:%
%:%628=217%:%
%:%629=218%:%
%:%630=218%:%
%:%631=219%:%
%:%632=220%:%
%:%635=221%:%
%:%639=221%:%
%:%640=221%:%
%:%641=222%:%
%:%642=222%:%
%:%643=223%:%
%:%644=223%:%
%:%649=223%:%
%:%652=224%:%
%:%653=225%:%
%:%654=225%:%
%:%655=226%:%
%:%656=227%:%
%:%659=228%:%
%:%663=228%:%
%:%664=228%:%
%:%665=229%:%
%:%666=229%:%
%:%667=230%:%
%:%668=230%:%
%:%673=230%:%
%:%676=231%:%
%:%677=232%:%
%:%678=232%:%
%:%679=233%:%
%:%680=234%:%
%:%681=235%:%
%:%682=235%:%
%:%683=236%:%
%:%684=237%:%
%:%685=238%:%
%:%686=238%:%
%:%689=239%:%
%:%693=239%:%
%:%694=239%:%
%:%699=239%:%
%:%702=240%:%
%:%703=240%:%
%:%704=241%:%
%:%705=242%:%
%:%706=242%:%
%:%707=243%:%
%:%710=244%:%
%:%714=244%:%
%:%715=244%:%
%:%716=245%:%
%:%717=245%:%
%:%718=246%:%
%:%719=246%:%
%:%720=247%:%
%:%721=247%:%
%:%722=248%:%
%:%723=248%:%
%:%724=249%:%
%:%725=249%:%
%:%726=250%:%
%:%727=250%:%
%:%732=250%:%
%:%737=251%:%
%:%742=252%:%

%
\begin{isabellebody}%
\setisabellecontext{Frequency{\isacharunderscore}{\kern0pt}Moment{\isacharunderscore}{\kern0pt}{\isadigit{0}}}%
%
\isadelimdocument
%
\endisadelimdocument
%
\isatagdocument
%
\isamarkupsection{Frequency Moment $0$%
}
\isamarkuptrue%
%
\endisatagdocument
{\isafolddocument}%
%
\isadelimdocument
%
\endisadelimdocument
%
\isadelimtheory
%
\endisadelimtheory
%
\isatagtheory
\isacommand{theory}\isamarkupfalse%
\ Frequency{\isacharunderscore}{\kern0pt}Moment{\isacharunderscore}{\kern0pt}{\isadigit{0}}\isanewline
\ \ \isakeyword{imports}\ Main\ Primes{\isacharunderscore}{\kern0pt}Ext\ Float{\isacharunderscore}{\kern0pt}Ext\ Median\ K{\isacharunderscore}{\kern0pt}Smallest\ Universal{\isacharunderscore}{\kern0pt}Hash{\isacharunderscore}{\kern0pt}Families{\isacharunderscore}{\kern0pt}Nat\ Encoding\isanewline
\ \ Frequency{\isacharunderscore}{\kern0pt}Moments\ Landau{\isacharunderscore}{\kern0pt}Ext\isanewline
\isakeyword{begin}%
\endisatagtheory
{\isafoldtheory}%
%
\isadelimtheory
%
\endisadelimtheory
%
\begin{isamarkuptext}%
This section contains a formalization of the algorithm for the zero-th frequency moment.
It is a KMV algorithm with a rounding method to match the space complexity of the best
algorithm described in \cite{baryossef2002}.%
\end{isamarkuptext}\isamarkuptrue%
%
\begin{isamarkuptext}%
In addition ot the Isabelle proof here, there is also and informal hand-writtend proof in
Appendix~\ref{sec:f0_proof}.%
\end{isamarkuptext}\isamarkuptrue%
\isacommand{type{\isacharunderscore}{\kern0pt}synonym}\isamarkupfalse%
\ f{\isadigit{0}}{\isacharunderscore}{\kern0pt}state\ {\isacharequal}{\kern0pt}\ {\isachardoublequoteopen}nat\ {\isasymtimes}\ nat\ {\isasymtimes}\ nat\ {\isasymtimes}\ nat\ {\isasymtimes}\ {\isacharparenleft}{\kern0pt}nat\ {\isasymRightarrow}\ {\isacharparenleft}{\kern0pt}int\ set\ list{\isacharparenright}{\kern0pt}{\isacharparenright}{\kern0pt}\ {\isasymtimes}\ {\isacharparenleft}{\kern0pt}nat\ {\isasymRightarrow}\ float\ set{\isacharparenright}{\kern0pt}{\isachardoublequoteclose}\isanewline
\isanewline
\isacommand{fun}\isamarkupfalse%
\ f{\isadigit{0}}{\isacharunderscore}{\kern0pt}init\ {\isacharcolon}{\kern0pt}{\isacharcolon}{\kern0pt}\ {\isachardoublequoteopen}rat\ {\isasymRightarrow}\ rat\ {\isasymRightarrow}\ nat\ {\isasymRightarrow}\ f{\isadigit{0}}{\isacharunderscore}{\kern0pt}state\ pmf{\isachardoublequoteclose}\ \isakeyword{where}\isanewline
\ \ {\isachardoublequoteopen}f{\isadigit{0}}{\isacharunderscore}{\kern0pt}init\ {\isasymdelta}\ {\isasymepsilon}\ n\ {\isacharequal}{\kern0pt}\isanewline
\ \ \ \ do\ {\isacharbraceleft}{\kern0pt}\isanewline
\ \ \ \ \ \ let\ s\ {\isacharequal}{\kern0pt}\ nat\ {\isasymlceil}{\isacharminus}{\kern0pt}{\isadigit{1}}{\isadigit{8}}\ {\isacharasterisk}{\kern0pt}\ ln\ {\isacharparenleft}{\kern0pt}real{\isacharunderscore}{\kern0pt}of{\isacharunderscore}{\kern0pt}rat\ {\isasymepsilon}{\isacharparenright}{\kern0pt}{\isasymrceil}{\isacharsemicolon}{\kern0pt}\isanewline
\ \ \ \ \ \ let\ t\ {\isacharequal}{\kern0pt}\ nat\ {\isasymlceil}{\isadigit{8}}{\isadigit{0}}\ {\isacharslash}{\kern0pt}\ {\isacharparenleft}{\kern0pt}real{\isacharunderscore}{\kern0pt}of{\isacharunderscore}{\kern0pt}rat\ {\isasymdelta}{\isacharparenright}{\kern0pt}\isactrlsup {\isadigit{2}}{\isasymrceil}{\isacharsemicolon}{\kern0pt}\isanewline
\ \ \ \ \ \ let\ p\ {\isacharequal}{\kern0pt}\ find{\isacharunderscore}{\kern0pt}prime{\isacharunderscore}{\kern0pt}above\ {\isacharparenleft}{\kern0pt}max\ n\ {\isadigit{1}}{\isadigit{9}}{\isacharparenright}{\kern0pt}{\isacharsemicolon}{\kern0pt}\isanewline
\ \ \ \ \ \ let\ r\ {\isacharequal}{\kern0pt}\ nat\ {\isacharparenleft}{\kern0pt}{\isadigit{4}}\ {\isacharasterisk}{\kern0pt}\ {\isasymlceil}log\ {\isadigit{2}}\ {\isacharparenleft}{\kern0pt}{\isadigit{1}}\ {\isacharslash}{\kern0pt}\ real{\isacharunderscore}{\kern0pt}of{\isacharunderscore}{\kern0pt}rat\ {\isasymdelta}{\isacharparenright}{\kern0pt}{\isasymrceil}\ {\isacharplus}{\kern0pt}\ {\isadigit{2}}{\isadigit{4}}{\isacharparenright}{\kern0pt}{\isacharsemicolon}{\kern0pt}\ \isanewline
\ \ \ \ \ \ h\ {\isasymleftarrow}\ prod{\isacharunderscore}{\kern0pt}pmf\ {\isacharbraceleft}{\kern0pt}{\isadigit{0}}{\isachardot}{\kern0pt}{\isachardot}{\kern0pt}{\isacharless}{\kern0pt}s{\isacharbraceright}{\kern0pt}\ {\isacharparenleft}{\kern0pt}{\isasymlambda}{\isacharunderscore}{\kern0pt}{\isachardot}{\kern0pt}\ pmf{\isacharunderscore}{\kern0pt}of{\isacharunderscore}{\kern0pt}set\ {\isacharparenleft}{\kern0pt}bounded{\isacharunderscore}{\kern0pt}degree{\isacharunderscore}{\kern0pt}polynomials\ {\isacharparenleft}{\kern0pt}ZFact\ {\isacharparenleft}{\kern0pt}int\ p{\isacharparenright}{\kern0pt}{\isacharparenright}{\kern0pt}\ {\isadigit{2}}{\isacharparenright}{\kern0pt}{\isacharparenright}{\kern0pt}{\isacharsemicolon}{\kern0pt}\isanewline
\ \ \ \ \ \ return{\isacharunderscore}{\kern0pt}pmf\ {\isacharparenleft}{\kern0pt}s{\isacharcomma}{\kern0pt}\ t{\isacharcomma}{\kern0pt}\ p{\isacharcomma}{\kern0pt}\ r{\isacharcomma}{\kern0pt}\ h{\isacharcomma}{\kern0pt}\ {\isacharparenleft}{\kern0pt}{\isasymlambda}{\isacharunderscore}{\kern0pt}\ {\isasymin}\ {\isacharbraceleft}{\kern0pt}{\isadigit{0}}{\isachardot}{\kern0pt}{\isachardot}{\kern0pt}{\isacharless}{\kern0pt}s{\isacharbraceright}{\kern0pt}{\isachardot}{\kern0pt}\ {\isacharbraceleft}{\kern0pt}{\isacharbraceright}{\kern0pt}{\isacharparenright}{\kern0pt}{\isacharparenright}{\kern0pt}\isanewline
\ \ \ \ {\isacharbraceright}{\kern0pt}{\isachardoublequoteclose}\isanewline
\isanewline
\isacommand{fun}\isamarkupfalse%
\ f{\isadigit{0}}{\isacharunderscore}{\kern0pt}update\ {\isacharcolon}{\kern0pt}{\isacharcolon}{\kern0pt}\ {\isachardoublequoteopen}nat\ {\isasymRightarrow}\ f{\isadigit{0}}{\isacharunderscore}{\kern0pt}state\ {\isasymRightarrow}\ f{\isadigit{0}}{\isacharunderscore}{\kern0pt}state\ pmf{\isachardoublequoteclose}\ \isakeyword{where}\isanewline
\ \ {\isachardoublequoteopen}f{\isadigit{0}}{\isacharunderscore}{\kern0pt}update\ x\ {\isacharparenleft}{\kern0pt}s{\isacharcomma}{\kern0pt}\ t{\isacharcomma}{\kern0pt}\ p{\isacharcomma}{\kern0pt}\ r{\isacharcomma}{\kern0pt}\ h{\isacharcomma}{\kern0pt}\ sketch{\isacharparenright}{\kern0pt}\ {\isacharequal}{\kern0pt}\ \isanewline
\ \ \ \ return{\isacharunderscore}{\kern0pt}pmf\ {\isacharparenleft}{\kern0pt}s{\isacharcomma}{\kern0pt}\ t{\isacharcomma}{\kern0pt}\ p{\isacharcomma}{\kern0pt}\ r{\isacharcomma}{\kern0pt}\ h{\isacharcomma}{\kern0pt}\ {\isasymlambda}i\ {\isasymin}\ {\isacharbraceleft}{\kern0pt}{\isadigit{0}}{\isachardot}{\kern0pt}{\isachardot}{\kern0pt}{\isacharless}{\kern0pt}s{\isacharbraceright}{\kern0pt}{\isachardot}{\kern0pt}\isanewline
\ \ \ \ \ \ least\ t\ {\isacharparenleft}{\kern0pt}insert\ {\isacharparenleft}{\kern0pt}float{\isacharunderscore}{\kern0pt}of\ {\isacharparenleft}{\kern0pt}truncate{\isacharunderscore}{\kern0pt}down\ r\ {\isacharparenleft}{\kern0pt}hash\ p\ x\ {\isacharparenleft}{\kern0pt}h\ i{\isacharparenright}{\kern0pt}{\isacharparenright}{\kern0pt}{\isacharparenright}{\kern0pt}{\isacharparenright}{\kern0pt}\ {\isacharparenleft}{\kern0pt}sketch\ i{\isacharparenright}{\kern0pt}{\isacharparenright}{\kern0pt}{\isacharparenright}{\kern0pt}{\isachardoublequoteclose}\isanewline
\isanewline
\isacommand{fun}\isamarkupfalse%
\ f{\isadigit{0}}{\isacharunderscore}{\kern0pt}result\ {\isacharcolon}{\kern0pt}{\isacharcolon}{\kern0pt}\ {\isachardoublequoteopen}f{\isadigit{0}}{\isacharunderscore}{\kern0pt}state\ {\isasymRightarrow}\ rat\ pmf{\isachardoublequoteclose}\ \isakeyword{where}\isanewline
\ \ {\isachardoublequoteopen}f{\isadigit{0}}{\isacharunderscore}{\kern0pt}result\ {\isacharparenleft}{\kern0pt}s{\isacharcomma}{\kern0pt}\ t{\isacharcomma}{\kern0pt}\ p{\isacharcomma}{\kern0pt}\ r{\isacharcomma}{\kern0pt}\ h{\isacharcomma}{\kern0pt}\ sketch{\isacharparenright}{\kern0pt}\ {\isacharequal}{\kern0pt}\ return{\isacharunderscore}{\kern0pt}pmf\ {\isacharparenleft}{\kern0pt}median\ {\isacharparenleft}{\kern0pt}{\isasymlambda}i\ {\isasymin}\ {\isacharbraceleft}{\kern0pt}{\isadigit{0}}{\isachardot}{\kern0pt}{\isachardot}{\kern0pt}{\isacharless}{\kern0pt}s{\isacharbraceright}{\kern0pt}{\isachardot}{\kern0pt}\isanewline
\ \ \ \ \ \ {\isacharparenleft}{\kern0pt}if\ card\ {\isacharparenleft}{\kern0pt}sketch\ i{\isacharparenright}{\kern0pt}\ {\isacharless}{\kern0pt}\ t\ then\ of{\isacharunderscore}{\kern0pt}nat\ {\isacharparenleft}{\kern0pt}card\ {\isacharparenleft}{\kern0pt}sketch\ i{\isacharparenright}{\kern0pt}{\isacharparenright}{\kern0pt}\ else\isanewline
\ \ \ \ \ \ \ \ rat{\isacharunderscore}{\kern0pt}of{\isacharunderscore}{\kern0pt}nat\ t{\isacharasterisk}{\kern0pt}\ rat{\isacharunderscore}{\kern0pt}of{\isacharunderscore}{\kern0pt}nat\ p\ {\isacharslash}{\kern0pt}\ rat{\isacharunderscore}{\kern0pt}of{\isacharunderscore}{\kern0pt}float\ \ {\isacharparenleft}{\kern0pt}Max\ {\isacharparenleft}{\kern0pt}sketch\ i{\isacharparenright}{\kern0pt}{\isacharparenright}{\kern0pt}{\isacharparenright}{\kern0pt}\isanewline
\ \ \ \ {\isacharparenright}{\kern0pt}\ s{\isacharparenright}{\kern0pt}{\isachardoublequoteclose}\isanewline
\isanewline
\isacommand{definition}\isamarkupfalse%
\ f{\isadigit{0}}{\isacharunderscore}{\kern0pt}sketch\ \isakeyword{where}\ \isanewline
\ \ {\isachardoublequoteopen}f{\isadigit{0}}{\isacharunderscore}{\kern0pt}sketch\ p\ r\ t\ h\ xs\ {\isacharequal}{\kern0pt}\ least\ t\ {\isacharparenleft}{\kern0pt}{\isacharparenleft}{\kern0pt}{\isasymlambda}x{\isachardot}{\kern0pt}\ float{\isacharunderscore}{\kern0pt}of\ {\isacharparenleft}{\kern0pt}truncate{\isacharunderscore}{\kern0pt}down\ r\ {\isacharparenleft}{\kern0pt}hash\ p\ x\ h{\isacharparenright}{\kern0pt}{\isacharparenright}{\kern0pt}{\isacharparenright}{\kern0pt}\ {\isacharbackquote}{\kern0pt}\ {\isacharparenleft}{\kern0pt}set\ xs{\isacharparenright}{\kern0pt}{\isacharparenright}{\kern0pt}{\isachardoublequoteclose}\isanewline
\isanewline
\isacommand{lemma}\isamarkupfalse%
\ f{\isadigit{0}}{\isacharunderscore}{\kern0pt}alg{\isacharunderscore}{\kern0pt}sketch{\isacharcolon}{\kern0pt}\isanewline
\ \ \isakeyword{fixes}\ n\ {\isacharcolon}{\kern0pt}{\isacharcolon}{\kern0pt}\ nat\isanewline
\ \ \isakeyword{fixes}\ as\ {\isacharcolon}{\kern0pt}{\isacharcolon}{\kern0pt}\ {\isachardoublequoteopen}nat\ list{\isachardoublequoteclose}\isanewline
\ \ \isakeyword{assumes}\ {\isachardoublequoteopen}{\isasymepsilon}\ {\isasymin}\ {\isacharbraceleft}{\kern0pt}{\isadigit{0}}{\isacharless}{\kern0pt}{\isachardot}{\kern0pt}{\isachardot}{\kern0pt}{\isacharless}{\kern0pt}{\isadigit{1}}{\isacharbraceright}{\kern0pt}{\isachardoublequoteclose}\isanewline
\ \ \isakeyword{assumes}\ {\isachardoublequoteopen}{\isasymdelta}\ {\isasymin}\ {\isacharbraceleft}{\kern0pt}{\isadigit{0}}{\isacharless}{\kern0pt}{\isachardot}{\kern0pt}{\isachardot}{\kern0pt}{\isacharless}{\kern0pt}{\isadigit{1}}{\isacharbraceright}{\kern0pt}{\isachardoublequoteclose}\isanewline
\ \ \isakeyword{defines}\ {\isachardoublequoteopen}sketch\ {\isasymequiv}\ fold\ {\isacharparenleft}{\kern0pt}{\isasymlambda}a\ state{\isachardot}{\kern0pt}\ state\ {\isasymbind}\ f{\isadigit{0}}{\isacharunderscore}{\kern0pt}update\ a{\isacharparenright}{\kern0pt}\ as\ {\isacharparenleft}{\kern0pt}f{\isadigit{0}}{\isacharunderscore}{\kern0pt}init\ {\isasymdelta}\ {\isasymepsilon}\ n{\isacharparenright}{\kern0pt}{\isachardoublequoteclose}\isanewline
\ \ \isakeyword{defines}\ {\isachardoublequoteopen}t\ {\isasymequiv}\ nat\ {\isasymlceil}{\isadigit{8}}{\isadigit{0}}\ {\isacharslash}{\kern0pt}\ {\isacharparenleft}{\kern0pt}real{\isacharunderscore}{\kern0pt}of{\isacharunderscore}{\kern0pt}rat\ {\isasymdelta}{\isacharparenright}{\kern0pt}\isactrlsup {\isadigit{2}}{\isasymrceil}{\isachardoublequoteclose}\isanewline
\ \ \isakeyword{defines}\ {\isachardoublequoteopen}s\ {\isasymequiv}\ nat\ {\isasymlceil}{\isacharminus}{\kern0pt}{\isacharparenleft}{\kern0pt}{\isadigit{1}}{\isadigit{8}}\ {\isacharasterisk}{\kern0pt}\ ln\ {\isacharparenleft}{\kern0pt}real{\isacharunderscore}{\kern0pt}of{\isacharunderscore}{\kern0pt}rat\ {\isasymepsilon}{\isacharparenright}{\kern0pt}{\isacharparenright}{\kern0pt}{\isasymrceil}{\isachardoublequoteclose}\isanewline
\ \ \isakeyword{defines}\ {\isachardoublequoteopen}p\ {\isasymequiv}\ find{\isacharunderscore}{\kern0pt}prime{\isacharunderscore}{\kern0pt}above\ {\isacharparenleft}{\kern0pt}max\ n\ {\isadigit{1}}{\isadigit{9}}{\isacharparenright}{\kern0pt}{\isachardoublequoteclose}\isanewline
\ \ \isakeyword{defines}\ {\isachardoublequoteopen}r\ {\isasymequiv}\ nat\ {\isacharparenleft}{\kern0pt}{\isadigit{4}}\ {\isacharasterisk}{\kern0pt}\ {\isasymlceil}log\ {\isadigit{2}}\ {\isacharparenleft}{\kern0pt}{\isadigit{1}}\ {\isacharslash}{\kern0pt}\ real{\isacharunderscore}{\kern0pt}of{\isacharunderscore}{\kern0pt}rat\ {\isasymdelta}{\isacharparenright}{\kern0pt}{\isasymrceil}\ {\isacharplus}{\kern0pt}\ {\isadigit{2}}{\isadigit{4}}{\isacharparenright}{\kern0pt}{\isachardoublequoteclose}\isanewline
\ \ \isakeyword{shows}\ {\isachardoublequoteopen}sketch\ {\isacharequal}{\kern0pt}\ map{\isacharunderscore}{\kern0pt}pmf\ {\isacharparenleft}{\kern0pt}{\isasymlambda}x{\isachardot}{\kern0pt}\ {\isacharparenleft}{\kern0pt}s{\isacharcomma}{\kern0pt}t{\isacharcomma}{\kern0pt}p{\isacharcomma}{\kern0pt}r{\isacharcomma}{\kern0pt}\ x{\isacharcomma}{\kern0pt}\ {\isasymlambda}i\ {\isasymin}\ {\isacharbraceleft}{\kern0pt}{\isadigit{0}}{\isachardot}{\kern0pt}{\isachardot}{\kern0pt}{\isacharless}{\kern0pt}s{\isacharbraceright}{\kern0pt}{\isachardot}{\kern0pt}\ f{\isadigit{0}}{\isacharunderscore}{\kern0pt}sketch\ p\ r\ t\ {\isacharparenleft}{\kern0pt}x\ i{\isacharparenright}{\kern0pt}\ as{\isacharparenright}{\kern0pt}{\isacharparenright}{\kern0pt}\isanewline
\ \ \ \ {\isacharparenleft}{\kern0pt}prod{\isacharunderscore}{\kern0pt}pmf\ {\isacharbraceleft}{\kern0pt}{\isadigit{0}}{\isachardot}{\kern0pt}{\isachardot}{\kern0pt}{\isacharless}{\kern0pt}s{\isacharbraceright}{\kern0pt}\ {\isacharparenleft}{\kern0pt}{\isasymlambda}{\isacharunderscore}{\kern0pt}{\isachardot}{\kern0pt}\ pmf{\isacharunderscore}{\kern0pt}of{\isacharunderscore}{\kern0pt}set\ {\isacharparenleft}{\kern0pt}bounded{\isacharunderscore}{\kern0pt}degree{\isacharunderscore}{\kern0pt}polynomials\ {\isacharparenleft}{\kern0pt}ZFact\ {\isacharparenleft}{\kern0pt}int\ p{\isacharparenright}{\kern0pt}{\isacharparenright}{\kern0pt}\ {\isadigit{2}}{\isacharparenright}{\kern0pt}{\isacharparenright}{\kern0pt}{\isacharparenright}{\kern0pt}{\isachardoublequoteclose}\ \isanewline
%
\isadelimproof
%
\endisadelimproof
%
\isatagproof
\isacommand{proof}\isamarkupfalse%
\ {\isacharparenleft}{\kern0pt}subst\ sketch{\isacharunderscore}{\kern0pt}def{\isacharcomma}{\kern0pt}\ induction\ as\ rule{\isacharcolon}{\kern0pt}rev{\isacharunderscore}{\kern0pt}induct{\isacharparenright}{\kern0pt}\isanewline
\ \ \isacommand{case}\isamarkupfalse%
\ Nil\isanewline
\ \ \isacommand{then}\isamarkupfalse%
\ \isacommand{show}\isamarkupfalse%
\ {\isacharquery}{\kern0pt}case\isanewline
\ \ \ \ \isacommand{apply}\isamarkupfalse%
\ {\isacharparenleft}{\kern0pt}simp\ add{\isacharcolon}{\kern0pt}s{\isacharunderscore}{\kern0pt}def{\isacharbrackleft}{\kern0pt}symmetric{\isacharbrackright}{\kern0pt}\ p{\isacharunderscore}{\kern0pt}def{\isacharbrackleft}{\kern0pt}symmetric{\isacharbrackright}{\kern0pt}\ map{\isacharunderscore}{\kern0pt}pmf{\isacharunderscore}{\kern0pt}def{\isacharbrackleft}{\kern0pt}symmetric{\isacharbrackright}{\kern0pt}\ t{\isacharunderscore}{\kern0pt}def{\isacharbrackleft}{\kern0pt}symmetric{\isacharbrackright}{\kern0pt}\ r{\isacharunderscore}{\kern0pt}def{\isacharbrackleft}{\kern0pt}symmetric{\isacharbrackright}{\kern0pt}{\isacharparenright}{\kern0pt}\isanewline
\ \ \ \ \isacommand{apply}\isamarkupfalse%
\ {\isacharparenleft}{\kern0pt}rule\ arg{\isacharunderscore}{\kern0pt}cong{\isadigit{2}}{\isacharbrackleft}{\kern0pt}\isakeyword{where}\ f{\isacharequal}{\kern0pt}{\isachardoublequoteopen}map{\isacharunderscore}{\kern0pt}pmf{\isachardoublequoteclose}{\isacharbrackright}{\kern0pt}{\isacharparenright}{\kern0pt}\isanewline
\ \ \ \ \ \isacommand{apply}\isamarkupfalse%
\ {\isacharparenleft}{\kern0pt}rule\ ext{\isacharparenright}{\kern0pt}\isanewline
\ \ \ \ \ \isacommand{apply}\isamarkupfalse%
\ simp\isanewline
\ \ \ \ \isacommand{by}\isamarkupfalse%
\ {\isacharparenleft}{\kern0pt}rule\ ext{\isacharcomma}{\kern0pt}\ simp\ add{\isacharcolon}{\kern0pt}f{\isadigit{0}}{\isacharunderscore}{\kern0pt}sketch{\isacharunderscore}{\kern0pt}def\ least{\isacharunderscore}{\kern0pt}def{\isacharcomma}{\kern0pt}\ simp{\isacharparenright}{\kern0pt}\isanewline
\isacommand{next}\isamarkupfalse%
\isanewline
\ \ \isacommand{case}\isamarkupfalse%
\ {\isacharparenleft}{\kern0pt}snoc\ x\ xs{\isacharparenright}{\kern0pt}\isanewline
\ \ \isacommand{then}\isamarkupfalse%
\ \isacommand{show}\isamarkupfalse%
\ {\isacharquery}{\kern0pt}case\isanewline
\ \ \ \ \isacommand{apply}\isamarkupfalse%
\ {\isacharparenleft}{\kern0pt}simp\ add{\isacharcolon}{\kern0pt}map{\isacharunderscore}{\kern0pt}pmf{\isacharunderscore}{\kern0pt}def{\isacharparenright}{\kern0pt}\isanewline
\ \ \ \ \isacommand{apply}\isamarkupfalse%
\ {\isacharparenleft}{\kern0pt}subst\ bind{\isacharunderscore}{\kern0pt}assoc{\isacharunderscore}{\kern0pt}pmf{\isacharparenright}{\kern0pt}\isanewline
\ \ \ \ \isacommand{apply}\isamarkupfalse%
\ {\isacharparenleft}{\kern0pt}subst\ bind{\isacharunderscore}{\kern0pt}return{\isacharunderscore}{\kern0pt}pmf{\isacharparenright}{\kern0pt}\isanewline
\ \ \ \ \isacommand{apply}\isamarkupfalse%
\ {\isacharparenleft}{\kern0pt}rule\ arg{\isacharunderscore}{\kern0pt}cong{\isadigit{2}}{\isacharbrackleft}{\kern0pt}\isakeyword{where}\ f{\isacharequal}{\kern0pt}{\isachardoublequoteopen}bind{\isacharunderscore}{\kern0pt}pmf{\isachardoublequoteclose}{\isacharbrackright}{\kern0pt}{\isacharcomma}{\kern0pt}\ simp{\isacharparenright}{\kern0pt}\isanewline
\ \ \ \ \isacommand{apply}\isamarkupfalse%
\ {\isacharparenleft}{\kern0pt}simp{\isacharparenright}{\kern0pt}\isanewline
\ \ \ \ \isacommand{apply}\isamarkupfalse%
\ {\isacharparenleft}{\kern0pt}rule\ ext{\isacharcomma}{\kern0pt}\ rule\ arg{\isacharunderscore}{\kern0pt}cong{\isacharbrackleft}{\kern0pt}\isakeyword{where}\ f{\isacharequal}{\kern0pt}{\isachardoublequoteopen}return{\isacharunderscore}{\kern0pt}pmf{\isachardoublequoteclose}{\isacharbrackright}{\kern0pt}{\isacharcomma}{\kern0pt}\ simp{\isacharparenright}{\kern0pt}\isanewline
\ \ \ \ \isacommand{apply}\isamarkupfalse%
\ {\isacharparenleft}{\kern0pt}rule\ ext{\isacharparenright}{\kern0pt}\isanewline
\ \ \ \ \isacommand{apply}\isamarkupfalse%
\ {\isacharparenleft}{\kern0pt}simp\ add{\isacharcolon}{\kern0pt}f{\isadigit{0}}{\isacharunderscore}{\kern0pt}sketch{\isacharunderscore}{\kern0pt}def{\isacharparenright}{\kern0pt}\isanewline
\ \ \ \ \isacommand{by}\isamarkupfalse%
\ {\isacharparenleft}{\kern0pt}subst\ least{\isacharunderscore}{\kern0pt}insert{\isacharcomma}{\kern0pt}\ simp{\isacharcomma}{\kern0pt}\ simp{\isacharparenright}{\kern0pt}\isanewline
\isacommand{qed}\isamarkupfalse%
%
\endisatagproof
{\isafoldproof}%
%
\isadelimproof
\isanewline
%
\endisadelimproof
\isanewline
\isacommand{lemma}\isamarkupfalse%
\ {\isacharparenleft}{\kern0pt}\isakeyword{in}\ prob{\isacharunderscore}{\kern0pt}space{\isacharparenright}{\kern0pt}\ prob{\isacharunderscore}{\kern0pt}sub{\isacharunderscore}{\kern0pt}additive{\isacharcolon}{\kern0pt}\isanewline
\ \ \isakeyword{assumes}\ {\isachardoublequoteopen}Collect\ P\ {\isasymin}\ sets\ M{\isachardoublequoteclose}\isanewline
\ \ \isakeyword{assumes}\ {\isachardoublequoteopen}Collect\ Q\ {\isasymin}\ sets\ M{\isachardoublequoteclose}\isanewline
\ \ \isakeyword{shows}\ {\isachardoublequoteopen}{\isasymP}{\isacharparenleft}{\kern0pt}{\isasymomega}\ in\ M{\isachardot}{\kern0pt}\ P\ {\isasymomega}\ {\isasymor}\ Q\ {\isasymomega}{\isacharparenright}{\kern0pt}\ {\isasymle}\ {\isasymP}{\isacharparenleft}{\kern0pt}{\isasymomega}\ in\ M{\isachardot}{\kern0pt}\ P\ {\isasymomega}{\isacharparenright}{\kern0pt}\ {\isacharplus}{\kern0pt}\ {\isasymP}{\isacharparenleft}{\kern0pt}{\isasymomega}\ in\ M{\isachardot}{\kern0pt}\ Q\ {\isasymomega}{\isacharparenright}{\kern0pt}{\isachardoublequoteclose}\isanewline
%
\isadelimproof
%
\endisadelimproof
%
\isatagproof
\isacommand{proof}\isamarkupfalse%
\ {\isacharminus}{\kern0pt}\isanewline
\ \ \isacommand{have}\isamarkupfalse%
\ {\isachardoublequoteopen}{\isasymP}{\isacharparenleft}{\kern0pt}{\isasymomega}\ in\ M{\isachardot}{\kern0pt}\ P\ {\isasymomega}\ {\isasymor}\ Q\ {\isasymomega}{\isacharparenright}{\kern0pt}\ {\isacharequal}{\kern0pt}\ measure\ M\ {\isacharparenleft}{\kern0pt}{\isacharbraceleft}{\kern0pt}{\isasymomega}\ {\isasymin}\ space\ M{\isachardot}{\kern0pt}\ P\ {\isasymomega}{\isacharbraceright}{\kern0pt}\ {\isasymunion}\ {\isacharbraceleft}{\kern0pt}{\isasymomega}\ {\isasymin}\ space\ M{\isachardot}{\kern0pt}\ Q\ {\isasymomega}{\isacharbraceright}{\kern0pt}{\isacharparenright}{\kern0pt}{\isachardoublequoteclose}\isanewline
\ \ \ \ \isacommand{apply}\isamarkupfalse%
\ {\isacharparenleft}{\kern0pt}rule\ arg{\isacharunderscore}{\kern0pt}cong{\isadigit{2}}{\isacharbrackleft}{\kern0pt}\isakeyword{where}\ f{\isacharequal}{\kern0pt}{\isachardoublequoteopen}measure{\isachardoublequoteclose}{\isacharbrackright}{\kern0pt}{\isacharcomma}{\kern0pt}\ simp{\isacharparenright}{\kern0pt}\isanewline
\ \ \ \ \isacommand{by}\isamarkupfalse%
\ {\isacharparenleft}{\kern0pt}subst\ set{\isacharunderscore}{\kern0pt}eq{\isacharunderscore}{\kern0pt}iff{\isacharcomma}{\kern0pt}\ rule\ allI{\isacharcomma}{\kern0pt}\ blast{\isacharparenright}{\kern0pt}\isanewline
\ \ \isacommand{also}\isamarkupfalse%
\ \isacommand{have}\isamarkupfalse%
\ {\isachardoublequoteopen}{\isachardot}{\kern0pt}{\isachardot}{\kern0pt}{\isachardot}{\kern0pt}\ {\isasymle}\ measure\ M\ {\isacharbraceleft}{\kern0pt}{\isasymomega}\ {\isasymin}\ space\ M{\isachardot}{\kern0pt}\ P\ {\isasymomega}{\isacharbraceright}{\kern0pt}\ {\isacharplus}{\kern0pt}\ measure\ M\ {\isacharbraceleft}{\kern0pt}{\isasymomega}\ {\isasymin}\ space\ M{\isachardot}{\kern0pt}\ Q\ {\isasymomega}{\isacharbraceright}{\kern0pt}{\isachardoublequoteclose}\isanewline
\ \ \ \ \isacommand{apply}\isamarkupfalse%
\ {\isacharparenleft}{\kern0pt}rule\ measure{\isacharunderscore}{\kern0pt}subadditive{\isacharparenright}{\kern0pt}\isanewline
\ \ \ \ \isacommand{apply}\isamarkupfalse%
\ {\isacharparenleft}{\kern0pt}metis\ {\isacharparenleft}{\kern0pt}no{\isacharunderscore}{\kern0pt}types{\isacharcomma}{\kern0pt}\ lifting{\isacharparenright}{\kern0pt}\ Collect{\isacharunderscore}{\kern0pt}cong\ mem{\isacharunderscore}{\kern0pt}Collect{\isacharunderscore}{\kern0pt}eq\ sets{\isachardot}{\kern0pt}sets{\isacharunderscore}{\kern0pt}into{\isacharunderscore}{\kern0pt}space\ subsetD\ assms{\isacharparenleft}{\kern0pt}{\isadigit{1}}{\isacharparenright}{\kern0pt}{\isacharparenright}{\kern0pt}\isanewline
\ \ \ \ \isacommand{apply}\isamarkupfalse%
\ {\isacharparenleft}{\kern0pt}metis\ {\isacharparenleft}{\kern0pt}no{\isacharunderscore}{\kern0pt}types{\isacharcomma}{\kern0pt}\ lifting{\isacharparenright}{\kern0pt}\ Collect{\isacharunderscore}{\kern0pt}cong\ mem{\isacharunderscore}{\kern0pt}Collect{\isacharunderscore}{\kern0pt}eq\ sets{\isachardot}{\kern0pt}sets{\isacharunderscore}{\kern0pt}into{\isacharunderscore}{\kern0pt}space\ subsetD\ assms{\isacharparenleft}{\kern0pt}{\isadigit{2}}{\isacharparenright}{\kern0pt}{\isacharparenright}{\kern0pt}\isanewline
\ \ \ \ \isacommand{by}\isamarkupfalse%
\ simp{\isacharplus}{\kern0pt}\isanewline
\ \ \isacommand{finally}\isamarkupfalse%
\ \isacommand{show}\isamarkupfalse%
\ {\isacharquery}{\kern0pt}thesis\ \isacommand{by}\isamarkupfalse%
\ simp\isanewline
\isacommand{qed}\isamarkupfalse%
%
\endisatagproof
{\isafoldproof}%
%
\isadelimproof
\isanewline
%
\endisadelimproof
\isanewline
\isacommand{lemma}\isamarkupfalse%
\ {\isacharparenleft}{\kern0pt}\isakeyword{in}\ prob{\isacharunderscore}{\kern0pt}space{\isacharparenright}{\kern0pt}\ prob{\isacharunderscore}{\kern0pt}sub{\isacharunderscore}{\kern0pt}additiveI{\isacharcolon}{\kern0pt}\isanewline
\ \ \isakeyword{assumes}\ {\isachardoublequoteopen}Collect\ P\ {\isasymin}\ sets\ M{\isachardoublequoteclose}\isanewline
\ \ \isakeyword{assumes}\ {\isachardoublequoteopen}Collect\ Q\ {\isasymin}\ sets\ M{\isachardoublequoteclose}\isanewline
\ \ \isakeyword{assumes}\ {\isachardoublequoteopen}{\isasymP}{\isacharparenleft}{\kern0pt}{\isasymomega}\ in\ M{\isachardot}{\kern0pt}\ P\ {\isasymomega}{\isacharparenright}{\kern0pt}\ {\isasymle}\ r{\isadigit{1}}{\isachardoublequoteclose}\isanewline
\ \ \isakeyword{assumes}\ {\isachardoublequoteopen}{\isasymP}{\isacharparenleft}{\kern0pt}{\isasymomega}\ in\ M{\isachardot}{\kern0pt}\ Q\ {\isasymomega}{\isacharparenright}{\kern0pt}\ {\isasymle}\ r{\isadigit{2}}{\isachardoublequoteclose}\isanewline
\ \ \isakeyword{shows}\ {\isachardoublequoteopen}{\isasymP}{\isacharparenleft}{\kern0pt}{\isasymomega}\ in\ M{\isachardot}{\kern0pt}\ P\ {\isasymomega}\ {\isasymor}\ Q\ {\isasymomega}{\isacharparenright}{\kern0pt}\ {\isasymle}\ r{\isadigit{1}}\ {\isacharplus}{\kern0pt}\ r{\isadigit{2}}{\isachardoublequoteclose}\isanewline
%
\isadelimproof
%
\endisadelimproof
%
\isatagproof
\isacommand{proof}\isamarkupfalse%
\ {\isacharminus}{\kern0pt}\isanewline
\ \ \isacommand{have}\isamarkupfalse%
\ {\isachardoublequoteopen}{\isasymP}{\isacharparenleft}{\kern0pt}{\isasymomega}\ in\ M{\isachardot}{\kern0pt}\ P\ {\isasymomega}\ {\isasymor}\ Q\ {\isasymomega}{\isacharparenright}{\kern0pt}\ {\isasymle}\ {\isasymP}{\isacharparenleft}{\kern0pt}{\isasymomega}\ in\ M{\isachardot}{\kern0pt}\ P\ {\isasymomega}{\isacharparenright}{\kern0pt}\ {\isacharplus}{\kern0pt}\ {\isasymP}{\isacharparenleft}{\kern0pt}{\isasymomega}\ in\ M{\isachardot}{\kern0pt}\ Q\ {\isasymomega}{\isacharparenright}{\kern0pt}{\isachardoublequoteclose}\isanewline
\ \ \ \ \isacommand{by}\isamarkupfalse%
\ {\isacharparenleft}{\kern0pt}rule\ \ prob{\isacharunderscore}{\kern0pt}sub{\isacharunderscore}{\kern0pt}additive{\isacharbrackleft}{\kern0pt}OF\ assms{\isacharparenleft}{\kern0pt}{\isadigit{1}}{\isacharparenright}{\kern0pt}\ assms{\isacharparenleft}{\kern0pt}{\isadigit{2}}{\isacharparenright}{\kern0pt}{\isacharbrackright}{\kern0pt}{\isacharparenright}{\kern0pt}\isanewline
\ \ \isacommand{also}\isamarkupfalse%
\ \isacommand{have}\isamarkupfalse%
\ {\isachardoublequoteopen}{\isachardot}{\kern0pt}{\isachardot}{\kern0pt}{\isachardot}{\kern0pt}\ {\isasymle}\ r{\isadigit{1}}\ {\isacharplus}{\kern0pt}\ r{\isadigit{2}}{\isachardoublequoteclose}\isanewline
\ \ \ \ \isacommand{by}\isamarkupfalse%
\ {\isacharparenleft}{\kern0pt}rule\ add{\isacharunderscore}{\kern0pt}mono{\isacharcomma}{\kern0pt}\ metis\ assms{\isacharparenleft}{\kern0pt}{\isadigit{3}}{\isacharparenright}{\kern0pt}{\isacharcomma}{\kern0pt}\ metis\ assms{\isacharparenleft}{\kern0pt}{\isadigit{4}}{\isacharparenright}{\kern0pt}{\isacharparenright}{\kern0pt}\isanewline
\ \ \isacommand{finally}\isamarkupfalse%
\ \isacommand{show}\isamarkupfalse%
\ {\isacharquery}{\kern0pt}thesis\ \isacommand{by}\isamarkupfalse%
\ simp\isanewline
\isacommand{qed}\isamarkupfalse%
%
\endisatagproof
{\isafoldproof}%
%
\isadelimproof
\isanewline
%
\endisadelimproof
\isanewline
\isacommand{lemma}\isamarkupfalse%
\ {\isacharparenleft}{\kern0pt}\isakeyword{in}\ prob{\isacharunderscore}{\kern0pt}space{\isacharparenright}{\kern0pt}\ prob{\isacharunderscore}{\kern0pt}mono{\isacharcolon}{\kern0pt}\isanewline
\ \ \isakeyword{assumes}\ {\isachardoublequoteopen}Collect\ Q\ {\isasymin}\ sets\ M{\isachardoublequoteclose}\isanewline
\ \ \isakeyword{assumes}\ {\isachardoublequoteopen}{\isasymAnd}{\isasymomega}{\isachardot}{\kern0pt}\ {\isasymomega}\ {\isasymin}\ space\ M\ {\isasymLongrightarrow}\ P\ {\isasymomega}\ {\isasymLongrightarrow}\ Q\ {\isasymomega}{\isachardoublequoteclose}\isanewline
\ \ \isakeyword{shows}\ {\isachardoublequoteopen}{\isasymP}{\isacharparenleft}{\kern0pt}{\isasymomega}\ in\ M{\isachardot}{\kern0pt}\ P\ {\isasymomega}{\isacharparenright}{\kern0pt}\ {\isasymle}\ {\isasymP}{\isacharparenleft}{\kern0pt}{\isasymomega}\ in\ M{\isachardot}{\kern0pt}\ Q\ {\isasymomega}{\isacharparenright}{\kern0pt}{\isachardoublequoteclose}\isanewline
%
\isadelimproof
\ \ %
\endisadelimproof
%
\isatagproof
\isacommand{apply}\isamarkupfalse%
\ {\isacharparenleft}{\kern0pt}rule\ finite{\isacharunderscore}{\kern0pt}measure{\isachardot}{\kern0pt}finite{\isacharunderscore}{\kern0pt}measure{\isacharunderscore}{\kern0pt}mono{\isacharparenright}{\kern0pt}\isanewline
\ \ \ \ \isacommand{apply}\isamarkupfalse%
\ simp\isanewline
\ \ \isacommand{apply}\isamarkupfalse%
\ {\isacharparenleft}{\kern0pt}rule\ subsetI{\isacharcomma}{\kern0pt}\ simp\ add{\isacharcolon}{\kern0pt}assms{\isacharparenleft}{\kern0pt}{\isadigit{2}}{\isacharparenright}{\kern0pt}{\isacharparenright}{\kern0pt}\isanewline
\ \ \isacommand{by}\isamarkupfalse%
\ {\isacharparenleft}{\kern0pt}metis\ {\isacharparenleft}{\kern0pt}no{\isacharunderscore}{\kern0pt}types{\isacharcomma}{\kern0pt}\ lifting{\isacharparenright}{\kern0pt}\ assms{\isacharparenleft}{\kern0pt}{\isadigit{1}}{\isacharparenright}{\kern0pt}\ \ Collect{\isacharunderscore}{\kern0pt}cong\ mem{\isacharunderscore}{\kern0pt}Collect{\isacharunderscore}{\kern0pt}eq\ sets{\isachardot}{\kern0pt}sets{\isacharunderscore}{\kern0pt}into{\isacharunderscore}{\kern0pt}space\ subsetD{\isacharparenright}{\kern0pt}%
\endisatagproof
{\isafoldproof}%
%
\isadelimproof
\isanewline
%
\endisadelimproof
\isanewline
\isacommand{lemma}\isamarkupfalse%
\ in{\isacharunderscore}{\kern0pt}events{\isacharunderscore}{\kern0pt}pmf{\isacharcolon}{\kern0pt}\ {\isachardoublequoteopen}A\ {\isasymin}\ measure{\isacharunderscore}{\kern0pt}pmf{\isachardot}{\kern0pt}events\ {\isasymOmega}{\isachardoublequoteclose}\isanewline
%
\isadelimproof
\ \ %
\endisadelimproof
%
\isatagproof
\isacommand{by}\isamarkupfalse%
\ simp%
\endisatagproof
{\isafoldproof}%
%
\isadelimproof
\isanewline
%
\endisadelimproof
\isanewline
\isacommand{lemma}\isamarkupfalse%
\ pmf{\isacharunderscore}{\kern0pt}add{\isacharcolon}{\kern0pt}\isanewline
\ \ \isakeyword{assumes}\ \ {\isachardoublequoteopen}{\isasymAnd}x{\isachardot}{\kern0pt}\ x\ {\isasymin}\ P\ {\isasymLongrightarrow}\ x\ {\isasymin}\ set{\isacharunderscore}{\kern0pt}pmf\ {\isasymOmega}\ {\isasymLongrightarrow}\ x\ {\isasymin}\ Q\ {\isasymor}\ x\ {\isasymin}\ R{\isachardoublequoteclose}\isanewline
\ \ \isakeyword{shows}\ {\isachardoublequoteopen}measure\ {\isacharparenleft}{\kern0pt}measure{\isacharunderscore}{\kern0pt}pmf\ {\isasymOmega}{\isacharparenright}{\kern0pt}\ P\ {\isasymle}\ measure\ {\isacharparenleft}{\kern0pt}measure{\isacharunderscore}{\kern0pt}pmf\ {\isasymOmega}{\isacharparenright}{\kern0pt}\ Q\ {\isacharplus}{\kern0pt}\ measure\ {\isacharparenleft}{\kern0pt}measure{\isacharunderscore}{\kern0pt}pmf\ {\isasymOmega}{\isacharparenright}{\kern0pt}\ R{\isachardoublequoteclose}\isanewline
%
\isadelimproof
%
\endisadelimproof
%
\isatagproof
\isacommand{proof}\isamarkupfalse%
\ {\isacharminus}{\kern0pt}\isanewline
\ \ \isacommand{have}\isamarkupfalse%
\ {\isachardoublequoteopen}measure\ {\isacharparenleft}{\kern0pt}measure{\isacharunderscore}{\kern0pt}pmf\ {\isasymOmega}{\isacharparenright}{\kern0pt}\ P\ {\isasymle}\ measure\ {\isacharparenleft}{\kern0pt}measure{\isacharunderscore}{\kern0pt}pmf\ {\isasymOmega}{\isacharparenright}{\kern0pt}\ {\isacharparenleft}{\kern0pt}Q\ {\isasymunion}\ R{\isacharparenright}{\kern0pt}{\isachardoublequoteclose}\isanewline
\ \ \ \ \isacommand{apply}\isamarkupfalse%
\ {\isacharparenleft}{\kern0pt}rule\ pmf{\isacharunderscore}{\kern0pt}mono{\isacharunderscore}{\kern0pt}{\isadigit{1}}{\isacharparenright}{\kern0pt}\isanewline
\ \ \ \ \isacommand{using}\isamarkupfalse%
\ assms\ \isacommand{by}\isamarkupfalse%
\ blast\isanewline
\ \ \isacommand{also}\isamarkupfalse%
\ \isacommand{have}\isamarkupfalse%
\ {\isachardoublequoteopen}{\isachardot}{\kern0pt}{\isachardot}{\kern0pt}{\isachardot}{\kern0pt}\ {\isasymle}\ measure\ {\isacharparenleft}{\kern0pt}measure{\isacharunderscore}{\kern0pt}pmf\ {\isasymOmega}{\isacharparenright}{\kern0pt}\ Q\ {\isacharplus}{\kern0pt}\ measure\ {\isacharparenleft}{\kern0pt}measure{\isacharunderscore}{\kern0pt}pmf\ {\isasymOmega}{\isacharparenright}{\kern0pt}\ R{\isachardoublequoteclose}\isanewline
\ \ \ \ \isacommand{by}\isamarkupfalse%
\ {\isacharparenleft}{\kern0pt}rule\ measure{\isacharunderscore}{\kern0pt}subadditive{\isacharcomma}{\kern0pt}\ simp{\isacharplus}{\kern0pt}{\isacharparenright}{\kern0pt}\isanewline
\ \ \isacommand{finally}\isamarkupfalse%
\ \isacommand{show}\isamarkupfalse%
\ {\isacharquery}{\kern0pt}thesis\ \isacommand{by}\isamarkupfalse%
\ simp\isanewline
\isacommand{qed}\isamarkupfalse%
%
\endisatagproof
{\isafoldproof}%
%
\isadelimproof
\isanewline
%
\endisadelimproof
\isanewline
\isacommand{lemma}\isamarkupfalse%
\ pmf{\isacharunderscore}{\kern0pt}mono{\isacharcolon}{\kern0pt}\isanewline
\ \ \isakeyword{assumes}\ {\isachardoublequoteopen}{\isasymAnd}x{\isachardot}{\kern0pt}\ x\ {\isasymin}\ P\ {\isasymLongrightarrow}\ x\ {\isasymin}\ Q{\isachardoublequoteclose}\isanewline
\ \ \isakeyword{shows}\ {\isachardoublequoteopen}measure\ {\isacharparenleft}{\kern0pt}measure{\isacharunderscore}{\kern0pt}pmf\ {\isasymOmega}{\isacharparenright}{\kern0pt}\ P\ {\isasymle}\ measure\ {\isacharparenleft}{\kern0pt}measure{\isacharunderscore}{\kern0pt}pmf\ {\isasymOmega}{\isacharparenright}{\kern0pt}\ Q{\isachardoublequoteclose}\isanewline
%
\isadelimproof
\ \ %
\endisadelimproof
%
\isatagproof
\isacommand{apply}\isamarkupfalse%
\ {\isacharparenleft}{\kern0pt}rule\ pmf{\isacharunderscore}{\kern0pt}mono{\isacharunderscore}{\kern0pt}{\isadigit{1}}{\isacharparenright}{\kern0pt}\ \isacommand{using}\isamarkupfalse%
\ assms\ \isacommand{by}\isamarkupfalse%
\ auto%
\endisatagproof
{\isafoldproof}%
%
\isadelimproof
\isanewline
%
\endisadelimproof
\isanewline
\isacommand{lemma}\isamarkupfalse%
\ abs{\isacharunderscore}{\kern0pt}ge{\isacharunderscore}{\kern0pt}iff{\isacharcolon}{\kern0pt}\ {\isachardoublequoteopen}{\isacharparenleft}{\kern0pt}{\isacharparenleft}{\kern0pt}x{\isacharcolon}{\kern0pt}{\isacharcolon}{\kern0pt}real{\isacharparenright}{\kern0pt}\ {\isasymle}\ abs\ y{\isacharparenright}{\kern0pt}\ {\isacharequal}{\kern0pt}\ {\isacharparenleft}{\kern0pt}x\ {\isasymle}\ y\ {\isasymor}\ x\ {\isasymle}\ {\isacharminus}{\kern0pt}y{\isacharparenright}{\kern0pt}{\isachardoublequoteclose}\isanewline
%
\isadelimproof
\ \ %
\endisadelimproof
%
\isatagproof
\isacommand{by}\isamarkupfalse%
\ linarith%
\endisatagproof
{\isafoldproof}%
%
\isadelimproof
\isanewline
%
\endisadelimproof
\isanewline
\isacommand{lemma}\isamarkupfalse%
\ two{\isacharunderscore}{\kern0pt}powr{\isacharunderscore}{\kern0pt}{\isadigit{0}}{\isacharcolon}{\kern0pt}\ {\isachardoublequoteopen}{\isadigit{2}}\ powr\ {\isacharparenleft}{\kern0pt}{\isadigit{0}}{\isacharcolon}{\kern0pt}{\isacharcolon}{\kern0pt}real{\isacharparenright}{\kern0pt}\ {\isacharequal}{\kern0pt}\ {\isadigit{1}}{\isachardoublequoteclose}\isanewline
%
\isadelimproof
\ \ %
\endisadelimproof
%
\isatagproof
\isacommand{by}\isamarkupfalse%
\ simp%
\endisatagproof
{\isafoldproof}%
%
\isadelimproof
\isanewline
%
\endisadelimproof
\isanewline
\isacommand{lemma}\isamarkupfalse%
\ count{\isacharunderscore}{\kern0pt}nat{\isacharunderscore}{\kern0pt}abs{\isacharunderscore}{\kern0pt}diff{\isacharunderscore}{\kern0pt}{\isadigit{2}}{\isacharcolon}{\kern0pt}\isanewline
\ \ \isakeyword{fixes}\ x\ {\isacharcolon}{\kern0pt}{\isacharcolon}{\kern0pt}\ nat\isanewline
\ \ \isakeyword{fixes}\ q\ {\isacharcolon}{\kern0pt}{\isacharcolon}{\kern0pt}\ real\isanewline
\ \ \isakeyword{assumes}\ {\isachardoublequoteopen}q\ {\isasymge}\ {\isadigit{0}}{\isachardoublequoteclose}\isanewline
\ \ \isakeyword{defines}\ {\isachardoublequoteopen}A\ {\isasymequiv}\ {\isacharbraceleft}{\kern0pt}{\isacharparenleft}{\kern0pt}k{\isacharcolon}{\kern0pt}{\isacharcolon}{\kern0pt}nat{\isacharparenright}{\kern0pt}{\isachardot}{\kern0pt}\ abs\ {\isacharparenleft}{\kern0pt}real\ x\ {\isacharminus}{\kern0pt}\ real\ k{\isacharparenright}{\kern0pt}\ {\isasymle}\ q\ {\isasymand}\ k\ {\isasymnoteq}\ x{\isacharbraceright}{\kern0pt}{\isachardoublequoteclose}\isanewline
\ \ \isakeyword{shows}\ {\isachardoublequoteopen}real\ {\isacharparenleft}{\kern0pt}card\ A{\isacharparenright}{\kern0pt}\ {\isasymle}\ {\isadigit{2}}\ {\isacharasterisk}{\kern0pt}\ q{\isachardoublequoteclose}\ \isakeyword{and}\ {\isachardoublequoteopen}finite\ A{\isachardoublequoteclose}\isanewline
%
\isadelimproof
%
\endisadelimproof
%
\isatagproof
\isacommand{proof}\isamarkupfalse%
\ {\isacharminus}{\kern0pt}\isanewline
\ \ \isacommand{have}\isamarkupfalse%
\ a{\isacharcolon}{\kern0pt}\ {\isachardoublequoteopen}of{\isacharunderscore}{\kern0pt}nat\ x\ {\isasymin}\ {\isacharbraceleft}{\kern0pt}{\isasymlceil}real\ x{\isacharminus}{\kern0pt}q{\isasymrceil}{\isachardot}{\kern0pt}{\isachardot}{\kern0pt}{\isasymlfloor}real\ x{\isacharplus}{\kern0pt}q{\isasymrfloor}{\isacharbraceright}{\kern0pt}{\isachardoublequoteclose}\isanewline
\ \ \ \ \isacommand{using}\isamarkupfalse%
\ assms\ \isanewline
\ \ \ \ \isacommand{by}\isamarkupfalse%
\ {\isacharparenleft}{\kern0pt}simp\ add{\isacharcolon}{\kern0pt}\ ceiling{\isacharunderscore}{\kern0pt}le{\isacharunderscore}{\kern0pt}iff{\isacharparenright}{\kern0pt}\isanewline
\ \ \isanewline
\ \ \isacommand{have}\isamarkupfalse%
\ {\isachardoublequoteopen}card\ A\ {\isacharequal}{\kern0pt}\ card\ {\isacharparenleft}{\kern0pt}int\ {\isacharbackquote}{\kern0pt}\ A{\isacharparenright}{\kern0pt}{\isachardoublequoteclose}\isanewline
\ \ \ \ \isacommand{by}\isamarkupfalse%
\ {\isacharparenleft}{\kern0pt}rule\ card{\isacharunderscore}{\kern0pt}image{\isacharbrackleft}{\kern0pt}symmetric{\isacharbrackright}{\kern0pt}{\isacharcomma}{\kern0pt}\ simp{\isacharparenright}{\kern0pt}\isanewline
\ \ \isacommand{also}\isamarkupfalse%
\ \isacommand{have}\isamarkupfalse%
\ {\isachardoublequoteopen}{\isachardot}{\kern0pt}{\isachardot}{\kern0pt}{\isachardot}{\kern0pt}\ {\isasymle}\ card\ {\isacharparenleft}{\kern0pt}{\isacharbraceleft}{\kern0pt}{\isasymlceil}real\ x{\isacharminus}{\kern0pt}q{\isasymrceil}{\isachardot}{\kern0pt}{\isachardot}{\kern0pt}{\isasymlfloor}real\ x{\isacharplus}{\kern0pt}q{\isasymrfloor}{\isacharbraceright}{\kern0pt}\ {\isacharminus}{\kern0pt}\ {\isacharbraceleft}{\kern0pt}of{\isacharunderscore}{\kern0pt}nat\ x{\isacharbraceright}{\kern0pt}{\isacharparenright}{\kern0pt}{\isachardoublequoteclose}\isanewline
\ \ \ \ \isacommand{apply}\isamarkupfalse%
\ {\isacharparenleft}{\kern0pt}rule\ card{\isacharunderscore}{\kern0pt}mono{\isacharcomma}{\kern0pt}\ simp{\isacharparenright}{\kern0pt}\isanewline
\ \ \ \ \isacommand{apply}\isamarkupfalse%
\ {\isacharparenleft}{\kern0pt}rule\ image{\isacharunderscore}{\kern0pt}subsetI{\isacharparenright}{\kern0pt}\isanewline
\ \ \ \ \isacommand{apply}\isamarkupfalse%
\ {\isacharparenleft}{\kern0pt}simp\ add{\isacharcolon}{\kern0pt}A{\isacharunderscore}{\kern0pt}def\ abs{\isacharunderscore}{\kern0pt}le{\isacharunderscore}{\kern0pt}iff{\isacharparenright}{\kern0pt}\isanewline
\ \ \ \ \isacommand{by}\isamarkupfalse%
\ linarith\isanewline
\ \ \isacommand{also}\isamarkupfalse%
\ \isacommand{have}\isamarkupfalse%
\ {\isachardoublequoteopen}{\isachardot}{\kern0pt}{\isachardot}{\kern0pt}{\isachardot}{\kern0pt}\ {\isacharequal}{\kern0pt}\ card\ {\isacharbraceleft}{\kern0pt}{\isasymlceil}real\ x{\isacharminus}{\kern0pt}q{\isasymrceil}{\isachardot}{\kern0pt}{\isachardot}{\kern0pt}{\isasymlfloor}real\ x{\isacharplus}{\kern0pt}q{\isasymrfloor}{\isacharbraceright}{\kern0pt}\ {\isacharminus}{\kern0pt}\ {\isadigit{1}}{\isachardoublequoteclose}\isanewline
\ \ \ \ \isacommand{by}\isamarkupfalse%
\ {\isacharparenleft}{\kern0pt}rule\ card{\isacharunderscore}{\kern0pt}Diff{\isacharunderscore}{\kern0pt}singleton{\isacharcomma}{\kern0pt}\ rule\ a{\isacharparenright}{\kern0pt}\isanewline
\ \ \isacommand{also}\isamarkupfalse%
\ \isacommand{have}\isamarkupfalse%
\ {\isachardoublequoteopen}{\isachardot}{\kern0pt}{\isachardot}{\kern0pt}{\isachardot}{\kern0pt}\ {\isacharequal}{\kern0pt}\ int\ {\isacharparenleft}{\kern0pt}card\ {\isacharbraceleft}{\kern0pt}{\isasymlceil}real\ x{\isacharminus}{\kern0pt}q{\isasymrceil}{\isachardot}{\kern0pt}{\isachardot}{\kern0pt}{\isasymlfloor}real\ x{\isacharplus}{\kern0pt}q{\isasymrfloor}{\isacharbraceright}{\kern0pt}{\isacharparenright}{\kern0pt}\ {\isacharminus}{\kern0pt}\ int\ {\isadigit{1}}{\isachardoublequoteclose}\isanewline
\ \ \ \ \isacommand{apply}\isamarkupfalse%
\ {\isacharparenleft}{\kern0pt}rule\ of{\isacharunderscore}{\kern0pt}nat{\isacharunderscore}{\kern0pt}diff{\isacharparenright}{\kern0pt}\isanewline
\ \ \ \ \isacommand{by}\isamarkupfalse%
\ {\isacharparenleft}{\kern0pt}metis\ a\ card{\isacharunderscore}{\kern0pt}{\isadigit{0}}{\isacharunderscore}{\kern0pt}eq\ empty{\isacharunderscore}{\kern0pt}iff\ finite{\isacharunderscore}{\kern0pt}atLeastAtMost{\isacharunderscore}{\kern0pt}int\ less{\isacharunderscore}{\kern0pt}one\ linorder{\isacharunderscore}{\kern0pt}not{\isacharunderscore}{\kern0pt}le{\isacharparenright}{\kern0pt}\isanewline
\ \ \isacommand{also}\isamarkupfalse%
\ \isacommand{have}\isamarkupfalse%
\ {\isachardoublequoteopen}{\isachardot}{\kern0pt}{\isachardot}{\kern0pt}{\isachardot}{\kern0pt}\ {\isasymle}\ {\isasymlfloor}q{\isacharplus}{\kern0pt}real\ x{\isasymrfloor}{\isacharplus}{\kern0pt}{\isadigit{1}}\ {\isacharminus}{\kern0pt}{\isasymlceil}real\ x{\isacharminus}{\kern0pt}q{\isasymrceil}\ {\isacharminus}{\kern0pt}\ {\isadigit{1}}{\isachardoublequoteclose}\isanewline
\ \ \ \ \isacommand{using}\isamarkupfalse%
\ assms\isanewline
\ \ \ \ \isacommand{apply}\isamarkupfalse%
\ simp\isanewline
\ \ \ \ \isacommand{by}\isamarkupfalse%
\ linarith\isanewline
\ \ \isacommand{also}\isamarkupfalse%
\ \isacommand{have}\isamarkupfalse%
\ {\isachardoublequoteopen}{\isachardot}{\kern0pt}{\isachardot}{\kern0pt}{\isachardot}{\kern0pt}\ {\isasymle}\ {\isadigit{2}}{\isacharasterisk}{\kern0pt}q{\isachardoublequoteclose}\isanewline
\ \ \ \ \isacommand{by}\isamarkupfalse%
\ linarith\isanewline
\ \ \isacommand{finally}\isamarkupfalse%
\ \isacommand{show}\isamarkupfalse%
\ {\isachardoublequoteopen}card\ A\ {\isasymle}\ {\isadigit{2}}\ {\isacharasterisk}{\kern0pt}\ q{\isachardoublequoteclose}\isanewline
\ \ \ \ \isacommand{by}\isamarkupfalse%
\ simp\isanewline
\ \ \isacommand{show}\isamarkupfalse%
\ {\isachardoublequoteopen}finite\ A{\isachardoublequoteclose}\isanewline
\ \ \ \ \isacommand{apply}\isamarkupfalse%
\ {\isacharparenleft}{\kern0pt}simp\ add{\isacharcolon}{\kern0pt}A{\isacharunderscore}{\kern0pt}def{\isacharparenright}{\kern0pt}\isanewline
\ \ \ \ \isacommand{apply}\isamarkupfalse%
\ {\isacharparenleft}{\kern0pt}rule\ finite{\isacharunderscore}{\kern0pt}subset{\isacharbrackleft}{\kern0pt}\isakeyword{where}\ B{\isacharequal}{\kern0pt}{\isachardoublequoteopen}{\isacharbraceleft}{\kern0pt}{\isadigit{0}}{\isachardot}{\kern0pt}{\isachardot}{\kern0pt}x{\isacharplus}{\kern0pt}nat\ {\isasymlceil}q{\isasymrceil}{\isacharbraceright}{\kern0pt}{\isachardoublequoteclose}{\isacharbrackright}{\kern0pt}{\isacharparenright}{\kern0pt}\isanewline
\ \ \ \ \isacommand{apply}\isamarkupfalse%
\ {\isacharparenleft}{\kern0pt}rule\ subsetI{\isacharcomma}{\kern0pt}\ simp\ add{\isacharcolon}{\kern0pt}abs{\isacharunderscore}{\kern0pt}le{\isacharunderscore}{\kern0pt}iff{\isacharparenright}{\kern0pt}\isanewline
\ \ \ \ \isacommand{using}\isamarkupfalse%
\ assms\ \isacommand{apply}\isamarkupfalse%
\ linarith\ \isacommand{by}\isamarkupfalse%
\ simp\isanewline
\isacommand{qed}\isamarkupfalse%
%
\endisatagproof
{\isafoldproof}%
%
\isadelimproof
\isanewline
%
\endisadelimproof
\isanewline
\isacommand{lemma}\isamarkupfalse%
\ f{\isadigit{0}}{\isacharunderscore}{\kern0pt}collision{\isacharunderscore}{\kern0pt}prob{\isacharcolon}{\kern0pt}\isanewline
\ \ \isakeyword{fixes}\ p\ {\isacharcolon}{\kern0pt}{\isacharcolon}{\kern0pt}\ nat\isanewline
\ \ \isakeyword{assumes}\ {\isachardoublequoteopen}Factorial{\isacharunderscore}{\kern0pt}Ring{\isachardot}{\kern0pt}prime\ p{\isachardoublequoteclose}\isanewline
\ \ \isakeyword{defines}\ {\isachardoublequoteopen}{\isasymOmega}\ {\isasymequiv}\ pmf{\isacharunderscore}{\kern0pt}of{\isacharunderscore}{\kern0pt}set\ {\isacharparenleft}{\kern0pt}bounded{\isacharunderscore}{\kern0pt}degree{\isacharunderscore}{\kern0pt}polynomials\ {\isacharparenleft}{\kern0pt}ZFact\ {\isacharparenleft}{\kern0pt}int\ p{\isacharparenright}{\kern0pt}{\isacharparenright}{\kern0pt}\ {\isadigit{2}}{\isacharparenright}{\kern0pt}{\isachardoublequoteclose}\isanewline
\ \ \isakeyword{assumes}\ {\isachardoublequoteopen}M\ {\isasymsubseteq}\ {\isacharbraceleft}{\kern0pt}{\isadigit{0}}{\isachardot}{\kern0pt}{\isachardot}{\kern0pt}{\isacharless}{\kern0pt}p{\isacharbraceright}{\kern0pt}{\isachardoublequoteclose}\isanewline
\ \ \isakeyword{assumes}\ {\isachardoublequoteopen}c\ {\isasymge}\ {\isadigit{1}}{\isachardoublequoteclose}\isanewline
\ \ \isakeyword{assumes}\ {\isachardoublequoteopen}r\ {\isasymge}\ {\isadigit{1}}{\isachardoublequoteclose}\isanewline
\ \ \isakeyword{shows}\ {\isachardoublequoteopen}{\isasymP}{\isacharparenleft}{\kern0pt}{\isasymomega}\ in\ measure{\isacharunderscore}{\kern0pt}pmf\ {\isasymOmega}{\isachardot}{\kern0pt}\ \isanewline
\ \ \ \ {\isasymexists}x\ {\isasymin}\ M{\isachardot}{\kern0pt}\ {\isasymexists}y\ {\isasymin}\ M{\isachardot}{\kern0pt}\isanewline
\ \ \ \ x\ {\isasymnoteq}\ y\ {\isasymand}\isanewline
\ \ \ \ truncate{\isacharunderscore}{\kern0pt}down\ r\ {\isacharparenleft}{\kern0pt}hash\ p\ x\ {\isasymomega}{\isacharparenright}{\kern0pt}\ {\isasymle}\ c\ {\isasymand}\isanewline
\ \ \ \ truncate{\isacharunderscore}{\kern0pt}down\ r\ {\isacharparenleft}{\kern0pt}hash\ p\ x\ {\isasymomega}{\isacharparenright}{\kern0pt}\ {\isacharequal}{\kern0pt}\ truncate{\isacharunderscore}{\kern0pt}down\ r\ {\isacharparenleft}{\kern0pt}hash\ p\ y\ {\isasymomega}{\isacharparenright}{\kern0pt}{\isacharparenright}{\kern0pt}\ {\isasymle}\ \isanewline
\ \ \ \ {\isadigit{6}}\ {\isacharasterisk}{\kern0pt}\ {\isacharparenleft}{\kern0pt}real\ {\isacharparenleft}{\kern0pt}card\ M{\isacharparenright}{\kern0pt}{\isacharparenright}{\kern0pt}\isactrlsup {\isadigit{2}}\ {\isacharasterisk}{\kern0pt}\ c\isactrlsup {\isadigit{2}}\ {\isacharasterisk}{\kern0pt}\ {\isadigit{2}}\ powr\ {\isacharminus}{\kern0pt}r\ {\isacharslash}{\kern0pt}\ {\isacharparenleft}{\kern0pt}real\ p{\isacharparenright}{\kern0pt}\ \isactrlsup {\isadigit{2}}\ {\isacharplus}{\kern0pt}\ {\isadigit{1}}{\isacharslash}{\kern0pt}real\ p{\isachardoublequoteclose}\ {\isacharparenleft}{\kern0pt}\isakeyword{is}\ {\isachardoublequoteopen}{\isasymP}{\isacharparenleft}{\kern0pt}{\isasymomega}\ in\ {\isacharunderscore}{\kern0pt}{\isachardot}{\kern0pt}\ {\isacharquery}{\kern0pt}l\ {\isasymomega}{\isacharparenright}{\kern0pt}\ {\isasymle}\ {\isacharquery}{\kern0pt}r{\isadigit{1}}\ {\isacharplus}{\kern0pt}\ {\isacharquery}{\kern0pt}r{\isadigit{2}}{\isachardoublequoteclose}{\isacharparenright}{\kern0pt}\isanewline
%
\isadelimproof
%
\endisadelimproof
%
\isatagproof
\isacommand{proof}\isamarkupfalse%
\ {\isacharminus}{\kern0pt}\isanewline
\ \ \isacommand{have}\isamarkupfalse%
\ p{\isacharunderscore}{\kern0pt}ge{\isacharunderscore}{\kern0pt}{\isadigit{0}}{\isacharcolon}{\kern0pt}\ {\isachardoublequoteopen}p\ {\isachargreater}{\kern0pt}\ {\isadigit{0}}{\isachardoublequoteclose}\isanewline
\ \ \ \ \isacommand{using}\isamarkupfalse%
\ assms\ prime{\isacharunderscore}{\kern0pt}gt{\isacharunderscore}{\kern0pt}{\isadigit{0}}{\isacharunderscore}{\kern0pt}nat\ \isacommand{by}\isamarkupfalse%
\ blast\isanewline
\isanewline
\ \ \isacommand{have}\isamarkupfalse%
\ c{\isacharunderscore}{\kern0pt}ge{\isacharunderscore}{\kern0pt}{\isadigit{0}}{\isacharcolon}{\kern0pt}\ {\isachardoublequoteopen}c{\isasymge}{\isadigit{0}}{\isachardoublequoteclose}\ \isacommand{using}\isamarkupfalse%
\ assms\ \isacommand{by}\isamarkupfalse%
\ simp\isanewline
\ \ \isanewline
\ \ \isacommand{have}\isamarkupfalse%
\ two{\isacharunderscore}{\kern0pt}pow{\isacharunderscore}{\kern0pt}r{\isacharunderscore}{\kern0pt}le{\isacharunderscore}{\kern0pt}{\isadigit{1}}{\isacharcolon}{\kern0pt}\ {\isachardoublequoteopen}{\isadigit{2}}\ powr\ {\isacharparenleft}{\kern0pt}{\isacharminus}{\kern0pt}real\ r{\isacharparenright}{\kern0pt}\ {\isasymle}\ {\isadigit{1}}{\isachardoublequoteclose}\ \isanewline
\ \ \ \ \isacommand{by}\isamarkupfalse%
\ {\isacharparenleft}{\kern0pt}subst\ two{\isacharunderscore}{\kern0pt}powr{\isacharunderscore}{\kern0pt}{\isadigit{0}}{\isacharbrackleft}{\kern0pt}symmetric{\isacharbrackright}{\kern0pt}{\isacharcomma}{\kern0pt}\ rule\ powr{\isacharunderscore}{\kern0pt}mono{\isacharcomma}{\kern0pt}\ simp{\isacharcomma}{\kern0pt}\ simp{\isacharparenright}{\kern0pt}\isanewline
\isanewline
\ \ \isacommand{have}\isamarkupfalse%
\ f{\isacharunderscore}{\kern0pt}M{\isacharcolon}{\kern0pt}\ {\isachardoublequoteopen}finite\ M{\isachardoublequoteclose}\isanewline
\ \ \ \ \isacommand{by}\isamarkupfalse%
\ {\isacharparenleft}{\kern0pt}rule\ finite{\isacharunderscore}{\kern0pt}subset{\isacharbrackleft}{\kern0pt}\isakeyword{where}\ B{\isacharequal}{\kern0pt}{\isachardoublequoteopen}{\isacharbraceleft}{\kern0pt}{\isadigit{0}}{\isachardot}{\kern0pt}{\isachardot}{\kern0pt}{\isacharless}{\kern0pt}p{\isacharbraceright}{\kern0pt}{\isachardoublequoteclose}{\isacharbrackright}{\kern0pt}{\isacharcomma}{\kern0pt}\ metis\ assms{\isacharparenleft}{\kern0pt}{\isadigit{3}}{\isacharparenright}{\kern0pt}{\isacharcomma}{\kern0pt}\ simp{\isacharparenright}{\kern0pt}\isanewline
\isanewline
\ \ \isacommand{have}\isamarkupfalse%
\ a{\isadigit{2}}{\isacharcolon}{\kern0pt}\ {\isachardoublequoteopen}{\isasymAnd}{\isasymomega}\ x{\isachardot}{\kern0pt}\ x\ {\isacharless}{\kern0pt}\ p\ {\isasymLongrightarrow}\ {\isasymomega}\ {\isasymin}\ bounded{\isacharunderscore}{\kern0pt}degree{\isacharunderscore}{\kern0pt}polynomials\ {\isacharparenleft}{\kern0pt}ZFact\ p{\isacharparenright}{\kern0pt}\ {\isadigit{2}}\ {\isasymLongrightarrow}\ hash\ p\ x\ {\isasymomega}\ {\isacharless}{\kern0pt}\ p{\isachardoublequoteclose}\ \isanewline
\ \ \ \ \isacommand{using}\isamarkupfalse%
\ hash{\isacharunderscore}{\kern0pt}range{\isacharbrackleft}{\kern0pt}OF\ p{\isacharunderscore}{\kern0pt}ge{\isacharunderscore}{\kern0pt}{\isadigit{0}}{\isacharbrackright}{\kern0pt}\ \isacommand{by}\isamarkupfalse%
\ simp\isanewline
\ \ \isacommand{have}\isamarkupfalse%
\ {\isachardoublequoteopen}{\isasymAnd}{\isasymomega}{\isachardot}{\kern0pt}\ degree\ {\isasymomega}\ {\isasymge}\ {\isadigit{1}}\ {\isasymLongrightarrow}\ {\isasymomega}\ {\isasymin}\ bounded{\isacharunderscore}{\kern0pt}degree{\isacharunderscore}{\kern0pt}polynomials\ {\isacharparenleft}{\kern0pt}ZFact\ p{\isacharparenright}{\kern0pt}\ {\isadigit{2}}\ {\isasymLongrightarrow}\ degree\ {\isasymomega}\ {\isacharequal}{\kern0pt}\ {\isadigit{1}}{\isachardoublequoteclose}\isanewline
\ \ \ \ \isacommand{apply}\isamarkupfalse%
\ {\isacharparenleft}{\kern0pt}simp\ add{\isacharcolon}{\kern0pt}bounded{\isacharunderscore}{\kern0pt}degree{\isacharunderscore}{\kern0pt}polynomials{\isacharunderscore}{\kern0pt}def{\isacharparenright}{\kern0pt}\ \isanewline
\ \ \ \ \isacommand{by}\isamarkupfalse%
\ {\isacharparenleft}{\kern0pt}metis\ One{\isacharunderscore}{\kern0pt}nat{\isacharunderscore}{\kern0pt}def\ Suc{\isacharunderscore}{\kern0pt}{\isadigit{1}}\ le{\isacharunderscore}{\kern0pt}less{\isacharunderscore}{\kern0pt}Suc{\isacharunderscore}{\kern0pt}eq\ less{\isacharunderscore}{\kern0pt}imp{\isacharunderscore}{\kern0pt}diff{\isacharunderscore}{\kern0pt}less\ list{\isachardot}{\kern0pt}size{\isacharparenleft}{\kern0pt}{\isadigit{3}}{\isacharparenright}{\kern0pt}\ pos{\isadigit{2}}{\isacharparenright}{\kern0pt}\isanewline
\ \ \isacommand{hence}\isamarkupfalse%
\ a{\isadigit{3}}{\isacharcolon}{\kern0pt}\ {\isachardoublequoteopen}{\isasymAnd}{\isasymomega}\ x\ y{\isachardot}{\kern0pt}\ x\ {\isacharless}{\kern0pt}\ p\ {\isasymLongrightarrow}\ y\ {\isacharless}{\kern0pt}\ p\ {\isasymLongrightarrow}\ \ x\ {\isasymnoteq}\ y\ {\isasymLongrightarrow}\ degree\ {\isasymomega}\ {\isasymge}\ {\isadigit{1}}\ {\isasymLongrightarrow}\ \isanewline
\ \ \ \ {\isasymomega}\ {\isasymin}\ bounded{\isacharunderscore}{\kern0pt}degree{\isacharunderscore}{\kern0pt}polynomials\ {\isacharparenleft}{\kern0pt}ZFact\ p{\isacharparenright}{\kern0pt}\ {\isadigit{2}}\ {\isasymLongrightarrow}\ \isanewline
\ \ \ \ hash\ p\ x\ {\isasymomega}\ {\isasymnoteq}\ hash\ p\ y\ {\isasymomega}{\isachardoublequoteclose}\ \isanewline
\ \ \ \ \isacommand{using}\isamarkupfalse%
\ hash{\isacharunderscore}{\kern0pt}inj{\isacharunderscore}{\kern0pt}if{\isacharunderscore}{\kern0pt}degree{\isacharunderscore}{\kern0pt}{\isadigit{1}}{\isacharbrackleft}{\kern0pt}OF\ assms{\isacharparenleft}{\kern0pt}{\isadigit{1}}{\isacharparenright}{\kern0pt}{\isacharbrackright}{\kern0pt}\ \isanewline
\ \ \ \ \isacommand{by}\isamarkupfalse%
\ {\isacharparenleft}{\kern0pt}meson\ atLeastLessThan{\isacharunderscore}{\kern0pt}iff\ inj{\isacharunderscore}{\kern0pt}on{\isacharunderscore}{\kern0pt}def\ less{\isacharunderscore}{\kern0pt}nat{\isacharunderscore}{\kern0pt}zero{\isacharunderscore}{\kern0pt}code\ linorder{\isacharunderscore}{\kern0pt}not{\isacharunderscore}{\kern0pt}less{\isacharparenright}{\kern0pt}\isanewline
\isanewline
\ \ \isacommand{have}\isamarkupfalse%
\ a{\isadigit{1}}{\isacharcolon}{\kern0pt}\ \isanewline
\ \ \ \ {\isachardoublequoteopen}{\isasymAnd}x\ y{\isachardot}{\kern0pt}\ x\ {\isacharless}{\kern0pt}\ y\ {\isasymLongrightarrow}\ x\ {\isasymin}\ M\ {\isasymLongrightarrow}\ y\ {\isasymin}\ M\ {\isasymLongrightarrow}\ measure\ {\isasymOmega}\ \isanewline
\ \ \ \ {\isacharbraceleft}{\kern0pt}{\isasymomega}{\isachardot}{\kern0pt}\ degree\ {\isasymomega}\ {\isasymge}\ {\isadigit{1}}\ {\isasymand}\ truncate{\isacharunderscore}{\kern0pt}down\ r\ {\isacharparenleft}{\kern0pt}hash\ p\ x\ {\isasymomega}{\isacharparenright}{\kern0pt}\ {\isasymle}\ c\ {\isasymand}\isanewline
\ \ \ \ truncate{\isacharunderscore}{\kern0pt}down\ r\ {\isacharparenleft}{\kern0pt}hash\ p\ x\ {\isasymomega}{\isacharparenright}{\kern0pt}\ {\isacharequal}{\kern0pt}\ truncate{\isacharunderscore}{\kern0pt}down\ r\ {\isacharparenleft}{\kern0pt}hash\ p\ y\ {\isasymomega}{\isacharparenright}{\kern0pt}{\isacharbraceright}{\kern0pt}\ {\isasymle}\ \isanewline
\ \ \ \ {\isadigit{1}}{\isadigit{2}}\ {\isacharasterisk}{\kern0pt}\ c\isactrlsup {\isadigit{2}}\ {\isacharasterisk}{\kern0pt}\ {\isadigit{2}}\ powr\ {\isacharparenleft}{\kern0pt}{\isacharminus}{\kern0pt}real\ r{\isacharparenright}{\kern0pt}\ {\isacharslash}{\kern0pt}{\isacharparenleft}{\kern0pt}real\ p{\isacharparenright}{\kern0pt}\isactrlsup {\isadigit{2}}{\isachardoublequoteclose}\isanewline
\ \ \isacommand{proof}\isamarkupfalse%
\ {\isacharminus}{\kern0pt}\isanewline
\ \ \ \ \isacommand{fix}\isamarkupfalse%
\ x\ y\isanewline
\ \ \ \ \isacommand{assume}\isamarkupfalse%
\ a{\isadigit{1}}{\isacharunderscore}{\kern0pt}{\isadigit{1}}{\isacharcolon}{\kern0pt}\ {\isachardoublequoteopen}x\ {\isasymin}\ M{\isachardoublequoteclose}\isanewline
\ \ \ \ \isacommand{assume}\isamarkupfalse%
\ a{\isadigit{1}}{\isacharunderscore}{\kern0pt}{\isadigit{2}}{\isacharcolon}{\kern0pt}\ {\isachardoublequoteopen}y\ {\isasymin}\ M{\isachardoublequoteclose}\isanewline
\ \ \ \ \isacommand{assume}\isamarkupfalse%
\ a{\isadigit{1}}{\isacharunderscore}{\kern0pt}{\isadigit{3}}{\isacharcolon}{\kern0pt}\ {\isachardoublequoteopen}x\ {\isacharless}{\kern0pt}\ y{\isachardoublequoteclose}\isanewline
\isanewline
\ \ \ \ \isacommand{have}\isamarkupfalse%
\ a{\isadigit{1}}{\isacharunderscore}{\kern0pt}{\isadigit{4}}{\isacharcolon}{\kern0pt}\ {\isachardoublequoteopen}{\isasymAnd}u\ v{\isachardot}{\kern0pt}\ truncate{\isacharunderscore}{\kern0pt}down\ r\ {\isacharparenleft}{\kern0pt}real\ u{\isacharparenright}{\kern0pt}\ {\isasymle}\ c\ {\isasymLongrightarrow}\ \isanewline
\ \ \ \ \ \ \ \ truncate{\isacharunderscore}{\kern0pt}down\ r\ {\isacharparenleft}{\kern0pt}real\ u{\isacharparenright}{\kern0pt}\ {\isacharequal}{\kern0pt}\ truncate{\isacharunderscore}{\kern0pt}down\ r\ {\isacharparenleft}{\kern0pt}real\ v{\isacharparenright}{\kern0pt}\ {\isasymLongrightarrow}\isanewline
\ \ \ \ \ \ \ \ real\ u\ {\isasymle}\ {\isadigit{2}}\ {\isacharasterisk}{\kern0pt}\ c\ {\isasymand}\ {\isasymbar}real\ u\ {\isacharminus}{\kern0pt}\ real\ v{\isasymbar}\ {\isasymle}\ {\isadigit{2}}\ {\isacharasterisk}{\kern0pt}\ c\ {\isacharasterisk}{\kern0pt}\ {\isadigit{2}}\ powr\ {\isacharparenleft}{\kern0pt}{\isacharminus}{\kern0pt}real\ r{\isacharparenright}{\kern0pt}{\isachardoublequoteclose}\isanewline
\ \ \ \ \isacommand{proof}\isamarkupfalse%
\ {\isacharminus}{\kern0pt}\isanewline
\ \ \ \ \ \ \isacommand{fix}\isamarkupfalse%
\ u\ v\isanewline
\ \ \ \ \ \ \isacommand{assume}\isamarkupfalse%
\ a{\isacharunderscore}{\kern0pt}{\isadigit{1}}{\isacharcolon}{\kern0pt}{\isachardoublequoteopen}truncate{\isacharunderscore}{\kern0pt}down\ r\ {\isacharparenleft}{\kern0pt}real\ u{\isacharparenright}{\kern0pt}\ {\isasymle}\ c{\isachardoublequoteclose}\isanewline
\ \ \ \ \ \ \isacommand{assume}\isamarkupfalse%
\ a{\isacharunderscore}{\kern0pt}{\isadigit{2}}{\isacharcolon}{\kern0pt}{\isachardoublequoteopen}truncate{\isacharunderscore}{\kern0pt}down\ r\ {\isacharparenleft}{\kern0pt}real\ u{\isacharparenright}{\kern0pt}\ {\isacharequal}{\kern0pt}\ truncate{\isacharunderscore}{\kern0pt}down\ r\ {\isacharparenleft}{\kern0pt}real\ v{\isacharparenright}{\kern0pt}{\isachardoublequoteclose}\isanewline
\ \ \ \ \ \ \isacommand{have}\isamarkupfalse%
\ a{\isacharunderscore}{\kern0pt}{\isadigit{3}}{\isacharcolon}{\kern0pt}\ {\isachardoublequoteopen}{\isadigit{2}}\ {\isacharasterisk}{\kern0pt}\ {\isadigit{2}}\ powr\ {\isacharparenleft}{\kern0pt}{\isacharminus}{\kern0pt}\ real\ r{\isacharparenright}{\kern0pt}\ {\isacharequal}{\kern0pt}\ {\isadigit{2}}\ powr\ {\isacharparenleft}{\kern0pt}{\isadigit{1}}\ {\isacharminus}{\kern0pt}real\ r{\isacharparenright}{\kern0pt}{\isachardoublequoteclose}\isanewline
\ \ \ \ \ \ \ \ \isacommand{by}\isamarkupfalse%
\ {\isacharparenleft}{\kern0pt}simp\ add{\isacharcolon}{\kern0pt}\ divide{\isacharunderscore}{\kern0pt}powr{\isacharunderscore}{\kern0pt}uminus\ powr{\isacharunderscore}{\kern0pt}diff{\isacharparenright}{\kern0pt}\isanewline
\isanewline
\ \ \ \ \ \ \isacommand{have}\isamarkupfalse%
\ a{\isacharunderscore}{\kern0pt}{\isadigit{4}}{\isacharunderscore}{\kern0pt}{\isadigit{1}}{\isacharcolon}{\kern0pt}\ {\isachardoublequoteopen}{\isadigit{1}}\ {\isasymle}\ {\isadigit{2}}\ {\isacharasterisk}{\kern0pt}\ {\isacharparenleft}{\kern0pt}{\isadigit{1}}\ {\isacharminus}{\kern0pt}\ {\isadigit{2}}\ powr\ {\isacharparenleft}{\kern0pt}{\isacharminus}{\kern0pt}\ real\ r{\isacharparenright}{\kern0pt}{\isacharparenright}{\kern0pt}{\isachardoublequoteclose}\isanewline
\ \ \ \ \ \ \ \ \isacommand{apply}\isamarkupfalse%
\ {\isacharparenleft}{\kern0pt}simp{\isacharcomma}{\kern0pt}\ subst\ a{\isacharunderscore}{\kern0pt}{\isadigit{3}}{\isacharcomma}{\kern0pt}\ subst\ {\isacharparenleft}{\kern0pt}{\isadigit{2}}{\isacharparenright}{\kern0pt}\ two{\isacharunderscore}{\kern0pt}powr{\isacharunderscore}{\kern0pt}{\isadigit{0}}{\isacharbrackleft}{\kern0pt}symmetric{\isacharbrackright}{\kern0pt}{\isacharparenright}{\kern0pt}\isanewline
\ \ \ \ \ \ \ \ \isacommand{apply}\isamarkupfalse%
\ {\isacharparenleft}{\kern0pt}rule\ powr{\isacharunderscore}{\kern0pt}mono{\isacharparenright}{\kern0pt}\isanewline
\ \ \ \ \ \ \ \ \isacommand{using}\isamarkupfalse%
\ assms{\isacharparenleft}{\kern0pt}{\isadigit{5}}{\isacharparenright}{\kern0pt}\ \isacommand{by}\isamarkupfalse%
\ simp{\isacharplus}{\kern0pt}\isanewline
\isanewline
\ \ \ \ \ \ \isacommand{have}\isamarkupfalse%
\ a{\isacharunderscore}{\kern0pt}{\isadigit{4}}{\isacharcolon}{\kern0pt}\ {\isachardoublequoteopen}{\isacharparenleft}{\kern0pt}c{\isacharasterisk}{\kern0pt}{\isadigit{1}}{\isacharparenright}{\kern0pt}\ {\isacharslash}{\kern0pt}\ {\isacharparenleft}{\kern0pt}{\isadigit{1}}\ {\isacharminus}{\kern0pt}\ {\isadigit{2}}\ powr\ {\isacharparenleft}{\kern0pt}{\isacharminus}{\kern0pt}real\ r{\isacharparenright}{\kern0pt}{\isacharparenright}{\kern0pt}\ {\isasymle}\ c\ {\isacharasterisk}{\kern0pt}\ {\isadigit{2}}{\isachardoublequoteclose}\ \isanewline
\ \ \ \ \ \ \ \ \isacommand{apply}\isamarkupfalse%
\ {\isacharparenleft}{\kern0pt}subst\ pos{\isacharunderscore}{\kern0pt}divide{\isacharunderscore}{\kern0pt}le{\isacharunderscore}{\kern0pt}eq{\isacharcomma}{\kern0pt}\ simp{\isacharparenright}{\kern0pt}\isanewline
\ \ \ \ \ \ \ \ \ \isacommand{apply}\isamarkupfalse%
\ {\isacharparenleft}{\kern0pt}subst\ two{\isacharunderscore}{\kern0pt}powr{\isacharunderscore}{\kern0pt}{\isadigit{0}}{\isacharbrackleft}{\kern0pt}symmetric{\isacharbrackright}{\kern0pt}{\isacharparenright}{\kern0pt}\isanewline
\ \ \ \ \ \ \ \ \ \isacommand{apply}\isamarkupfalse%
\ {\isacharparenleft}{\kern0pt}rule\ powr{\isacharunderscore}{\kern0pt}less{\isacharunderscore}{\kern0pt}mono{\isacharparenright}{\kern0pt}\ \isacommand{using}\isamarkupfalse%
\ assms{\isacharparenleft}{\kern0pt}{\isadigit{5}}{\isacharparenright}{\kern0pt}\ \isacommand{apply}\isamarkupfalse%
\ simp\isanewline
\ \ \ \ \ \ \ \ \ \isacommand{apply}\isamarkupfalse%
\ simp\isanewline
\ \ \ \ \ \ \ \ \isacommand{using}\isamarkupfalse%
\ a{\isacharunderscore}{\kern0pt}{\isadigit{4}}{\isacharunderscore}{\kern0pt}{\isadigit{1}}\ \isanewline
\ \ \ \ \ \ \ \ \isacommand{by}\isamarkupfalse%
\ {\isacharparenleft}{\kern0pt}metis\ {\isacharparenleft}{\kern0pt}no{\isacharunderscore}{\kern0pt}types{\isacharcomma}{\kern0pt}\ opaque{\isacharunderscore}{\kern0pt}lifting{\isacharparenright}{\kern0pt}\ c{\isacharunderscore}{\kern0pt}ge{\isacharunderscore}{\kern0pt}{\isadigit{0}}\ mult{\isachardot}{\kern0pt}left{\isacharunderscore}{\kern0pt}commute\ mult{\isachardot}{\kern0pt}right{\isacharunderscore}{\kern0pt}neutral\ mult{\isacharunderscore}{\kern0pt}left{\isacharunderscore}{\kern0pt}mono{\isacharparenright}{\kern0pt}\isanewline
\isanewline
\ \ \ \ \ \ \isacommand{have}\isamarkupfalse%
\ a{\isacharunderscore}{\kern0pt}{\isadigit{5}}{\isacharcolon}{\kern0pt}\ {\isachardoublequoteopen}{\isasymAnd}x{\isachardot}{\kern0pt}\ truncate{\isacharunderscore}{\kern0pt}down\ r\ {\isacharparenleft}{\kern0pt}real\ x{\isacharparenright}{\kern0pt}\ {\isasymle}\ \ c\ \ {\isasymLongrightarrow}\ real\ x\ {\isasymle}\ c\ {\isacharasterisk}{\kern0pt}\ {\isadigit{2}}{\isachardoublequoteclose}\isanewline
\ \ \ \ \ \ \ \ \isacommand{apply}\isamarkupfalse%
\ {\isacharparenleft}{\kern0pt}rule\ order{\isacharunderscore}{\kern0pt}trans{\isacharbrackleft}{\kern0pt}OF\ {\isacharunderscore}{\kern0pt}\ a{\isacharunderscore}{\kern0pt}{\isadigit{4}}{\isacharbrackright}{\kern0pt}{\isacharparenright}{\kern0pt}\isanewline
\ \ \ \ \ \ \ \ \isacommand{apply}\isamarkupfalse%
\ {\isacharparenleft}{\kern0pt}subst\ pos{\isacharunderscore}{\kern0pt}le{\isacharunderscore}{\kern0pt}divide{\isacharunderscore}{\kern0pt}eq{\isacharparenright}{\kern0pt}\isanewline
\ \ \ \ \ \ \ \ \ \isacommand{apply}\isamarkupfalse%
\ {\isacharparenleft}{\kern0pt}simp{\isacharcomma}{\kern0pt}\ subst\ two{\isacharunderscore}{\kern0pt}powr{\isacharunderscore}{\kern0pt}{\isadigit{0}}{\isacharbrackleft}{\kern0pt}symmetric{\isacharbrackright}{\kern0pt}{\isacharparenright}{\kern0pt}\isanewline
\ \ \ \ \ \ \ \ \ \isacommand{apply}\isamarkupfalse%
\ {\isacharparenleft}{\kern0pt}rule\ powr{\isacharunderscore}{\kern0pt}less{\isacharunderscore}{\kern0pt}mono{\isacharparenright}{\kern0pt}\ \isacommand{using}\isamarkupfalse%
\ assms{\isacharparenleft}{\kern0pt}{\isadigit{5}}{\isacharparenright}{\kern0pt}\ \isacommand{apply}\isamarkupfalse%
\ simp\isanewline
\ \ \ \ \ \ \ \ \isacommand{apply}\isamarkupfalse%
\ simp\isanewline
\ \ \ \ \ \ \ \ \isacommand{using}\isamarkupfalse%
\ \ truncate{\isacharunderscore}{\kern0pt}down{\isacharunderscore}{\kern0pt}pos{\isacharbrackleft}{\kern0pt}OF\ of{\isacharunderscore}{\kern0pt}nat{\isacharunderscore}{\kern0pt}{\isadigit{0}}{\isacharunderscore}{\kern0pt}le{\isacharunderscore}{\kern0pt}iff{\isacharbrackright}{\kern0pt}\ order{\isacharunderscore}{\kern0pt}trans\ \isacommand{apply}\isamarkupfalse%
\ simp\ \isacommand{by}\isamarkupfalse%
\ blast\isanewline
\isanewline
\ \ \ \ \ \ \isacommand{have}\isamarkupfalse%
\ a{\isacharunderscore}{\kern0pt}{\isadigit{6}}{\isacharcolon}{\kern0pt}\ {\isachardoublequoteopen}real\ u\ {\isasymle}\ c\ {\isacharasterisk}{\kern0pt}\ {\isadigit{2}}{\isachardoublequoteclose}\isanewline
\ \ \ \ \ \ \ \ \isacommand{using}\isamarkupfalse%
\ a{\isacharunderscore}{\kern0pt}{\isadigit{1}}\ a{\isacharunderscore}{\kern0pt}{\isadigit{5}}\ \isacommand{by}\isamarkupfalse%
\ simp\isanewline
\ \ \ \ \ \ \isacommand{have}\isamarkupfalse%
\ a{\isacharunderscore}{\kern0pt}{\isadigit{7}}{\isacharcolon}{\kern0pt}\ {\isachardoublequoteopen}real\ v\ {\isasymle}\ c\ {\isacharasterisk}{\kern0pt}\ {\isadigit{2}}{\isachardoublequoteclose}\ \isanewline
\ \ \ \ \ \ \ \ \isacommand{using}\isamarkupfalse%
\ a{\isacharunderscore}{\kern0pt}{\isadigit{1}}\ a{\isacharunderscore}{\kern0pt}{\isadigit{2}}\ a{\isacharunderscore}{\kern0pt}{\isadigit{5}}\ \isacommand{by}\isamarkupfalse%
\ simp\isanewline
\ \ \ \ \ \ \isacommand{have}\isamarkupfalse%
\ {\isachardoublequoteopen}\ {\isasymbar}real\ u\ {\isacharminus}{\kern0pt}\ real\ v{\isasymbar}\ {\isasymle}\ {\isacharparenleft}{\kern0pt}max\ {\isasymbar}real\ u{\isasymbar}\ {\isasymbar}real\ v{\isasymbar}{\isacharparenright}{\kern0pt}\ {\isacharasterisk}{\kern0pt}\ {\isadigit{2}}\ powr\ {\isacharparenleft}{\kern0pt}{\isacharminus}{\kern0pt}real\ r{\isacharparenright}{\kern0pt}{\isachardoublequoteclose}\isanewline
\ \ \ \ \ \ \ \ \isacommand{apply}\isamarkupfalse%
\ {\isacharparenleft}{\kern0pt}rule\ truncate{\isacharunderscore}{\kern0pt}down{\isacharunderscore}{\kern0pt}eq{\isacharparenright}{\kern0pt}\ \isacommand{using}\isamarkupfalse%
\ a{\isacharunderscore}{\kern0pt}{\isadigit{2}}\ \isacommand{by}\isamarkupfalse%
\ simp\ \isanewline
\ \ \ \ \ \ \isacommand{also}\isamarkupfalse%
\ \isacommand{have}\isamarkupfalse%
\ {\isachardoublequoteopen}{\isachardot}{\kern0pt}{\isachardot}{\kern0pt}{\isachardot}{\kern0pt}\ {\isasymle}\ {\isacharparenleft}{\kern0pt}c\ {\isacharasterisk}{\kern0pt}\ {\isadigit{2}}{\isacharparenright}{\kern0pt}\ {\isacharasterisk}{\kern0pt}\ {\isadigit{2}}\ powr\ {\isacharparenleft}{\kern0pt}{\isacharminus}{\kern0pt}real\ r{\isacharparenright}{\kern0pt}{\isachardoublequoteclose}\isanewline
\ \ \ \ \ \ \ \ \isacommand{apply}\isamarkupfalse%
\ {\isacharparenleft}{\kern0pt}rule\ mult{\isacharunderscore}{\kern0pt}right{\isacharunderscore}{\kern0pt}mono{\isacharparenright}{\kern0pt}\ \isacommand{using}\isamarkupfalse%
\ a{\isacharunderscore}{\kern0pt}{\isadigit{6}}\ a{\isacharunderscore}{\kern0pt}{\isadigit{7}}\ \isacommand{by}\isamarkupfalse%
\ simp{\isacharplus}{\kern0pt}\isanewline
\ \ \ \ \ \ \isacommand{finally}\isamarkupfalse%
\ \isacommand{have}\isamarkupfalse%
\ a{\isacharunderscore}{\kern0pt}{\isadigit{8}}{\isacharcolon}{\kern0pt}\ {\isachardoublequoteopen}{\isasymbar}real\ u\ {\isacharminus}{\kern0pt}\ real\ v{\isasymbar}\ {\isasymle}\ {\isadigit{2}}\ {\isacharasterisk}{\kern0pt}\ c\ {\isacharasterisk}{\kern0pt}\ {\isadigit{2}}\ powr\ {\isacharparenleft}{\kern0pt}{\isacharminus}{\kern0pt}real\ r{\isacharparenright}{\kern0pt}{\isachardoublequoteclose}\isanewline
\ \ \ \ \ \ \ \ \isacommand{by}\isamarkupfalse%
\ simp\isanewline
\isanewline
\ \ \ \ \ \ \isacommand{show}\isamarkupfalse%
\ {\isachardoublequoteopen}real\ u\ {\isasymle}\ {\isadigit{2}}{\isacharasterisk}{\kern0pt}\ c\ {\isasymand}\ {\isasymbar}real\ u\ {\isacharminus}{\kern0pt}\ real\ v{\isasymbar}\ {\isasymle}\ {\isadigit{2}}\ {\isacharasterisk}{\kern0pt}\ c\ {\isacharasterisk}{\kern0pt}\ {\isadigit{2}}\ powr\ {\isacharparenleft}{\kern0pt}{\isacharminus}{\kern0pt}real\ r{\isacharparenright}{\kern0pt}{\isachardoublequoteclose}\ \isanewline
\ \ \ \ \ \ \ \ \isacommand{using}\isamarkupfalse%
\ a{\isacharunderscore}{\kern0pt}{\isadigit{6}}\ a{\isacharunderscore}{\kern0pt}{\isadigit{8}}\ \isacommand{by}\isamarkupfalse%
\ simp\isanewline
\ \ \ \ \isacommand{qed}\isamarkupfalse%
\isanewline
\isanewline
\ \ \ \ \isacommand{have}\isamarkupfalse%
\ {\isachardoublequoteopen}measure\ {\isasymOmega}\ {\isacharbraceleft}{\kern0pt}{\isasymomega}{\isachardot}{\kern0pt}\ degree\ {\isasymomega}\ {\isasymge}\ {\isadigit{1}}\ {\isasymand}\ truncate{\isacharunderscore}{\kern0pt}down\ r\ {\isacharparenleft}{\kern0pt}hash\ p\ x\ {\isasymomega}{\isacharparenright}{\kern0pt}\ {\isasymle}\ c\ {\isasymand}\isanewline
\ \ \ \ \ \ truncate{\isacharunderscore}{\kern0pt}down\ r\ {\isacharparenleft}{\kern0pt}hash\ p\ x\ {\isasymomega}{\isacharparenright}{\kern0pt}\ {\isacharequal}{\kern0pt}\ truncate{\isacharunderscore}{\kern0pt}down\ r\ {\isacharparenleft}{\kern0pt}hash\ p\ y\ {\isasymomega}{\isacharparenright}{\kern0pt}{\isacharbraceright}{\kern0pt}\ {\isasymle}\isanewline
\ \ \ \ \ \ measure\ {\isasymOmega}\ {\isacharparenleft}{\kern0pt}{\isasymUnion}\ i\ {\isasymin}\ {\isacharbraceleft}{\kern0pt}{\isacharparenleft}{\kern0pt}u{\isacharcomma}{\kern0pt}v{\isacharparenright}{\kern0pt}\ {\isasymin}\ {\isacharbraceleft}{\kern0pt}{\isadigit{0}}{\isachardot}{\kern0pt}{\isachardot}{\kern0pt}{\isacharless}{\kern0pt}p{\isacharbraceright}{\kern0pt}\ {\isasymtimes}\ {\isacharbraceleft}{\kern0pt}{\isadigit{0}}{\isachardot}{\kern0pt}{\isachardot}{\kern0pt}{\isacharless}{\kern0pt}p{\isacharbraceright}{\kern0pt}{\isachardot}{\kern0pt}\ u\ {\isasymnoteq}\ v\ {\isasymand}\isanewline
\ \ \ \ \ \ truncate{\isacharunderscore}{\kern0pt}down\ r\ u\ {\isasymle}\ c\ {\isasymand}\ truncate{\isacharunderscore}{\kern0pt}down\ r\ u\ {\isacharequal}{\kern0pt}\ truncate{\isacharunderscore}{\kern0pt}down\ r\ v{\isacharbraceright}{\kern0pt}{\isachardot}{\kern0pt}\isanewline
\ \ \ \ \ \ {\isacharbraceleft}{\kern0pt}{\isasymomega}{\isachardot}{\kern0pt}\ \ hash\ p\ x\ {\isasymomega}\ {\isacharequal}{\kern0pt}\ fst\ i\ {\isasymand}\ hash\ p\ y\ {\isasymomega}\ {\isacharequal}{\kern0pt}\ snd\ i{\isacharbraceright}{\kern0pt}{\isacharparenright}{\kern0pt}{\isachardoublequoteclose}\isanewline
\ \ \ \ \ \ \isacommand{apply}\isamarkupfalse%
\ {\isacharparenleft}{\kern0pt}rule\ pmf{\isacharunderscore}{\kern0pt}mono{\isacharunderscore}{\kern0pt}{\isadigit{1}}{\isacharparenright}{\kern0pt}\isanewline
\ \ \ \ \ \ \isacommand{apply}\isamarkupfalse%
\ {\isacharparenleft}{\kern0pt}simp\ add{\isacharcolon}{\kern0pt}\ {\isasymOmega}{\isacharunderscore}{\kern0pt}def{\isacharparenright}{\kern0pt}\isanewline
\ \ \ \ \ \ \isacommand{apply}\isamarkupfalse%
\ {\isacharparenleft}{\kern0pt}subst\ {\isacharparenleft}{\kern0pt}asm{\isacharparenright}{\kern0pt}\ set{\isacharunderscore}{\kern0pt}pmf{\isacharunderscore}{\kern0pt}of{\isacharunderscore}{\kern0pt}set{\isacharparenright}{\kern0pt}\isanewline
\ \ \ \ \ \ \ \ \isacommand{apply}\isamarkupfalse%
\ {\isacharparenleft}{\kern0pt}rule\ ne{\isacharunderscore}{\kern0pt}bounded{\isacharunderscore}{\kern0pt}degree{\isacharunderscore}{\kern0pt}polynomials{\isacharparenright}{\kern0pt}\isanewline
\ \ \ \ \ \ \isacommand{apply}\isamarkupfalse%
\ {\isacharparenleft}{\kern0pt}rule\ fin{\isacharunderscore}{\kern0pt}bounded{\isacharunderscore}{\kern0pt}degree{\isacharunderscore}{\kern0pt}polynomials{\isacharbrackleft}{\kern0pt}OF\ p{\isacharunderscore}{\kern0pt}ge{\isacharunderscore}{\kern0pt}{\isadigit{0}}{\isacharbrackright}{\kern0pt}{\isacharparenright}{\kern0pt}\isanewline
\ \ \ \ \ \ \isacommand{by}\isamarkupfalse%
\ {\isacharparenleft}{\kern0pt}metis\ assms{\isacharparenleft}{\kern0pt}{\isadigit{3}}{\isacharparenright}{\kern0pt}\ a{\isadigit{2}}\ a{\isadigit{3}}\ a{\isadigit{1}}{\isacharunderscore}{\kern0pt}{\isadigit{1}}\ a{\isadigit{1}}{\isacharunderscore}{\kern0pt}{\isadigit{2}}\ a{\isadigit{1}}{\isacharunderscore}{\kern0pt}{\isadigit{3}}\ \ One{\isacharunderscore}{\kern0pt}nat{\isacharunderscore}{\kern0pt}def\ less{\isacharunderscore}{\kern0pt}not{\isacharunderscore}{\kern0pt}refl{\isadigit{3}}\ atLeastLessThan{\isacharunderscore}{\kern0pt}iff\ subset{\isacharunderscore}{\kern0pt}eq{\isacharparenright}{\kern0pt}\isanewline
\ \ \ \ \isacommand{also}\isamarkupfalse%
\ \isacommand{have}\isamarkupfalse%
\ {\isachardoublequoteopen}{\isachardot}{\kern0pt}{\isachardot}{\kern0pt}{\isachardot}{\kern0pt}\ {\isasymle}\ {\isacharparenleft}{\kern0pt}{\isasymSum}\ i{\isasymin}\ {\isacharbraceleft}{\kern0pt}{\isacharparenleft}{\kern0pt}u{\isacharcomma}{\kern0pt}v{\isacharparenright}{\kern0pt}\ {\isasymin}\ {\isacharbraceleft}{\kern0pt}{\isadigit{0}}{\isachardot}{\kern0pt}{\isachardot}{\kern0pt}{\isacharless}{\kern0pt}p{\isacharbraceright}{\kern0pt}\ {\isasymtimes}\ {\isacharbraceleft}{\kern0pt}{\isadigit{0}}{\isachardot}{\kern0pt}{\isachardot}{\kern0pt}{\isacharless}{\kern0pt}p{\isacharbraceright}{\kern0pt}{\isachardot}{\kern0pt}\ u\ {\isasymnoteq}\ v\ {\isasymand}\isanewline
\ \ \ \ \ \ truncate{\isacharunderscore}{\kern0pt}down\ r\ u\ {\isasymle}\ c\ {\isasymand}\ truncate{\isacharunderscore}{\kern0pt}down\ r\ u\ {\isacharequal}{\kern0pt}\ truncate{\isacharunderscore}{\kern0pt}down\ r\ v{\isacharbraceright}{\kern0pt}{\isachardot}{\kern0pt}\ \isanewline
\ \ \ \ \ \ measure\ {\isasymOmega}\ \ {\isacharbraceleft}{\kern0pt}{\isasymomega}{\isachardot}{\kern0pt}\ hash\ p\ x\ {\isasymomega}\ {\isacharequal}{\kern0pt}\ fst\ i\ {\isasymand}\ hash\ p\ y\ {\isasymomega}\ {\isacharequal}{\kern0pt}\ snd\ i{\isacharbraceright}{\kern0pt}{\isacharparenright}{\kern0pt}{\isachardoublequoteclose}\isanewline
\ \ \ \ \ \ \isacommand{apply}\isamarkupfalse%
\ {\isacharparenleft}{\kern0pt}rule\ measure{\isacharunderscore}{\kern0pt}UNION{\isacharunderscore}{\kern0pt}le{\isacharparenright}{\kern0pt}\isanewline
\ \ \ \ \ \ \ \isacommand{apply}\isamarkupfalse%
\ {\isacharparenleft}{\kern0pt}rule\ finite{\isacharunderscore}{\kern0pt}subset{\isacharbrackleft}{\kern0pt}\isakeyword{where}\ B{\isacharequal}{\kern0pt}{\isachardoublequoteopen}{\isacharbraceleft}{\kern0pt}{\isadigit{0}}{\isachardot}{\kern0pt}{\isachardot}{\kern0pt}{\isacharless}{\kern0pt}p{\isacharbraceright}{\kern0pt}\ {\isasymtimes}\ {\isacharbraceleft}{\kern0pt}{\isadigit{0}}{\isachardot}{\kern0pt}{\isachardot}{\kern0pt}{\isacharless}{\kern0pt}p{\isacharbraceright}{\kern0pt}{\isachardoublequoteclose}{\isacharbrackright}{\kern0pt}{\isacharcomma}{\kern0pt}\ rule\ subsetI{\isacharcomma}{\kern0pt}\ simp\ add{\isacharcolon}{\kern0pt}case{\isacharunderscore}{\kern0pt}prod{\isacharunderscore}{\kern0pt}beta\ mem{\isacharunderscore}{\kern0pt}Times{\isacharunderscore}{\kern0pt}iff{\isacharcomma}{\kern0pt}\ simp{\isacharparenright}{\kern0pt}\isanewline
\ \ \ \ \ \ \isacommand{by}\isamarkupfalse%
\ simp\isanewline
\ \ \ \ \isacommand{also}\isamarkupfalse%
\ \isacommand{have}\isamarkupfalse%
\ {\isachardoublequoteopen}{\isachardot}{\kern0pt}{\isachardot}{\kern0pt}{\isachardot}{\kern0pt}\ {\isasymle}\ {\isacharparenleft}{\kern0pt}{\isasymSum}\ i{\isasymin}\ {\isacharbraceleft}{\kern0pt}{\isacharparenleft}{\kern0pt}u{\isacharcomma}{\kern0pt}v{\isacharparenright}{\kern0pt}\ {\isasymin}\ {\isacharbraceleft}{\kern0pt}{\isadigit{0}}{\isachardot}{\kern0pt}{\isachardot}{\kern0pt}{\isacharless}{\kern0pt}p{\isacharbraceright}{\kern0pt}\ {\isasymtimes}\ {\isacharbraceleft}{\kern0pt}{\isadigit{0}}{\isachardot}{\kern0pt}{\isachardot}{\kern0pt}{\isacharless}{\kern0pt}p{\isacharbraceright}{\kern0pt}{\isachardot}{\kern0pt}\ u\ {\isasymnoteq}\ v\ {\isasymand}\isanewline
\ \ \ \ \ \ truncate{\isacharunderscore}{\kern0pt}down\ r\ u\ {\isasymle}\ c\ {\isasymand}\ truncate{\isacharunderscore}{\kern0pt}down\ r\ u\ {\isacharequal}{\kern0pt}\ truncate{\isacharunderscore}{\kern0pt}down\ r\ v{\isacharbraceright}{\kern0pt}{\isachardot}{\kern0pt}\ \isanewline
\ \ \ \ \ \ {\isasymP}{\isacharparenleft}{\kern0pt}{\isasymomega}\ in\ {\isasymOmega}{\isachardot}{\kern0pt}\ {\isacharparenleft}{\kern0pt}{\isasymforall}u\ {\isasymin}\ UNIV{\isachardot}{\kern0pt}\ hash\ p\ {\isacharparenleft}{\kern0pt}if\ u\ then\ x\ else\ y{\isacharparenright}{\kern0pt}\ {\isasymomega}\ {\isacharequal}{\kern0pt}\ {\isacharparenleft}{\kern0pt}if\ u\ then\ {\isacharparenleft}{\kern0pt}fst\ i{\isacharparenright}{\kern0pt}\ else\ {\isacharparenleft}{\kern0pt}snd\ i{\isacharparenright}{\kern0pt}{\isacharparenright}{\kern0pt}{\isacharparenright}{\kern0pt}{\isacharparenright}{\kern0pt}{\isacharparenright}{\kern0pt}{\isachardoublequoteclose}\ \isanewline
\ \ \ \ \ \ \isacommand{apply}\isamarkupfalse%
\ {\isacharparenleft}{\kern0pt}rule\ sum{\isacharunderscore}{\kern0pt}mono{\isacharparenright}{\kern0pt}\isanewline
\ \ \ \ \ \ \isacommand{apply}\isamarkupfalse%
\ {\isacharparenleft}{\kern0pt}rule\ pmf{\isacharunderscore}{\kern0pt}mono{\isacharparenright}{\kern0pt}\isanewline
\ \ \ \ \ \ \isacommand{by}\isamarkupfalse%
\ {\isacharparenleft}{\kern0pt}simp\ add{\isacharcolon}{\kern0pt}case{\isacharunderscore}{\kern0pt}prod{\isacharunderscore}{\kern0pt}beta{\isacharparenright}{\kern0pt}\isanewline
\ \ \ \ \isacommand{also}\isamarkupfalse%
\ \isacommand{have}\isamarkupfalse%
\ {\isachardoublequoteopen}{\isachardot}{\kern0pt}{\isachardot}{\kern0pt}{\isachardot}{\kern0pt}\ {\isasymle}\ {\isacharparenleft}{\kern0pt}{\isasymSum}\ i{\isasymin}\ {\isacharbraceleft}{\kern0pt}{\isacharparenleft}{\kern0pt}u{\isacharcomma}{\kern0pt}v{\isacharparenright}{\kern0pt}\ {\isasymin}\ {\isacharbraceleft}{\kern0pt}{\isadigit{0}}{\isachardot}{\kern0pt}{\isachardot}{\kern0pt}{\isacharless}{\kern0pt}p{\isacharbraceright}{\kern0pt}\ {\isasymtimes}\ {\isacharbraceleft}{\kern0pt}{\isadigit{0}}{\isachardot}{\kern0pt}{\isachardot}{\kern0pt}{\isacharless}{\kern0pt}p{\isacharbraceright}{\kern0pt}{\isachardot}{\kern0pt}\ u\ {\isasymnoteq}\ v\ {\isasymand}\isanewline
\ \ \ \ \ \ truncate{\isacharunderscore}{\kern0pt}down\ r\ u\ {\isasymle}\ c\ {\isasymand}\ truncate{\isacharunderscore}{\kern0pt}down\ r\ u\ {\isacharequal}{\kern0pt}\ truncate{\isacharunderscore}{\kern0pt}down\ r\ v{\isacharbraceright}{\kern0pt}{\isachardot}{\kern0pt}\ {\isadigit{1}}{\isacharslash}{\kern0pt}{\isacharparenleft}{\kern0pt}real\ p{\isacharparenright}{\kern0pt}\isactrlsup {\isadigit{2}}{\isacharparenright}{\kern0pt}{\isachardoublequoteclose}\isanewline
\ \ \ \ \ \ \isacommand{apply}\isamarkupfalse%
\ {\isacharparenleft}{\kern0pt}rule\ sum{\isacharunderscore}{\kern0pt}mono{\isacharparenright}{\kern0pt}\isanewline
\ \ \ \ \ \ \isacommand{apply}\isamarkupfalse%
\ {\isacharparenleft}{\kern0pt}simp\ only{\isacharcolon}{\kern0pt}{\isasymOmega}{\isacharunderscore}{\kern0pt}def{\isacharparenright}{\kern0pt}\isanewline
\ \ \ \ \ \ \isacommand{apply}\isamarkupfalse%
\ {\isacharparenleft}{\kern0pt}subst\ hash{\isacharunderscore}{\kern0pt}prob{\isacharunderscore}{\kern0pt}{\isadigit{2}}{\isacharbrackleft}{\kern0pt}OF\ assms{\isacharparenleft}{\kern0pt}{\isadigit{1}}{\isacharparenright}{\kern0pt}{\isacharbrackright}{\kern0pt}{\isacharparenright}{\kern0pt}\isanewline
\ \ \ \ \ \ \ \ \ \ \isacommand{using}\isamarkupfalse%
\ a{\isadigit{1}}{\isacharunderscore}{\kern0pt}{\isadigit{3}}\ \isacommand{apply}\isamarkupfalse%
\ {\isacharparenleft}{\kern0pt}simp\ add{\isacharcolon}{\kern0pt}\ inj{\isacharunderscore}{\kern0pt}on{\isacharunderscore}{\kern0pt}def{\isacharparenright}{\kern0pt}\isanewline
\ \ \ \ \ \ \ \ \ \isacommand{using}\isamarkupfalse%
\ a{\isadigit{1}}{\isacharunderscore}{\kern0pt}{\isadigit{1}}\ assms{\isacharparenleft}{\kern0pt}{\isadigit{3}}{\isacharparenright}{\kern0pt}\ a{\isadigit{1}}{\isacharunderscore}{\kern0pt}{\isadigit{3}}\ a{\isadigit{1}}{\isacharunderscore}{\kern0pt}{\isadigit{2}}\ \isacommand{apply}\isamarkupfalse%
\ auto{\isacharbrackleft}{\kern0pt}{\isadigit{1}}{\isacharbrackright}{\kern0pt}\isanewline
\ \ \ \ \ \ \ \ \ \isacommand{by}\isamarkupfalse%
\ force{\isacharplus}{\kern0pt}\isanewline
\ \ \ \ \isacommand{also}\isamarkupfalse%
\ \isacommand{have}\isamarkupfalse%
\ {\isachardoublequoteopen}{\isachardot}{\kern0pt}{\isachardot}{\kern0pt}{\isachardot}{\kern0pt}\ {\isacharequal}{\kern0pt}\ {\isadigit{1}}{\isacharslash}{\kern0pt}{\isacharparenleft}{\kern0pt}real\ p{\isacharparenright}{\kern0pt}\isactrlsup {\isadigit{2}}\ {\isacharasterisk}{\kern0pt}\ \isanewline
\ \ \ \ \ \ card\ {\isacharbraceleft}{\kern0pt}{\isacharparenleft}{\kern0pt}u{\isacharcomma}{\kern0pt}v{\isacharparenright}{\kern0pt}\ {\isasymin}\ {\isacharbraceleft}{\kern0pt}{\isadigit{0}}{\isachardot}{\kern0pt}{\isachardot}{\kern0pt}{\isacharless}{\kern0pt}p{\isacharbraceright}{\kern0pt}\ {\isasymtimes}\ {\isacharbraceleft}{\kern0pt}{\isadigit{0}}{\isachardot}{\kern0pt}{\isachardot}{\kern0pt}{\isacharless}{\kern0pt}p{\isacharbraceright}{\kern0pt}{\isachardot}{\kern0pt}\ u\ {\isasymnoteq}\ v\ {\isasymand}\ truncate{\isacharunderscore}{\kern0pt}down\ r\ u\ {\isasymle}\ c\ {\isasymand}\ truncate{\isacharunderscore}{\kern0pt}down\ r\ u\ {\isacharequal}{\kern0pt}\ truncate{\isacharunderscore}{\kern0pt}down\ r\ v{\isacharbraceright}{\kern0pt}{\isachardoublequoteclose}\isanewline
\ \ \ \ \ \ \isacommand{by}\isamarkupfalse%
\ simp\isanewline
\ \ \ \ \isacommand{also}\isamarkupfalse%
\ \isacommand{have}\isamarkupfalse%
\ {\isachardoublequoteopen}{\isachardot}{\kern0pt}{\isachardot}{\kern0pt}{\isachardot}{\kern0pt}\ {\isasymle}\ {\isadigit{1}}{\isacharslash}{\kern0pt}{\isacharparenleft}{\kern0pt}real\ p{\isacharparenright}{\kern0pt}\isactrlsup {\isadigit{2}}\ {\isacharasterisk}{\kern0pt}\ \isanewline
\ \ \ \ \ \ card\ {\isacharbraceleft}{\kern0pt}{\isacharparenleft}{\kern0pt}u{\isacharcomma}{\kern0pt}v{\isacharparenright}{\kern0pt}\ {\isasymin}\ {\isacharbraceleft}{\kern0pt}{\isadigit{0}}{\isachardot}{\kern0pt}{\isachardot}{\kern0pt}{\isacharless}{\kern0pt}p{\isacharbraceright}{\kern0pt}\ {\isasymtimes}\ {\isacharbraceleft}{\kern0pt}{\isadigit{0}}{\isachardot}{\kern0pt}{\isachardot}{\kern0pt}{\isacharless}{\kern0pt}p{\isacharbraceright}{\kern0pt}{\isachardot}{\kern0pt}\ u\ {\isasymnoteq}\ v\ {\isasymand}\ real\ u\ {\isasymle}\ {\isadigit{2}}\ {\isacharasterisk}{\kern0pt}\ c\ {\isasymand}\ abs\ {\isacharparenleft}{\kern0pt}real\ u\ {\isacharminus}{\kern0pt}\ real\ v{\isacharparenright}{\kern0pt}\ {\isasymle}\ {\isadigit{2}}\ {\isacharasterisk}{\kern0pt}\ c\ {\isacharasterisk}{\kern0pt}\ {\isadigit{2}}\ powr\ {\isacharparenleft}{\kern0pt}{\isacharminus}{\kern0pt}real\ r{\isacharparenright}{\kern0pt}{\isacharbraceright}{\kern0pt}{\isachardoublequoteclose}\isanewline
\ \ \ \ \ \ \isacommand{apply}\isamarkupfalse%
\ {\isacharparenleft}{\kern0pt}rule\ mult{\isacharunderscore}{\kern0pt}left{\isacharunderscore}{\kern0pt}mono{\isacharcomma}{\kern0pt}\ rule\ of{\isacharunderscore}{\kern0pt}nat{\isacharunderscore}{\kern0pt}mono{\isacharcomma}{\kern0pt}\ rule\ card{\isacharunderscore}{\kern0pt}mono{\isacharparenright}{\kern0pt}\isanewline
\ \ \ \ \ \ \ \ \isacommand{apply}\isamarkupfalse%
\ {\isacharparenleft}{\kern0pt}rule\ finite{\isacharunderscore}{\kern0pt}subset{\isacharbrackleft}{\kern0pt}\isakeyword{where}\ B{\isacharequal}{\kern0pt}{\isachardoublequoteopen}{\isacharbraceleft}{\kern0pt}{\isadigit{0}}{\isachardot}{\kern0pt}{\isachardot}{\kern0pt}{\isacharless}{\kern0pt}p{\isacharbraceright}{\kern0pt}{\isasymtimes}{\isacharbraceleft}{\kern0pt}{\isadigit{0}}{\isachardot}{\kern0pt}{\isachardot}{\kern0pt}{\isacharless}{\kern0pt}p{\isacharbraceright}{\kern0pt}{\isachardoublequoteclose}{\isacharbrackright}{\kern0pt}{\isacharcomma}{\kern0pt}\ rule\ subsetI{\isacharcomma}{\kern0pt}\ simp\ add{\isacharcolon}{\kern0pt}mem{\isacharunderscore}{\kern0pt}Times{\isacharunderscore}{\kern0pt}iff\ case{\isacharunderscore}{\kern0pt}prod{\isacharunderscore}{\kern0pt}beta{\isacharcomma}{\kern0pt}\ simp{\isacharparenright}{\kern0pt}\isanewline
\ \ \ \ \ \ \ \isacommand{apply}\isamarkupfalse%
\ {\isacharparenleft}{\kern0pt}rule\ subsetI{\isacharcomma}{\kern0pt}\ simp\ add{\isacharcolon}{\kern0pt}case{\isacharunderscore}{\kern0pt}prod{\isacharunderscore}{\kern0pt}beta{\isacharparenright}{\kern0pt}\isanewline
\ \ \ \ \ \ \isacommand{by}\isamarkupfalse%
\ {\isacharparenleft}{\kern0pt}metis\ a{\isadigit{1}}{\isacharunderscore}{\kern0pt}{\isadigit{4}}{\isacharcomma}{\kern0pt}\ simp{\isacharparenright}{\kern0pt}\isanewline
\ \ \ \ \isacommand{also}\isamarkupfalse%
\ \isacommand{have}\isamarkupfalse%
\ {\isachardoublequoteopen}{\isachardot}{\kern0pt}{\isachardot}{\kern0pt}{\isachardot}{\kern0pt}\ {\isasymle}\ {\isadigit{1}}{\isacharslash}{\kern0pt}{\isacharparenleft}{\kern0pt}real\ p{\isacharparenright}{\kern0pt}\isactrlsup {\isadigit{2}}\ {\isacharasterisk}{\kern0pt}\ card\ {\isacharparenleft}{\kern0pt}{\isasymUnion}u{\isacharprime}{\kern0pt}\ {\isasymin}\ {\isacharbraceleft}{\kern0pt}u{\isachardot}{\kern0pt}\ u\ {\isacharless}{\kern0pt}\ p\ {\isasymand}\ real\ u\ {\isasymle}\ {\isadigit{2}}\ {\isacharasterisk}{\kern0pt}\ c{\isacharbraceright}{\kern0pt}{\isachardot}{\kern0pt}\isanewline
\ \ \ \ \ \ \ \ {\isacharbraceleft}{\kern0pt}{\isacharparenleft}{\kern0pt}u{\isacharcolon}{\kern0pt}{\isacharcolon}{\kern0pt}nat{\isacharcomma}{\kern0pt}v{\isacharcolon}{\kern0pt}{\isacharcolon}{\kern0pt}nat{\isacharparenright}{\kern0pt}{\isachardot}{\kern0pt}\ u\ {\isacharequal}{\kern0pt}\ u{\isacharprime}{\kern0pt}\ {\isasymand}\ abs\ {\isacharparenleft}{\kern0pt}real\ u\ {\isacharminus}{\kern0pt}\ real\ v{\isacharparenright}{\kern0pt}\ {\isasymle}\ {\isadigit{2}}\ {\isacharasterisk}{\kern0pt}\ c\ {\isacharasterisk}{\kern0pt}\ {\isadigit{2}}\ powr\ {\isacharparenleft}{\kern0pt}{\isacharminus}{\kern0pt}real\ r{\isacharparenright}{\kern0pt}\ {\isasymand}\ v\ {\isacharless}{\kern0pt}\ p\ {\isasymand}\ v\ {\isasymnoteq}\ u{\isacharprime}{\kern0pt}{\isacharbraceright}{\kern0pt}{\isacharparenright}{\kern0pt}{\isachardoublequoteclose}\isanewline
\ \ \ \ \ \ \isacommand{apply}\isamarkupfalse%
\ {\isacharparenleft}{\kern0pt}rule\ mult{\isacharunderscore}{\kern0pt}left{\isacharunderscore}{\kern0pt}mono{\isacharparenright}{\kern0pt}\isanewline
\ \ \ \ \ \ \ \isacommand{apply}\isamarkupfalse%
\ {\isacharparenleft}{\kern0pt}rule\ of{\isacharunderscore}{\kern0pt}nat{\isacharunderscore}{\kern0pt}mono{\isacharparenright}{\kern0pt}\isanewline
\ \ \ \ \ \ \ \isacommand{apply}\isamarkupfalse%
\ {\isacharparenleft}{\kern0pt}rule\ card{\isacharunderscore}{\kern0pt}mono{\isacharcomma}{\kern0pt}\ simp\ add{\isacharcolon}{\kern0pt}case{\isacharunderscore}{\kern0pt}prod{\isacharunderscore}{\kern0pt}beta{\isacharparenright}{\kern0pt}\isanewline
\ \ \ \ \ \ \ \ \isacommand{apply}\isamarkupfalse%
\ {\isacharparenleft}{\kern0pt}rule\ allI{\isacharcomma}{\kern0pt}\ rule\ impI{\isacharparenright}{\kern0pt}\isanewline
\ \ \ \ \ \ \isacommand{apply}\isamarkupfalse%
\ {\isacharparenleft}{\kern0pt}rule\ finite{\isacharunderscore}{\kern0pt}subset{\isacharbrackleft}{\kern0pt}\isakeyword{where}\ B{\isacharequal}{\kern0pt}{\isachardoublequoteopen}{\isacharbraceleft}{\kern0pt}{\isadigit{0}}{\isachardot}{\kern0pt}{\isachardot}{\kern0pt}{\isacharless}{\kern0pt}p{\isacharbraceright}{\kern0pt}{\isasymtimes}{\isacharbraceleft}{\kern0pt}{\isadigit{0}}{\isachardot}{\kern0pt}{\isachardot}{\kern0pt}{\isacharless}{\kern0pt}p{\isacharbraceright}{\kern0pt}{\isachardoublequoteclose}{\isacharbrackright}{\kern0pt}{\isacharcomma}{\kern0pt}\ rule\ subsetI{\isacharcomma}{\kern0pt}\ simp\ add{\isacharcolon}{\kern0pt}case{\isacharunderscore}{\kern0pt}prod{\isacharunderscore}{\kern0pt}beta\ mem{\isacharunderscore}{\kern0pt}Times{\isacharunderscore}{\kern0pt}iff{\isacharcomma}{\kern0pt}\ simp{\isacharparenright}{\kern0pt}\isanewline
\ \ \ \ \ \ \ \isacommand{apply}\isamarkupfalse%
\ {\isacharparenleft}{\kern0pt}rule\ subsetI{\isacharcomma}{\kern0pt}\ simp\ add{\isacharcolon}{\kern0pt}case{\isacharunderscore}{\kern0pt}prod{\isacharunderscore}{\kern0pt}beta{\isacharparenright}{\kern0pt}\isanewline
\ \ \ \ \ \ \isacommand{by}\isamarkupfalse%
\ simp\isanewline
\ \ \ \ \isacommand{also}\isamarkupfalse%
\ \isacommand{have}\isamarkupfalse%
\ {\isachardoublequoteopen}{\isachardot}{\kern0pt}{\isachardot}{\kern0pt}{\isachardot}{\kern0pt}\ {\isasymle}\ {\isadigit{1}}{\isacharslash}{\kern0pt}{\isacharparenleft}{\kern0pt}real\ p{\isacharparenright}{\kern0pt}\isactrlsup {\isadigit{2}}\ {\isacharasterisk}{\kern0pt}\ {\isacharparenleft}{\kern0pt}{\isasymSum}\ u{\isacharprime}{\kern0pt}\ {\isasymin}\ {\isacharbraceleft}{\kern0pt}u{\isachardot}{\kern0pt}\ u\ {\isacharless}{\kern0pt}\ p\ {\isasymand}\ real\ u\ {\isasymle}\ {\isadigit{2}}\ {\isacharasterisk}{\kern0pt}\ c{\isacharbraceright}{\kern0pt}{\isachardot}{\kern0pt}\isanewline
\ \ \ \ \ \ card\ \ {\isacharbraceleft}{\kern0pt}{\isacharparenleft}{\kern0pt}u{\isacharcolon}{\kern0pt}{\isacharcolon}{\kern0pt}nat{\isacharcomma}{\kern0pt}v{\isacharcolon}{\kern0pt}{\isacharcolon}{\kern0pt}nat{\isacharparenright}{\kern0pt}{\isachardot}{\kern0pt}\ u\ {\isacharequal}{\kern0pt}\ u{\isacharprime}{\kern0pt}\ {\isasymand}\ abs\ {\isacharparenleft}{\kern0pt}real\ u\ {\isacharminus}{\kern0pt}\ real\ v{\isacharparenright}{\kern0pt}\ {\isasymle}\ {\isadigit{2}}\ {\isacharasterisk}{\kern0pt}\ c\ {\isacharasterisk}{\kern0pt}\ {\isadigit{2}}\ powr\ {\isacharparenleft}{\kern0pt}{\isacharminus}{\kern0pt}real\ r{\isacharparenright}{\kern0pt}\ {\isasymand}\ v\ {\isacharless}{\kern0pt}\ p\ {\isasymand}\ v\ {\isasymnoteq}\ u{\isacharprime}{\kern0pt}{\isacharbraceright}{\kern0pt}{\isacharparenright}{\kern0pt}{\isachardoublequoteclose}\isanewline
\ \ \ \ \ \ \isacommand{apply}\isamarkupfalse%
\ {\isacharparenleft}{\kern0pt}rule\ mult{\isacharunderscore}{\kern0pt}left{\isacharunderscore}{\kern0pt}mono{\isacharparenright}{\kern0pt}\isanewline
\ \ \ \ \ \ \ \isacommand{apply}\isamarkupfalse%
\ {\isacharparenleft}{\kern0pt}rule\ of{\isacharunderscore}{\kern0pt}nat{\isacharunderscore}{\kern0pt}mono{\isacharparenright}{\kern0pt}\isanewline
\ \ \ \ \ \ \isacommand{by}\isamarkupfalse%
\ {\isacharparenleft}{\kern0pt}rule\ card{\isacharunderscore}{\kern0pt}UN{\isacharunderscore}{\kern0pt}le{\isacharcomma}{\kern0pt}\ simp{\isacharcomma}{\kern0pt}\ simp{\isacharparenright}{\kern0pt}\isanewline
\ \ \ \ \isacommand{also}\isamarkupfalse%
\ \isacommand{have}\isamarkupfalse%
\ {\isachardoublequoteopen}{\isachardot}{\kern0pt}{\isachardot}{\kern0pt}{\isachardot}{\kern0pt}\ {\isacharequal}{\kern0pt}\ {\isadigit{1}}{\isacharslash}{\kern0pt}{\isacharparenleft}{\kern0pt}real\ p{\isacharparenright}{\kern0pt}\isactrlsup {\isadigit{2}}\ {\isacharasterisk}{\kern0pt}\ {\isacharparenleft}{\kern0pt}{\isasymSum}\ u{\isacharprime}{\kern0pt}\ {\isasymin}\ {\isacharbraceleft}{\kern0pt}u{\isachardot}{\kern0pt}\ u\ {\isacharless}{\kern0pt}\ p\ {\isasymand}\ \ real\ u\ {\isasymle}\ {\isadigit{2}}\ {\isacharasterisk}{\kern0pt}\ c{\isacharbraceright}{\kern0pt}{\isachardot}{\kern0pt}\isanewline
\ \ \ \ \ \ card\ {\isacharparenleft}{\kern0pt}{\isacharparenleft}{\kern0pt}{\isasymlambda}x{\isachardot}{\kern0pt}\ {\isacharparenleft}{\kern0pt}u{\isacharprime}{\kern0pt}\ {\isacharcomma}{\kern0pt}x{\isacharparenright}{\kern0pt}{\isacharparenright}{\kern0pt}\ {\isacharbackquote}{\kern0pt}\ {\isacharbraceleft}{\kern0pt}{\isacharparenleft}{\kern0pt}v{\isacharcolon}{\kern0pt}{\isacharcolon}{\kern0pt}nat{\isacharparenright}{\kern0pt}{\isachardot}{\kern0pt}\ abs\ {\isacharparenleft}{\kern0pt}real\ u{\isacharprime}{\kern0pt}\ {\isacharminus}{\kern0pt}\ real\ v{\isacharparenright}{\kern0pt}\ {\isasymle}\ {\isadigit{2}}\ {\isacharasterisk}{\kern0pt}\ c\ {\isacharasterisk}{\kern0pt}\ {\isadigit{2}}\ powr\ {\isacharparenleft}{\kern0pt}{\isacharminus}{\kern0pt}real\ r{\isacharparenright}{\kern0pt}\ {\isasymand}\ v\ {\isacharless}{\kern0pt}\ p\ {\isasymand}\ v\ {\isasymnoteq}\ u{\isacharprime}{\kern0pt}{\isacharbraceright}{\kern0pt}{\isacharparenright}{\kern0pt}{\isacharparenright}{\kern0pt}{\isachardoublequoteclose}\isanewline
\ \ \ \ \ \ \isacommand{apply}\isamarkupfalse%
\ {\isacharparenleft}{\kern0pt}simp{\isacharcomma}{\kern0pt}\ rule\ disjI{\isadigit{2}}{\isacharcomma}{\kern0pt}\ rule\ sum{\isachardot}{\kern0pt}cong{\isacharcomma}{\kern0pt}\ simp{\isacharparenright}{\kern0pt}\isanewline
\ \ \ \ \ \ \isacommand{apply}\isamarkupfalse%
\ {\isacharparenleft}{\kern0pt}simp{\isacharcomma}{\kern0pt}\ rule\ arg{\isacharunderscore}{\kern0pt}cong{\isacharbrackleft}{\kern0pt}\isakeyword{where}\ f{\isacharequal}{\kern0pt}{\isachardoublequoteopen}card{\isachardoublequoteclose}{\isacharbrackright}{\kern0pt}{\isacharcomma}{\kern0pt}\ subst\ set{\isacharunderscore}{\kern0pt}eq{\isacharunderscore}{\kern0pt}iff{\isacharparenright}{\kern0pt}\isanewline
\ \ \ \ \ \ \isacommand{by}\isamarkupfalse%
\ blast\isanewline
\ \ \ \ \isacommand{also}\isamarkupfalse%
\ \isacommand{have}\isamarkupfalse%
\ {\isachardoublequoteopen}{\isachardot}{\kern0pt}{\isachardot}{\kern0pt}{\isachardot}{\kern0pt}\ {\isasymle}\ {\isadigit{1}}{\isacharslash}{\kern0pt}{\isacharparenleft}{\kern0pt}real\ p{\isacharparenright}{\kern0pt}\isactrlsup {\isadigit{2}}\ {\isacharasterisk}{\kern0pt}\ {\isacharparenleft}{\kern0pt}{\isasymSum}\ u{\isacharprime}{\kern0pt}\ {\isasymin}\ {\isacharbraceleft}{\kern0pt}u{\isachardot}{\kern0pt}\ u\ {\isacharless}{\kern0pt}\ p\ {\isasymand}\ real\ u\ {\isasymle}\ {\isadigit{2}}\ {\isacharasterisk}{\kern0pt}\ c{\isacharbraceright}{\kern0pt}{\isachardot}{\kern0pt}\isanewline
\ \ \ \ \ \ card\ {\isacharbraceleft}{\kern0pt}{\isacharparenleft}{\kern0pt}v{\isacharcolon}{\kern0pt}{\isacharcolon}{\kern0pt}nat{\isacharparenright}{\kern0pt}{\isachardot}{\kern0pt}\ abs\ {\isacharparenleft}{\kern0pt}real\ u{\isacharprime}{\kern0pt}\ {\isacharminus}{\kern0pt}\ real\ v{\isacharparenright}{\kern0pt}\ {\isasymle}\ {\isadigit{2}}\ {\isacharasterisk}{\kern0pt}\ c\ {\isacharasterisk}{\kern0pt}\ {\isadigit{2}}\ powr\ {\isacharparenleft}{\kern0pt}{\isacharminus}{\kern0pt}real\ r{\isacharparenright}{\kern0pt}\ {\isasymand}\ v\ {\isacharless}{\kern0pt}\ p\ {\isasymand}\ v\ {\isasymnoteq}\ u{\isacharprime}{\kern0pt}{\isacharbraceright}{\kern0pt}{\isacharparenright}{\kern0pt}{\isachardoublequoteclose}\isanewline
\ \ \ \ \ \ \isacommand{apply}\isamarkupfalse%
\ {\isacharparenleft}{\kern0pt}rule\ mult{\isacharunderscore}{\kern0pt}left{\isacharunderscore}{\kern0pt}mono{\isacharparenright}{\kern0pt}\isanewline
\ \ \ \ \ \ \ \isacommand{apply}\isamarkupfalse%
\ {\isacharparenleft}{\kern0pt}rule\ of{\isacharunderscore}{\kern0pt}nat{\isacharunderscore}{\kern0pt}mono{\isacharcomma}{\kern0pt}\ rule\ sum{\isacharunderscore}{\kern0pt}mono{\isacharcomma}{\kern0pt}\ rule\ card{\isacharunderscore}{\kern0pt}image{\isacharunderscore}{\kern0pt}le{\isacharcomma}{\kern0pt}\ simp{\isacharparenright}{\kern0pt}\isanewline
\ \ \ \ \ \ \isacommand{by}\isamarkupfalse%
\ simp\isanewline
\ \ \ \ \isacommand{also}\isamarkupfalse%
\ \isacommand{have}\isamarkupfalse%
\ {\isachardoublequoteopen}{\isachardot}{\kern0pt}{\isachardot}{\kern0pt}{\isachardot}{\kern0pt}\ {\isasymle}\ {\isadigit{1}}{\isacharslash}{\kern0pt}{\isacharparenleft}{\kern0pt}real\ p{\isacharparenright}{\kern0pt}\isactrlsup {\isadigit{2}}\ {\isacharasterisk}{\kern0pt}\ {\isacharparenleft}{\kern0pt}{\isasymSum}\ u{\isacharprime}{\kern0pt}\ {\isasymin}\ {\isacharbraceleft}{\kern0pt}u{\isachardot}{\kern0pt}\ u\ {\isacharless}{\kern0pt}\ p\ {\isasymand}\ real\ u\ {\isasymle}\ {\isadigit{2}}\ {\isacharasterisk}{\kern0pt}\ c{\isacharbraceright}{\kern0pt}{\isachardot}{\kern0pt}\isanewline
\ \ \ \ \ \ card\ {\isacharbraceleft}{\kern0pt}{\isacharparenleft}{\kern0pt}v{\isacharcolon}{\kern0pt}{\isacharcolon}{\kern0pt}nat{\isacharparenright}{\kern0pt}{\isachardot}{\kern0pt}\ abs\ {\isacharparenleft}{\kern0pt}real\ u{\isacharprime}{\kern0pt}\ {\isacharminus}{\kern0pt}\ real\ v{\isacharparenright}{\kern0pt}\ {\isasymle}\ {\isadigit{2}}\ {\isacharasterisk}{\kern0pt}\ c\ {\isacharasterisk}{\kern0pt}\ {\isadigit{2}}\ powr\ {\isacharparenleft}{\kern0pt}{\isacharminus}{\kern0pt}real\ r{\isacharparenright}{\kern0pt}\ {\isasymand}\ v\ {\isasymnoteq}\ u{\isacharprime}{\kern0pt}{\isacharbraceright}{\kern0pt}{\isacharparenright}{\kern0pt}{\isachardoublequoteclose}\isanewline
\ \ \ \ \ \ \isacommand{apply}\isamarkupfalse%
\ {\isacharparenleft}{\kern0pt}rule\ mult{\isacharunderscore}{\kern0pt}left{\isacharunderscore}{\kern0pt}mono{\isacharparenright}{\kern0pt}\isanewline
\ \ \ \ \ \ \ \isacommand{apply}\isamarkupfalse%
\ {\isacharparenleft}{\kern0pt}rule\ of{\isacharunderscore}{\kern0pt}nat{\isacharunderscore}{\kern0pt}mono{\isacharcomma}{\kern0pt}\ rule\ sum{\isacharunderscore}{\kern0pt}mono{\isacharcomma}{\kern0pt}\ rule\ card{\isacharunderscore}{\kern0pt}mono{\isacharparenright}{\kern0pt}\isanewline
\ \ \ \ \ \ \ \ \isacommand{apply}\isamarkupfalse%
\ {\isacharparenleft}{\kern0pt}rule\ count{\isacharunderscore}{\kern0pt}nat{\isacharunderscore}{\kern0pt}abs{\isacharunderscore}{\kern0pt}diff{\isacharunderscore}{\kern0pt}{\isadigit{2}}{\isacharparenleft}{\kern0pt}{\isadigit{2}}{\isacharparenright}{\kern0pt}{\isacharcomma}{\kern0pt}\ simp{\isacharparenright}{\kern0pt}\isanewline
\ \ \ \ \ \ \isacommand{by}\isamarkupfalse%
\ {\isacharparenleft}{\kern0pt}rule\ subsetI{\isacharcomma}{\kern0pt}\ simp{\isacharcomma}{\kern0pt}\ simp{\isacharparenright}{\kern0pt}\isanewline
\ \ \ \ \isacommand{also}\isamarkupfalse%
\ \isacommand{have}\isamarkupfalse%
\ {\isachardoublequoteopen}{\isachardot}{\kern0pt}{\isachardot}{\kern0pt}{\isachardot}{\kern0pt}\ {\isasymle}\ {\isadigit{1}}{\isacharslash}{\kern0pt}{\isacharparenleft}{\kern0pt}real\ p{\isacharparenright}{\kern0pt}\isactrlsup {\isadigit{2}}\ {\isacharasterisk}{\kern0pt}\ {\isacharparenleft}{\kern0pt}{\isasymSum}\ u{\isacharprime}{\kern0pt}\ {\isasymin}\ {\isacharbraceleft}{\kern0pt}u{\isachardot}{\kern0pt}\ u\ {\isacharless}{\kern0pt}\ p\ {\isasymand}\ real\ u\ {\isasymle}\ {\isadigit{2}}\ {\isacharasterisk}{\kern0pt}\ c{\isacharbraceright}{\kern0pt}{\isachardot}{\kern0pt}\isanewline
\ \ \ \ \ \ {\isadigit{2}}\ {\isacharasterisk}{\kern0pt}\ {\isacharparenleft}{\kern0pt}{\isadigit{2}}\ {\isacharasterisk}{\kern0pt}\ c\ {\isacharasterisk}{\kern0pt}\ {\isadigit{2}}\ powr\ {\isacharparenleft}{\kern0pt}{\isacharminus}{\kern0pt}real\ r{\isacharparenright}{\kern0pt}{\isacharparenright}{\kern0pt}{\isacharparenright}{\kern0pt}{\isachardoublequoteclose}\isanewline
\ \ \ \ \ \ \isacommand{apply}\isamarkupfalse%
\ {\isacharparenleft}{\kern0pt}rule\ mult{\isacharunderscore}{\kern0pt}left{\isacharunderscore}{\kern0pt}mono{\isacharparenright}{\kern0pt}\isanewline
\ \ \ \ \ \ \ \isacommand{apply}\isamarkupfalse%
\ {\isacharparenleft}{\kern0pt}subst\ of{\isacharunderscore}{\kern0pt}nat{\isacharunderscore}{\kern0pt}sum{\isacharparenright}{\kern0pt}\isanewline
\ \ \ \ \ \ \ \isacommand{apply}\isamarkupfalse%
\ {\isacharparenleft}{\kern0pt}rule\ sum{\isacharunderscore}{\kern0pt}mono{\isacharparenright}{\kern0pt}\isanewline
\ \ \ \ \ \ \ \isacommand{apply}\isamarkupfalse%
\ {\isacharparenleft}{\kern0pt}rule\ count{\isacharunderscore}{\kern0pt}nat{\isacharunderscore}{\kern0pt}abs{\isacharunderscore}{\kern0pt}diff{\isacharunderscore}{\kern0pt}{\isadigit{2}}{\isacharparenleft}{\kern0pt}{\isadigit{1}}{\isacharparenright}{\kern0pt}{\isacharcomma}{\kern0pt}\ simp{\isacharparenright}{\kern0pt}\isanewline
\ \ \ \ \ \ \isacommand{by}\isamarkupfalse%
\ simp\isanewline
\ \ \ \ \isacommand{also}\isamarkupfalse%
\ \isacommand{have}\isamarkupfalse%
\ {\isachardoublequoteopen}{\isachardot}{\kern0pt}{\isachardot}{\kern0pt}{\isachardot}{\kern0pt}\ {\isasymle}\ \ {\isadigit{1}}{\isacharslash}{\kern0pt}{\isacharparenleft}{\kern0pt}real\ p{\isacharparenright}{\kern0pt}\isactrlsup {\isadigit{2}}\ {\isacharasterisk}{\kern0pt}\ {\isacharparenleft}{\kern0pt}real\ {\isacharparenleft}{\kern0pt}card\ {\isacharbraceleft}{\kern0pt}u{\isachardot}{\kern0pt}\ u\ {\isasymle}\ nat\ {\isacharparenleft}{\kern0pt}{\isasymlfloor}{\isadigit{2}}\ {\isacharasterisk}{\kern0pt}\ c\ {\isasymrfloor}{\isacharparenright}{\kern0pt}{\isacharbraceright}{\kern0pt}{\isacharparenright}{\kern0pt}\ {\isacharasterisk}{\kern0pt}\ {\isacharparenleft}{\kern0pt}{\isadigit{2}}\ {\isacharasterisk}{\kern0pt}\ {\isacharparenleft}{\kern0pt}{\isadigit{2}}\ {\isacharasterisk}{\kern0pt}\ c\ {\isacharasterisk}{\kern0pt}\ {\isadigit{2}}\ powr\ {\isacharparenleft}{\kern0pt}{\isacharminus}{\kern0pt}real\ r{\isacharparenright}{\kern0pt}{\isacharparenright}{\kern0pt}{\isacharparenright}{\kern0pt}{\isacharparenright}{\kern0pt}{\isachardoublequoteclose}\isanewline
\ \ \ \ \ \ \isacommand{apply}\isamarkupfalse%
\ {\isacharparenleft}{\kern0pt}rule\ mult{\isacharunderscore}{\kern0pt}left{\isacharunderscore}{\kern0pt}mono{\isacharparenright}{\kern0pt}\isanewline
\ \ \ \ \ \ \ \isacommand{apply}\isamarkupfalse%
\ {\isacharparenleft}{\kern0pt}subst\ sum{\isacharunderscore}{\kern0pt}constant{\isacharparenright}{\kern0pt}\isanewline
\ \ \ \ \ \ \ \isacommand{apply}\isamarkupfalse%
\ {\isacharparenleft}{\kern0pt}rule\ mult{\isacharunderscore}{\kern0pt}right{\isacharunderscore}{\kern0pt}mono{\isacharparenright}{\kern0pt}\isanewline
\ \ \ \ \ \ \ \ \isacommand{apply}\isamarkupfalse%
\ {\isacharparenleft}{\kern0pt}rule\ of{\isacharunderscore}{\kern0pt}nat{\isacharunderscore}{\kern0pt}mono{\isacharcomma}{\kern0pt}\ rule\ card{\isacharunderscore}{\kern0pt}mono{\isacharcomma}{\kern0pt}\ simp{\isacharparenright}{\kern0pt}\isanewline
\ \ \ \ \ \ \ \ \isacommand{apply}\isamarkupfalse%
\ {\isacharparenleft}{\kern0pt}rule\ subsetI{\isacharcomma}{\kern0pt}\ simp{\isacharparenright}{\kern0pt}\ \isacommand{using}\isamarkupfalse%
\ c{\isacharunderscore}{\kern0pt}ge{\isacharunderscore}{\kern0pt}{\isadigit{0}}\ le{\isacharunderscore}{\kern0pt}nat{\isacharunderscore}{\kern0pt}floor\ \isacommand{apply}\isamarkupfalse%
\ blast\isanewline
\ \ \ \ \ \ \ \isacommand{apply}\isamarkupfalse%
\ {\isacharparenleft}{\kern0pt}simp\ add{\isacharcolon}{\kern0pt}\ c{\isacharunderscore}{\kern0pt}ge{\isacharunderscore}{\kern0pt}{\isadigit{0}}{\isacharparenright}{\kern0pt}\isanewline
\ \ \ \ \ \ \isacommand{by}\isamarkupfalse%
\ simp\isanewline
\ \ \ \ \isacommand{also}\isamarkupfalse%
\ \isacommand{have}\isamarkupfalse%
\ {\isachardoublequoteopen}{\isachardot}{\kern0pt}{\isachardot}{\kern0pt}{\isachardot}{\kern0pt}\ {\isasymle}\ \ {\isadigit{1}}{\isacharslash}{\kern0pt}{\isacharparenleft}{\kern0pt}real\ p{\isacharparenright}{\kern0pt}\isactrlsup {\isadigit{2}}\ {\isacharasterisk}{\kern0pt}\ {\isacharparenleft}{\kern0pt}{\isacharparenleft}{\kern0pt}{\isadigit{3}}\ {\isacharasterisk}{\kern0pt}\ c{\isacharparenright}{\kern0pt}\ {\isacharasterisk}{\kern0pt}\ {\isacharparenleft}{\kern0pt}{\isadigit{2}}\ {\isacharasterisk}{\kern0pt}\ {\isacharparenleft}{\kern0pt}{\isadigit{2}}\ {\isacharasterisk}{\kern0pt}\ c\ {\isacharasterisk}{\kern0pt}\ {\isadigit{2}}\ powr\ {\isacharparenleft}{\kern0pt}{\isacharminus}{\kern0pt}real\ r{\isacharparenright}{\kern0pt}{\isacharparenright}{\kern0pt}{\isacharparenright}{\kern0pt}{\isacharparenright}{\kern0pt}{\isachardoublequoteclose}\isanewline
\ \ \ \ \ \ \isacommand{apply}\isamarkupfalse%
\ {\isacharparenleft}{\kern0pt}rule\ mult{\isacharunderscore}{\kern0pt}left{\isacharunderscore}{\kern0pt}mono{\isacharparenright}{\kern0pt}\isanewline
\ \ \ \ \ \ \ \isacommand{apply}\isamarkupfalse%
\ {\isacharparenleft}{\kern0pt}rule\ mult{\isacharunderscore}{\kern0pt}right{\isacharunderscore}{\kern0pt}mono{\isacharparenright}{\kern0pt}\isanewline
\ \ \ \ \ \ \isacommand{apply}\isamarkupfalse%
\ simp\ \isacommand{using}\isamarkupfalse%
\ assms{\isacharparenleft}{\kern0pt}{\isadigit{4}}{\isacharparenright}{\kern0pt}\ \isacommand{apply}\isamarkupfalse%
\ linarith\isanewline
\ \ \ \ \ \ \isacommand{by}\isamarkupfalse%
\ {\isacharparenleft}{\kern0pt}simp\ add{\isacharcolon}{\kern0pt}\ c{\isacharunderscore}{\kern0pt}ge{\isacharunderscore}{\kern0pt}{\isadigit{0}}{\isacharparenright}{\kern0pt}{\isacharplus}{\kern0pt}\isanewline
\ \ \ \ \isacommand{also}\isamarkupfalse%
\ \isacommand{have}\isamarkupfalse%
\ {\isachardoublequoteopen}{\isachardot}{\kern0pt}{\isachardot}{\kern0pt}{\isachardot}{\kern0pt}\ {\isacharequal}{\kern0pt}\ {\isadigit{1}}{\isadigit{2}}\ \ {\isacharasterisk}{\kern0pt}\ c\isactrlsup {\isadigit{2}}\ {\isacharasterisk}{\kern0pt}\ {\isadigit{2}}\ powr\ {\isacharparenleft}{\kern0pt}{\isacharminus}{\kern0pt}real\ r{\isacharparenright}{\kern0pt}\ {\isacharslash}{\kern0pt}{\isacharparenleft}{\kern0pt}real\ p{\isacharparenright}{\kern0pt}\isactrlsup {\isadigit{2}}{\isachardoublequoteclose}\isanewline
\ \ \ \ \ \ \isacommand{by}\isamarkupfalse%
\ {\isacharparenleft}{\kern0pt}simp\ add{\isacharcolon}{\kern0pt}ac{\isacharunderscore}{\kern0pt}simps\ power{\isadigit{2}}{\isacharunderscore}{\kern0pt}eq{\isacharunderscore}{\kern0pt}square{\isacharparenright}{\kern0pt}\ \isanewline
\ \ \ \ \isacommand{finally}\isamarkupfalse%
\ \isacommand{show}\isamarkupfalse%
\ {\isachardoublequoteopen}measure\ {\isasymOmega}\ {\isacharbraceleft}{\kern0pt}{\isasymomega}{\isachardot}{\kern0pt}\ degree\ {\isasymomega}\ {\isasymge}\ {\isadigit{1}}\ {\isasymand}\ truncate{\isacharunderscore}{\kern0pt}down\ r\ {\isacharparenleft}{\kern0pt}hash\ p\ x\ {\isasymomega}{\isacharparenright}{\kern0pt}\ {\isasymle}\ c\ {\isasymand}\isanewline
\ \ \ \ \ \ truncate{\isacharunderscore}{\kern0pt}down\ r\ {\isacharparenleft}{\kern0pt}hash\ p\ x\ {\isasymomega}{\isacharparenright}{\kern0pt}\ {\isacharequal}{\kern0pt}\ truncate{\isacharunderscore}{\kern0pt}down\ r\ {\isacharparenleft}{\kern0pt}hash\ p\ y\ {\isasymomega}{\isacharparenright}{\kern0pt}{\isacharbraceright}{\kern0pt}\ {\isasymle}\ \ {\isadigit{1}}{\isadigit{2}}\ \ {\isacharasterisk}{\kern0pt}\ c\isactrlsup {\isadigit{2}}\ {\isacharasterisk}{\kern0pt}\ {\isadigit{2}}\ powr\ {\isacharparenleft}{\kern0pt}{\isacharminus}{\kern0pt}real\ r{\isacharparenright}{\kern0pt}\ {\isacharslash}{\kern0pt}{\isacharparenleft}{\kern0pt}real\ p{\isacharparenright}{\kern0pt}\isactrlsup {\isadigit{2}}{\isachardoublequoteclose}\isanewline
\ \ \ \ \ \ \isacommand{by}\isamarkupfalse%
\ simp\isanewline
\ \ \isacommand{qed}\isamarkupfalse%
\isanewline
\isanewline
\ \ \isacommand{have}\isamarkupfalse%
\ {\isachardoublequoteopen}{\isasymP}{\isacharparenleft}{\kern0pt}{\isasymomega}\ in\ measure{\isacharunderscore}{\kern0pt}pmf\ {\isasymOmega}{\isachardot}{\kern0pt}\ {\isacharquery}{\kern0pt}l\ {\isasymomega}\ {\isasymand}\ degree\ {\isasymomega}\ {\isasymge}\ {\isadigit{1}}{\isacharparenright}{\kern0pt}\ {\isasymle}\ \isanewline
\ \ \ \ measure\ {\isasymOmega}\ {\isacharparenleft}{\kern0pt}{\isasymUnion}\ i\ {\isasymin}\ {\isacharbraceleft}{\kern0pt}{\isacharparenleft}{\kern0pt}x{\isacharcomma}{\kern0pt}y{\isacharparenright}{\kern0pt}\ {\isasymin}\ M\ {\isasymtimes}\ M{\isachardot}{\kern0pt}\ x\ {\isacharless}{\kern0pt}\ y{\isacharbraceright}{\kern0pt}{\isachardot}{\kern0pt}\ {\isacharbraceleft}{\kern0pt}{\isasymomega}{\isachardot}{\kern0pt}\ \isanewline
\ \ \ \ degree\ {\isasymomega}\ {\isasymge}\ {\isadigit{1}}\ {\isasymand}\ truncate{\isacharunderscore}{\kern0pt}down\ r\ {\isacharparenleft}{\kern0pt}hash\ p\ {\isacharparenleft}{\kern0pt}fst\ i{\isacharparenright}{\kern0pt}\ {\isasymomega}{\isacharparenright}{\kern0pt}\ {\isasymle}\ c\ {\isasymand}\isanewline
\ \ \ \ truncate{\isacharunderscore}{\kern0pt}down\ r\ {\isacharparenleft}{\kern0pt}hash\ p\ {\isacharparenleft}{\kern0pt}fst\ i{\isacharparenright}{\kern0pt}\ {\isasymomega}{\isacharparenright}{\kern0pt}\ {\isacharequal}{\kern0pt}\ truncate{\isacharunderscore}{\kern0pt}down\ r\ {\isacharparenleft}{\kern0pt}hash\ p\ {\isacharparenleft}{\kern0pt}snd\ i{\isacharparenright}{\kern0pt}\ {\isasymomega}{\isacharparenright}{\kern0pt}{\isacharbraceright}{\kern0pt}{\isacharparenright}{\kern0pt}{\isachardoublequoteclose}\isanewline
\ \ \ \ \isacommand{apply}\isamarkupfalse%
\ {\isacharparenleft}{\kern0pt}rule\ pmf{\isacharunderscore}{\kern0pt}mono{\isacharparenright}{\kern0pt}\isanewline
\ \ \ \ \isacommand{apply}\isamarkupfalse%
\ {\isacharparenleft}{\kern0pt}simp{\isacharparenright}{\kern0pt}\ \isanewline
\ \ \ \ \isacommand{by}\isamarkupfalse%
\ {\isacharparenleft}{\kern0pt}metis\ linorder{\isacharunderscore}{\kern0pt}neqE{\isacharunderscore}{\kern0pt}nat{\isacharparenright}{\kern0pt}\isanewline
\ \ \isacommand{also}\isamarkupfalse%
\ \isacommand{have}\isamarkupfalse%
\ {\isachardoublequoteopen}{\isachardot}{\kern0pt}{\isachardot}{\kern0pt}{\isachardot}{\kern0pt}\ {\isasymle}\ {\isacharparenleft}{\kern0pt}{\isasymSum}\ i\ {\isasymin}\ {\isacharbraceleft}{\kern0pt}{\isacharparenleft}{\kern0pt}x{\isacharcomma}{\kern0pt}y{\isacharparenright}{\kern0pt}\ {\isasymin}\ M\ {\isasymtimes}\ M{\isachardot}{\kern0pt}\ x\ {\isacharless}{\kern0pt}\ y{\isacharbraceright}{\kern0pt}{\isachardot}{\kern0pt}\ measure\ {\isasymOmega}\ \isanewline
\ \ \ \ {\isacharbraceleft}{\kern0pt}{\isasymomega}{\isachardot}{\kern0pt}\ degree\ {\isasymomega}\ {\isasymge}\ {\isadigit{1}}\ {\isasymand}\ truncate{\isacharunderscore}{\kern0pt}down\ r\ {\isacharparenleft}{\kern0pt}hash\ p\ {\isacharparenleft}{\kern0pt}fst\ i{\isacharparenright}{\kern0pt}\ {\isasymomega}{\isacharparenright}{\kern0pt}\ {\isasymle}\ c\ {\isasymand}\isanewline
\ \ \ \ truncate{\isacharunderscore}{\kern0pt}down\ r\ {\isacharparenleft}{\kern0pt}hash\ p\ {\isacharparenleft}{\kern0pt}fst\ i{\isacharparenright}{\kern0pt}\ {\isasymomega}{\isacharparenright}{\kern0pt}\ {\isacharequal}{\kern0pt}\ truncate{\isacharunderscore}{\kern0pt}down\ r\ {\isacharparenleft}{\kern0pt}hash\ p\ {\isacharparenleft}{\kern0pt}snd\ i{\isacharparenright}{\kern0pt}\ {\isasymomega}{\isacharparenright}{\kern0pt}{\isacharbraceright}{\kern0pt}{\isacharparenright}{\kern0pt}{\isachardoublequoteclose}\isanewline
\ \ \ \ \isacommand{apply}\isamarkupfalse%
\ {\isacharparenleft}{\kern0pt}rule\ measure{\isacharunderscore}{\kern0pt}UNION{\isacharunderscore}{\kern0pt}le{\isacharparenright}{\kern0pt}\isanewline
\ \ \ \ \ \isacommand{apply}\isamarkupfalse%
\ {\isacharparenleft}{\kern0pt}rule\ finite{\isacharunderscore}{\kern0pt}subset{\isacharbrackleft}{\kern0pt}\isakeyword{where}\ B{\isacharequal}{\kern0pt}{\isachardoublequoteopen}M\ {\isasymtimes}\ M{\isachardoublequoteclose}{\isacharbrackright}{\kern0pt}{\isacharcomma}{\kern0pt}\ rule\ subsetI{\isacharcomma}{\kern0pt}\ simp\ add{\isacharcolon}{\kern0pt}case{\isacharunderscore}{\kern0pt}prod{\isacharunderscore}{\kern0pt}beta\ mem{\isacharunderscore}{\kern0pt}Times{\isacharunderscore}{\kern0pt}iff{\isacharparenright}{\kern0pt}\isanewline
\ \ \ \ \ \isacommand{apply}\isamarkupfalse%
\ {\isacharparenleft}{\kern0pt}rule\ finite{\isacharunderscore}{\kern0pt}cartesian{\isacharunderscore}{\kern0pt}product{\isacharbrackleft}{\kern0pt}OF\ f{\isacharunderscore}{\kern0pt}M\ f{\isacharunderscore}{\kern0pt}M{\isacharbrackright}{\kern0pt}{\isacharparenright}{\kern0pt}\isanewline
\ \ \ \ \isacommand{by}\isamarkupfalse%
\ simp\isanewline
\ \ \isacommand{also}\isamarkupfalse%
\ \isacommand{have}\isamarkupfalse%
\ {\isachardoublequoteopen}{\isachardot}{\kern0pt}{\isachardot}{\kern0pt}{\isachardot}{\kern0pt}\ {\isasymle}\ {\isacharparenleft}{\kern0pt}{\isasymSum}\ i\ {\isasymin}\ {\isacharbraceleft}{\kern0pt}{\isacharparenleft}{\kern0pt}x{\isacharcomma}{\kern0pt}y{\isacharparenright}{\kern0pt}\ {\isasymin}\ M\ {\isasymtimes}\ M{\isachardot}{\kern0pt}\ x\ {\isacharless}{\kern0pt}\ y{\isacharbraceright}{\kern0pt}{\isachardot}{\kern0pt}\ {\isadigit{1}}{\isadigit{2}}\ \ {\isacharasterisk}{\kern0pt}\ c\isactrlsup {\isadigit{2}}\ {\isacharasterisk}{\kern0pt}\ {\isadigit{2}}\ powr\ {\isacharparenleft}{\kern0pt}{\isacharminus}{\kern0pt}real\ r{\isacharparenright}{\kern0pt}\ {\isacharslash}{\kern0pt}{\isacharparenleft}{\kern0pt}real\ p{\isacharparenright}{\kern0pt}\isactrlsup {\isadigit{2}}{\isacharparenright}{\kern0pt}{\isachardoublequoteclose}\isanewline
\ \ \ \ \isacommand{apply}\isamarkupfalse%
\ {\isacharparenleft}{\kern0pt}rule\ sum{\isacharunderscore}{\kern0pt}mono{\isacharparenright}{\kern0pt}\isanewline
\ \ \ \ \isacommand{using}\isamarkupfalse%
\ a{\isadigit{1}}\ \isacommand{by}\isamarkupfalse%
\ {\isacharparenleft}{\kern0pt}simp\ add{\isacharcolon}{\kern0pt}case{\isacharunderscore}{\kern0pt}prod{\isacharunderscore}{\kern0pt}beta{\isacharparenright}{\kern0pt}\isanewline
\ \ \isacommand{also}\isamarkupfalse%
\ \isacommand{have}\isamarkupfalse%
\ {\isachardoublequoteopen}{\isachardot}{\kern0pt}{\isachardot}{\kern0pt}{\isachardot}{\kern0pt}\ {\isacharequal}{\kern0pt}\ \ {\isacharparenleft}{\kern0pt}{\isadigit{1}}{\isadigit{2}}\ {\isacharasterisk}{\kern0pt}\ c\isactrlsup {\isadigit{2}}\ \ {\isacharasterisk}{\kern0pt}\ {\isadigit{2}}\ powr\ {\isacharparenleft}{\kern0pt}{\isacharminus}{\kern0pt}real\ r{\isacharparenright}{\kern0pt}\ {\isacharslash}{\kern0pt}{\isacharparenleft}{\kern0pt}real\ p{\isacharparenright}{\kern0pt}\isactrlsup {\isadigit{2}}{\isacharparenright}{\kern0pt}\ {\isacharasterisk}{\kern0pt}\ card\ \ {\isacharbraceleft}{\kern0pt}{\isacharparenleft}{\kern0pt}x{\isacharcomma}{\kern0pt}y{\isacharparenright}{\kern0pt}\ {\isasymin}\ M\ {\isasymtimes}\ M{\isachardot}{\kern0pt}\ x\ {\isacharless}{\kern0pt}\ y{\isacharbraceright}{\kern0pt}{\isachardoublequoteclose}\isanewline
\ \ \ \ \isacommand{by}\isamarkupfalse%
\ simp\isanewline
\ \ \isacommand{also}\isamarkupfalse%
\ \isacommand{have}\isamarkupfalse%
\ {\isachardoublequoteopen}{\isachardot}{\kern0pt}{\isachardot}{\kern0pt}{\isachardot}{\kern0pt}\ {\isasymle}\ {\isacharparenleft}{\kern0pt}{\isadigit{1}}{\isadigit{2}}\ {\isacharasterisk}{\kern0pt}\ c\isactrlsup {\isadigit{2}}\ {\isacharasterisk}{\kern0pt}\ {\isadigit{2}}\ powr\ {\isacharparenleft}{\kern0pt}{\isacharminus}{\kern0pt}real\ r{\isacharparenright}{\kern0pt}\ {\isacharslash}{\kern0pt}{\isacharparenleft}{\kern0pt}real\ p{\isacharparenright}{\kern0pt}\isactrlsup {\isadigit{2}}{\isacharparenright}{\kern0pt}\ {\isacharasterisk}{\kern0pt}\ {\isacharparenleft}{\kern0pt}{\isacharparenleft}{\kern0pt}real\ {\isacharparenleft}{\kern0pt}card\ M{\isacharparenright}{\kern0pt}{\isacharparenright}{\kern0pt}\isactrlsup {\isadigit{2}}{\isacharslash}{\kern0pt}real\ {\isadigit{2}}{\isacharparenright}{\kern0pt}{\isachardoublequoteclose}\isanewline
\ \ \ \ \isacommand{apply}\isamarkupfalse%
\ {\isacharparenleft}{\kern0pt}rule\ mult{\isacharunderscore}{\kern0pt}left{\isacharunderscore}{\kern0pt}mono{\isacharparenright}{\kern0pt}\isanewline
\ \ \ \ \ \isacommand{apply}\isamarkupfalse%
\ {\isacharparenleft}{\kern0pt}subst\ pos{\isacharunderscore}{\kern0pt}le{\isacharunderscore}{\kern0pt}divide{\isacharunderscore}{\kern0pt}eq{\isacharcomma}{\kern0pt}\ simp{\isacharparenright}{\kern0pt}\isanewline
\ \ \ \ \ \isacommand{apply}\isamarkupfalse%
\ {\isacharparenleft}{\kern0pt}subst\ mult{\isachardot}{\kern0pt}commute{\isacharparenright}{\kern0pt}\isanewline
\ \ \ \ \ \isacommand{apply}\isamarkupfalse%
\ {\isacharparenleft}{\kern0pt}subst\ of{\isacharunderscore}{\kern0pt}nat{\isacharunderscore}{\kern0pt}mult{\isacharbrackleft}{\kern0pt}symmetric{\isacharbrackright}{\kern0pt}{\isacharparenright}{\kern0pt}\isanewline
\ \ \ \ \ \isacommand{apply}\isamarkupfalse%
\ {\isacharparenleft}{\kern0pt}subst\ card{\isacharunderscore}{\kern0pt}ordered{\isacharunderscore}{\kern0pt}pairs{\isacharcomma}{\kern0pt}\ rule\ finite{\isacharunderscore}{\kern0pt}subset{\isacharbrackleft}{\kern0pt}OF\ assms{\isacharparenleft}{\kern0pt}{\isadigit{3}}{\isacharparenright}{\kern0pt}{\isacharbrackright}{\kern0pt}{\isacharcomma}{\kern0pt}\ simp{\isacharparenright}{\kern0pt}\isanewline
\ \ \ \ \ \isacommand{apply}\isamarkupfalse%
\ {\isacharparenleft}{\kern0pt}subst\ of{\isacharunderscore}{\kern0pt}nat{\isacharunderscore}{\kern0pt}power{\isacharbrackleft}{\kern0pt}symmetric{\isacharbrackright}{\kern0pt}{\isacharcomma}{\kern0pt}\ rule\ of{\isacharunderscore}{\kern0pt}nat{\isacharunderscore}{\kern0pt}mono{\isacharparenright}{\kern0pt}\isanewline
\ \ \ \ \ \isacommand{apply}\isamarkupfalse%
\ {\isacharparenleft}{\kern0pt}simp\ add{\isacharcolon}{\kern0pt}power{\isadigit{2}}{\isacharunderscore}{\kern0pt}eq{\isacharunderscore}{\kern0pt}square{\isacharparenright}{\kern0pt}\isanewline
\ \ \ \ \isacommand{by}\isamarkupfalse%
\ {\isacharparenleft}{\kern0pt}simp\ add{\isacharcolon}{\kern0pt}c{\isacharunderscore}{\kern0pt}ge{\isacharunderscore}{\kern0pt}{\isadigit{0}}{\isacharparenright}{\kern0pt}\isanewline
\ \ \isacommand{also}\isamarkupfalse%
\ \isacommand{have}\isamarkupfalse%
\ {\isachardoublequoteopen}{\isachardot}{\kern0pt}{\isachardot}{\kern0pt}{\isachardot}{\kern0pt}\ {\isacharequal}{\kern0pt}\ {\isadigit{6}}\ {\isacharasterisk}{\kern0pt}\ {\isacharparenleft}{\kern0pt}real\ {\isacharparenleft}{\kern0pt}card\ M{\isacharparenright}{\kern0pt}{\isacharparenright}{\kern0pt}\isactrlsup {\isadigit{2}}\ {\isacharasterisk}{\kern0pt}\ c\isactrlsup {\isadigit{2}}\ {\isacharasterisk}{\kern0pt}\ {\isadigit{2}}\ powr\ {\isacharparenleft}{\kern0pt}{\isacharminus}{\kern0pt}real\ r{\isacharparenright}{\kern0pt}\ {\isacharslash}{\kern0pt}{\isacharparenleft}{\kern0pt}real\ p{\isacharparenright}{\kern0pt}\isactrlsup {\isadigit{2}}{\isachardoublequoteclose}\isanewline
\ \ \ \ \isacommand{by}\isamarkupfalse%
\ {\isacharparenleft}{\kern0pt}simp\ add{\isacharcolon}{\kern0pt}algebra{\isacharunderscore}{\kern0pt}simps{\isacharparenright}{\kern0pt}\isanewline
\ \ \isacommand{finally}\isamarkupfalse%
\ \isacommand{have}\isamarkupfalse%
\ a{\isacharcolon}{\kern0pt}{\isachardoublequoteopen}{\isasymP}{\isacharparenleft}{\kern0pt}{\isasymomega}\ in\ measure{\isacharunderscore}{\kern0pt}pmf\ {\isasymOmega}{\isachardot}{\kern0pt}\ {\isacharquery}{\kern0pt}l\ {\isasymomega}\ {\isasymand}\ degree\ {\isasymomega}\ {\isasymge}\ {\isadigit{1}}{\isacharparenright}{\kern0pt}\ {\isasymle}\ {\isacharquery}{\kern0pt}r{\isadigit{1}}{\isachardoublequoteclose}\ \isacommand{by}\isamarkupfalse%
\ simp\isanewline
\isanewline
\ \ \isacommand{have}\isamarkupfalse%
\ b{\isadigit{1}}{\isacharcolon}{\kern0pt}\ {\isachardoublequoteopen}bounded{\isacharunderscore}{\kern0pt}degree{\isacharunderscore}{\kern0pt}polynomials\ {\isacharparenleft}{\kern0pt}ZFact\ {\isacharparenleft}{\kern0pt}int\ p{\isacharparenright}{\kern0pt}{\isacharparenright}{\kern0pt}\ {\isadigit{2}}\ {\isasyminter}\ {\isacharbraceleft}{\kern0pt}{\isasymomega}{\isachardot}{\kern0pt}\ length\ {\isasymomega}\ {\isasymle}\ Suc\ {\isadigit{0}}{\isacharbraceright}{\kern0pt}\isanewline
\ \ \ \ {\isacharequal}{\kern0pt}\ bounded{\isacharunderscore}{\kern0pt}degree{\isacharunderscore}{\kern0pt}polynomials\ {\isacharparenleft}{\kern0pt}ZFact\ {\isacharparenleft}{\kern0pt}int\ p{\isacharparenright}{\kern0pt}{\isacharparenright}{\kern0pt}\ {\isadigit{1}}{\isachardoublequoteclose}\isanewline
\ \ \ \ \isacommand{apply}\isamarkupfalse%
\ {\isacharparenleft}{\kern0pt}rule\ order{\isacharunderscore}{\kern0pt}antisym{\isacharparenright}{\kern0pt}\isanewline
\ \ \ \ \ \isacommand{apply}\isamarkupfalse%
\ {\isacharparenleft}{\kern0pt}rule\ subsetI{\isacharcomma}{\kern0pt}\ simp\ add{\isacharcolon}{\kern0pt}bounded{\isacharunderscore}{\kern0pt}degree{\isacharunderscore}{\kern0pt}polynomials{\isacharunderscore}{\kern0pt}def{\isacharparenright}{\kern0pt}\ \isanewline
\ \ \ \ \isacommand{by}\isamarkupfalse%
\ {\isacharparenleft}{\kern0pt}rule\ subsetI{\isacharcomma}{\kern0pt}\ simp\ add{\isacharcolon}{\kern0pt}bounded{\isacharunderscore}{\kern0pt}degree{\isacharunderscore}{\kern0pt}polynomials{\isacharunderscore}{\kern0pt}def{\isacharcomma}{\kern0pt}\ fastforce{\isacharparenright}{\kern0pt}\ \isanewline
\isanewline
\ \ \isacommand{have}\isamarkupfalse%
\ b{\isacharcolon}{\kern0pt}\ {\isachardoublequoteopen}\ {\isasymP}{\isacharparenleft}{\kern0pt}{\isasymomega}\ in\ measure{\isacharunderscore}{\kern0pt}pmf\ {\isasymOmega}{\isachardot}{\kern0pt}\ degree\ {\isasymomega}\ {\isacharless}{\kern0pt}\ {\isadigit{1}}{\isacharparenright}{\kern0pt}\ {\isasymle}\ {\isacharquery}{\kern0pt}r{\isadigit{2}}{\isachardoublequoteclose}\ \isanewline
\ \ \ \ \isacommand{apply}\isamarkupfalse%
\ {\isacharparenleft}{\kern0pt}simp\ add{\isacharcolon}{\kern0pt}{\isasymOmega}{\isacharunderscore}{\kern0pt}def{\isacharparenright}{\kern0pt}\isanewline
\ \ \ \ \isacommand{apply}\isamarkupfalse%
\ {\isacharparenleft}{\kern0pt}subst\ measure{\isacharunderscore}{\kern0pt}pmf{\isacharunderscore}{\kern0pt}of{\isacharunderscore}{\kern0pt}set{\isacharparenright}{\kern0pt}\ \isanewline
\ \ \ \ \ \ \ \ \isacommand{apply}\isamarkupfalse%
\ {\isacharparenleft}{\kern0pt}rule\ ne{\isacharunderscore}{\kern0pt}bounded{\isacharunderscore}{\kern0pt}degree{\isacharunderscore}{\kern0pt}polynomials{\isacharparenright}{\kern0pt}\isanewline
\ \ \ \ \ \ \isacommand{apply}\isamarkupfalse%
\ {\isacharparenleft}{\kern0pt}rule\ fin{\isacharunderscore}{\kern0pt}bounded{\isacharunderscore}{\kern0pt}degree{\isacharunderscore}{\kern0pt}polynomials{\isacharbrackleft}{\kern0pt}OF\ p{\isacharunderscore}{\kern0pt}ge{\isacharunderscore}{\kern0pt}{\isadigit{0}}{\isacharbrackright}{\kern0pt}{\isacharparenright}{\kern0pt}\isanewline
\ \ \ \ \isacommand{apply}\isamarkupfalse%
\ {\isacharparenleft}{\kern0pt}subst\ card{\isacharunderscore}{\kern0pt}bounded{\isacharunderscore}{\kern0pt}degree{\isacharunderscore}{\kern0pt}polynomials{\isacharbrackleft}{\kern0pt}OF\ p{\isacharunderscore}{\kern0pt}ge{\isacharunderscore}{\kern0pt}{\isadigit{0}}{\isacharbrackright}{\kern0pt}{\isacharcomma}{\kern0pt}\ subst\ b{\isadigit{1}}{\isacharparenright}{\kern0pt}\isanewline
\ \ \ \ \isacommand{apply}\isamarkupfalse%
\ {\isacharparenleft}{\kern0pt}subst\ card{\isacharunderscore}{\kern0pt}bounded{\isacharunderscore}{\kern0pt}degree{\isacharunderscore}{\kern0pt}polynomials{\isacharbrackleft}{\kern0pt}OF\ p{\isacharunderscore}{\kern0pt}ge{\isacharunderscore}{\kern0pt}{\isadigit{0}}{\isacharbrackright}{\kern0pt}{\isacharparenright}{\kern0pt}\ \isanewline
\ \ \ \ \isacommand{apply}\isamarkupfalse%
\ {\isacharparenleft}{\kern0pt}simp\ add{\isacharcolon}{\kern0pt}zfact{\isacharunderscore}{\kern0pt}card{\isacharbrackleft}{\kern0pt}OF\ p{\isacharunderscore}{\kern0pt}ge{\isacharunderscore}{\kern0pt}{\isadigit{0}}{\isacharbrackright}{\kern0pt}{\isacharparenright}{\kern0pt}\isanewline
\ \ \ \ \isacommand{by}\isamarkupfalse%
\ {\isacharparenleft}{\kern0pt}subst\ pos{\isacharunderscore}{\kern0pt}divide{\isacharunderscore}{\kern0pt}le{\isacharunderscore}{\kern0pt}eq{\isacharcomma}{\kern0pt}\ simp\ add{\isacharcolon}{\kern0pt}p{\isacharunderscore}{\kern0pt}ge{\isacharunderscore}{\kern0pt}{\isadigit{0}}{\isacharcomma}{\kern0pt}\ simp\ add{\isacharcolon}{\kern0pt}power{\isadigit{2}}{\isacharunderscore}{\kern0pt}eq{\isacharunderscore}{\kern0pt}square{\isacharparenright}{\kern0pt}\isanewline
\isanewline
\ \ \isacommand{have}\isamarkupfalse%
\ {\isachardoublequoteopen}{\isasymP}{\isacharparenleft}{\kern0pt}{\isasymomega}\ in\ measure{\isacharunderscore}{\kern0pt}pmf\ {\isasymOmega}{\isachardot}{\kern0pt}\ {\isacharquery}{\kern0pt}l\ {\isasymomega}{\isacharparenright}{\kern0pt}\ {\isasymle}\isanewline
\ \ \ \ {\isasymP}{\isacharparenleft}{\kern0pt}{\isasymomega}\ in\ measure{\isacharunderscore}{\kern0pt}pmf\ {\isasymOmega}{\isachardot}{\kern0pt}\ {\isacharquery}{\kern0pt}l\ {\isasymomega}\ {\isasymand}\ degree\ {\isasymomega}\ {\isasymge}\ {\isadigit{1}}{\isacharparenright}{\kern0pt}\ {\isacharplus}{\kern0pt}\ {\isasymP}{\isacharparenleft}{\kern0pt}{\isasymomega}\ in\ measure{\isacharunderscore}{\kern0pt}pmf\ {\isasymOmega}{\isachardot}{\kern0pt}\ degree\ {\isasymomega}\ {\isacharless}{\kern0pt}\ {\isadigit{1}}{\isacharparenright}{\kern0pt}{\isachardoublequoteclose}\isanewline
\ \ \ \ \isacommand{by}\isamarkupfalse%
\ {\isacharparenleft}{\kern0pt}rule\ pmf{\isacharunderscore}{\kern0pt}add{\isacharcomma}{\kern0pt}\ simp{\isacharcomma}{\kern0pt}\ linarith{\isacharparenright}{\kern0pt}\isanewline
\ \ \isacommand{also}\isamarkupfalse%
\ \isacommand{have}\isamarkupfalse%
\ {\isachardoublequoteopen}{\isachardot}{\kern0pt}{\isachardot}{\kern0pt}{\isachardot}{\kern0pt}\ {\isasymle}\ {\isacharquery}{\kern0pt}r{\isadigit{1}}\ {\isacharplus}{\kern0pt}\ {\isacharquery}{\kern0pt}r{\isadigit{2}}{\isachardoublequoteclose}\ \isacommand{by}\isamarkupfalse%
\ {\isacharparenleft}{\kern0pt}rule\ add{\isacharunderscore}{\kern0pt}mono{\isacharcomma}{\kern0pt}\ metis\ a{\isacharcomma}{\kern0pt}\ metis\ b{\isacharparenright}{\kern0pt}\isanewline
\ \ \isacommand{finally}\isamarkupfalse%
\ \isacommand{show}\isamarkupfalse%
\ {\isacharquery}{\kern0pt}thesis\ \isacommand{by}\isamarkupfalse%
\ simp\isanewline
\isacommand{qed}\isamarkupfalse%
%
\endisatagproof
{\isafoldproof}%
%
\isadelimproof
\isanewline
%
\endisadelimproof
\isanewline
\isacommand{lemma}\isamarkupfalse%
\ inters{\isacharunderscore}{\kern0pt}compr{\isacharcolon}{\kern0pt}\ {\isachardoublequoteopen}A\ {\isasyminter}\ {\isacharbraceleft}{\kern0pt}x{\isachardot}{\kern0pt}\ P\ x{\isacharbraceright}{\kern0pt}\ {\isacharequal}{\kern0pt}\ {\isacharbraceleft}{\kern0pt}x\ {\isasymin}\ A{\isachardot}{\kern0pt}\ P\ x{\isacharbraceright}{\kern0pt}{\isachardoublequoteclose}\isanewline
%
\isadelimproof
\ \ %
\endisadelimproof
%
\isatagproof
\isacommand{by}\isamarkupfalse%
\ blast%
\endisatagproof
{\isafoldproof}%
%
\isadelimproof
\isanewline
%
\endisadelimproof
\isanewline
\isacommand{lemma}\isamarkupfalse%
\ of{\isacharunderscore}{\kern0pt}bool{\isacharunderscore}{\kern0pt}square{\isacharcolon}{\kern0pt}\ {\isachardoublequoteopen}{\isacharparenleft}{\kern0pt}of{\isacharunderscore}{\kern0pt}bool\ x{\isacharparenright}{\kern0pt}\isactrlsup {\isadigit{2}}\ {\isacharequal}{\kern0pt}\ {\isacharparenleft}{\kern0pt}{\isacharparenleft}{\kern0pt}of{\isacharunderscore}{\kern0pt}bool\ x{\isacharparenright}{\kern0pt}{\isacharcolon}{\kern0pt}{\isacharcolon}{\kern0pt}real{\isacharparenright}{\kern0pt}{\isachardoublequoteclose}\isanewline
%
\isadelimproof
\ \ %
\endisadelimproof
%
\isatagproof
\isacommand{by}\isamarkupfalse%
\ {\isacharparenleft}{\kern0pt}cases\ x{\isacharcomma}{\kern0pt}\ simp{\isacharcomma}{\kern0pt}\ simp{\isacharparenright}{\kern0pt}%
\endisatagproof
{\isafoldproof}%
%
\isadelimproof
\isanewline
%
\endisadelimproof
\isanewline
\isacommand{theorem}\isamarkupfalse%
\ f{\isadigit{0}}{\isacharunderscore}{\kern0pt}alg{\isacharunderscore}{\kern0pt}correct{\isacharcolon}{\kern0pt}\isanewline
\ \ \isakeyword{assumes}\ {\isachardoublequoteopen}{\isasymepsilon}\ {\isasymin}\ {\isacharbraceleft}{\kern0pt}{\isadigit{0}}{\isacharless}{\kern0pt}{\isachardot}{\kern0pt}{\isachardot}{\kern0pt}{\isacharless}{\kern0pt}{\isadigit{1}}{\isacharbraceright}{\kern0pt}{\isachardoublequoteclose}\isanewline
\ \ \isakeyword{assumes}\ {\isachardoublequoteopen}{\isasymdelta}\ {\isasymin}\ {\isacharbraceleft}{\kern0pt}{\isadigit{0}}{\isacharless}{\kern0pt}{\isachardot}{\kern0pt}{\isachardot}{\kern0pt}{\isacharless}{\kern0pt}{\isadigit{1}}{\isacharbraceright}{\kern0pt}{\isachardoublequoteclose}\isanewline
\ \ \isakeyword{assumes}\ {\isachardoublequoteopen}set\ as\ {\isasymsubseteq}\ {\isacharbraceleft}{\kern0pt}{\isadigit{0}}{\isachardot}{\kern0pt}{\isachardot}{\kern0pt}{\isacharless}{\kern0pt}n{\isacharbraceright}{\kern0pt}{\isachardoublequoteclose}\isanewline
\ \ \isakeyword{defines}\ {\isachardoublequoteopen}M\ {\isasymequiv}\ fold\ {\isacharparenleft}{\kern0pt}{\isasymlambda}a\ state{\isachardot}{\kern0pt}\ state\ {\isasymbind}\ f{\isadigit{0}}{\isacharunderscore}{\kern0pt}update\ a{\isacharparenright}{\kern0pt}\ as\ {\isacharparenleft}{\kern0pt}f{\isadigit{0}}{\isacharunderscore}{\kern0pt}init\ {\isasymdelta}\ {\isasymepsilon}\ n{\isacharparenright}{\kern0pt}\ {\isasymbind}\ f{\isadigit{0}}{\isacharunderscore}{\kern0pt}result{\isachardoublequoteclose}\isanewline
\ \ \isakeyword{shows}\ {\isachardoublequoteopen}{\isasymP}{\isacharparenleft}{\kern0pt}{\isasymomega}\ in\ measure{\isacharunderscore}{\kern0pt}pmf\ M{\isachardot}{\kern0pt}\ {\isasymbar}{\isasymomega}\ {\isacharminus}{\kern0pt}\ F\ {\isadigit{0}}\ as{\isasymbar}\ {\isasymle}\ {\isasymdelta}\ {\isacharasterisk}{\kern0pt}\ F\ {\isadigit{0}}\ as{\isacharparenright}{\kern0pt}\ {\isasymge}\ {\isadigit{1}}\ {\isacharminus}{\kern0pt}\ of{\isacharunderscore}{\kern0pt}rat\ {\isasymepsilon}{\isachardoublequoteclose}\isanewline
%
\isadelimproof
%
\endisadelimproof
%
\isatagproof
\isacommand{proof}\isamarkupfalse%
\ {\isacharminus}{\kern0pt}\isanewline
\ \ \isacommand{define}\isamarkupfalse%
\ s\ \isakeyword{where}\ {\isachardoublequoteopen}s\ {\isacharequal}{\kern0pt}\ nat\ {\isasymlceil}{\isacharminus}{\kern0pt}{\isacharparenleft}{\kern0pt}{\isadigit{1}}{\isadigit{8}}{\isacharasterisk}{\kern0pt}\ ln\ {\isacharparenleft}{\kern0pt}real{\isacharunderscore}{\kern0pt}of{\isacharunderscore}{\kern0pt}rat\ {\isasymepsilon}{\isacharparenright}{\kern0pt}{\isacharparenright}{\kern0pt}{\isasymrceil}{\isachardoublequoteclose}\isanewline
\ \ \isacommand{define}\isamarkupfalse%
\ t\ \isakeyword{where}\ {\isachardoublequoteopen}t\ {\isacharequal}{\kern0pt}\ nat\ {\isasymlceil}{\isadigit{8}}{\isadigit{0}}\ {\isacharslash}{\kern0pt}\ {\isacharparenleft}{\kern0pt}real{\isacharunderscore}{\kern0pt}of{\isacharunderscore}{\kern0pt}rat\ {\isasymdelta}{\isacharparenright}{\kern0pt}\isactrlsup {\isadigit{2}}{\isasymrceil}{\isachardoublequoteclose}\isanewline
\ \ \isacommand{define}\isamarkupfalse%
\ p\ \isakeyword{where}\ {\isachardoublequoteopen}p\ {\isacharequal}{\kern0pt}\ find{\isacharunderscore}{\kern0pt}prime{\isacharunderscore}{\kern0pt}above\ {\isacharparenleft}{\kern0pt}max\ n\ {\isadigit{1}}{\isadigit{9}}{\isacharparenright}{\kern0pt}{\isachardoublequoteclose}\isanewline
\ \ \isacommand{define}\isamarkupfalse%
\ r\ \isakeyword{where}\ {\isachardoublequoteopen}r\ {\isacharequal}{\kern0pt}\ nat\ {\isacharparenleft}{\kern0pt}{\isadigit{4}}\ {\isacharasterisk}{\kern0pt}\ {\isasymlceil}log\ {\isadigit{2}}\ {\isacharparenleft}{\kern0pt}{\isadigit{1}}\ {\isacharslash}{\kern0pt}\ real{\isacharunderscore}{\kern0pt}of{\isacharunderscore}{\kern0pt}rat\ {\isasymdelta}{\isacharparenright}{\kern0pt}{\isasymrceil}\ {\isacharplus}{\kern0pt}\ {\isadigit{2}}{\isadigit{4}}{\isacharparenright}{\kern0pt}{\isachardoublequoteclose}\isanewline
\ \ \isacommand{define}\isamarkupfalse%
\ g\ \isakeyword{where}\ {\isachardoublequoteopen}g\ {\isacharequal}{\kern0pt}\ {\isacharparenleft}{\kern0pt}{\isasymlambda}S{\isachardot}{\kern0pt}\ if\ card\ S\ {\isacharless}{\kern0pt}\ t\ then\ rat{\isacharunderscore}{\kern0pt}of{\isacharunderscore}{\kern0pt}nat\ {\isacharparenleft}{\kern0pt}card\ S{\isacharparenright}{\kern0pt}\ else\ of{\isacharunderscore}{\kern0pt}nat\ t\ {\isacharasterisk}{\kern0pt}\ rat{\isacharunderscore}{\kern0pt}of{\isacharunderscore}{\kern0pt}nat\ p\ {\isacharslash}{\kern0pt}\ rat{\isacharunderscore}{\kern0pt}of{\isacharunderscore}{\kern0pt}float\ {\isacharparenleft}{\kern0pt}Max\ S{\isacharparenright}{\kern0pt}{\isacharparenright}{\kern0pt}{\isachardoublequoteclose}\isanewline
\ \ \isacommand{define}\isamarkupfalse%
\ g{\isacharprime}{\kern0pt}\ \isakeyword{where}\ {\isachardoublequoteopen}g{\isacharprime}{\kern0pt}\ {\isacharequal}{\kern0pt}\ {\isacharparenleft}{\kern0pt}{\isasymlambda}S{\isachardot}{\kern0pt}\ if\ card\ S\ {\isacharless}{\kern0pt}\ t\ then\ real\ {\isacharparenleft}{\kern0pt}card\ S{\isacharparenright}{\kern0pt}\ else\ real\ t\ {\isacharasterisk}{\kern0pt}\ real\ p\ {\isacharslash}{\kern0pt}\ Max\ S{\isacharparenright}{\kern0pt}{\isachardoublequoteclose}\isanewline
\ \ \isacommand{define}\isamarkupfalse%
\ h\ \isakeyword{where}\ {\isachardoublequoteopen}h\ {\isacharequal}{\kern0pt}\ {\isacharparenleft}{\kern0pt}{\isasymlambda}{\isasymomega}{\isachardot}{\kern0pt}\ least\ t\ {\isacharparenleft}{\kern0pt}{\isacharparenleft}{\kern0pt}{\isasymlambda}x{\isachardot}{\kern0pt}\ truncate{\isacharunderscore}{\kern0pt}down\ r\ {\isacharparenleft}{\kern0pt}hash\ p\ x\ {\isasymomega}{\isacharparenright}{\kern0pt}{\isacharparenright}{\kern0pt}\ {\isacharbackquote}{\kern0pt}\ set\ as{\isacharparenright}{\kern0pt}{\isacharparenright}{\kern0pt}{\isachardoublequoteclose}\isanewline
\ \ \isacommand{define}\isamarkupfalse%
\ {\isasymOmega}\isactrlsub {\isadigit{0}}\ \isakeyword{where}\ {\isachardoublequoteopen}{\isasymOmega}\isactrlsub {\isadigit{0}}\ {\isacharequal}{\kern0pt}\ prod{\isacharunderscore}{\kern0pt}pmf\ {\isacharbraceleft}{\kern0pt}{\isadigit{0}}{\isachardot}{\kern0pt}{\isachardot}{\kern0pt}{\isacharless}{\kern0pt}s{\isacharbraceright}{\kern0pt}\ {\isacharparenleft}{\kern0pt}{\isasymlambda}{\isacharunderscore}{\kern0pt}{\isachardot}{\kern0pt}\ pmf{\isacharunderscore}{\kern0pt}of{\isacharunderscore}{\kern0pt}set\ {\isacharparenleft}{\kern0pt}bounded{\isacharunderscore}{\kern0pt}degree{\isacharunderscore}{\kern0pt}polynomials\ {\isacharparenleft}{\kern0pt}ZFact\ {\isacharparenleft}{\kern0pt}int\ p{\isacharparenright}{\kern0pt}{\isacharparenright}{\kern0pt}\ {\isadigit{2}}{\isacharparenright}{\kern0pt}{\isacharparenright}{\kern0pt}{\isachardoublequoteclose}\isanewline
\ \ \isacommand{define}\isamarkupfalse%
\ {\isasymOmega}\isactrlsub {\isadigit{1}}\ \isakeyword{where}\ {\isachardoublequoteopen}{\isasymOmega}\isactrlsub {\isadigit{1}}\ {\isacharequal}{\kern0pt}\ pmf{\isacharunderscore}{\kern0pt}of{\isacharunderscore}{\kern0pt}set\ {\isacharparenleft}{\kern0pt}bounded{\isacharunderscore}{\kern0pt}degree{\isacharunderscore}{\kern0pt}polynomials\ {\isacharparenleft}{\kern0pt}ZFact\ {\isacharparenleft}{\kern0pt}int\ p{\isacharparenright}{\kern0pt}{\isacharparenright}{\kern0pt}\ {\isadigit{2}}{\isacharparenright}{\kern0pt}{\isachardoublequoteclose}\isanewline
\ \ \isacommand{define}\isamarkupfalse%
\ m\ \isakeyword{where}\ {\isachardoublequoteopen}m\ {\isacharequal}{\kern0pt}\ card\ {\isacharparenleft}{\kern0pt}set\ as{\isacharparenright}{\kern0pt}{\isachardoublequoteclose}\isanewline
\isanewline
\ \ \isacommand{define}\isamarkupfalse%
\ f\ \isakeyword{where}\ {\isachardoublequoteopen}f\ {\isacharequal}{\kern0pt}\ {\isacharparenleft}{\kern0pt}{\isasymlambda}r\ {\isasymomega}{\isachardot}{\kern0pt}\ card\ {\isacharbraceleft}{\kern0pt}x\ {\isasymin}\ set\ as{\isachardot}{\kern0pt}\ int\ {\isacharparenleft}{\kern0pt}hash\ p\ x\ {\isasymomega}{\isacharparenright}{\kern0pt}\ {\isasymle}\ r{\isacharbraceright}{\kern0pt}{\isacharparenright}{\kern0pt}{\isachardoublequoteclose}\isanewline
\ \ \isacommand{define}\isamarkupfalse%
\ {\isasymdelta}{\isacharprime}{\kern0pt}\ \isakeyword{where}\ {\isachardoublequoteopen}{\isasymdelta}{\isacharprime}{\kern0pt}\ {\isacharequal}{\kern0pt}\ {\isadigit{3}}{\isacharasterisk}{\kern0pt}\ real{\isacharunderscore}{\kern0pt}of{\isacharunderscore}{\kern0pt}rat\ {\isasymdelta}\ {\isacharslash}{\kern0pt}{\isadigit{4}}{\isachardoublequoteclose}\isanewline
\ \ \isacommand{define}\isamarkupfalse%
\ a\ \isakeyword{where}\ {\isachardoublequoteopen}a\ {\isacharequal}{\kern0pt}\ {\isasymlfloor}real\ t\ {\isacharasterisk}{\kern0pt}\ p\ {\isacharslash}{\kern0pt}\ {\isacharparenleft}{\kern0pt}m\ {\isacharasterisk}{\kern0pt}\ {\isacharparenleft}{\kern0pt}{\isadigit{1}}{\isacharplus}{\kern0pt}{\isasymdelta}{\isacharprime}{\kern0pt}{\isacharparenright}{\kern0pt}{\isacharparenright}{\kern0pt}{\isasymrfloor}{\isachardoublequoteclose}\isanewline
\ \ \isacommand{define}\isamarkupfalse%
\ b\ \isakeyword{where}\ {\isachardoublequoteopen}b\ {\isacharequal}{\kern0pt}\ {\isasymlceil}real\ t\ {\isacharasterisk}{\kern0pt}\ p\ {\isacharslash}{\kern0pt}\ {\isacharparenleft}{\kern0pt}m\ {\isacharasterisk}{\kern0pt}\ {\isacharparenleft}{\kern0pt}{\isadigit{1}}{\isacharminus}{\kern0pt}{\isasymdelta}{\isacharprime}{\kern0pt}{\isacharparenright}{\kern0pt}{\isacharparenright}{\kern0pt}{\isacharminus}{\kern0pt}{\isadigit{1}}{\isasymrceil}{\isachardoublequoteclose}\isanewline
\isanewline
\ \ \isacommand{define}\isamarkupfalse%
\ has{\isacharunderscore}{\kern0pt}no{\isacharunderscore}{\kern0pt}collision\ \isakeyword{where}\ {\isachardoublequoteopen}has{\isacharunderscore}{\kern0pt}no{\isacharunderscore}{\kern0pt}collision\ {\isacharequal}{\kern0pt}\ {\isacharparenleft}{\kern0pt}{\isasymlambda}{\isasymomega}{\isachardot}{\kern0pt}\ {\isasymforall}x{\isasymin}\ set\ as{\isachardot}{\kern0pt}\ {\isasymforall}y\ {\isasymin}\ set\ as{\isachardot}{\kern0pt}\isanewline
\ \ \ \ {\isacharparenleft}{\kern0pt}truncate{\isacharunderscore}{\kern0pt}down\ r\ {\isacharparenleft}{\kern0pt}hash\ p\ x\ {\isasymomega}{\isacharparenright}{\kern0pt}\ {\isacharequal}{\kern0pt}\ truncate{\isacharunderscore}{\kern0pt}down\ r\ {\isacharparenleft}{\kern0pt}hash\ p\ y\ {\isasymomega}{\isacharparenright}{\kern0pt}\ {\isasymlongrightarrow}\ x\ {\isacharequal}{\kern0pt}\ y{\isacharparenright}{\kern0pt}\ {\isasymor}\ \isanewline
\ \ \ \ truncate{\isacharunderscore}{\kern0pt}down\ r\ {\isacharparenleft}{\kern0pt}hash\ p\ x\ {\isasymomega}{\isacharparenright}{\kern0pt}\ {\isachargreater}{\kern0pt}\ b{\isacharparenright}{\kern0pt}{\isachardoublequoteclose}\isanewline
\isanewline
\ \ \isacommand{have}\isamarkupfalse%
\ s{\isacharunderscore}{\kern0pt}ge{\isacharunderscore}{\kern0pt}{\isadigit{0}}{\isacharcolon}{\kern0pt}\ {\isachardoublequoteopen}s\ {\isachargreater}{\kern0pt}\ {\isadigit{0}}{\isachardoublequoteclose}\ \isanewline
\ \ \ \ \isacommand{using}\isamarkupfalse%
\ assms{\isacharparenleft}{\kern0pt}{\isadigit{1}}{\isacharparenright}{\kern0pt}\ \isacommand{by}\isamarkupfalse%
\ {\isacharparenleft}{\kern0pt}simp\ add{\isacharcolon}{\kern0pt}s{\isacharunderscore}{\kern0pt}def{\isacharparenright}{\kern0pt}\isanewline
\isanewline
\ \ \isacommand{have}\isamarkupfalse%
\ t{\isacharunderscore}{\kern0pt}ge{\isacharunderscore}{\kern0pt}{\isadigit{0}}{\isacharcolon}{\kern0pt}\ {\isachardoublequoteopen}t\ {\isachargreater}{\kern0pt}\ {\isadigit{0}}{\isachardoublequoteclose}\isanewline
\ \ \ \ \isacommand{using}\isamarkupfalse%
\ assms\ \isacommand{by}\isamarkupfalse%
\ {\isacharparenleft}{\kern0pt}simp\ add{\isacharcolon}{\kern0pt}t{\isacharunderscore}{\kern0pt}def{\isacharparenright}{\kern0pt}\isanewline
\isanewline
\ \ \isacommand{have}\isamarkupfalse%
\ {\isasymdelta}{\isacharunderscore}{\kern0pt}ge{\isacharunderscore}{\kern0pt}{\isadigit{0}}{\isacharcolon}{\kern0pt}\ {\isachardoublequoteopen}{\isasymdelta}\ {\isachargreater}{\kern0pt}\ {\isadigit{0}}{\isachardoublequoteclose}\ \isacommand{using}\isamarkupfalse%
\ assms\ \isacommand{by}\isamarkupfalse%
\ simp\isanewline
\ \ \isacommand{have}\isamarkupfalse%
\ {\isasymdelta}{\isacharunderscore}{\kern0pt}le{\isacharunderscore}{\kern0pt}{\isadigit{1}}{\isacharcolon}{\kern0pt}\ {\isachardoublequoteopen}{\isasymdelta}\ {\isacharless}{\kern0pt}\ {\isadigit{1}}{\isachardoublequoteclose}\ \isacommand{using}\isamarkupfalse%
\ assms\ \isacommand{by}\isamarkupfalse%
\ simp\isanewline
\isanewline
\ \ \isacommand{have}\isamarkupfalse%
\ r{\isacharunderscore}{\kern0pt}bound{\isacharcolon}{\kern0pt}\ {\isachardoublequoteopen}{\isadigit{4}}\ {\isacharasterisk}{\kern0pt}\ log\ {\isadigit{2}}\ {\isacharparenleft}{\kern0pt}{\isadigit{1}}\ {\isacharslash}{\kern0pt}\ real{\isacharunderscore}{\kern0pt}of{\isacharunderscore}{\kern0pt}rat\ {\isasymdelta}{\isacharparenright}{\kern0pt}\ {\isacharplus}{\kern0pt}\ {\isadigit{2}}{\isadigit{4}}\ {\isasymle}\ r{\isachardoublequoteclose}\isanewline
\ \ \ \ \isacommand{apply}\isamarkupfalse%
\ {\isacharparenleft}{\kern0pt}simp\ add{\isacharcolon}{\kern0pt}r{\isacharunderscore}{\kern0pt}def{\isacharparenright}{\kern0pt}\ \ \ \ \ \ \ \ \ \ \ \ \ \ \ \ \ \ \ \ \ \ \ \ \ \ \ \ \ \ \isanewline
\ \ \ \ \isacommand{apply}\isamarkupfalse%
\ {\isacharparenleft}{\kern0pt}subst\ of{\isacharunderscore}{\kern0pt}nat{\isacharunderscore}{\kern0pt}nat{\isacharparenright}{\kern0pt}\isanewline
\ \ \ \ \ \isacommand{apply}\isamarkupfalse%
\ {\isacharparenleft}{\kern0pt}rule\ add{\isacharunderscore}{\kern0pt}nonneg{\isacharunderscore}{\kern0pt}nonneg{\isacharparenright}{\kern0pt}\isanewline
\ \ \ \ \ \ \isacommand{apply}\isamarkupfalse%
\ {\isacharparenleft}{\kern0pt}rule\ mult{\isacharunderscore}{\kern0pt}nonneg{\isacharunderscore}{\kern0pt}nonneg{\isacharcomma}{\kern0pt}\ simp{\isacharparenright}{\kern0pt}\isanewline
\ \ \ \ \ \ \isacommand{apply}\isamarkupfalse%
\ {\isacharparenleft}{\kern0pt}subst\ zero{\isacharunderscore}{\kern0pt}le{\isacharunderscore}{\kern0pt}ceiling{\isacharcomma}{\kern0pt}\ subst\ log{\isacharunderscore}{\kern0pt}divide{\isacharcomma}{\kern0pt}\ simp{\isacharcomma}{\kern0pt}\ simp{\isacharcomma}{\kern0pt}\ simp{\isacharcomma}{\kern0pt}\ simp\ add{\isacharcolon}{\kern0pt}{\isasymdelta}{\isacharunderscore}{\kern0pt}ge{\isacharunderscore}{\kern0pt}{\isadigit{0}}{\isacharcomma}{\kern0pt}\ simp{\isacharparenright}{\kern0pt}\isanewline
\ \ \ \ \ \ \isacommand{apply}\isamarkupfalse%
\ {\isacharparenleft}{\kern0pt}subst\ log{\isacharunderscore}{\kern0pt}less{\isacharunderscore}{\kern0pt}one{\isacharunderscore}{\kern0pt}cancel{\isacharunderscore}{\kern0pt}iff{\isacharcomma}{\kern0pt}\ simp{\isacharcomma}{\kern0pt}\ simp\ add{\isacharcolon}{\kern0pt}{\isasymdelta}{\isacharunderscore}{\kern0pt}ge{\isacharunderscore}{\kern0pt}{\isadigit{0}}{\isacharparenright}{\kern0pt}\isanewline
\ \ \ \ \isacommand{by}\isamarkupfalse%
\ {\isacharparenleft}{\kern0pt}rule\ order{\isacharunderscore}{\kern0pt}less{\isacharunderscore}{\kern0pt}le{\isacharunderscore}{\kern0pt}trans{\isacharbrackleft}{\kern0pt}\isakeyword{where}\ y{\isacharequal}{\kern0pt}{\isachardoublequoteopen}{\isadigit{1}}{\isachardoublequoteclose}{\isacharbrackright}{\kern0pt}{\isacharcomma}{\kern0pt}\ simp\ add{\isacharcolon}{\kern0pt}{\isasymdelta}{\isacharunderscore}{\kern0pt}le{\isacharunderscore}{\kern0pt}{\isadigit{1}}{\isacharcomma}{\kern0pt}\ simp{\isacharplus}{\kern0pt}{\isacharparenright}{\kern0pt}\isanewline
\isanewline
\ \ \isacommand{have}\isamarkupfalse%
\ {\isachardoublequoteopen}{\isadigit{1}}\ {\isasymle}\ {\isadigit{0}}\ {\isacharplus}{\kern0pt}\ {\isacharparenleft}{\kern0pt}{\isadigit{2}}{\isadigit{4}}{\isacharcolon}{\kern0pt}{\isacharcolon}{\kern0pt}real{\isacharparenright}{\kern0pt}{\isachardoublequoteclose}\ \isacommand{by}\isamarkupfalse%
\ simp\isanewline
\ \ \isacommand{also}\isamarkupfalse%
\ \isacommand{have}\isamarkupfalse%
\ {\isachardoublequoteopen}{\isachardot}{\kern0pt}{\isachardot}{\kern0pt}{\isachardot}{\kern0pt}\ {\isasymle}\ {\isadigit{4}}\ {\isacharasterisk}{\kern0pt}\ log\ {\isadigit{2}}\ {\isacharparenleft}{\kern0pt}{\isadigit{1}}\ {\isacharslash}{\kern0pt}\ real{\isacharunderscore}{\kern0pt}of{\isacharunderscore}{\kern0pt}rat\ {\isasymdelta}{\isacharparenright}{\kern0pt}\ {\isacharplus}{\kern0pt}\ {\isadigit{2}}{\isadigit{4}}{\isachardoublequoteclose}\isanewline
\ \ \ \ \isacommand{apply}\isamarkupfalse%
\ {\isacharparenleft}{\kern0pt}rule\ add{\isacharunderscore}{\kern0pt}mono{\isacharcomma}{\kern0pt}\ simp{\isacharparenright}{\kern0pt}\isanewline
\ \ \ \ \isacommand{apply}\isamarkupfalse%
\ {\isacharparenleft}{\kern0pt}subst\ zero{\isacharunderscore}{\kern0pt}le{\isacharunderscore}{\kern0pt}log{\isacharunderscore}{\kern0pt}cancel{\isacharunderscore}{\kern0pt}iff{\isacharparenright}{\kern0pt}\isanewline
\ \ \ \ \isacommand{using}\isamarkupfalse%
\ assms\ \isacommand{by}\isamarkupfalse%
\ simp{\isacharplus}{\kern0pt}\isanewline
\ \ \isacommand{also}\isamarkupfalse%
\ \isacommand{have}\isamarkupfalse%
\ {\isachardoublequoteopen}{\isachardot}{\kern0pt}{\isachardot}{\kern0pt}{\isachardot}{\kern0pt}\ {\isasymle}\ r{\isachardoublequoteclose}\ \isacommand{using}\isamarkupfalse%
\ r{\isacharunderscore}{\kern0pt}bound\ \isacommand{by}\isamarkupfalse%
\ simp\isanewline
\ \ \isacommand{finally}\isamarkupfalse%
\ \isacommand{have}\isamarkupfalse%
\ r{\isacharunderscore}{\kern0pt}ge{\isacharunderscore}{\kern0pt}{\isadigit{0}}{\isacharcolon}{\kern0pt}\ {\isachardoublequoteopen}{\isadigit{1}}\ {\isasymle}\ r{\isachardoublequoteclose}\ \isacommand{by}\isamarkupfalse%
\ simp\isanewline
\isanewline
\ \ \isacommand{have}\isamarkupfalse%
\ {\isachardoublequoteopen}{\isadigit{2}}\ powr\ {\isacharparenleft}{\kern0pt}{\isacharminus}{\kern0pt}real\ r{\isacharparenright}{\kern0pt}\ {\isasymle}\ {\isadigit{2}}\ powr\ {\isacharparenleft}{\kern0pt}{\isacharminus}{\kern0pt}{\isacharparenleft}{\kern0pt}{\isadigit{4}}\ {\isacharasterisk}{\kern0pt}\ log\ {\isadigit{2}}\ {\isacharparenleft}{\kern0pt}{\isadigit{1}}\ {\isacharslash}{\kern0pt}\ real{\isacharunderscore}{\kern0pt}of{\isacharunderscore}{\kern0pt}rat\ {\isasymdelta}{\isacharparenright}{\kern0pt}\ {\isacharplus}{\kern0pt}\ {\isadigit{2}}{\isadigit{4}}{\isacharparenright}{\kern0pt}{\isacharparenright}{\kern0pt}{\isachardoublequoteclose}\isanewline
\ \ \ \ \isacommand{apply}\isamarkupfalse%
\ {\isacharparenleft}{\kern0pt}rule\ powr{\isacharunderscore}{\kern0pt}mono{\isacharparenright}{\kern0pt}\ \isacommand{using}\isamarkupfalse%
\ r{\isacharunderscore}{\kern0pt}bound\ \isacommand{apply}\isamarkupfalse%
\ linarith\ \isacommand{by}\isamarkupfalse%
\ simp\isanewline
\ \ \isacommand{also}\isamarkupfalse%
\ \isacommand{have}\isamarkupfalse%
\ {\isachardoublequoteopen}{\isachardot}{\kern0pt}{\isachardot}{\kern0pt}{\isachardot}{\kern0pt}\ {\isacharequal}{\kern0pt}\ {\isadigit{2}}\ powr\ {\isacharparenleft}{\kern0pt}{\isacharminus}{\kern0pt}{\isadigit{4}}\ {\isacharasterisk}{\kern0pt}\ log\ {\isadigit{2}}\ {\isacharparenleft}{\kern0pt}{\isadigit{1}}\ {\isacharslash}{\kern0pt}real{\isacharunderscore}{\kern0pt}of{\isacharunderscore}{\kern0pt}rat\ {\isasymdelta}{\isacharparenright}{\kern0pt}\ {\isacharminus}{\kern0pt}{\isadigit{2}}{\isadigit{4}}{\isacharparenright}{\kern0pt}{\isachardoublequoteclose}\isanewline
\ \ \ \ \isacommand{by}\isamarkupfalse%
\ {\isacharparenleft}{\kern0pt}rule\ arg{\isacharunderscore}{\kern0pt}cong{\isadigit{2}}{\isacharbrackleft}{\kern0pt}\isakeyword{where}\ f{\isacharequal}{\kern0pt}{\isachardoublequoteopen}{\isacharparenleft}{\kern0pt}powr{\isacharparenright}{\kern0pt}{\isachardoublequoteclose}{\isacharbrackright}{\kern0pt}{\isacharcomma}{\kern0pt}\ simp{\isacharcomma}{\kern0pt}\ simp\ add{\isacharcolon}{\kern0pt}algebra{\isacharunderscore}{\kern0pt}simps{\isacharparenright}{\kern0pt}\isanewline
\ \ \isacommand{also}\isamarkupfalse%
\ \isacommand{have}\isamarkupfalse%
\ {\isachardoublequoteopen}{\isachardot}{\kern0pt}{\isachardot}{\kern0pt}{\isachardot}{\kern0pt}\ {\isasymle}\ {\isadigit{2}}\ powr\ {\isacharparenleft}{\kern0pt}\ {\isacharminus}{\kern0pt}{\isadigit{1}}\ {\isacharasterisk}{\kern0pt}\ log\ {\isadigit{2}}\ {\isacharparenleft}{\kern0pt}{\isadigit{1}}\ {\isacharslash}{\kern0pt}real{\isacharunderscore}{\kern0pt}of{\isacharunderscore}{\kern0pt}rat\ {\isasymdelta}{\isacharparenright}{\kern0pt}\ {\isacharminus}{\kern0pt}{\isadigit{4}}{\isacharparenright}{\kern0pt}{\isachardoublequoteclose}\isanewline
\ \ \ \ \isacommand{apply}\isamarkupfalse%
\ {\isacharparenleft}{\kern0pt}rule\ powr{\isacharunderscore}{\kern0pt}mono{\isacharparenright}{\kern0pt}\isanewline
\ \ \ \ \ \isacommand{apply}\isamarkupfalse%
\ {\isacharparenleft}{\kern0pt}rule\ diff{\isacharunderscore}{\kern0pt}mono{\isacharparenright}{\kern0pt}\isanewline
\ \ \ \ \isacommand{using}\isamarkupfalse%
\ assms{\isacharparenleft}{\kern0pt}{\isadigit{2}}{\isacharparenright}{\kern0pt}\ \isacommand{by}\isamarkupfalse%
\ simp{\isacharplus}{\kern0pt}\isanewline
\ \ \isacommand{also}\isamarkupfalse%
\ \isacommand{have}\isamarkupfalse%
\ {\isachardoublequoteopen}{\isachardot}{\kern0pt}{\isachardot}{\kern0pt}{\isachardot}{\kern0pt}\ {\isacharequal}{\kern0pt}\ real{\isacharunderscore}{\kern0pt}of{\isacharunderscore}{\kern0pt}rat\ {\isasymdelta}\ {\isacharslash}{\kern0pt}\ {\isadigit{1}}{\isadigit{6}}{\isachardoublequoteclose}\isanewline
\ \ \ \ \isacommand{apply}\isamarkupfalse%
\ {\isacharparenleft}{\kern0pt}subst\ powr{\isacharunderscore}{\kern0pt}diff{\isacharparenright}{\kern0pt}\isanewline
\ \ \ \ \isacommand{apply}\isamarkupfalse%
\ {\isacharparenleft}{\kern0pt}subst\ log{\isacharunderscore}{\kern0pt}divide{\isacharcomma}{\kern0pt}\ simp{\isacharcomma}{\kern0pt}\ simp{\isacharcomma}{\kern0pt}\ simp{\isacharcomma}{\kern0pt}\ simp\ add{\isacharcolon}{\kern0pt}{\isasymdelta}{\isacharunderscore}{\kern0pt}ge{\isacharunderscore}{\kern0pt}{\isadigit{0}}{\isacharcomma}{\kern0pt}\ simp{\isacharparenright}{\kern0pt}\isanewline
\ \ \ \ \isacommand{by}\isamarkupfalse%
\ {\isacharparenleft}{\kern0pt}subst\ powr{\isacharunderscore}{\kern0pt}log{\isacharunderscore}{\kern0pt}cancel{\isacharcomma}{\kern0pt}\ simp{\isacharcomma}{\kern0pt}\ simp{\isacharcomma}{\kern0pt}\ simp\ add{\isacharcolon}{\kern0pt}{\isasymdelta}{\isacharunderscore}{\kern0pt}ge{\isacharunderscore}{\kern0pt}{\isadigit{0}}{\isacharcomma}{\kern0pt}\ simp{\isacharparenright}{\kern0pt}\isanewline
\ \ \isacommand{also}\isamarkupfalse%
\ \isacommand{have}\isamarkupfalse%
\ {\isachardoublequoteopen}{\isachardot}{\kern0pt}{\isachardot}{\kern0pt}{\isachardot}{\kern0pt}\ {\isacharless}{\kern0pt}\ real{\isacharunderscore}{\kern0pt}of{\isacharunderscore}{\kern0pt}rat\ {\isasymdelta}\ {\isacharslash}{\kern0pt}\ {\isadigit{8}}{\isachardoublequoteclose}\isanewline
\ \ \ \ \isacommand{by}\isamarkupfalse%
\ {\isacharparenleft}{\kern0pt}subst\ pos{\isacharunderscore}{\kern0pt}divide{\isacharunderscore}{\kern0pt}less{\isacharunderscore}{\kern0pt}eq{\isacharcomma}{\kern0pt}\ simp{\isacharcomma}{\kern0pt}\ simp\ add{\isacharcolon}{\kern0pt}{\isasymdelta}{\isacharunderscore}{\kern0pt}ge{\isacharunderscore}{\kern0pt}{\isadigit{0}}{\isacharparenright}{\kern0pt}\isanewline
\ \ \isacommand{finally}\isamarkupfalse%
\ \isacommand{have}\isamarkupfalse%
\ r{\isacharunderscore}{\kern0pt}le{\isacharunderscore}{\kern0pt}{\isasymdelta}{\isacharcolon}{\kern0pt}\ {\isachardoublequoteopen}{\isadigit{2}}\ powr\ {\isacharparenleft}{\kern0pt}{\isacharminus}{\kern0pt}real\ r{\isacharparenright}{\kern0pt}\ {\isacharless}{\kern0pt}\ {\isacharparenleft}{\kern0pt}real{\isacharunderscore}{\kern0pt}of{\isacharunderscore}{\kern0pt}rat\ {\isasymdelta}{\isacharparenright}{\kern0pt}{\isacharslash}{\kern0pt}\ {\isadigit{8}}{\isachardoublequoteclose}\isanewline
\ \ \ \ \isacommand{by}\isamarkupfalse%
\ simp\isanewline
\isanewline
\ \ \isacommand{have}\isamarkupfalse%
\ r{\isacharunderscore}{\kern0pt}le{\isacharunderscore}{\kern0pt}t{\isadigit{2}}{\isacharcolon}{\kern0pt}\ {\isachardoublequoteopen}{\isadigit{1}}{\isadigit{8}}\ {\isacharasterisk}{\kern0pt}\ {\isadigit{9}}{\isadigit{6}}\ {\isacharasterisk}{\kern0pt}\ {\isacharparenleft}{\kern0pt}real\ t{\isacharparenright}{\kern0pt}\isactrlsup {\isadigit{2}}\ {\isacharasterisk}{\kern0pt}\ {\isadigit{2}}\ powr\ {\isacharparenleft}{\kern0pt}{\isacharminus}{\kern0pt}real\ r{\isacharparenright}{\kern0pt}\ {\isasymle}\ \isanewline
\ \ \ \ {\isadigit{1}}{\isadigit{8}}\ {\isacharasterisk}{\kern0pt}\ {\isadigit{9}}{\isadigit{6}}\ {\isacharasterisk}{\kern0pt}\ {\isacharparenleft}{\kern0pt}{\isadigit{8}}{\isadigit{0}}\ {\isacharslash}{\kern0pt}\ {\isacharparenleft}{\kern0pt}real{\isacharunderscore}{\kern0pt}of{\isacharunderscore}{\kern0pt}rat\ {\isasymdelta}{\isacharparenright}{\kern0pt}\isactrlsup {\isadigit{2}}{\isacharplus}{\kern0pt}{\isadigit{1}}{\isacharparenright}{\kern0pt}\isactrlsup {\isadigit{2}}\ {\isacharasterisk}{\kern0pt}\ {\isadigit{2}}\ powr\ {\isacharparenleft}{\kern0pt}{\isacharminus}{\kern0pt}{\isadigit{4}}\ {\isacharasterisk}{\kern0pt}\ log\ {\isadigit{2}}\ {\isacharparenleft}{\kern0pt}{\isadigit{1}}\ {\isacharslash}{\kern0pt}\ real{\isacharunderscore}{\kern0pt}of{\isacharunderscore}{\kern0pt}rat\ {\isasymdelta}{\isacharparenright}{\kern0pt}\ {\isacharminus}{\kern0pt}\ {\isadigit{2}}{\isadigit{4}}{\isacharparenright}{\kern0pt}{\isachardoublequoteclose}\isanewline
\ \ \ \ \isacommand{apply}\isamarkupfalse%
\ {\isacharparenleft}{\kern0pt}rule\ mult{\isacharunderscore}{\kern0pt}mono{\isacharparenright}{\kern0pt}\isanewline
\ \ \ \ \ \ \ \isacommand{apply}\isamarkupfalse%
\ {\isacharparenleft}{\kern0pt}rule\ mult{\isacharunderscore}{\kern0pt}left{\isacharunderscore}{\kern0pt}mono{\isacharparenright}{\kern0pt}\isanewline
\ \ \ \ \ \ \ \ \isacommand{apply}\isamarkupfalse%
\ {\isacharparenleft}{\kern0pt}rule\ power{\isacharunderscore}{\kern0pt}mono{\isacharparenright}{\kern0pt}\isanewline
\ \ \ \ \ \ \ \ \ \isacommand{apply}\isamarkupfalse%
\ {\isacharparenleft}{\kern0pt}simp\ add{\isacharcolon}{\kern0pt}t{\isacharunderscore}{\kern0pt}def{\isacharparenright}{\kern0pt}\ \isacommand{using}\isamarkupfalse%
\ t{\isacharunderscore}{\kern0pt}def\ t{\isacharunderscore}{\kern0pt}ge{\isacharunderscore}{\kern0pt}{\isadigit{0}}\ \isacommand{apply}\isamarkupfalse%
\ linarith\isanewline
\ \ \ \ \ \ \ \ \isacommand{apply}\isamarkupfalse%
\ simp\isanewline
\ \ \ \ \ \ \ \isacommand{apply}\isamarkupfalse%
\ simp\isanewline
\ \ \ \ \ \ \isacommand{apply}\isamarkupfalse%
\ {\isacharparenleft}{\kern0pt}rule\ powr{\isacharunderscore}{\kern0pt}mono{\isacharparenright}{\kern0pt}\ \isacommand{using}\isamarkupfalse%
\ r{\isacharunderscore}{\kern0pt}bound\ \isacommand{apply}\isamarkupfalse%
\ linarith\isanewline
\ \ \ \ \isacommand{by}\isamarkupfalse%
\ simp{\isacharplus}{\kern0pt}\isanewline
\ \ \isacommand{also}\isamarkupfalse%
\ \isacommand{have}\isamarkupfalse%
\ {\isachardoublequoteopen}{\isachardot}{\kern0pt}{\isachardot}{\kern0pt}{\isachardot}{\kern0pt}\ {\isasymle}\ {\isadigit{1}}{\isadigit{8}}\ {\isacharasterisk}{\kern0pt}\ {\isadigit{9}}{\isadigit{6}}\ {\isacharasterisk}{\kern0pt}\ {\isacharparenleft}{\kern0pt}{\isadigit{8}}{\isadigit{0}}\ {\isacharslash}{\kern0pt}\ {\isacharparenleft}{\kern0pt}real{\isacharunderscore}{\kern0pt}of{\isacharunderscore}{\kern0pt}rat\ {\isasymdelta}{\isacharparenright}{\kern0pt}\isactrlsup {\isadigit{2}}\ {\isacharplus}{\kern0pt}\ {\isadigit{1}}\ {\isacharslash}{\kern0pt}\ \ {\isacharparenleft}{\kern0pt}real{\isacharunderscore}{\kern0pt}of{\isacharunderscore}{\kern0pt}rat\ {\isasymdelta}{\isacharparenright}{\kern0pt}\isactrlsup {\isadigit{2}}{\isacharparenright}{\kern0pt}\isactrlsup {\isadigit{2}}\ {\isacharasterisk}{\kern0pt}\ {\isacharparenleft}{\kern0pt}{\isadigit{2}}\ powr\ {\isacharparenleft}{\kern0pt}{\isacharminus}{\kern0pt}{\isadigit{4}}\ {\isacharasterisk}{\kern0pt}\ log\ {\isadigit{2}}\ {\isacharparenleft}{\kern0pt}{\isadigit{1}}\ {\isacharslash}{\kern0pt}\ real{\isacharunderscore}{\kern0pt}of{\isacharunderscore}{\kern0pt}rat\ {\isasymdelta}{\isacharparenright}{\kern0pt}{\isacharparenright}{\kern0pt}\ {\isacharasterisk}{\kern0pt}\ {\isadigit{2}}\ powr\ {\isacharparenleft}{\kern0pt}{\isacharminus}{\kern0pt}{\isadigit{2}}{\isadigit{4}}{\isacharparenright}{\kern0pt}{\isacharparenright}{\kern0pt}{\isachardoublequoteclose}\isanewline
\ \ \ \ \isacommand{apply}\isamarkupfalse%
\ {\isacharparenleft}{\kern0pt}rule\ mult{\isacharunderscore}{\kern0pt}mono{\isacharparenright}{\kern0pt}\isanewline
\ \ \ \ \ \ \ \isacommand{apply}\isamarkupfalse%
\ {\isacharparenleft}{\kern0pt}rule\ mult{\isacharunderscore}{\kern0pt}left{\isacharunderscore}{\kern0pt}mono{\isacharparenright}{\kern0pt}\isanewline
\ \ \ \ \ \ \ \ \isacommand{apply}\isamarkupfalse%
\ {\isacharparenleft}{\kern0pt}rule\ power{\isacharunderscore}{\kern0pt}mono{\isacharparenright}{\kern0pt}\isanewline
\ \ \ \ \ \ \ \ \ \isacommand{apply}\isamarkupfalse%
\ {\isacharparenleft}{\kern0pt}rule\ add{\isacharunderscore}{\kern0pt}mono{\isacharcomma}{\kern0pt}\ simp{\isacharparenright}{\kern0pt}\ \isacommand{using}\isamarkupfalse%
\ assms{\isacharparenleft}{\kern0pt}{\isadigit{2}}{\isacharparenright}{\kern0pt}\ \isacommand{apply}\isamarkupfalse%
\ {\isacharparenleft}{\kern0pt}simp\ add{\isacharcolon}{\kern0pt}\ power{\isacharunderscore}{\kern0pt}le{\isacharunderscore}{\kern0pt}one{\isacharparenright}{\kern0pt}\isanewline
\ \ \ \ \isacommand{by}\isamarkupfalse%
\ {\isacharparenleft}{\kern0pt}simp\ add{\isacharcolon}{\kern0pt}powr{\isacharunderscore}{\kern0pt}diff{\isacharparenright}{\kern0pt}{\isacharplus}{\kern0pt}\isanewline
\ \ \isacommand{also}\isamarkupfalse%
\ \isacommand{have}\isamarkupfalse%
\ {\isachardoublequoteopen}{\isachardot}{\kern0pt}{\isachardot}{\kern0pt}{\isachardot}{\kern0pt}\ {\isacharequal}{\kern0pt}\ {\isadigit{1}}{\isadigit{8}}\ {\isacharasterisk}{\kern0pt}\ {\isadigit{9}}{\isadigit{6}}\ {\isacharasterisk}{\kern0pt}\ {\isacharparenleft}{\kern0pt}{\isadigit{8}}{\isadigit{1}}\isactrlsup {\isadigit{2}}\ {\isacharslash}{\kern0pt}\ {\isacharparenleft}{\kern0pt}real{\isacharunderscore}{\kern0pt}of{\isacharunderscore}{\kern0pt}rat\ {\isasymdelta}{\isacharparenright}{\kern0pt}{\isacharcircum}{\kern0pt}{\isadigit{4}}{\isacharparenright}{\kern0pt}\ {\isacharasterisk}{\kern0pt}\ {\isacharparenleft}{\kern0pt}{\isadigit{2}}\ powr\ {\isacharparenleft}{\kern0pt}log\ {\isadigit{2}}\ {\isacharparenleft}{\kern0pt}{\isacharparenleft}{\kern0pt}real{\isacharunderscore}{\kern0pt}of{\isacharunderscore}{\kern0pt}rat\ {\isasymdelta}{\isacharparenright}{\kern0pt}{\isacharcircum}{\kern0pt}{\isadigit{4}}{\isacharparenright}{\kern0pt}{\isacharparenright}{\kern0pt}\ \ {\isacharasterisk}{\kern0pt}\ {\isadigit{2}}\ powr\ {\isacharparenleft}{\kern0pt}{\isacharminus}{\kern0pt}{\isadigit{2}}{\isadigit{4}}{\isacharparenright}{\kern0pt}{\isacharparenright}{\kern0pt}{\isachardoublequoteclose}\isanewline
\ \ \ \ \isacommand{apply}\isamarkupfalse%
\ {\isacharparenleft}{\kern0pt}rule\ arg{\isacharunderscore}{\kern0pt}cong{\isadigit{2}}{\isacharbrackleft}{\kern0pt}\isakeyword{where}\ f{\isacharequal}{\kern0pt}{\isachardoublequoteopen}{\isacharparenleft}{\kern0pt}{\isacharasterisk}{\kern0pt}{\isacharparenright}{\kern0pt}{\isachardoublequoteclose}{\isacharbrackright}{\kern0pt}{\isacharparenright}{\kern0pt}\isanewline
\ \ \ \ \ \isacommand{apply}\isamarkupfalse%
\ {\isacharparenleft}{\kern0pt}rule\ arg{\isacharunderscore}{\kern0pt}cong{\isadigit{2}}{\isacharbrackleft}{\kern0pt}\isakeyword{where}\ f{\isacharequal}{\kern0pt}{\isachardoublequoteopen}{\isacharparenleft}{\kern0pt}{\isacharasterisk}{\kern0pt}{\isacharparenright}{\kern0pt}{\isachardoublequoteclose}{\isacharbrackright}{\kern0pt}{\isacharcomma}{\kern0pt}\ simp{\isacharparenright}{\kern0pt}\isanewline
\ \ \ \ \isacommand{apply}\isamarkupfalse%
\ {\isacharparenleft}{\kern0pt}simp\ add{\isacharcolon}{\kern0pt}power{\isadigit{2}}{\isacharunderscore}{\kern0pt}eq{\isacharunderscore}{\kern0pt}square\ power{\isadigit{4}}{\isacharunderscore}{\kern0pt}eq{\isacharunderscore}{\kern0pt}xxxx{\isacharparenright}{\kern0pt}\isanewline
\ \ \ \ \isacommand{apply}\isamarkupfalse%
\ {\isacharparenleft}{\kern0pt}rule\ arg{\isacharunderscore}{\kern0pt}cong{\isadigit{2}}{\isacharbrackleft}{\kern0pt}\isakeyword{where}\ f{\isacharequal}{\kern0pt}{\isachardoublequoteopen}{\isacharparenleft}{\kern0pt}{\isacharasterisk}{\kern0pt}{\isacharparenright}{\kern0pt}{\isachardoublequoteclose}{\isacharbrackright}{\kern0pt}{\isacharparenright}{\kern0pt}\isanewline
\ \ \ \ \ \isacommand{apply}\isamarkupfalse%
\ {\isacharparenleft}{\kern0pt}rule\ arg{\isacharunderscore}{\kern0pt}cong{\isadigit{2}}{\isacharbrackleft}{\kern0pt}\isakeyword{where}\ f{\isacharequal}{\kern0pt}{\isachardoublequoteopen}{\isacharparenleft}{\kern0pt}powr{\isacharparenright}{\kern0pt}{\isachardoublequoteclose}{\isacharbrackright}{\kern0pt}{\isacharcomma}{\kern0pt}\ simp{\isacharparenright}{\kern0pt}\isanewline
\ \ \ \ \ \isacommand{apply}\isamarkupfalse%
\ {\isacharparenleft}{\kern0pt}subst\ log{\isacharunderscore}{\kern0pt}nat{\isacharunderscore}{\kern0pt}power{\isacharcomma}{\kern0pt}\ simp\ add{\isacharcolon}{\kern0pt}{\isasymdelta}{\isacharunderscore}{\kern0pt}ge{\isacharunderscore}{\kern0pt}{\isadigit{0}}{\isacharparenright}{\kern0pt}\isanewline
\ \ \ \ \ \isacommand{apply}\isamarkupfalse%
\ {\isacharparenleft}{\kern0pt}subst\ log{\isacharunderscore}{\kern0pt}divide{\isacharcomma}{\kern0pt}\ simp{\isacharcomma}{\kern0pt}\ simp{\isacharcomma}{\kern0pt}\ simp{\isacharcomma}{\kern0pt}\ simp\ add{\isacharcolon}{\kern0pt}{\isasymdelta}{\isacharunderscore}{\kern0pt}ge{\isacharunderscore}{\kern0pt}{\isadigit{0}}{\isacharparenright}{\kern0pt}\ \isanewline
\ \ \ \ \isacommand{by}\isamarkupfalse%
\ simp{\isacharplus}{\kern0pt}\isanewline
\ \ \isacommand{also}\isamarkupfalse%
\ \isacommand{have}\isamarkupfalse%
\ {\isachardoublequoteopen}{\isachardot}{\kern0pt}{\isachardot}{\kern0pt}{\isachardot}{\kern0pt}\ {\isacharequal}{\kern0pt}\ {\isadigit{1}}{\isadigit{8}}\ {\isacharasterisk}{\kern0pt}\ {\isadigit{9}}{\isadigit{6}}\ {\isacharasterisk}{\kern0pt}\ {\isadigit{8}}{\isadigit{1}}\isactrlsup {\isadigit{2}}\ {\isacharasterisk}{\kern0pt}\ {\isadigit{2}}\ powr\ {\isacharparenleft}{\kern0pt}{\isacharminus}{\kern0pt}{\isadigit{2}}{\isadigit{4}}{\isacharparenright}{\kern0pt}{\isachardoublequoteclose}\isanewline
\ \ \ \ \isacommand{apply}\isamarkupfalse%
\ {\isacharparenleft}{\kern0pt}subst\ powr{\isacharunderscore}{\kern0pt}log{\isacharunderscore}{\kern0pt}cancel{\isacharcomma}{\kern0pt}\ simp{\isacharcomma}{\kern0pt}\ simp{\isacharcomma}{\kern0pt}\ simp{\isacharparenright}{\kern0pt}\ \isacommand{using}\isamarkupfalse%
\ {\isasymdelta}{\isacharunderscore}{\kern0pt}ge{\isacharunderscore}{\kern0pt}{\isadigit{0}}\ \isacommand{apply}\isamarkupfalse%
\ blast\isanewline
\ \ \ \ \isacommand{apply}\isamarkupfalse%
\ {\isacharparenleft}{\kern0pt}simp\ add{\isacharcolon}{\kern0pt}algebra{\isacharunderscore}{\kern0pt}simps{\isacharparenright}{\kern0pt}\ \isacommand{using}\isamarkupfalse%
\ {\isasymdelta}{\isacharunderscore}{\kern0pt}ge{\isacharunderscore}{\kern0pt}{\isadigit{0}}\ \isacommand{by}\isamarkupfalse%
\ blast\isanewline
\ \ \isacommand{also}\isamarkupfalse%
\ \isacommand{have}\isamarkupfalse%
\ {\isachardoublequoteopen}{\isachardot}{\kern0pt}{\isachardot}{\kern0pt}{\isachardot}{\kern0pt}\ {\isasymle}\ {\isadigit{1}}{\isachardoublequoteclose}\isanewline
\ \ \ \ \isacommand{by}\isamarkupfalse%
\ simp\isanewline
\ \ \isacommand{finally}\isamarkupfalse%
\ \isacommand{have}\isamarkupfalse%
\ r{\isacharunderscore}{\kern0pt}le{\isacharunderscore}{\kern0pt}t{\isadigit{2}}{\isacharcolon}{\kern0pt}\ {\isachardoublequoteopen}{\isadigit{1}}{\isadigit{8}}\ {\isacharasterisk}{\kern0pt}\ {\isadigit{9}}{\isadigit{6}}\ {\isacharasterisk}{\kern0pt}\ {\isacharparenleft}{\kern0pt}real\ t{\isacharparenright}{\kern0pt}\isactrlsup {\isadigit{2}}\ {\isacharasterisk}{\kern0pt}\ {\isadigit{2}}\ powr\ {\isacharparenleft}{\kern0pt}{\isacharminus}{\kern0pt}real\ r{\isacharparenright}{\kern0pt}\ {\isasymle}\ {\isadigit{1}}{\isachardoublequoteclose}\isanewline
\ \ \ \ \isacommand{by}\isamarkupfalse%
\ simp\isanewline
\isanewline
\ \ \isacommand{have}\isamarkupfalse%
\ {\isasymdelta}{\isacharprime}{\kern0pt}{\isacharunderscore}{\kern0pt}ge{\isacharunderscore}{\kern0pt}{\isadigit{0}}{\isacharcolon}{\kern0pt}\ {\isachardoublequoteopen}{\isasymdelta}{\isacharprime}{\kern0pt}\ {\isachargreater}{\kern0pt}\ {\isadigit{0}}{\isachardoublequoteclose}\ \isacommand{using}\isamarkupfalse%
\ assms\ \isacommand{by}\isamarkupfalse%
\ {\isacharparenleft}{\kern0pt}simp\ add{\isacharcolon}{\kern0pt}{\isasymdelta}{\isacharprime}{\kern0pt}{\isacharunderscore}{\kern0pt}def{\isacharparenright}{\kern0pt}\isanewline
\ \ \isacommand{have}\isamarkupfalse%
\ {\isasymdelta}{\isacharprime}{\kern0pt}{\isacharunderscore}{\kern0pt}le{\isacharunderscore}{\kern0pt}{\isadigit{1}}{\isacharcolon}{\kern0pt}\ {\isachardoublequoteopen}{\isasymdelta}{\isacharprime}{\kern0pt}\ {\isacharless}{\kern0pt}\ {\isadigit{1}}{\isachardoublequoteclose}\isanewline
\ \ \ \ \isacommand{apply}\isamarkupfalse%
\ {\isacharparenleft}{\kern0pt}rule\ order{\isacharunderscore}{\kern0pt}less{\isacharunderscore}{\kern0pt}le{\isacharunderscore}{\kern0pt}trans{\isacharbrackleft}{\kern0pt}\isakeyword{where}\ y{\isacharequal}{\kern0pt}{\isachardoublequoteopen}{\isadigit{3}}{\isacharslash}{\kern0pt}{\isadigit{4}}{\isachardoublequoteclose}{\isacharbrackright}{\kern0pt}{\isacharparenright}{\kern0pt}\isanewline
\ \ \ \ \isacommand{using}\isamarkupfalse%
\ assms\ \isacommand{by}\isamarkupfalse%
\ {\isacharparenleft}{\kern0pt}simp\ add{\isacharcolon}{\kern0pt}{\isasymdelta}{\isacharprime}{\kern0pt}{\isacharunderscore}{\kern0pt}def{\isacharparenright}{\kern0pt}{\isacharplus}{\kern0pt}\isanewline
\ \isanewline
\ \ \isacommand{have}\isamarkupfalse%
\ {\isachardoublequoteopen}t\ {\isasymle}\ {\isadigit{8}}{\isadigit{0}}\ {\isacharslash}{\kern0pt}\ {\isacharparenleft}{\kern0pt}real{\isacharunderscore}{\kern0pt}of{\isacharunderscore}{\kern0pt}rat\ {\isasymdelta}{\isacharparenright}{\kern0pt}\isactrlsup {\isadigit{2}}\ {\isacharplus}{\kern0pt}\ {\isadigit{1}}{\isachardoublequoteclose}\isanewline
\ \ \ \ \isacommand{using}\isamarkupfalse%
\ t{\isacharunderscore}{\kern0pt}def\ t{\isacharunderscore}{\kern0pt}ge{\isacharunderscore}{\kern0pt}{\isadigit{0}}\ \isacommand{by}\isamarkupfalse%
\ linarith\isanewline
\ \ \isacommand{also}\isamarkupfalse%
\ \isacommand{have}\isamarkupfalse%
\ {\isachardoublequoteopen}{\isachardot}{\kern0pt}{\isachardot}{\kern0pt}{\isachardot}{\kern0pt}\ {\isacharequal}{\kern0pt}\ {\isadigit{4}}{\isadigit{5}}\ {\isacharslash}{\kern0pt}\ {\isacharparenleft}{\kern0pt}{\isasymdelta}{\isacharprime}{\kern0pt}{\isacharparenright}{\kern0pt}\isactrlsup {\isadigit{2}}\ {\isacharplus}{\kern0pt}\ {\isadigit{1}}{\isachardoublequoteclose}\isanewline
\ \ \ \ \isacommand{by}\isamarkupfalse%
\ {\isacharparenleft}{\kern0pt}simp\ add{\isacharcolon}{\kern0pt}{\isasymdelta}{\isacharprime}{\kern0pt}{\isacharunderscore}{\kern0pt}def\ algebra{\isacharunderscore}{\kern0pt}simps\ power{\isadigit{2}}{\isacharunderscore}{\kern0pt}eq{\isacharunderscore}{\kern0pt}square{\isacharparenright}{\kern0pt}\isanewline
\ \ \isacommand{also}\isamarkupfalse%
\ \isacommand{have}\isamarkupfalse%
\ {\isachardoublequoteopen}{\isachardot}{\kern0pt}{\isachardot}{\kern0pt}{\isachardot}{\kern0pt}\ {\isasymle}\ {\isadigit{4}}{\isadigit{5}}\ {\isacharslash}{\kern0pt}\ {\isasymdelta}{\isacharprime}{\kern0pt}\isactrlsup {\isadigit{2}}\ {\isacharplus}{\kern0pt}\ {\isadigit{1}}\ {\isacharslash}{\kern0pt}\ {\isasymdelta}{\isacharprime}{\kern0pt}\isactrlsup {\isadigit{2}}{\isachardoublequoteclose}\isanewline
\ \ \ \ \isacommand{apply}\isamarkupfalse%
\ {\isacharparenleft}{\kern0pt}rule\ add{\isacharunderscore}{\kern0pt}mono{\isacharcomma}{\kern0pt}\ simp{\isacharparenright}{\kern0pt}\isanewline
\ \ \ \ \isacommand{apply}\isamarkupfalse%
\ {\isacharparenleft}{\kern0pt}subst\ pos{\isacharunderscore}{\kern0pt}le{\isacharunderscore}{\kern0pt}divide{\isacharunderscore}{\kern0pt}eq{\isacharcomma}{\kern0pt}\ simp\ add{\isacharcolon}{\kern0pt}{\isasymdelta}{\isacharprime}{\kern0pt}{\isacharunderscore}{\kern0pt}def{\isacharparenright}{\kern0pt}\isanewline
\ \ \ \ \isacommand{using}\isamarkupfalse%
\ assms\ \isacommand{apply}\isamarkupfalse%
\ force\isanewline
\ \ \ \ \isacommand{apply}\isamarkupfalse%
\ {\isacharparenleft}{\kern0pt}simp\ add{\isacharcolon}{\kern0pt}\ {\isasymdelta}{\isacharprime}{\kern0pt}{\isacharunderscore}{\kern0pt}def\ algebra{\isacharunderscore}{\kern0pt}simps{\isacharparenright}{\kern0pt}\isanewline
\ \ \ \ \isacommand{apply}\isamarkupfalse%
\ {\isacharparenleft}{\kern0pt}subst\ power{\isacharunderscore}{\kern0pt}le{\isacharunderscore}{\kern0pt}one{\isacharunderscore}{\kern0pt}iff{\isacharparenright}{\kern0pt}\isanewline
\ \ \ \ \isacommand{using}\isamarkupfalse%
\ assms\ \isacommand{apply}\isamarkupfalse%
\ simp\isanewline
\ \ \ \ \isacommand{apply}\isamarkupfalse%
\ {\isacharparenleft}{\kern0pt}subst\ pos{\isacharunderscore}{\kern0pt}divide{\isacharunderscore}{\kern0pt}le{\isacharunderscore}{\kern0pt}eq{\isacharcomma}{\kern0pt}\ simp{\isacharcomma}{\kern0pt}\ simp{\isacharparenright}{\kern0pt}\isanewline
\ \ \ \ \isacommand{apply}\isamarkupfalse%
\ {\isacharparenleft}{\kern0pt}rule\ order{\isacharunderscore}{\kern0pt}trans{\isacharbrackleft}{\kern0pt}\isakeyword{where}\ y{\isacharequal}{\kern0pt}{\isachardoublequoteopen}{\isadigit{3}}{\isachardoublequoteclose}{\isacharbrackright}{\kern0pt}{\isacharparenright}{\kern0pt}\isanewline
\ \ \ \ \isacommand{using}\isamarkupfalse%
\ assms{\isacharparenleft}{\kern0pt}{\isadigit{2}}{\isacharparenright}{\kern0pt}\ \isacommand{by}\isamarkupfalse%
\ simp{\isacharplus}{\kern0pt}\isanewline
\ \ \isacommand{also}\isamarkupfalse%
\ \isacommand{have}\isamarkupfalse%
\ {\isachardoublequoteopen}{\isachardot}{\kern0pt}{\isachardot}{\kern0pt}{\isachardot}{\kern0pt}\ {\isacharequal}{\kern0pt}\ {\isadigit{4}}{\isadigit{6}}{\isacharslash}{\kern0pt}\ {\isasymdelta}{\isacharprime}{\kern0pt}\isactrlsup {\isadigit{2}}{\isachardoublequoteclose}\isanewline
\ \ \ \ \isacommand{by}\isamarkupfalse%
\ simp\isanewline
\ \ \isacommand{finally}\isamarkupfalse%
\ \isacommand{have}\isamarkupfalse%
\ t{\isacharunderscore}{\kern0pt}le{\isacharunderscore}{\kern0pt}{\isasymdelta}{\isacharprime}{\kern0pt}{\isacharcolon}{\kern0pt}\ {\isachardoublequoteopen}t\ {\isasymle}\ {\isadigit{4}}{\isadigit{6}}{\isacharslash}{\kern0pt}\ {\isasymdelta}{\isacharprime}{\kern0pt}\isactrlsup {\isadigit{2}}{\isachardoublequoteclose}\ \isacommand{by}\isamarkupfalse%
\ simp\isanewline
\isanewline
\ \ \isacommand{have}\isamarkupfalse%
\ {\isachardoublequoteopen}{\isadigit{4}}{\isadigit{5}}\ {\isacharslash}{\kern0pt}\ {\isasymdelta}{\isacharprime}{\kern0pt}\isactrlsup {\isadigit{2}}\ {\isacharequal}{\kern0pt}\ {\isadigit{8}}{\isadigit{0}}{\isacharslash}{\kern0pt}\ {\isacharparenleft}{\kern0pt}real{\isacharunderscore}{\kern0pt}of{\isacharunderscore}{\kern0pt}rat\ {\isasymdelta}{\isacharparenright}{\kern0pt}\isactrlsup {\isadigit{2}}{\isachardoublequoteclose}\isanewline
\ \ \ \ \isacommand{by}\isamarkupfalse%
\ {\isacharparenleft}{\kern0pt}simp\ add{\isacharcolon}{\kern0pt}{\isasymdelta}{\isacharprime}{\kern0pt}{\isacharunderscore}{\kern0pt}def\ power{\isadigit{2}}{\isacharunderscore}{\kern0pt}eq{\isacharunderscore}{\kern0pt}square{\isacharparenright}{\kern0pt}\isanewline
\ \ \isacommand{also}\isamarkupfalse%
\ \isacommand{have}\isamarkupfalse%
\ {\isachardoublequoteopen}{\isachardot}{\kern0pt}{\isachardot}{\kern0pt}{\isachardot}{\kern0pt}\ {\isasymle}\ t{\isachardoublequoteclose}\isanewline
\ \ \ \ \isacommand{using}\isamarkupfalse%
\ t{\isacharunderscore}{\kern0pt}ge{\isacharunderscore}{\kern0pt}{\isadigit{0}}\ t{\isacharunderscore}{\kern0pt}def\ of{\isacharunderscore}{\kern0pt}nat{\isacharunderscore}{\kern0pt}ceiling\ \isacommand{by}\isamarkupfalse%
\ blast\isanewline
\ \ \isacommand{finally}\isamarkupfalse%
\ \isacommand{have}\isamarkupfalse%
\ t{\isacharunderscore}{\kern0pt}ge{\isacharunderscore}{\kern0pt}{\isasymdelta}{\isacharprime}{\kern0pt}{\isacharcolon}{\kern0pt}\ {\isachardoublequoteopen}\ {\isadigit{4}}{\isadigit{5}}\ {\isacharslash}{\kern0pt}\ {\isasymdelta}{\isacharprime}{\kern0pt}\isactrlsup {\isadigit{2}}\ {\isasymle}\ t{\isachardoublequoteclose}\ \isacommand{by}\isamarkupfalse%
\ simp\isanewline
\isanewline
\ \ \isacommand{have}\isamarkupfalse%
\ p{\isacharunderscore}{\kern0pt}prime{\isacharcolon}{\kern0pt}\ {\isachardoublequoteopen}Factorial{\isacharunderscore}{\kern0pt}Ring{\isachardot}{\kern0pt}prime\ p{\isachardoublequoteclose}\ \isanewline
\ \ \ \ \isacommand{using}\isamarkupfalse%
\ p{\isacharunderscore}{\kern0pt}def\ find{\isacharunderscore}{\kern0pt}prime{\isacharunderscore}{\kern0pt}above{\isacharunderscore}{\kern0pt}is{\isacharunderscore}{\kern0pt}prime\ \isacommand{by}\isamarkupfalse%
\ simp\isanewline
\ \ \isacommand{have}\isamarkupfalse%
\ p{\isacharunderscore}{\kern0pt}ge{\isacharunderscore}{\kern0pt}{\isadigit{1}}{\isadigit{8}}{\isacharcolon}{\kern0pt}\ {\isachardoublequoteopen}p\ {\isasymge}\ {\isadigit{1}}{\isadigit{8}}{\isachardoublequoteclose}\ \isanewline
\ \ \ \ \isacommand{apply}\isamarkupfalse%
\ {\isacharparenleft}{\kern0pt}rule\ order{\isacharunderscore}{\kern0pt}trans{\isacharbrackleft}{\kern0pt}\isakeyword{where}\ y{\isacharequal}{\kern0pt}{\isachardoublequoteopen}{\isadigit{1}}{\isadigit{9}}{\isachardoublequoteclose}{\isacharbrackright}{\kern0pt}{\isacharcomma}{\kern0pt}\ simp{\isacharparenright}{\kern0pt}\isanewline
\ \ \ \ \isacommand{using}\isamarkupfalse%
\ p{\isacharunderscore}{\kern0pt}def\ find{\isacharunderscore}{\kern0pt}prime{\isacharunderscore}{\kern0pt}above{\isacharunderscore}{\kern0pt}lower{\isacharunderscore}{\kern0pt}bound\ max{\isachardot}{\kern0pt}bounded{\isacharunderscore}{\kern0pt}iff\ \isacommand{by}\isamarkupfalse%
\ blast\isanewline
\ \ \isacommand{hence}\isamarkupfalse%
\ p{\isacharunderscore}{\kern0pt}ge{\isacharunderscore}{\kern0pt}{\isadigit{0}}{\isacharcolon}{\kern0pt}\ {\isachardoublequoteopen}p\ {\isachargreater}{\kern0pt}\ {\isadigit{0}}{\isachardoublequoteclose}\ \isacommand{by}\isamarkupfalse%
\ simp\isanewline
\isanewline
\ \ \isacommand{have}\isamarkupfalse%
\ {\isachardoublequoteopen}m\ {\isasymle}\ card\ {\isacharbraceleft}{\kern0pt}{\isadigit{0}}{\isachardot}{\kern0pt}{\isachardot}{\kern0pt}{\isacharless}{\kern0pt}n{\isacharbraceright}{\kern0pt}{\isachardoublequoteclose}\ \isanewline
\ \ \ \ \isacommand{apply}\isamarkupfalse%
\ {\isacharparenleft}{\kern0pt}subst\ m{\isacharunderscore}{\kern0pt}def{\isacharparenright}{\kern0pt}\isanewline
\ \ \ \ \isacommand{by}\isamarkupfalse%
\ {\isacharparenleft}{\kern0pt}rule\ card{\isacharunderscore}{\kern0pt}mono{\isacharcomma}{\kern0pt}\ simp{\isacharcomma}{\kern0pt}\ simp\ add{\isacharcolon}{\kern0pt}assms{\isacharparenleft}{\kern0pt}{\isadigit{3}}{\isacharparenright}{\kern0pt}{\isacharparenright}{\kern0pt}\isanewline
\ \ \isacommand{also}\isamarkupfalse%
\ \isacommand{have}\isamarkupfalse%
\ {\isachardoublequoteopen}{\isachardot}{\kern0pt}{\isachardot}{\kern0pt}{\isachardot}{\kern0pt}\ {\isasymle}\ p{\isachardoublequoteclose}\isanewline
\ \ \ \ \isacommand{by}\isamarkupfalse%
\ {\isacharparenleft}{\kern0pt}metis\ p{\isacharunderscore}{\kern0pt}def\ find{\isacharunderscore}{\kern0pt}prime{\isacharunderscore}{\kern0pt}above{\isacharunderscore}{\kern0pt}lower{\isacharunderscore}{\kern0pt}bound\ card{\isacharunderscore}{\kern0pt}atLeastLessThan\ diff{\isacharunderscore}{\kern0pt}zero\ max{\isacharunderscore}{\kern0pt}def\ order{\isacharunderscore}{\kern0pt}trans{\isacharparenright}{\kern0pt}\isanewline
\ \ \isacommand{finally}\isamarkupfalse%
\ \isacommand{have}\isamarkupfalse%
\ m{\isacharunderscore}{\kern0pt}le{\isacharunderscore}{\kern0pt}p{\isacharcolon}{\kern0pt}\ {\isachardoublequoteopen}m\ {\isasymle}\ p{\isachardoublequoteclose}\ \isacommand{by}\isamarkupfalse%
\ simp\isanewline
\isanewline
\ \ \isacommand{have}\isamarkupfalse%
\ xs{\isacharunderscore}{\kern0pt}le{\isacharunderscore}{\kern0pt}p{\isacharcolon}{\kern0pt}\ {\isachardoublequoteopen}{\isasymAnd}x{\isachardot}{\kern0pt}\ x\ {\isasymin}\ set\ as\ {\isasymLongrightarrow}\ x\ {\isacharless}{\kern0pt}\ p{\isachardoublequoteclose}\ \isanewline
\ \ \ \ \isacommand{apply}\isamarkupfalse%
\ {\isacharparenleft}{\kern0pt}rule\ order{\isacharunderscore}{\kern0pt}less{\isacharunderscore}{\kern0pt}le{\isacharunderscore}{\kern0pt}trans{\isacharbrackleft}{\kern0pt}\isakeyword{where}\ y{\isacharequal}{\kern0pt}{\isachardoublequoteopen}n{\isachardoublequoteclose}{\isacharbrackright}{\kern0pt}{\isacharparenright}{\kern0pt}\isanewline
\ \ \ \ \isacommand{using}\isamarkupfalse%
\ assms{\isacharparenleft}{\kern0pt}{\isadigit{3}}{\isacharparenright}{\kern0pt}\ atLeastLessThan{\isacharunderscore}{\kern0pt}iff\ \isacommand{apply}\isamarkupfalse%
\ blast\isanewline
\ \ \ \ \isacommand{by}\isamarkupfalse%
\ {\isacharparenleft}{\kern0pt}metis\ p{\isacharunderscore}{\kern0pt}def\ find{\isacharunderscore}{\kern0pt}prime{\isacharunderscore}{\kern0pt}above{\isacharunderscore}{\kern0pt}lower{\isacharunderscore}{\kern0pt}bound\ max{\isacharunderscore}{\kern0pt}def\ order{\isacharunderscore}{\kern0pt}trans{\isacharparenright}{\kern0pt}\isanewline
\isanewline
\ \ \isacommand{have}\isamarkupfalse%
\ m{\isacharunderscore}{\kern0pt}eq{\isacharunderscore}{\kern0pt}F{\isacharunderscore}{\kern0pt}{\isadigit{0}}{\isacharcolon}{\kern0pt}\ {\isachardoublequoteopen}real\ m\ {\isacharequal}{\kern0pt}\ of{\isacharunderscore}{\kern0pt}rat\ {\isacharparenleft}{\kern0pt}F\ {\isadigit{0}}\ as{\isacharparenright}{\kern0pt}{\isachardoublequoteclose}\isanewline
\ \ \ \ \isacommand{by}\isamarkupfalse%
\ {\isacharparenleft}{\kern0pt}simp\ add{\isacharcolon}{\kern0pt}m{\isacharunderscore}{\kern0pt}def\ F{\isacharunderscore}{\kern0pt}def{\isacharparenright}{\kern0pt}\isanewline
\isanewline
\ \ \isacommand{have}\isamarkupfalse%
\ fin{\isacharunderscore}{\kern0pt}omega{\isacharunderscore}{\kern0pt}{\isadigit{1}}{\isacharcolon}{\kern0pt}\ {\isachardoublequoteopen}finite\ {\isacharparenleft}{\kern0pt}set{\isacharunderscore}{\kern0pt}pmf\ {\isasymOmega}\isactrlsub {\isadigit{1}}{\isacharparenright}{\kern0pt}{\isachardoublequoteclose}\isanewline
\ \ \ \ \isacommand{apply}\isamarkupfalse%
\ {\isacharparenleft}{\kern0pt}simp\ add{\isacharcolon}{\kern0pt}{\isasymOmega}\isactrlsub {\isadigit{1}}{\isacharunderscore}{\kern0pt}def{\isacharparenright}{\kern0pt}\isanewline
\ \ \ \ \isacommand{by}\isamarkupfalse%
\ {\isacharparenleft}{\kern0pt}metis\ fin{\isacharunderscore}{\kern0pt}bounded{\isacharunderscore}{\kern0pt}degree{\isacharunderscore}{\kern0pt}polynomials{\isacharbrackleft}{\kern0pt}OF\ p{\isacharunderscore}{\kern0pt}ge{\isacharunderscore}{\kern0pt}{\isadigit{0}}{\isacharbrackright}{\kern0pt}\ ne{\isacharunderscore}{\kern0pt}bounded{\isacharunderscore}{\kern0pt}degree{\isacharunderscore}{\kern0pt}polynomials\ set{\isacharunderscore}{\kern0pt}pmf{\isacharunderscore}{\kern0pt}of{\isacharunderscore}{\kern0pt}set{\isacharparenright}{\kern0pt}\isanewline
\isanewline
\ \ \isacommand{have}\isamarkupfalse%
\ exp{\isacharunderscore}{\kern0pt}var{\isacharunderscore}{\kern0pt}f{\isacharcolon}{\kern0pt}\ {\isachardoublequoteopen}{\isasymAnd}a{\isachardot}{\kern0pt}\ a\ {\isasymge}\ {\isacharminus}{\kern0pt}{\isadigit{1}}\ {\isasymLongrightarrow}\ a\ {\isacharless}{\kern0pt}\ int\ p\ {\isasymLongrightarrow}\ \isanewline
\ \ \ \ prob{\isacharunderscore}{\kern0pt}space{\isachardot}{\kern0pt}expectation\ {\isasymOmega}\isactrlsub {\isadigit{1}}\ {\isacharparenleft}{\kern0pt}{\isasymlambda}{\isasymomega}{\isachardot}{\kern0pt}\ real\ {\isacharparenleft}{\kern0pt}f\ a\ {\isasymomega}{\isacharparenright}{\kern0pt}{\isacharparenright}{\kern0pt}\ {\isacharequal}{\kern0pt}\ real\ m\ {\isacharasterisk}{\kern0pt}\ {\isacharparenleft}{\kern0pt}real{\isacharunderscore}{\kern0pt}of{\isacharunderscore}{\kern0pt}int\ a{\isacharplus}{\kern0pt}{\isadigit{1}}{\isacharparenright}{\kern0pt}\ {\isacharslash}{\kern0pt}\ p\ {\isasymand}\isanewline
\ \ \ \ prob{\isacharunderscore}{\kern0pt}space{\isachardot}{\kern0pt}variance\ {\isasymOmega}\isactrlsub {\isadigit{1}}\ {\isacharparenleft}{\kern0pt}{\isasymlambda}{\isasymomega}{\isachardot}{\kern0pt}\ real\ {\isacharparenleft}{\kern0pt}f\ a\ {\isasymomega}{\isacharparenright}{\kern0pt}{\isacharparenright}{\kern0pt}\ {\isasymle}\ real\ m\ {\isacharasterisk}{\kern0pt}\ {\isacharparenleft}{\kern0pt}real{\isacharunderscore}{\kern0pt}of{\isacharunderscore}{\kern0pt}int\ a{\isacharplus}{\kern0pt}{\isadigit{1}}{\isacharparenright}{\kern0pt}\ {\isacharslash}{\kern0pt}\ p{\isachardoublequoteclose}\isanewline
\ \ \isacommand{proof}\isamarkupfalse%
\ {\isacharminus}{\kern0pt}\isanewline
\ \ \ \ \isacommand{fix}\isamarkupfalse%
\ a\ {\isacharcolon}{\kern0pt}{\isacharcolon}{\kern0pt}\ int\isanewline
\ \ \ \ \isacommand{assume}\isamarkupfalse%
\ a{\isacharunderscore}{\kern0pt}ge{\isacharunderscore}{\kern0pt}m{\isadigit{1}}{\isacharcolon}{\kern0pt}\ {\isachardoublequoteopen}a\ {\isasymge}\ {\isacharminus}{\kern0pt}{\isadigit{1}}{\isachardoublequoteclose}\isanewline
\ \ \ \ \isacommand{assume}\isamarkupfalse%
\ a{\isacharunderscore}{\kern0pt}le{\isacharunderscore}{\kern0pt}p{\isacharcolon}{\kern0pt}\ {\isachardoublequoteopen}a\ {\isacharless}{\kern0pt}\ int\ p{\isachardoublequoteclose}\isanewline
\ \ \ \ \isacommand{have}\isamarkupfalse%
\ xs{\isacharunderscore}{\kern0pt}subs{\isacharunderscore}{\kern0pt}p{\isacharcolon}{\kern0pt}\ {\isachardoublequoteopen}set\ as\ {\isasymsubseteq}\ {\isacharbraceleft}{\kern0pt}{\isadigit{0}}{\isachardot}{\kern0pt}{\isachardot}{\kern0pt}{\isacharless}{\kern0pt}p{\isacharbraceright}{\kern0pt}{\isachardoublequoteclose}\isanewline
\ \ \ \ \ \ \isacommand{using}\isamarkupfalse%
\ xs{\isacharunderscore}{\kern0pt}le{\isacharunderscore}{\kern0pt}p\ \ \isanewline
\ \ \ \ \ \ \isacommand{by}\isamarkupfalse%
\ {\isacharparenleft}{\kern0pt}simp\ add{\isacharcolon}{\kern0pt}\ subset{\isacharunderscore}{\kern0pt}iff{\isacharparenright}{\kern0pt}\isanewline
\isanewline
\ \ \ \ \isacommand{have}\isamarkupfalse%
\ exp{\isacharunderscore}{\kern0pt}single{\isacharcolon}{\kern0pt}\ {\isachardoublequoteopen}{\isasymAnd}x{\isachardot}{\kern0pt}\ x\ {\isasymin}\ set\ as\ {\isasymLongrightarrow}\ prob{\isacharunderscore}{\kern0pt}space{\isachardot}{\kern0pt}expectation\ {\isasymOmega}\isactrlsub {\isadigit{1}}\ {\isacharparenleft}{\kern0pt}{\isasymlambda}{\isasymomega}{\isachardot}{\kern0pt}\ of{\isacharunderscore}{\kern0pt}bool\ {\isacharparenleft}{\kern0pt}int\ {\isacharparenleft}{\kern0pt}hash\ p\ x\ {\isasymomega}{\isacharparenright}{\kern0pt}\ {\isasymle}\ a{\isacharparenright}{\kern0pt}{\isacharparenright}{\kern0pt}\ {\isacharequal}{\kern0pt}\ \isanewline
\ \ \ \ \ \ {\isacharparenleft}{\kern0pt}real{\isacharunderscore}{\kern0pt}of{\isacharunderscore}{\kern0pt}int\ a{\isacharplus}{\kern0pt}{\isadigit{1}}{\isacharparenright}{\kern0pt}{\isacharslash}{\kern0pt}real\ p{\isachardoublequoteclose}\isanewline
\ \ \ \ \isacommand{proof}\isamarkupfalse%
\ {\isacharminus}{\kern0pt}\isanewline
\ \ \ \ \ \ \isacommand{fix}\isamarkupfalse%
\ x\isanewline
\ \ \ \ \ \ \isacommand{assume}\isamarkupfalse%
\ {\isachardoublequoteopen}x\ {\isasymin}\ set\ as{\isachardoublequoteclose}\isanewline
\ \ \ \ \ \ \isacommand{hence}\isamarkupfalse%
\ x{\isacharunderscore}{\kern0pt}le{\isacharunderscore}{\kern0pt}p{\isacharcolon}{\kern0pt}\ {\isachardoublequoteopen}x\ {\isacharless}{\kern0pt}\ p{\isachardoublequoteclose}\ \isacommand{using}\isamarkupfalse%
\ xs{\isacharunderscore}{\kern0pt}le{\isacharunderscore}{\kern0pt}p\ \isacommand{by}\isamarkupfalse%
\ simp\isanewline
\ \ \ \ \ \ \isacommand{have}\isamarkupfalse%
\ {\isachardoublequoteopen}prob{\isacharunderscore}{\kern0pt}space{\isachardot}{\kern0pt}expectation\ {\isasymOmega}\isactrlsub {\isadigit{1}}\ {\isacharparenleft}{\kern0pt}{\isasymlambda}{\isasymomega}{\isachardot}{\kern0pt}\ of{\isacharunderscore}{\kern0pt}bool\ {\isacharparenleft}{\kern0pt}int\ {\isacharparenleft}{\kern0pt}hash\ p\ x\ {\isasymomega}{\isacharparenright}{\kern0pt}\ {\isasymle}\ a{\isacharparenright}{\kern0pt}{\isacharparenright}{\kern0pt}\ {\isacharequal}{\kern0pt}\ \isanewline
\ \ \ \ \ \ \ \ measure\ {\isasymOmega}\isactrlsub {\isadigit{1}}\ {\isacharparenleft}{\kern0pt}{\isacharbraceleft}{\kern0pt}{\isasymomega}{\isachardot}{\kern0pt}\ int\ {\isacharparenleft}{\kern0pt}hash\ p\ x\ {\isasymomega}{\isacharparenright}{\kern0pt}\ {\isasymle}\ a{\isacharbraceright}{\kern0pt}\ {\isasyminter}\ space\ {\isasymOmega}\isactrlsub {\isadigit{1}}{\isacharparenright}{\kern0pt}{\isachardoublequoteclose}\isanewline
\ \ \ \ \ \ \ \ \isacommand{apply}\isamarkupfalse%
\ {\isacharparenleft}{\kern0pt}subst\ Bochner{\isacharunderscore}{\kern0pt}Integration{\isachardot}{\kern0pt}integral{\isacharunderscore}{\kern0pt}indicator{\isacharbrackleft}{\kern0pt}\isakeyword{where}\ M{\isacharequal}{\kern0pt}{\isachardoublequoteopen}measure{\isacharunderscore}{\kern0pt}pmf\ {\isasymOmega}\isactrlsub {\isadigit{1}}{\isachardoublequoteclose}{\isacharcomma}{\kern0pt}\ symmetric{\isacharbrackright}{\kern0pt}{\isacharparenright}{\kern0pt}\isanewline
\ \ \ \ \ \ \ \ \isacommand{apply}\isamarkupfalse%
\ {\isacharparenleft}{\kern0pt}rule\ arg{\isacharunderscore}{\kern0pt}cong{\isadigit{2}}{\isacharbrackleft}{\kern0pt}\isakeyword{where}\ f{\isacharequal}{\kern0pt}{\isachardoublequoteopen}integral\isactrlsup L{\isachardoublequoteclose}{\isacharbrackright}{\kern0pt}{\isacharcomma}{\kern0pt}\ simp{\isacharparenright}{\kern0pt}\isanewline
\ \ \ \ \ \ \ \ \isacommand{by}\isamarkupfalse%
\ {\isacharparenleft}{\kern0pt}rule\ ext{\isacharcomma}{\kern0pt}\ simp{\isacharparenright}{\kern0pt}\isanewline
\ \ \ \ \ \ \isacommand{also}\isamarkupfalse%
\ \isacommand{have}\isamarkupfalse%
\ {\isachardoublequoteopen}{\isachardot}{\kern0pt}{\isachardot}{\kern0pt}{\isachardot}{\kern0pt}\ {\isacharequal}{\kern0pt}\ {\isasymP}{\isacharparenleft}{\kern0pt}{\isasymomega}\ in\ {\isasymOmega}\isactrlsub {\isadigit{1}}{\isachardot}{\kern0pt}\ hash\ p\ x\ {\isasymomega}\ {\isasymin}\ {\isacharbraceleft}{\kern0pt}k{\isachardot}{\kern0pt}\ int\ k\ {\isasymle}\ a{\isacharbraceright}{\kern0pt}{\isacharparenright}{\kern0pt}{\isachardoublequoteclose}\isanewline
\ \ \ \ \ \ \ \ \isacommand{by}\isamarkupfalse%
\ simp\isanewline
\ \ \ \ \ \ \isacommand{also}\isamarkupfalse%
\ \isacommand{have}\isamarkupfalse%
\ {\isachardoublequoteopen}{\isachardot}{\kern0pt}{\isachardot}{\kern0pt}{\isachardot}{\kern0pt}\ {\isacharequal}{\kern0pt}\ card\ {\isacharparenleft}{\kern0pt}{\isacharbraceleft}{\kern0pt}k{\isachardot}{\kern0pt}\ int\ k\ {\isasymle}\ a{\isacharbraceright}{\kern0pt}\ {\isasyminter}\ {\isacharbraceleft}{\kern0pt}{\isadigit{0}}{\isachardot}{\kern0pt}{\isachardot}{\kern0pt}{\isacharless}{\kern0pt}p{\isacharbraceright}{\kern0pt}{\isacharparenright}{\kern0pt}\ {\isacharslash}{\kern0pt}\ real\ p{\isachardoublequoteclose}\isanewline
\ \ \ \ \ \ \ \ \isacommand{apply}\isamarkupfalse%
\ {\isacharparenleft}{\kern0pt}simp\ only{\isacharcolon}{\kern0pt}{\isasymOmega}\isactrlsub {\isadigit{1}}{\isacharunderscore}{\kern0pt}def{\isacharparenright}{\kern0pt}\isanewline
\ \ \ \ \ \ \ \ \isacommand{by}\isamarkupfalse%
\ {\isacharparenleft}{\kern0pt}rule\ hash{\isacharunderscore}{\kern0pt}prob{\isacharunderscore}{\kern0pt}range{\isacharbrackleft}{\kern0pt}OF\ p{\isacharunderscore}{\kern0pt}prime\ x{\isacharunderscore}{\kern0pt}le{\isacharunderscore}{\kern0pt}p{\isacharbrackright}{\kern0pt}{\isacharcomma}{\kern0pt}\ simp{\isacharparenright}{\kern0pt}\isanewline
\ \ \ \ \ \ \isacommand{also}\isamarkupfalse%
\ \isacommand{have}\isamarkupfalse%
\ {\isachardoublequoteopen}{\isachardot}{\kern0pt}{\isachardot}{\kern0pt}{\isachardot}{\kern0pt}\ {\isacharequal}{\kern0pt}\ card\ {\isacharbraceleft}{\kern0pt}{\isadigit{0}}{\isachardot}{\kern0pt}{\isachardot}{\kern0pt}{\isacharless}{\kern0pt}nat\ {\isacharparenleft}{\kern0pt}a{\isacharplus}{\kern0pt}{\isadigit{1}}{\isacharparenright}{\kern0pt}{\isacharbraceright}{\kern0pt}\ {\isacharslash}{\kern0pt}\ real\ p{\isachardoublequoteclose}\isanewline
\ \ \ \ \ \ \ \ \isacommand{apply}\isamarkupfalse%
\ {\isacharparenleft}{\kern0pt}rule\ arg{\isacharunderscore}{\kern0pt}cong{\isadigit{2}}{\isacharbrackleft}{\kern0pt}\isakeyword{where}\ f{\isacharequal}{\kern0pt}{\isachardoublequoteopen}{\isacharparenleft}{\kern0pt}{\isacharslash}{\kern0pt}{\isacharparenright}{\kern0pt}{\isachardoublequoteclose}{\isacharbrackright}{\kern0pt}{\isacharparenright}{\kern0pt}\isanewline
\ \ \ \ \ \ \ \ \ \isacommand{apply}\isamarkupfalse%
\ {\isacharparenleft}{\kern0pt}rule\ arg{\isacharunderscore}{\kern0pt}cong{\isacharbrackleft}{\kern0pt}\isakeyword{where}\ f{\isacharequal}{\kern0pt}{\isachardoublequoteopen}real{\isachardoublequoteclose}{\isacharbrackright}{\kern0pt}{\isacharcomma}{\kern0pt}\ rule\ arg{\isacharunderscore}{\kern0pt}cong{\isacharbrackleft}{\kern0pt}\isakeyword{where}\ f{\isacharequal}{\kern0pt}{\isachardoublequoteopen}card{\isachardoublequoteclose}{\isacharbrackright}{\kern0pt}{\isacharparenright}{\kern0pt}\isanewline
\ \ \ \ \ \ \ \ \ \isacommand{apply}\isamarkupfalse%
\ {\isacharparenleft}{\kern0pt}subst\ set{\isacharunderscore}{\kern0pt}eq{\isacharunderscore}{\kern0pt}iff{\isacharcomma}{\kern0pt}\ rule\ allI{\isacharparenright}{\kern0pt}\isanewline
\ \ \ \ \ \ \ \ \ \isacommand{apply}\isamarkupfalse%
\ {\isacharparenleft}{\kern0pt}cases\ {\isachardoublequoteopen}a\ {\isasymge}\ {\isadigit{0}}{\isachardoublequoteclose}{\isacharparenright}{\kern0pt}\isanewline
\ \ \ \ \ \ \ \ \ \ \isacommand{using}\isamarkupfalse%
\ a{\isacharunderscore}{\kern0pt}le{\isacharunderscore}{\kern0pt}p\ \isacommand{apply}\isamarkupfalse%
\ {\isacharparenleft}{\kern0pt}simp{\isacharcomma}{\kern0pt}\ linarith{\isacharparenright}{\kern0pt}\ \isanewline
\ \ \ \ \ \ \ \ \isacommand{by}\isamarkupfalse%
\ simp{\isacharplus}{\kern0pt}\isanewline
\ \ \ \ \ \ \isacommand{also}\isamarkupfalse%
\ \isacommand{have}\isamarkupfalse%
\ {\isachardoublequoteopen}{\isachardot}{\kern0pt}{\isachardot}{\kern0pt}{\isachardot}{\kern0pt}\ {\isacharequal}{\kern0pt}\ \ {\isacharparenleft}{\kern0pt}real{\isacharunderscore}{\kern0pt}of{\isacharunderscore}{\kern0pt}int\ a{\isacharplus}{\kern0pt}{\isadigit{1}}{\isacharparenright}{\kern0pt}{\isacharslash}{\kern0pt}real\ p{\isachardoublequoteclose}\isanewline
\ \ \ \ \ \ \ \ \isacommand{using}\isamarkupfalse%
\ a{\isacharunderscore}{\kern0pt}ge{\isacharunderscore}{\kern0pt}m{\isadigit{1}}\ \isacommand{by}\isamarkupfalse%
\ simp\isanewline
\ \ \ \ \ \ \isacommand{finally}\isamarkupfalse%
\ \isacommand{show}\isamarkupfalse%
\ {\isachardoublequoteopen}prob{\isacharunderscore}{\kern0pt}space{\isachardot}{\kern0pt}expectation\ {\isasymOmega}\isactrlsub {\isadigit{1}}\ {\isacharparenleft}{\kern0pt}{\isasymlambda}{\isasymomega}{\isachardot}{\kern0pt}\ of{\isacharunderscore}{\kern0pt}bool\ {\isacharparenleft}{\kern0pt}int\ {\isacharparenleft}{\kern0pt}hash\ p\ x\ {\isasymomega}{\isacharparenright}{\kern0pt}\ {\isasymle}\ a{\isacharparenright}{\kern0pt}{\isacharparenright}{\kern0pt}\ {\isacharequal}{\kern0pt}\ \isanewline
\ \ \ \ \ \ \ \ {\isacharparenleft}{\kern0pt}real{\isacharunderscore}{\kern0pt}of{\isacharunderscore}{\kern0pt}int\ a{\isacharplus}{\kern0pt}{\isadigit{1}}{\isacharparenright}{\kern0pt}{\isacharslash}{\kern0pt}real\ p{\isachardoublequoteclose}\isanewline
\ \ \ \ \ \ \ \ \isacommand{by}\isamarkupfalse%
\ simp\isanewline
\ \ \ \ \isacommand{qed}\isamarkupfalse%
\isanewline
\ \ \ \ \isacommand{have}\isamarkupfalse%
\ {\isachardoublequoteopen}prob{\isacharunderscore}{\kern0pt}space{\isachardot}{\kern0pt}expectation\ {\isasymOmega}\isactrlsub {\isadigit{1}}\ {\isacharparenleft}{\kern0pt}{\isasymlambda}{\isasymomega}{\isachardot}{\kern0pt}\ real\ {\isacharparenleft}{\kern0pt}f\ a\ {\isasymomega}{\isacharparenright}{\kern0pt}{\isacharparenright}{\kern0pt}\ {\isacharequal}{\kern0pt}\ \isanewline
\ \ \ \ \ \ prob{\isacharunderscore}{\kern0pt}space{\isachardot}{\kern0pt}expectation\ {\isasymOmega}\isactrlsub {\isadigit{1}}\ {\isacharparenleft}{\kern0pt}{\isasymlambda}{\isasymomega}{\isachardot}{\kern0pt}\ {\isacharparenleft}{\kern0pt}{\isasymSum}x\ {\isasymin}\ set\ as{\isachardot}{\kern0pt}\ of{\isacharunderscore}{\kern0pt}bool\ {\isacharparenleft}{\kern0pt}int\ {\isacharparenleft}{\kern0pt}hash\ p\ x\ {\isasymomega}{\isacharparenright}{\kern0pt}\ {\isasymle}\ a{\isacharparenright}{\kern0pt}{\isacharparenright}{\kern0pt}{\isacharparenright}{\kern0pt}{\isachardoublequoteclose}\isanewline
\ \ \ \ \ \ \isacommand{by}\isamarkupfalse%
\ {\isacharparenleft}{\kern0pt}simp\ add{\isacharcolon}{\kern0pt}f{\isacharunderscore}{\kern0pt}def\ inters{\isacharunderscore}{\kern0pt}compr{\isacharparenright}{\kern0pt}\isanewline
\ \ \ \ \isacommand{also}\isamarkupfalse%
\ \isacommand{have}\isamarkupfalse%
\ {\isachardoublequoteopen}{\isachardot}{\kern0pt}{\isachardot}{\kern0pt}{\isachardot}{\kern0pt}\ {\isacharequal}{\kern0pt}\ \ {\isacharparenleft}{\kern0pt}{\isasymSum}x\ {\isasymin}\ set\ as{\isachardot}{\kern0pt}\ prob{\isacharunderscore}{\kern0pt}space{\isachardot}{\kern0pt}expectation\ {\isasymOmega}\isactrlsub {\isadigit{1}}\ {\isacharparenleft}{\kern0pt}{\isasymlambda}{\isasymomega}{\isachardot}{\kern0pt}\ of{\isacharunderscore}{\kern0pt}bool\ {\isacharparenleft}{\kern0pt}int\ {\isacharparenleft}{\kern0pt}hash\ p\ x\ {\isasymomega}{\isacharparenright}{\kern0pt}\ {\isasymle}\ a{\isacharparenright}{\kern0pt}{\isacharparenright}{\kern0pt}{\isacharparenright}{\kern0pt}{\isachardoublequoteclose}\isanewline
\ \ \ \ \ \ \isacommand{apply}\isamarkupfalse%
\ {\isacharparenleft}{\kern0pt}rule\ Bochner{\isacharunderscore}{\kern0pt}Integration{\isachardot}{\kern0pt}integral{\isacharunderscore}{\kern0pt}sum{\isacharparenright}{\kern0pt}\isanewline
\ \ \ \ \ \ \isacommand{by}\isamarkupfalse%
\ {\isacharparenleft}{\kern0pt}rule\ integrable{\isacharunderscore}{\kern0pt}measure{\isacharunderscore}{\kern0pt}pmf{\isacharunderscore}{\kern0pt}finite{\isacharbrackleft}{\kern0pt}OF\ fin{\isacharunderscore}{\kern0pt}omega{\isacharunderscore}{\kern0pt}{\isadigit{1}}{\isacharbrackright}{\kern0pt}{\isacharparenright}{\kern0pt}\isanewline
\ \ \ \ \isacommand{also}\isamarkupfalse%
\ \isacommand{have}\isamarkupfalse%
\ {\isachardoublequoteopen}{\isachardot}{\kern0pt}{\isachardot}{\kern0pt}{\isachardot}{\kern0pt}\ {\isacharequal}{\kern0pt}\ {\isacharparenleft}{\kern0pt}{\isasymSum}\ x\ {\isasymin}\ set\ as{\isachardot}{\kern0pt}\ {\isacharparenleft}{\kern0pt}a{\isacharplus}{\kern0pt}{\isadigit{1}}{\isacharparenright}{\kern0pt}{\isacharslash}{\kern0pt}real\ p{\isacharparenright}{\kern0pt}{\isachardoublequoteclose}\isanewline
\ \ \ \ \ \ \isacommand{by}\isamarkupfalse%
\ {\isacharparenleft}{\kern0pt}rule\ sum{\isachardot}{\kern0pt}cong{\isacharcomma}{\kern0pt}\ simp{\isacharcomma}{\kern0pt}\ subst\ exp{\isacharunderscore}{\kern0pt}single{\isacharcomma}{\kern0pt}\ simp{\isacharcomma}{\kern0pt}\ simp{\isacharparenright}{\kern0pt}\isanewline
\ \ \ \ \isacommand{also}\isamarkupfalse%
\ \isacommand{have}\isamarkupfalse%
\ {\isachardoublequoteopen}{\isachardot}{\kern0pt}{\isachardot}{\kern0pt}{\isachardot}{\kern0pt}\ {\isacharequal}{\kern0pt}\ real\ m\ {\isacharasterisk}{\kern0pt}\ {\isacharparenleft}{\kern0pt}real{\isacharunderscore}{\kern0pt}of{\isacharunderscore}{\kern0pt}int\ a\ {\isacharplus}{\kern0pt}\ {\isadigit{1}}{\isacharparenright}{\kern0pt}\ {\isacharslash}{\kern0pt}real\ p{\isachardoublequoteclose}\isanewline
\ \ \ \ \ \ \isacommand{by}\isamarkupfalse%
\ {\isacharparenleft}{\kern0pt}simp\ add{\isacharcolon}{\kern0pt}m{\isacharunderscore}{\kern0pt}def{\isacharparenright}{\kern0pt}\isanewline
\ \ \ \ \isacommand{finally}\isamarkupfalse%
\ \isacommand{have}\isamarkupfalse%
\ r{\isacharunderscore}{\kern0pt}{\isadigit{1}}{\isacharcolon}{\kern0pt}\ {\isachardoublequoteopen}prob{\isacharunderscore}{\kern0pt}space{\isachardot}{\kern0pt}expectation\ {\isasymOmega}\isactrlsub {\isadigit{1}}\ {\isacharparenleft}{\kern0pt}{\isasymlambda}{\isasymomega}{\isachardot}{\kern0pt}\ real\ {\isacharparenleft}{\kern0pt}f\ a\ {\isasymomega}{\isacharparenright}{\kern0pt}{\isacharparenright}{\kern0pt}\ {\isacharequal}{\kern0pt}\ real\ m\ {\isacharasterisk}{\kern0pt}\ {\isacharparenleft}{\kern0pt}real{\isacharunderscore}{\kern0pt}of{\isacharunderscore}{\kern0pt}int\ a{\isacharplus}{\kern0pt}{\isadigit{1}}{\isacharparenright}{\kern0pt}\ {\isacharslash}{\kern0pt}\ p{\isachardoublequoteclose}\isanewline
\ \ \ \ \ \ \isacommand{by}\isamarkupfalse%
\ simp\isanewline
\isanewline
\ \ \ \ \isacommand{have}\isamarkupfalse%
\ {\isachardoublequoteopen}prob{\isacharunderscore}{\kern0pt}space{\isachardot}{\kern0pt}variance\ {\isasymOmega}\isactrlsub {\isadigit{1}}\ {\isacharparenleft}{\kern0pt}{\isasymlambda}{\isasymomega}{\isachardot}{\kern0pt}\ real\ {\isacharparenleft}{\kern0pt}f\ a\ {\isasymomega}{\isacharparenright}{\kern0pt}{\isacharparenright}{\kern0pt}\ {\isacharequal}{\kern0pt}\ \isanewline
\ \ \ \ \ \ prob{\isacharunderscore}{\kern0pt}space{\isachardot}{\kern0pt}variance\ {\isasymOmega}\isactrlsub {\isadigit{1}}\ {\isacharparenleft}{\kern0pt}{\isasymlambda}{\isasymomega}{\isachardot}{\kern0pt}\ {\isacharparenleft}{\kern0pt}{\isasymSum}x\ {\isasymin}\ set\ as{\isachardot}{\kern0pt}\ of{\isacharunderscore}{\kern0pt}bool\ {\isacharparenleft}{\kern0pt}int\ {\isacharparenleft}{\kern0pt}hash\ p\ x\ {\isasymomega}{\isacharparenright}{\kern0pt}\ {\isasymle}\ a{\isacharparenright}{\kern0pt}{\isacharparenright}{\kern0pt}{\isacharparenright}{\kern0pt}{\isachardoublequoteclose}\isanewline
\ \ \ \ \ \ \isacommand{by}\isamarkupfalse%
\ {\isacharparenleft}{\kern0pt}simp\ add{\isacharcolon}{\kern0pt}f{\isacharunderscore}{\kern0pt}def\ inters{\isacharunderscore}{\kern0pt}compr{\isacharparenright}{\kern0pt}\isanewline
\ \ \ \ \isacommand{also}\isamarkupfalse%
\ \isacommand{have}\isamarkupfalse%
\ {\isachardoublequoteopen}{\isachardot}{\kern0pt}{\isachardot}{\kern0pt}{\isachardot}{\kern0pt}\ {\isacharequal}{\kern0pt}\ {\isacharparenleft}{\kern0pt}{\isasymSum}x\ {\isasymin}\ set\ as{\isachardot}{\kern0pt}\ prob{\isacharunderscore}{\kern0pt}space{\isachardot}{\kern0pt}variance\ {\isasymOmega}\isactrlsub {\isadigit{1}}\ {\isacharparenleft}{\kern0pt}{\isasymlambda}{\isasymomega}{\isachardot}{\kern0pt}\ of{\isacharunderscore}{\kern0pt}bool\ {\isacharparenleft}{\kern0pt}int\ {\isacharparenleft}{\kern0pt}hash\ p\ x\ {\isasymomega}{\isacharparenright}{\kern0pt}\ {\isasymle}\ a{\isacharparenright}{\kern0pt}{\isacharparenright}{\kern0pt}{\isacharparenright}{\kern0pt}{\isachardoublequoteclose}\isanewline
\ \ \ \ \ \ \isacommand{apply}\isamarkupfalse%
\ {\isacharparenleft}{\kern0pt}rule\ prob{\isacharunderscore}{\kern0pt}space{\isachardot}{\kern0pt}var{\isacharunderscore}{\kern0pt}sum{\isacharunderscore}{\kern0pt}pairwise{\isacharunderscore}{\kern0pt}indep{\isacharunderscore}{\kern0pt}{\isadigit{2}}{\isacharcomma}{\kern0pt}\ simp\ add{\isacharcolon}{\kern0pt}prob{\isacharunderscore}{\kern0pt}space{\isacharunderscore}{\kern0pt}measure{\isacharunderscore}{\kern0pt}pmf{\isacharcomma}{\kern0pt}\ simp{\isacharcomma}{\kern0pt}\ simp{\isacharparenright}{\kern0pt}\isanewline
\ \ \ \ \ \ \ \isacommand{apply}\isamarkupfalse%
\ {\isacharparenleft}{\kern0pt}rule\ integrable{\isacharunderscore}{\kern0pt}measure{\isacharunderscore}{\kern0pt}pmf{\isacharunderscore}{\kern0pt}finite{\isacharbrackleft}{\kern0pt}OF\ fin{\isacharunderscore}{\kern0pt}omega{\isacharunderscore}{\kern0pt}{\isadigit{1}}{\isacharbrackright}{\kern0pt}{\isacharparenright}{\kern0pt}\isanewline
\ \ \ \ \ \ \isacommand{apply}\isamarkupfalse%
\ {\isacharparenleft}{\kern0pt}rule\ prob{\isacharunderscore}{\kern0pt}space{\isachardot}{\kern0pt}indep{\isacharunderscore}{\kern0pt}vars{\isacharunderscore}{\kern0pt}compose{\isadigit{2}}{\isacharbrackleft}{\kern0pt}\isakeyword{where}\ Y{\isacharequal}{\kern0pt}{\isachardoublequoteopen}{\isasymlambda}i\ x{\isachardot}{\kern0pt}\ of{\isacharunderscore}{\kern0pt}bool\ {\isacharparenleft}{\kern0pt}int\ x\ {\isasymle}\ a{\isacharparenright}{\kern0pt}{\isachardoublequoteclose}\ \isakeyword{and}\ M{\isacharprime}{\kern0pt}{\isacharequal}{\kern0pt}{\isachardoublequoteopen}{\isasymlambda}{\isacharunderscore}{\kern0pt}{\isachardot}{\kern0pt}\ measure{\isacharunderscore}{\kern0pt}pmf\ {\isacharparenleft}{\kern0pt}pmf{\isacharunderscore}{\kern0pt}of{\isacharunderscore}{\kern0pt}set\ {\isacharbraceleft}{\kern0pt}{\isadigit{0}}{\isachardot}{\kern0pt}{\isachardot}{\kern0pt}{\isacharless}{\kern0pt}p{\isacharbraceright}{\kern0pt}{\isacharparenright}{\kern0pt}{\isachardoublequoteclose}{\isacharbrackright}{\kern0pt}{\isacharparenright}{\kern0pt}\isanewline
\ \ \ \ \ \ \ \ \isacommand{apply}\isamarkupfalse%
\ {\isacharparenleft}{\kern0pt}simp\ add{\isacharcolon}{\kern0pt}prob{\isacharunderscore}{\kern0pt}space{\isacharunderscore}{\kern0pt}measure{\isacharunderscore}{\kern0pt}pmf{\isacharparenright}{\kern0pt}\isanewline
\ \ \ \ \ \ \ \isacommand{using}\isamarkupfalse%
\ hash{\isacharunderscore}{\kern0pt}k{\isacharunderscore}{\kern0pt}wise{\isacharunderscore}{\kern0pt}indep{\isacharbrackleft}{\kern0pt}OF\ p{\isacharunderscore}{\kern0pt}prime{\isacharcomma}{\kern0pt}\ \isakeyword{where}\ n{\isacharequal}{\kern0pt}{\isachardoublequoteopen}{\isadigit{2}}{\isachardoublequoteclose}{\isacharbrackright}{\kern0pt}\ xs{\isacharunderscore}{\kern0pt}subs{\isacharunderscore}{\kern0pt}p\isanewline
\ \ \ \ \ \ \ \isacommand{apply}\isamarkupfalse%
\ {\isacharparenleft}{\kern0pt}simp\ add{\isacharcolon}{\kern0pt}measure{\isacharunderscore}{\kern0pt}pmf{\isachardot}{\kern0pt}k{\isacharunderscore}{\kern0pt}wise{\isacharunderscore}{\kern0pt}indep{\isacharunderscore}{\kern0pt}vars{\isacharunderscore}{\kern0pt}def\ {\isasymOmega}\isactrlsub {\isadigit{1}}{\isacharunderscore}{\kern0pt}def{\isacharparenright}{\kern0pt}\ \isanewline
\ \ \ \ \ \ \ \isacommand{apply}\isamarkupfalse%
\ {\isacharparenleft}{\kern0pt}metis\ le{\isacharunderscore}{\kern0pt}refl\ order{\isacharunderscore}{\kern0pt}trans\ subset{\isacharunderscore}{\kern0pt}eq{\isacharunderscore}{\kern0pt}atLeast{\isadigit{0}}{\isacharunderscore}{\kern0pt}lessThan{\isacharunderscore}{\kern0pt}finite{\isacharparenright}{\kern0pt}\ \isanewline
\ \ \ \ \ \ \isacommand{by}\isamarkupfalse%
\ simp\isanewline
\ \ \ \ \isacommand{also}\isamarkupfalse%
\ \isacommand{have}\isamarkupfalse%
\ {\isachardoublequoteopen}{\isachardot}{\kern0pt}{\isachardot}{\kern0pt}{\isachardot}{\kern0pt}\ {\isasymle}\ {\isacharparenleft}{\kern0pt}{\isasymSum}\ x\ {\isasymin}\ set\ as{\isachardot}{\kern0pt}\ {\isacharparenleft}{\kern0pt}a{\isacharplus}{\kern0pt}{\isadigit{1}}{\isacharparenright}{\kern0pt}{\isacharslash}{\kern0pt}real\ p{\isacharparenright}{\kern0pt}{\isachardoublequoteclose}\isanewline
\ \ \ \ \ \ \isacommand{apply}\isamarkupfalse%
\ {\isacharparenleft}{\kern0pt}rule\ sum{\isacharunderscore}{\kern0pt}mono{\isacharparenright}{\kern0pt}\isanewline
\ \ \ \ \ \ \isacommand{apply}\isamarkupfalse%
\ {\isacharparenleft}{\kern0pt}subst\ prob{\isacharunderscore}{\kern0pt}space{\isachardot}{\kern0pt}variance{\isacharunderscore}{\kern0pt}eq{\isacharbrackleft}{\kern0pt}OF\ prob{\isacharunderscore}{\kern0pt}space{\isacharunderscore}{\kern0pt}measure{\isacharunderscore}{\kern0pt}pmf{\isacharbrackright}{\kern0pt}{\isacharparenright}{\kern0pt}\isanewline
\ \ \ \ \ \ \ \isacommand{apply}\isamarkupfalse%
\ {\isacharparenleft}{\kern0pt}rule\ integrable{\isacharunderscore}{\kern0pt}measure{\isacharunderscore}{\kern0pt}pmf{\isacharunderscore}{\kern0pt}finite{\isacharbrackleft}{\kern0pt}OF\ fin{\isacharunderscore}{\kern0pt}omega{\isacharunderscore}{\kern0pt}{\isadigit{1}}{\isacharbrackright}{\kern0pt}{\isacharparenright}{\kern0pt}\isanewline
\ \ \ \ \ \ \ \isacommand{apply}\isamarkupfalse%
\ {\isacharparenleft}{\kern0pt}rule\ integrable{\isacharunderscore}{\kern0pt}measure{\isacharunderscore}{\kern0pt}pmf{\isacharunderscore}{\kern0pt}finite{\isacharbrackleft}{\kern0pt}OF\ fin{\isacharunderscore}{\kern0pt}omega{\isacharunderscore}{\kern0pt}{\isadigit{1}}{\isacharbrackright}{\kern0pt}{\isacharparenright}{\kern0pt}\isanewline
\ \ \ \ \ \ \isacommand{apply}\isamarkupfalse%
\ {\isacharparenleft}{\kern0pt}simp\ add{\isacharcolon}{\kern0pt}of{\isacharunderscore}{\kern0pt}bool{\isacharunderscore}{\kern0pt}square{\isacharparenright}{\kern0pt}\isanewline
\ \ \ \ \ \ \isacommand{apply}\isamarkupfalse%
\ {\isacharparenleft}{\kern0pt}subst\ exp{\isacharunderscore}{\kern0pt}single{\isacharcomma}{\kern0pt}\ simp{\isacharparenright}{\kern0pt}\isanewline
\ \ \ \ \ \ \isacommand{by}\isamarkupfalse%
\ simp\isanewline
\ \ \ \ \isacommand{also}\isamarkupfalse%
\ \isacommand{have}\isamarkupfalse%
\ {\isachardoublequoteopen}{\isachardot}{\kern0pt}{\isachardot}{\kern0pt}{\isachardot}{\kern0pt}\ {\isacharequal}{\kern0pt}\ real\ m\ {\isacharasterisk}{\kern0pt}\ {\isacharparenleft}{\kern0pt}real{\isacharunderscore}{\kern0pt}of{\isacharunderscore}{\kern0pt}int\ a\ {\isacharplus}{\kern0pt}\ {\isadigit{1}}{\isacharparenright}{\kern0pt}\ {\isacharslash}{\kern0pt}real\ p{\isachardoublequoteclose}\isanewline
\ \ \ \ \ \ \isacommand{by}\isamarkupfalse%
\ {\isacharparenleft}{\kern0pt}simp\ add{\isacharcolon}{\kern0pt}m{\isacharunderscore}{\kern0pt}def{\isacharparenright}{\kern0pt}\isanewline
\ \ \ \ \isacommand{finally}\isamarkupfalse%
\ \isacommand{have}\isamarkupfalse%
\ r{\isacharunderscore}{\kern0pt}{\isadigit{2}}{\isacharcolon}{\kern0pt}\ {\isachardoublequoteopen}prob{\isacharunderscore}{\kern0pt}space{\isachardot}{\kern0pt}variance\ {\isasymOmega}\isactrlsub {\isadigit{1}}\ {\isacharparenleft}{\kern0pt}{\isasymlambda}{\isasymomega}{\isachardot}{\kern0pt}\ real\ {\isacharparenleft}{\kern0pt}f\ a\ {\isasymomega}{\isacharparenright}{\kern0pt}{\isacharparenright}{\kern0pt}\ {\isasymle}\ real\ m\ {\isacharasterisk}{\kern0pt}\ {\isacharparenleft}{\kern0pt}real{\isacharunderscore}{\kern0pt}of{\isacharunderscore}{\kern0pt}int\ a{\isacharplus}{\kern0pt}{\isadigit{1}}{\isacharparenright}{\kern0pt}\ {\isacharslash}{\kern0pt}\ p{\isachardoublequoteclose}\isanewline
\ \ \ \ \ \ \isacommand{by}\isamarkupfalse%
\ simp\isanewline
\ \ \ \ \isacommand{show}\isamarkupfalse%
\isanewline
\ \ \ \ \ \ {\isachardoublequoteopen}prob{\isacharunderscore}{\kern0pt}space{\isachardot}{\kern0pt}expectation\ {\isasymOmega}\isactrlsub {\isadigit{1}}\ {\isacharparenleft}{\kern0pt}{\isasymlambda}{\isasymomega}{\isachardot}{\kern0pt}\ real\ {\isacharparenleft}{\kern0pt}f\ a\ {\isasymomega}{\isacharparenright}{\kern0pt}{\isacharparenright}{\kern0pt}\ {\isacharequal}{\kern0pt}\ real\ m\ {\isacharasterisk}{\kern0pt}\ {\isacharparenleft}{\kern0pt}real{\isacharunderscore}{\kern0pt}of{\isacharunderscore}{\kern0pt}int\ a{\isacharplus}{\kern0pt}{\isadigit{1}}{\isacharparenright}{\kern0pt}\ {\isacharslash}{\kern0pt}\ p\ {\isasymand}\isanewline
\ \ \ \ \ \ \ prob{\isacharunderscore}{\kern0pt}space{\isachardot}{\kern0pt}variance\ {\isasymOmega}\isactrlsub {\isadigit{1}}\ {\isacharparenleft}{\kern0pt}{\isasymlambda}{\isasymomega}{\isachardot}{\kern0pt}\ real\ {\isacharparenleft}{\kern0pt}f\ a\ {\isasymomega}{\isacharparenright}{\kern0pt}{\isacharparenright}{\kern0pt}\ {\isasymle}\ real\ m\ {\isacharasterisk}{\kern0pt}\ {\isacharparenleft}{\kern0pt}real{\isacharunderscore}{\kern0pt}of{\isacharunderscore}{\kern0pt}int\ a{\isacharplus}{\kern0pt}{\isadigit{1}}{\isacharparenright}{\kern0pt}\ {\isacharslash}{\kern0pt}\ p{\isachardoublequoteclose}\isanewline
\ \ \ \ \ \ \isacommand{using}\isamarkupfalse%
\ r{\isacharunderscore}{\kern0pt}{\isadigit{1}}\ r{\isacharunderscore}{\kern0pt}{\isadigit{2}}\ \isacommand{by}\isamarkupfalse%
\ auto\isanewline
\ \ \isacommand{qed}\isamarkupfalse%
\isanewline
\isanewline
\ \ \isacommand{have}\isamarkupfalse%
\ exp{\isacharunderscore}{\kern0pt}f{\isacharcolon}{\kern0pt}\ {\isachardoublequoteopen}{\isasymAnd}a{\isachardot}{\kern0pt}\ \ a\ {\isasymge}\ {\isacharminus}{\kern0pt}{\isadigit{1}}\ {\isasymLongrightarrow}\ a\ {\isacharless}{\kern0pt}\ int\ p\ {\isasymLongrightarrow}\ prob{\isacharunderscore}{\kern0pt}space{\isachardot}{\kern0pt}expectation\ {\isasymOmega}\isactrlsub {\isadigit{1}}\ {\isacharparenleft}{\kern0pt}{\isasymlambda}{\isasymomega}{\isachardot}{\kern0pt}\ real\ {\isacharparenleft}{\kern0pt}f\ a\ {\isasymomega}{\isacharparenright}{\kern0pt}{\isacharparenright}{\kern0pt}\ {\isacharequal}{\kern0pt}\isanewline
\ \ \ \ real\ m\ {\isacharasterisk}{\kern0pt}\ {\isacharparenleft}{\kern0pt}real{\isacharunderscore}{\kern0pt}of{\isacharunderscore}{\kern0pt}int\ a{\isacharplus}{\kern0pt}{\isadigit{1}}{\isacharparenright}{\kern0pt}\ {\isacharslash}{\kern0pt}\ p{\isachardoublequoteclose}\ \isacommand{using}\isamarkupfalse%
\ exp{\isacharunderscore}{\kern0pt}var{\isacharunderscore}{\kern0pt}f\ \isacommand{by}\isamarkupfalse%
\ blast\isanewline
\isanewline
\ \ \isacommand{have}\isamarkupfalse%
\ var{\isacharunderscore}{\kern0pt}f{\isacharcolon}{\kern0pt}\ {\isachardoublequoteopen}{\isasymAnd}a{\isachardot}{\kern0pt}\ a\ {\isasymge}\ {\isacharminus}{\kern0pt}{\isadigit{1}}\ {\isasymLongrightarrow}\ a\ {\isacharless}{\kern0pt}\ int\ p\ {\isasymLongrightarrow}\ prob{\isacharunderscore}{\kern0pt}space{\isachardot}{\kern0pt}variance\ {\isasymOmega}\isactrlsub {\isadigit{1}}\ {\isacharparenleft}{\kern0pt}{\isasymlambda}{\isasymomega}{\isachardot}{\kern0pt}\ real\ {\isacharparenleft}{\kern0pt}f\ a\ {\isasymomega}{\isacharparenright}{\kern0pt}{\isacharparenright}{\kern0pt}\ {\isasymle}\isanewline
\ \ \ \ real\ m\ {\isacharasterisk}{\kern0pt}\ {\isacharparenleft}{\kern0pt}real{\isacharunderscore}{\kern0pt}of{\isacharunderscore}{\kern0pt}int\ a{\isacharplus}{\kern0pt}{\isadigit{1}}{\isacharparenright}{\kern0pt}\ {\isacharslash}{\kern0pt}\ p{\isachardoublequoteclose}\ \isacommand{using}\isamarkupfalse%
\ exp{\isacharunderscore}{\kern0pt}var{\isacharunderscore}{\kern0pt}f\ \isacommand{by}\isamarkupfalse%
\ blast\isanewline
\isanewline
\ \ \isacommand{have}\isamarkupfalse%
\ b{\isacharcolon}{\kern0pt}\ {\isachardoublequoteopen}{\isasymP}{\isacharparenleft}{\kern0pt}{\isasymomega}\ in\ measure{\isacharunderscore}{\kern0pt}pmf\ {\isasymOmega}\isactrlsub {\isadigit{1}}{\isachardot}{\kern0pt}\ \isanewline
\ \ \ \ of{\isacharunderscore}{\kern0pt}rat\ {\isasymdelta}\ {\isacharasterisk}{\kern0pt}\ real{\isacharunderscore}{\kern0pt}of{\isacharunderscore}{\kern0pt}rat\ {\isacharparenleft}{\kern0pt}F\ {\isadigit{0}}\ as{\isacharparenright}{\kern0pt}\ {\isacharless}{\kern0pt}\ {\isasymbar}g{\isacharprime}{\kern0pt}\ {\isacharparenleft}{\kern0pt}h\ {\isasymomega}{\isacharparenright}{\kern0pt}\ {\isacharminus}{\kern0pt}\ of{\isacharunderscore}{\kern0pt}rat\ {\isacharparenleft}{\kern0pt}F\ {\isadigit{0}}\ as{\isacharparenright}{\kern0pt}{\isasymbar}{\isacharparenright}{\kern0pt}\ {\isasymle}\ {\isadigit{1}}{\isacharslash}{\kern0pt}{\isadigit{3}}{\isachardoublequoteclose}\isanewline
\ \ \isacommand{proof}\isamarkupfalse%
\ {\isacharparenleft}{\kern0pt}cases\ {\isachardoublequoteopen}card\ {\isacharparenleft}{\kern0pt}set\ as{\isacharparenright}{\kern0pt}\ {\isasymge}\ t{\isachardoublequoteclose}{\isacharparenright}{\kern0pt}\isanewline
\ \ \ \ \isacommand{case}\isamarkupfalse%
\ True\isanewline
\ \ \ \ \isacommand{hence}\isamarkupfalse%
\ t{\isacharunderscore}{\kern0pt}le{\isacharunderscore}{\kern0pt}m{\isacharcolon}{\kern0pt}\ {\isachardoublequoteopen}t\ {\isasymle}\ card\ {\isacharparenleft}{\kern0pt}set\ as{\isacharparenright}{\kern0pt}{\isachardoublequoteclose}\ \isacommand{by}\isamarkupfalse%
\ simp\isanewline
\ \ \ \ \isacommand{have}\isamarkupfalse%
\ m{\isacharunderscore}{\kern0pt}ge{\isacharunderscore}{\kern0pt}{\isadigit{0}}{\isacharcolon}{\kern0pt}\ {\isachardoublequoteopen}real\ m\ {\isachargreater}{\kern0pt}\ {\isadigit{0}}{\isachardoublequoteclose}\isanewline
\ \ \ \ \ \ \isacommand{using}\isamarkupfalse%
\ m{\isacharunderscore}{\kern0pt}def\ True\ t{\isacharunderscore}{\kern0pt}ge{\isacharunderscore}{\kern0pt}{\isadigit{0}}\ \isacommand{by}\isamarkupfalse%
\ simp\isanewline
\ \ \isanewline
\ \ \ \ \isacommand{have}\isamarkupfalse%
\ b{\isacharunderscore}{\kern0pt}le{\isacharunderscore}{\kern0pt}tpm\ {\isacharcolon}{\kern0pt}{\isachardoublequoteopen}b\ {\isasymle}\ real\ t\ {\isacharasterisk}{\kern0pt}\ real\ p\ {\isacharslash}{\kern0pt}\ {\isacharparenleft}{\kern0pt}real\ m\ {\isacharasterisk}{\kern0pt}\ {\isacharparenleft}{\kern0pt}{\isadigit{1}}\ {\isacharminus}{\kern0pt}\ {\isasymdelta}{\isacharprime}{\kern0pt}{\isacharparenright}{\kern0pt}{\isacharparenright}{\kern0pt}{\isachardoublequoteclose}\isanewline
\ \ \ \ \ \ \isacommand{by}\isamarkupfalse%
\ {\isacharparenleft}{\kern0pt}simp\ add{\isacharcolon}{\kern0pt}b{\isacharunderscore}{\kern0pt}def{\isacharparenright}{\kern0pt}\isanewline
\ \ \ \ \isacommand{also}\isamarkupfalse%
\ \isacommand{have}\isamarkupfalse%
\ {\isachardoublequoteopen}{\isachardot}{\kern0pt}{\isachardot}{\kern0pt}{\isachardot}{\kern0pt}\ {\isasymle}\ real\ t\ {\isacharasterisk}{\kern0pt}\ real\ p\ {\isacharslash}{\kern0pt}\ {\isacharparenleft}{\kern0pt}real\ m\ {\isacharasterisk}{\kern0pt}\ {\isacharparenleft}{\kern0pt}{\isadigit{1}}{\isacharslash}{\kern0pt}{\isadigit{4}}{\isacharparenright}{\kern0pt}{\isacharparenright}{\kern0pt}{\isachardoublequoteclose}\isanewline
\ \ \ \ \ \ \isacommand{apply}\isamarkupfalse%
\ {\isacharparenleft}{\kern0pt}rule\ divide{\isacharunderscore}{\kern0pt}left{\isacharunderscore}{\kern0pt}mono{\isacharparenright}{\kern0pt}\isanewline
\ \ \ \ \ \ \ \ \isacommand{apply}\isamarkupfalse%
\ {\isacharparenleft}{\kern0pt}rule\ mult{\isacharunderscore}{\kern0pt}left{\isacharunderscore}{\kern0pt}mono{\isacharparenright}{\kern0pt}\isanewline
\ \ \ \ \ \ \ \ \isacommand{using}\isamarkupfalse%
\ assms\ \isacommand{apply}\isamarkupfalse%
\ {\isacharparenleft}{\kern0pt}simp\ add{\isacharcolon}{\kern0pt}{\isasymdelta}{\isacharprime}{\kern0pt}{\isacharunderscore}{\kern0pt}def{\isacharparenright}{\kern0pt}\isanewline
\ \ \ \ \ \ \isacommand{using}\isamarkupfalse%
\ m{\isacharunderscore}{\kern0pt}ge{\isacharunderscore}{\kern0pt}{\isadigit{0}}\ {\isasymdelta}{\isacharprime}{\kern0pt}{\isacharunderscore}{\kern0pt}le{\isacharunderscore}{\kern0pt}{\isadigit{1}}\ \isacommand{by}\isamarkupfalse%
\ {\isacharparenleft}{\kern0pt}auto\ intro{\isacharbang}{\kern0pt}{\isacharcolon}{\kern0pt}mult{\isacharunderscore}{\kern0pt}pos{\isacharunderscore}{\kern0pt}pos{\isacharparenright}{\kern0pt}\isanewline
\ \ \ \ \isacommand{finally}\isamarkupfalse%
\ \isacommand{have}\isamarkupfalse%
\ b{\isacharunderscore}{\kern0pt}le{\isacharunderscore}{\kern0pt}tpm{\isacharcolon}{\kern0pt}\ {\isachardoublequoteopen}b\ {\isasymle}\ {\isadigit{4}}\ {\isacharasterisk}{\kern0pt}\ real\ t\ {\isacharasterisk}{\kern0pt}\ real\ p\ {\isacharslash}{\kern0pt}\ real\ m{\isachardoublequoteclose}\isanewline
\ \ \ \ \ \ \isacommand{by}\isamarkupfalse%
\ {\isacharparenleft}{\kern0pt}simp\ add{\isacharcolon}{\kern0pt}algebra{\isacharunderscore}{\kern0pt}simps{\isacharparenright}{\kern0pt}\isanewline
\isanewline
\ \ \ \ \isacommand{have}\isamarkupfalse%
\ a{\isacharunderscore}{\kern0pt}ge{\isacharunderscore}{\kern0pt}{\isadigit{0}}{\isacharcolon}{\kern0pt}\ {\isachardoublequoteopen}a\ {\isasymge}\ {\isadigit{0}}{\isachardoublequoteclose}\ \isanewline
\ \ \ \ \ \ \isacommand{apply}\isamarkupfalse%
\ {\isacharparenleft}{\kern0pt}simp\ add{\isacharcolon}{\kern0pt}a{\isacharunderscore}{\kern0pt}def{\isacharparenright}{\kern0pt}\isanewline
\ \ \ \ \ \ \isacommand{apply}\isamarkupfalse%
\ {\isacharparenleft}{\kern0pt}rule\ divide{\isacharunderscore}{\kern0pt}nonneg{\isacharunderscore}{\kern0pt}nonneg{\isacharcomma}{\kern0pt}\ simp{\isacharparenright}{\kern0pt}\isanewline
\ \ \ \ \ \ \isacommand{using}\isamarkupfalse%
\ {\isasymdelta}{\isacharprime}{\kern0pt}{\isacharunderscore}{\kern0pt}ge{\isacharunderscore}{\kern0pt}{\isadigit{0}}\ \isacommand{by}\isamarkupfalse%
\ simp\isanewline
\ \ \ \ \isacommand{have}\isamarkupfalse%
\ b{\isacharunderscore}{\kern0pt}ge{\isacharunderscore}{\kern0pt}{\isadigit{0}}{\isacharcolon}{\kern0pt}\ {\isachardoublequoteopen}b\ {\isachargreater}{\kern0pt}\ {\isadigit{0}}{\isachardoublequoteclose}\ \isanewline
\ \ \ \ \ \ \isacommand{apply}\isamarkupfalse%
\ {\isacharparenleft}{\kern0pt}simp\ add{\isacharcolon}{\kern0pt}b{\isacharunderscore}{\kern0pt}def{\isacharparenright}{\kern0pt}\isanewline
\ \ \ \ \ \ \isacommand{apply}\isamarkupfalse%
\ {\isacharparenleft}{\kern0pt}subst\ pos{\isacharunderscore}{\kern0pt}less{\isacharunderscore}{\kern0pt}divide{\isacharunderscore}{\kern0pt}eq{\isacharparenright}{\kern0pt}\isanewline
\ \ \ \ \ \ \ \isacommand{apply}\isamarkupfalse%
\ {\isacharparenleft}{\kern0pt}rule\ mult{\isacharunderscore}{\kern0pt}pos{\isacharunderscore}{\kern0pt}pos{\isacharparenright}{\kern0pt}\isanewline
\ \ \ \ \ \ \isacommand{using}\isamarkupfalse%
\ True\ m{\isacharunderscore}{\kern0pt}def\ t{\isacharunderscore}{\kern0pt}ge{\isacharunderscore}{\kern0pt}{\isadigit{0}}\ \isacommand{apply}\isamarkupfalse%
\ simp\isanewline
\ \ \ \ \ \ \isacommand{using}\isamarkupfalse%
\ {\isasymdelta}{\isacharprime}{\kern0pt}{\isacharunderscore}{\kern0pt}le{\isacharunderscore}{\kern0pt}{\isadigit{1}}\ \isacommand{apply}\isamarkupfalse%
\ simp\isanewline
\ \ \ \ \ \ \isacommand{apply}\isamarkupfalse%
\ simp\isanewline
\ \ \ \ \ \ \isacommand{apply}\isamarkupfalse%
\ {\isacharparenleft}{\kern0pt}subst\ mult{\isachardot}{\kern0pt}commute{\isacharparenright}{\kern0pt}\isanewline
\ \ \ \ \ \ \isacommand{apply}\isamarkupfalse%
\ {\isacharparenleft}{\kern0pt}rule\ order{\isacharunderscore}{\kern0pt}less{\isacharunderscore}{\kern0pt}le{\isacharunderscore}{\kern0pt}trans{\isacharbrackleft}{\kern0pt}\isakeyword{where}\ y{\isacharequal}{\kern0pt}{\isachardoublequoteopen}real\ m{\isachardoublequoteclose}{\isacharbrackright}{\kern0pt}{\isacharparenright}{\kern0pt}\ \isacommand{using}\isamarkupfalse%
\ {\isasymdelta}{\isacharprime}{\kern0pt}{\isacharunderscore}{\kern0pt}ge{\isacharunderscore}{\kern0pt}{\isadigit{0}}\ m{\isacharunderscore}{\kern0pt}ge{\isacharunderscore}{\kern0pt}{\isadigit{0}}\ \isacommand{apply}\isamarkupfalse%
\ simp\isanewline
\ \ \ \ \ \ \isacommand{apply}\isamarkupfalse%
\ {\isacharparenleft}{\kern0pt}rule\ order{\isacharunderscore}{\kern0pt}trans{\isacharbrackleft}{\kern0pt}\isakeyword{where}\ y{\isacharequal}{\kern0pt}{\isachardoublequoteopen}{\isadigit{1}}\ {\isacharasterisk}{\kern0pt}\ real\ p{\isachardoublequoteclose}{\isacharbrackright}{\kern0pt}{\isacharparenright}{\kern0pt}\ \isacommand{using}\isamarkupfalse%
\ m{\isacharunderscore}{\kern0pt}le{\isacharunderscore}{\kern0pt}p\ \isacommand{apply}\isamarkupfalse%
\ simp\isanewline
\ \ \ \ \ \ \isacommand{apply}\isamarkupfalse%
\ {\isacharparenleft}{\kern0pt}rule\ mult{\isacharunderscore}{\kern0pt}right{\isacharunderscore}{\kern0pt}mono{\isacharparenright}{\kern0pt}\ \isacommand{using}\isamarkupfalse%
\ t{\isacharunderscore}{\kern0pt}ge{\isacharunderscore}{\kern0pt}{\isadigit{0}}\ \isacommand{apply}\isamarkupfalse%
\ simp\isanewline
\ \ \ \ \ \ \isacommand{by}\isamarkupfalse%
\ simp\isanewline
\ \ \ \ \isacommand{hence}\isamarkupfalse%
\ b{\isacharunderscore}{\kern0pt}ge{\isacharunderscore}{\kern0pt}{\isadigit{1}}{\isacharcolon}{\kern0pt}\ {\isachardoublequoteopen}real{\isacharunderscore}{\kern0pt}of{\isacharunderscore}{\kern0pt}int\ b\ {\isasymge}\ {\isadigit{1}}{\isachardoublequoteclose}\isanewline
\ \ \ \ \ \ \isacommand{by}\isamarkupfalse%
\ linarith\isanewline
\isanewline
\ \ \ \ \isacommand{have}\isamarkupfalse%
\ a{\isacharunderscore}{\kern0pt}le{\isacharunderscore}{\kern0pt}p{\isacharcolon}{\kern0pt}\ {\isachardoublequoteopen}a\ {\isacharless}{\kern0pt}\ real\ p{\isachardoublequoteclose}\isanewline
\ \ \ \ \ \ \isacommand{apply}\isamarkupfalse%
\ {\isacharparenleft}{\kern0pt}rule\ order{\isacharunderscore}{\kern0pt}le{\isacharunderscore}{\kern0pt}less{\isacharunderscore}{\kern0pt}trans{\isacharbrackleft}{\kern0pt}\isakeyword{where}\ y{\isacharequal}{\kern0pt}{\isachardoublequoteopen}real\ t\ {\isacharasterisk}{\kern0pt}\ real\ p\ {\isacharslash}{\kern0pt}\ {\isacharparenleft}{\kern0pt}real\ m\ {\isacharasterisk}{\kern0pt}\ {\isacharparenleft}{\kern0pt}{\isadigit{1}}\ {\isacharplus}{\kern0pt}\ {\isasymdelta}{\isacharprime}{\kern0pt}{\isacharparenright}{\kern0pt}{\isacharparenright}{\kern0pt}{\isachardoublequoteclose}{\isacharbrackright}{\kern0pt}{\isacharparenright}{\kern0pt}\isanewline
\ \ \ \ \ \ \ \isacommand{apply}\isamarkupfalse%
\ {\isacharparenleft}{\kern0pt}simp\ add{\isacharcolon}{\kern0pt}a{\isacharunderscore}{\kern0pt}def{\isacharparenright}{\kern0pt}\isanewline
\ \ \ \ \ \ \isacommand{apply}\isamarkupfalse%
\ {\isacharparenleft}{\kern0pt}subst\ pos{\isacharunderscore}{\kern0pt}divide{\isacharunderscore}{\kern0pt}less{\isacharunderscore}{\kern0pt}eq{\isacharparenright}{\kern0pt}\ \isacommand{using}\isamarkupfalse%
\ m{\isacharunderscore}{\kern0pt}ge{\isacharunderscore}{\kern0pt}{\isadigit{0}}\ {\isasymdelta}{\isacharprime}{\kern0pt}{\isacharunderscore}{\kern0pt}ge{\isacharunderscore}{\kern0pt}{\isadigit{0}}\ \isacommand{apply}\isamarkupfalse%
\ force\isanewline
\ \ \ \ \ \ \isacommand{apply}\isamarkupfalse%
\ {\isacharparenleft}{\kern0pt}subst\ mult{\isachardot}{\kern0pt}commute{\isacharparenright}{\kern0pt}\isanewline
\ \ \ \ \ \ \isacommand{apply}\isamarkupfalse%
\ {\isacharparenleft}{\kern0pt}rule\ mult{\isacharunderscore}{\kern0pt}strict{\isacharunderscore}{\kern0pt}left{\isacharunderscore}{\kern0pt}mono{\isacharparenright}{\kern0pt}\isanewline
\ \ \ \ \ \ \ \isacommand{apply}\isamarkupfalse%
\ {\isacharparenleft}{\kern0pt}rule\ order{\isacharunderscore}{\kern0pt}le{\isacharunderscore}{\kern0pt}less{\isacharunderscore}{\kern0pt}trans{\isacharbrackleft}{\kern0pt}\isakeyword{where}\ y{\isacharequal}{\kern0pt}{\isachardoublequoteopen}real\ m{\isachardoublequoteclose}{\isacharbrackright}{\kern0pt}{\isacharparenright}{\kern0pt}\ \isacommand{using}\isamarkupfalse%
\ True\ m{\isacharunderscore}{\kern0pt}def\ \isacommand{apply}\isamarkupfalse%
\ linarith\isanewline
\ \ \ \ \ \ \isacommand{using}\isamarkupfalse%
\ {\isasymdelta}{\isacharprime}{\kern0pt}{\isacharunderscore}{\kern0pt}ge{\isacharunderscore}{\kern0pt}{\isadigit{0}}\ m{\isacharunderscore}{\kern0pt}ge{\isacharunderscore}{\kern0pt}{\isadigit{0}}\ \isacommand{apply}\isamarkupfalse%
\ force\isanewline
\ \ \ \ \ \ \isacommand{using}\isamarkupfalse%
\ p{\isacharunderscore}{\kern0pt}ge{\isacharunderscore}{\kern0pt}{\isadigit{0}}\ \isacommand{by}\isamarkupfalse%
\ simp\isanewline
\ \ \ \ \isacommand{hence}\isamarkupfalse%
\ a{\isacharunderscore}{\kern0pt}le{\isacharunderscore}{\kern0pt}p{\isacharcolon}{\kern0pt}\ {\isachardoublequoteopen}a\ {\isacharless}{\kern0pt}\ int\ p{\isachardoublequoteclose}\isanewline
\ \ \ \ \ \ \isacommand{by}\isamarkupfalse%
\ linarith\isanewline
\isanewline
\ \ \ \ \isacommand{have}\isamarkupfalse%
\ {\isachardoublequoteopen}{\isasymP}{\isacharparenleft}{\kern0pt}{\isasymomega}\ in\ measure{\isacharunderscore}{\kern0pt}pmf\ {\isasymOmega}\isactrlsub {\isadigit{1}}{\isachardot}{\kern0pt}\ f\ a\ {\isasymomega}\ {\isasymge}\ t{\isacharparenright}{\kern0pt}\ {\isasymle}\ \isanewline
\ \ \ \ \ \ {\isasymP}{\isacharparenleft}{\kern0pt}{\isasymomega}\ in\ measure{\isacharunderscore}{\kern0pt}pmf\ {\isasymOmega}\isactrlsub {\isadigit{1}}{\isachardot}{\kern0pt}\ abs\ {\isacharparenleft}{\kern0pt}real\ {\isacharparenleft}{\kern0pt}f\ a\ {\isasymomega}{\isacharparenright}{\kern0pt}\ {\isacharminus}{\kern0pt}\ prob{\isacharunderscore}{\kern0pt}space{\isachardot}{\kern0pt}expectation\ {\isacharparenleft}{\kern0pt}measure{\isacharunderscore}{\kern0pt}pmf\ {\isasymOmega}\isactrlsub {\isadigit{1}}{\isacharparenright}{\kern0pt}\ {\isacharparenleft}{\kern0pt}{\isasymlambda}{\isasymomega}{\isachardot}{\kern0pt}\ real\ {\isacharparenleft}{\kern0pt}f\ a\ {\isasymomega}{\isacharparenright}{\kern0pt}{\isacharparenright}{\kern0pt}{\isacharparenright}{\kern0pt}\ \isanewline
\ \ \ \ \ \ {\isasymge}\ {\isadigit{3}}\ {\isacharasterisk}{\kern0pt}\ sqrt\ {\isacharparenleft}{\kern0pt}m\ {\isacharasterisk}{\kern0pt}{\isacharparenleft}{\kern0pt}real{\isacharunderscore}{\kern0pt}of{\isacharunderscore}{\kern0pt}int\ a{\isacharplus}{\kern0pt}{\isadigit{1}}{\isacharparenright}{\kern0pt}{\isacharslash}{\kern0pt}p{\isacharparenright}{\kern0pt}{\isacharparenright}{\kern0pt}{\isachardoublequoteclose}\isanewline
\ \ \ \ \isacommand{proof}\isamarkupfalse%
\ {\isacharparenleft}{\kern0pt}rule\ prob{\isacharunderscore}{\kern0pt}space{\isachardot}{\kern0pt}prob{\isacharunderscore}{\kern0pt}mono{\isacharbrackleft}{\kern0pt}OF\ prob{\isacharunderscore}{\kern0pt}space{\isacharunderscore}{\kern0pt}measure{\isacharunderscore}{\kern0pt}pmf\ in{\isacharunderscore}{\kern0pt}events{\isacharunderscore}{\kern0pt}pmf{\isacharbrackright}{\kern0pt}{\isacharparenright}{\kern0pt}\isanewline
\ \ \ \ \ \ \isacommand{fix}\isamarkupfalse%
\ {\isasymomega}\isanewline
\ \ \ \ \ \ \isacommand{assume}\isamarkupfalse%
\ {\isachardoublequoteopen}{\isasymomega}\ {\isasymin}\ space\ {\isacharparenleft}{\kern0pt}measure{\isacharunderscore}{\kern0pt}pmf\ {\isasymOmega}\isactrlsub {\isadigit{1}}{\isacharparenright}{\kern0pt}{\isachardoublequoteclose}\isanewline
\ \ \ \ \ \ \isacommand{assume}\isamarkupfalse%
\ t{\isacharunderscore}{\kern0pt}le{\isacharcolon}{\kern0pt}\ {\isachardoublequoteopen}t\ {\isasymle}\ f\ a\ {\isasymomega}{\isachardoublequoteclose}\isanewline
\ \ \ \ \ \ \isacommand{have}\isamarkupfalse%
\ {\isachardoublequoteopen}real\ m\ {\isacharasterisk}{\kern0pt}\ {\isacharparenleft}{\kern0pt}of{\isacharunderscore}{\kern0pt}int\ a\ {\isacharplus}{\kern0pt}\ {\isadigit{1}}{\isacharparenright}{\kern0pt}\ {\isacharslash}{\kern0pt}\ p\ {\isacharequal}{\kern0pt}\ real\ m\ {\isacharasterisk}{\kern0pt}\ {\isacharparenleft}{\kern0pt}of{\isacharunderscore}{\kern0pt}int\ a{\isacharparenright}{\kern0pt}\ {\isacharslash}{\kern0pt}\ p\ {\isacharplus}{\kern0pt}\ real\ m\ {\isacharslash}{\kern0pt}\ p{\isachardoublequoteclose}\isanewline
\ \ \ \ \ \ \ \ \isacommand{by}\isamarkupfalse%
\ {\isacharparenleft}{\kern0pt}simp\ add{\isacharcolon}{\kern0pt}algebra{\isacharunderscore}{\kern0pt}simps\ add{\isacharunderscore}{\kern0pt}divide{\isacharunderscore}{\kern0pt}distrib{\isacharparenright}{\kern0pt}\isanewline
\ \ \ \ \ \ \isacommand{also}\isamarkupfalse%
\ \isacommand{have}\isamarkupfalse%
\ {\isachardoublequoteopen}{\isachardot}{\kern0pt}{\isachardot}{\kern0pt}{\isachardot}{\kern0pt}\ {\isasymle}\ \ real\ m\ {\isacharasterisk}{\kern0pt}\ {\isacharparenleft}{\kern0pt}real\ t\ {\isacharasterisk}{\kern0pt}\ real\ p\ {\isacharslash}{\kern0pt}\ {\isacharparenleft}{\kern0pt}real\ m\ {\isacharasterisk}{\kern0pt}\ {\isacharparenleft}{\kern0pt}{\isadigit{1}}{\isacharplus}{\kern0pt}{\isasymdelta}{\isacharprime}{\kern0pt}{\isacharparenright}{\kern0pt}{\isacharparenright}{\kern0pt}{\isacharparenright}{\kern0pt}\ {\isacharslash}{\kern0pt}\ real\ p\ {\isacharplus}{\kern0pt}\ {\isadigit{1}}{\isachardoublequoteclose}\isanewline
\ \ \ \ \ \ \ \ \isacommand{apply}\isamarkupfalse%
\ {\isacharparenleft}{\kern0pt}rule\ add{\isacharunderscore}{\kern0pt}mono{\isacharparenright}{\kern0pt}\isanewline
\ \ \ \ \ \ \ \ \ \isacommand{apply}\isamarkupfalse%
\ {\isacharparenleft}{\kern0pt}rule\ divide{\isacharunderscore}{\kern0pt}right{\isacharunderscore}{\kern0pt}mono{\isacharparenright}{\kern0pt}\isanewline
\ \ \ \ \ \ \ \ \ \ \isacommand{apply}\isamarkupfalse%
\ {\isacharparenleft}{\kern0pt}rule\ mult{\isacharunderscore}{\kern0pt}mono{\isacharcomma}{\kern0pt}\ simp{\isacharcomma}{\kern0pt}\ simp\ add{\isacharcolon}{\kern0pt}a{\isacharunderscore}{\kern0pt}def{\isacharcomma}{\kern0pt}\ simp{\isacharcomma}{\kern0pt}\ simp\ add{\isacharcolon}{\kern0pt}a{\isacharunderscore}{\kern0pt}ge{\isacharunderscore}{\kern0pt}{\isadigit{0}}{\isacharparenright}{\kern0pt}\isanewline
\ \ \ \ \ \ \ \ \ \isacommand{apply}\isamarkupfalse%
\ {\isacharparenleft}{\kern0pt}simp{\isacharparenright}{\kern0pt}\isanewline
\ \ \ \ \ \ \ \ \isacommand{using}\isamarkupfalse%
\ m{\isacharunderscore}{\kern0pt}le{\isacharunderscore}{\kern0pt}p\ \isacommand{by}\isamarkupfalse%
\ {\isacharparenleft}{\kern0pt}simp\ add{\isacharcolon}{\kern0pt}\ p{\isacharunderscore}{\kern0pt}ge{\isacharunderscore}{\kern0pt}{\isadigit{0}}{\isacharparenright}{\kern0pt}\isanewline
\ \ \ \ \ \ \isacommand{also}\isamarkupfalse%
\ \isacommand{have}\isamarkupfalse%
\ {\isachardoublequoteopen}{\isachardot}{\kern0pt}{\isachardot}{\kern0pt}{\isachardot}{\kern0pt}\ {\isasymle}\ real\ t\ {\isacharslash}{\kern0pt}\ {\isacharparenleft}{\kern0pt}{\isadigit{1}}{\isacharplus}{\kern0pt}{\isasymdelta}{\isacharprime}{\kern0pt}{\isacharparenright}{\kern0pt}\ {\isacharplus}{\kern0pt}\ {\isadigit{1}}{\isachardoublequoteclose}\isanewline
\ \ \ \ \ \ \ \ \isacommand{apply}\isamarkupfalse%
\ {\isacharparenleft}{\kern0pt}rule\ add{\isacharunderscore}{\kern0pt}mono{\isacharparenright}{\kern0pt}\isanewline
\ \ \ \ \ \ \ \ \ \isacommand{apply}\isamarkupfalse%
\ {\isacharparenleft}{\kern0pt}subst\ pos{\isacharunderscore}{\kern0pt}le{\isacharunderscore}{\kern0pt}divide{\isacharunderscore}{\kern0pt}eq{\isacharparenright}{\kern0pt}\ \isacommand{using}\isamarkupfalse%
\ {\isasymdelta}{\isacharprime}{\kern0pt}{\isacharunderscore}{\kern0pt}ge{\isacharunderscore}{\kern0pt}{\isadigit{0}}\ \isacommand{apply}\isamarkupfalse%
\ simp\isanewline
\ \ \ \ \ \ \ \ \isacommand{by}\isamarkupfalse%
\ simp{\isacharplus}{\kern0pt}\isanewline
\ \ \ \ \ \ \isacommand{finally}\isamarkupfalse%
\ \isacommand{have}\isamarkupfalse%
\ a{\isacharunderscore}{\kern0pt}le{\isacharunderscore}{\kern0pt}{\isadigit{1}}{\isacharcolon}{\kern0pt}\ {\isachardoublequoteopen}real\ m\ {\isacharasterisk}{\kern0pt}\ {\isacharparenleft}{\kern0pt}of{\isacharunderscore}{\kern0pt}int\ a\ {\isacharplus}{\kern0pt}\ {\isadigit{1}}{\isacharparenright}{\kern0pt}\ {\isacharslash}{\kern0pt}\ p\ {\isasymle}\ t\ {\isacharslash}{\kern0pt}\ {\isacharparenleft}{\kern0pt}{\isadigit{1}}{\isacharplus}{\kern0pt}\ {\isasymdelta}{\isacharprime}{\kern0pt}{\isacharparenright}{\kern0pt}\ {\isacharplus}{\kern0pt}\ {\isadigit{1}}{\isachardoublequoteclose}\isanewline
\ \ \ \ \ \ \ \ \isacommand{by}\isamarkupfalse%
\ simp\isanewline
\ \ \ \ \ \ \isacommand{have}\isamarkupfalse%
\ a{\isacharunderscore}{\kern0pt}le{\isacharcolon}{\kern0pt}\ {\isachardoublequoteopen}{\isadigit{3}}\ {\isacharasterisk}{\kern0pt}\ sqrt\ {\isacharparenleft}{\kern0pt}real\ m\ {\isacharasterisk}{\kern0pt}\ {\isacharparenleft}{\kern0pt}of{\isacharunderscore}{\kern0pt}int\ a\ {\isacharplus}{\kern0pt}\ {\isadigit{1}}{\isacharparenright}{\kern0pt}\ {\isacharslash}{\kern0pt}\ real\ p{\isacharparenright}{\kern0pt}\ {\isacharplus}{\kern0pt}\ real\ m\ {\isacharasterisk}{\kern0pt}\ {\isacharparenleft}{\kern0pt}of{\isacharunderscore}{\kern0pt}int\ a\ {\isacharplus}{\kern0pt}\ {\isadigit{1}}{\isacharparenright}{\kern0pt}\ {\isacharslash}{\kern0pt}\ real\ p\ {\isasymle}\ \isanewline
\ \ \ \ \ \ \ \ {\isadigit{3}}\ {\isacharasterisk}{\kern0pt}\ sqrt\ {\isacharparenleft}{\kern0pt}t\ {\isacharslash}{\kern0pt}\ {\isacharparenleft}{\kern0pt}{\isadigit{1}}{\isacharplus}{\kern0pt}{\isasymdelta}{\isacharprime}{\kern0pt}{\isacharparenright}{\kern0pt}\ {\isacharplus}{\kern0pt}\ {\isadigit{1}}{\isacharparenright}{\kern0pt}\ {\isacharplus}{\kern0pt}\ {\isacharparenleft}{\kern0pt}t\ {\isacharslash}{\kern0pt}\ {\isacharparenleft}{\kern0pt}{\isadigit{1}}{\isacharplus}{\kern0pt}{\isasymdelta}{\isacharprime}{\kern0pt}{\isacharparenright}{\kern0pt}\ {\isacharplus}{\kern0pt}\ {\isadigit{1}}{\isacharparenright}{\kern0pt}{\isachardoublequoteclose}\isanewline
\ \ \ \ \ \ \ \ \isacommand{apply}\isamarkupfalse%
\ {\isacharparenleft}{\kern0pt}rule\ add{\isacharunderscore}{\kern0pt}mono{\isacharparenright}{\kern0pt}\isanewline
\ \ \ \ \ \ \ \ \ \isacommand{apply}\isamarkupfalse%
\ {\isacharparenleft}{\kern0pt}rule\ mult{\isacharunderscore}{\kern0pt}left{\isacharunderscore}{\kern0pt}mono{\isacharparenright}{\kern0pt}\isanewline
\ \ \ \ \ \ \ \ \ \ \isacommand{apply}\isamarkupfalse%
\ {\isacharparenleft}{\kern0pt}subst\ real{\isacharunderscore}{\kern0pt}sqrt{\isacharunderscore}{\kern0pt}le{\isacharunderscore}{\kern0pt}iff{\isacharcomma}{\kern0pt}\ simp\ add{\isacharcolon}{\kern0pt}a{\isacharunderscore}{\kern0pt}le{\isacharunderscore}{\kern0pt}{\isadigit{1}}{\isacharparenright}{\kern0pt}\isanewline
\ \ \ \ \ \ \ \ \ \isacommand{apply}\isamarkupfalse%
\ simp\isanewline
\ \ \ \ \ \ \ \ \isacommand{by}\isamarkupfalse%
\ {\isacharparenleft}{\kern0pt}simp\ add{\isacharcolon}{\kern0pt}a{\isacharunderscore}{\kern0pt}le{\isacharunderscore}{\kern0pt}{\isadigit{1}}{\isacharparenright}{\kern0pt}\isanewline
\ \ \ \ \ \ \isacommand{also}\isamarkupfalse%
\ \isacommand{have}\isamarkupfalse%
\ {\isachardoublequoteopen}{\isachardot}{\kern0pt}{\isachardot}{\kern0pt}{\isachardot}{\kern0pt}\ {\isasymle}\ {\isadigit{3}}\ {\isacharasterisk}{\kern0pt}\ sqrt\ {\isacharparenleft}{\kern0pt}real\ t{\isacharplus}{\kern0pt}{\isadigit{1}}{\isacharparenright}{\kern0pt}\ {\isacharplus}{\kern0pt}\ {\isacharparenleft}{\kern0pt}{\isacharparenleft}{\kern0pt}t\ {\isacharminus}{\kern0pt}\ {\isasymdelta}{\isacharprime}{\kern0pt}\ {\isacharasterisk}{\kern0pt}\ t\ {\isacharslash}{\kern0pt}\ {\isacharparenleft}{\kern0pt}{\isadigit{1}}{\isacharplus}{\kern0pt}{\isasymdelta}{\isacharprime}{\kern0pt}{\isacharparenright}{\kern0pt}{\isacharparenright}{\kern0pt}\ {\isacharplus}{\kern0pt}\ {\isadigit{1}}{\isacharparenright}{\kern0pt}{\isachardoublequoteclose}\isanewline
\ \ \ \ \ \ \ \ \isacommand{apply}\isamarkupfalse%
\ {\isacharparenleft}{\kern0pt}rule\ add{\isacharunderscore}{\kern0pt}mono{\isacharparenright}{\kern0pt}\isanewline
\ \ \ \ \ \ \ \ \ \isacommand{apply}\isamarkupfalse%
\ {\isacharparenleft}{\kern0pt}rule\ mult{\isacharunderscore}{\kern0pt}left{\isacharunderscore}{\kern0pt}mono{\isacharparenright}{\kern0pt}\isanewline
\ \ \ \ \ \ \ \ \ \ \isacommand{apply}\isamarkupfalse%
\ {\isacharparenleft}{\kern0pt}subst\ real{\isacharunderscore}{\kern0pt}sqrt{\isacharunderscore}{\kern0pt}le{\isacharunderscore}{\kern0pt}iff{\isacharcomma}{\kern0pt}\ simp{\isacharparenright}{\kern0pt}\isanewline
\ \ \ \ \ \ \ \ \ \ \isacommand{apply}\isamarkupfalse%
\ {\isacharparenleft}{\kern0pt}subst\ pos{\isacharunderscore}{\kern0pt}divide{\isacharunderscore}{\kern0pt}le{\isacharunderscore}{\kern0pt}eq{\isacharparenright}{\kern0pt}\ \isacommand{using}\isamarkupfalse%
\ {\isasymdelta}{\isacharprime}{\kern0pt}{\isacharunderscore}{\kern0pt}ge{\isacharunderscore}{\kern0pt}{\isadigit{0}}\ \isacommand{apply}\isamarkupfalse%
\ simp\isanewline
\ \ \ \ \ \ \ \ \ \ \isacommand{using}\isamarkupfalse%
\ {\isasymdelta}{\isacharprime}{\kern0pt}{\isacharunderscore}{\kern0pt}ge{\isacharunderscore}{\kern0pt}{\isadigit{0}}\ \isacommand{apply}\isamarkupfalse%
\ {\isacharparenleft}{\kern0pt}simp\ add{\isacharcolon}{\kern0pt}\ t{\isacharunderscore}{\kern0pt}ge{\isacharunderscore}{\kern0pt}{\isadigit{0}}{\isacharparenright}{\kern0pt}\isanewline
\ \ \ \ \ \ \ \ \ \isacommand{apply}\isamarkupfalse%
\ simp\isanewline
\ \ \ \ \ \ \ \ \isacommand{apply}\isamarkupfalse%
\ {\isacharparenleft}{\kern0pt}rule\ add{\isacharunderscore}{\kern0pt}mono{\isacharparenright}{\kern0pt}\isanewline
\ \ \ \ \ \ \ \ \ \isacommand{apply}\isamarkupfalse%
\ {\isacharparenleft}{\kern0pt}subst\ pos{\isacharunderscore}{\kern0pt}divide{\isacharunderscore}{\kern0pt}le{\isacharunderscore}{\kern0pt}eq{\isacharparenright}{\kern0pt}\ \isacommand{using}\isamarkupfalse%
\ {\isasymdelta}{\isacharprime}{\kern0pt}{\isacharunderscore}{\kern0pt}ge{\isacharunderscore}{\kern0pt}{\isadigit{0}}\ \isacommand{apply}\isamarkupfalse%
\ simp\isanewline
\ \ \ \ \ \ \ \ \ \isacommand{apply}\isamarkupfalse%
\ {\isacharparenleft}{\kern0pt}subst\ left{\isacharunderscore}{\kern0pt}diff{\isacharunderscore}{\kern0pt}distrib{\isacharcomma}{\kern0pt}\ simp{\isacharcomma}{\kern0pt}\ simp\ add{\isacharcolon}{\kern0pt}algebra{\isacharunderscore}{\kern0pt}simps{\isacharparenright}{\kern0pt}\isanewline
\ \ \ \ \ \ \ \ \isacommand{using}\isamarkupfalse%
\ {\isasymdelta}{\isacharprime}{\kern0pt}{\isacharunderscore}{\kern0pt}ge{\isacharunderscore}{\kern0pt}{\isadigit{0}}\ \isacommand{by}\isamarkupfalse%
\ simp{\isacharplus}{\kern0pt}\isanewline
\ \ \ \ \ \ \isacommand{also}\isamarkupfalse%
\ \isacommand{have}\isamarkupfalse%
\ {\isachardoublequoteopen}{\isachardot}{\kern0pt}{\isachardot}{\kern0pt}{\isachardot}{\kern0pt}\ {\isasymle}\ {\isadigit{3}}\ {\isacharasterisk}{\kern0pt}\ sqrt\ {\isacharparenleft}{\kern0pt}{\isadigit{4}}{\isadigit{6}}\ {\isacharslash}{\kern0pt}\ {\isasymdelta}{\isacharprime}{\kern0pt}\isactrlsup {\isadigit{2}}\ {\isacharplus}{\kern0pt}\ {\isadigit{1}}\ {\isacharslash}{\kern0pt}\ {\isasymdelta}{\isacharprime}{\kern0pt}\isactrlsup {\isadigit{2}}{\isacharparenright}{\kern0pt}\ {\isacharplus}{\kern0pt}\ {\isacharparenleft}{\kern0pt}t\ {\isacharminus}{\kern0pt}\ {\isasymdelta}{\isacharprime}{\kern0pt}\ {\isacharasterisk}{\kern0pt}\ t{\isacharslash}{\kern0pt}{\isadigit{2}}{\isacharparenright}{\kern0pt}\ {\isacharplus}{\kern0pt}\ {\isadigit{1}}\ {\isacharslash}{\kern0pt}\ {\isasymdelta}{\isacharprime}{\kern0pt}{\isachardoublequoteclose}\isanewline
\ \ \ \ \ \ \ \ \isacommand{apply}\isamarkupfalse%
\ {\isacharparenleft}{\kern0pt}subst\ add{\isachardot}{\kern0pt}assoc{\isacharbrackleft}{\kern0pt}symmetric{\isacharbrackright}{\kern0pt}{\isacharparenright}{\kern0pt}\isanewline
\ \ \ \ \ \ \ \ \isacommand{apply}\isamarkupfalse%
\ {\isacharparenleft}{\kern0pt}rule\ add{\isacharunderscore}{\kern0pt}mono{\isacharparenright}{\kern0pt}\isanewline
\ \ \ \ \ \ \ \ \ \isacommand{apply}\isamarkupfalse%
\ {\isacharparenleft}{\kern0pt}rule\ add{\isacharunderscore}{\kern0pt}mono{\isacharparenright}{\kern0pt}\isanewline
\ \ \ \ \ \ \ \ \ \ \isacommand{apply}\isamarkupfalse%
\ {\isacharparenleft}{\kern0pt}rule\ mult{\isacharunderscore}{\kern0pt}left{\isacharunderscore}{\kern0pt}mono{\isacharparenright}{\kern0pt}\isanewline
\ \ \ \ \ \ \ \ \ \ \ \isacommand{apply}\isamarkupfalse%
\ {\isacharparenleft}{\kern0pt}subst\ real{\isacharunderscore}{\kern0pt}sqrt{\isacharunderscore}{\kern0pt}le{\isacharunderscore}{\kern0pt}iff{\isacharparenright}{\kern0pt}\isanewline
\ \ \ \ \ \ \ \ \ \ \ \isacommand{apply}\isamarkupfalse%
\ {\isacharparenleft}{\kern0pt}rule\ add{\isacharunderscore}{\kern0pt}mono{\isacharcomma}{\kern0pt}\ metis\ t{\isacharunderscore}{\kern0pt}le{\isacharunderscore}{\kern0pt}{\isasymdelta}{\isacharprime}{\kern0pt}{\isacharparenright}{\kern0pt}\isanewline
\ \ \ \ \ \ \ \ \ \ \ \isacommand{apply}\isamarkupfalse%
\ {\isacharparenleft}{\kern0pt}subst\ pos{\isacharunderscore}{\kern0pt}le{\isacharunderscore}{\kern0pt}divide{\isacharunderscore}{\kern0pt}eq{\isacharparenright}{\kern0pt}\ \isacommand{using}\isamarkupfalse%
\ {\isasymdelta}{\isacharprime}{\kern0pt}{\isacharunderscore}{\kern0pt}ge{\isacharunderscore}{\kern0pt}{\isadigit{0}}\ \isacommand{apply}\isamarkupfalse%
\ simp\isanewline
\ \ \ \ \ \ \ \ \ \ \ \isacommand{apply}\isamarkupfalse%
\ {\isacharparenleft}{\kern0pt}metis\ \ {\isasymdelta}{\isacharprime}{\kern0pt}{\isacharunderscore}{\kern0pt}le{\isacharunderscore}{\kern0pt}{\isadigit{1}}\ {\isasymdelta}{\isacharprime}{\kern0pt}{\isacharunderscore}{\kern0pt}ge{\isacharunderscore}{\kern0pt}{\isadigit{0}}\ \ less{\isacharunderscore}{\kern0pt}eq{\isacharunderscore}{\kern0pt}real{\isacharunderscore}{\kern0pt}def\ mult{\isacharunderscore}{\kern0pt}{\isadigit{1}}\ power{\isacharunderscore}{\kern0pt}le{\isacharunderscore}{\kern0pt}one{\isacharparenright}{\kern0pt}\isanewline
\ \ \ \ \ \ \ \ \ \ \isacommand{apply}\isamarkupfalse%
\ simp\isanewline
\ \ \ \ \ \ \ \ \ \isacommand{apply}\isamarkupfalse%
\ simp\isanewline
\ \ \ \ \ \ \ \ \ \isacommand{apply}\isamarkupfalse%
\ {\isacharparenleft}{\kern0pt}subst\ pos{\isacharunderscore}{\kern0pt}le{\isacharunderscore}{\kern0pt}divide{\isacharunderscore}{\kern0pt}eq{\isacharparenright}{\kern0pt}\ \isacommand{using}\isamarkupfalse%
\ {\isasymdelta}{\isacharprime}{\kern0pt}{\isacharunderscore}{\kern0pt}ge{\isacharunderscore}{\kern0pt}{\isadigit{0}}\ \isacommand{apply}\isamarkupfalse%
\ simp\isanewline
\ \ \ \ \ \ \ \ \ \isacommand{using}\isamarkupfalse%
\ {\isasymdelta}{\isacharprime}{\kern0pt}{\isacharunderscore}{\kern0pt}le{\isacharunderscore}{\kern0pt}{\isadigit{1}}\ {\isasymdelta}{\isacharprime}{\kern0pt}{\isacharunderscore}{\kern0pt}ge{\isacharunderscore}{\kern0pt}{\isadigit{0}}\ \isanewline
\ \ \ \ \ \ \ \ \ \isacommand{apply}\isamarkupfalse%
\ {\isacharparenleft}{\kern0pt}metis\ add{\isacharunderscore}{\kern0pt}mono\ less{\isacharunderscore}{\kern0pt}eq{\isacharunderscore}{\kern0pt}real{\isacharunderscore}{\kern0pt}def\ mult{\isacharunderscore}{\kern0pt}eq{\isacharunderscore}{\kern0pt}{\isadigit{0}}{\isacharunderscore}{\kern0pt}iff\ mult{\isacharunderscore}{\kern0pt}left{\isacharunderscore}{\kern0pt}mono\ of{\isacharunderscore}{\kern0pt}nat{\isacharunderscore}{\kern0pt}{\isadigit{0}}{\isacharunderscore}{\kern0pt}le{\isacharunderscore}{\kern0pt}iff\ one{\isacharunderscore}{\kern0pt}add{\isacharunderscore}{\kern0pt}one{\isacharparenright}{\kern0pt}\isanewline
\ \ \ \ \ \ \ \ \isacommand{using}\isamarkupfalse%
\ {\isasymdelta}{\isacharprime}{\kern0pt}{\isacharunderscore}{\kern0pt}le{\isacharunderscore}{\kern0pt}{\isadigit{1}}\ {\isasymdelta}{\isacharprime}{\kern0pt}{\isacharunderscore}{\kern0pt}ge{\isacharunderscore}{\kern0pt}{\isadigit{0}}\ \isacommand{by}\isamarkupfalse%
\ simp\isanewline
\ \ \ \ \ \ \isacommand{also}\isamarkupfalse%
\ \isacommand{have}\isamarkupfalse%
\ {\isachardoublequoteopen}{\isachardot}{\kern0pt}{\isachardot}{\kern0pt}{\isachardot}{\kern0pt}\ {\isasymle}\ {\isacharparenleft}{\kern0pt}{\isadigit{2}}{\isadigit{1}}\ {\isacharslash}{\kern0pt}\ {\isasymdelta}{\isacharprime}{\kern0pt}\ {\isacharplus}{\kern0pt}\ {\isacharparenleft}{\kern0pt}t\ {\isacharminus}{\kern0pt}\ {\isadigit{4}}{\isadigit{5}}\ {\isacharslash}{\kern0pt}\ {\isacharparenleft}{\kern0pt}{\isadigit{2}}{\isacharasterisk}{\kern0pt}{\isasymdelta}{\isacharprime}{\kern0pt}{\isacharparenright}{\kern0pt}{\isacharparenright}{\kern0pt}{\isacharparenright}{\kern0pt}\ {\isacharplus}{\kern0pt}\ {\isadigit{1}}\ {\isacharslash}{\kern0pt}\ {\isasymdelta}{\isacharprime}{\kern0pt}{\isachardoublequoteclose}\ \isanewline
\ \ \ \ \ \ \ \ \isacommand{apply}\isamarkupfalse%
\ {\isacharparenleft}{\kern0pt}rule\ add{\isacharunderscore}{\kern0pt}mono{\isacharparenright}{\kern0pt}\isanewline
\ \ \ \ \ \ \ \ \ \isacommand{apply}\isamarkupfalse%
\ {\isacharparenleft}{\kern0pt}rule\ add{\isacharunderscore}{\kern0pt}mono{\isacharparenright}{\kern0pt}\isanewline
\ \ \ \ \ \ \ \ \ \ \isacommand{apply}\isamarkupfalse%
\ {\isacharparenleft}{\kern0pt}simp\ add{\isacharcolon}{\kern0pt}real{\isacharunderscore}{\kern0pt}sqrt{\isacharunderscore}{\kern0pt}divide{\isacharcomma}{\kern0pt}\ subst\ abs{\isacharunderscore}{\kern0pt}of{\isacharunderscore}{\kern0pt}nonneg{\isacharparenright}{\kern0pt}\ \isacommand{using}\isamarkupfalse%
\ {\isasymdelta}{\isacharprime}{\kern0pt}{\isacharunderscore}{\kern0pt}ge{\isacharunderscore}{\kern0pt}{\isadigit{0}}\ \isacommand{apply}\isamarkupfalse%
\ linarith\isanewline
\ \ \ \ \ \ \ \ \isacommand{using}\isamarkupfalse%
\ {\isasymdelta}{\isacharprime}{\kern0pt}{\isacharunderscore}{\kern0pt}ge{\isacharunderscore}{\kern0pt}{\isadigit{0}}\ \isacommand{apply}\isamarkupfalse%
\ {\isacharparenleft}{\kern0pt}simp\ add{\isacharcolon}{\kern0pt}\ divide{\isacharunderscore}{\kern0pt}le{\isacharunderscore}{\kern0pt}cancel{\isacharparenright}{\kern0pt}\isanewline
\ \ \ \ \ \ \ \ \ \ \isacommand{apply}\isamarkupfalse%
\ {\isacharparenleft}{\kern0pt}rule\ real{\isacharunderscore}{\kern0pt}le{\isacharunderscore}{\kern0pt}lsqrt{\isacharcomma}{\kern0pt}\ simp{\isacharcomma}{\kern0pt}\ simp{\isacharcomma}{\kern0pt}\ simp{\isacharparenright}{\kern0pt}\isanewline
\ \ \ \ \ \ \ \ \ \isacommand{apply}\isamarkupfalse%
\ simp\isanewline
\ \ \ \ \ \ \ \ \ \isacommand{apply}\isamarkupfalse%
\ {\isacharparenleft}{\kern0pt}metis\ {\isasymdelta}{\isacharprime}{\kern0pt}{\isacharunderscore}{\kern0pt}ge{\isacharunderscore}{\kern0pt}{\isadigit{0}}\ t{\isacharunderscore}{\kern0pt}ge{\isacharunderscore}{\kern0pt}{\isasymdelta}{\isacharprime}{\kern0pt}\ less{\isacharunderscore}{\kern0pt}eq{\isacharunderscore}{\kern0pt}real{\isacharunderscore}{\kern0pt}def\ mult{\isacharunderscore}{\kern0pt}left{\isacharunderscore}{\kern0pt}mono\ power{\isadigit{2}}{\isacharunderscore}{\kern0pt}eq{\isacharunderscore}{\kern0pt}square\ real{\isacharunderscore}{\kern0pt}divide{\isacharunderscore}{\kern0pt}square{\isacharunderscore}{\kern0pt}eq\ times{\isacharunderscore}{\kern0pt}divide{\isacharunderscore}{\kern0pt}eq{\isacharunderscore}{\kern0pt}right{\isacharparenright}{\kern0pt}\isanewline
\ \ \ \ \ \ \ \ \isacommand{by}\isamarkupfalse%
\ simp\isanewline
\ \ \ \ \ \ \isacommand{also}\isamarkupfalse%
\ \isacommand{have}\isamarkupfalse%
\ {\isachardoublequoteopen}{\isachardot}{\kern0pt}{\isachardot}{\kern0pt}{\isachardot}{\kern0pt}\ {\isasymle}\ t{\isachardoublequoteclose}\ \isacommand{using}\isamarkupfalse%
\ {\isasymdelta}{\isacharprime}{\kern0pt}{\isacharunderscore}{\kern0pt}ge{\isacharunderscore}{\kern0pt}{\isadigit{0}}\ \isacommand{by}\isamarkupfalse%
\ simp\isanewline
\ \ \ \ \ \ \isacommand{also}\isamarkupfalse%
\ \isacommand{have}\isamarkupfalse%
\ {\isachardoublequoteopen}{\isachardot}{\kern0pt}{\isachardot}{\kern0pt}{\isachardot}{\kern0pt}\ {\isasymle}\ f\ a\ {\isasymomega}{\isachardoublequoteclose}\ \isacommand{using}\isamarkupfalse%
\ t{\isacharunderscore}{\kern0pt}le\ \isacommand{by}\isamarkupfalse%
\ simp\isanewline
\ \ \ \ \ \ \isacommand{finally}\isamarkupfalse%
\ \isacommand{have}\isamarkupfalse%
\ t{\isacharunderscore}{\kern0pt}le{\isacharcolon}{\kern0pt}\ {\isachardoublequoteopen}{\isadigit{3}}\ {\isacharasterisk}{\kern0pt}\ sqrt\ {\isacharparenleft}{\kern0pt}real\ m\ {\isacharasterisk}{\kern0pt}\ {\isacharparenleft}{\kern0pt}of{\isacharunderscore}{\kern0pt}int\ a\ {\isacharplus}{\kern0pt}\ {\isadigit{1}}{\isacharparenright}{\kern0pt}\ {\isacharslash}{\kern0pt}\ real\ p{\isacharparenright}{\kern0pt}\ {\isasymle}\ f\ a\ {\isasymomega}\ {\isacharminus}{\kern0pt}\ real\ m\ {\isacharasterisk}{\kern0pt}\ {\isacharparenleft}{\kern0pt}of{\isacharunderscore}{\kern0pt}int\ a\ {\isacharplus}{\kern0pt}\ {\isadigit{1}}{\isacharparenright}{\kern0pt}\ {\isacharslash}{\kern0pt}\ real\ p{\isachardoublequoteclose}\isanewline
\ \ \ \ \ \ \ \ \isacommand{by}\isamarkupfalse%
\ {\isacharparenleft}{\kern0pt}simp\ add{\isacharcolon}{\kern0pt}algebra{\isacharunderscore}{\kern0pt}simps{\isacharparenright}{\kern0pt}\isanewline
\ \ \ \ \ \ \isacommand{show}\isamarkupfalse%
\ {\isachardoublequoteopen}\ {\isadigit{3}}\ {\isacharasterisk}{\kern0pt}\ sqrt\ {\isacharparenleft}{\kern0pt}real\ m\ {\isacharasterisk}{\kern0pt}\ {\isacharparenleft}{\kern0pt}real{\isacharunderscore}{\kern0pt}of{\isacharunderscore}{\kern0pt}int\ a\ {\isacharplus}{\kern0pt}\ {\isadigit{1}}{\isacharparenright}{\kern0pt}\ {\isacharslash}{\kern0pt}\ real\ p{\isacharparenright}{\kern0pt}\ {\isasymle}\ \isanewline
\ \ \ \ \ \ \ \ {\isasymbar}real\ {\isacharparenleft}{\kern0pt}f\ a\ {\isasymomega}{\isacharparenright}{\kern0pt}\ {\isacharminus}{\kern0pt}\ measure{\isacharunderscore}{\kern0pt}pmf{\isachardot}{\kern0pt}expectation\ {\isasymOmega}\isactrlsub {\isadigit{1}}\ {\isacharparenleft}{\kern0pt}{\isasymlambda}{\isasymomega}{\isachardot}{\kern0pt}\ real\ {\isacharparenleft}{\kern0pt}f\ a\ {\isasymomega}{\isacharparenright}{\kern0pt}{\isacharparenright}{\kern0pt}{\isasymbar}{\isachardoublequoteclose}\isanewline
\ \ \ \ \ \ \ \ \isacommand{apply}\isamarkupfalse%
\ {\isacharparenleft}{\kern0pt}subst\ exp{\isacharunderscore}{\kern0pt}f{\isacharparenright}{\kern0pt}\ \isacommand{using}\isamarkupfalse%
\ a{\isacharunderscore}{\kern0pt}ge{\isacharunderscore}{\kern0pt}{\isadigit{0}}\ a{\isacharunderscore}{\kern0pt}le{\isacharunderscore}{\kern0pt}p\ True\ \isacommand{apply}\isamarkupfalse%
\ {\isacharparenleft}{\kern0pt}simp{\isacharcomma}{\kern0pt}\ simp{\isacharparenright}{\kern0pt}\isanewline
\ \ \ \ \ \ \ \ \isacommand{apply}\isamarkupfalse%
\ {\isacharparenleft}{\kern0pt}subst\ abs{\isacharunderscore}{\kern0pt}ge{\isacharunderscore}{\kern0pt}iff{\isacharparenright}{\kern0pt}\isanewline
\ \ \ \ \ \ \ \ \isacommand{using}\isamarkupfalse%
\ t{\isacharunderscore}{\kern0pt}le\ \isacommand{by}\isamarkupfalse%
\ blast\isanewline
\ \ \ \ \isacommand{qed}\isamarkupfalse%
\isanewline
\ \ \ \ \isacommand{also}\isamarkupfalse%
\ \isacommand{have}\isamarkupfalse%
\ {\isachardoublequoteopen}{\isachardot}{\kern0pt}{\isachardot}{\kern0pt}{\isachardot}{\kern0pt}\ {\isasymle}\ prob{\isacharunderscore}{\kern0pt}space{\isachardot}{\kern0pt}variance\ {\isacharparenleft}{\kern0pt}measure{\isacharunderscore}{\kern0pt}pmf\ {\isasymOmega}\isactrlsub {\isadigit{1}}{\isacharparenright}{\kern0pt}\ {\isacharparenleft}{\kern0pt}{\isasymlambda}{\isasymomega}{\isachardot}{\kern0pt}\ real\ {\isacharparenleft}{\kern0pt}f\ a\ {\isasymomega}{\isacharparenright}{\kern0pt}{\isacharparenright}{\kern0pt}\ \isanewline
\ \ \ \ \ \ {\isacharslash}{\kern0pt}\ {\isacharparenleft}{\kern0pt}{\isadigit{3}}\ {\isacharasterisk}{\kern0pt}\ sqrt\ {\isacharparenleft}{\kern0pt}real\ m\ {\isacharasterisk}{\kern0pt}\ {\isacharparenleft}{\kern0pt}of{\isacharunderscore}{\kern0pt}int\ a\ {\isacharplus}{\kern0pt}\ {\isadigit{1}}{\isacharparenright}{\kern0pt}\ {\isacharslash}{\kern0pt}\ real\ p{\isacharparenright}{\kern0pt}{\isacharparenright}{\kern0pt}\isactrlsup {\isadigit{2}}{\isachardoublequoteclose}\isanewline
\ \ \ \ \ \ \isacommand{apply}\isamarkupfalse%
\ {\isacharparenleft}{\kern0pt}rule\ prob{\isacharunderscore}{\kern0pt}space{\isachardot}{\kern0pt}Chebyshev{\isacharunderscore}{\kern0pt}inequality{\isacharparenright}{\kern0pt}\isanewline
\ \ \ \ \ \ \ \ \ \isacommand{apply}\isamarkupfalse%
\ {\isacharparenleft}{\kern0pt}metis\ prob{\isacharunderscore}{\kern0pt}space{\isacharunderscore}{\kern0pt}measure{\isacharunderscore}{\kern0pt}pmf{\isacharparenright}{\kern0pt}\isanewline
\ \ \ \ \ \ \ \ \isacommand{apply}\isamarkupfalse%
\ simp\isanewline
\ \ \ \ \ \ \ \isacommand{apply}\isamarkupfalse%
\ {\isacharparenleft}{\kern0pt}rule\ integrable{\isacharunderscore}{\kern0pt}measure{\isacharunderscore}{\kern0pt}pmf{\isacharunderscore}{\kern0pt}finite{\isacharbrackleft}{\kern0pt}OF\ fin{\isacharunderscore}{\kern0pt}omega{\isacharunderscore}{\kern0pt}{\isadigit{1}}{\isacharbrackright}{\kern0pt}{\isacharparenright}{\kern0pt}\isanewline
\ \ \ \ \ \ \ \isacommand{apply}\isamarkupfalse%
\ simp\isanewline
\ \ \ \ \ \ \isacommand{using}\isamarkupfalse%
\ t{\isacharunderscore}{\kern0pt}ge{\isacharunderscore}{\kern0pt}{\isadigit{0}}\ a{\isacharunderscore}{\kern0pt}ge{\isacharunderscore}{\kern0pt}{\isadigit{0}}\ p{\isacharunderscore}{\kern0pt}ge{\isacharunderscore}{\kern0pt}{\isadigit{0}}\ m{\isacharunderscore}{\kern0pt}ge{\isacharunderscore}{\kern0pt}{\isadigit{0}}\ m{\isacharunderscore}{\kern0pt}eq{\isacharunderscore}{\kern0pt}F{\isacharunderscore}{\kern0pt}{\isadigit{0}}\ \isacommand{by}\isamarkupfalse%
\ auto\isanewline
\ \ \ \ \isacommand{also}\isamarkupfalse%
\ \isacommand{have}\isamarkupfalse%
\ {\isachardoublequoteopen}{\isachardot}{\kern0pt}{\isachardot}{\kern0pt}{\isachardot}{\kern0pt}\ {\isasymle}\ {\isadigit{1}}{\isacharslash}{\kern0pt}{\isadigit{9}}{\isachardoublequoteclose}\isanewline
\ \ \ \ \ \ \isacommand{apply}\isamarkupfalse%
\ {\isacharparenleft}{\kern0pt}subst\ pos{\isacharunderscore}{\kern0pt}divide{\isacharunderscore}{\kern0pt}le{\isacharunderscore}{\kern0pt}eq{\isacharparenright}{\kern0pt}\ \isacommand{using}\isamarkupfalse%
\ a{\isacharunderscore}{\kern0pt}ge{\isacharunderscore}{\kern0pt}{\isadigit{0}}\ p{\isacharunderscore}{\kern0pt}ge{\isacharunderscore}{\kern0pt}{\isadigit{0}}\ m{\isacharunderscore}{\kern0pt}ge{\isacharunderscore}{\kern0pt}{\isadigit{0}}\ m{\isacharunderscore}{\kern0pt}eq{\isacharunderscore}{\kern0pt}F{\isacharunderscore}{\kern0pt}{\isadigit{0}}\ \isacommand{apply}\isamarkupfalse%
\ force\isanewline
\ \ \ \ \ \ \isacommand{apply}\isamarkupfalse%
\ simp\isanewline
\ \ \ \ \ \ \isacommand{apply}\isamarkupfalse%
\ {\isacharparenleft}{\kern0pt}subst\ real{\isacharunderscore}{\kern0pt}sqrt{\isacharunderscore}{\kern0pt}pow{\isadigit{2}}{\isacharparenright}{\kern0pt}\ \isacommand{using}\isamarkupfalse%
\ a{\isacharunderscore}{\kern0pt}ge{\isacharunderscore}{\kern0pt}{\isadigit{0}}\ p{\isacharunderscore}{\kern0pt}ge{\isacharunderscore}{\kern0pt}{\isadigit{0}}\ m{\isacharunderscore}{\kern0pt}ge{\isacharunderscore}{\kern0pt}{\isadigit{0}}\ m{\isacharunderscore}{\kern0pt}eq{\isacharunderscore}{\kern0pt}F{\isacharunderscore}{\kern0pt}{\isadigit{0}}\ \isacommand{apply}\isamarkupfalse%
\ force\isanewline
\ \ \ \ \ \ \isacommand{apply}\isamarkupfalse%
\ {\isacharparenleft}{\kern0pt}rule\ var{\isacharunderscore}{\kern0pt}f{\isacharparenright}{\kern0pt}\ \isacommand{using}\isamarkupfalse%
\ a{\isacharunderscore}{\kern0pt}ge{\isacharunderscore}{\kern0pt}{\isadigit{0}}\ \isacommand{apply}\isamarkupfalse%
\ linarith\isanewline
\ \ \ \ \ \ \isacommand{using}\isamarkupfalse%
\ a{\isacharunderscore}{\kern0pt}le{\isacharunderscore}{\kern0pt}p\ \isacommand{by}\isamarkupfalse%
\ simp\isanewline
\ \ \ \ \isacommand{finally}\isamarkupfalse%
\ \isacommand{have}\isamarkupfalse%
\ case{\isacharunderscore}{\kern0pt}{\isadigit{1}}{\isacharcolon}{\kern0pt}\ {\isachardoublequoteopen}{\isasymP}{\isacharparenleft}{\kern0pt}{\isasymomega}\ in\ measure{\isacharunderscore}{\kern0pt}pmf\ {\isasymOmega}\isactrlsub {\isadigit{1}}{\isachardot}{\kern0pt}\ f\ a\ {\isasymomega}\ {\isasymge}\ t{\isacharparenright}{\kern0pt}\ {\isasymle}\ {\isadigit{1}}{\isacharslash}{\kern0pt}{\isadigit{9}}{\isachardoublequoteclose}\isanewline
\ \ \ \ \ \ \isacommand{by}\isamarkupfalse%
\ simp\isanewline
\isanewline
\ \ \ \ \isacommand{have}\isamarkupfalse%
\ case{\isacharunderscore}{\kern0pt}{\isadigit{2}}{\isacharcolon}{\kern0pt}\ {\isachardoublequoteopen}{\isasymP}{\isacharparenleft}{\kern0pt}{\isasymomega}\ in\ measure{\isacharunderscore}{\kern0pt}pmf\ {\isasymOmega}\isactrlsub {\isadigit{1}}{\isachardot}{\kern0pt}\ f\ b\ {\isasymomega}\ {\isacharless}{\kern0pt}\ t{\isacharparenright}{\kern0pt}\ {\isasymle}\ {\isadigit{1}}{\isacharslash}{\kern0pt}{\isadigit{9}}{\isachardoublequoteclose}\isanewline
\ \ \ \ \isacommand{proof}\isamarkupfalse%
\ {\isacharparenleft}{\kern0pt}cases\ {\isachardoublequoteopen}b\ {\isacharless}{\kern0pt}\ p{\isachardoublequoteclose}{\isacharparenright}{\kern0pt}\isanewline
\ \ \ \ \ \ \isacommand{case}\isamarkupfalse%
\ True\isanewline
\ \ \ \ \ \ \isacommand{have}\isamarkupfalse%
\ {\isachardoublequoteopen}{\isasymP}{\isacharparenleft}{\kern0pt}{\isasymomega}\ in\ measure{\isacharunderscore}{\kern0pt}pmf\ {\isasymOmega}\isactrlsub {\isadigit{1}}{\isachardot}{\kern0pt}\ f\ b\ {\isasymomega}\ {\isacharless}{\kern0pt}\ t{\isacharparenright}{\kern0pt}\ {\isasymle}\ \isanewline
\ \ \ \ \ \ \ \ {\isasymP}{\isacharparenleft}{\kern0pt}{\isasymomega}\ in\ measure{\isacharunderscore}{\kern0pt}pmf\ {\isasymOmega}\isactrlsub {\isadigit{1}}{\isachardot}{\kern0pt}\ abs\ {\isacharparenleft}{\kern0pt}real\ {\isacharparenleft}{\kern0pt}f\ b\ {\isasymomega}{\isacharparenright}{\kern0pt}\ {\isacharminus}{\kern0pt}\ prob{\isacharunderscore}{\kern0pt}space{\isachardot}{\kern0pt}expectation\ {\isacharparenleft}{\kern0pt}measure{\isacharunderscore}{\kern0pt}pmf\ {\isasymOmega}\isactrlsub {\isadigit{1}}{\isacharparenright}{\kern0pt}\ {\isacharparenleft}{\kern0pt}{\isasymlambda}{\isasymomega}{\isachardot}{\kern0pt}\ real\ {\isacharparenleft}{\kern0pt}f\ b\ {\isasymomega}{\isacharparenright}{\kern0pt}{\isacharparenright}{\kern0pt}{\isacharparenright}{\kern0pt}\ \isanewline
\ \ \ \ \ \ \ \ {\isasymge}\ {\isadigit{3}}\ {\isacharasterisk}{\kern0pt}\ sqrt\ {\isacharparenleft}{\kern0pt}m\ {\isacharasterisk}{\kern0pt}{\isacharparenleft}{\kern0pt}real{\isacharunderscore}{\kern0pt}of{\isacharunderscore}{\kern0pt}int\ b{\isacharplus}{\kern0pt}{\isadigit{1}}{\isacharparenright}{\kern0pt}{\isacharslash}{\kern0pt}p{\isacharparenright}{\kern0pt}{\isacharparenright}{\kern0pt}{\isachardoublequoteclose}\isanewline
\ \ \ \ \ \ \isacommand{proof}\isamarkupfalse%
\ {\isacharparenleft}{\kern0pt}rule\ prob{\isacharunderscore}{\kern0pt}space{\isachardot}{\kern0pt}prob{\isacharunderscore}{\kern0pt}mono{\isacharbrackleft}{\kern0pt}OF\ prob{\isacharunderscore}{\kern0pt}space{\isacharunderscore}{\kern0pt}measure{\isacharunderscore}{\kern0pt}pmf\ in{\isacharunderscore}{\kern0pt}events{\isacharunderscore}{\kern0pt}pmf{\isacharbrackright}{\kern0pt}{\isacharparenright}{\kern0pt}\isanewline
\ \ \ \ \ \ \ \ \isacommand{fix}\isamarkupfalse%
\ {\isasymomega}\isanewline
\ \ \ \ \ \ \ \ \isacommand{assume}\isamarkupfalse%
\ {\isachardoublequoteopen}{\isasymomega}\ {\isasymin}\ space\ {\isacharparenleft}{\kern0pt}measure{\isacharunderscore}{\kern0pt}pmf\ {\isasymOmega}\isactrlsub {\isadigit{1}}{\isacharparenright}{\kern0pt}{\isachardoublequoteclose}\isanewline
\ \ \ \ \ \ \ \ \isacommand{have}\isamarkupfalse%
\ aux{\isacharcolon}{\kern0pt}\ {\isachardoublequoteopen}{\isacharparenleft}{\kern0pt}real\ t\ {\isacharplus}{\kern0pt}\ {\isadigit{3}}\ {\isacharasterisk}{\kern0pt}\ sqrt\ {\isacharparenleft}{\kern0pt}real\ t\ {\isacharslash}{\kern0pt}\ {\isacharparenleft}{\kern0pt}{\isadigit{1}}\ {\isacharminus}{\kern0pt}\ {\isasymdelta}{\isacharprime}{\kern0pt}{\isacharparenright}{\kern0pt}\ {\isacharplus}{\kern0pt}\ {\isadigit{1}}{\isacharparenright}{\kern0pt}{\isacharparenright}{\kern0pt}\ {\isacharasterisk}{\kern0pt}\ {\isacharparenleft}{\kern0pt}{\isadigit{1}}\ {\isacharminus}{\kern0pt}\ {\isasymdelta}{\isacharprime}{\kern0pt}{\isacharparenright}{\kern0pt}\ {\isacharequal}{\kern0pt}\isanewline
\ \ \ \ \ \ \ \ \ \ \ real\ t\ {\isacharminus}{\kern0pt}\ {\isasymdelta}{\isacharprime}{\kern0pt}\ {\isacharasterisk}{\kern0pt}\ t\ {\isacharplus}{\kern0pt}\ {\isadigit{3}}\ {\isacharasterisk}{\kern0pt}\ {\isacharparenleft}{\kern0pt}{\isacharparenleft}{\kern0pt}{\isadigit{1}}{\isacharminus}{\kern0pt}{\isasymdelta}{\isacharprime}{\kern0pt}{\isacharparenright}{\kern0pt}\ {\isacharasterisk}{\kern0pt}\ sqrt{\isacharparenleft}{\kern0pt}\ real\ t\ {\isacharslash}{\kern0pt}\ {\isacharparenleft}{\kern0pt}{\isadigit{1}}{\isacharminus}{\kern0pt}{\isasymdelta}{\isacharprime}{\kern0pt}{\isacharparenright}{\kern0pt}\ {\isacharplus}{\kern0pt}\ {\isadigit{1}}{\isacharparenright}{\kern0pt}{\isacharparenright}{\kern0pt}{\isachardoublequoteclose}\isanewline
\ \ \ \ \ \ \ \ \ \ \isacommand{by}\isamarkupfalse%
\ {\isacharparenleft}{\kern0pt}simp\ add{\isacharcolon}{\kern0pt}algebra{\isacharunderscore}{\kern0pt}simps{\isacharparenright}{\kern0pt}\isanewline
\ \ \ \ \ \ \ \ \isacommand{also}\isamarkupfalse%
\ \isacommand{have}\isamarkupfalse%
\ {\isachardoublequoteopen}{\isachardot}{\kern0pt}{\isachardot}{\kern0pt}{\isachardot}{\kern0pt}\ {\isacharequal}{\kern0pt}\ real\ t\ {\isacharminus}{\kern0pt}\ {\isasymdelta}{\isacharprime}{\kern0pt}\ {\isacharasterisk}{\kern0pt}\ t\ {\isacharplus}{\kern0pt}\ {\isadigit{3}}\ {\isacharasterisk}{\kern0pt}\ sqrt\ {\isacharparenleft}{\kern0pt}\ \ {\isacharparenleft}{\kern0pt}{\isadigit{1}}{\isacharminus}{\kern0pt}{\isasymdelta}{\isacharprime}{\kern0pt}{\isacharparenright}{\kern0pt}\isactrlsup {\isadigit{2}}\ {\isacharasterisk}{\kern0pt}\ {\isacharparenleft}{\kern0pt}real\ t\ {\isacharslash}{\kern0pt}\ \ {\isacharparenleft}{\kern0pt}{\isadigit{1}}{\isacharminus}{\kern0pt}{\isasymdelta}{\isacharprime}{\kern0pt}{\isacharparenright}{\kern0pt}\ {\isacharplus}{\kern0pt}\ \ {\isadigit{1}}{\isacharparenright}{\kern0pt}{\isacharparenright}{\kern0pt}{\isachardoublequoteclose}\isanewline
\ \ \ \ \ \ \ \ \ \ \isacommand{apply}\isamarkupfalse%
\ {\isacharparenleft}{\kern0pt}subst\ real{\isacharunderscore}{\kern0pt}sqrt{\isacharunderscore}{\kern0pt}mult{\isacharparenright}{\kern0pt}\isanewline
\ \ \ \ \ \ \ \ \ \ \isacommand{apply}\isamarkupfalse%
\ {\isacharparenleft}{\kern0pt}subst\ real{\isacharunderscore}{\kern0pt}sqrt{\isacharunderscore}{\kern0pt}abs{\isacharparenright}{\kern0pt}\isanewline
\ \ \ \ \ \ \ \ \ \ \isacommand{apply}\isamarkupfalse%
\ {\isacharparenleft}{\kern0pt}subst\ abs{\isacharunderscore}{\kern0pt}of{\isacharunderscore}{\kern0pt}nonneg{\isacharparenright}{\kern0pt}\isanewline
\ \ \ \ \ \ \ \ \ \ \isacommand{using}\isamarkupfalse%
\ {\isasymdelta}{\isacharprime}{\kern0pt}{\isacharunderscore}{\kern0pt}le{\isacharunderscore}{\kern0pt}{\isadigit{1}}\ \isacommand{by}\isamarkupfalse%
\ simp{\isacharplus}{\kern0pt}\isanewline
\ \ \ \ \ \ \ \ \isacommand{also}\isamarkupfalse%
\ \isacommand{have}\isamarkupfalse%
\ {\isachardoublequoteopen}{\isachardot}{\kern0pt}{\isachardot}{\kern0pt}{\isachardot}{\kern0pt}\ {\isacharequal}{\kern0pt}\ real\ t\ {\isacharminus}{\kern0pt}\ {\isasymdelta}{\isacharprime}{\kern0pt}\ {\isacharasterisk}{\kern0pt}\ t\ {\isacharplus}{\kern0pt}\ {\isadigit{3}}\ {\isacharasterisk}{\kern0pt}\ sqrt\ {\isacharparenleft}{\kern0pt}\ real\ t\ {\isacharasterisk}{\kern0pt}\ {\isacharparenleft}{\kern0pt}{\isadigit{1}}{\isacharminus}{\kern0pt}\ {\isasymdelta}{\isacharprime}{\kern0pt}{\isacharparenright}{\kern0pt}\ {\isacharplus}{\kern0pt}\ {\isacharparenleft}{\kern0pt}{\isadigit{1}}{\isacharminus}{\kern0pt}{\isasymdelta}{\isacharprime}{\kern0pt}{\isacharparenright}{\kern0pt}\isactrlsup {\isadigit{2}}{\isacharparenright}{\kern0pt}{\isachardoublequoteclose}\isanewline
\ \ \ \ \ \ \ \ \ \ \isacommand{by}\isamarkupfalse%
\ {\isacharparenleft}{\kern0pt}simp\ add{\isacharcolon}{\kern0pt}power{\isadigit{2}}{\isacharunderscore}{\kern0pt}eq{\isacharunderscore}{\kern0pt}square\ distrib{\isacharunderscore}{\kern0pt}left{\isacharparenright}{\kern0pt}\isanewline
\ \ \ \ \ \ \ \ \isacommand{also}\isamarkupfalse%
\ \isacommand{have}\isamarkupfalse%
\ {\isachardoublequoteopen}{\isachardot}{\kern0pt}{\isachardot}{\kern0pt}{\isachardot}{\kern0pt}\ {\isasymle}\ real\ t\ {\isacharminus}{\kern0pt}\ {\isadigit{4}}{\isadigit{5}}{\isacharslash}{\kern0pt}\ {\isasymdelta}{\isacharprime}{\kern0pt}\ {\isacharplus}{\kern0pt}\ {\isadigit{3}}\ {\isacharasterisk}{\kern0pt}\ sqrt\ {\isacharparenleft}{\kern0pt}\ real\ t\ \ {\isacharplus}{\kern0pt}\ {\isadigit{1}}{\isacharparenright}{\kern0pt}{\isachardoublequoteclose}\isanewline
\ \ \ \ \ \ \ \ \ \ \isacommand{apply}\isamarkupfalse%
\ {\isacharparenleft}{\kern0pt}rule\ add{\isacharunderscore}{\kern0pt}mono{\isacharcomma}{\kern0pt}\ simp{\isacharparenright}{\kern0pt}\isanewline
\ \ \ \ \ \ \ \ \ \ \ \isacommand{apply}\isamarkupfalse%
\ {\isacharparenleft}{\kern0pt}subst\ mult{\isachardot}{\kern0pt}commute{\isacharcomma}{\kern0pt}\ subst\ pos{\isacharunderscore}{\kern0pt}divide{\isacharunderscore}{\kern0pt}le{\isacharunderscore}{\kern0pt}eq{\isacharbrackleft}{\kern0pt}symmetric{\isacharbrackright}{\kern0pt}{\isacharparenright}{\kern0pt}\isanewline
\ \ \ \ \ \ \ \ \ \ \ \ \isacommand{using}\isamarkupfalse%
\ {\isasymdelta}{\isacharprime}{\kern0pt}{\isacharunderscore}{\kern0pt}ge{\isacharunderscore}{\kern0pt}{\isadigit{0}}\ \isacommand{apply}\isamarkupfalse%
\ blast\isanewline
\ \ \ \ \ \ \ \ \ \ \ \isacommand{using}\isamarkupfalse%
\ t{\isacharunderscore}{\kern0pt}ge{\isacharunderscore}{\kern0pt}{\isasymdelta}{\isacharprime}{\kern0pt}\ \isacommand{apply}\isamarkupfalse%
\ {\isacharparenleft}{\kern0pt}simp\ add{\isacharcolon}{\kern0pt}power{\isadigit{2}}{\isacharunderscore}{\kern0pt}eq{\isacharunderscore}{\kern0pt}square{\isacharparenright}{\kern0pt}\isanewline
\ \ \ \ \ \ \ \ \ \ \isacommand{apply}\isamarkupfalse%
\ simp\isanewline
\ \ \ \ \ \ \ \ \ \ \isacommand{apply}\isamarkupfalse%
\ {\isacharparenleft}{\kern0pt}rule\ add{\isacharunderscore}{\kern0pt}mono{\isacharparenright}{\kern0pt}\isanewline
\ \ \ \ \ \ \ \ \ \ \ \isacommand{using}\isamarkupfalse%
\ {\isasymdelta}{\isacharprime}{\kern0pt}{\isacharunderscore}{\kern0pt}le{\isacharunderscore}{\kern0pt}{\isadigit{1}}\ {\isasymdelta}{\isacharprime}{\kern0pt}{\isacharunderscore}{\kern0pt}ge{\isacharunderscore}{\kern0pt}{\isadigit{0}}\ \isacommand{by}\isamarkupfalse%
\ {\isacharparenleft}{\kern0pt}simp\ add{\isacharcolon}{\kern0pt}\ power{\isacharunderscore}{\kern0pt}le{\isacharunderscore}{\kern0pt}one\ t{\isacharunderscore}{\kern0pt}ge{\isacharunderscore}{\kern0pt}{\isadigit{0}}{\isacharparenright}{\kern0pt}{\isacharplus}{\kern0pt}\isanewline
\ \ \ \ \ \ \ \ \isacommand{also}\isamarkupfalse%
\ \isacommand{have}\isamarkupfalse%
\ {\isachardoublequoteopen}{\isachardot}{\kern0pt}{\isachardot}{\kern0pt}{\isachardot}{\kern0pt}\ {\isasymle}\ real\ t\ {\isacharminus}{\kern0pt}\ {\isadigit{4}}{\isadigit{5}}{\isacharslash}{\kern0pt}\ {\isasymdelta}{\isacharprime}{\kern0pt}\ {\isacharplus}{\kern0pt}\ {\isadigit{3}}\ {\isacharasterisk}{\kern0pt}\ sqrt\ {\isacharparenleft}{\kern0pt}\ {\isadigit{4}}{\isadigit{6}}\ {\isacharslash}{\kern0pt}\ {\isasymdelta}{\isacharprime}{\kern0pt}\isactrlsup {\isadigit{2}}\ {\isacharplus}{\kern0pt}\ {\isadigit{1}}\ {\isacharslash}{\kern0pt}\ {\isasymdelta}{\isacharprime}{\kern0pt}\isactrlsup {\isadigit{2}}{\isacharparenright}{\kern0pt}{\isachardoublequoteclose}\isanewline
\ \ \ \ \ \ \ \ \ \ \isacommand{apply}\isamarkupfalse%
\ {\isacharparenleft}{\kern0pt}rule\ add{\isacharunderscore}{\kern0pt}mono{\isacharcomma}{\kern0pt}\ simp{\isacharparenright}{\kern0pt}\isanewline
\ \ \ \ \ \ \ \ \ \ \isacommand{apply}\isamarkupfalse%
\ {\isacharparenleft}{\kern0pt}rule\ mult{\isacharunderscore}{\kern0pt}left{\isacharunderscore}{\kern0pt}mono{\isacharparenright}{\kern0pt}\isanewline
\ \ \ \ \ \ \ \ \ \ \ \isacommand{apply}\isamarkupfalse%
\ {\isacharparenleft}{\kern0pt}subst\ real{\isacharunderscore}{\kern0pt}sqrt{\isacharunderscore}{\kern0pt}le{\isacharunderscore}{\kern0pt}iff{\isacharparenright}{\kern0pt}\isanewline
\ \ \ \ \ \ \ \ \ \ \ \isacommand{apply}\isamarkupfalse%
\ {\isacharparenleft}{\kern0pt}rule\ add{\isacharunderscore}{\kern0pt}mono{\isacharcomma}{\kern0pt}\ metis\ t{\isacharunderscore}{\kern0pt}le{\isacharunderscore}{\kern0pt}{\isasymdelta}{\isacharprime}{\kern0pt}{\isacharparenright}{\kern0pt}\isanewline
\ \ \ \ \ \ \ \ \ \ \ \isacommand{apply}\isamarkupfalse%
\ {\isacharparenleft}{\kern0pt}meson\ {\isasymdelta}{\isacharprime}{\kern0pt}{\isacharunderscore}{\kern0pt}ge{\isacharunderscore}{\kern0pt}{\isadigit{0}}\ {\isasymdelta}{\isacharprime}{\kern0pt}{\isacharunderscore}{\kern0pt}le{\isacharunderscore}{\kern0pt}{\isadigit{1}}\ le{\isacharunderscore}{\kern0pt}divide{\isacharunderscore}{\kern0pt}eq{\isacharunderscore}{\kern0pt}{\isadigit{1}}{\isacharunderscore}{\kern0pt}pos\ less{\isacharunderscore}{\kern0pt}eq{\isacharunderscore}{\kern0pt}real{\isacharunderscore}{\kern0pt}def\ power{\isacharunderscore}{\kern0pt}le{\isacharunderscore}{\kern0pt}one{\isacharunderscore}{\kern0pt}iff\ zero{\isacharunderscore}{\kern0pt}less{\isacharunderscore}{\kern0pt}power{\isacharparenright}{\kern0pt}\isanewline
\ \ \ \ \ \ \ \ \ \ \isacommand{by}\isamarkupfalse%
\ simp\isanewline
\ \ \ \ \ \ \ \ \isacommand{also}\isamarkupfalse%
\ \isacommand{have}\isamarkupfalse%
\ {\isachardoublequoteopen}{\isachardot}{\kern0pt}{\isachardot}{\kern0pt}{\isachardot}{\kern0pt}\ {\isacharequal}{\kern0pt}\ real\ t\ {\isacharplus}{\kern0pt}\ {\isacharparenleft}{\kern0pt}{\isadigit{3}}\ {\isacharasterisk}{\kern0pt}\ sqrt{\isacharparenleft}{\kern0pt}{\isadigit{4}}{\isadigit{7}}{\isacharparenright}{\kern0pt}\ {\isacharminus}{\kern0pt}\ {\isadigit{4}}{\isadigit{5}}{\isacharparenright}{\kern0pt}{\isacharslash}{\kern0pt}\ {\isasymdelta}{\isacharprime}{\kern0pt}{\isachardoublequoteclose}\isanewline
\ \ \ \ \ \ \ \ \ \ \isacommand{apply}\isamarkupfalse%
\ {\isacharparenleft}{\kern0pt}simp\ add{\isacharcolon}{\kern0pt}real{\isacharunderscore}{\kern0pt}sqrt{\isacharunderscore}{\kern0pt}divide{\isacharparenright}{\kern0pt}\isanewline
\ \ \ \ \ \ \ \ \ \ \isacommand{apply}\isamarkupfalse%
\ {\isacharparenleft}{\kern0pt}subst\ abs{\isacharunderscore}{\kern0pt}of{\isacharunderscore}{\kern0pt}nonneg{\isacharparenright}{\kern0pt}\isanewline
\ \ \ \ \ \ \ \ \ \ \isacommand{using}\isamarkupfalse%
\ {\isasymdelta}{\isacharprime}{\kern0pt}{\isacharunderscore}{\kern0pt}ge{\isacharunderscore}{\kern0pt}{\isadigit{0}}\ \isacommand{by}\isamarkupfalse%
\ {\isacharparenleft}{\kern0pt}simp\ add{\isacharcolon}{\kern0pt}\ diff{\isacharunderscore}{\kern0pt}divide{\isacharunderscore}{\kern0pt}distrib{\isacharparenright}{\kern0pt}{\isacharplus}{\kern0pt}\isanewline
\ \ \ \ \ \ \ \ \isacommand{also}\isamarkupfalse%
\ \isacommand{have}\isamarkupfalse%
\ {\isachardoublequoteopen}{\isachardot}{\kern0pt}{\isachardot}{\kern0pt}{\isachardot}{\kern0pt}\ {\isasymle}\ t{\isachardoublequoteclose}\isanewline
\ \ \ \ \ \ \ \ \ \ \isacommand{apply}\isamarkupfalse%
\ simp\isanewline
\ \ \ \ \ \ \ \ \ \ \isacommand{apply}\isamarkupfalse%
\ {\isacharparenleft}{\kern0pt}subst\ pos{\isacharunderscore}{\kern0pt}divide{\isacharunderscore}{\kern0pt}le{\isacharunderscore}{\kern0pt}eq{\isacharparenright}{\kern0pt}\isanewline
\ \ \ \ \ \ \ \ \ \ \isacommand{using}\isamarkupfalse%
\ {\isasymdelta}{\isacharprime}{\kern0pt}{\isacharunderscore}{\kern0pt}ge{\isacharunderscore}{\kern0pt}{\isadigit{0}}\ \isacommand{apply}\isamarkupfalse%
\ simp\ \isanewline
\ \ \ \ \ \ \ \ \ \ \isacommand{apply}\isamarkupfalse%
\ simp\isanewline
\ \ \ \ \ \ \ \ \ \ \isacommand{by}\isamarkupfalse%
\ {\isacharparenleft}{\kern0pt}rule\ real{\isacharunderscore}{\kern0pt}le{\isacharunderscore}{\kern0pt}lsqrt{\isacharcomma}{\kern0pt}\ simp{\isacharplus}{\kern0pt}{\isacharparenright}{\kern0pt}\isanewline
\ \ \ \ \ \ \ \ \isacommand{finally}\isamarkupfalse%
\ \isacommand{have}\isamarkupfalse%
\ aux{\isacharcolon}{\kern0pt}\ {\isachardoublequoteopen}{\isacharparenleft}{\kern0pt}real\ t\ {\isacharplus}{\kern0pt}\ {\isadigit{3}}\ {\isacharasterisk}{\kern0pt}\ sqrt\ {\isacharparenleft}{\kern0pt}real\ t\ {\isacharslash}{\kern0pt}\ {\isacharparenleft}{\kern0pt}{\isadigit{1}}\ {\isacharminus}{\kern0pt}\ {\isasymdelta}{\isacharprime}{\kern0pt}{\isacharparenright}{\kern0pt}\ {\isacharplus}{\kern0pt}\ {\isadigit{1}}{\isacharparenright}{\kern0pt}{\isacharparenright}{\kern0pt}\ {\isacharasterisk}{\kern0pt}\ {\isacharparenleft}{\kern0pt}{\isadigit{1}}\ {\isacharminus}{\kern0pt}\ {\isasymdelta}{\isacharprime}{\kern0pt}{\isacharparenright}{\kern0pt}\ {\isasymle}\ real\ t\ {\isachardoublequoteclose}\isanewline
\ \ \ \ \ \ \ \ \ \ \isacommand{by}\isamarkupfalse%
\ simp\isanewline
\ \ \ \ \ \ \ \ \isacommand{assume}\isamarkupfalse%
\ t{\isacharunderscore}{\kern0pt}ge{\isacharcolon}{\kern0pt}\ {\isachardoublequoteopen}f\ b\ {\isasymomega}\ {\isacharless}{\kern0pt}\ t{\isachardoublequoteclose}\isanewline
\ \ \ \ \ \ \ \ \isacommand{have}\isamarkupfalse%
\ {\isachardoublequoteopen}real\ {\isacharparenleft}{\kern0pt}f\ b\ {\isasymomega}{\isacharparenright}{\kern0pt}\ {\isacharplus}{\kern0pt}\ {\isadigit{3}}\ {\isacharasterisk}{\kern0pt}\ sqrt\ {\isacharparenleft}{\kern0pt}real\ m\ {\isacharasterisk}{\kern0pt}\ {\isacharparenleft}{\kern0pt}real{\isacharunderscore}{\kern0pt}of{\isacharunderscore}{\kern0pt}int\ b\ {\isacharplus}{\kern0pt}\ {\isadigit{1}}{\isacharparenright}{\kern0pt}\ {\isacharslash}{\kern0pt}\ real\ p{\isacharparenright}{\kern0pt}\ \isanewline
\ \ \ \ \ \ \ \ \ \ {\isasymle}\ real\ t\ {\isacharplus}{\kern0pt}\ {\isadigit{3}}\ {\isacharasterisk}{\kern0pt}\ sqrt\ {\isacharparenleft}{\kern0pt}real\ m\ {\isacharasterisk}{\kern0pt}\ real{\isacharunderscore}{\kern0pt}of{\isacharunderscore}{\kern0pt}int\ b\ {\isacharslash}{\kern0pt}\ real\ p\ {\isacharplus}{\kern0pt}\ {\isadigit{1}}{\isacharparenright}{\kern0pt}{\isachardoublequoteclose}\isanewline
\ \ \ \ \ \ \ \ \ \ \isacommand{apply}\isamarkupfalse%
\ {\isacharparenleft}{\kern0pt}rule\ add{\isacharunderscore}{\kern0pt}mono{\isacharparenright}{\kern0pt}\isanewline
\ \ \ \ \ \ \ \ \ \ \isacommand{using}\isamarkupfalse%
\ t{\isacharunderscore}{\kern0pt}ge\ \isacommand{apply}\isamarkupfalse%
\ linarith\isanewline
\ \ \ \ \ \ \ \ \ \ \isacommand{using}\isamarkupfalse%
\ m{\isacharunderscore}{\kern0pt}le{\isacharunderscore}{\kern0pt}p\ \isacommand{by}\isamarkupfalse%
\ {\isacharparenleft}{\kern0pt}simp\ add{\isacharcolon}{\kern0pt}\ algebra{\isacharunderscore}{\kern0pt}simps\ add{\isacharunderscore}{\kern0pt}divide{\isacharunderscore}{\kern0pt}distrib\ p{\isacharunderscore}{\kern0pt}ge{\isacharunderscore}{\kern0pt}{\isadigit{0}}{\isacharparenright}{\kern0pt}\isanewline
\ \ \ \ \ \ \ \ \isacommand{also}\isamarkupfalse%
\ \isacommand{have}\isamarkupfalse%
\ {\isachardoublequoteopen}{\isachardot}{\kern0pt}{\isachardot}{\kern0pt}{\isachardot}{\kern0pt}\ {\isasymle}\ real\ t\ {\isacharplus}{\kern0pt}\ {\isadigit{3}}\ {\isacharasterisk}{\kern0pt}\ sqrt\ {\isacharparenleft}{\kern0pt}real\ m\ {\isacharasterisk}{\kern0pt}\ {\isacharparenleft}{\kern0pt}real\ t\ {\isacharasterisk}{\kern0pt}\ real\ p\ {\isacharslash}{\kern0pt}\ {\isacharparenleft}{\kern0pt}real\ m\ {\isacharasterisk}{\kern0pt}\ {\isacharparenleft}{\kern0pt}{\isadigit{1}}{\isacharminus}{\kern0pt}\ {\isasymdelta}{\isacharprime}{\kern0pt}{\isacharparenright}{\kern0pt}{\isacharparenright}{\kern0pt}{\isacharparenright}{\kern0pt}\ {\isacharslash}{\kern0pt}\ real\ p\ {\isacharplus}{\kern0pt}\ {\isadigit{1}}{\isacharparenright}{\kern0pt}{\isachardoublequoteclose}\isanewline
\ \ \ \ \ \ \ \ \ \ \isacommand{apply}\isamarkupfalse%
\ {\isacharparenleft}{\kern0pt}rule\ add{\isacharunderscore}{\kern0pt}mono{\isacharcomma}{\kern0pt}\ simp{\isacharparenright}{\kern0pt}\isanewline
\ \ \ \ \ \ \ \ \ \ \isacommand{apply}\isamarkupfalse%
\ {\isacharparenleft}{\kern0pt}rule\ mult{\isacharunderscore}{\kern0pt}left{\isacharunderscore}{\kern0pt}mono{\isacharparenright}{\kern0pt}\isanewline
\ \ \ \ \ \ \ \ \ \ \ \isacommand{apply}\isamarkupfalse%
\ {\isacharparenleft}{\kern0pt}subst\ real{\isacharunderscore}{\kern0pt}sqrt{\isacharunderscore}{\kern0pt}le{\isacharunderscore}{\kern0pt}iff{\isacharparenright}{\kern0pt}\isanewline
\ \ \ \ \ \ \ \ \ \ \ \isacommand{apply}\isamarkupfalse%
\ {\isacharparenleft}{\kern0pt}rule\ add{\isacharunderscore}{\kern0pt}mono{\isacharparenright}{\kern0pt}\isanewline
\ \ \ \ \ \ \ \ \ \ \ \ \isacommand{apply}\isamarkupfalse%
\ {\isacharparenleft}{\kern0pt}rule\ divide{\isacharunderscore}{\kern0pt}right{\isacharunderscore}{\kern0pt}mono{\isacharparenright}{\kern0pt}\isanewline
\ \ \ \ \ \ \ \ \ \ \ \ \ \isacommand{apply}\isamarkupfalse%
\ {\isacharparenleft}{\kern0pt}rule\ mult{\isacharunderscore}{\kern0pt}left{\isacharunderscore}{\kern0pt}mono{\isacharparenright}{\kern0pt}\isanewline
\ \ \ \ \ \ \ \ \ \ \isacommand{apply}\isamarkupfalse%
\ {\isacharparenleft}{\kern0pt}simp\ add{\isacharcolon}{\kern0pt}b{\isacharunderscore}{\kern0pt}def{\isacharparenright}{\kern0pt}\isanewline
\ \ \ \ \ \ \ \ \ \ \isacommand{by}\isamarkupfalse%
\ simp{\isacharplus}{\kern0pt}\ \isanewline
\ \ \ \ \ \ \ \ \isacommand{also}\isamarkupfalse%
\ \isacommand{have}\isamarkupfalse%
\ {\isachardoublequoteopen}{\isachardot}{\kern0pt}{\isachardot}{\kern0pt}{\isachardot}{\kern0pt}\ {\isasymle}\ real\ t\ {\isacharplus}{\kern0pt}\ {\isadigit{3}}\ {\isacharasterisk}{\kern0pt}\ sqrt{\isacharparenleft}{\kern0pt}real\ t\ {\isacharslash}{\kern0pt}\ {\isacharparenleft}{\kern0pt}{\isadigit{1}}{\isacharminus}{\kern0pt}{\isasymdelta}{\isacharprime}{\kern0pt}{\isacharparenright}{\kern0pt}\ {\isacharplus}{\kern0pt}\ {\isadigit{1}}{\isacharparenright}{\kern0pt}{\isachardoublequoteclose}\isanewline
\ \ \ \ \ \ \ \ \ \ \isacommand{apply}\isamarkupfalse%
\ {\isacharparenleft}{\kern0pt}simp\ add{\isacharcolon}{\kern0pt}p{\isacharunderscore}{\kern0pt}ge{\isacharunderscore}{\kern0pt}{\isadigit{0}}{\isacharparenright}{\kern0pt}\isanewline
\ \ \ \ \ \ \ \ \ \ \isacommand{using}\isamarkupfalse%
\ t{\isacharunderscore}{\kern0pt}ge{\isacharunderscore}{\kern0pt}{\isadigit{0}}\ t{\isacharunderscore}{\kern0pt}le{\isacharunderscore}{\kern0pt}m\ m{\isacharunderscore}{\kern0pt}def\ \isacommand{by}\isamarkupfalse%
\ linarith\isanewline
\ \ \ \ \ \ \ \ \isacommand{also}\isamarkupfalse%
\ \isacommand{have}\isamarkupfalse%
\ {\isachardoublequoteopen}{\isachardot}{\kern0pt}{\isachardot}{\kern0pt}{\isachardot}{\kern0pt}\ {\isasymle}\ real\ t\ {\isacharslash}{\kern0pt}\ {\isacharparenleft}{\kern0pt}{\isadigit{1}}{\isacharminus}{\kern0pt}{\isasymdelta}{\isacharprime}{\kern0pt}{\isacharparenright}{\kern0pt}{\isachardoublequoteclose}\ \isanewline
\ \ \ \ \ \ \ \ \ \ \isacommand{apply}\isamarkupfalse%
\ {\isacharparenleft}{\kern0pt}subst\ pos{\isacharunderscore}{\kern0pt}le{\isacharunderscore}{\kern0pt}divide{\isacharunderscore}{\kern0pt}eq{\isacharparenright}{\kern0pt}\ \isacommand{using}\isamarkupfalse%
\ {\isasymdelta}{\isacharprime}{\kern0pt}{\isacharunderscore}{\kern0pt}le{\isacharunderscore}{\kern0pt}{\isadigit{1}}\ aux\ \isacommand{by}\isamarkupfalse%
\ simp{\isacharplus}{\kern0pt}\isanewline
\ \ \ \ \ \ \ \ \isacommand{also}\isamarkupfalse%
\ \isacommand{have}\isamarkupfalse%
\ {\isachardoublequoteopen}{\isachardot}{\kern0pt}{\isachardot}{\kern0pt}{\isachardot}{\kern0pt}\ {\isacharequal}{\kern0pt}\ real\ m\ {\isacharasterisk}{\kern0pt}\ {\isacharparenleft}{\kern0pt}real\ t\ {\isacharasterisk}{\kern0pt}\ real\ p\ {\isacharslash}{\kern0pt}\ {\isacharparenleft}{\kern0pt}real\ m\ {\isacharasterisk}{\kern0pt}\ {\isacharparenleft}{\kern0pt}{\isadigit{1}}{\isacharminus}{\kern0pt}{\isasymdelta}{\isacharprime}{\kern0pt}{\isacharparenright}{\kern0pt}{\isacharparenright}{\kern0pt}{\isacharparenright}{\kern0pt}\ {\isacharslash}{\kern0pt}\ real\ p{\isachardoublequoteclose}\ \isanewline
\ \ \ \ \ \ \ \ \ \ \isacommand{apply}\isamarkupfalse%
\ {\isacharparenleft}{\kern0pt}simp\ add{\isacharcolon}{\kern0pt}p{\isacharunderscore}{\kern0pt}ge{\isacharunderscore}{\kern0pt}{\isadigit{0}}{\isacharparenright}{\kern0pt}\isanewline
\ \ \ \ \ \ \ \ \ \ \isacommand{using}\isamarkupfalse%
\ t{\isacharunderscore}{\kern0pt}ge{\isacharunderscore}{\kern0pt}{\isadigit{0}}\ t{\isacharunderscore}{\kern0pt}le{\isacharunderscore}{\kern0pt}m\ m{\isacharunderscore}{\kern0pt}def\ \isacommand{by}\isamarkupfalse%
\ linarith\isanewline
\ \ \ \ \ \ \ \ \isacommand{also}\isamarkupfalse%
\ \isacommand{have}\isamarkupfalse%
\ {\isachardoublequoteopen}{\isachardot}{\kern0pt}{\isachardot}{\kern0pt}{\isachardot}{\kern0pt}\ {\isasymle}\ \ real\ m\ {\isacharasterisk}{\kern0pt}\ {\isacharparenleft}{\kern0pt}real{\isacharunderscore}{\kern0pt}of{\isacharunderscore}{\kern0pt}int\ b\ {\isacharplus}{\kern0pt}\ {\isadigit{1}}{\isacharparenright}{\kern0pt}\ {\isacharslash}{\kern0pt}\ real\ p{\isachardoublequoteclose}\ \ \ \ \ \ \isanewline
\ \ \ \ \ \ \ \ \ \ \isacommand{apply}\isamarkupfalse%
\ {\isacharparenleft}{\kern0pt}rule\ divide{\isacharunderscore}{\kern0pt}right{\isacharunderscore}{\kern0pt}mono{\isacharparenright}{\kern0pt}\isanewline
\ \ \ \ \ \ \ \ \ \ \ \isacommand{apply}\isamarkupfalse%
\ {\isacharparenleft}{\kern0pt}rule\ mult{\isacharunderscore}{\kern0pt}left{\isacharunderscore}{\kern0pt}mono{\isacharparenright}{\kern0pt}\isanewline
\ \ \ \ \ \ \ \ \ \ \isacommand{by}\isamarkupfalse%
\ {\isacharparenleft}{\kern0pt}simp\ add{\isacharcolon}{\kern0pt}b{\isacharunderscore}{\kern0pt}def{\isacharparenright}{\kern0pt}{\isacharplus}{\kern0pt}\isanewline
\ \ \ \ \ \ \ \ \isacommand{finally}\isamarkupfalse%
\ \isacommand{have}\isamarkupfalse%
\ t{\isacharunderscore}{\kern0pt}ge{\isacharcolon}{\kern0pt}\ {\isachardoublequoteopen}real\ {\isacharparenleft}{\kern0pt}f\ b\ {\isasymomega}{\isacharparenright}{\kern0pt}\ {\isacharplus}{\kern0pt}\ {\isadigit{3}}\ {\isacharasterisk}{\kern0pt}\ sqrt\ {\isacharparenleft}{\kern0pt}real\ m\ {\isacharasterisk}{\kern0pt}\ {\isacharparenleft}{\kern0pt}real{\isacharunderscore}{\kern0pt}of{\isacharunderscore}{\kern0pt}int\ b\ {\isacharplus}{\kern0pt}\ {\isadigit{1}}{\isacharparenright}{\kern0pt}\ {\isacharslash}{\kern0pt}\ real\ p{\isacharparenright}{\kern0pt}\ \isanewline
\ \ \ \ \ \ \ \ \ \ {\isasymle}\ real\ m\ {\isacharasterisk}{\kern0pt}\ {\isacharparenleft}{\kern0pt}real{\isacharunderscore}{\kern0pt}of{\isacharunderscore}{\kern0pt}int\ b\ {\isacharplus}{\kern0pt}\ {\isadigit{1}}{\isacharparenright}{\kern0pt}\ {\isacharslash}{\kern0pt}\ real\ p{\isachardoublequoteclose}\isanewline
\ \ \ \ \ \ \ \ \ \ \isacommand{by}\isamarkupfalse%
\ simp\isanewline
\ \ \ \ \ \ \ \ \isacommand{show}\isamarkupfalse%
\ {\isachardoublequoteopen}\ {\isadigit{3}}\ {\isacharasterisk}{\kern0pt}\ sqrt\ {\isacharparenleft}{\kern0pt}real\ m\ {\isacharasterisk}{\kern0pt}\ {\isacharparenleft}{\kern0pt}real{\isacharunderscore}{\kern0pt}of{\isacharunderscore}{\kern0pt}int\ b\ {\isacharplus}{\kern0pt}\ {\isadigit{1}}{\isacharparenright}{\kern0pt}\ {\isacharslash}{\kern0pt}\ real\ p{\isacharparenright}{\kern0pt}\ {\isasymle}\ \isanewline
\ \ \ \ \ \ \ \ \ \ {\isasymbar}real\ {\isacharparenleft}{\kern0pt}f\ b\ {\isasymomega}{\isacharparenright}{\kern0pt}\ {\isacharminus}{\kern0pt}\ measure{\isacharunderscore}{\kern0pt}pmf{\isachardot}{\kern0pt}expectation\ {\isasymOmega}\isactrlsub {\isadigit{1}}\ {\isacharparenleft}{\kern0pt}{\isasymlambda}{\isasymomega}{\isachardot}{\kern0pt}\ real\ {\isacharparenleft}{\kern0pt}f\ b\ {\isasymomega}{\isacharparenright}{\kern0pt}{\isacharparenright}{\kern0pt}{\isasymbar}{\isachardoublequoteclose}\ \ \isanewline
\ \ \ \ \ \ \ \ \ \ \isacommand{apply}\isamarkupfalse%
\ {\isacharparenleft}{\kern0pt}subst\ exp{\isacharunderscore}{\kern0pt}f{\isacharparenright}{\kern0pt}\ \isacommand{using}\isamarkupfalse%
\ b{\isacharunderscore}{\kern0pt}ge{\isacharunderscore}{\kern0pt}{\isadigit{0}}\ True\ \isacommand{apply}\isamarkupfalse%
\ {\isacharparenleft}{\kern0pt}simp{\isacharcomma}{\kern0pt}\ simp{\isacharparenright}{\kern0pt}\isanewline
\ \ \ \ \ \ \ \ \ \ \isacommand{apply}\isamarkupfalse%
\ {\isacharparenleft}{\kern0pt}subst\ abs{\isacharunderscore}{\kern0pt}ge{\isacharunderscore}{\kern0pt}iff{\isacharparenright}{\kern0pt}\isanewline
\ \ \ \ \ \ \ \ \ \ \isacommand{using}\isamarkupfalse%
\ t{\isacharunderscore}{\kern0pt}ge\ \isacommand{by}\isamarkupfalse%
\ force\isanewline
\ \ \ \ \ \ \isacommand{qed}\isamarkupfalse%
\isanewline
\ \ \ \ \ \ \isacommand{also}\isamarkupfalse%
\ \isacommand{have}\isamarkupfalse%
\ {\isachardoublequoteopen}{\isachardot}{\kern0pt}{\isachardot}{\kern0pt}{\isachardot}{\kern0pt}\ {\isasymle}\ prob{\isacharunderscore}{\kern0pt}space{\isachardot}{\kern0pt}variance\ {\isacharparenleft}{\kern0pt}measure{\isacharunderscore}{\kern0pt}pmf\ {\isasymOmega}\isactrlsub {\isadigit{1}}{\isacharparenright}{\kern0pt}\ {\isacharparenleft}{\kern0pt}{\isasymlambda}{\isasymomega}{\isachardot}{\kern0pt}\ real\ {\isacharparenleft}{\kern0pt}f\ b\ {\isasymomega}{\isacharparenright}{\kern0pt}{\isacharparenright}{\kern0pt}\ \isanewline
\ \ \ \ \ \ \ \ {\isacharslash}{\kern0pt}\ {\isacharparenleft}{\kern0pt}{\isadigit{3}}\ {\isacharasterisk}{\kern0pt}\ sqrt\ {\isacharparenleft}{\kern0pt}real\ m\ {\isacharasterisk}{\kern0pt}\ {\isacharparenleft}{\kern0pt}real{\isacharunderscore}{\kern0pt}of{\isacharunderscore}{\kern0pt}int\ b\ {\isacharplus}{\kern0pt}\ {\isadigit{1}}{\isacharparenright}{\kern0pt}\ {\isacharslash}{\kern0pt}\ real\ p{\isacharparenright}{\kern0pt}{\isacharparenright}{\kern0pt}\isactrlsup {\isadigit{2}}{\isachardoublequoteclose}\ \isanewline
\ \ \ \ \ \ \ \ \isacommand{apply}\isamarkupfalse%
\ {\isacharparenleft}{\kern0pt}rule\ prob{\isacharunderscore}{\kern0pt}space{\isachardot}{\kern0pt}Chebyshev{\isacharunderscore}{\kern0pt}inequality{\isacharparenright}{\kern0pt}\isanewline
\ \ \ \ \ \ \ \ \ \ \ \isacommand{apply}\isamarkupfalse%
\ {\isacharparenleft}{\kern0pt}metis\ prob{\isacharunderscore}{\kern0pt}space{\isacharunderscore}{\kern0pt}measure{\isacharunderscore}{\kern0pt}pmf{\isacharparenright}{\kern0pt}\isanewline
\ \ \ \ \ \ \ \ \ \ \isacommand{apply}\isamarkupfalse%
\ simp\isanewline
\ \ \ \ \ \ \ \ \ \isacommand{apply}\isamarkupfalse%
\ {\isacharparenleft}{\kern0pt}rule\ integrable{\isacharunderscore}{\kern0pt}measure{\isacharunderscore}{\kern0pt}pmf{\isacharunderscore}{\kern0pt}finite{\isacharbrackleft}{\kern0pt}OF\ fin{\isacharunderscore}{\kern0pt}omega{\isacharunderscore}{\kern0pt}{\isadigit{1}}{\isacharbrackright}{\kern0pt}{\isacharparenright}{\kern0pt}\isanewline
\ \ \ \ \ \ \ \ \ \isacommand{apply}\isamarkupfalse%
\ simp\isanewline
\ \ \ \ \ \ \ \ \isacommand{using}\isamarkupfalse%
\ t{\isacharunderscore}{\kern0pt}ge{\isacharunderscore}{\kern0pt}{\isadigit{0}}\ b{\isacharunderscore}{\kern0pt}ge{\isacharunderscore}{\kern0pt}{\isadigit{0}}\ p{\isacharunderscore}{\kern0pt}ge{\isacharunderscore}{\kern0pt}{\isadigit{0}}\ m{\isacharunderscore}{\kern0pt}ge{\isacharunderscore}{\kern0pt}{\isadigit{0}}\ m{\isacharunderscore}{\kern0pt}eq{\isacharunderscore}{\kern0pt}F{\isacharunderscore}{\kern0pt}{\isadigit{0}}\ \isacommand{by}\isamarkupfalse%
\ auto\isanewline
\ \ \ \ \ \ \isacommand{also}\isamarkupfalse%
\ \isacommand{have}\isamarkupfalse%
\ {\isachardoublequoteopen}{\isachardot}{\kern0pt}{\isachardot}{\kern0pt}{\isachardot}{\kern0pt}\ {\isasymle}\ {\isadigit{1}}{\isacharslash}{\kern0pt}{\isadigit{9}}{\isachardoublequoteclose}\isanewline
\ \ \ \ \ \ \ \ \isacommand{apply}\isamarkupfalse%
\ {\isacharparenleft}{\kern0pt}subst\ pos{\isacharunderscore}{\kern0pt}divide{\isacharunderscore}{\kern0pt}le{\isacharunderscore}{\kern0pt}eq{\isacharparenright}{\kern0pt}\ \isanewline
\ \ \ \ \ \ \ \ \isacommand{using}\isamarkupfalse%
\ b{\isacharunderscore}{\kern0pt}ge{\isacharunderscore}{\kern0pt}{\isadigit{0}}\ p{\isacharunderscore}{\kern0pt}ge{\isacharunderscore}{\kern0pt}{\isadigit{0}}\ m{\isacharunderscore}{\kern0pt}ge{\isacharunderscore}{\kern0pt}{\isadigit{0}}\ m{\isacharunderscore}{\kern0pt}eq{\isacharunderscore}{\kern0pt}F{\isacharunderscore}{\kern0pt}{\isadigit{0}}\ \isacommand{apply}\isamarkupfalse%
\ force\isanewline
\ \ \ \ \ \ \ \ \isacommand{apply}\isamarkupfalse%
\ simp\isanewline
\ \ \ \ \ \ \ \ \isacommand{apply}\isamarkupfalse%
\ {\isacharparenleft}{\kern0pt}subst\ real{\isacharunderscore}{\kern0pt}sqrt{\isacharunderscore}{\kern0pt}pow{\isadigit{2}}{\isacharparenright}{\kern0pt}\isanewline
\ \ \ \ \ \ \ \ \isacommand{using}\isamarkupfalse%
\ b{\isacharunderscore}{\kern0pt}ge{\isacharunderscore}{\kern0pt}{\isadigit{0}}\ p{\isacharunderscore}{\kern0pt}ge{\isacharunderscore}{\kern0pt}{\isadigit{0}}\ m{\isacharunderscore}{\kern0pt}ge{\isacharunderscore}{\kern0pt}{\isadigit{0}}\ m{\isacharunderscore}{\kern0pt}eq{\isacharunderscore}{\kern0pt}F{\isacharunderscore}{\kern0pt}{\isadigit{0}}\ \isacommand{apply}\isamarkupfalse%
\ force\isanewline
\ \ \ \ \ \ \ \ \isacommand{apply}\isamarkupfalse%
\ {\isacharparenleft}{\kern0pt}rule\ var{\isacharunderscore}{\kern0pt}f{\isacharparenright}{\kern0pt}\ \isacommand{using}\isamarkupfalse%
\ b{\isacharunderscore}{\kern0pt}ge{\isacharunderscore}{\kern0pt}{\isadigit{0}}\ \isacommand{apply}\isamarkupfalse%
\ linarith\isanewline
\ \ \ \ \ \ \ \ \isacommand{using}\isamarkupfalse%
\ True\ \isacommand{by}\isamarkupfalse%
\ simp\isanewline
\ \ \ \ \ \ \isacommand{finally}\isamarkupfalse%
\ \isacommand{show}\isamarkupfalse%
\ {\isacharquery}{\kern0pt}thesis\isanewline
\ \ \ \ \ \ \ \ \isacommand{by}\isamarkupfalse%
\ simp\isanewline
\ \ \ \ \isacommand{next}\isamarkupfalse%
\isanewline
\ \ \ \ \ \ \isacommand{case}\isamarkupfalse%
\ False\isanewline
\ \ \ \ \ \ \isacommand{have}\isamarkupfalse%
\ {\isachardoublequoteopen}{\isasymP}{\isacharparenleft}{\kern0pt}{\isasymomega}\ in\ {\isasymOmega}\isactrlsub {\isadigit{1}}{\isachardot}{\kern0pt}\ f\ b\ {\isasymomega}\ {\isacharless}{\kern0pt}\ t{\isacharparenright}{\kern0pt}\ {\isasymle}\ {\isasymP}{\isacharparenleft}{\kern0pt}{\isasymomega}\ in\ {\isasymOmega}\isactrlsub {\isadigit{1}}{\isachardot}{\kern0pt}\ False{\isacharparenright}{\kern0pt}{\isachardoublequoteclose}\isanewline
\ \ \ \ \ \ \isacommand{proof}\isamarkupfalse%
\ {\isacharparenleft}{\kern0pt}rule\ pmf{\isacharunderscore}{\kern0pt}mono{\isacharunderscore}{\kern0pt}{\isadigit{1}}{\isacharparenright}{\kern0pt}\isanewline
\ \ \ \ \ \ \ \ \isacommand{fix}\isamarkupfalse%
\ {\isasymomega}\isanewline
\ \ \ \ \ \ \ \ \isacommand{assume}\isamarkupfalse%
\ a{\isacharunderscore}{\kern0pt}{\isadigit{1}}{\isacharcolon}{\kern0pt}{\isachardoublequoteopen}{\isasymomega}\ {\isasymin}\ {\isacharbraceleft}{\kern0pt}{\isasymomega}\ {\isasymin}\ space\ {\isacharparenleft}{\kern0pt}measure{\isacharunderscore}{\kern0pt}pmf\ {\isasymOmega}\isactrlsub {\isadigit{1}}{\isacharparenright}{\kern0pt}{\isachardot}{\kern0pt}\ f\ b\ {\isasymomega}\ {\isacharless}{\kern0pt}\ t{\isacharbraceright}{\kern0pt}{\isachardoublequoteclose}\isanewline
\ \ \ \ \ \ \ \ \isacommand{assume}\isamarkupfalse%
\ a{\isacharunderscore}{\kern0pt}{\isadigit{2}}{\isacharcolon}{\kern0pt}{\isachardoublequoteopen}{\isasymomega}\ {\isasymin}\ set{\isacharunderscore}{\kern0pt}pmf\ {\isasymOmega}\isactrlsub {\isadigit{1}}{\isachardoublequoteclose}\isanewline
\ \ \ \ \ \ \ \ \isacommand{have}\isamarkupfalse%
\ a{\isacharcolon}{\kern0pt}{\isachardoublequoteopen}{\isasymAnd}x{\isachardot}{\kern0pt}\ x\ {\isacharless}{\kern0pt}\ p\ {\isasymLongrightarrow}\ hash\ p\ x\ {\isasymomega}\ {\isacharless}{\kern0pt}\ p{\isachardoublequoteclose}\ \isanewline
\ \ \ \ \ \ \ \ \ \ \isacommand{using}\isamarkupfalse%
\ hash{\isacharunderscore}{\kern0pt}range{\isacharbrackleft}{\kern0pt}OF\ p{\isacharunderscore}{\kern0pt}ge{\isacharunderscore}{\kern0pt}{\isadigit{0}}{\isacharbrackright}{\kern0pt}\ \ a{\isacharunderscore}{\kern0pt}{\isadigit{2}}\isanewline
\ \ \ \ \ \ \ \ \ \ \isacommand{by}\isamarkupfalse%
\ {\isacharparenleft}{\kern0pt}simp\ add{\isacharcolon}{\kern0pt}{\isasymOmega}\isactrlsub {\isadigit{1}}{\isacharunderscore}{\kern0pt}def\ set{\isacharunderscore}{\kern0pt}pmf{\isacharunderscore}{\kern0pt}of{\isacharunderscore}{\kern0pt}set{\isacharbrackleft}{\kern0pt}OF\ ne{\isacharunderscore}{\kern0pt}bounded{\isacharunderscore}{\kern0pt}degree{\isacharunderscore}{\kern0pt}polynomials\ fin{\isacharunderscore}{\kern0pt}bounded{\isacharunderscore}{\kern0pt}degree{\isacharunderscore}{\kern0pt}polynomials{\isacharbrackleft}{\kern0pt}OF\ p{\isacharunderscore}{\kern0pt}ge{\isacharunderscore}{\kern0pt}{\isadigit{0}}{\isacharbrackright}{\kern0pt}{\isacharbrackright}{\kern0pt}{\isacharparenright}{\kern0pt}\isanewline
\ \ \ \ \ \ \ \ \isacommand{have}\isamarkupfalse%
\ {\isachardoublequoteopen}t\ {\isasymle}\ card\ {\isacharparenleft}{\kern0pt}set\ as{\isacharparenright}{\kern0pt}{\isachardoublequoteclose}\isanewline
\ \ \ \ \ \ \ \ \ \ \isacommand{using}\isamarkupfalse%
\ True\ \isacommand{by}\isamarkupfalse%
\ simp\isanewline
\ \ \ \ \ \ \ \ \isacommand{also}\isamarkupfalse%
\ \isacommand{have}\isamarkupfalse%
\ {\isachardoublequoteopen}{\isachardot}{\kern0pt}{\isachardot}{\kern0pt}{\isachardot}{\kern0pt}\ {\isasymle}\ f\ b\ {\isasymomega}{\isachardoublequoteclose}\isanewline
\ \ \ \ \ \ \ \ \ \ \isacommand{apply}\isamarkupfalse%
\ {\isacharparenleft}{\kern0pt}simp\ add{\isacharcolon}{\kern0pt}f{\isacharunderscore}{\kern0pt}def{\isacharparenright}{\kern0pt}\isanewline
\ \ \ \ \ \ \ \ \ \ \isacommand{apply}\isamarkupfalse%
\ {\isacharparenleft}{\kern0pt}rule\ card{\isacharunderscore}{\kern0pt}mono{\isacharcomma}{\kern0pt}\ simp{\isacharparenright}{\kern0pt}\isanewline
\ \ \ \ \ \ \ \ \ \ \isacommand{apply}\isamarkupfalse%
\ {\isacharparenleft}{\kern0pt}rule\ subsetI{\isacharparenright}{\kern0pt}\isanewline
\ \ \ \ \ \ \ \ \ \ \isacommand{by}\isamarkupfalse%
\ {\isacharparenleft}{\kern0pt}metis\ {\isacharparenleft}{\kern0pt}no{\isacharunderscore}{\kern0pt}types{\isacharcomma}{\kern0pt}\ lifting{\isacharparenright}{\kern0pt}\ False\ a\ xs{\isacharunderscore}{\kern0pt}le{\isacharunderscore}{\kern0pt}p\ \ linorder{\isacharunderscore}{\kern0pt}linear\ mem{\isacharunderscore}{\kern0pt}Collect{\isacharunderscore}{\kern0pt}eq\ of{\isacharunderscore}{\kern0pt}nat{\isacharunderscore}{\kern0pt}less{\isacharunderscore}{\kern0pt}iff\ order{\isacharunderscore}{\kern0pt}le{\isacharunderscore}{\kern0pt}less{\isacharunderscore}{\kern0pt}trans{\isacharparenright}{\kern0pt}\isanewline
\ \ \ \ \ \ \ \ \isacommand{also}\isamarkupfalse%
\ \isacommand{have}\isamarkupfalse%
\ {\isachardoublequoteopen}{\isachardot}{\kern0pt}{\isachardot}{\kern0pt}{\isachardot}{\kern0pt}\ {\isacharless}{\kern0pt}\ t{\isachardoublequoteclose}\ \isacommand{using}\isamarkupfalse%
\ a{\isacharunderscore}{\kern0pt}{\isadigit{1}}\ \isacommand{by}\isamarkupfalse%
\ simp\isanewline
\ \ \ \ \ \ \ \ \isacommand{finally}\isamarkupfalse%
\ \isacommand{have}\isamarkupfalse%
\ {\isachardoublequoteopen}False{\isachardoublequoteclose}\ \isacommand{by}\isamarkupfalse%
\ auto\isanewline
\ \ \ \ \ \ \ \ \isacommand{thus}\isamarkupfalse%
\ {\isachardoublequoteopen}{\isasymomega}\ {\isasymin}\ {\isacharbraceleft}{\kern0pt}{\isasymomega}\ {\isasymin}\ space\ {\isacharparenleft}{\kern0pt}measure{\isacharunderscore}{\kern0pt}pmf\ {\isasymOmega}\isactrlsub {\isadigit{1}}{\isacharparenright}{\kern0pt}{\isachardot}{\kern0pt}\ False{\isacharbraceright}{\kern0pt}{\isachardoublequoteclose}\isanewline
\ \ \ \ \ \ \ \ \ \ \isacommand{by}\isamarkupfalse%
\ simp\isanewline
\ \ \ \ \ \ \isacommand{qed}\isamarkupfalse%
\isanewline
\ \ \ \ \ \ \isacommand{also}\isamarkupfalse%
\ \isacommand{have}\isamarkupfalse%
\ {\isachardoublequoteopen}{\isachardot}{\kern0pt}{\isachardot}{\kern0pt}{\isachardot}{\kern0pt}\ {\isacharequal}{\kern0pt}\ {\isadigit{0}}{\isachardoublequoteclose}\ \isacommand{by}\isamarkupfalse%
\ auto\isanewline
\ \ \ \ \ \ \isacommand{finally}\isamarkupfalse%
\ \isacommand{show}\isamarkupfalse%
\ {\isacharquery}{\kern0pt}thesis\ \isacommand{by}\isamarkupfalse%
\ simp\isanewline
\ \ \ \ \isacommand{qed}\isamarkupfalse%
\isanewline
\isanewline
\ \ \ \ \isacommand{have}\isamarkupfalse%
\ {\isachardoublequoteopen}{\isasymP}{\isacharparenleft}{\kern0pt}{\isasymomega}\ in\ measure{\isacharunderscore}{\kern0pt}pmf\ {\isasymOmega}\isactrlsub {\isadigit{1}}{\isachardot}{\kern0pt}\ {\isasymnot}has{\isacharunderscore}{\kern0pt}no{\isacharunderscore}{\kern0pt}collision\ {\isasymomega}{\isacharparenright}{\kern0pt}\ {\isasymle}\isanewline
\ \ \ \ \ \ {\isasymP}{\isacharparenleft}{\kern0pt}{\isasymomega}\ in\ measure{\isacharunderscore}{\kern0pt}pmf\ {\isasymOmega}\isactrlsub {\isadigit{1}}{\isachardot}{\kern0pt}\ {\isasymexists}x\ {\isasymin}\ set\ as{\isachardot}{\kern0pt}\ {\isasymexists}y\ {\isasymin}\ set\ as{\isachardot}{\kern0pt}\ x\ {\isasymnoteq}\ y\ {\isasymand}\ \isanewline
\ \ \ \ \ \ truncate{\isacharunderscore}{\kern0pt}down\ r\ {\isacharparenleft}{\kern0pt}real\ {\isacharparenleft}{\kern0pt}hash\ p\ x\ {\isasymomega}{\isacharparenright}{\kern0pt}{\isacharparenright}{\kern0pt}\ {\isasymle}\ real{\isacharunderscore}{\kern0pt}of{\isacharunderscore}{\kern0pt}int\ b\ {\isasymand}\ \isanewline
\ \ \ \ \ \ truncate{\isacharunderscore}{\kern0pt}down\ r\ {\isacharparenleft}{\kern0pt}real\ {\isacharparenleft}{\kern0pt}hash\ p\ x\ {\isasymomega}{\isacharparenright}{\kern0pt}{\isacharparenright}{\kern0pt}\ {\isacharequal}{\kern0pt}\ truncate{\isacharunderscore}{\kern0pt}down\ r\ {\isacharparenleft}{\kern0pt}real\ {\isacharparenleft}{\kern0pt}hash\ p\ y\ {\isasymomega}{\isacharparenright}{\kern0pt}{\isacharparenright}{\kern0pt}{\isacharparenright}{\kern0pt}{\isachardoublequoteclose}\ \isanewline
\ \ \ \ \ \ \isacommand{apply}\isamarkupfalse%
\ {\isacharparenleft}{\kern0pt}rule\ pmf{\isacharunderscore}{\kern0pt}mono{\isacharunderscore}{\kern0pt}{\isadigit{1}}{\isacharparenright}{\kern0pt}\isanewline
\ \ \ \ \ \ \isacommand{apply}\isamarkupfalse%
\ {\isacharparenleft}{\kern0pt}simp\ add{\isacharcolon}{\kern0pt}has{\isacharunderscore}{\kern0pt}no{\isacharunderscore}{\kern0pt}collision{\isacharunderscore}{\kern0pt}def\ {\isasymOmega}\isactrlsub {\isadigit{1}}{\isacharunderscore}{\kern0pt}def{\isacharparenright}{\kern0pt}\ \isanewline
\ \ \ \ \ \ \isacommand{by}\isamarkupfalse%
\ force\isanewline
\ \ \ \ \isacommand{also}\isamarkupfalse%
\ \isacommand{have}\isamarkupfalse%
\ {\isachardoublequoteopen}{\isachardot}{\kern0pt}{\isachardot}{\kern0pt}{\isachardot}{\kern0pt}\ {\isasymle}\ {\isadigit{6}}\ {\isacharasterisk}{\kern0pt}\ {\isacharparenleft}{\kern0pt}real\ {\isacharparenleft}{\kern0pt}card\ {\isacharparenleft}{\kern0pt}set\ as{\isacharparenright}{\kern0pt}{\isacharparenright}{\kern0pt}{\isacharparenright}{\kern0pt}\isactrlsup {\isadigit{2}}\ {\isacharasterisk}{\kern0pt}\ {\isacharparenleft}{\kern0pt}real{\isacharunderscore}{\kern0pt}of{\isacharunderscore}{\kern0pt}int\ b{\isacharparenright}{\kern0pt}\isactrlsup {\isadigit{2}}\ \isanewline
\ \ \ \ \ \ \ {\isacharasterisk}{\kern0pt}\ {\isadigit{2}}\ powr\ {\isacharminus}{\kern0pt}\ real\ r\ {\isacharslash}{\kern0pt}\ {\isacharparenleft}{\kern0pt}real\ p{\isacharparenright}{\kern0pt}\isactrlsup {\isadigit{2}}\ {\isacharplus}{\kern0pt}\ {\isadigit{1}}\ {\isacharslash}{\kern0pt}\ real\ p{\isachardoublequoteclose}\isanewline
\ \ \ \ \ \ \isacommand{apply}\isamarkupfalse%
\ {\isacharparenleft}{\kern0pt}simp\ only{\isacharcolon}{\kern0pt}\ {\isasymOmega}\isactrlsub {\isadigit{1}}{\isacharunderscore}{\kern0pt}def{\isacharparenright}{\kern0pt}\isanewline
\ \ \ \ \ \ \isacommand{apply}\isamarkupfalse%
\ {\isacharparenleft}{\kern0pt}rule\ f{\isadigit{0}}{\isacharunderscore}{\kern0pt}collision{\isacharunderscore}{\kern0pt}prob{\isacharbrackleft}{\kern0pt}\isakeyword{where}\ c{\isacharequal}{\kern0pt}{\isachardoublequoteopen}real{\isacharunderscore}{\kern0pt}of{\isacharunderscore}{\kern0pt}int\ b{\isachardoublequoteclose}{\isacharbrackright}{\kern0pt}{\isacharparenright}{\kern0pt}\isanewline
\ \ \ \ \ \ \ \ \isacommand{apply}\isamarkupfalse%
\ {\isacharparenleft}{\kern0pt}metis\ p{\isacharunderscore}{\kern0pt}prime{\isacharparenright}{\kern0pt}\isanewline
\ \ \ \ \ \ \ \isacommand{apply}\isamarkupfalse%
\ {\isacharparenleft}{\kern0pt}rule\ subsetI{\isacharcomma}{\kern0pt}\ simp\ add{\isacharcolon}{\kern0pt}xs{\isacharunderscore}{\kern0pt}le{\isacharunderscore}{\kern0pt}p{\isacharparenright}{\kern0pt}\isanewline
\ \ \ \ \ \ \ \isacommand{apply}\isamarkupfalse%
\ {\isacharparenleft}{\kern0pt}\ metis\ b{\isacharunderscore}{\kern0pt}ge{\isacharunderscore}{\kern0pt}{\isadigit{1}}{\isacharparenright}{\kern0pt}\isanewline
\ \ \ \ \ \ \isacommand{by}\isamarkupfalse%
\ {\isacharparenleft}{\kern0pt}metis\ r{\isacharunderscore}{\kern0pt}ge{\isacharunderscore}{\kern0pt}{\isadigit{0}}{\isacharparenright}{\kern0pt}\isanewline
\ \ \ \ \isacommand{also}\isamarkupfalse%
\ \isacommand{have}\isamarkupfalse%
\ {\isachardoublequoteopen}{\isachardot}{\kern0pt}{\isachardot}{\kern0pt}{\isachardot}{\kern0pt}\ {\isasymle}\ {\isadigit{6}}\ {\isacharasterisk}{\kern0pt}\ {\isacharparenleft}{\kern0pt}real\ m{\isacharparenright}{\kern0pt}\isactrlsup {\isadigit{2}}\ {\isacharasterisk}{\kern0pt}\ {\isacharparenleft}{\kern0pt}real{\isacharunderscore}{\kern0pt}of{\isacharunderscore}{\kern0pt}int\ b{\isacharparenright}{\kern0pt}\isactrlsup {\isadigit{2}}\ {\isacharasterisk}{\kern0pt}\ {\isadigit{2}}\ powr\ {\isacharminus}{\kern0pt}\ real\ r\ {\isacharslash}{\kern0pt}\ {\isacharparenleft}{\kern0pt}real\ p{\isacharparenright}{\kern0pt}\isactrlsup {\isadigit{2}}\ {\isacharplus}{\kern0pt}\ {\isadigit{1}}\ {\isacharslash}{\kern0pt}\ real\ p{\isachardoublequoteclose}\isanewline
\ \ \ \ \ \ \isacommand{apply}\isamarkupfalse%
\ {\isacharparenleft}{\kern0pt}rule\ add{\isacharunderscore}{\kern0pt}mono{\isacharparenright}{\kern0pt}\isanewline
\ \ \ \ \ \ \ \isacommand{apply}\isamarkupfalse%
\ {\isacharparenleft}{\kern0pt}rule\ divide{\isacharunderscore}{\kern0pt}right{\isacharunderscore}{\kern0pt}mono{\isacharparenright}{\kern0pt}\isanewline
\ \ \ \ \ \ \ \ \isacommand{apply}\isamarkupfalse%
\ {\isacharparenleft}{\kern0pt}rule\ mult{\isacharunderscore}{\kern0pt}right{\isacharunderscore}{\kern0pt}mono{\isacharparenright}{\kern0pt}\isanewline
\ \ \ \ \ \ \ \ \ \isacommand{apply}\isamarkupfalse%
\ {\isacharparenleft}{\kern0pt}rule\ mult{\isacharunderscore}{\kern0pt}mono{\isacharparenright}{\kern0pt}\isanewline
\ \ \ \ \ \ \ \ \ \ \ \ \isacommand{apply}\isamarkupfalse%
\ {\isacharparenleft}{\kern0pt}simp\ add{\isacharcolon}{\kern0pt}m{\isacharunderscore}{\kern0pt}def{\isacharparenright}{\kern0pt}\isanewline
\ \ \ \ \ \ \ \ \ \ \ \isacommand{apply}\isamarkupfalse%
\ {\isacharparenleft}{\kern0pt}rule\ power{\isacharunderscore}{\kern0pt}mono{\isacharcomma}{\kern0pt}\ simp{\isacharparenright}{\kern0pt}\isanewline
\ \ \ \ \ \ \isacommand{using}\isamarkupfalse%
\ b{\isacharunderscore}{\kern0pt}ge{\isacharunderscore}{\kern0pt}{\isadigit{0}}\ \isacommand{by}\isamarkupfalse%
\ simp{\isacharplus}{\kern0pt}\isanewline
\ \ \ \ \isacommand{also}\isamarkupfalse%
\ \isacommand{have}\isamarkupfalse%
\ {\isachardoublequoteopen}{\isachardot}{\kern0pt}{\isachardot}{\kern0pt}{\isachardot}{\kern0pt}\ {\isasymle}\ {\isadigit{6}}\ {\isacharasterisk}{\kern0pt}\ {\isacharparenleft}{\kern0pt}real\ m{\isacharparenright}{\kern0pt}\isactrlsup {\isadigit{2}}\ {\isacharasterisk}{\kern0pt}\ {\isacharparenleft}{\kern0pt}{\isadigit{4}}\ {\isacharasterisk}{\kern0pt}\ real\ t\ {\isacharasterisk}{\kern0pt}\ real\ p\ {\isacharslash}{\kern0pt}\ real\ m{\isacharparenright}{\kern0pt}\isactrlsup {\isadigit{2}}\ {\isacharasterisk}{\kern0pt}\ {\isacharparenleft}{\kern0pt}{\isadigit{2}}\ powr\ {\isacharminus}{\kern0pt}\ real\ r{\isacharparenright}{\kern0pt}\ {\isacharslash}{\kern0pt}\ {\isacharparenleft}{\kern0pt}real\ p{\isacharparenright}{\kern0pt}\isactrlsup {\isadigit{2}}\ {\isacharplus}{\kern0pt}\ {\isadigit{1}}\ {\isacharslash}{\kern0pt}\ real\ p{\isachardoublequoteclose}\isanewline
\ \ \ \ \ \ \isacommand{apply}\isamarkupfalse%
\ {\isacharparenleft}{\kern0pt}rule\ add{\isacharunderscore}{\kern0pt}mono{\isacharparenright}{\kern0pt}\isanewline
\ \ \ \ \ \ \ \isacommand{apply}\isamarkupfalse%
\ {\isacharparenleft}{\kern0pt}rule\ divide{\isacharunderscore}{\kern0pt}right{\isacharunderscore}{\kern0pt}mono{\isacharparenright}{\kern0pt}\isanewline
\ \ \ \ \ \ \ \ \isacommand{apply}\isamarkupfalse%
\ {\isacharparenleft}{\kern0pt}rule\ mult{\isacharunderscore}{\kern0pt}right{\isacharunderscore}{\kern0pt}mono{\isacharparenright}{\kern0pt}\isanewline
\ \ \ \ \ \ \ \ \isacommand{apply}\isamarkupfalse%
\ {\isacharparenleft}{\kern0pt}rule\ mult{\isacharunderscore}{\kern0pt}left{\isacharunderscore}{\kern0pt}mono{\isacharparenright}{\kern0pt}\isanewline
\ \ \ \ \ \ \isacommand{apply}\isamarkupfalse%
\ {\isacharparenleft}{\kern0pt}simp\ add{\isacharcolon}{\kern0pt}b{\isacharunderscore}{\kern0pt}def{\isacharparenright}{\kern0pt}\ \isanewline
\ \ \ \ \ \ \isacommand{using}\isamarkupfalse%
\ b{\isacharunderscore}{\kern0pt}def\ b{\isacharunderscore}{\kern0pt}ge{\isacharunderscore}{\kern0pt}{\isadigit{1}}\ b{\isacharunderscore}{\kern0pt}le{\isacharunderscore}{\kern0pt}tpm\ \isacommand{apply}\isamarkupfalse%
\ force\isanewline
\ \ \ \ \ \ \ \ \ \isacommand{apply}\isamarkupfalse%
\ simp\isanewline
\ \ \ \ \ \ \ \ \isacommand{apply}\isamarkupfalse%
\ simp\isanewline
\ \ \ \ \ \ \ \isacommand{apply}\isamarkupfalse%
\ simp\isanewline
\ \ \ \ \ \ \isacommand{by}\isamarkupfalse%
\ simp\ \isanewline
\ \ \ \ \isacommand{also}\isamarkupfalse%
\ \isacommand{have}\isamarkupfalse%
\ {\isachardoublequoteopen}{\isachardot}{\kern0pt}{\isachardot}{\kern0pt}{\isachardot}{\kern0pt}\ {\isacharequal}{\kern0pt}\ {\isadigit{9}}{\isadigit{6}}\ {\isacharasterisk}{\kern0pt}\ {\isacharparenleft}{\kern0pt}real\ t{\isacharparenright}{\kern0pt}\isactrlsup {\isadigit{2}}\ {\isacharasterisk}{\kern0pt}\ {\isacharparenleft}{\kern0pt}{\isadigit{2}}\ powr\ {\isacharminus}{\kern0pt}real\ r{\isacharparenright}{\kern0pt}\ {\isacharplus}{\kern0pt}\ {\isadigit{1}}\ {\isacharslash}{\kern0pt}\ real\ p{\isachardoublequoteclose}\isanewline
\ \ \ \ \ \ \isacommand{using}\isamarkupfalse%
\ p{\isacharunderscore}{\kern0pt}ge{\isacharunderscore}{\kern0pt}{\isadigit{0}}\ m{\isacharunderscore}{\kern0pt}ge{\isacharunderscore}{\kern0pt}{\isadigit{0}}\ t{\isacharunderscore}{\kern0pt}ge{\isacharunderscore}{\kern0pt}{\isadigit{0}}\ \isacommand{by}\isamarkupfalse%
\ {\isacharparenleft}{\kern0pt}simp\ add{\isacharcolon}{\kern0pt}algebra{\isacharunderscore}{\kern0pt}simps\ power{\isadigit{2}}{\isacharunderscore}{\kern0pt}eq{\isacharunderscore}{\kern0pt}square{\isacharparenright}{\kern0pt}\isanewline
\ \ \ \ \isacommand{also}\isamarkupfalse%
\ \isacommand{have}\isamarkupfalse%
\ {\isachardoublequoteopen}{\isachardot}{\kern0pt}{\isachardot}{\kern0pt}{\isachardot}{\kern0pt}\ {\isasymle}\ {\isadigit{1}}{\isacharslash}{\kern0pt}{\isadigit{1}}{\isadigit{8}}\ {\isacharplus}{\kern0pt}\ {\isadigit{1}}{\isacharslash}{\kern0pt}{\isadigit{1}}{\isadigit{8}}{\isachardoublequoteclose}\isanewline
\ \ \ \ \ \ \isacommand{apply}\isamarkupfalse%
\ {\isacharparenleft}{\kern0pt}rule\ add{\isacharunderscore}{\kern0pt}mono{\isacharparenright}{\kern0pt}\isanewline
\ \ \ \ \ \ \isacommand{apply}\isamarkupfalse%
\ {\isacharparenleft}{\kern0pt}subst\ pos{\isacharunderscore}{\kern0pt}le{\isacharunderscore}{\kern0pt}divide{\isacharunderscore}{\kern0pt}eq{\isacharcomma}{\kern0pt}\ simp{\isacharparenright}{\kern0pt}\isanewline
\ \ \ \ \ \ \isacommand{using}\isamarkupfalse%
\ r{\isacharunderscore}{\kern0pt}le{\isacharunderscore}{\kern0pt}t{\isadigit{2}}\ \isacommand{apply}\isamarkupfalse%
\ simp\isanewline
\ \ \ \ \ \ \isacommand{using}\isamarkupfalse%
\ p{\isacharunderscore}{\kern0pt}ge{\isacharunderscore}{\kern0pt}{\isadigit{1}}{\isadigit{8}}\ \isacommand{by}\isamarkupfalse%
\ simp\isanewline
\ \ \ \ \isacommand{also}\isamarkupfalse%
\ \isacommand{have}\isamarkupfalse%
\ {\isachardoublequoteopen}{\isachardot}{\kern0pt}{\isachardot}{\kern0pt}{\isachardot}{\kern0pt}\ {\isacharequal}{\kern0pt}\ {\isadigit{1}}{\isacharslash}{\kern0pt}{\isadigit{9}}{\isachardoublequoteclose}\ \isacommand{by}\isamarkupfalse%
\ {\isacharparenleft}{\kern0pt}simp{\isacharparenright}{\kern0pt}\isanewline
\ \ \ \ \isacommand{finally}\isamarkupfalse%
\ \isacommand{have}\isamarkupfalse%
\ case{\isacharunderscore}{\kern0pt}{\isadigit{3}}{\isacharcolon}{\kern0pt}\ {\isachardoublequoteopen}{\isasymP}{\isacharparenleft}{\kern0pt}{\isasymomega}\ in\ measure{\isacharunderscore}{\kern0pt}pmf\ {\isasymOmega}\isactrlsub {\isadigit{1}}{\isachardot}{\kern0pt}\ {\isasymnot}has{\isacharunderscore}{\kern0pt}no{\isacharunderscore}{\kern0pt}collision\ {\isasymomega}{\isacharparenright}{\kern0pt}\ {\isasymle}\ {\isadigit{1}}{\isacharslash}{\kern0pt}{\isadigit{9}}{\isachardoublequoteclose}\ \isanewline
\ \ \ \ \ \ \isacommand{by}\isamarkupfalse%
\ simp\isanewline
\isanewline
\ \ \ \ \isacommand{have}\isamarkupfalse%
\ {\isachardoublequoteopen}{\isasymP}{\isacharparenleft}{\kern0pt}{\isasymomega}\ in\ measure{\isacharunderscore}{\kern0pt}pmf\ {\isasymOmega}\isactrlsub {\isadigit{1}}{\isachardot}{\kern0pt}\isanewline
\ \ \ \ \ \ \ \ real{\isacharunderscore}{\kern0pt}of{\isacharunderscore}{\kern0pt}rat\ {\isasymdelta}\ {\isacharasterisk}{\kern0pt}\ real{\isacharunderscore}{\kern0pt}of{\isacharunderscore}{\kern0pt}rat\ {\isacharparenleft}{\kern0pt}F\ {\isadigit{0}}\ as{\isacharparenright}{\kern0pt}\ {\isacharless}{\kern0pt}\ {\isasymbar}g{\isacharprime}{\kern0pt}\ {\isacharparenleft}{\kern0pt}h\ {\isasymomega}{\isacharparenright}{\kern0pt}\ {\isacharminus}{\kern0pt}\ real{\isacharunderscore}{\kern0pt}of{\isacharunderscore}{\kern0pt}rat\ {\isacharparenleft}{\kern0pt}F\ {\isadigit{0}}\ as{\isacharparenright}{\kern0pt}{\isasymbar}{\isacharparenright}{\kern0pt}\ {\isasymle}\ \isanewline
\ \ \ \ \ \ {\isasymP}{\isacharparenleft}{\kern0pt}{\isasymomega}\ in\ measure{\isacharunderscore}{\kern0pt}pmf\ {\isasymOmega}\isactrlsub {\isadigit{1}}{\isachardot}{\kern0pt}\ f\ a\ {\isasymomega}\ {\isasymge}\ t\ {\isasymor}\ f\ b\ {\isasymomega}\ {\isacharless}{\kern0pt}\ t\ {\isasymor}\ {\isasymnot}{\isacharparenleft}{\kern0pt}has{\isacharunderscore}{\kern0pt}no{\isacharunderscore}{\kern0pt}collision\ {\isasymomega}{\isacharparenright}{\kern0pt}{\isacharparenright}{\kern0pt}{\isachardoublequoteclose}\isanewline
\ \ \ \ \isacommand{proof}\isamarkupfalse%
\ {\isacharparenleft}{\kern0pt}rule\ prob{\isacharunderscore}{\kern0pt}space{\isachardot}{\kern0pt}prob{\isacharunderscore}{\kern0pt}mono{\isacharbrackleft}{\kern0pt}OF\ prob{\isacharunderscore}{\kern0pt}space{\isacharunderscore}{\kern0pt}measure{\isacharunderscore}{\kern0pt}pmf\ in{\isacharunderscore}{\kern0pt}events{\isacharunderscore}{\kern0pt}pmf{\isacharbrackright}{\kern0pt}{\isacharcomma}{\kern0pt}\ rule\ ccontr{\isacharparenright}{\kern0pt}\isanewline
\ \ \ \ \ \ \isacommand{fix}\isamarkupfalse%
\ {\isasymomega}\isanewline
\ \ \ \ \ \ \isacommand{assume}\isamarkupfalse%
\ {\isachardoublequoteopen}{\isasymomega}\ {\isasymin}\ space\ {\isacharparenleft}{\kern0pt}measure{\isacharunderscore}{\kern0pt}pmf\ {\isasymOmega}\isactrlsub {\isadigit{1}}{\isacharparenright}{\kern0pt}{\isachardoublequoteclose}\isanewline
\ \ \ \ \ \ \isacommand{assume}\isamarkupfalse%
\ est{\isacharcolon}{\kern0pt}\ {\isachardoublequoteopen}real{\isacharunderscore}{\kern0pt}of{\isacharunderscore}{\kern0pt}rat\ {\isasymdelta}\ {\isacharasterisk}{\kern0pt}\ real{\isacharunderscore}{\kern0pt}of{\isacharunderscore}{\kern0pt}rat\ {\isacharparenleft}{\kern0pt}F\ {\isadigit{0}}\ as{\isacharparenright}{\kern0pt}\ {\isacharless}{\kern0pt}\ {\isasymbar}g{\isacharprime}{\kern0pt}\ {\isacharparenleft}{\kern0pt}h\ {\isasymomega}{\isacharparenright}{\kern0pt}\ {\isacharminus}{\kern0pt}\ real{\isacharunderscore}{\kern0pt}of{\isacharunderscore}{\kern0pt}rat\ {\isacharparenleft}{\kern0pt}F\ {\isadigit{0}}\ as{\isacharparenright}{\kern0pt}{\isasymbar}{\isachardoublequoteclose}\isanewline
\ \ \ \ \ \ \isacommand{assume}\isamarkupfalse%
\ {\isachardoublequoteopen}{\isasymnot}{\isacharparenleft}{\kern0pt}\ t\ {\isasymle}\ f\ a\ {\isasymomega}\ {\isasymor}\ f\ b\ {\isasymomega}\ {\isacharless}{\kern0pt}\ t\ {\isasymor}\ {\isasymnot}\ has{\isacharunderscore}{\kern0pt}no{\isacharunderscore}{\kern0pt}collision\ {\isasymomega}{\isacharparenright}{\kern0pt}{\isachardoublequoteclose}\isanewline
\ \ \ \ \ \ \isacommand{hence}\isamarkupfalse%
\ lb{\isacharcolon}{\kern0pt}\ {\isachardoublequoteopen}f\ a\ {\isasymomega}\ {\isacharless}{\kern0pt}\ t{\isachardoublequoteclose}\ \isakeyword{and}\ ub{\isacharcolon}{\kern0pt}\ {\isachardoublequoteopen}f\ b\ {\isasymomega}\ {\isasymge}\ t{\isachardoublequoteclose}\ \isakeyword{and}\ no{\isacharunderscore}{\kern0pt}col{\isacharcolon}{\kern0pt}\ {\isachardoublequoteopen}has{\isacharunderscore}{\kern0pt}no{\isacharunderscore}{\kern0pt}collision\ {\isasymomega}{\isachardoublequoteclose}\ \isacommand{by}\isamarkupfalse%
\ simp{\isacharplus}{\kern0pt}\isanewline
\isanewline
\ \ \ \ \ \ \isacommand{define}\isamarkupfalse%
\ y\ \isakeyword{where}\ {\isachardoublequoteopen}y\ {\isacharequal}{\kern0pt}\ \ nth{\isacharunderscore}{\kern0pt}mset\ {\isacharparenleft}{\kern0pt}t{\isacharminus}{\kern0pt}{\isadigit{1}}{\isacharparenright}{\kern0pt}\ {\isacharbraceleft}{\kern0pt}{\isacharhash}{\kern0pt}int\ {\isacharparenleft}{\kern0pt}hash\ p\ x\ {\isasymomega}{\isacharparenright}{\kern0pt}{\isachardot}{\kern0pt}\ x\ {\isasymin}{\isacharhash}{\kern0pt}\ mset{\isacharunderscore}{\kern0pt}set\ {\isacharparenleft}{\kern0pt}set\ as{\isacharparenright}{\kern0pt}{\isacharhash}{\kern0pt}{\isacharbraceright}{\kern0pt}{\isachardoublequoteclose}\isanewline
\ \ \ \ \ \ \isacommand{define}\isamarkupfalse%
\ y{\isacharprime}{\kern0pt}\ \isakeyword{where}\ {\isachardoublequoteopen}y{\isacharprime}{\kern0pt}\ {\isacharequal}{\kern0pt}\ \ nth{\isacharunderscore}{\kern0pt}mset\ {\isacharparenleft}{\kern0pt}t{\isacharminus}{\kern0pt}{\isadigit{1}}{\isacharparenright}{\kern0pt}\ {\isacharbraceleft}{\kern0pt}{\isacharhash}{\kern0pt}truncate{\isacharunderscore}{\kern0pt}down\ r\ {\isacharparenleft}{\kern0pt}hash\ p\ x\ {\isasymomega}{\isacharparenright}{\kern0pt}{\isachardot}{\kern0pt}\ x\ {\isasymin}{\isacharhash}{\kern0pt}\ mset{\isacharunderscore}{\kern0pt}set\ {\isacharparenleft}{\kern0pt}set\ as{\isacharparenright}{\kern0pt}{\isacharhash}{\kern0pt}{\isacharbraceright}{\kern0pt}{\isachardoublequoteclose}\isanewline
\isanewline
\ \ \ \ \ \ \isacommand{have}\isamarkupfalse%
\ {\isachardoublequoteopen}a\ {\isacharless}{\kern0pt}\ y{\isachardoublequoteclose}\ \isanewline
\ \ \ \ \ \ \ \ \isacommand{apply}\isamarkupfalse%
\ {\isacharparenleft}{\kern0pt}subst\ y{\isacharunderscore}{\kern0pt}def{\isacharcomma}{\kern0pt}\ rule\ nth{\isacharunderscore}{\kern0pt}mset{\isacharunderscore}{\kern0pt}bound{\isacharunderscore}{\kern0pt}left{\isacharunderscore}{\kern0pt}excl{\isacharparenright}{\kern0pt}\isanewline
\ \ \ \ \ \ \ \ \ \isacommand{apply}\isamarkupfalse%
\ {\isacharparenleft}{\kern0pt}simp{\isacharparenright}{\kern0pt}\isanewline
\ \ \ \ \ \ \ \ \isacommand{using}\isamarkupfalse%
\ True\ t{\isacharunderscore}{\kern0pt}ge{\isacharunderscore}{\kern0pt}{\isadigit{0}}\ \isacommand{apply}\isamarkupfalse%
\ linarith\isanewline
\ \ \ \ \ \ \ \ \isacommand{using}\isamarkupfalse%
\ lb\ \isanewline
\ \ \ \ \ \ \ \ \isacommand{by}\isamarkupfalse%
\ {\isacharparenleft}{\kern0pt}simp\ add{\isacharcolon}{\kern0pt}f{\isacharunderscore}{\kern0pt}def\ swap{\isacharunderscore}{\kern0pt}filter{\isacharunderscore}{\kern0pt}image\ count{\isacharunderscore}{\kern0pt}le{\isacharunderscore}{\kern0pt}def{\isacharparenright}{\kern0pt}\isanewline
\ \ \ \ \ \ \isacommand{hence}\isamarkupfalse%
\ rank{\isacharunderscore}{\kern0pt}t{\isacharunderscore}{\kern0pt}lb{\isacharcolon}{\kern0pt}\ {\isachardoublequoteopen}a\ {\isacharplus}{\kern0pt}\ {\isadigit{1}}\ {\isasymle}\ y{\isachardoublequoteclose}\ \isanewline
\ \ \ \ \ \ \ \ \isacommand{by}\isamarkupfalse%
\ linarith\isanewline
\ \ \ \ \isanewline
\ \ \ \ \ \ \isacommand{have}\isamarkupfalse%
\ rank{\isacharunderscore}{\kern0pt}t{\isacharunderscore}{\kern0pt}ub{\isacharcolon}{\kern0pt}\ {\isachardoublequoteopen}y\ {\isasymle}\ b{\isachardoublequoteclose}\ \isanewline
\ \ \ \ \ \ \ \ \isacommand{apply}\isamarkupfalse%
\ {\isacharparenleft}{\kern0pt}subst\ y{\isacharunderscore}{\kern0pt}def{\isacharcomma}{\kern0pt}\ rule\ nth{\isacharunderscore}{\kern0pt}mset{\isacharunderscore}{\kern0pt}bound{\isacharunderscore}{\kern0pt}right{\isacharparenright}{\kern0pt}\isanewline
\ \ \ \ \ \ \ \ \ \isacommand{apply}\isamarkupfalse%
\ simp\ \isacommand{using}\isamarkupfalse%
\ True\ t{\isacharunderscore}{\kern0pt}ge{\isacharunderscore}{\kern0pt}{\isadigit{0}}\ \isacommand{apply}\isamarkupfalse%
\ linarith\isanewline
\ \ \ \ \ \ \ \ \isacommand{using}\isamarkupfalse%
\ ub\ t{\isacharunderscore}{\kern0pt}ge{\isacharunderscore}{\kern0pt}{\isadigit{0}}\isanewline
\ \ \ \ \ \ \ \ \isacommand{by}\isamarkupfalse%
\ {\isacharparenleft}{\kern0pt}simp\ add{\isacharcolon}{\kern0pt}f{\isacharunderscore}{\kern0pt}def\ swap{\isacharunderscore}{\kern0pt}filter{\isacharunderscore}{\kern0pt}image\ count{\isacharunderscore}{\kern0pt}le{\isacharunderscore}{\kern0pt}def{\isacharparenright}{\kern0pt}\isanewline
\isanewline
\ \ \ \ \ \ \isacommand{have}\isamarkupfalse%
\ y{\isacharunderscore}{\kern0pt}ge{\isacharunderscore}{\kern0pt}{\isadigit{0}}{\isacharcolon}{\kern0pt}\ {\isachardoublequoteopen}real{\isacharunderscore}{\kern0pt}of{\isacharunderscore}{\kern0pt}int\ y\ {\isasymge}\ {\isadigit{0}}{\isachardoublequoteclose}\ \isacommand{using}\isamarkupfalse%
\ rank{\isacharunderscore}{\kern0pt}t{\isacharunderscore}{\kern0pt}lb\ a{\isacharunderscore}{\kern0pt}ge{\isacharunderscore}{\kern0pt}{\isadigit{0}}\ \isacommand{by}\isamarkupfalse%
\ linarith\isanewline
\ \ \ \ \ \ \isacommand{have}\isamarkupfalse%
\ y{\isacharprime}{\kern0pt}{\isacharunderscore}{\kern0pt}eq{\isacharcolon}{\kern0pt}\ {\isachardoublequoteopen}y{\isacharprime}{\kern0pt}\ {\isacharequal}{\kern0pt}\ truncate{\isacharunderscore}{\kern0pt}down\ r\ y{\isachardoublequoteclose}\isanewline
\ \ \ \ \ \ \ \ \isacommand{apply}\isamarkupfalse%
\ {\isacharparenleft}{\kern0pt}subst\ y{\isacharunderscore}{\kern0pt}def{\isacharcomma}{\kern0pt}\ subst\ y{\isacharprime}{\kern0pt}{\isacharunderscore}{\kern0pt}def{\isacharcomma}{\kern0pt}\ subst\ nth{\isacharunderscore}{\kern0pt}mset{\isacharunderscore}{\kern0pt}commute{\isacharunderscore}{\kern0pt}mono{\isacharbrackleft}{\kern0pt}\isakeyword{where}\ f{\isacharequal}{\kern0pt}{\isachardoublequoteopen}{\isacharparenleft}{\kern0pt}{\isasymlambda}x{\isachardot}{\kern0pt}\ truncate{\isacharunderscore}{\kern0pt}down\ r\ {\isacharparenleft}{\kern0pt}of{\isacharunderscore}{\kern0pt}int\ x{\isacharparenright}{\kern0pt}{\isacharparenright}{\kern0pt}{\isachardoublequoteclose}{\isacharbrackright}{\kern0pt}{\isacharparenright}{\kern0pt}\ \isanewline
\ \ \ \ \ \ \ \ \ \ \isacommand{apply}\isamarkupfalse%
\ {\isacharparenleft}{\kern0pt}metis\ truncate{\isacharunderscore}{\kern0pt}down{\isacharunderscore}{\kern0pt}mono\ mono{\isacharunderscore}{\kern0pt}def\ of{\isacharunderscore}{\kern0pt}int{\isacharunderscore}{\kern0pt}le{\isacharunderscore}{\kern0pt}iff{\isacharparenright}{\kern0pt}\isanewline
\ \ \ \ \ \ \ \ \ \isacommand{apply}\isamarkupfalse%
\ simp\ \isacommand{using}\isamarkupfalse%
\ True\ t{\isacharunderscore}{\kern0pt}ge{\isacharunderscore}{\kern0pt}{\isadigit{0}}\ \isacommand{apply}\isamarkupfalse%
\ linarith\isanewline
\ \ \ \ \ \ \ \ \isacommand{by}\isamarkupfalse%
\ {\isacharparenleft}{\kern0pt}simp\ add{\isacharcolon}{\kern0pt}\ multiset{\isachardot}{\kern0pt}map{\isacharunderscore}{\kern0pt}comp\ comp{\isacharunderscore}{\kern0pt}def{\isacharparenright}{\kern0pt}\isanewline
\ \ \ \ \ \ \isacommand{have}\isamarkupfalse%
\ {\isachardoublequoteopen}real{\isacharunderscore}{\kern0pt}of{\isacharunderscore}{\kern0pt}int\ {\isacharparenleft}{\kern0pt}a{\isacharplus}{\kern0pt}{\isadigit{1}}{\isacharparenright}{\kern0pt}\ {\isacharasterisk}{\kern0pt}\ {\isacharparenleft}{\kern0pt}{\isadigit{1}}\ {\isacharminus}{\kern0pt}\ {\isadigit{2}}\ powr\ {\isacharminus}{\kern0pt}real\ r{\isacharparenright}{\kern0pt}\ {\isasymle}\ real{\isacharunderscore}{\kern0pt}of{\isacharunderscore}{\kern0pt}int\ y\ {\isacharasterisk}{\kern0pt}\ {\isacharparenleft}{\kern0pt}{\isadigit{1}}\ {\isacharminus}{\kern0pt}\ {\isadigit{2}}\ powr\ {\isacharparenleft}{\kern0pt}{\isacharminus}{\kern0pt}real\ r{\isacharparenright}{\kern0pt}{\isacharparenright}{\kern0pt}{\isachardoublequoteclose}\isanewline
\ \ \ \ \ \ \ \ \isacommand{apply}\isamarkupfalse%
\ {\isacharparenleft}{\kern0pt}rule\ mult{\isacharunderscore}{\kern0pt}right{\isacharunderscore}{\kern0pt}mono{\isacharparenright}{\kern0pt}\isanewline
\ \ \ \ \ \ \ \ \isacommand{using}\isamarkupfalse%
\ rank{\isacharunderscore}{\kern0pt}t{\isacharunderscore}{\kern0pt}lb\ of{\isacharunderscore}{\kern0pt}int{\isacharunderscore}{\kern0pt}le{\isacharunderscore}{\kern0pt}iff\ \isacommand{apply}\isamarkupfalse%
\ blast\isanewline
\ \ \ \ \ \ \ \ \isacommand{apply}\isamarkupfalse%
\ simp\isanewline
\ \ \ \ \ \ \ \ \isacommand{apply}\isamarkupfalse%
\ {\isacharparenleft}{\kern0pt}subst\ two{\isacharunderscore}{\kern0pt}powr{\isacharunderscore}{\kern0pt}{\isadigit{0}}{\isacharbrackleft}{\kern0pt}symmetric{\isacharbrackright}{\kern0pt}{\isacharparenright}{\kern0pt}\isanewline
\ \ \ \ \ \ \ \ \isacommand{by}\isamarkupfalse%
\ {\isacharparenleft}{\kern0pt}rule\ powr{\isacharunderscore}{\kern0pt}mono{\isacharcomma}{\kern0pt}\ simp{\isacharcomma}{\kern0pt}\ simp{\isacharparenright}{\kern0pt}\isanewline
\ \ \ \ \ \ \isacommand{also}\isamarkupfalse%
\ \isacommand{have}\isamarkupfalse%
\ {\isachardoublequoteopen}{\isachardot}{\kern0pt}{\isachardot}{\kern0pt}{\isachardot}{\kern0pt}\ {\isasymle}\ y{\isacharprime}{\kern0pt}{\isachardoublequoteclose}\isanewline
\ \ \ \ \ \ \ \ \isacommand{apply}\isamarkupfalse%
\ {\isacharparenleft}{\kern0pt}subst\ y{\isacharprime}{\kern0pt}{\isacharunderscore}{\kern0pt}eq{\isacharparenright}{\kern0pt}\isanewline
\ \ \ \ \ \ \ \ \isacommand{using}\isamarkupfalse%
\ truncate{\isacharunderscore}{\kern0pt}down{\isacharunderscore}{\kern0pt}pos{\isacharbrackleft}{\kern0pt}OF\ y{\isacharunderscore}{\kern0pt}ge{\isacharunderscore}{\kern0pt}{\isadigit{0}}{\isacharbrackright}{\kern0pt}\ \isacommand{by}\isamarkupfalse%
\ simp\isanewline
\ \ \ \ \ \ \isacommand{finally}\isamarkupfalse%
\ \isacommand{have}\isamarkupfalse%
\ rank{\isacharunderscore}{\kern0pt}t{\isacharunderscore}{\kern0pt}lb{\isacharprime}{\kern0pt}{\isacharcolon}{\kern0pt}\ {\isachardoublequoteopen}{\isacharparenleft}{\kern0pt}a{\isacharplus}{\kern0pt}{\isadigit{1}}{\isacharparenright}{\kern0pt}\ {\isacharasterisk}{\kern0pt}\ {\isacharparenleft}{\kern0pt}{\isadigit{1}}\ {\isacharminus}{\kern0pt}\ {\isadigit{2}}\ powr\ {\isacharparenleft}{\kern0pt}{\isacharminus}{\kern0pt}real\ r{\isacharparenright}{\kern0pt}{\isacharparenright}{\kern0pt}\ {\isasymle}\ y{\isacharprime}{\kern0pt}{\isachardoublequoteclose}\ \isacommand{by}\isamarkupfalse%
\ simp\isanewline
\isanewline
\ \ \ \ \ \ \isacommand{have}\isamarkupfalse%
\ {\isachardoublequoteopen}y{\isacharprime}{\kern0pt}\ {\isasymle}\ real{\isacharunderscore}{\kern0pt}of{\isacharunderscore}{\kern0pt}int\ y{\isachardoublequoteclose}\isanewline
\ \ \ \ \ \ \ \ \isacommand{by}\isamarkupfalse%
\ {\isacharparenleft}{\kern0pt}subst\ y{\isacharprime}{\kern0pt}{\isacharunderscore}{\kern0pt}eq{\isacharcomma}{\kern0pt}\ rule\ truncate{\isacharunderscore}{\kern0pt}down{\isacharunderscore}{\kern0pt}le{\isacharcomma}{\kern0pt}\ simp{\isacharparenright}{\kern0pt}\isanewline
\ \ \ \ \ \ \isacommand{also}\isamarkupfalse%
\ \isacommand{have}\isamarkupfalse%
\ {\isachardoublequoteopen}{\isachardot}{\kern0pt}{\isachardot}{\kern0pt}{\isachardot}{\kern0pt}\ {\isasymle}\ real{\isacharunderscore}{\kern0pt}of{\isacharunderscore}{\kern0pt}int\ b{\isachardoublequoteclose}\isanewline
\ \ \ \ \ \ \ \ \isacommand{using}\isamarkupfalse%
\ rank{\isacharunderscore}{\kern0pt}t{\isacharunderscore}{\kern0pt}ub\ of{\isacharunderscore}{\kern0pt}int{\isacharunderscore}{\kern0pt}le{\isacharunderscore}{\kern0pt}iff\ \isacommand{by}\isamarkupfalse%
\ blast\isanewline
\ \ \ \ \ \ \isacommand{finally}\isamarkupfalse%
\ \isacommand{have}\isamarkupfalse%
\ rank{\isacharunderscore}{\kern0pt}t{\isacharunderscore}{\kern0pt}ub{\isacharprime}{\kern0pt}{\isacharcolon}{\kern0pt}\ {\isachardoublequoteopen}y{\isacharprime}{\kern0pt}\ {\isasymle}\ b{\isachardoublequoteclose}\isanewline
\ \ \ \ \ \ \ \ \isacommand{by}\isamarkupfalse%
\ simp\isanewline
\isanewline
\ \ \ \ \ \ \isacommand{have}\isamarkupfalse%
\ {\isachardoublequoteopen}{\isadigit{0}}\ {\isacharless}{\kern0pt}\ {\isacharparenleft}{\kern0pt}a{\isacharplus}{\kern0pt}{\isadigit{1}}{\isacharparenright}{\kern0pt}\ {\isacharasterisk}{\kern0pt}\ {\isacharparenleft}{\kern0pt}{\isadigit{1}}{\isacharminus}{\kern0pt}{\isadigit{2}}\ powr\ {\isacharparenleft}{\kern0pt}{\isacharminus}{\kern0pt}real\ r{\isacharparenright}{\kern0pt}{\isacharparenright}{\kern0pt}{\isachardoublequoteclose}\isanewline
\ \ \ \ \ \ \ \ \isacommand{apply}\isamarkupfalse%
\ {\isacharparenleft}{\kern0pt}rule\ mult{\isacharunderscore}{\kern0pt}pos{\isacharunderscore}{\kern0pt}pos{\isacharparenright}{\kern0pt}\isanewline
\ \ \ \ \ \ \ \ \isacommand{using}\isamarkupfalse%
\ a{\isacharunderscore}{\kern0pt}ge{\isacharunderscore}{\kern0pt}{\isadigit{0}}\ \isacommand{apply}\isamarkupfalse%
\ linarith\isanewline
\ \ \ \ \ \ \ \ \isacommand{apply}\isamarkupfalse%
\ simp\isanewline
\ \ \ \ \ \ \ \ \isacommand{apply}\isamarkupfalse%
\ {\isacharparenleft}{\kern0pt}subst\ two{\isacharunderscore}{\kern0pt}powr{\isacharunderscore}{\kern0pt}{\isadigit{0}}{\isacharbrackleft}{\kern0pt}symmetric{\isacharbrackright}{\kern0pt}{\isacharparenright}{\kern0pt}\isanewline
\ \ \ \ \ \ \ \ \isacommand{apply}\isamarkupfalse%
\ {\isacharparenleft}{\kern0pt}rule\ powr{\isacharunderscore}{\kern0pt}less{\isacharunderscore}{\kern0pt}mono{\isacharparenright}{\kern0pt}\isanewline
\ \ \ \ \ \ \ \ \isacommand{using}\isamarkupfalse%
\ r{\isacharunderscore}{\kern0pt}ge{\isacharunderscore}{\kern0pt}{\isadigit{0}}\ \isacommand{by}\isamarkupfalse%
\ auto\isanewline
\ \ \ \ \ \ \isacommand{hence}\isamarkupfalse%
\ y{\isacharprime}{\kern0pt}{\isacharunderscore}{\kern0pt}pos{\isacharcolon}{\kern0pt}\ {\isachardoublequoteopen}y{\isacharprime}{\kern0pt}\ {\isachargreater}{\kern0pt}\ {\isadigit{0}}{\isachardoublequoteclose}\ \isacommand{using}\isamarkupfalse%
\ rank{\isacharunderscore}{\kern0pt}t{\isacharunderscore}{\kern0pt}lb{\isacharprime}{\kern0pt}\ \isacommand{by}\isamarkupfalse%
\ linarith\isanewline
\isanewline
\ \ \ \ \ \ \isacommand{have}\isamarkupfalse%
\ no{\isacharunderscore}{\kern0pt}col{\isacharprime}{\kern0pt}{\isacharcolon}{\kern0pt}\ {\isachardoublequoteopen}{\isasymAnd}x{\isachardot}{\kern0pt}\ x\ {\isasymle}\ y{\isacharprime}{\kern0pt}\ {\isasymLongrightarrow}\ count\ {\isacharbraceleft}{\kern0pt}{\isacharhash}{\kern0pt}truncate{\isacharunderscore}{\kern0pt}down\ r\ {\isacharparenleft}{\kern0pt}real\ {\isacharparenleft}{\kern0pt}hash\ p\ x\ {\isasymomega}{\isacharparenright}{\kern0pt}{\isacharparenright}{\kern0pt}{\isachardot}{\kern0pt}\ x\ {\isasymin}{\isacharhash}{\kern0pt}\ mset{\isacharunderscore}{\kern0pt}set\ {\isacharparenleft}{\kern0pt}set\ as{\isacharparenright}{\kern0pt}{\isacharhash}{\kern0pt}{\isacharbraceright}{\kern0pt}\ x\ {\isasymle}\ {\isadigit{1}}{\isachardoublequoteclose}\isanewline
\ \ \ \ \ \ \ \ \isacommand{apply}\isamarkupfalse%
\ {\isacharparenleft}{\kern0pt}subst\ count{\isacharunderscore}{\kern0pt}image{\isacharunderscore}{\kern0pt}mset{\isacharcomma}{\kern0pt}\ simp\ add{\isacharcolon}{\kern0pt}vimage{\isacharunderscore}{\kern0pt}def\ card{\isacharunderscore}{\kern0pt}le{\isacharunderscore}{\kern0pt}Suc{\isadigit{0}}{\isacharunderscore}{\kern0pt}iff{\isacharunderscore}{\kern0pt}eq{\isacharparenright}{\kern0pt}\isanewline
\ \ \ \ \ \ \ \ \isacommand{using}\isamarkupfalse%
\ \ rank{\isacharunderscore}{\kern0pt}t{\isacharunderscore}{\kern0pt}ub{\isacharprime}{\kern0pt}\ no{\isacharunderscore}{\kern0pt}col\ \isacommand{apply}\isamarkupfalse%
\ {\isacharparenleft}{\kern0pt}subst\ {\isacharparenleft}{\kern0pt}asm{\isacharparenright}{\kern0pt}\ has{\isacharunderscore}{\kern0pt}no{\isacharunderscore}{\kern0pt}collision{\isacharunderscore}{\kern0pt}def{\isacharparenright}{\kern0pt}\isanewline
\ \ \ \ \ \ \ \ \isacommand{by}\isamarkupfalse%
\ force\isanewline
\isanewline
\ \ \ \ \ \ \isacommand{have}\isamarkupfalse%
\ h{\isacharunderscore}{\kern0pt}{\isadigit{1}}{\isacharcolon}{\kern0pt}\ {\isachardoublequoteopen}Max\ {\isacharparenleft}{\kern0pt}h\ {\isasymomega}{\isacharparenright}{\kern0pt}\ {\isacharequal}{\kern0pt}\ y{\isacharprime}{\kern0pt}{\isachardoublequoteclose}\isanewline
\ \ \ \ \ \ \ \ \isacommand{apply}\isamarkupfalse%
\ {\isacharparenleft}{\kern0pt}simp\ add{\isacharcolon}{\kern0pt}h{\isacharunderscore}{\kern0pt}def\ y{\isacharprime}{\kern0pt}{\isacharunderscore}{\kern0pt}def{\isacharparenright}{\kern0pt}\isanewline
\ \ \ \ \ \ \ \ \isacommand{apply}\isamarkupfalse%
\ {\isacharparenleft}{\kern0pt}subst\ nth{\isacharunderscore}{\kern0pt}mset{\isacharunderscore}{\kern0pt}max{\isacharparenright}{\kern0pt}\isanewline
\ \ \ \ \ \ \ \ \isacommand{using}\isamarkupfalse%
\ True\ t{\isacharunderscore}{\kern0pt}ge{\isacharunderscore}{\kern0pt}{\isadigit{0}}\ \isacommand{apply}\isamarkupfalse%
\ simp\isanewline
\ \ \ \ \ \ \ \ \isacommand{using}\isamarkupfalse%
\ no{\isacharunderscore}{\kern0pt}col{\isacharprime}{\kern0pt}\ \isacommand{apply}\isamarkupfalse%
\ {\isacharparenleft}{\kern0pt}simp\ add{\isacharcolon}{\kern0pt}y{\isacharprime}{\kern0pt}{\isacharunderscore}{\kern0pt}def{\isacharparenright}{\kern0pt}\isanewline
\ \ \ \ \ \ \ \ \isacommand{using}\isamarkupfalse%
\ t{\isacharunderscore}{\kern0pt}ge{\isacharunderscore}{\kern0pt}{\isadigit{0}}\isanewline
\ \ \ \ \ \ \ \ \isacommand{by}\isamarkupfalse%
\ simp\isanewline
\isanewline
\ \ \ \ \ \ \isacommand{have}\isamarkupfalse%
\ {\isachardoublequoteopen}card\ {\isacharparenleft}{\kern0pt}h\ {\isasymomega}{\isacharparenright}{\kern0pt}\ {\isacharequal}{\kern0pt}\ card\ {\isacharparenleft}{\kern0pt}least\ {\isacharparenleft}{\kern0pt}{\isacharparenleft}{\kern0pt}t{\isacharminus}{\kern0pt}{\isadigit{1}}{\isacharparenright}{\kern0pt}{\isacharplus}{\kern0pt}{\isadigit{1}}{\isacharparenright}{\kern0pt}\ {\isacharparenleft}{\kern0pt}set{\isacharunderscore}{\kern0pt}mset\ {\isacharbraceleft}{\kern0pt}{\isacharhash}{\kern0pt}truncate{\isacharunderscore}{\kern0pt}down\ r\ {\isacharparenleft}{\kern0pt}hash\ p\ x\ {\isasymomega}{\isacharparenright}{\kern0pt}{\isachardot}{\kern0pt}\ x\ {\isasymin}{\isacharhash}{\kern0pt}\ mset{\isacharunderscore}{\kern0pt}set\ {\isacharparenleft}{\kern0pt}set\ as{\isacharparenright}{\kern0pt}{\isacharhash}{\kern0pt}{\isacharbraceright}{\kern0pt}{\isacharparenright}{\kern0pt}{\isacharparenright}{\kern0pt}{\isachardoublequoteclose}\isanewline
\ \ \ \ \ \ \ \ \isacommand{using}\isamarkupfalse%
\ t{\isacharunderscore}{\kern0pt}ge{\isacharunderscore}{\kern0pt}{\isadigit{0}}\isanewline
\ \ \ \ \ \ \ \ \isacommand{by}\isamarkupfalse%
\ {\isacharparenleft}{\kern0pt}simp\ add{\isacharcolon}{\kern0pt}h{\isacharunderscore}{\kern0pt}def{\isacharparenright}{\kern0pt}\isanewline
\ \ \ \ \ \ \isacommand{also}\isamarkupfalse%
\ \isacommand{have}\isamarkupfalse%
\ {\isachardoublequoteopen}{\isachardot}{\kern0pt}{\isachardot}{\kern0pt}{\isachardot}{\kern0pt}\ {\isacharequal}{\kern0pt}\ {\isacharparenleft}{\kern0pt}t{\isacharminus}{\kern0pt}{\isadigit{1}}{\isacharparenright}{\kern0pt}\ {\isacharplus}{\kern0pt}{\isadigit{1}}{\isachardoublequoteclose}\isanewline
\ \ \ \ \ \ \ \ \isacommand{apply}\isamarkupfalse%
\ {\isacharparenleft}{\kern0pt}rule\ nth{\isacharunderscore}{\kern0pt}mset{\isacharunderscore}{\kern0pt}max{\isacharparenleft}{\kern0pt}{\isadigit{2}}{\isacharparenright}{\kern0pt}{\isacharparenright}{\kern0pt}\isanewline
\ \ \ \ \ \ \ \ \ \isacommand{using}\isamarkupfalse%
\ True\ t{\isacharunderscore}{\kern0pt}ge{\isacharunderscore}{\kern0pt}{\isadigit{0}}\ \isacommand{apply}\isamarkupfalse%
\ simp\isanewline
\ \ \ \ \ \ \ \ \isacommand{using}\isamarkupfalse%
\ no{\isacharunderscore}{\kern0pt}col{\isacharprime}{\kern0pt}\ \isacommand{by}\isamarkupfalse%
\ {\isacharparenleft}{\kern0pt}simp\ add{\isacharcolon}{\kern0pt}y{\isacharprime}{\kern0pt}{\isacharunderscore}{\kern0pt}def{\isacharparenright}{\kern0pt}\isanewline
\ \ \ \ \ \ \isacommand{also}\isamarkupfalse%
\ \isacommand{have}\isamarkupfalse%
\ {\isachardoublequoteopen}{\isachardot}{\kern0pt}{\isachardot}{\kern0pt}{\isachardot}{\kern0pt}\ {\isacharequal}{\kern0pt}\ t{\isachardoublequoteclose}\ \isacommand{using}\isamarkupfalse%
\ t{\isacharunderscore}{\kern0pt}ge{\isacharunderscore}{\kern0pt}{\isadigit{0}}\ \isacommand{by}\isamarkupfalse%
\ simp\isanewline
\ \ \ \ \ \ \isacommand{finally}\isamarkupfalse%
\ \isacommand{have}\isamarkupfalse%
\ h{\isacharunderscore}{\kern0pt}{\isadigit{2}}{\isacharcolon}{\kern0pt}\ {\isachardoublequoteopen}card\ {\isacharparenleft}{\kern0pt}h\ {\isasymomega}{\isacharparenright}{\kern0pt}\ {\isacharequal}{\kern0pt}\ t{\isachardoublequoteclose}\isanewline
\ \ \ \ \ \ \ \ \isacommand{by}\isamarkupfalse%
\ simp\isanewline
\ \ \ \ \ \ \isacommand{have}\isamarkupfalse%
\ h{\isacharunderscore}{\kern0pt}{\isadigit{3}}{\isacharcolon}{\kern0pt}\ {\isachardoublequoteopen}g{\isacharprime}{\kern0pt}\ {\isacharparenleft}{\kern0pt}h\ {\isasymomega}{\isacharparenright}{\kern0pt}\ {\isacharequal}{\kern0pt}\ real\ t\ {\isacharasterisk}{\kern0pt}\ real\ p\ {\isacharslash}{\kern0pt}\ y{\isacharprime}{\kern0pt}{\isachardoublequoteclose}\isanewline
\ \ \ \ \ \ \ \ \isacommand{using}\isamarkupfalse%
\ h{\isacharunderscore}{\kern0pt}{\isadigit{2}}\ h{\isacharunderscore}{\kern0pt}{\isadigit{1}}\ \isacommand{by}\isamarkupfalse%
\ {\isacharparenleft}{\kern0pt}simp\ add{\isacharcolon}{\kern0pt}g{\isacharprime}{\kern0pt}{\isacharunderscore}{\kern0pt}def{\isacharparenright}{\kern0pt}\isanewline
\isanewline
\ \ \ \ \ \ \isacommand{have}\isamarkupfalse%
\ {\isachardoublequoteopen}{\isacharparenleft}{\kern0pt}real\ t{\isacharparenright}{\kern0pt}\ {\isacharasterisk}{\kern0pt}\ real\ p\ {\isasymle}\ \ {\isacharparenleft}{\kern0pt}{\isadigit{1}}\ {\isacharplus}{\kern0pt}\ {\isasymdelta}{\isacharprime}{\kern0pt}{\isacharparenright}{\kern0pt}\ {\isacharasterisk}{\kern0pt}\ real\ m\ {\isacharasterisk}{\kern0pt}\ {\isacharparenleft}{\kern0pt}{\isacharparenleft}{\kern0pt}real\ t{\isacharparenright}{\kern0pt}\ {\isacharasterisk}{\kern0pt}\ real\ p\ {\isacharslash}{\kern0pt}\ {\isacharparenleft}{\kern0pt}real\ m\ {\isacharasterisk}{\kern0pt}\ {\isacharparenleft}{\kern0pt}{\isadigit{1}}\ {\isacharplus}{\kern0pt}\ {\isasymdelta}{\isacharprime}{\kern0pt}{\isacharparenright}{\kern0pt}{\isacharparenright}{\kern0pt}{\isacharparenright}{\kern0pt}{\isachardoublequoteclose}\ \isanewline
\ \ \ \ \ \ \ \ \isacommand{apply}\isamarkupfalse%
\ {\isacharparenleft}{\kern0pt}simp{\isacharparenright}{\kern0pt}\isanewline
\ \ \ \ \ \ \ \ \isacommand{using}\isamarkupfalse%
\ {\isasymdelta}{\isacharprime}{\kern0pt}{\isacharunderscore}{\kern0pt}le{\isacharunderscore}{\kern0pt}{\isadigit{1}}\ m{\isacharunderscore}{\kern0pt}def\ True\ t{\isacharunderscore}{\kern0pt}ge{\isacharunderscore}{\kern0pt}{\isadigit{0}}\ {\isasymdelta}{\isacharprime}{\kern0pt}{\isacharunderscore}{\kern0pt}ge{\isacharunderscore}{\kern0pt}{\isadigit{0}}\ \isacommand{by}\isamarkupfalse%
\ linarith\isanewline
\ \ \ \ \ \ \isacommand{also}\isamarkupfalse%
\ \isacommand{have}\isamarkupfalse%
\ {\isachardoublequoteopen}{\isachardot}{\kern0pt}{\isachardot}{\kern0pt}{\isachardot}{\kern0pt}\ {\isasymle}\ {\isacharparenleft}{\kern0pt}{\isadigit{1}}{\isacharplus}{\kern0pt}{\isasymdelta}{\isacharprime}{\kern0pt}{\isacharparenright}{\kern0pt}\ {\isacharasterisk}{\kern0pt}\ m\ {\isacharasterisk}{\kern0pt}\ {\isacharparenleft}{\kern0pt}a{\isacharplus}{\kern0pt}{\isadigit{1}}{\isacharparenright}{\kern0pt}{\isachardoublequoteclose}\isanewline
\ \ \ \ \ \ \ \ \isacommand{apply}\isamarkupfalse%
\ {\isacharparenleft}{\kern0pt}rule\ mult{\isacharunderscore}{\kern0pt}left{\isacharunderscore}{\kern0pt}mono{\isacharparenright}{\kern0pt}\isanewline
\ \ \ \ \ \ \ \ \ \isacommand{apply}\isamarkupfalse%
\ {\isacharparenleft}{\kern0pt}simp\ add{\isacharcolon}{\kern0pt}a{\isacharunderscore}{\kern0pt}def{\isacharparenright}{\kern0pt}\isanewline
\ \ \ \ \ \ \ \ \isacommand{using}\isamarkupfalse%
\ {\isasymdelta}{\isacharprime}{\kern0pt}{\isacharunderscore}{\kern0pt}ge{\isacharunderscore}{\kern0pt}{\isadigit{0}}\ \isacommand{by}\isamarkupfalse%
\ simp\isanewline
\ \ \ \ \ \ \isacommand{also}\isamarkupfalse%
\ \isacommand{have}\isamarkupfalse%
\ {\isachardoublequoteopen}{\isachardot}{\kern0pt}{\isachardot}{\kern0pt}{\isachardot}{\kern0pt}\ {\isacharless}{\kern0pt}\ {\isacharparenleft}{\kern0pt}{\isacharparenleft}{\kern0pt}{\isadigit{1}}\ {\isacharplus}{\kern0pt}\ real{\isacharunderscore}{\kern0pt}of{\isacharunderscore}{\kern0pt}rat\ {\isasymdelta}{\isacharparenright}{\kern0pt}{\isacharasterisk}{\kern0pt}{\isacharparenleft}{\kern0pt}{\isadigit{1}}{\isacharminus}{\kern0pt}real{\isacharunderscore}{\kern0pt}of{\isacharunderscore}{\kern0pt}rat\ {\isasymdelta}{\isacharslash}{\kern0pt}{\isadigit{8}}{\isacharparenright}{\kern0pt}{\isacharparenright}{\kern0pt}\ {\isacharasterisk}{\kern0pt}\ m\ {\isacharasterisk}{\kern0pt}\ {\isacharparenleft}{\kern0pt}a{\isacharplus}{\kern0pt}{\isadigit{1}}{\isacharparenright}{\kern0pt}{\isachardoublequoteclose}\isanewline
\ \ \ \ \ \ \ \ \isacommand{apply}\isamarkupfalse%
\ {\isacharparenleft}{\kern0pt}rule\ mult{\isacharunderscore}{\kern0pt}strict{\isacharunderscore}{\kern0pt}right{\isacharunderscore}{\kern0pt}mono{\isacharparenright}{\kern0pt}\isanewline
\ \ \ \ \ \ \ \ \ \isacommand{apply}\isamarkupfalse%
\ {\isacharparenleft}{\kern0pt}rule\ mult{\isacharunderscore}{\kern0pt}strict{\isacharunderscore}{\kern0pt}right{\isacharunderscore}{\kern0pt}mono{\isacharparenright}{\kern0pt}\isanewline
\ \ \ \ \ \ \ \ \ \ \isacommand{apply}\isamarkupfalse%
\ {\isacharparenleft}{\kern0pt}simp\ add{\isacharcolon}{\kern0pt}{\isasymdelta}{\isacharprime}{\kern0pt}{\isacharunderscore}{\kern0pt}def\ distrib{\isacharunderscore}{\kern0pt}left\ distrib{\isacharunderscore}{\kern0pt}right\ left{\isacharunderscore}{\kern0pt}diff{\isacharunderscore}{\kern0pt}distrib\ right{\isacharunderscore}{\kern0pt}diff{\isacharunderscore}{\kern0pt}distrib{\isacharparenright}{\kern0pt}\isanewline
\ \ \ \ \ \ \ \ \isacommand{using}\isamarkupfalse%
\ True\ m{\isacharunderscore}{\kern0pt}def\ t{\isacharunderscore}{\kern0pt}ge{\isacharunderscore}{\kern0pt}{\isadigit{0}}\ a{\isacharunderscore}{\kern0pt}ge{\isacharunderscore}{\kern0pt}{\isadigit{0}}\ assms{\isacharparenleft}{\kern0pt}{\isadigit{2}}{\isacharparenright}{\kern0pt}\ \isacommand{by}\isamarkupfalse%
\ auto\isanewline
\ \ \ \ \ \ \isacommand{also}\isamarkupfalse%
\ \isacommand{have}\isamarkupfalse%
\ {\isachardoublequoteopen}{\isachardot}{\kern0pt}{\isachardot}{\kern0pt}{\isachardot}{\kern0pt}\ {\isasymle}\ {\isacharparenleft}{\kern0pt}{\isacharparenleft}{\kern0pt}{\isadigit{1}}\ {\isacharplus}{\kern0pt}\ real{\isacharunderscore}{\kern0pt}of{\isacharunderscore}{\kern0pt}rat\ {\isasymdelta}{\isacharparenright}{\kern0pt}{\isacharasterisk}{\kern0pt}{\isacharparenleft}{\kern0pt}{\isadigit{1}}{\isacharminus}{\kern0pt}{\isadigit{2}}\ powr\ {\isacharparenleft}{\kern0pt}{\isacharminus}{\kern0pt}r{\isacharparenright}{\kern0pt}{\isacharparenright}{\kern0pt}{\isacharparenright}{\kern0pt}\ {\isacharasterisk}{\kern0pt}\ m\ {\isacharasterisk}{\kern0pt}\ {\isacharparenleft}{\kern0pt}a{\isacharplus}{\kern0pt}{\isadigit{1}}{\isacharparenright}{\kern0pt}{\isachardoublequoteclose}\isanewline
\ \ \ \ \ \ \ \ \isacommand{apply}\isamarkupfalse%
\ {\isacharparenleft}{\kern0pt}rule\ mult{\isacharunderscore}{\kern0pt}right{\isacharunderscore}{\kern0pt}mono{\isacharparenright}{\kern0pt}\isanewline
\ \ \ \ \ \ \ \ \ \isacommand{apply}\isamarkupfalse%
\ {\isacharparenleft}{\kern0pt}rule\ mult{\isacharunderscore}{\kern0pt}right{\isacharunderscore}{\kern0pt}mono{\isacharparenright}{\kern0pt}\isanewline
\ \ \ \ \ \ \ \ \ \ \isacommand{apply}\isamarkupfalse%
\ {\isacharparenleft}{\kern0pt}rule\ mult{\isacharunderscore}{\kern0pt}left{\isacharunderscore}{\kern0pt}mono{\isacharparenright}{\kern0pt}\isanewline
\ \ \ \ \ \ \ \ \isacommand{using}\isamarkupfalse%
\ r{\isacharunderscore}{\kern0pt}le{\isacharunderscore}{\kern0pt}{\isasymdelta}\ assms{\isacharparenleft}{\kern0pt}{\isadigit{2}}{\isacharparenright}{\kern0pt}\ a{\isacharunderscore}{\kern0pt}ge{\isacharunderscore}{\kern0pt}{\isadigit{0}}\ \isacommand{by}\isamarkupfalse%
\ auto\isanewline
\ \ \ \ \ \ \isacommand{also}\isamarkupfalse%
\ \isacommand{have}\isamarkupfalse%
\ {\isachardoublequoteopen}{\isachardot}{\kern0pt}{\isachardot}{\kern0pt}{\isachardot}{\kern0pt}\ {\isacharequal}{\kern0pt}\ {\isacharparenleft}{\kern0pt}{\isadigit{1}}\ {\isacharplus}{\kern0pt}\ real{\isacharunderscore}{\kern0pt}of{\isacharunderscore}{\kern0pt}rat\ {\isasymdelta}{\isacharparenright}{\kern0pt}\ {\isacharasterisk}{\kern0pt}\ m\ {\isacharasterisk}{\kern0pt}\ {\isacharparenleft}{\kern0pt}{\isacharparenleft}{\kern0pt}a{\isacharplus}{\kern0pt}{\isadigit{1}}{\isacharparenright}{\kern0pt}\ {\isacharasterisk}{\kern0pt}\ {\isacharparenleft}{\kern0pt}{\isadigit{1}}{\isacharminus}{\kern0pt}{\isadigit{2}}\ powr\ {\isacharparenleft}{\kern0pt}{\isacharminus}{\kern0pt}real\ r{\isacharparenright}{\kern0pt}{\isacharparenright}{\kern0pt}{\isacharparenright}{\kern0pt}{\isachardoublequoteclose}\ \isanewline
\ \ \ \ \ \ \ \ \isacommand{by}\isamarkupfalse%
\ simp\isanewline
\ \ \ \ \ \ \isacommand{also}\isamarkupfalse%
\ \isacommand{have}\isamarkupfalse%
\ {\isachardoublequoteopen}{\isachardot}{\kern0pt}{\isachardot}{\kern0pt}{\isachardot}{\kern0pt}\ {\isasymle}\ {\isacharparenleft}{\kern0pt}{\isadigit{1}}\ {\isacharplus}{\kern0pt}\ real{\isacharunderscore}{\kern0pt}of{\isacharunderscore}{\kern0pt}rat\ {\isasymdelta}{\isacharparenright}{\kern0pt}\ {\isacharasterisk}{\kern0pt}\ m\ {\isacharasterisk}{\kern0pt}\ y{\isacharprime}{\kern0pt}{\isachardoublequoteclose}\isanewline
\ \ \ \ \ \ \ \ \isacommand{apply}\isamarkupfalse%
\ {\isacharparenleft}{\kern0pt}rule\ mult{\isacharunderscore}{\kern0pt}left{\isacharunderscore}{\kern0pt}mono{\isacharcomma}{\kern0pt}\ metis\ rank{\isacharunderscore}{\kern0pt}t{\isacharunderscore}{\kern0pt}lb{\isacharprime}{\kern0pt}{\isacharparenright}{\kern0pt}\isanewline
\ \ \ \ \ \ \ \ \isacommand{using}\isamarkupfalse%
\ assms\ \isacommand{by}\isamarkupfalse%
\ simp\isanewline
\ \ \ \ \ \ \isacommand{finally}\isamarkupfalse%
\ \isacommand{have}\isamarkupfalse%
\ {\isachardoublequoteopen}real\ t\ {\isacharasterisk}{\kern0pt}\ real\ p\ {\isacharless}{\kern0pt}\ {\isacharparenleft}{\kern0pt}{\isadigit{1}}\ {\isacharplus}{\kern0pt}\ real{\isacharunderscore}{\kern0pt}of{\isacharunderscore}{\kern0pt}rat\ {\isasymdelta}{\isacharparenright}{\kern0pt}\ {\isacharasterisk}{\kern0pt}\ m\ {\isacharasterisk}{\kern0pt}\ y{\isacharprime}{\kern0pt}{\isachardoublequoteclose}\ \isacommand{by}\isamarkupfalse%
\ simp\isanewline
\ \ \ \ \ \ \isacommand{hence}\isamarkupfalse%
\ f{\isacharunderscore}{\kern0pt}{\isadigit{1}}{\isacharcolon}{\kern0pt}\ {\isachardoublequoteopen}g{\isacharprime}{\kern0pt}\ {\isacharparenleft}{\kern0pt}h\ {\isasymomega}{\isacharparenright}{\kern0pt}\ {\isacharless}{\kern0pt}\ {\isacharparenleft}{\kern0pt}{\isadigit{1}}\ {\isacharplus}{\kern0pt}\ real{\isacharunderscore}{\kern0pt}of{\isacharunderscore}{\kern0pt}rat\ {\isasymdelta}{\isacharparenright}{\kern0pt}\ {\isacharasterisk}{\kern0pt}\ m{\isachardoublequoteclose}\isanewline
\ \ \ \ \ \ \ \ \isacommand{apply}\isamarkupfalse%
\ {\isacharparenleft}{\kern0pt}simp\ add{\isacharcolon}{\kern0pt}\ h{\isacharunderscore}{\kern0pt}{\isadigit{3}}{\isacharparenright}{\kern0pt}\isanewline
\ \ \ \ \ \ \ \ \isacommand{by}\isamarkupfalse%
\ {\isacharparenleft}{\kern0pt}subst\ pos{\isacharunderscore}{\kern0pt}divide{\isacharunderscore}{\kern0pt}less{\isacharunderscore}{\kern0pt}eq{\isacharcomma}{\kern0pt}\ metis\ y{\isacharprime}{\kern0pt}{\isacharunderscore}{\kern0pt}pos{\isacharcomma}{\kern0pt}\ simp{\isacharparenright}{\kern0pt}\isanewline
\ \ \ \ \ \ \isacommand{have}\isamarkupfalse%
\ {\isachardoublequoteopen}{\isacharparenleft}{\kern0pt}{\isadigit{1}}\ {\isacharminus}{\kern0pt}\ real{\isacharunderscore}{\kern0pt}of{\isacharunderscore}{\kern0pt}rat\ {\isasymdelta}{\isacharparenright}{\kern0pt}\ {\isacharasterisk}{\kern0pt}\ m\ {\isacharasterisk}{\kern0pt}\ y{\isacharprime}{\kern0pt}\ {\isasymle}\ {\isacharparenleft}{\kern0pt}{\isadigit{1}}\ {\isacharminus}{\kern0pt}\ real{\isacharunderscore}{\kern0pt}of{\isacharunderscore}{\kern0pt}rat\ {\isasymdelta}{\isacharparenright}{\kern0pt}\ {\isacharasterisk}{\kern0pt}\ m\ {\isacharasterisk}{\kern0pt}\ b{\isachardoublequoteclose}\ \isanewline
\ \ \ \ \ \ \ \ \isacommand{apply}\isamarkupfalse%
\ {\isacharparenleft}{\kern0pt}rule\ mult{\isacharunderscore}{\kern0pt}left{\isacharunderscore}{\kern0pt}mono{\isacharcomma}{\kern0pt}\ metis\ rank{\isacharunderscore}{\kern0pt}t{\isacharunderscore}{\kern0pt}ub{\isacharprime}{\kern0pt}{\isacharparenright}{\kern0pt}\isanewline
\ \ \ \ \ \ \ \ \isacommand{using}\isamarkupfalse%
\ assms\ \isacommand{by}\isamarkupfalse%
\ simp\isanewline
\ \ \ \ \ \ \isacommand{also}\isamarkupfalse%
\ \isacommand{have}\isamarkupfalse%
\ {\isachardoublequoteopen}{\isachardot}{\kern0pt}{\isachardot}{\kern0pt}{\isachardot}{\kern0pt}\ {\isacharequal}{\kern0pt}\ {\isacharparenleft}{\kern0pt}{\isacharparenleft}{\kern0pt}{\isadigit{1}}{\isacharminus}{\kern0pt}real{\isacharunderscore}{\kern0pt}of{\isacharunderscore}{\kern0pt}rat\ {\isasymdelta}{\isacharparenright}{\kern0pt}{\isacharparenright}{\kern0pt}\ \ {\isacharasterisk}{\kern0pt}\ {\isacharparenleft}{\kern0pt}real\ m\ {\isacharasterisk}{\kern0pt}\ b{\isacharparenright}{\kern0pt}{\isachardoublequoteclose}\isanewline
\ \ \ \ \ \ \ \ \isacommand{by}\isamarkupfalse%
\ simp\isanewline
\ \ \ \ \ \ \isacommand{also}\isamarkupfalse%
\ \isacommand{have}\isamarkupfalse%
\ {\isachardoublequoteopen}{\isachardot}{\kern0pt}{\isachardot}{\kern0pt}{\isachardot}{\kern0pt}\ {\isacharless}{\kern0pt}\ {\isacharparenleft}{\kern0pt}{\isadigit{1}}{\isacharminus}{\kern0pt}{\isasymdelta}{\isacharprime}{\kern0pt}{\isacharparenright}{\kern0pt}\ {\isacharasterisk}{\kern0pt}\ {\isacharparenleft}{\kern0pt}real\ m\ {\isacharasterisk}{\kern0pt}\ b{\isacharparenright}{\kern0pt}{\isachardoublequoteclose}\ \isanewline
\ \ \ \ \ \ \ \ \isacommand{apply}\isamarkupfalse%
\ {\isacharparenleft}{\kern0pt}rule\ mult{\isacharunderscore}{\kern0pt}strict{\isacharunderscore}{\kern0pt}right{\isacharunderscore}{\kern0pt}mono{\isacharparenright}{\kern0pt}\isanewline
\ \ \ \ \ \ \ \ \ \isacommand{apply}\isamarkupfalse%
\ {\isacharparenleft}{\kern0pt}simp\ add{\isacharcolon}{\kern0pt}\ {\isasymdelta}{\isacharprime}{\kern0pt}{\isacharunderscore}{\kern0pt}def\ algebra{\isacharunderscore}{\kern0pt}simps{\isacharparenright}{\kern0pt}\isanewline
\ \ \ \ \ \ \ \ \isacommand{using}\isamarkupfalse%
\ assms\ \isacommand{apply}\isamarkupfalse%
\ simp\isanewline
\ \ \ \ \ \ \ \ \isacommand{using}\isamarkupfalse%
\ r{\isacharunderscore}{\kern0pt}le{\isacharunderscore}{\kern0pt}{\isasymdelta}\ m{\isacharunderscore}{\kern0pt}eq{\isacharunderscore}{\kern0pt}F{\isacharunderscore}{\kern0pt}{\isadigit{0}}\ m{\isacharunderscore}{\kern0pt}ge{\isacharunderscore}{\kern0pt}{\isadigit{0}}\ b{\isacharunderscore}{\kern0pt}ge{\isacharunderscore}{\kern0pt}{\isadigit{0}}\ \isacommand{by}\isamarkupfalse%
\ simp\isanewline
\ \ \ \ \ \ \isacommand{also}\isamarkupfalse%
\ \isacommand{have}\isamarkupfalse%
\ {\isachardoublequoteopen}{\isachardot}{\kern0pt}{\isachardot}{\kern0pt}{\isachardot}{\kern0pt}\ {\isasymle}\ {\isacharparenleft}{\kern0pt}{\isadigit{1}}{\isacharminus}{\kern0pt}{\isasymdelta}{\isacharprime}{\kern0pt}{\isacharparenright}{\kern0pt}\ {\isacharasterisk}{\kern0pt}\ {\isacharparenleft}{\kern0pt}real\ m\ {\isacharasterisk}{\kern0pt}\ {\isacharparenleft}{\kern0pt}real\ t\ {\isacharasterisk}{\kern0pt}\ real\ p\ {\isacharslash}{\kern0pt}\ {\isacharparenleft}{\kern0pt}real\ m\ {\isacharasterisk}{\kern0pt}\ {\isacharparenleft}{\kern0pt}{\isadigit{1}}{\isacharminus}{\kern0pt}{\isasymdelta}{\isacharprime}{\kern0pt}{\isacharparenright}{\kern0pt}{\isacharparenright}{\kern0pt}{\isacharparenright}{\kern0pt}{\isacharparenright}{\kern0pt}{\isachardoublequoteclose}\isanewline
\ \ \ \ \ \ \ \ \isacommand{apply}\isamarkupfalse%
\ {\isacharparenleft}{\kern0pt}rule\ mult{\isacharunderscore}{\kern0pt}left{\isacharunderscore}{\kern0pt}mono{\isacharparenright}{\kern0pt}\isanewline
\ \ \ \ \ \ \ \ \isacommand{apply}\isamarkupfalse%
\ {\isacharparenleft}{\kern0pt}rule\ mult{\isacharunderscore}{\kern0pt}left{\isacharunderscore}{\kern0pt}mono{\isacharparenright}{\kern0pt}\isanewline
\ \ \ \ \ \ \ \ \ \ \isacommand{apply}\isamarkupfalse%
\ {\isacharparenleft}{\kern0pt}simp\ add{\isacharcolon}{\kern0pt}b{\isacharunderscore}{\kern0pt}def{\isacharcomma}{\kern0pt}\ simp{\isacharparenright}{\kern0pt}\isanewline
\ \ \ \ \ \ \ \ \isacommand{using}\isamarkupfalse%
\ {\isasymdelta}{\isacharprime}{\kern0pt}{\isacharunderscore}{\kern0pt}ge{\isacharunderscore}{\kern0pt}{\isadigit{0}}\ {\isasymdelta}{\isacharprime}{\kern0pt}{\isacharunderscore}{\kern0pt}le{\isacharunderscore}{\kern0pt}{\isadigit{1}}\ \isacommand{by}\isamarkupfalse%
\ force\isanewline
\ \ \ \ \ \ \isacommand{also}\isamarkupfalse%
\ \isacommand{have}\isamarkupfalse%
\ {\isachardoublequoteopen}{\isachardot}{\kern0pt}{\isachardot}{\kern0pt}{\isachardot}{\kern0pt}\ {\isacharequal}{\kern0pt}\ real\ t\ {\isacharasterisk}{\kern0pt}\ real\ p{\isachardoublequoteclose}\isanewline
\ \ \ \ \ \ \ \ \isacommand{apply}\isamarkupfalse%
\ {\isacharparenleft}{\kern0pt}simp{\isacharparenright}{\kern0pt}\isanewline
\ \ \ \ \ \ \ \ \isacommand{using}\isamarkupfalse%
\ {\isasymdelta}{\isacharprime}{\kern0pt}{\isacharunderscore}{\kern0pt}ge{\isacharunderscore}{\kern0pt}{\isadigit{0}}\ {\isasymdelta}{\isacharprime}{\kern0pt}{\isacharunderscore}{\kern0pt}le{\isacharunderscore}{\kern0pt}{\isadigit{1}}\ t{\isacharunderscore}{\kern0pt}ge{\isacharunderscore}{\kern0pt}{\isadigit{0}}\ p{\isacharunderscore}{\kern0pt}ge{\isacharunderscore}{\kern0pt}{\isadigit{0}}\ \isacommand{apply}\isamarkupfalse%
\ simp\isanewline
\ \ \ \ \ \ \ \ \isacommand{using}\isamarkupfalse%
\ True\ m{\isacharunderscore}{\kern0pt}def\ order{\isacharunderscore}{\kern0pt}less{\isacharunderscore}{\kern0pt}le{\isacharunderscore}{\kern0pt}trans\ \isacommand{by}\isamarkupfalse%
\ blast\isanewline
\ \ \ \ \ \ \isacommand{finally}\isamarkupfalse%
\ \isacommand{have}\isamarkupfalse%
\ {\isachardoublequoteopen}{\isacharparenleft}{\kern0pt}{\isadigit{1}}\ {\isacharminus}{\kern0pt}\ real{\isacharunderscore}{\kern0pt}of{\isacharunderscore}{\kern0pt}rat\ {\isasymdelta}{\isacharparenright}{\kern0pt}\ {\isacharasterisk}{\kern0pt}\ m\ {\isacharasterisk}{\kern0pt}\ y{\isacharprime}{\kern0pt}\ {\isacharless}{\kern0pt}\ real\ t\ {\isacharasterisk}{\kern0pt}\ real\ p{\isachardoublequoteclose}\ \isacommand{by}\isamarkupfalse%
\ simp\isanewline
\ \ \ \ \ \ \isacommand{hence}\isamarkupfalse%
\ f{\isacharunderscore}{\kern0pt}{\isadigit{2}}{\isacharcolon}{\kern0pt}\ {\isachardoublequoteopen}g{\isacharprime}{\kern0pt}\ {\isacharparenleft}{\kern0pt}h\ {\isasymomega}{\isacharparenright}{\kern0pt}\ {\isachargreater}{\kern0pt}\ {\isacharparenleft}{\kern0pt}{\isadigit{1}}\ {\isacharminus}{\kern0pt}\ real{\isacharunderscore}{\kern0pt}of{\isacharunderscore}{\kern0pt}rat\ {\isasymdelta}{\isacharparenright}{\kern0pt}\ {\isacharasterisk}{\kern0pt}\ m{\isachardoublequoteclose}\isanewline
\ \ \ \ \ \ \ \ \isacommand{apply}\isamarkupfalse%
\ {\isacharparenleft}{\kern0pt}simp\ add{\isacharcolon}{\kern0pt}\ h{\isacharunderscore}{\kern0pt}{\isadigit{3}}{\isacharparenright}{\kern0pt}\isanewline
\ \ \ \ \ \ \ \ \isacommand{by}\isamarkupfalse%
\ {\isacharparenleft}{\kern0pt}subst\ pos{\isacharunderscore}{\kern0pt}less{\isacharunderscore}{\kern0pt}divide{\isacharunderscore}{\kern0pt}eq{\isacharcomma}{\kern0pt}\ metis\ y{\isacharprime}{\kern0pt}{\isacharunderscore}{\kern0pt}pos{\isacharcomma}{\kern0pt}\ simp{\isacharparenright}{\kern0pt}\isanewline
\ \ \ \ \ \ \isacommand{have}\isamarkupfalse%
\ {\isachardoublequoteopen}abs\ {\isacharparenleft}{\kern0pt}g{\isacharprime}{\kern0pt}\ {\isacharparenleft}{\kern0pt}h\ {\isasymomega}{\isacharparenright}{\kern0pt}\ {\isacharminus}{\kern0pt}\ real{\isacharunderscore}{\kern0pt}of{\isacharunderscore}{\kern0pt}rat\ {\isacharparenleft}{\kern0pt}F\ {\isadigit{0}}\ as{\isacharparenright}{\kern0pt}{\isacharparenright}{\kern0pt}\ {\isacharless}{\kern0pt}\ real{\isacharunderscore}{\kern0pt}of{\isacharunderscore}{\kern0pt}rat\ {\isasymdelta}\ {\isacharasterisk}{\kern0pt}\ {\isacharparenleft}{\kern0pt}real{\isacharunderscore}{\kern0pt}of{\isacharunderscore}{\kern0pt}rat\ {\isacharparenleft}{\kern0pt}F\ {\isadigit{0}}\ as{\isacharparenright}{\kern0pt}{\isacharparenright}{\kern0pt}{\isachardoublequoteclose}\isanewline
\ \ \ \ \ \ \ \ \isacommand{apply}\isamarkupfalse%
\ {\isacharparenleft}{\kern0pt}subst\ abs{\isacharunderscore}{\kern0pt}less{\isacharunderscore}{\kern0pt}iff{\isacharparenright}{\kern0pt}\ \isacommand{using}\isamarkupfalse%
\ f{\isacharunderscore}{\kern0pt}{\isadigit{1}}\ f{\isacharunderscore}{\kern0pt}{\isadigit{2}}\isanewline
\ \ \ \ \ \ \ \ \isacommand{by}\isamarkupfalse%
\ {\isacharparenleft}{\kern0pt}simp\ add{\isacharcolon}{\kern0pt}algebra{\isacharunderscore}{\kern0pt}simps\ m{\isacharunderscore}{\kern0pt}eq{\isacharunderscore}{\kern0pt}F{\isacharunderscore}{\kern0pt}{\isadigit{0}}{\isacharparenright}{\kern0pt}\isanewline
\ \ \ \ \ \ \isacommand{thus}\isamarkupfalse%
\ {\isachardoublequoteopen}False{\isachardoublequoteclose}\isanewline
\ \ \ \ \ \ \ \ \isacommand{using}\isamarkupfalse%
\ est\ \isacommand{by}\isamarkupfalse%
\ linarith\isanewline
\ \ \ \ \isacommand{qed}\isamarkupfalse%
\isanewline
\ \ \ \ \isacommand{also}\isamarkupfalse%
\ \isacommand{have}\isamarkupfalse%
\ {\isachardoublequoteopen}{\isachardot}{\kern0pt}{\isachardot}{\kern0pt}{\isachardot}{\kern0pt}\ {\isasymle}\ {\isadigit{1}}{\isacharslash}{\kern0pt}{\isadigit{9}}\ {\isacharplus}{\kern0pt}\ {\isacharparenleft}{\kern0pt}{\isadigit{1}}{\isacharslash}{\kern0pt}{\isadigit{9}}\ {\isacharplus}{\kern0pt}\ {\isadigit{1}}{\isacharslash}{\kern0pt}{\isadigit{9}}{\isacharparenright}{\kern0pt}{\isachardoublequoteclose}\isanewline
\ \ \ \ \ \ \isacommand{apply}\isamarkupfalse%
\ {\isacharparenleft}{\kern0pt}rule\ prob{\isacharunderscore}{\kern0pt}space{\isachardot}{\kern0pt}prob{\isacharunderscore}{\kern0pt}sub{\isacharunderscore}{\kern0pt}additiveI{\isacharcomma}{\kern0pt}\ simp\ add{\isacharcolon}{\kern0pt}prob{\isacharunderscore}{\kern0pt}space{\isacharunderscore}{\kern0pt}measure{\isacharunderscore}{\kern0pt}pmf{\isacharcomma}{\kern0pt}\ simp{\isacharcomma}{\kern0pt}\ simp{\isacharparenright}{\kern0pt}\isanewline
\ \ \ \ \ \ \ \isacommand{apply}\isamarkupfalse%
\ {\isacharparenleft}{\kern0pt}rule\ case{\isacharunderscore}{\kern0pt}{\isadigit{1}}{\isacharparenright}{\kern0pt}\isanewline
\ \ \ \ \ \ \isacommand{apply}\isamarkupfalse%
\ {\isacharparenleft}{\kern0pt}rule\ prob{\isacharunderscore}{\kern0pt}space{\isachardot}{\kern0pt}prob{\isacharunderscore}{\kern0pt}sub{\isacharunderscore}{\kern0pt}additiveI{\isacharcomma}{\kern0pt}\ simp\ add{\isacharcolon}{\kern0pt}prob{\isacharunderscore}{\kern0pt}space{\isacharunderscore}{\kern0pt}measure{\isacharunderscore}{\kern0pt}pmf{\isacharcomma}{\kern0pt}\ simp{\isacharcomma}{\kern0pt}\ simp{\isacharparenright}{\kern0pt}\isanewline
\ \ \ \ \ \ \isacommand{by}\isamarkupfalse%
\ {\isacharparenleft}{\kern0pt}rule\ case{\isacharunderscore}{\kern0pt}{\isadigit{2}}{\isacharcomma}{\kern0pt}\ rule\ case{\isacharunderscore}{\kern0pt}{\isadigit{3}}{\isacharparenright}{\kern0pt}\isanewline
\ \ \ \ \isacommand{also}\isamarkupfalse%
\ \isacommand{have}\isamarkupfalse%
\ {\isachardoublequoteopen}{\isachardot}{\kern0pt}{\isachardot}{\kern0pt}{\isachardot}{\kern0pt}\ {\isacharequal}{\kern0pt}\ {\isadigit{1}}{\isacharslash}{\kern0pt}{\isadigit{3}}{\isachardoublequoteclose}\ \isacommand{by}\isamarkupfalse%
\ simp\isanewline
\ \ \ \ \isacommand{finally}\isamarkupfalse%
\ \isacommand{show}\isamarkupfalse%
\ {\isacharquery}{\kern0pt}thesis\ \isacommand{by}\isamarkupfalse%
\ simp\isanewline
\ \ \isacommand{next}\isamarkupfalse%
\isanewline
\ \ \ \ \isacommand{case}\isamarkupfalse%
\ False\isanewline
\ \ \ \ \isacommand{have}\isamarkupfalse%
\ {\isachardoublequoteopen}{\isasymP}{\isacharparenleft}{\kern0pt}{\isasymomega}\ in\ measure{\isacharunderscore}{\kern0pt}pmf\ {\isasymOmega}\isactrlsub {\isadigit{1}}{\isachardot}{\kern0pt}\ real{\isacharunderscore}{\kern0pt}of{\isacharunderscore}{\kern0pt}rat\ {\isasymdelta}\ {\isacharasterisk}{\kern0pt}\ real{\isacharunderscore}{\kern0pt}of{\isacharunderscore}{\kern0pt}rat\ {\isacharparenleft}{\kern0pt}F\ {\isadigit{0}}\ as{\isacharparenright}{\kern0pt}\ {\isacharless}{\kern0pt}\ {\isasymbar}g{\isacharprime}{\kern0pt}\ {\isacharparenleft}{\kern0pt}h\ {\isasymomega}{\isacharparenright}{\kern0pt}\ {\isacharminus}{\kern0pt}\ real{\isacharunderscore}{\kern0pt}of{\isacharunderscore}{\kern0pt}rat\ {\isacharparenleft}{\kern0pt}F\ {\isadigit{0}}\ as{\isacharparenright}{\kern0pt}{\isasymbar}{\isacharparenright}{\kern0pt}\ {\isasymle}\isanewline
\ \ \ \ \ \ {\isasymP}{\isacharparenleft}{\kern0pt}{\isasymomega}\ in\ measure{\isacharunderscore}{\kern0pt}pmf\ {\isasymOmega}\isactrlsub {\isadigit{1}}{\isachardot}{\kern0pt}\ {\isasymexists}x\ {\isasymin}\ set\ as{\isachardot}{\kern0pt}\ {\isasymexists}y\ {\isasymin}\ set\ as{\isachardot}{\kern0pt}\ x\ {\isasymnoteq}\ y\ {\isasymand}\ \isanewline
\ \ \ \ \ \ truncate{\isacharunderscore}{\kern0pt}down\ r\ {\isacharparenleft}{\kern0pt}real\ {\isacharparenleft}{\kern0pt}hash\ p\ x\ {\isasymomega}{\isacharparenright}{\kern0pt}{\isacharparenright}{\kern0pt}\ {\isasymle}\ real\ p\ {\isasymand}\ \isanewline
\ \ \ \ \ \ truncate{\isacharunderscore}{\kern0pt}down\ r\ {\isacharparenleft}{\kern0pt}real\ {\isacharparenleft}{\kern0pt}hash\ p\ x\ {\isasymomega}{\isacharparenright}{\kern0pt}{\isacharparenright}{\kern0pt}\ {\isacharequal}{\kern0pt}\ truncate{\isacharunderscore}{\kern0pt}down\ r\ {\isacharparenleft}{\kern0pt}real\ {\isacharparenleft}{\kern0pt}hash\ p\ y\ {\isasymomega}{\isacharparenright}{\kern0pt}{\isacharparenright}{\kern0pt}{\isacharparenright}{\kern0pt}{\isachardoublequoteclose}\ \isanewline
\ \ \ \ \isacommand{proof}\isamarkupfalse%
\ {\isacharparenleft}{\kern0pt}rule\ pmf{\isacharunderscore}{\kern0pt}mono{\isacharunderscore}{\kern0pt}{\isadigit{1}}{\isacharparenright}{\kern0pt}\isanewline
\ \ \ \ \ \ \isacommand{fix}\isamarkupfalse%
\ {\isasymomega}\isanewline
\ \ \ \ \ \ \isacommand{assume}\isamarkupfalse%
\ a{\isacharcolon}{\kern0pt}{\isachardoublequoteopen}{\isasymomega}\ {\isasymin}\ {\isacharbraceleft}{\kern0pt}{\isasymomega}\ {\isasymin}\ space\ {\isacharparenleft}{\kern0pt}measure{\isacharunderscore}{\kern0pt}pmf\ {\isasymOmega}\isactrlsub {\isadigit{1}}{\isacharparenright}{\kern0pt}{\isachardot}{\kern0pt}\isanewline
\ \ \ \ \ \ \ \ \ \ \ \ \ \ real{\isacharunderscore}{\kern0pt}of{\isacharunderscore}{\kern0pt}rat\ {\isasymdelta}\ {\isacharasterisk}{\kern0pt}\ real{\isacharunderscore}{\kern0pt}of{\isacharunderscore}{\kern0pt}rat\ {\isacharparenleft}{\kern0pt}F\ {\isadigit{0}}\ as{\isacharparenright}{\kern0pt}\ {\isacharless}{\kern0pt}\ {\isasymbar}g{\isacharprime}{\kern0pt}\ {\isacharparenleft}{\kern0pt}h\ {\isasymomega}{\isacharparenright}{\kern0pt}\ {\isacharminus}{\kern0pt}\ real{\isacharunderscore}{\kern0pt}of{\isacharunderscore}{\kern0pt}rat\ {\isacharparenleft}{\kern0pt}F\ {\isadigit{0}}\ as{\isacharparenright}{\kern0pt}{\isasymbar}{\isacharbraceright}{\kern0pt}{\isachardoublequoteclose}\isanewline
\ \ \ \ \ \ \isacommand{assume}\isamarkupfalse%
\ b{\isacharcolon}{\kern0pt}{\isachardoublequoteopen}{\isasymomega}\ {\isasymin}\ set{\isacharunderscore}{\kern0pt}pmf\ {\isasymOmega}\isactrlsub {\isadigit{1}}{\isachardoublequoteclose}\ \isanewline
\ \ \ \ \ \ \isacommand{have}\isamarkupfalse%
\ a{\isacharunderscore}{\kern0pt}{\isadigit{1}}{\isacharcolon}{\kern0pt}\ {\isachardoublequoteopen}card\ {\isacharparenleft}{\kern0pt}set\ as{\isacharparenright}{\kern0pt}\ {\isacharless}{\kern0pt}\ t{\isachardoublequoteclose}\ \isacommand{using}\isamarkupfalse%
\ False\ \isacommand{by}\isamarkupfalse%
\ auto\isanewline
\ \ \ \ \ \ \isacommand{have}\isamarkupfalse%
\ a{\isacharunderscore}{\kern0pt}{\isadigit{2}}{\isacharcolon}{\kern0pt}{\isachardoublequoteopen}card\ {\isacharparenleft}{\kern0pt}h\ {\isasymomega}{\isacharparenright}{\kern0pt}\ {\isacharequal}{\kern0pt}\ card\ {\isacharparenleft}{\kern0pt}{\isacharparenleft}{\kern0pt}{\isasymlambda}x{\isachardot}{\kern0pt}\ truncate{\isacharunderscore}{\kern0pt}down\ r\ {\isacharparenleft}{\kern0pt}real\ {\isacharparenleft}{\kern0pt}hash\ p\ x\ {\isasymomega}{\isacharparenright}{\kern0pt}{\isacharparenright}{\kern0pt}{\isacharparenright}{\kern0pt}\ {\isacharbackquote}{\kern0pt}\ {\isacharparenleft}{\kern0pt}set\ as{\isacharparenright}{\kern0pt}{\isacharparenright}{\kern0pt}{\isachardoublequoteclose}\isanewline
\ \ \ \ \ \ \ \ \isacommand{apply}\isamarkupfalse%
\ {\isacharparenleft}{\kern0pt}simp\ add{\isacharcolon}{\kern0pt}h{\isacharunderscore}{\kern0pt}def{\isacharparenright}{\kern0pt}\isanewline
\ \ \ \ \ \ \ \ \isacommand{apply}\isamarkupfalse%
\ {\isacharparenleft}{\kern0pt}subst\ card{\isacharunderscore}{\kern0pt}least{\isacharcomma}{\kern0pt}\ simp{\isacharparenright}{\kern0pt}\isanewline
\ \ \ \ \ \ \ \ \isacommand{apply}\isamarkupfalse%
\ {\isacharparenleft}{\kern0pt}rule\ min{\isachardot}{\kern0pt}absorb{\isadigit{4}}{\isacharparenright}{\kern0pt}\isanewline
\ \ \ \ \ \ \ \ \isacommand{using}\isamarkupfalse%
\ card{\isacharunderscore}{\kern0pt}image{\isacharunderscore}{\kern0pt}le\ a{\isacharunderscore}{\kern0pt}{\isadigit{1}}\ order{\isacharunderscore}{\kern0pt}le{\isacharunderscore}{\kern0pt}less{\isacharunderscore}{\kern0pt}trans{\isacharbrackleft}{\kern0pt}OF\ {\isacharunderscore}{\kern0pt}\ a{\isacharunderscore}{\kern0pt}{\isadigit{1}}{\isacharbrackright}{\kern0pt}\ \isacommand{by}\isamarkupfalse%
\ blast\isanewline
\ \ \ \ \ \ \isacommand{have}\isamarkupfalse%
\ {\isachardoublequoteopen}card\ {\isacharparenleft}{\kern0pt}h\ {\isasymomega}{\isacharparenright}{\kern0pt}\ {\isacharless}{\kern0pt}\ t{\isachardoublequoteclose}\isanewline
\ \ \ \ \ \ \ \ \isacommand{by}\isamarkupfalse%
\ {\isacharparenleft}{\kern0pt}metis\ List{\isachardot}{\kern0pt}finite{\isacharunderscore}{\kern0pt}set\ \ a{\isacharunderscore}{\kern0pt}{\isadigit{1}}\ a{\isacharunderscore}{\kern0pt}{\isadigit{2}}\ card{\isacharunderscore}{\kern0pt}image{\isacharunderscore}{\kern0pt}le\ \ order{\isacharunderscore}{\kern0pt}le{\isacharunderscore}{\kern0pt}less{\isacharunderscore}{\kern0pt}trans{\isacharparenright}{\kern0pt}\isanewline
\ \ \ \ \ \ \isacommand{hence}\isamarkupfalse%
\ {\isachardoublequoteopen}g{\isacharprime}{\kern0pt}\ {\isacharparenleft}{\kern0pt}h\ {\isasymomega}{\isacharparenright}{\kern0pt}\ {\isacharequal}{\kern0pt}\ card\ {\isacharparenleft}{\kern0pt}h\ {\isasymomega}{\isacharparenright}{\kern0pt}{\isachardoublequoteclose}\ \isacommand{by}\isamarkupfalse%
\ {\isacharparenleft}{\kern0pt}simp\ add{\isacharcolon}{\kern0pt}g{\isacharprime}{\kern0pt}{\isacharunderscore}{\kern0pt}def{\isacharparenright}{\kern0pt}\isanewline
\ \ \ \ \ \ \isacommand{hence}\isamarkupfalse%
\ {\isachardoublequoteopen}card\ {\isacharparenleft}{\kern0pt}h\ {\isasymomega}{\isacharparenright}{\kern0pt}\ {\isasymnoteq}\ real{\isacharunderscore}{\kern0pt}of{\isacharunderscore}{\kern0pt}rat\ {\isacharparenleft}{\kern0pt}F\ {\isadigit{0}}\ as{\isacharparenright}{\kern0pt}{\isachardoublequoteclose}\isanewline
\ \ \ \ \ \ \ \ \isacommand{using}\isamarkupfalse%
\ a\ assms{\isacharparenleft}{\kern0pt}{\isadigit{2}}{\isacharparenright}{\kern0pt}\ \isacommand{apply}\isamarkupfalse%
\ simp\ \isanewline
\ \ \ \ \ \ \ \ \isacommand{by}\isamarkupfalse%
\ {\isacharparenleft}{\kern0pt}metis\ abs{\isacharunderscore}{\kern0pt}zero\ cancel{\isacharunderscore}{\kern0pt}comm{\isacharunderscore}{\kern0pt}monoid{\isacharunderscore}{\kern0pt}add{\isacharunderscore}{\kern0pt}class{\isachardot}{\kern0pt}diff{\isacharunderscore}{\kern0pt}cancel\ of{\isacharunderscore}{\kern0pt}nat{\isacharunderscore}{\kern0pt}less{\isacharunderscore}{\kern0pt}{\isadigit{0}}{\isacharunderscore}{\kern0pt}iff\ pos{\isacharunderscore}{\kern0pt}prod{\isacharunderscore}{\kern0pt}lt\ zero{\isacharunderscore}{\kern0pt}less{\isacharunderscore}{\kern0pt}of{\isacharunderscore}{\kern0pt}rat{\isacharunderscore}{\kern0pt}iff{\isacharparenright}{\kern0pt}\isanewline
\ \ \ \ \ \ \isacommand{hence}\isamarkupfalse%
\ {\isachardoublequoteopen}card\ {\isacharparenleft}{\kern0pt}h\ {\isasymomega}{\isacharparenright}{\kern0pt}\ {\isasymnoteq}\ card\ {\isacharparenleft}{\kern0pt}set\ as{\isacharparenright}{\kern0pt}{\isachardoublequoteclose}\isanewline
\ \ \ \ \ \ \ \ \isacommand{using}\isamarkupfalse%
\ m{\isacharunderscore}{\kern0pt}def\ m{\isacharunderscore}{\kern0pt}eq{\isacharunderscore}{\kern0pt}F{\isacharunderscore}{\kern0pt}{\isadigit{0}}\ \isacommand{by}\isamarkupfalse%
\ linarith\isanewline
\ \ \ \ \ \ \isacommand{hence}\isamarkupfalse%
\ {\isachardoublequoteopen}{\isasymnot}inj{\isacharunderscore}{\kern0pt}on\ {\isacharparenleft}{\kern0pt}{\isasymlambda}x{\isachardot}{\kern0pt}\ truncate{\isacharunderscore}{\kern0pt}down\ r\ {\isacharparenleft}{\kern0pt}real\ {\isacharparenleft}{\kern0pt}hash\ p\ x\ {\isasymomega}{\isacharparenright}{\kern0pt}{\isacharparenright}{\kern0pt}{\isacharparenright}{\kern0pt}\ {\isacharparenleft}{\kern0pt}set\ as{\isacharparenright}{\kern0pt}{\isachardoublequoteclose}\isanewline
\ \ \ \ \ \ \ \ \isacommand{apply}\isamarkupfalse%
\ {\isacharparenleft}{\kern0pt}simp\ add{\isacharcolon}{\kern0pt}a{\isacharunderscore}{\kern0pt}{\isadigit{2}}{\isacharparenright}{\kern0pt}\ \isanewline
\ \ \ \ \ \ \ \ \isacommand{using}\isamarkupfalse%
\ card{\isacharunderscore}{\kern0pt}image\ \isacommand{by}\isamarkupfalse%
\ blast\isanewline
\ \ \ \ \ \ \isacommand{moreover}\isamarkupfalse%
\ \isacommand{have}\isamarkupfalse%
\ {\isachardoublequoteopen}{\isasymAnd}x{\isachardot}{\kern0pt}\ x\ {\isasymin}\ set\ as\ {\isasymLongrightarrow}\ truncate{\isacharunderscore}{\kern0pt}down\ r\ {\isacharparenleft}{\kern0pt}real\ {\isacharparenleft}{\kern0pt}hash\ p\ x\ {\isasymomega}{\isacharparenright}{\kern0pt}{\isacharparenright}{\kern0pt}\ {\isasymle}\ real\ p{\isachardoublequoteclose}\isanewline
\ \ \ \ \ \ \isacommand{proof}\isamarkupfalse%
\ {\isacharminus}{\kern0pt}\isanewline
\ \ \ \ \ \ \ \ \isacommand{fix}\isamarkupfalse%
\ x\isanewline
\ \ \ \ \ \ \ \ \isacommand{assume}\isamarkupfalse%
\ a{\isacharcolon}{\kern0pt}{\isachardoublequoteopen}x\ {\isasymin}\ set\ as{\isachardoublequoteclose}\isanewline
\ \ \ \ \ \ \ \ \isacommand{show}\isamarkupfalse%
\ {\isachardoublequoteopen}truncate{\isacharunderscore}{\kern0pt}down\ r\ {\isacharparenleft}{\kern0pt}real\ {\isacharparenleft}{\kern0pt}hash\ p\ x\ {\isasymomega}{\isacharparenright}{\kern0pt}{\isacharparenright}{\kern0pt}\ {\isasymle}\ real\ p{\isachardoublequoteclose}\isanewline
\ \ \ \ \ \ \ \ \ \ \isacommand{apply}\isamarkupfalse%
\ {\isacharparenleft}{\kern0pt}rule\ truncate{\isacharunderscore}{\kern0pt}down{\isacharunderscore}{\kern0pt}le{\isacharparenright}{\kern0pt}\isanewline
\ \ \ \ \ \ \ \ \ \ \isacommand{using}\isamarkupfalse%
\ hash{\isacharunderscore}{\kern0pt}range{\isacharbrackleft}{\kern0pt}OF\ p{\isacharunderscore}{\kern0pt}ge{\isacharunderscore}{\kern0pt}{\isadigit{0}}\ {\isacharunderscore}{\kern0pt}\ xs{\isacharunderscore}{\kern0pt}le{\isacharunderscore}{\kern0pt}p{\isacharbrackleft}{\kern0pt}OF\ a{\isacharbrackright}{\kern0pt}{\isacharbrackright}{\kern0pt}\ \ b\isanewline
\ \ \ \ \ \ \ \ \ \ \isacommand{apply}\isamarkupfalse%
\ {\isacharparenleft}{\kern0pt}simp\ add{\isacharcolon}{\kern0pt}{\isasymOmega}\isactrlsub {\isadigit{1}}{\isacharunderscore}{\kern0pt}def\ set{\isacharunderscore}{\kern0pt}pmf{\isacharunderscore}{\kern0pt}of{\isacharunderscore}{\kern0pt}set{\isacharbrackleft}{\kern0pt}OF\ ne{\isacharunderscore}{\kern0pt}bounded{\isacharunderscore}{\kern0pt}degree{\isacharunderscore}{\kern0pt}polynomials\ fin{\isacharunderscore}{\kern0pt}bounded{\isacharunderscore}{\kern0pt}degree{\isacharunderscore}{\kern0pt}polynomials{\isacharbrackleft}{\kern0pt}OF\ p{\isacharunderscore}{\kern0pt}ge{\isacharunderscore}{\kern0pt}{\isadigit{0}}{\isacharbrackright}{\kern0pt}{\isacharbrackright}{\kern0pt}{\isacharparenright}{\kern0pt}\isanewline
\ \ \ \ \ \ \ \ \ \ \isacommand{using}\isamarkupfalse%
\ le{\isacharunderscore}{\kern0pt}eq{\isacharunderscore}{\kern0pt}less{\isacharunderscore}{\kern0pt}or{\isacharunderscore}{\kern0pt}eq\ \isacommand{by}\isamarkupfalse%
\ blast\isanewline
\ \ \ \ \ \ \isacommand{qed}\isamarkupfalse%
\isanewline
\ \ \ \ \ \isacommand{ultimately}\isamarkupfalse%
\ \isacommand{show}\isamarkupfalse%
\ {\isachardoublequoteopen}{\isasymomega}\ {\isasymin}\ {\isacharbraceleft}{\kern0pt}{\isasymomega}\ {\isasymin}\ space\ {\isacharparenleft}{\kern0pt}measure{\isacharunderscore}{\kern0pt}pmf\ {\isasymOmega}\isactrlsub {\isadigit{1}}{\isacharparenright}{\kern0pt}{\isachardot}{\kern0pt}\ {\isasymexists}x\ {\isasymin}\ set\ as{\isachardot}{\kern0pt}\ {\isasymexists}y\ {\isasymin}\ set\ as{\isachardot}{\kern0pt}\ x\ {\isasymnoteq}\ y\ {\isasymand}\isanewline
\ \ \ \ \ \ \ \ truncate{\isacharunderscore}{\kern0pt}down\ r\ {\isacharparenleft}{\kern0pt}real\ {\isacharparenleft}{\kern0pt}hash\ p\ x\ {\isasymomega}{\isacharparenright}{\kern0pt}{\isacharparenright}{\kern0pt}\ {\isasymle}\ real\ p\ {\isasymand}\ \isanewline
\ \ \ \ \ \ \ \ truncate{\isacharunderscore}{\kern0pt}down\ r\ {\isacharparenleft}{\kern0pt}real\ {\isacharparenleft}{\kern0pt}hash\ p\ x\ {\isasymomega}{\isacharparenright}{\kern0pt}{\isacharparenright}{\kern0pt}\ {\isacharequal}{\kern0pt}\ truncate{\isacharunderscore}{\kern0pt}down\ r\ {\isacharparenleft}{\kern0pt}real\ {\isacharparenleft}{\kern0pt}hash\ p\ y\ {\isasymomega}{\isacharparenright}{\kern0pt}{\isacharparenright}{\kern0pt}{\isacharbraceright}{\kern0pt}{\isachardoublequoteclose}\isanewline
\ \ \ \ \ \ \ \isacommand{apply}\isamarkupfalse%
\ {\isacharparenleft}{\kern0pt}simp\ add{\isacharcolon}{\kern0pt}inj{\isacharunderscore}{\kern0pt}on{\isacharunderscore}{\kern0pt}def{\isacharparenright}{\kern0pt}\ \isacommand{by}\isamarkupfalse%
\ blast\isanewline
\ \ \ \ \isacommand{qed}\isamarkupfalse%
\isanewline
\ \ \ \ \isacommand{also}\isamarkupfalse%
\ \isacommand{have}\isamarkupfalse%
\ {\isachardoublequoteopen}{\isachardot}{\kern0pt}{\isachardot}{\kern0pt}{\isachardot}{\kern0pt}\ {\isasymle}\ {\isadigit{6}}\ {\isacharasterisk}{\kern0pt}\ {\isacharparenleft}{\kern0pt}real\ {\isacharparenleft}{\kern0pt}card\ {\isacharparenleft}{\kern0pt}set\ as{\isacharparenright}{\kern0pt}{\isacharparenright}{\kern0pt}{\isacharparenright}{\kern0pt}\isactrlsup {\isadigit{2}}\ {\isacharasterisk}{\kern0pt}\ {\isacharparenleft}{\kern0pt}real\ p{\isacharparenright}{\kern0pt}\isactrlsup {\isadigit{2}}\ {\isacharasterisk}{\kern0pt}\ {\isadigit{2}}\ powr\ {\isacharminus}{\kern0pt}\ real\ r\ {\isacharslash}{\kern0pt}\ {\isacharparenleft}{\kern0pt}real\ p{\isacharparenright}{\kern0pt}\isactrlsup {\isadigit{2}}\ {\isacharplus}{\kern0pt}\ {\isadigit{1}}\ {\isacharslash}{\kern0pt}\ real\ p{\isachardoublequoteclose}\isanewline
\ \ \ \ \ \ \isacommand{apply}\isamarkupfalse%
\ {\isacharparenleft}{\kern0pt}simp\ only{\isacharcolon}{\kern0pt}{\isasymOmega}\isactrlsub {\isadigit{1}}{\isacharunderscore}{\kern0pt}def{\isacharparenright}{\kern0pt}\isanewline
\ \ \ \ \ \ \isacommand{apply}\isamarkupfalse%
\ {\isacharparenleft}{\kern0pt}rule\ f{\isadigit{0}}{\isacharunderscore}{\kern0pt}collision{\isacharunderscore}{\kern0pt}prob{\isacharparenright}{\kern0pt}\isanewline
\ \ \ \ \ \ \ \ \isacommand{apply}\isamarkupfalse%
\ {\isacharparenleft}{\kern0pt}metis\ p{\isacharunderscore}{\kern0pt}prime{\isacharparenright}{\kern0pt}\isanewline
\ \ \ \ \ \ \ \isacommand{apply}\isamarkupfalse%
\ {\isacharparenleft}{\kern0pt}rule\ subsetI{\isacharcomma}{\kern0pt}\ simp\ add{\isacharcolon}{\kern0pt}xs{\isacharunderscore}{\kern0pt}le{\isacharunderscore}{\kern0pt}p{\isacharparenright}{\kern0pt}\isanewline
\ \ \ \ \ \ \isacommand{using}\isamarkupfalse%
\ p{\isacharunderscore}{\kern0pt}ge{\isacharunderscore}{\kern0pt}{\isadigit{0}}\ r{\isacharunderscore}{\kern0pt}ge{\isacharunderscore}{\kern0pt}{\isadigit{0}}\ \isacommand{by}\isamarkupfalse%
\ simp{\isacharplus}{\kern0pt}\isanewline
\ \ \ \ \isacommand{also}\isamarkupfalse%
\ \isacommand{have}\isamarkupfalse%
\ {\isachardoublequoteopen}{\isachardot}{\kern0pt}{\isachardot}{\kern0pt}{\isachardot}{\kern0pt}\ {\isacharequal}{\kern0pt}\ {\isadigit{6}}\ {\isacharasterisk}{\kern0pt}\ {\isacharparenleft}{\kern0pt}real\ {\isacharparenleft}{\kern0pt}card\ {\isacharparenleft}{\kern0pt}set\ as{\isacharparenright}{\kern0pt}{\isacharparenright}{\kern0pt}{\isacharparenright}{\kern0pt}\isactrlsup {\isadigit{2}}\ {\isacharasterisk}{\kern0pt}\ {\isadigit{2}}\ powr\ {\isacharparenleft}{\kern0pt}{\isacharminus}{\kern0pt}\ real\ r{\isacharparenright}{\kern0pt}\ {\isacharplus}{\kern0pt}\ {\isadigit{1}}\ {\isacharslash}{\kern0pt}\ real\ p{\isachardoublequoteclose}\isanewline
\ \ \ \ \ \ \isacommand{apply}\isamarkupfalse%
\ {\isacharparenleft}{\kern0pt}simp\ add{\isacharcolon}{\kern0pt}ac{\isacharunderscore}{\kern0pt}simps\ power{\isadigit{2}}{\isacharunderscore}{\kern0pt}eq{\isacharunderscore}{\kern0pt}square{\isacharparenright}{\kern0pt}\isanewline
\ \ \ \ \ \ \isacommand{using}\isamarkupfalse%
\ p{\isacharunderscore}{\kern0pt}ge{\isacharunderscore}{\kern0pt}{\isadigit{0}}\ \isacommand{by}\isamarkupfalse%
\ blast\isanewline
\ \ \ \ \isacommand{also}\isamarkupfalse%
\ \isacommand{have}\isamarkupfalse%
\ {\isachardoublequoteopen}{\isachardot}{\kern0pt}{\isachardot}{\kern0pt}{\isachardot}{\kern0pt}\ {\isasymle}\ {\isadigit{6}}\ {\isacharasterisk}{\kern0pt}\ {\isacharparenleft}{\kern0pt}real\ t{\isacharparenright}{\kern0pt}\isactrlsup {\isadigit{2}}\ {\isacharasterisk}{\kern0pt}\ {\isadigit{2}}\ powr\ {\isacharparenleft}{\kern0pt}{\isacharminus}{\kern0pt}real\ r{\isacharparenright}{\kern0pt}\ {\isacharplus}{\kern0pt}\ {\isadigit{1}}\ {\isacharslash}{\kern0pt}\ real\ p{\isachardoublequoteclose}\isanewline
\ \ \ \ \ \ \isacommand{apply}\isamarkupfalse%
\ {\isacharparenleft}{\kern0pt}rule\ add{\isacharunderscore}{\kern0pt}mono{\isacharparenright}{\kern0pt}\isanewline
\ \ \ \ \ \ \ \isacommand{apply}\isamarkupfalse%
\ {\isacharparenleft}{\kern0pt}rule\ mult{\isacharunderscore}{\kern0pt}right{\isacharunderscore}{\kern0pt}mono{\isacharparenright}{\kern0pt}\isanewline
\ \ \ \ \ \ \ \ \isacommand{apply}\isamarkupfalse%
\ {\isacharparenleft}{\kern0pt}rule\ mult{\isacharunderscore}{\kern0pt}left{\isacharunderscore}{\kern0pt}mono{\isacharparenright}{\kern0pt}\isanewline
\ \ \ \ \ \ \ \ \ \isacommand{apply}\isamarkupfalse%
\ {\isacharparenleft}{\kern0pt}rule\ power{\isacharunderscore}{\kern0pt}mono{\isacharparenright}{\kern0pt}\ \isacommand{using}\isamarkupfalse%
\ False\ \isacommand{apply}\isamarkupfalse%
\ simp\isanewline
\ \ \ \ \ \ \isacommand{by}\isamarkupfalse%
\ simp{\isacharplus}{\kern0pt}\isanewline
\ \ \ \ \isacommand{also}\isamarkupfalse%
\ \isacommand{have}\isamarkupfalse%
\ {\isachardoublequoteopen}{\isachardot}{\kern0pt}{\isachardot}{\kern0pt}{\isachardot}{\kern0pt}\ {\isasymle}\ {\isadigit{1}}{\isacharslash}{\kern0pt}{\isadigit{6}}\ {\isacharplus}{\kern0pt}\ {\isadigit{1}}{\isacharslash}{\kern0pt}{\isadigit{6}}{\isachardoublequoteclose}\isanewline
\ \ \ \ \ \ \isacommand{apply}\isamarkupfalse%
\ {\isacharparenleft}{\kern0pt}rule\ add{\isacharunderscore}{\kern0pt}mono{\isacharparenright}{\kern0pt}\isanewline
\ \ \ \ \ \ \isacommand{apply}\isamarkupfalse%
\ {\isacharparenleft}{\kern0pt}subst\ pos{\isacharunderscore}{\kern0pt}le{\isacharunderscore}{\kern0pt}divide{\isacharunderscore}{\kern0pt}eq{\isacharcomma}{\kern0pt}\ simp{\isacharparenright}{\kern0pt}\isanewline
\ \ \ \ \ \ \isacommand{using}\isamarkupfalse%
\ r{\isacharunderscore}{\kern0pt}le{\isacharunderscore}{\kern0pt}t{\isadigit{2}}\ \isacommand{apply}\isamarkupfalse%
\ simp\isanewline
\ \ \ \ \ \ \isacommand{using}\isamarkupfalse%
\ p{\isacharunderscore}{\kern0pt}ge{\isacharunderscore}{\kern0pt}{\isadigit{1}}{\isadigit{8}}\ \isacommand{by}\isamarkupfalse%
\ simp\isanewline
\ \ \ \ \isacommand{also}\isamarkupfalse%
\ \isacommand{have}\isamarkupfalse%
\ {\isachardoublequoteopen}{\isachardot}{\kern0pt}{\isachardot}{\kern0pt}{\isachardot}{\kern0pt}\ {\isasymle}\ {\isadigit{1}}{\isacharslash}{\kern0pt}{\isadigit{3}}{\isachardoublequoteclose}\ \isacommand{by}\isamarkupfalse%
\ simp\isanewline
\ \ \ \ \isacommand{finally}\isamarkupfalse%
\ \isacommand{show}\isamarkupfalse%
\ {\isacharquery}{\kern0pt}thesis\ \isacommand{by}\isamarkupfalse%
\ simp\isanewline
\ \ \isacommand{qed}\isamarkupfalse%
\isanewline
\isanewline
\ \ \isacommand{have}\isamarkupfalse%
\ f{\isadigit{0}}{\isacharunderscore}{\kern0pt}result{\isacharunderscore}{\kern0pt}elim{\isacharcolon}{\kern0pt}\ {\isachardoublequoteopen}{\isasymAnd}x{\isachardot}{\kern0pt}\ f{\isadigit{0}}{\isacharunderscore}{\kern0pt}result\ {\isacharparenleft}{\kern0pt}s{\isacharcomma}{\kern0pt}\ t{\isacharcomma}{\kern0pt}\ p{\isacharcomma}{\kern0pt}\ r{\isacharcomma}{\kern0pt}\ x{\isacharcomma}{\kern0pt}\ {\isasymlambda}i{\isasymin}{\isacharbraceleft}{\kern0pt}{\isadigit{0}}{\isachardot}{\kern0pt}{\isachardot}{\kern0pt}{\isacharless}{\kern0pt}s{\isacharbraceright}{\kern0pt}{\isachardot}{\kern0pt}\ f{\isadigit{0}}{\isacharunderscore}{\kern0pt}sketch\ p\ r\ t\ {\isacharparenleft}{\kern0pt}x\ i{\isacharparenright}{\kern0pt}\ as{\isacharparenright}{\kern0pt}\ {\isacharequal}{\kern0pt}\isanewline
\ \ \ \ return{\isacharunderscore}{\kern0pt}pmf\ {\isacharparenleft}{\kern0pt}median\ {\isacharparenleft}{\kern0pt}{\isasymlambda}i{\isachardot}{\kern0pt}\ g\ {\isacharparenleft}{\kern0pt}f{\isadigit{0}}{\isacharunderscore}{\kern0pt}sketch\ p\ r\ t\ {\isacharparenleft}{\kern0pt}x\ i{\isacharparenright}{\kern0pt}\ as{\isacharparenright}{\kern0pt}{\isacharparenright}{\kern0pt}\ s{\isacharparenright}{\kern0pt}{\isachardoublequoteclose}\isanewline
\ \ \ \ \isacommand{apply}\isamarkupfalse%
\ {\isacharparenleft}{\kern0pt}simp\ add{\isacharcolon}{\kern0pt}g{\isacharunderscore}{\kern0pt}def{\isacharparenright}{\kern0pt}\isanewline
\ \ \ \ \isacommand{apply}\isamarkupfalse%
\ {\isacharparenleft}{\kern0pt}rule\ median{\isacharunderscore}{\kern0pt}cong{\isacharparenright}{\kern0pt}\isanewline
\ \ \ \ \isacommand{by}\isamarkupfalse%
\ simp\isanewline
\isanewline
\ \ \isacommand{have}\isamarkupfalse%
\ real{\isacharunderscore}{\kern0pt}g{\isacharunderscore}{\kern0pt}{\isadigit{2}}{\isacharcolon}{\kern0pt}{\isachardoublequoteopen}{\isasymAnd}{\isasymomega}{\isachardot}{\kern0pt}\ real{\isacharunderscore}{\kern0pt}of{\isacharunderscore}{\kern0pt}float\ {\isacharbackquote}{\kern0pt}\ f{\isadigit{0}}{\isacharunderscore}{\kern0pt}sketch\ p\ r\ t\ {\isasymomega}\ as\ {\isacharequal}{\kern0pt}\ h\ {\isasymomega}{\isachardoublequoteclose}\ \isanewline
\ \ \ \ \isacommand{apply}\isamarkupfalse%
\ {\isacharparenleft}{\kern0pt}simp\ add{\isacharcolon}{\kern0pt}g{\isacharunderscore}{\kern0pt}def\ g{\isacharprime}{\kern0pt}{\isacharunderscore}{\kern0pt}def\ h{\isacharunderscore}{\kern0pt}def\ f{\isadigit{0}}{\isacharunderscore}{\kern0pt}sketch{\isacharunderscore}{\kern0pt}def{\isacharparenright}{\kern0pt}\isanewline
\ \ \ \ \isacommand{apply}\isamarkupfalse%
\ {\isacharparenleft}{\kern0pt}subst\ least{\isacharunderscore}{\kern0pt}mono{\isacharunderscore}{\kern0pt}commute{\isacharcomma}{\kern0pt}\ simp{\isacharparenright}{\kern0pt}\isanewline
\ \ \ \ \ \isacommand{apply}\isamarkupfalse%
\ {\isacharparenleft}{\kern0pt}meson\ less{\isacharunderscore}{\kern0pt}float{\isachardot}{\kern0pt}rep{\isacharunderscore}{\kern0pt}eq\ strict{\isacharunderscore}{\kern0pt}mono{\isacharunderscore}{\kern0pt}onI{\isacharparenright}{\kern0pt}\isanewline
\ \ \ \ \isacommand{by}\isamarkupfalse%
\ {\isacharparenleft}{\kern0pt}simp\ add{\isacharcolon}{\kern0pt}image{\isacharunderscore}{\kern0pt}comp\ float{\isacharunderscore}{\kern0pt}of{\isacharunderscore}{\kern0pt}inverse{\isacharbrackleft}{\kern0pt}OF\ truncate{\isacharunderscore}{\kern0pt}down{\isacharunderscore}{\kern0pt}float{\isacharbrackright}{\kern0pt}{\isacharparenright}{\kern0pt}\isanewline
\isanewline
\ \ \isacommand{have}\isamarkupfalse%
\ card{\isacharunderscore}{\kern0pt}eq{\isacharcolon}{\kern0pt}\ {\isachardoublequoteopen}{\isasymAnd}{\isasymomega}{\isachardot}{\kern0pt}\ card\ {\isacharparenleft}{\kern0pt}f{\isadigit{0}}{\isacharunderscore}{\kern0pt}sketch\ p\ r\ t\ {\isasymomega}\ as{\isacharparenright}{\kern0pt}\ {\isacharequal}{\kern0pt}\ card\ {\isacharparenleft}{\kern0pt}h\ {\isasymomega}{\isacharparenright}{\kern0pt}{\isachardoublequoteclose}\ \isanewline
\ \ \ \ \isacommand{apply}\isamarkupfalse%
\ {\isacharparenleft}{\kern0pt}subst\ real{\isacharunderscore}{\kern0pt}g{\isacharunderscore}{\kern0pt}{\isadigit{2}}{\isacharbrackleft}{\kern0pt}symmetric{\isacharbrackright}{\kern0pt}{\isacharparenright}{\kern0pt}\ \isanewline
\ \ \ \ \isacommand{apply}\isamarkupfalse%
\ {\isacharparenleft}{\kern0pt}rule\ card{\isacharunderscore}{\kern0pt}image{\isacharbrackleft}{\kern0pt}symmetric{\isacharbrackright}{\kern0pt}{\isacharparenright}{\kern0pt}\ \isanewline
\ \ \ \ \isacommand{using}\isamarkupfalse%
\ inj{\isacharunderscore}{\kern0pt}on{\isacharunderscore}{\kern0pt}def\ real{\isacharunderscore}{\kern0pt}of{\isacharunderscore}{\kern0pt}float{\isacharunderscore}{\kern0pt}inject\ \isacommand{by}\isamarkupfalse%
\ blast\isanewline
\isanewline
\ \ \isacommand{have}\isamarkupfalse%
\ real{\isacharunderscore}{\kern0pt}g{\isacharcolon}{\kern0pt}\ {\isachardoublequoteopen}{\isasymAnd}{\isasymomega}{\isachardot}{\kern0pt}\ real{\isacharunderscore}{\kern0pt}of{\isacharunderscore}{\kern0pt}rat\ {\isacharparenleft}{\kern0pt}g\ {\isacharparenleft}{\kern0pt}f{\isadigit{0}}{\isacharunderscore}{\kern0pt}sketch\ p\ r\ t\ {\isasymomega}\ as{\isacharparenright}{\kern0pt}{\isacharparenright}{\kern0pt}\ {\isacharequal}{\kern0pt}\ g{\isacharprime}{\kern0pt}\ {\isacharparenleft}{\kern0pt}h\ {\isasymomega}{\isacharparenright}{\kern0pt}{\isachardoublequoteclose}\isanewline
\ \ \ \ \isacommand{apply}\isamarkupfalse%
\ {\isacharparenleft}{\kern0pt}simp\ add{\isacharcolon}{\kern0pt}g{\isacharunderscore}{\kern0pt}def\ g{\isacharprime}{\kern0pt}{\isacharunderscore}{\kern0pt}def\ card{\isacharunderscore}{\kern0pt}eq\ of{\isacharunderscore}{\kern0pt}rat{\isacharunderscore}{\kern0pt}divide\ of{\isacharunderscore}{\kern0pt}rat{\isacharunderscore}{\kern0pt}mult\ of{\isacharunderscore}{\kern0pt}rat{\isacharunderscore}{\kern0pt}add\ real{\isacharunderscore}{\kern0pt}of{\isacharunderscore}{\kern0pt}rat{\isacharunderscore}{\kern0pt}of{\isacharunderscore}{\kern0pt}float{\isacharparenright}{\kern0pt}\isanewline
\ \ \ \ \isacommand{apply}\isamarkupfalse%
\ {\isacharparenleft}{\kern0pt}rule\ impI{\isacharparenright}{\kern0pt}\isanewline
\ \ \ \ \isacommand{apply}\isamarkupfalse%
\ {\isacharparenleft}{\kern0pt}subst\ mono{\isacharunderscore}{\kern0pt}Max{\isacharunderscore}{\kern0pt}commute{\isacharbrackleft}{\kern0pt}\isakeyword{where}\ f{\isacharequal}{\kern0pt}{\isachardoublequoteopen}real{\isacharunderscore}{\kern0pt}of{\isacharunderscore}{\kern0pt}float{\isachardoublequoteclose}{\isacharbrackright}{\kern0pt}{\isacharparenright}{\kern0pt}\isanewline
\ \ \ \ \isacommand{using}\isamarkupfalse%
\ less{\isacharunderscore}{\kern0pt}eq{\isacharunderscore}{\kern0pt}float{\isachardot}{\kern0pt}rep{\isacharunderscore}{\kern0pt}eq\ mono{\isacharunderscore}{\kern0pt}def\ \isacommand{apply}\isamarkupfalse%
\ blast\isanewline
\ \ \ \ \ \ \isacommand{apply}\isamarkupfalse%
\ {\isacharparenleft}{\kern0pt}simp\ add{\isacharcolon}{\kern0pt}f{\isadigit{0}}{\isacharunderscore}{\kern0pt}sketch{\isacharunderscore}{\kern0pt}def{\isacharcomma}{\kern0pt}\ simp\ add{\isacharcolon}{\kern0pt}least{\isacharunderscore}{\kern0pt}def{\isacharparenright}{\kern0pt}\isanewline
\ \ \ \ \isacommand{using}\isamarkupfalse%
\ card{\isacharunderscore}{\kern0pt}eq{\isacharbrackleft}{\kern0pt}symmetric{\isacharbrackright}{\kern0pt}\ card{\isacharunderscore}{\kern0pt}gt{\isacharunderscore}{\kern0pt}{\isadigit{0}}{\isacharunderscore}{\kern0pt}iff\ t{\isacharunderscore}{\kern0pt}ge{\isacharunderscore}{\kern0pt}{\isadigit{0}}\ \isacommand{apply}\isamarkupfalse%
\ {\isacharparenleft}{\kern0pt}simp{\isacharcomma}{\kern0pt}\ force{\isacharparenright}{\kern0pt}\ \isanewline
\ \ \ \ \isacommand{by}\isamarkupfalse%
\ {\isacharparenleft}{\kern0pt}simp\ add{\isacharcolon}{\kern0pt}real{\isacharunderscore}{\kern0pt}g{\isacharunderscore}{\kern0pt}{\isadigit{2}}{\isacharparenright}{\kern0pt}\isanewline
\ \isanewline
\ \ \isacommand{have}\isamarkupfalse%
\ {\isachardoublequoteopen}{\isadigit{1}}{\isacharminus}{\kern0pt}real{\isacharunderscore}{\kern0pt}of{\isacharunderscore}{\kern0pt}rat\ {\isasymepsilon}\ {\isasymle}\ {\isasymP}{\isacharparenleft}{\kern0pt}{\isasymomega}\ in\ measure{\isacharunderscore}{\kern0pt}pmf\ {\isasymOmega}\isactrlsub {\isadigit{0}}{\isachardot}{\kern0pt}\isanewline
\ \ \ \ \ \ {\isasymbar}median\ {\isacharparenleft}{\kern0pt}{\isasymlambda}i{\isachardot}{\kern0pt}\ g{\isacharprime}{\kern0pt}\ {\isacharparenleft}{\kern0pt}h\ {\isacharparenleft}{\kern0pt}{\isasymomega}\ i{\isacharparenright}{\kern0pt}{\isacharparenright}{\kern0pt}{\isacharparenright}{\kern0pt}\ s\ {\isacharminus}{\kern0pt}\ real{\isacharunderscore}{\kern0pt}of{\isacharunderscore}{\kern0pt}rat\ {\isacharparenleft}{\kern0pt}F\ {\isadigit{0}}\ as{\isacharparenright}{\kern0pt}{\isasymbar}\ {\isasymle}\ \ real{\isacharunderscore}{\kern0pt}of{\isacharunderscore}{\kern0pt}rat\ {\isasymdelta}\ {\isacharasterisk}{\kern0pt}\ real{\isacharunderscore}{\kern0pt}of{\isacharunderscore}{\kern0pt}rat\ {\isacharparenleft}{\kern0pt}F\ {\isadigit{0}}\ as{\isacharparenright}{\kern0pt}{\isacharparenright}{\kern0pt}{\isachardoublequoteclose}\isanewline
\ \ \ \ \isacommand{apply}\isamarkupfalse%
\ {\isacharparenleft}{\kern0pt}rule\ prob{\isacharunderscore}{\kern0pt}space{\isachardot}{\kern0pt}median{\isacharunderscore}{\kern0pt}bound{\isacharunderscore}{\kern0pt}{\isadigit{2}}{\isacharcomma}{\kern0pt}\ simp\ add{\isacharcolon}{\kern0pt}prob{\isacharunderscore}{\kern0pt}space{\isacharunderscore}{\kern0pt}measure{\isacharunderscore}{\kern0pt}pmf{\isacharparenright}{\kern0pt}\isanewline
\ \ \ \ \ \ \ \isacommand{using}\isamarkupfalse%
\ assms\ \isacommand{apply}\isamarkupfalse%
\ simp\ \isanewline
\ \ \ \ \ \ \isacommand{apply}\isamarkupfalse%
\ {\isacharparenleft}{\kern0pt}subst\ {\isasymOmega}\isactrlsub {\isadigit{0}}{\isacharunderscore}{\kern0pt}def{\isacharparenright}{\kern0pt}\isanewline
\ \ \ \ \ \ \isacommand{apply}\isamarkupfalse%
\ {\isacharparenleft}{\kern0pt}rule\ indep{\isacharunderscore}{\kern0pt}vars{\isacharunderscore}{\kern0pt}restrict{\isacharunderscore}{\kern0pt}intro\ {\isacharbrackleft}{\kern0pt}\isakeyword{where}\ f{\isacharequal}{\kern0pt}{\isachardoublequoteopen}{\isasymlambda}j{\isachardot}{\kern0pt}\ {\isacharbraceleft}{\kern0pt}j{\isacharbraceright}{\kern0pt}{\isachardoublequoteclose}{\isacharbrackright}{\kern0pt}{\isacharcomma}{\kern0pt}\ simp{\isacharcomma}{\kern0pt}\ simp\ add{\isacharcolon}{\kern0pt}disjoint{\isacharunderscore}{\kern0pt}family{\isacharunderscore}{\kern0pt}on{\isacharunderscore}{\kern0pt}def{\isacharcomma}{\kern0pt}\ simp\ add{\isacharcolon}{\kern0pt}\ s{\isacharunderscore}{\kern0pt}ge{\isacharunderscore}{\kern0pt}{\isadigit{0}}{\isacharcomma}{\kern0pt}\ simp{\isacharcomma}{\kern0pt}\ simp{\isacharcomma}{\kern0pt}\ simp{\isacharparenright}{\kern0pt}\isanewline
\ \ \ \ \ \isacommand{apply}\isamarkupfalse%
\ {\isacharparenleft}{\kern0pt}simp\ add{\isacharcolon}{\kern0pt}s{\isacharunderscore}{\kern0pt}def{\isacharparenright}{\kern0pt}\ \isacommand{using}\isamarkupfalse%
\ of{\isacharunderscore}{\kern0pt}nat{\isacharunderscore}{\kern0pt}ceiling\ \isacommand{apply}\isamarkupfalse%
\ blast\isanewline
\ \ \ \ \isacommand{apply}\isamarkupfalse%
\ simp\isanewline
\ \ \ \ \isacommand{apply}\isamarkupfalse%
\ {\isacharparenleft}{\kern0pt}subst\ {\isasymOmega}\isactrlsub {\isadigit{0}}{\isacharunderscore}{\kern0pt}def{\isacharparenright}{\kern0pt}\isanewline
\ \ \ \ \isacommand{apply}\isamarkupfalse%
\ {\isacharparenleft}{\kern0pt}subst\ prob{\isacharunderscore}{\kern0pt}prod{\isacharunderscore}{\kern0pt}pmf{\isacharunderscore}{\kern0pt}slice{\isacharcomma}{\kern0pt}\ simp{\isacharcomma}{\kern0pt}\ simp{\isacharparenright}{\kern0pt}\isanewline
\ \ \ \ \isacommand{using}\isamarkupfalse%
\ b\ \isacommand{by}\isamarkupfalse%
\ {\isacharparenleft}{\kern0pt}simp\ add{\isacharcolon}{\kern0pt}{\isasymOmega}\isactrlsub {\isadigit{1}}{\isacharunderscore}{\kern0pt}def{\isacharparenright}{\kern0pt}\ \isanewline
\ \ \isacommand{also}\isamarkupfalse%
\ \isacommand{have}\isamarkupfalse%
\ {\isachardoublequoteopen}{\isachardot}{\kern0pt}{\isachardot}{\kern0pt}{\isachardot}{\kern0pt}\ {\isacharequal}{\kern0pt}\ {\isasymP}{\isacharparenleft}{\kern0pt}{\isasymomega}\ in\ measure{\isacharunderscore}{\kern0pt}pmf\ {\isasymOmega}\isactrlsub {\isadigit{0}}{\isachardot}{\kern0pt}\ \isanewline
\ \ \ \ \ \ {\isasymbar}median\ {\isacharparenleft}{\kern0pt}{\isasymlambda}i{\isachardot}{\kern0pt}\ g\ {\isacharparenleft}{\kern0pt}f{\isadigit{0}}{\isacharunderscore}{\kern0pt}sketch\ p\ r\ t\ {\isacharparenleft}{\kern0pt}{\isasymomega}\ i{\isacharparenright}{\kern0pt}\ as{\isacharparenright}{\kern0pt}{\isacharparenright}{\kern0pt}\ s\ {\isacharminus}{\kern0pt}\ F\ {\isadigit{0}}\ as{\isasymbar}\ {\isasymle}\ \ {\isasymdelta}\ {\isacharasterisk}{\kern0pt}\ F\ {\isadigit{0}}\ as{\isacharparenright}{\kern0pt}{\isachardoublequoteclose}\isanewline
\ \ \ \ \isacommand{apply}\isamarkupfalse%
\ {\isacharparenleft}{\kern0pt}rule\ arg{\isacharunderscore}{\kern0pt}cong{\isadigit{2}}{\isacharbrackleft}{\kern0pt}\isakeyword{where}\ f{\isacharequal}{\kern0pt}{\isachardoublequoteopen}measure{\isachardoublequoteclose}{\isacharbrackright}{\kern0pt}{\isacharcomma}{\kern0pt}\ simp{\isacharparenright}{\kern0pt}\isanewline
\ \ \ \ \isacommand{apply}\isamarkupfalse%
\ {\isacharparenleft}{\kern0pt}rule\ Collect{\isacharunderscore}{\kern0pt}cong{\isacharcomma}{\kern0pt}\ simp{\isacharcomma}{\kern0pt}\ subst\ real{\isacharunderscore}{\kern0pt}g{\isacharbrackleft}{\kern0pt}symmetric{\isacharbrackright}{\kern0pt}{\isacharparenright}{\kern0pt}\isanewline
\ \ \ \ \isacommand{apply}\isamarkupfalse%
\ {\isacharparenleft}{\kern0pt}subst\ of{\isacharunderscore}{\kern0pt}rat{\isacharunderscore}{\kern0pt}mult{\isacharbrackleft}{\kern0pt}symmetric{\isacharbrackright}{\kern0pt}{\isacharcomma}{\kern0pt}\ subst\ median{\isacharunderscore}{\kern0pt}rat{\isacharbrackleft}{\kern0pt}OF\ s{\isacharunderscore}{\kern0pt}ge{\isacharunderscore}{\kern0pt}{\isadigit{0}}{\isacharcomma}{\kern0pt}\ symmetric{\isacharbrackright}{\kern0pt}{\isacharparenright}{\kern0pt}\isanewline
\ \ \ \ \isacommand{apply}\isamarkupfalse%
\ {\isacharparenleft}{\kern0pt}subst\ of{\isacharunderscore}{\kern0pt}rat{\isacharunderscore}{\kern0pt}diff{\isacharbrackleft}{\kern0pt}symmetric{\isacharbrackright}{\kern0pt}{\isacharcomma}{\kern0pt}\ simp{\isacharparenright}{\kern0pt}\isanewline
\ \ \ \ \isacommand{using}\isamarkupfalse%
\ of{\isacharunderscore}{\kern0pt}rat{\isacharunderscore}{\kern0pt}less{\isacharunderscore}{\kern0pt}eq\ \isacommand{by}\isamarkupfalse%
\ blast\isanewline
\ \ \isacommand{finally}\isamarkupfalse%
\ \isacommand{have}\isamarkupfalse%
\ a{\isacharcolon}{\kern0pt}{\isachardoublequoteopen}{\isasymP}{\isacharparenleft}{\kern0pt}{\isasymomega}\ in\ measure{\isacharunderscore}{\kern0pt}pmf\ {\isasymOmega}\isactrlsub {\isadigit{0}}{\isachardot}{\kern0pt}\ \ \isanewline
\ \ \ \ \ \ {\isasymbar}median\ {\isacharparenleft}{\kern0pt}{\isasymlambda}i{\isachardot}{\kern0pt}\ g\ {\isacharparenleft}{\kern0pt}f{\isadigit{0}}{\isacharunderscore}{\kern0pt}sketch\ p\ r\ t\ {\isacharparenleft}{\kern0pt}{\isasymomega}\ i{\isacharparenright}{\kern0pt}\ as{\isacharparenright}{\kern0pt}{\isacharparenright}{\kern0pt}\ s\ {\isacharminus}{\kern0pt}\ F\ {\isadigit{0}}\ as{\isasymbar}\ {\isasymle}\ {\isasymdelta}\ {\isacharasterisk}{\kern0pt}\ F\ {\isadigit{0}}\ as{\isacharparenright}{\kern0pt}\ {\isasymge}\ {\isadigit{1}}{\isacharminus}{\kern0pt}real{\isacharunderscore}{\kern0pt}of{\isacharunderscore}{\kern0pt}rat\ {\isasymepsilon}{\isachardoublequoteclose}\isanewline
\ \ \ \ \isacommand{by}\isamarkupfalse%
\ blast\isanewline
\isanewline
\ \ \isacommand{show}\isamarkupfalse%
\ {\isacharquery}{\kern0pt}thesis\isanewline
\ \ \ \ \isacommand{apply}\isamarkupfalse%
\ {\isacharparenleft}{\kern0pt}subst\ M{\isacharunderscore}{\kern0pt}def{\isacharparenright}{\kern0pt}\isanewline
\ \ \ \ \isacommand{apply}\isamarkupfalse%
\ {\isacharparenleft}{\kern0pt}subst\ f{\isadigit{0}}{\isacharunderscore}{\kern0pt}alg{\isacharunderscore}{\kern0pt}sketch{\isacharbrackleft}{\kern0pt}OF\ assms{\isacharparenleft}{\kern0pt}{\isadigit{1}}{\isacharparenright}{\kern0pt}\ assms{\isacharparenleft}{\kern0pt}{\isadigit{2}}{\isacharparenright}{\kern0pt}{\isacharbrackright}{\kern0pt}{\isacharcomma}{\kern0pt}\ simp{\isacharparenright}{\kern0pt}\isanewline
\ \ \ \ \isacommand{apply}\isamarkupfalse%
\ {\isacharparenleft}{\kern0pt}simp\ add{\isacharcolon}{\kern0pt}t{\isacharunderscore}{\kern0pt}def{\isacharbrackleft}{\kern0pt}symmetric{\isacharbrackright}{\kern0pt}\ p{\isacharunderscore}{\kern0pt}def{\isacharbrackleft}{\kern0pt}symmetric{\isacharbrackright}{\kern0pt}\ r{\isacharunderscore}{\kern0pt}def{\isacharbrackleft}{\kern0pt}symmetric{\isacharbrackright}{\kern0pt}\ s{\isacharunderscore}{\kern0pt}def{\isacharbrackleft}{\kern0pt}symmetric{\isacharbrackright}{\kern0pt}\ map{\isacharunderscore}{\kern0pt}pmf{\isacharunderscore}{\kern0pt}def{\isacharparenright}{\kern0pt}\isanewline
\ \ \ \ \isacommand{apply}\isamarkupfalse%
\ {\isacharparenleft}{\kern0pt}subst\ bind{\isacharunderscore}{\kern0pt}assoc{\isacharunderscore}{\kern0pt}pmf{\isacharparenright}{\kern0pt}\isanewline
\ \ \ \ \isacommand{apply}\isamarkupfalse%
\ {\isacharparenleft}{\kern0pt}subst\ bind{\isacharunderscore}{\kern0pt}return{\isacharunderscore}{\kern0pt}pmf{\isacharparenright}{\kern0pt}\isanewline
\ \ \ \ \isacommand{apply}\isamarkupfalse%
\ {\isacharparenleft}{\kern0pt}subst\ f{\isadigit{0}}{\isacharunderscore}{\kern0pt}result{\isacharunderscore}{\kern0pt}elim{\isacharparenright}{\kern0pt}\isanewline
\ \ \ \ \isacommand{apply}\isamarkupfalse%
\ {\isacharparenleft}{\kern0pt}subst\ map{\isacharunderscore}{\kern0pt}pmf{\isacharunderscore}{\kern0pt}def{\isacharbrackleft}{\kern0pt}symmetric{\isacharbrackright}{\kern0pt}{\isacharparenright}{\kern0pt}\isanewline
\ \ \ \ \isacommand{using}\isamarkupfalse%
\ a\ \isacommand{by}\isamarkupfalse%
\ {\isacharparenleft}{\kern0pt}simp\ add{\isacharcolon}{\kern0pt}{\isasymOmega}\isactrlsub {\isadigit{0}}{\isacharunderscore}{\kern0pt}def{\isacharbrackleft}{\kern0pt}symmetric{\isacharbrackright}{\kern0pt}{\isacharparenright}{\kern0pt}\isanewline
\isacommand{qed}\isamarkupfalse%
%
\endisatagproof
{\isafoldproof}%
%
\isadelimproof
\isanewline
%
\endisadelimproof
\isanewline
\isacommand{fun}\isamarkupfalse%
\ f{\isadigit{0}}{\isacharunderscore}{\kern0pt}space{\isacharunderscore}{\kern0pt}usage\ {\isacharcolon}{\kern0pt}{\isacharcolon}{\kern0pt}\ {\isachardoublequoteopen}{\isacharparenleft}{\kern0pt}nat\ {\isasymtimes}\ rat\ {\isasymtimes}\ rat{\isacharparenright}{\kern0pt}\ {\isasymRightarrow}\ real{\isachardoublequoteclose}\ \isakeyword{where}\isanewline
\ \ {\isachardoublequoteopen}f{\isadigit{0}}{\isacharunderscore}{\kern0pt}space{\isacharunderscore}{\kern0pt}usage\ {\isacharparenleft}{\kern0pt}n{\isacharcomma}{\kern0pt}\ {\isasymepsilon}{\isacharcomma}{\kern0pt}\ {\isasymdelta}{\isacharparenright}{\kern0pt}\ {\isacharequal}{\kern0pt}\ {\isacharparenleft}{\kern0pt}\isanewline
\ \ \ \ let\ s\ {\isacharequal}{\kern0pt}\ nat\ {\isasymlceil}{\isacharminus}{\kern0pt}{\isadigit{1}}{\isadigit{8}}\ {\isacharasterisk}{\kern0pt}\ ln\ {\isacharparenleft}{\kern0pt}real{\isacharunderscore}{\kern0pt}of{\isacharunderscore}{\kern0pt}rat\ {\isasymepsilon}{\isacharparenright}{\kern0pt}{\isasymrceil}\ in\ \isanewline
\ \ \ \ let\ r\ {\isacharequal}{\kern0pt}\ nat\ {\isacharparenleft}{\kern0pt}{\isadigit{4}}\ {\isacharasterisk}{\kern0pt}\ {\isasymlceil}log\ {\isadigit{2}}\ {\isacharparenleft}{\kern0pt}{\isadigit{1}}\ {\isacharslash}{\kern0pt}\ real{\isacharunderscore}{\kern0pt}of{\isacharunderscore}{\kern0pt}rat\ {\isasymdelta}{\isacharparenright}{\kern0pt}{\isasymrceil}\ {\isacharplus}{\kern0pt}\ {\isadigit{2}}{\isadigit{4}}{\isacharparenright}{\kern0pt}\ in\isanewline
\ \ \ \ let\ t\ {\isacharequal}{\kern0pt}\ nat\ {\isasymlceil}{\isadigit{8}}{\isadigit{0}}\ {\isacharslash}{\kern0pt}\ {\isacharparenleft}{\kern0pt}real{\isacharunderscore}{\kern0pt}of{\isacharunderscore}{\kern0pt}rat\ {\isasymdelta}{\isacharparenright}{\kern0pt}\isactrlsup {\isadigit{2}}\ {\isasymrceil}\ in\isanewline
\ \ \ \ {\isadigit{8}}\ {\isacharplus}{\kern0pt}\isanewline
\ \ \ \ {\isadigit{2}}\ {\isacharasterisk}{\kern0pt}\ log\ {\isadigit{2}}\ {\isacharparenleft}{\kern0pt}real\ s\ {\isacharplus}{\kern0pt}\ {\isadigit{1}}{\isacharparenright}{\kern0pt}\ {\isacharplus}{\kern0pt}\isanewline
\ \ \ \ {\isadigit{2}}\ {\isacharasterisk}{\kern0pt}\ log\ {\isadigit{2}}\ {\isacharparenleft}{\kern0pt}real\ t\ {\isacharplus}{\kern0pt}\ {\isadigit{1}}{\isacharparenright}{\kern0pt}\ {\isacharplus}{\kern0pt}\isanewline
\ \ \ \ {\isadigit{2}}\ {\isacharasterisk}{\kern0pt}\ log\ {\isadigit{2}}\ {\isacharparenleft}{\kern0pt}real\ n\ {\isacharplus}{\kern0pt}\ {\isadigit{1}}{\isadigit{0}}{\isacharparenright}{\kern0pt}\ {\isacharplus}{\kern0pt}\isanewline
\ \ \ \ {\isadigit{2}}\ {\isacharasterisk}{\kern0pt}\ log\ {\isadigit{2}}\ {\isacharparenleft}{\kern0pt}real\ r\ {\isacharplus}{\kern0pt}\ {\isadigit{1}}{\isacharparenright}{\kern0pt}\ {\isacharplus}{\kern0pt}\isanewline
\ \ \ \ real\ s\ {\isacharasterisk}{\kern0pt}\ {\isacharparenleft}{\kern0pt}{\isadigit{1}}{\isadigit{2}}\ {\isacharplus}{\kern0pt}\ {\isadigit{4}}\ {\isacharasterisk}{\kern0pt}\ log\ {\isadigit{2}}\ {\isacharparenleft}{\kern0pt}{\isadigit{1}}{\isadigit{0}}\ {\isacharplus}{\kern0pt}\ real\ n{\isacharparenright}{\kern0pt}\ {\isacharplus}{\kern0pt}\isanewline
\ \ \ \ real\ t\ {\isacharasterisk}{\kern0pt}\ {\isacharparenleft}{\kern0pt}{\isadigit{1}}{\isadigit{1}}\ {\isacharplus}{\kern0pt}\ {\isadigit{4}}\ {\isacharasterisk}{\kern0pt}\ r\ {\isacharplus}{\kern0pt}\ {\isadigit{2}}\ {\isacharasterisk}{\kern0pt}\ log\ {\isadigit{2}}\ {\isacharparenleft}{\kern0pt}log\ {\isadigit{2}}\ {\isacharparenleft}{\kern0pt}real\ n\ {\isacharplus}{\kern0pt}\ {\isadigit{9}}{\isacharparenright}{\kern0pt}{\isacharparenright}{\kern0pt}{\isacharparenright}{\kern0pt}{\isacharparenright}{\kern0pt}{\isacharparenright}{\kern0pt}{\isachardoublequoteclose}\isanewline
\isanewline
\isacommand{definition}\isamarkupfalse%
\ encode{\isacharunderscore}{\kern0pt}state\ \isakeyword{where}\isanewline
\ \ {\isachardoublequoteopen}encode{\isacharunderscore}{\kern0pt}state\ {\isacharequal}{\kern0pt}\ \isanewline
\ \ \ \ N\isactrlsub S\ {\isasymtimes}\isactrlsub D\ {\isacharparenleft}{\kern0pt}{\isasymlambda}s{\isachardot}{\kern0pt}\ \isanewline
\ \ \ \ N\isactrlsub S\ {\isasymtimes}\isactrlsub S\ {\isacharparenleft}{\kern0pt}\isanewline
\ \ \ \ N\isactrlsub S\ {\isasymtimes}\isactrlsub D\ {\isacharparenleft}{\kern0pt}{\isasymlambda}p{\isachardot}{\kern0pt}\ \isanewline
\ \ \ \ N\isactrlsub S\ {\isasymtimes}\isactrlsub S\ {\isacharparenleft}{\kern0pt}\ \isanewline
\ \ \ \ {\isacharparenleft}{\kern0pt}{\isacharbrackleft}{\kern0pt}{\isadigit{0}}{\isachardot}{\kern0pt}{\isachardot}{\kern0pt}{\isacharless}{\kern0pt}s{\isacharbrackright}{\kern0pt}\ {\isasymrightarrow}\isactrlsub S\ {\isacharparenleft}{\kern0pt}list\isactrlsub S\ {\isacharparenleft}{\kern0pt}zfact\isactrlsub S\ p{\isacharparenright}{\kern0pt}{\isacharparenright}{\kern0pt}{\isacharparenright}{\kern0pt}\ {\isasymtimes}\isactrlsub S\isanewline
\ \ \ \ {\isacharparenleft}{\kern0pt}{\isacharbrackleft}{\kern0pt}{\isadigit{0}}{\isachardot}{\kern0pt}{\isachardot}{\kern0pt}{\isacharless}{\kern0pt}s{\isacharbrackright}{\kern0pt}\ {\isasymrightarrow}\isactrlsub S\ {\isacharparenleft}{\kern0pt}set\isactrlsub S\ F\isactrlsub S{\isacharparenright}{\kern0pt}{\isacharparenright}{\kern0pt}{\isacharparenright}{\kern0pt}{\isacharparenright}{\kern0pt}{\isacharparenright}{\kern0pt}{\isacharparenright}{\kern0pt}{\isachardoublequoteclose}\isanewline
\isanewline
\isacommand{lemma}\isamarkupfalse%
\ {\isachardoublequoteopen}inj{\isacharunderscore}{\kern0pt}on\ encode{\isacharunderscore}{\kern0pt}state\ {\isacharparenleft}{\kern0pt}dom\ encode{\isacharunderscore}{\kern0pt}state{\isacharparenright}{\kern0pt}{\isachardoublequoteclose}\isanewline
%
\isadelimproof
\ \ %
\endisadelimproof
%
\isatagproof
\isacommand{apply}\isamarkupfalse%
\ {\isacharparenleft}{\kern0pt}rule\ encoding{\isacharunderscore}{\kern0pt}imp{\isacharunderscore}{\kern0pt}inj{\isacharparenright}{\kern0pt}\isanewline
\ \ \isacommand{apply}\isamarkupfalse%
\ {\isacharparenleft}{\kern0pt}simp\ add{\isacharcolon}{\kern0pt}\ encode{\isacharunderscore}{\kern0pt}state{\isacharunderscore}{\kern0pt}def{\isacharparenright}{\kern0pt}\isanewline
\ \ \isacommand{apply}\isamarkupfalse%
\ {\isacharparenleft}{\kern0pt}rule\ dependent{\isacharunderscore}{\kern0pt}encoding{\isacharcomma}{\kern0pt}\ metis\ nat{\isacharunderscore}{\kern0pt}encoding{\isacharparenright}{\kern0pt}\isanewline
\ \ \isacommand{apply}\isamarkupfalse%
\ {\isacharparenleft}{\kern0pt}rule\ prod{\isacharunderscore}{\kern0pt}encoding{\isacharcomma}{\kern0pt}\ metis\ nat{\isacharunderscore}{\kern0pt}encoding{\isacharparenright}{\kern0pt}\isanewline
\ \ \isacommand{apply}\isamarkupfalse%
\ {\isacharparenleft}{\kern0pt}rule\ dependent{\isacharunderscore}{\kern0pt}encoding{\isacharcomma}{\kern0pt}\ metis\ nat{\isacharunderscore}{\kern0pt}encoding{\isacharparenright}{\kern0pt}\isanewline
\ \ \isacommand{apply}\isamarkupfalse%
\ {\isacharparenleft}{\kern0pt}rule\ prod{\isacharunderscore}{\kern0pt}encoding{\isacharcomma}{\kern0pt}\ metis\ nat{\isacharunderscore}{\kern0pt}encoding{\isacharparenright}{\kern0pt}\isanewline
\ \ \isacommand{apply}\isamarkupfalse%
\ {\isacharparenleft}{\kern0pt}rule\ prod{\isacharunderscore}{\kern0pt}encoding{\isacharcomma}{\kern0pt}\ metis\ encode{\isacharunderscore}{\kern0pt}extensional\ list{\isacharunderscore}{\kern0pt}encoding\ zfact{\isacharunderscore}{\kern0pt}encoding{\isacharparenright}{\kern0pt}\isanewline
\ \ \isacommand{by}\isamarkupfalse%
\ {\isacharparenleft}{\kern0pt}rule\ encode{\isacharunderscore}{\kern0pt}extensional{\isacharcomma}{\kern0pt}\ rule\ encode{\isacharunderscore}{\kern0pt}set{\isacharcomma}{\kern0pt}\ rule\ encode{\isacharunderscore}{\kern0pt}float{\isacharparenright}{\kern0pt}%
\endisatagproof
{\isafoldproof}%
%
\isadelimproof
\isanewline
%
\endisadelimproof
\isanewline
\isacommand{lemma}\isamarkupfalse%
\ f{\isacharunderscore}{\kern0pt}subset{\isacharcolon}{\kern0pt}\isanewline
\ \ \isakeyword{assumes}\ {\isachardoublequoteopen}g\ {\isacharbackquote}{\kern0pt}\ A\ {\isasymsubseteq}\ h\ {\isacharbackquote}{\kern0pt}\ B{\isachardoublequoteclose}\isanewline
\ \ \isakeyword{shows}\ {\isachardoublequoteopen}{\isacharparenleft}{\kern0pt}{\isasymlambda}x{\isachardot}{\kern0pt}\ f\ {\isacharparenleft}{\kern0pt}g\ x{\isacharparenright}{\kern0pt}{\isacharparenright}{\kern0pt}\ {\isacharbackquote}{\kern0pt}\ A\ {\isasymsubseteq}\ {\isacharparenleft}{\kern0pt}{\isasymlambda}x{\isachardot}{\kern0pt}\ f\ {\isacharparenleft}{\kern0pt}h\ x{\isacharparenright}{\kern0pt}{\isacharparenright}{\kern0pt}\ {\isacharbackquote}{\kern0pt}\ B{\isachardoublequoteclose}\isanewline
%
\isadelimproof
\ \ %
\endisadelimproof
%
\isatagproof
\isacommand{using}\isamarkupfalse%
\ assms\ \isacommand{by}\isamarkupfalse%
\ auto%
\endisatagproof
{\isafoldproof}%
%
\isadelimproof
\isanewline
%
\endisadelimproof
\isanewline
\isacommand{theorem}\isamarkupfalse%
\ f{\isadigit{0}}{\isacharunderscore}{\kern0pt}exact{\isacharunderscore}{\kern0pt}space{\isacharunderscore}{\kern0pt}usage{\isacharcolon}{\kern0pt}\isanewline
\ \ \isakeyword{assumes}\ {\isachardoublequoteopen}{\isasymepsilon}\ {\isasymin}\ {\isacharbraceleft}{\kern0pt}{\isadigit{0}}{\isacharless}{\kern0pt}{\isachardot}{\kern0pt}{\isachardot}{\kern0pt}{\isacharless}{\kern0pt}{\isadigit{1}}{\isacharbraceright}{\kern0pt}{\isachardoublequoteclose}\isanewline
\ \ \isakeyword{assumes}\ {\isachardoublequoteopen}{\isasymdelta}\ {\isasymin}\ {\isacharbraceleft}{\kern0pt}{\isadigit{0}}{\isacharless}{\kern0pt}{\isachardot}{\kern0pt}{\isachardot}{\kern0pt}{\isacharless}{\kern0pt}{\isadigit{1}}{\isacharbraceright}{\kern0pt}{\isachardoublequoteclose}\isanewline
\ \ \isakeyword{assumes}\ {\isachardoublequoteopen}set\ as\ {\isasymsubseteq}\ {\isacharbraceleft}{\kern0pt}{\isadigit{0}}{\isachardot}{\kern0pt}{\isachardot}{\kern0pt}{\isacharless}{\kern0pt}n{\isacharbraceright}{\kern0pt}{\isachardoublequoteclose}\isanewline
\ \ \isakeyword{defines}\ {\isachardoublequoteopen}M\ {\isasymequiv}\ fold\ {\isacharparenleft}{\kern0pt}{\isasymlambda}a\ state{\isachardot}{\kern0pt}\ state\ {\isasymbind}\ f{\isadigit{0}}{\isacharunderscore}{\kern0pt}update\ a{\isacharparenright}{\kern0pt}\ as\ {\isacharparenleft}{\kern0pt}f{\isadigit{0}}{\isacharunderscore}{\kern0pt}init\ {\isasymdelta}\ {\isasymepsilon}\ n{\isacharparenright}{\kern0pt}{\isachardoublequoteclose}\isanewline
\ \ \isakeyword{shows}\ {\isachardoublequoteopen}AE\ {\isasymomega}\ in\ M{\isachardot}{\kern0pt}\ bit{\isacharunderscore}{\kern0pt}count\ {\isacharparenleft}{\kern0pt}encode{\isacharunderscore}{\kern0pt}state\ {\isasymomega}{\isacharparenright}{\kern0pt}\ {\isasymle}\ f{\isadigit{0}}{\isacharunderscore}{\kern0pt}space{\isacharunderscore}{\kern0pt}usage\ {\isacharparenleft}{\kern0pt}n{\isacharcomma}{\kern0pt}\ {\isasymepsilon}{\isacharcomma}{\kern0pt}\ {\isasymdelta}{\isacharparenright}{\kern0pt}{\isachardoublequoteclose}\isanewline
%
\isadelimproof
%
\endisadelimproof
%
\isatagproof
\isacommand{proof}\isamarkupfalse%
\ {\isacharminus}{\kern0pt}\isanewline
\ \ \isacommand{define}\isamarkupfalse%
\ s\ \isakeyword{where}\ {\isachardoublequoteopen}s\ {\isacharequal}{\kern0pt}\ nat\ {\isasymlceil}{\isacharminus}{\kern0pt}{\isacharparenleft}{\kern0pt}{\isadigit{1}}{\isadigit{8}}{\isacharasterisk}{\kern0pt}\ ln\ {\isacharparenleft}{\kern0pt}real{\isacharunderscore}{\kern0pt}of{\isacharunderscore}{\kern0pt}rat\ {\isasymepsilon}{\isacharparenright}{\kern0pt}{\isacharparenright}{\kern0pt}{\isasymrceil}{\isachardoublequoteclose}\isanewline
\ \ \isacommand{define}\isamarkupfalse%
\ t\ \isakeyword{where}\ {\isachardoublequoteopen}t\ {\isacharequal}{\kern0pt}\ nat\ {\isasymlceil}{\isadigit{8}}{\isadigit{0}}\ {\isacharslash}{\kern0pt}\ {\isacharparenleft}{\kern0pt}real{\isacharunderscore}{\kern0pt}of{\isacharunderscore}{\kern0pt}rat\ {\isasymdelta}{\isacharparenright}{\kern0pt}\isactrlsup {\isadigit{2}}{\isasymrceil}{\isachardoublequoteclose}\isanewline
\ \ \isacommand{define}\isamarkupfalse%
\ p\ \isakeyword{where}\ {\isachardoublequoteopen}p\ {\isacharequal}{\kern0pt}\ find{\isacharunderscore}{\kern0pt}prime{\isacharunderscore}{\kern0pt}above\ {\isacharparenleft}{\kern0pt}max\ n\ {\isadigit{1}}{\isadigit{9}}{\isacharparenright}{\kern0pt}{\isachardoublequoteclose}\isanewline
\ \ \isacommand{define}\isamarkupfalse%
\ r\ \isakeyword{where}\ {\isachardoublequoteopen}r\ {\isacharequal}{\kern0pt}\ nat\ {\isacharparenleft}{\kern0pt}{\isadigit{4}}\ {\isacharasterisk}{\kern0pt}\ {\isasymlceil}log\ {\isadigit{2}}\ {\isacharparenleft}{\kern0pt}{\isadigit{1}}\ {\isacharslash}{\kern0pt}\ real{\isacharunderscore}{\kern0pt}of{\isacharunderscore}{\kern0pt}rat\ {\isasymdelta}{\isacharparenright}{\kern0pt}{\isasymrceil}\ {\isacharplus}{\kern0pt}\ {\isadigit{2}}{\isadigit{4}}{\isacharparenright}{\kern0pt}{\isachardoublequoteclose}\isanewline
\isanewline
\ \ \isacommand{have}\isamarkupfalse%
\ n{\isacharunderscore}{\kern0pt}le{\isacharunderscore}{\kern0pt}p{\isacharcolon}{\kern0pt}\ {\isachardoublequoteopen}n\ {\isasymle}\ p{\isachardoublequoteclose}\ \isanewline
\ \ \ \ \isacommand{apply}\isamarkupfalse%
\ {\isacharparenleft}{\kern0pt}rule\ order{\isacharunderscore}{\kern0pt}trans{\isacharbrackleft}{\kern0pt}\isakeyword{where}\ y{\isacharequal}{\kern0pt}{\isachardoublequoteopen}max\ n\ {\isadigit{1}}{\isadigit{9}}{\isachardoublequoteclose}{\isacharbrackright}{\kern0pt}{\isacharcomma}{\kern0pt}\ simp{\isacharparenright}{\kern0pt}\isanewline
\ \ \ \ \isacommand{apply}\isamarkupfalse%
\ {\isacharparenleft}{\kern0pt}subst\ p{\isacharunderscore}{\kern0pt}def{\isacharparenright}{\kern0pt}\isanewline
\ \ \ \ \isacommand{by}\isamarkupfalse%
\ {\isacharparenleft}{\kern0pt}rule\ find{\isacharunderscore}{\kern0pt}prime{\isacharunderscore}{\kern0pt}above{\isacharunderscore}{\kern0pt}lower{\isacharunderscore}{\kern0pt}bound{\isacharparenright}{\kern0pt}\isanewline
\isanewline
\ \ \isacommand{have}\isamarkupfalse%
\ p{\isacharunderscore}{\kern0pt}ge{\isacharunderscore}{\kern0pt}{\isadigit{0}}{\isacharcolon}{\kern0pt}\ {\isachardoublequoteopen}p\ {\isachargreater}{\kern0pt}\ {\isadigit{0}}{\isachardoublequoteclose}\isanewline
\ \ \ \ \isacommand{apply}\isamarkupfalse%
\ {\isacharparenleft}{\kern0pt}rule\ prime{\isacharunderscore}{\kern0pt}gt{\isacharunderscore}{\kern0pt}{\isadigit{0}}{\isacharunderscore}{\kern0pt}nat{\isacharparenright}{\kern0pt}\isanewline
\ \ \ \ \isacommand{by}\isamarkupfalse%
\ {\isacharparenleft}{\kern0pt}simp\ add{\isacharcolon}{\kern0pt}p{\isacharunderscore}{\kern0pt}def\ find{\isacharunderscore}{\kern0pt}prime{\isacharunderscore}{\kern0pt}above{\isacharunderscore}{\kern0pt}is{\isacharunderscore}{\kern0pt}prime{\isacharparenright}{\kern0pt}\isanewline
\isanewline
\ \ \isacommand{have}\isamarkupfalse%
\ p{\isacharunderscore}{\kern0pt}le{\isacharunderscore}{\kern0pt}n{\isacharcolon}{\kern0pt}\ {\isachardoublequoteopen}p\ {\isasymle}\ {\isadigit{2}}{\isacharasterisk}{\kern0pt}n\ {\isacharplus}{\kern0pt}\ {\isadigit{1}}{\isadigit{9}}{\isachardoublequoteclose}\isanewline
\ \ \ \ \isacommand{apply}\isamarkupfalse%
\ {\isacharparenleft}{\kern0pt}simp\ add{\isacharcolon}{\kern0pt}p{\isacharunderscore}{\kern0pt}def{\isacharparenright}{\kern0pt}\isanewline
\ \ \ \ \isacommand{apply}\isamarkupfalse%
\ {\isacharparenleft}{\kern0pt}cases\ {\isachardoublequoteopen}n\ {\isasymle}\ {\isadigit{1}}{\isadigit{9}}{\isachardoublequoteclose}{\isacharcomma}{\kern0pt}\ simp\ add{\isacharcolon}{\kern0pt}find{\isacharunderscore}{\kern0pt}prime{\isacharunderscore}{\kern0pt}above{\isachardot}{\kern0pt}simps{\isacharparenright}{\kern0pt}\ \isanewline
\ \ \ \ \isacommand{apply}\isamarkupfalse%
\ {\isacharparenleft}{\kern0pt}rule\ order{\isacharunderscore}{\kern0pt}trans{\isacharbrackleft}{\kern0pt}\isakeyword{where}\ y{\isacharequal}{\kern0pt}{\isachardoublequoteopen}{\isadigit{2}}{\isacharasterisk}{\kern0pt}n{\isacharplus}{\kern0pt}{\isadigit{2}}{\isachardoublequoteclose}{\isacharbrackright}{\kern0pt}{\isacharcomma}{\kern0pt}\ simp\ add{\isacharcolon}{\kern0pt}find{\isacharunderscore}{\kern0pt}prime{\isacharunderscore}{\kern0pt}above{\isacharunderscore}{\kern0pt}upper{\isacharunderscore}{\kern0pt}bound{\isacharbrackleft}{\kern0pt}simplified{\isacharbrackright}{\kern0pt}{\isacharparenright}{\kern0pt}\isanewline
\ \ \ \ \isacommand{by}\isamarkupfalse%
\ simp\isanewline
\isanewline
\ \ \isacommand{have}\isamarkupfalse%
\ log{\isacharunderscore}{\kern0pt}{\isadigit{2}}{\isacharunderscore}{\kern0pt}{\isadigit{4}}{\isacharcolon}{\kern0pt}\ {\isachardoublequoteopen}log\ {\isadigit{2}}\ {\isadigit{4}}\ {\isacharequal}{\kern0pt}\ {\isadigit{2}}{\isachardoublequoteclose}\ \isanewline
\ \ \ \ \isacommand{by}\isamarkupfalse%
\ {\isacharparenleft}{\kern0pt}metis\ log{\isadigit{2}}{\isacharunderscore}{\kern0pt}of{\isacharunderscore}{\kern0pt}power{\isacharunderscore}{\kern0pt}eq\ mult{\isacharunderscore}{\kern0pt}{\isadigit{2}}\ numeral{\isacharunderscore}{\kern0pt}Bit{\isadigit{0}}\ of{\isacharunderscore}{\kern0pt}nat{\isacharunderscore}{\kern0pt}numeral\ power{\isadigit{2}}{\isacharunderscore}{\kern0pt}eq{\isacharunderscore}{\kern0pt}square{\isacharparenright}{\kern0pt}\isanewline
\isanewline
\ \ \isacommand{have}\isamarkupfalse%
\ b{\isacharunderscore}{\kern0pt}{\isadigit{4}}{\isacharunderscore}{\kern0pt}{\isadigit{2}}{\isadigit{2}}{\isacharcolon}{\kern0pt}\ {\isachardoublequoteopen}{\isasymAnd}y{\isachardot}{\kern0pt}\ y\ {\isasymin}\ {\isacharbraceleft}{\kern0pt}{\isadigit{0}}{\isachardot}{\kern0pt}{\isachardot}{\kern0pt}{\isacharless}{\kern0pt}p{\isacharbraceright}{\kern0pt}\ {\isasymLongrightarrow}\ bit{\isacharunderscore}{\kern0pt}count\ {\isacharparenleft}{\kern0pt}F\isactrlsub S\ {\isacharparenleft}{\kern0pt}float{\isacharunderscore}{\kern0pt}of\ {\isacharparenleft}{\kern0pt}truncate{\isacharunderscore}{\kern0pt}down\ r\ y{\isacharparenright}{\kern0pt}{\isacharparenright}{\kern0pt}{\isacharparenright}{\kern0pt}\ {\isasymle}\ \isanewline
\ \ \ \ ereal\ {\isacharparenleft}{\kern0pt}{\isadigit{1}}{\isadigit{0}}\ {\isacharplus}{\kern0pt}\ {\isadigit{4}}\ {\isacharasterisk}{\kern0pt}\ real\ r\ {\isacharplus}{\kern0pt}\ {\isadigit{2}}\ {\isacharasterisk}{\kern0pt}\ log\ {\isadigit{2}}\ {\isacharparenleft}{\kern0pt}log\ {\isadigit{2}}\ {\isacharparenleft}{\kern0pt}n{\isacharplus}{\kern0pt}{\isadigit{9}}{\isacharparenright}{\kern0pt}{\isacharparenright}{\kern0pt}{\isacharparenright}{\kern0pt}{\isachardoublequoteclose}\ \isanewline
\ \ \isacommand{proof}\isamarkupfalse%
\ {\isacharminus}{\kern0pt}\isanewline
\ \ \ \ \isacommand{fix}\isamarkupfalse%
\ y\isanewline
\ \ \ \ \isacommand{assume}\isamarkupfalse%
\ a{\isacharcolon}{\kern0pt}{\isachardoublequoteopen}y\ {\isasymin}\ {\isacharbraceleft}{\kern0pt}{\isadigit{0}}{\isachardot}{\kern0pt}{\isachardot}{\kern0pt}{\isacharless}{\kern0pt}p{\isacharbraceright}{\kern0pt}{\isachardoublequoteclose}\isanewline
\isanewline
\ \ \ \ \isacommand{show}\isamarkupfalse%
\ {\isachardoublequoteopen}\ bit{\isacharunderscore}{\kern0pt}count\ {\isacharparenleft}{\kern0pt}F\isactrlsub S\ {\isacharparenleft}{\kern0pt}float{\isacharunderscore}{\kern0pt}of\ {\isacharparenleft}{\kern0pt}truncate{\isacharunderscore}{\kern0pt}down\ r\ y{\isacharparenright}{\kern0pt}{\isacharparenright}{\kern0pt}{\isacharparenright}{\kern0pt}\ {\isasymle}\ ereal\ {\isacharparenleft}{\kern0pt}{\isadigit{1}}{\isadigit{0}}\ {\isacharplus}{\kern0pt}\ {\isadigit{4}}\ {\isacharasterisk}{\kern0pt}\ real\ r\ {\isacharplus}{\kern0pt}\ {\isadigit{2}}\ {\isacharasterisk}{\kern0pt}\ log\ {\isadigit{2}}\ {\isacharparenleft}{\kern0pt}log\ {\isadigit{2}}\ {\isacharparenleft}{\kern0pt}n{\isacharplus}{\kern0pt}{\isadigit{9}}{\isacharparenright}{\kern0pt}{\isacharparenright}{\kern0pt}{\isacharparenright}{\kern0pt}{\isachardoublequoteclose}\ \isanewline
\ \ \ \ \isacommand{proof}\isamarkupfalse%
\ {\isacharparenleft}{\kern0pt}cases\ {\isachardoublequoteopen}y\ {\isasymge}\ {\isadigit{1}}{\isachardoublequoteclose}{\isacharparenright}{\kern0pt}\isanewline
\ \ \ \ \ \ \isacommand{case}\isamarkupfalse%
\ True\isanewline
\isanewline
\ \ \ \ \ \ \isacommand{have}\isamarkupfalse%
\ b{\isacharunderscore}{\kern0pt}{\isadigit{4}}{\isacharunderscore}{\kern0pt}{\isadigit{2}}{\isadigit{3}}{\isacharcolon}{\kern0pt}\ {\isachardoublequoteopen}{\isadigit{0}}\ {\isacharless}{\kern0pt}\ {\isadigit{2}}\ {\isacharplus}{\kern0pt}\ log\ {\isadigit{2}}\ {\isacharparenleft}{\kern0pt}real\ p{\isacharparenright}{\kern0pt}{\isachardoublequoteclose}\ \isanewline
\ \ \ \ \ \ \ \isacommand{apply}\isamarkupfalse%
\ {\isacharparenleft}{\kern0pt}rule\ order{\isacharunderscore}{\kern0pt}less{\isacharunderscore}{\kern0pt}le{\isacharunderscore}{\kern0pt}trans{\isacharbrackleft}{\kern0pt}\isakeyword{where}\ y{\isacharequal}{\kern0pt}{\isachardoublequoteopen}{\isadigit{2}}{\isacharplus}{\kern0pt}log\ {\isadigit{2}}\ {\isadigit{1}}{\isachardoublequoteclose}{\isacharbrackright}{\kern0pt}{\isacharcomma}{\kern0pt}\ simp{\isacharparenright}{\kern0pt}\isanewline
\ \ \ \ \ \ \ \isacommand{using}\isamarkupfalse%
\ p{\isacharunderscore}{\kern0pt}ge{\isacharunderscore}{\kern0pt}{\isadigit{0}}\ \isacommand{by}\isamarkupfalse%
\ simp\isanewline
\isanewline
\ \ \ \ \ \ \isacommand{have}\isamarkupfalse%
\ {\isachardoublequoteopen}bit{\isacharunderscore}{\kern0pt}count\ {\isacharparenleft}{\kern0pt}F\isactrlsub S\ {\isacharparenleft}{\kern0pt}float{\isacharunderscore}{\kern0pt}of\ {\isacharparenleft}{\kern0pt}truncate{\isacharunderscore}{\kern0pt}down\ r\ y{\isacharparenright}{\kern0pt}{\isacharparenright}{\kern0pt}{\isacharparenright}{\kern0pt}\ {\isasymle}\ \ ereal\ {\isacharparenleft}{\kern0pt}{\isadigit{8}}\ {\isacharplus}{\kern0pt}\ {\isadigit{4}}\ {\isacharasterisk}{\kern0pt}\ real\ r\ {\isacharplus}{\kern0pt}\ {\isadigit{2}}\ {\isacharasterisk}{\kern0pt}\ log\ {\isadigit{2}}\ {\isacharparenleft}{\kern0pt}{\isadigit{2}}\ {\isacharplus}{\kern0pt}\ {\isasymbar}log\ {\isadigit{2}}\ {\isasymbar}real\ y{\isasymbar}{\isasymbar}{\isacharparenright}{\kern0pt}{\isacharparenright}{\kern0pt}{\isachardoublequoteclose}\isanewline
\ \ \ \ \ \ \ \ \isacommand{by}\isamarkupfalse%
\ {\isacharparenleft}{\kern0pt}rule\ truncate{\isacharunderscore}{\kern0pt}float{\isacharunderscore}{\kern0pt}bit{\isacharunderscore}{\kern0pt}count{\isacharparenright}{\kern0pt}\isanewline
\ \ \ \ \ \ \isacommand{also}\isamarkupfalse%
\ \isacommand{have}\isamarkupfalse%
\ {\isachardoublequoteopen}{\isachardot}{\kern0pt}{\isachardot}{\kern0pt}{\isachardot}{\kern0pt}\ {\isasymle}\ ereal\ {\isacharparenleft}{\kern0pt}{\isadigit{8}}\ {\isacharplus}{\kern0pt}\ {\isadigit{4}}\ {\isacharasterisk}{\kern0pt}\ real\ r\ {\isacharplus}{\kern0pt}\ {\isadigit{2}}\ {\isacharasterisk}{\kern0pt}\ log\ {\isadigit{2}}\ {\isacharparenleft}{\kern0pt}{\isadigit{2}}\ {\isacharplus}{\kern0pt}\ log\ {\isadigit{2}}\ p{\isacharparenright}{\kern0pt}{\isacharparenright}{\kern0pt}{\isachardoublequoteclose}\isanewline
\ \ \ \ \ \ \ \ \isacommand{apply}\isamarkupfalse%
\ {\isacharparenleft}{\kern0pt}simp{\isacharparenright}{\kern0pt}\isanewline
\ \ \ \ \ \ \ \ \isacommand{apply}\isamarkupfalse%
\ {\isacharparenleft}{\kern0pt}subst\ log{\isacharunderscore}{\kern0pt}le{\isacharunderscore}{\kern0pt}cancel{\isacharunderscore}{\kern0pt}iff{\isacharcomma}{\kern0pt}\ simp{\isacharcomma}{\kern0pt}\ simp{\isacharcomma}{\kern0pt}\ simp\ add{\isacharcolon}{\kern0pt}b{\isacharunderscore}{\kern0pt}{\isadigit{4}}{\isacharunderscore}{\kern0pt}{\isadigit{2}}{\isadigit{3}}{\isacharparenright}{\kern0pt}\isanewline
\ \ \ \ \ \ \ \ \isacommand{apply}\isamarkupfalse%
\ {\isacharparenleft}{\kern0pt}subst\ abs{\isacharunderscore}{\kern0pt}of{\isacharunderscore}{\kern0pt}nonneg{\isacharparenright}{\kern0pt}\ \isacommand{using}\isamarkupfalse%
\ True\ \isacommand{apply}\isamarkupfalse%
\ simp\isanewline
\ \ \ \ \ \ \ \ \isacommand{apply}\isamarkupfalse%
\ {\isacharparenleft}{\kern0pt}simp{\isacharcomma}{\kern0pt}\ subst\ log{\isacharunderscore}{\kern0pt}le{\isacharunderscore}{\kern0pt}cancel{\isacharunderscore}{\kern0pt}iff{\isacharcomma}{\kern0pt}\ simp{\isacharcomma}{\kern0pt}\ simp{\isacharparenright}{\kern0pt}\ \isacommand{using}\isamarkupfalse%
\ True\ \isacommand{apply}\isamarkupfalse%
\ simp\isanewline
\ \ \ \ \ \ \ \ \ \isacommand{apply}\isamarkupfalse%
\ {\isacharparenleft}{\kern0pt}simp\ add{\isacharcolon}{\kern0pt}p{\isacharunderscore}{\kern0pt}ge{\isacharunderscore}{\kern0pt}{\isadigit{0}}{\isacharparenright}{\kern0pt}\isanewline
\ \ \ \ \ \ \ \ \isacommand{using}\isamarkupfalse%
\ a\ \isacommand{by}\isamarkupfalse%
\ simp\isanewline
\ \ \ \ \ \ \isacommand{also}\isamarkupfalse%
\ \isacommand{have}\isamarkupfalse%
\ {\isachardoublequoteopen}{\isachardot}{\kern0pt}{\isachardot}{\kern0pt}{\isachardot}{\kern0pt}\ {\isasymle}\ ereal\ {\isacharparenleft}{\kern0pt}{\isadigit{8}}\ {\isacharplus}{\kern0pt}\ {\isadigit{4}}\ {\isacharasterisk}{\kern0pt}\ real\ r\ {\isacharplus}{\kern0pt}\ {\isadigit{2}}\ {\isacharasterisk}{\kern0pt}\ log\ {\isadigit{2}}\ {\isacharparenleft}{\kern0pt}log\ {\isadigit{2}}\ {\isadigit{4}}\ {\isacharplus}{\kern0pt}\ log\ {\isadigit{2}}\ {\isacharparenleft}{\kern0pt}{\isadigit{2}}\ {\isacharasterisk}{\kern0pt}\ n\ {\isacharplus}{\kern0pt}\ {\isadigit{1}}{\isadigit{9}}{\isacharparenright}{\kern0pt}{\isacharparenright}{\kern0pt}{\isacharparenright}{\kern0pt}{\isachardoublequoteclose}\isanewline
\ \ \ \ \ \ \ \ \isacommand{apply}\isamarkupfalse%
\ simp\isanewline
\ \ \ \ \ \ \ \ \isacommand{apply}\isamarkupfalse%
\ {\isacharparenleft}{\kern0pt}subst\ log{\isacharunderscore}{\kern0pt}le{\isacharunderscore}{\kern0pt}cancel{\isacharunderscore}{\kern0pt}iff{\isacharcomma}{\kern0pt}\ simp{\isacharcomma}{\kern0pt}\ simp\ add{\isacharcolon}{\kern0pt}{\isacharunderscore}{\kern0pt}b{\isacharunderscore}{\kern0pt}{\isadigit{4}}{\isacharunderscore}{\kern0pt}{\isadigit{2}}{\isadigit{3}}{\isacharparenright}{\kern0pt}\isanewline
\ \ \ \ \ \ \ \ \ \isacommand{apply}\isamarkupfalse%
\ {\isacharparenleft}{\kern0pt}rule\ add{\isacharunderscore}{\kern0pt}pos{\isacharunderscore}{\kern0pt}pos{\isacharcomma}{\kern0pt}\ simp{\isacharcomma}{\kern0pt}\ simp{\isacharparenright}{\kern0pt}\isanewline
\ \ \ \ \ \ \ \ \isacommand{apply}\isamarkupfalse%
\ {\isacharparenleft}{\kern0pt}rule\ add{\isacharunderscore}{\kern0pt}mono{\isacharparenright}{\kern0pt}\isanewline
\ \ \ \ \ \ \ \ \ \isacommand{apply}\isamarkupfalse%
\ {\isacharparenleft}{\kern0pt}metis\ dual{\isacharunderscore}{\kern0pt}order{\isachardot}{\kern0pt}refl\ log{\isadigit{2}}{\isacharunderscore}{\kern0pt}of{\isacharunderscore}{\kern0pt}power{\isacharunderscore}{\kern0pt}eq\ mult{\isacharunderscore}{\kern0pt}{\isadigit{2}}\ numeral{\isacharunderscore}{\kern0pt}Bit{\isadigit{0}}\ of{\isacharunderscore}{\kern0pt}nat{\isacharunderscore}{\kern0pt}numeral\ power{\isadigit{2}}{\isacharunderscore}{\kern0pt}eq{\isacharunderscore}{\kern0pt}square{\isacharparenright}{\kern0pt}\isanewline
\ \ \ \ \ \ \ \ \isacommand{apply}\isamarkupfalse%
\ {\isacharparenleft}{\kern0pt}subst\ log{\isacharunderscore}{\kern0pt}le{\isacharunderscore}{\kern0pt}cancel{\isacharunderscore}{\kern0pt}iff{\isacharcomma}{\kern0pt}\ simp{\isacharcomma}{\kern0pt}\ simp\ add{\isacharcolon}{\kern0pt}p{\isacharunderscore}{\kern0pt}ge{\isacharunderscore}{\kern0pt}{\isadigit{0}}{\isacharcomma}{\kern0pt}\ simp{\isacharparenright}{\kern0pt}\isanewline
\ \ \ \ \ \ \ \ \isacommand{using}\isamarkupfalse%
\ p{\isacharunderscore}{\kern0pt}le{\isacharunderscore}{\kern0pt}n\ \isacommand{by}\isamarkupfalse%
\ simp\isanewline
\ \ \ \ \ \ \isacommand{also}\isamarkupfalse%
\ \isacommand{have}\isamarkupfalse%
\ {\isachardoublequoteopen}{\isachardot}{\kern0pt}{\isachardot}{\kern0pt}{\isachardot}{\kern0pt}\ {\isasymle}\ ereal\ {\isacharparenleft}{\kern0pt}{\isadigit{8}}\ {\isacharplus}{\kern0pt}\ {\isadigit{4}}\ {\isacharasterisk}{\kern0pt}\ real\ r\ {\isacharplus}{\kern0pt}\ {\isadigit{2}}\ {\isacharasterisk}{\kern0pt}\ log\ {\isadigit{2}}\ {\isacharparenleft}{\kern0pt}log\ {\isadigit{2}}\ {\isacharparenleft}{\kern0pt}{\isacharparenleft}{\kern0pt}n{\isacharplus}{\kern0pt}{\isadigit{9}}{\isacharparenright}{\kern0pt}\ powr\ {\isadigit{2}}{\isacharparenright}{\kern0pt}{\isacharparenright}{\kern0pt}{\isacharparenright}{\kern0pt}{\isachardoublequoteclose}\isanewline
\ \ \ \ \ \ \ \ \isacommand{apply}\isamarkupfalse%
\ simp\isanewline
\ \ \ \ \ \ \ \ \isacommand{apply}\isamarkupfalse%
\ {\isacharparenleft}{\kern0pt}subst\ log{\isacharunderscore}{\kern0pt}le{\isacharunderscore}{\kern0pt}cancel{\isacharunderscore}{\kern0pt}iff{\isacharcomma}{\kern0pt}\ simp{\isacharcomma}{\kern0pt}\ rule\ add{\isacharunderscore}{\kern0pt}pos{\isacharunderscore}{\kern0pt}pos{\isacharcomma}{\kern0pt}\ simp{\isacharcomma}{\kern0pt}\ simp{\isacharcomma}{\kern0pt}\ simp{\isacharparenright}{\kern0pt}\isanewline
\ \ \ \ \ \ \ \ \isacommand{apply}\isamarkupfalse%
\ {\isacharparenleft}{\kern0pt}subst\ log{\isacharunderscore}{\kern0pt}mult{\isacharbrackleft}{\kern0pt}symmetric{\isacharbrackright}{\kern0pt}{\isacharcomma}{\kern0pt}\ simp{\isacharcomma}{\kern0pt}\ simp{\isacharcomma}{\kern0pt}\ simp{\isacharcomma}{\kern0pt}\ simp{\isacharparenright}{\kern0pt}\isanewline
\ \ \ \ \ \ \ \ \isacommand{by}\isamarkupfalse%
\ {\isacharparenleft}{\kern0pt}subst\ log{\isacharunderscore}{\kern0pt}le{\isacharunderscore}{\kern0pt}cancel{\isacharunderscore}{\kern0pt}iff{\isacharcomma}{\kern0pt}\ simp{\isacharcomma}{\kern0pt}\ simp{\isacharcomma}{\kern0pt}\ simp{\isacharcomma}{\kern0pt}\ simp\ add{\isacharcolon}{\kern0pt}power{\isadigit{2}}{\isacharunderscore}{\kern0pt}eq{\isacharunderscore}{\kern0pt}square\ algebra{\isacharunderscore}{\kern0pt}simps{\isacharparenright}{\kern0pt}\isanewline
\ \ \ \ \ \ \isacommand{also}\isamarkupfalse%
\ \isacommand{have}\isamarkupfalse%
\ {\isachardoublequoteopen}{\isachardot}{\kern0pt}{\isachardot}{\kern0pt}{\isachardot}{\kern0pt}\ {\isacharequal}{\kern0pt}\ ereal\ {\isacharparenleft}{\kern0pt}{\isadigit{1}}{\isadigit{0}}\ {\isacharplus}{\kern0pt}\ \ {\isadigit{4}}\ {\isacharasterisk}{\kern0pt}\ real\ r\ {\isacharplus}{\kern0pt}\ {\isadigit{2}}\ {\isacharasterisk}{\kern0pt}\ log\ {\isadigit{2}}\ {\isacharparenleft}{\kern0pt}log\ {\isadigit{2}}\ {\isacharparenleft}{\kern0pt}n\ {\isacharplus}{\kern0pt}\ {\isadigit{9}}{\isacharparenright}{\kern0pt}{\isacharparenright}{\kern0pt}{\isacharparenright}{\kern0pt}{\isachardoublequoteclose}\isanewline
\ \ \ \ \ \ \ \ \isacommand{apply}\isamarkupfalse%
\ {\isacharparenleft}{\kern0pt}subst\ log{\isacharunderscore}{\kern0pt}powr{\isacharcomma}{\kern0pt}\ simp{\isacharparenright}{\kern0pt}\isanewline
\ \ \ \ \ \ \ \ \isacommand{apply}\isamarkupfalse%
\ {\isacharparenleft}{\kern0pt}simp{\isacharparenright}{\kern0pt}\isanewline
\ \ \ \ \ \ \ \ \isacommand{apply}\isamarkupfalse%
\ {\isacharparenleft}{\kern0pt}subst\ {\isacharparenleft}{\kern0pt}{\isadigit{3}}{\isacharparenright}{\kern0pt}\ log{\isacharunderscore}{\kern0pt}{\isadigit{2}}{\isacharunderscore}{\kern0pt}{\isadigit{4}}{\isacharbrackleft}{\kern0pt}symmetric{\isacharbrackright}{\kern0pt}{\isacharparenright}{\kern0pt}\ \isanewline
\ \ \ \ \ \ \ \ \isacommand{by}\isamarkupfalse%
\ {\isacharparenleft}{\kern0pt}subst\ log{\isacharunderscore}{\kern0pt}mult{\isacharcomma}{\kern0pt}\ simp{\isacharcomma}{\kern0pt}\ simp{\isacharcomma}{\kern0pt}\ simp{\isacharcomma}{\kern0pt}\ simp{\isacharcomma}{\kern0pt}\ simp\ add{\isacharcolon}{\kern0pt}log{\isacharunderscore}{\kern0pt}{\isadigit{2}}{\isacharunderscore}{\kern0pt}{\isadigit{4}}{\isacharparenright}{\kern0pt}\isanewline
\ \ \ \ \ \ \isacommand{finally}\isamarkupfalse%
\ \isacommand{show}\isamarkupfalse%
\ {\isacharquery}{\kern0pt}thesis\ \isacommand{by}\isamarkupfalse%
\ simp\isanewline
\ \ \ \ \isacommand{next}\isamarkupfalse%
\isanewline
\ \ \ \ \ \ \isacommand{case}\isamarkupfalse%
\ False\isanewline
\ \ \ \ \ \ \isacommand{hence}\isamarkupfalse%
\ {\isachardoublequoteopen}y\ {\isacharequal}{\kern0pt}\ {\isadigit{0}}{\isachardoublequoteclose}\ \isacommand{using}\isamarkupfalse%
\ a\ \isacommand{by}\isamarkupfalse%
\ simp\isanewline
\ \ \ \ \ \ \isacommand{then}\isamarkupfalse%
\ \isacommand{show}\isamarkupfalse%
\ {\isacharquery}{\kern0pt}thesis\ \isacommand{by}\isamarkupfalse%
\ {\isacharparenleft}{\kern0pt}simp\ add{\isacharcolon}{\kern0pt}float{\isacharunderscore}{\kern0pt}bit{\isacharunderscore}{\kern0pt}count{\isacharunderscore}{\kern0pt}zero{\isacharparenright}{\kern0pt}\isanewline
\ \ \ \ \isacommand{qed}\isamarkupfalse%
\isanewline
\ \ \isacommand{qed}\isamarkupfalse%
\isanewline
\isanewline
\ \ \isacommand{have}\isamarkupfalse%
\ b{\isacharcolon}{\kern0pt}\ \isanewline
\ \ \ \ {\isachardoublequoteopen}{\isasymAnd}x{\isachardot}{\kern0pt}\ x\ {\isasymin}\ {\isacharparenleft}{\kern0pt}{\isacharbraceleft}{\kern0pt}{\isadigit{0}}{\isachardot}{\kern0pt}{\isachardot}{\kern0pt}{\isacharless}{\kern0pt}s{\isacharbraceright}{\kern0pt}\ {\isasymrightarrow}\isactrlsub E\ bounded{\isacharunderscore}{\kern0pt}degree{\isacharunderscore}{\kern0pt}polynomials\ {\isacharparenleft}{\kern0pt}ZFact\ {\isacharparenleft}{\kern0pt}int\ p{\isacharparenright}{\kern0pt}{\isacharparenright}{\kern0pt}\ {\isadigit{2}}{\isacharparenright}{\kern0pt}\ {\isasymLongrightarrow}\isanewline
\ \ \ \ \ \ \ \ bit{\isacharunderscore}{\kern0pt}count\ {\isacharparenleft}{\kern0pt}encode{\isacharunderscore}{\kern0pt}state\ {\isacharparenleft}{\kern0pt}s{\isacharcomma}{\kern0pt}\ t{\isacharcomma}{\kern0pt}\ p{\isacharcomma}{\kern0pt}\ r{\isacharcomma}{\kern0pt}\ x{\isacharcomma}{\kern0pt}\ {\isasymlambda}i{\isasymin}{\isacharbraceleft}{\kern0pt}{\isadigit{0}}{\isachardot}{\kern0pt}{\isachardot}{\kern0pt}{\isacharless}{\kern0pt}s{\isacharbraceright}{\kern0pt}{\isachardot}{\kern0pt}\ f{\isadigit{0}}{\isacharunderscore}{\kern0pt}sketch\ p\ r\ t\ {\isacharparenleft}{\kern0pt}x\ i{\isacharparenright}{\kern0pt}\ as{\isacharparenright}{\kern0pt}{\isacharparenright}{\kern0pt}\ {\isasymle}\ \isanewline
\ \ \ \ \ \ \ \ f{\isadigit{0}}{\isacharunderscore}{\kern0pt}space{\isacharunderscore}{\kern0pt}usage\ {\isacharparenleft}{\kern0pt}n{\isacharcomma}{\kern0pt}\ {\isasymepsilon}{\isacharcomma}{\kern0pt}\ {\isasymdelta}{\isacharparenright}{\kern0pt}{\isachardoublequoteclose}\isanewline
\ \ \isacommand{proof}\isamarkupfalse%
\ {\isacharminus}{\kern0pt}\isanewline
\ \ \ \ \isacommand{fix}\isamarkupfalse%
\ x\isanewline
\ \ \ \ \isacommand{assume}\isamarkupfalse%
\ b{\isacharunderscore}{\kern0pt}{\isadigit{1}}{\isacharcolon}{\kern0pt}{\isachardoublequoteopen}x\ {\isasymin}\ {\isacharbraceleft}{\kern0pt}{\isadigit{0}}{\isachardot}{\kern0pt}{\isachardot}{\kern0pt}{\isacharless}{\kern0pt}s{\isacharbraceright}{\kern0pt}\ {\isasymrightarrow}\isactrlsub E\ bounded{\isacharunderscore}{\kern0pt}degree{\isacharunderscore}{\kern0pt}polynomials\ {\isacharparenleft}{\kern0pt}ZFact\ {\isacharparenleft}{\kern0pt}int\ p{\isacharparenright}{\kern0pt}{\isacharparenright}{\kern0pt}\ {\isadigit{2}}{\isachardoublequoteclose}\isanewline
\ \ \ \ \isacommand{have}\isamarkupfalse%
\ b{\isacharunderscore}{\kern0pt}{\isadigit{2}}{\isacharcolon}{\kern0pt}\ {\isachardoublequoteopen}x\ {\isasymin}\ extensional\ {\isacharbraceleft}{\kern0pt}{\isadigit{0}}{\isachardot}{\kern0pt}{\isachardot}{\kern0pt}{\isacharless}{\kern0pt}s{\isacharbraceright}{\kern0pt}{\isachardoublequoteclose}\ \isacommand{using}\isamarkupfalse%
\ b{\isacharunderscore}{\kern0pt}{\isadigit{1}}\ \isacommand{by}\isamarkupfalse%
\ {\isacharparenleft}{\kern0pt}simp\ add{\isacharcolon}{\kern0pt}PiE{\isacharunderscore}{\kern0pt}def{\isacharparenright}{\kern0pt}\ \isanewline
\isanewline
\ \ \ \ \isacommand{have}\isamarkupfalse%
\ {\isachardoublequoteopen}{\isasymAnd}y{\isachardot}{\kern0pt}\ y\ {\isasymin}\ {\isacharbraceleft}{\kern0pt}{\isadigit{0}}{\isachardot}{\kern0pt}{\isachardot}{\kern0pt}{\isacharless}{\kern0pt}s{\isacharbraceright}{\kern0pt}\ {\isasymLongrightarrow}\ card\ {\isacharparenleft}{\kern0pt}f{\isadigit{0}}{\isacharunderscore}{\kern0pt}sketch\ p\ r\ t\ {\isacharparenleft}{\kern0pt}x\ y{\isacharparenright}{\kern0pt}\ as{\isacharparenright}{\kern0pt}\ {\isasymle}\ t\ {\isachardoublequoteclose}\isanewline
\ \ \ \ \ \ \isacommand{apply}\isamarkupfalse%
\ {\isacharparenleft}{\kern0pt}simp\ add{\isacharcolon}{\kern0pt}f{\isadigit{0}}{\isacharunderscore}{\kern0pt}sketch{\isacharunderscore}{\kern0pt}def{\isacharparenright}{\kern0pt}\isanewline
\ \ \ \ \ \ \isacommand{apply}\isamarkupfalse%
\ {\isacharparenleft}{\kern0pt}subst\ card{\isacharunderscore}{\kern0pt}least{\isacharcomma}{\kern0pt}\ simp{\isacharparenright}{\kern0pt}\isanewline
\ \ \ \ \ \ \isacommand{by}\isamarkupfalse%
\ simp\isanewline
\isanewline
\ \ \ \ \isacommand{hence}\isamarkupfalse%
\ b{\isacharunderscore}{\kern0pt}{\isadigit{3}}{\isacharcolon}{\kern0pt}\ {\isachardoublequoteopen}{\isasymAnd}y{\isachardot}{\kern0pt}\ y\ {\isasymin}\ {\isacharparenleft}{\kern0pt}{\isasymlambda}z{\isachardot}{\kern0pt}\ f{\isadigit{0}}{\isacharunderscore}{\kern0pt}sketch\ p\ r\ t\ {\isacharparenleft}{\kern0pt}x\ z{\isacharparenright}{\kern0pt}\ as{\isacharparenright}{\kern0pt}\ {\isacharbackquote}{\kern0pt}\ {\isacharbraceleft}{\kern0pt}{\isadigit{0}}{\isachardot}{\kern0pt}{\isachardot}{\kern0pt}{\isacharless}{\kern0pt}s{\isacharbraceright}{\kern0pt}\ {\isasymLongrightarrow}\ card\ y\ {\isasymle}\ t{\isachardoublequoteclose}\isanewline
\ \ \ \ \ \ \isacommand{by}\isamarkupfalse%
\ force\isanewline
\isanewline
\ \ \ \ \isacommand{have}\isamarkupfalse%
\ {\isachardoublequoteopen}{\isasymAnd}y{\isachardot}{\kern0pt}\ y\ {\isasymin}\ {\isacharbraceleft}{\kern0pt}{\isadigit{0}}{\isachardot}{\kern0pt}{\isachardot}{\kern0pt}{\isacharless}{\kern0pt}s{\isacharbraceright}{\kern0pt}\ {\isasymLongrightarrow}\ f{\isadigit{0}}{\isacharunderscore}{\kern0pt}sketch\ p\ r\ t\ {\isacharparenleft}{\kern0pt}x\ y{\isacharparenright}{\kern0pt}\ as\ {\isasymsubseteq}\ {\isacharparenleft}{\kern0pt}{\isasymlambda}k{\isachardot}{\kern0pt}\ float{\isacharunderscore}{\kern0pt}of\ {\isacharparenleft}{\kern0pt}truncate{\isacharunderscore}{\kern0pt}down\ r\ k{\isacharparenright}{\kern0pt}{\isacharparenright}{\kern0pt}\ {\isacharbackquote}{\kern0pt}\ {\isacharbraceleft}{\kern0pt}{\isadigit{0}}{\isachardot}{\kern0pt}{\isachardot}{\kern0pt}{\isacharless}{\kern0pt}p{\isacharbraceright}{\kern0pt}\ {\isachardoublequoteclose}\isanewline
\ \ \ \ \ \ \isacommand{apply}\isamarkupfalse%
\ {\isacharparenleft}{\kern0pt}simp\ add{\isacharcolon}{\kern0pt}f{\isadigit{0}}{\isacharunderscore}{\kern0pt}sketch{\isacharunderscore}{\kern0pt}def{\isacharparenright}{\kern0pt}\isanewline
\ \ \ \ \ \ \isacommand{apply}\isamarkupfalse%
\ {\isacharparenleft}{\kern0pt}rule\ order{\isacharunderscore}{\kern0pt}trans{\isacharbrackleft}{\kern0pt}OF\ least{\isacharunderscore}{\kern0pt}subset{\isacharbrackright}{\kern0pt}{\isacharparenright}{\kern0pt}\isanewline
\ \ \ \ \ \ \isacommand{apply}\isamarkupfalse%
\ {\isacharparenleft}{\kern0pt}rule\ f{\isacharunderscore}{\kern0pt}subset{\isacharbrackleft}{\kern0pt}\isakeyword{where}\ f{\isacharequal}{\kern0pt}{\isachardoublequoteopen}{\isasymlambda}x{\isachardot}{\kern0pt}\ float{\isacharunderscore}{\kern0pt}of\ {\isacharparenleft}{\kern0pt}truncate{\isacharunderscore}{\kern0pt}down\ r\ {\isacharparenleft}{\kern0pt}real\ x{\isacharparenright}{\kern0pt}{\isacharparenright}{\kern0pt}{\isachardoublequoteclose}{\isacharbrackright}{\kern0pt}{\isacharparenright}{\kern0pt}\isanewline
\ \ \ \ \ \ \isacommand{apply}\isamarkupfalse%
\ {\isacharparenleft}{\kern0pt}rule\ image{\isacharunderscore}{\kern0pt}subsetI{\isacharcomma}{\kern0pt}\ simp{\isacharparenright}{\kern0pt}\isanewline
\ \ \ \ \ \ \isacommand{apply}\isamarkupfalse%
\ {\isacharparenleft}{\kern0pt}rule\ hash{\isacharunderscore}{\kern0pt}range{\isacharbrackleft}{\kern0pt}OF\ p{\isacharunderscore}{\kern0pt}ge{\isacharunderscore}{\kern0pt}{\isadigit{0}}{\isacharcomma}{\kern0pt}\ \isakeyword{where}\ n{\isacharequal}{\kern0pt}{\isachardoublequoteopen}{\isadigit{2}}{\isachardoublequoteclose}{\isacharbrackright}{\kern0pt}{\isacharparenright}{\kern0pt}\isanewline
\ \ \ \ \ \ \ \isacommand{using}\isamarkupfalse%
\ b{\isacharunderscore}{\kern0pt}{\isadigit{1}}\ \isacommand{apply}\isamarkupfalse%
\ {\isacharparenleft}{\kern0pt}simp\ add{\isacharcolon}{\kern0pt}\ PiE{\isacharunderscore}{\kern0pt}iff{\isacharparenright}{\kern0pt}\isanewline
\ \ \ \ \ \ \isacommand{by}\isamarkupfalse%
\ {\isacharparenleft}{\kern0pt}metis\ assms{\isacharparenleft}{\kern0pt}{\isadigit{3}}{\isacharparenright}{\kern0pt}\ n{\isacharunderscore}{\kern0pt}le{\isacharunderscore}{\kern0pt}p\ order{\isacharunderscore}{\kern0pt}less{\isacharunderscore}{\kern0pt}le{\isacharunderscore}{\kern0pt}trans\ atLeastLessThan{\isacharunderscore}{\kern0pt}iff\ subset{\isacharunderscore}{\kern0pt}eq{\isacharparenright}{\kern0pt}\isanewline
\ \ \ \ \isacommand{hence}\isamarkupfalse%
\ b{\isacharunderscore}{\kern0pt}{\isadigit{4}}{\isacharcolon}{\kern0pt}\ {\isachardoublequoteopen}{\isasymAnd}y{\isachardot}{\kern0pt}\ y\ {\isasymin}\ {\isacharparenleft}{\kern0pt}{\isasymlambda}z{\isachardot}{\kern0pt}\ f{\isadigit{0}}{\isacharunderscore}{\kern0pt}sketch\ p\ r\ t\ {\isacharparenleft}{\kern0pt}x\ z{\isacharparenright}{\kern0pt}\ as{\isacharparenright}{\kern0pt}\ {\isacharbackquote}{\kern0pt}\ {\isacharbraceleft}{\kern0pt}{\isadigit{0}}{\isachardot}{\kern0pt}{\isachardot}{\kern0pt}{\isacharless}{\kern0pt}s{\isacharbraceright}{\kern0pt}\ {\isasymLongrightarrow}\ \isanewline
\ \ \ \ \ \ y\ {\isasymsubseteq}\ {\isacharparenleft}{\kern0pt}{\isasymlambda}k{\isachardot}{\kern0pt}\ float{\isacharunderscore}{\kern0pt}of\ {\isacharparenleft}{\kern0pt}truncate{\isacharunderscore}{\kern0pt}down\ r\ k{\isacharparenright}{\kern0pt}{\isacharparenright}{\kern0pt}\ {\isacharbackquote}{\kern0pt}\ {\isacharbraceleft}{\kern0pt}{\isadigit{0}}{\isachardot}{\kern0pt}{\isachardot}{\kern0pt}{\isacharless}{\kern0pt}p{\isacharbraceright}{\kern0pt}{\isachardoublequoteclose}\isanewline
\ \ \ \ \ \ \isacommand{by}\isamarkupfalse%
\ force\isanewline
\isanewline
\ \ \ \ \isacommand{have}\isamarkupfalse%
\ b{\isacharunderscore}{\kern0pt}{\isadigit{4}}{\isacharunderscore}{\kern0pt}{\isadigit{1}}{\isacharcolon}{\kern0pt}\ {\isachardoublequoteopen}{\isasymAnd}y\ z\ {\isachardot}{\kern0pt}\ y\ {\isasymin}\ {\isacharparenleft}{\kern0pt}{\isasymlambda}z{\isachardot}{\kern0pt}\ f{\isadigit{0}}{\isacharunderscore}{\kern0pt}sketch\ p\ r\ t\ {\isacharparenleft}{\kern0pt}x\ z{\isacharparenright}{\kern0pt}\ as{\isacharparenright}{\kern0pt}\ {\isacharbackquote}{\kern0pt}\ {\isacharbraceleft}{\kern0pt}{\isadigit{0}}{\isachardot}{\kern0pt}{\isachardot}{\kern0pt}{\isacharless}{\kern0pt}s{\isacharbraceright}{\kern0pt}\ {\isasymLongrightarrow}\ z\ {\isasymin}\ y\ {\isasymLongrightarrow}\ \isanewline
\ \ \ \ \ \ bit{\isacharunderscore}{\kern0pt}count\ {\isacharparenleft}{\kern0pt}F\isactrlsub S\ z{\isacharparenright}{\kern0pt}\ {\isasymle}\ ereal\ {\isacharparenleft}{\kern0pt}{\isadigit{1}}{\isadigit{0}}\ {\isacharplus}{\kern0pt}\ {\isadigit{4}}\ {\isacharasterisk}{\kern0pt}\ real\ r\ {\isacharplus}{\kern0pt}\ {\isadigit{2}}\ {\isacharasterisk}{\kern0pt}\ log\ {\isadigit{2}}\ {\isacharparenleft}{\kern0pt}log\ {\isadigit{2}}\ {\isacharparenleft}{\kern0pt}n{\isacharplus}{\kern0pt}{\isadigit{9}}{\isacharparenright}{\kern0pt}{\isacharparenright}{\kern0pt}{\isacharparenright}{\kern0pt}{\isachardoublequoteclose}\isanewline
\ \ \ \ \ \ \isacommand{using}\isamarkupfalse%
\ b{\isacharunderscore}{\kern0pt}{\isadigit{4}}{\isacharunderscore}{\kern0pt}{\isadigit{2}}{\isadigit{2}}\ b{\isacharunderscore}{\kern0pt}{\isadigit{4}}\ \isacommand{by}\isamarkupfalse%
\ blast\isanewline
\isanewline
\ \ \ \ \isacommand{have}\isamarkupfalse%
\ {\isachardoublequoteopen}{\isasymAnd}y{\isachardot}{\kern0pt}\ y\ {\isasymin}\ {\isacharbraceleft}{\kern0pt}{\isadigit{0}}{\isachardot}{\kern0pt}{\isachardot}{\kern0pt}{\isacharless}{\kern0pt}s{\isacharbraceright}{\kern0pt}\ {\isasymLongrightarrow}\ finite\ {\isacharparenleft}{\kern0pt}f{\isadigit{0}}{\isacharunderscore}{\kern0pt}sketch\ p\ r\ t\ {\isacharparenleft}{\kern0pt}x\ y{\isacharparenright}{\kern0pt}\ as{\isacharparenright}{\kern0pt}{\isachardoublequoteclose}\isanewline
\ \ \ \ \ \ \isacommand{apply}\isamarkupfalse%
\ {\isacharparenleft}{\kern0pt}simp\ add{\isacharcolon}{\kern0pt}f{\isadigit{0}}{\isacharunderscore}{\kern0pt}sketch{\isacharunderscore}{\kern0pt}def{\isacharparenright}{\kern0pt}\isanewline
\ \ \ \ \ \ \isacommand{by}\isamarkupfalse%
\ {\isacharparenleft}{\kern0pt}rule\ finite{\isacharunderscore}{\kern0pt}subset{\isacharbrackleft}{\kern0pt}OF\ least{\isacharunderscore}{\kern0pt}subset{\isacharbrackright}{\kern0pt}{\isacharcomma}{\kern0pt}\ simp{\isacharparenright}{\kern0pt}\isanewline
\ \ \ \ \isacommand{hence}\isamarkupfalse%
\ b{\isacharunderscore}{\kern0pt}{\isadigit{5}}{\isacharcolon}{\kern0pt}\ {\isachardoublequoteopen}{\isasymAnd}y{\isachardot}{\kern0pt}\ y\ {\isasymin}\ {\isacharparenleft}{\kern0pt}{\isasymlambda}z{\isachardot}{\kern0pt}\ f{\isadigit{0}}{\isacharunderscore}{\kern0pt}sketch\ p\ r\ t\ {\isacharparenleft}{\kern0pt}x\ z{\isacharparenright}{\kern0pt}\ as{\isacharparenright}{\kern0pt}\ {\isacharbackquote}{\kern0pt}\ {\isacharbraceleft}{\kern0pt}{\isadigit{0}}{\isachardot}{\kern0pt}{\isachardot}{\kern0pt}{\isacharless}{\kern0pt}s{\isacharbraceright}{\kern0pt}\ {\isasymLongrightarrow}\ finite\ y{\isachardoublequoteclose}\ \isacommand{by}\isamarkupfalse%
\ force\isanewline
\isanewline
\ \ \ \ \isacommand{have}\isamarkupfalse%
\ {\isachardoublequoteopen}bit{\isacharunderscore}{\kern0pt}count\ {\isacharparenleft}{\kern0pt}encode{\isacharunderscore}{\kern0pt}state\ {\isacharparenleft}{\kern0pt}s{\isacharcomma}{\kern0pt}\ t{\isacharcomma}{\kern0pt}\ p{\isacharcomma}{\kern0pt}\ r{\isacharcomma}{\kern0pt}\ x{\isacharcomma}{\kern0pt}\ {\isasymlambda}i{\isasymin}{\isacharbraceleft}{\kern0pt}{\isadigit{0}}{\isachardot}{\kern0pt}{\isachardot}{\kern0pt}{\isacharless}{\kern0pt}s{\isacharbraceright}{\kern0pt}{\isachardot}{\kern0pt}\ f{\isadigit{0}}{\isacharunderscore}{\kern0pt}sketch\ p\ r\ t\ {\isacharparenleft}{\kern0pt}x\ i{\isacharparenright}{\kern0pt}\ as{\isacharparenright}{\kern0pt}{\isacharparenright}{\kern0pt}\ {\isacharequal}{\kern0pt}\isanewline
\ \ \ \ \ \ bit{\isacharunderscore}{\kern0pt}count\ {\isacharparenleft}{\kern0pt}N\isactrlsub S\ s{\isacharparenright}{\kern0pt}\ {\isacharplus}{\kern0pt}\ bit{\isacharunderscore}{\kern0pt}count\ {\isacharparenleft}{\kern0pt}N\isactrlsub S\ t{\isacharparenright}{\kern0pt}\ {\isacharplus}{\kern0pt}\ \ bit{\isacharunderscore}{\kern0pt}count\ {\isacharparenleft}{\kern0pt}N\isactrlsub S\ p{\isacharparenright}{\kern0pt}\ {\isacharplus}{\kern0pt}\ bit{\isacharunderscore}{\kern0pt}count\ {\isacharparenleft}{\kern0pt}N\isactrlsub S\ r{\isacharparenright}{\kern0pt}\ {\isacharplus}{\kern0pt}\isanewline
\ \ \ \ \ \ bit{\isacharunderscore}{\kern0pt}count\ {\isacharparenleft}{\kern0pt}list\isactrlsub S\ {\isacharparenleft}{\kern0pt}list\isactrlsub S\ {\isacharparenleft}{\kern0pt}zfact\isactrlsub S\ p{\isacharparenright}{\kern0pt}{\isacharparenright}{\kern0pt}\ {\isacharparenleft}{\kern0pt}map\ x\ {\isacharbrackleft}{\kern0pt}{\isadigit{0}}{\isachardot}{\kern0pt}{\isachardot}{\kern0pt}{\isacharless}{\kern0pt}s{\isacharbrackright}{\kern0pt}{\isacharparenright}{\kern0pt}{\isacharparenright}{\kern0pt}\ {\isacharplus}{\kern0pt}\isanewline
\ \ \ \ \ \ bit{\isacharunderscore}{\kern0pt}count\ {\isacharparenleft}{\kern0pt}list\isactrlsub S\ {\isacharparenleft}{\kern0pt}set\isactrlsub S\ F\isactrlsub S{\isacharparenright}{\kern0pt}\ {\isacharparenleft}{\kern0pt}map\ {\isacharparenleft}{\kern0pt}{\isasymlambda}i{\isasymin}{\isacharbraceleft}{\kern0pt}{\isadigit{0}}{\isachardot}{\kern0pt}{\isachardot}{\kern0pt}{\isacharless}{\kern0pt}s{\isacharbraceright}{\kern0pt}{\isachardot}{\kern0pt}\ f{\isadigit{0}}{\isacharunderscore}{\kern0pt}sketch\ p\ r\ t\ {\isacharparenleft}{\kern0pt}x\ i{\isacharparenright}{\kern0pt}\ as{\isacharparenright}{\kern0pt}\ {\isacharbrackleft}{\kern0pt}{\isadigit{0}}{\isachardot}{\kern0pt}{\isachardot}{\kern0pt}{\isacharless}{\kern0pt}s{\isacharbrackright}{\kern0pt}{\isacharparenright}{\kern0pt}{\isacharparenright}{\kern0pt}{\isachardoublequoteclose}\isanewline
\ \ \ \ \ \ \isacommand{apply}\isamarkupfalse%
\ {\isacharparenleft}{\kern0pt}simp\ add{\isacharcolon}{\kern0pt}b{\isacharunderscore}{\kern0pt}{\isadigit{2}}\ encode{\isacharunderscore}{\kern0pt}state{\isacharunderscore}{\kern0pt}def\ dependent{\isacharunderscore}{\kern0pt}bit{\isacharunderscore}{\kern0pt}count\ prod{\isacharunderscore}{\kern0pt}bit{\isacharunderscore}{\kern0pt}count\isanewline
\ \ \ \ \ \ \ \ s{\isacharunderscore}{\kern0pt}def{\isacharbrackleft}{\kern0pt}symmetric{\isacharbrackright}{\kern0pt}\ t{\isacharunderscore}{\kern0pt}def{\isacharbrackleft}{\kern0pt}symmetric{\isacharbrackright}{\kern0pt}\ p{\isacharunderscore}{\kern0pt}def{\isacharbrackleft}{\kern0pt}symmetric{\isacharbrackright}{\kern0pt}\ r{\isacharunderscore}{\kern0pt}def{\isacharbrackleft}{\kern0pt}symmetric{\isacharbrackright}{\kern0pt}\ fun\isactrlsub S{\isacharunderscore}{\kern0pt}def\isanewline
\ \ \ \ \ \ \ \ del{\isacharcolon}{\kern0pt}N\isactrlsub S{\isachardot}{\kern0pt}simps\ encode{\isacharunderscore}{\kern0pt}prod{\isachardot}{\kern0pt}simps\ encode{\isacharunderscore}{\kern0pt}dependent{\isacharunderscore}{\kern0pt}sum{\isachardot}{\kern0pt}simps{\isacharparenright}{\kern0pt}\isanewline
\ \ \ \ \ \ \isacommand{by}\isamarkupfalse%
\ {\isacharparenleft}{\kern0pt}simp\ add{\isacharcolon}{\kern0pt}ac{\isacharunderscore}{\kern0pt}simps\ del{\isacharcolon}{\kern0pt}N\isactrlsub S{\isachardot}{\kern0pt}simps\ encode{\isacharunderscore}{\kern0pt}prod{\isachardot}{\kern0pt}simps\ encode{\isacharunderscore}{\kern0pt}dependent{\isacharunderscore}{\kern0pt}sum{\isachardot}{\kern0pt}simps{\isacharparenright}{\kern0pt}\isanewline
\ \ \ \ \isacommand{also}\isamarkupfalse%
\ \isacommand{have}\isamarkupfalse%
\ {\isachardoublequoteopen}{\isachardot}{\kern0pt}{\isachardot}{\kern0pt}{\isachardot}{\kern0pt}\ {\isasymle}\ ereal\ {\isacharparenleft}{\kern0pt}{\isadigit{2}}{\isacharasterisk}{\kern0pt}\ log\ {\isadigit{2}}\ {\isacharparenleft}{\kern0pt}real\ s\ {\isacharplus}{\kern0pt}\ {\isadigit{1}}{\isacharparenright}{\kern0pt}\ {\isacharplus}{\kern0pt}\ {\isadigit{1}}{\isacharparenright}{\kern0pt}\ {\isacharplus}{\kern0pt}\ ereal\ \ {\isacharparenleft}{\kern0pt}{\isadigit{2}}{\isacharasterisk}{\kern0pt}\ log\ {\isadigit{2}}\ {\isacharparenleft}{\kern0pt}real\ t\ {\isacharplus}{\kern0pt}\ {\isadigit{1}}{\isacharparenright}{\kern0pt}\ {\isacharplus}{\kern0pt}\ {\isadigit{1}}{\isacharparenright}{\kern0pt}\isanewline
\ \ \ \ \ \ {\isacharplus}{\kern0pt}\ ereal\ {\isacharparenleft}{\kern0pt}{\isadigit{2}}{\isacharasterisk}{\kern0pt}\ log\ {\isadigit{2}}\ {\isacharparenleft}{\kern0pt}real\ p\ {\isacharplus}{\kern0pt}\ {\isadigit{1}}{\isacharparenright}{\kern0pt}\ {\isacharplus}{\kern0pt}\ {\isadigit{1}}{\isacharparenright}{\kern0pt}\ {\isacharplus}{\kern0pt}\ ereal\ {\isacharparenleft}{\kern0pt}{\isadigit{2}}\ {\isacharasterisk}{\kern0pt}\ log\ {\isadigit{2}}\ {\isacharparenleft}{\kern0pt}real\ r\ {\isacharplus}{\kern0pt}\ {\isadigit{1}}{\isacharparenright}{\kern0pt}\ {\isacharplus}{\kern0pt}\ {\isadigit{1}}{\isacharparenright}{\kern0pt}\isanewline
\ \ \ \ \ \ {\isacharplus}{\kern0pt}\ {\isacharparenleft}{\kern0pt}ereal\ {\isacharparenleft}{\kern0pt}real\ s{\isacharparenright}{\kern0pt}\ {\isacharasterisk}{\kern0pt}\ {\isacharparenleft}{\kern0pt}ereal\ {\isacharparenleft}{\kern0pt}real\ {\isadigit{2}}\ {\isacharasterisk}{\kern0pt}\ {\isacharparenleft}{\kern0pt}{\isadigit{2}}\ {\isacharasterisk}{\kern0pt}\ log\ {\isadigit{2}}\ {\isacharparenleft}{\kern0pt}real\ p{\isacharparenright}{\kern0pt}\ {\isacharplus}{\kern0pt}\ {\isadigit{2}}{\isacharparenright}{\kern0pt}\ {\isacharplus}{\kern0pt}\ {\isadigit{1}}{\isacharparenright}{\kern0pt}\ {\isacharplus}{\kern0pt}\ {\isadigit{1}}{\isacharparenright}{\kern0pt}\ {\isacharplus}{\kern0pt}\ {\isadigit{1}}{\isacharparenright}{\kern0pt}\ \isanewline
\ \ \ \ \ \ {\isacharplus}{\kern0pt}\ {\isacharparenleft}{\kern0pt}ereal\ {\isacharparenleft}{\kern0pt}real\ s{\isacharparenright}{\kern0pt}\ {\isacharasterisk}{\kern0pt}\ {\isacharparenleft}{\kern0pt}{\isacharparenleft}{\kern0pt}ereal\ {\isacharparenleft}{\kern0pt}real\ t{\isacharparenright}{\kern0pt}\ {\isacharasterisk}{\kern0pt}\ {\isacharparenleft}{\kern0pt}ereal\ {\isacharparenleft}{\kern0pt}{\isadigit{1}}{\isadigit{0}}\ {\isacharplus}{\kern0pt}\ {\isadigit{4}}\ {\isacharasterisk}{\kern0pt}\ real\ r\ {\isacharplus}{\kern0pt}\ {\isadigit{2}}\ {\isacharasterisk}{\kern0pt}\ log\ {\isadigit{2}}\ {\isacharparenleft}{\kern0pt}log\ {\isadigit{2}}\ {\isacharparenleft}{\kern0pt}real\ {\isacharparenleft}{\kern0pt}n\ {\isacharplus}{\kern0pt}\ {\isadigit{9}}{\isacharparenright}{\kern0pt}{\isacharparenright}{\kern0pt}{\isacharparenright}{\kern0pt}{\isacharparenright}{\kern0pt}\isanewline
\ \ \ \ \ \ \ \ \ \ \ {\isacharplus}{\kern0pt}\ {\isadigit{1}}{\isacharparenright}{\kern0pt}\ {\isacharplus}{\kern0pt}\ {\isadigit{1}}{\isacharparenright}{\kern0pt}\ {\isacharplus}{\kern0pt}\ {\isadigit{1}}{\isacharparenright}{\kern0pt}\ {\isacharplus}{\kern0pt}\ {\isadigit{1}}{\isacharparenright}{\kern0pt}{\isachardoublequoteclose}\isanewline
\ \ \ \ \ \ \isacommand{apply}\isamarkupfalse%
\ {\isacharparenleft}{\kern0pt}rule\ add{\isacharunderscore}{\kern0pt}mono{\isacharcomma}{\kern0pt}\ rule\ add{\isacharunderscore}{\kern0pt}mono{\isacharcomma}{\kern0pt}\ rule\ add{\isacharunderscore}{\kern0pt}mono{\isacharcomma}{\kern0pt}\ rule\ add{\isacharunderscore}{\kern0pt}mono{\isacharcomma}{\kern0pt}\ rule\ add{\isacharunderscore}{\kern0pt}mono{\isacharparenright}{\kern0pt}\isanewline
\ \ \ \ \ \ \ \ \ \ \ \isacommand{apply}\isamarkupfalse%
\ {\isacharparenleft}{\kern0pt}metis\ nat{\isacharunderscore}{\kern0pt}bit{\isacharunderscore}{\kern0pt}count{\isacharparenright}{\kern0pt}\isanewline
\ \ \ \ \ \ \ \ \ \ \isacommand{apply}\isamarkupfalse%
\ {\isacharparenleft}{\kern0pt}metis\ nat{\isacharunderscore}{\kern0pt}bit{\isacharunderscore}{\kern0pt}count{\isacharparenright}{\kern0pt}\isanewline
\ \ \ \ \ \ \ \ \ \isacommand{apply}\isamarkupfalse%
\ {\isacharparenleft}{\kern0pt}metis\ nat{\isacharunderscore}{\kern0pt}bit{\isacharunderscore}{\kern0pt}count{\isacharparenright}{\kern0pt}\isanewline
\ \ \ \ \ \ \ \ \isacommand{apply}\isamarkupfalse%
\ {\isacharparenleft}{\kern0pt}metis\ nat{\isacharunderscore}{\kern0pt}bit{\isacharunderscore}{\kern0pt}count{\isacharparenright}{\kern0pt}\isanewline
\ \ \ \ \ \ \ \isacommand{apply}\isamarkupfalse%
\ {\isacharparenleft}{\kern0pt}rule\ list{\isacharunderscore}{\kern0pt}bit{\isacharunderscore}{\kern0pt}count{\isacharunderscore}{\kern0pt}est{\isacharbrackleft}{\kern0pt}\isakeyword{where}\ xs{\isacharequal}{\kern0pt}{\isachardoublequoteopen}map\ x\ {\isacharbrackleft}{\kern0pt}{\isadigit{0}}{\isachardot}{\kern0pt}{\isachardot}{\kern0pt}{\isacharless}{\kern0pt}s{\isacharbrackright}{\kern0pt}{\isachardoublequoteclose}{\isacharcomma}{\kern0pt}\ simplified{\isacharbrackright}{\kern0pt}{\isacharparenright}{\kern0pt}\isanewline
\ \ \ \ \ \ \ \isacommand{apply}\isamarkupfalse%
\ {\isacharparenleft}{\kern0pt}rule\ bounded{\isacharunderscore}{\kern0pt}degree{\isacharunderscore}{\kern0pt}polynomial{\isacharunderscore}{\kern0pt}bit{\isacharunderscore}{\kern0pt}count{\isacharbrackleft}{\kern0pt}OF\ p{\isacharunderscore}{\kern0pt}ge{\isacharunderscore}{\kern0pt}{\isadigit{0}}{\isacharbrackright}{\kern0pt}{\isacharparenright}{\kern0pt}\ \isacommand{using}\isamarkupfalse%
\ b{\isacharunderscore}{\kern0pt}{\isadigit{1}}\ \isacommand{apply}\isamarkupfalse%
\ blast\isanewline
\ \ \ \ \ \ \isacommand{apply}\isamarkupfalse%
\ {\isacharparenleft}{\kern0pt}rule\ list{\isacharunderscore}{\kern0pt}bit{\isacharunderscore}{\kern0pt}count{\isacharunderscore}{\kern0pt}est{\isacharbrackleft}{\kern0pt}\isakeyword{where}\ xs{\isacharequal}{\kern0pt}{\isachardoublequoteopen}map\ {\isacharparenleft}{\kern0pt}{\isasymlambda}i{\isasymin}{\isacharbraceleft}{\kern0pt}{\isadigit{0}}{\isachardot}{\kern0pt}{\isachardot}{\kern0pt}{\isacharless}{\kern0pt}s{\isacharbraceright}{\kern0pt}{\isachardot}{\kern0pt}\ f{\isadigit{0}}{\isacharunderscore}{\kern0pt}sketch\ p\ r\ t\ {\isacharparenleft}{\kern0pt}x\ i{\isacharparenright}{\kern0pt}\ as{\isacharparenright}{\kern0pt}\ {\isacharbrackleft}{\kern0pt}{\isadigit{0}}{\isachardot}{\kern0pt}{\isachardot}{\kern0pt}{\isacharless}{\kern0pt}s{\isacharbrackright}{\kern0pt}{\isachardoublequoteclose}{\isacharcomma}{\kern0pt}\ simplified{\isacharbrackright}{\kern0pt}{\isacharparenright}{\kern0pt}\isanewline
\ \ \ \ \ \ \isacommand{apply}\isamarkupfalse%
\ {\isacharparenleft}{\kern0pt}rule\ set{\isacharunderscore}{\kern0pt}bit{\isacharunderscore}{\kern0pt}count{\isacharunderscore}{\kern0pt}est{\isacharcomma}{\kern0pt}\ metis\ b{\isacharunderscore}{\kern0pt}{\isadigit{5}}{\isacharcomma}{\kern0pt}\ metis\ b{\isacharunderscore}{\kern0pt}{\isadigit{3}}{\isacharparenright}{\kern0pt}\isanewline
\ \ \ \ \ \ \isacommand{apply}\isamarkupfalse%
\ simp\ \isanewline
\ \ \ \ \ \ \isacommand{by}\isamarkupfalse%
\ {\isacharparenleft}{\kern0pt}metis\ b{\isacharunderscore}{\kern0pt}{\isadigit{4}}{\isacharunderscore}{\kern0pt}{\isadigit{1}}{\isacharparenright}{\kern0pt}\isanewline
\ \ \ \ \isacommand{also}\isamarkupfalse%
\ \isacommand{have}\isamarkupfalse%
\ {\isachardoublequoteopen}{\isachardot}{\kern0pt}{\isachardot}{\kern0pt}{\isachardot}{\kern0pt}\ {\isacharequal}{\kern0pt}\ ereal\ {\isacharparenleft}{\kern0pt}\ {\isadigit{6}}\ {\isacharplus}{\kern0pt}\ {\isadigit{2}}\ {\isacharasterisk}{\kern0pt}\ log\ {\isadigit{2}}\ {\isacharparenleft}{\kern0pt}real\ s\ {\isacharplus}{\kern0pt}\ {\isadigit{1}}{\isacharparenright}{\kern0pt}\ {\isacharplus}{\kern0pt}\ {\isadigit{2}}\ {\isacharasterisk}{\kern0pt}\ log\ {\isadigit{2}}\ {\isacharparenleft}{\kern0pt}real\ t\ {\isacharplus}{\kern0pt}\ {\isadigit{1}}{\isacharparenright}{\kern0pt}\ {\isacharplus}{\kern0pt}\ \isanewline
\ \ \ \ \ \ {\isadigit{2}}\ {\isacharasterisk}{\kern0pt}\ log\ {\isadigit{2}}\ {\isacharparenleft}{\kern0pt}real\ p\ {\isacharplus}{\kern0pt}\ {\isadigit{1}}{\isacharparenright}{\kern0pt}\ {\isacharplus}{\kern0pt}\ {\isadigit{2}}\ {\isacharasterisk}{\kern0pt}\ log\ {\isadigit{2}}\ {\isacharparenleft}{\kern0pt}real\ r\ {\isacharplus}{\kern0pt}\ {\isadigit{1}}{\isacharparenright}{\kern0pt}\ {\isacharplus}{\kern0pt}\ real\ s\ {\isacharasterisk}{\kern0pt}\ {\isacharparenleft}{\kern0pt}{\isadigit{8}}\ {\isacharplus}{\kern0pt}\ {\isadigit{4}}\ {\isacharasterisk}{\kern0pt}\ log\ {\isadigit{2}}\ {\isacharparenleft}{\kern0pt}real\ p{\isacharparenright}{\kern0pt}\ {\isacharplus}{\kern0pt}\ \isanewline
\ \ \ \ \ \ real\ t\ {\isacharasterisk}{\kern0pt}\ {\isacharparenleft}{\kern0pt}{\isadigit{1}}{\isadigit{1}}\ {\isacharplus}{\kern0pt}\ {\isacharparenleft}{\kern0pt}{\isadigit{4}}\ {\isacharasterisk}{\kern0pt}\ real\ r\ {\isacharplus}{\kern0pt}\ {\isadigit{2}}\ {\isacharasterisk}{\kern0pt}\ log\ {\isadigit{2}}\ {\isacharparenleft}{\kern0pt}log\ {\isadigit{2}}\ {\isacharparenleft}{\kern0pt}real\ n\ {\isacharplus}{\kern0pt}\ {\isadigit{9}}{\isacharparenright}{\kern0pt}{\isacharparenright}{\kern0pt}{\isacharparenright}{\kern0pt}{\isacharparenright}{\kern0pt}{\isacharparenright}{\kern0pt}{\isacharparenright}{\kern0pt}{\isachardoublequoteclose}\isanewline
\ \ \ \ \ \ \isacommand{apply}\isamarkupfalse%
\ {\isacharparenleft}{\kern0pt}simp{\isacharparenright}{\kern0pt}\isanewline
\ \ \ \ \ \ \isacommand{by}\isamarkupfalse%
\ {\isacharparenleft}{\kern0pt}subst\ distrib{\isacharunderscore}{\kern0pt}left{\isacharbrackleft}{\kern0pt}symmetric{\isacharbrackright}{\kern0pt}{\isacharcomma}{\kern0pt}\ simp{\isacharparenright}{\kern0pt}\ \isanewline
\ \ \ \ \isacommand{also}\isamarkupfalse%
\ \isacommand{have}\isamarkupfalse%
\ {\isachardoublequoteopen}{\isachardot}{\kern0pt}{\isachardot}{\kern0pt}{\isachardot}{\kern0pt}\ {\isasymle}\ ereal\ {\isacharparenleft}{\kern0pt}\ {\isadigit{6}}\ {\isacharplus}{\kern0pt}\ {\isadigit{2}}\ {\isacharasterisk}{\kern0pt}\ log\ {\isadigit{2}}\ {\isacharparenleft}{\kern0pt}real\ s\ {\isacharplus}{\kern0pt}\ {\isadigit{1}}{\isacharparenright}{\kern0pt}\ \ {\isacharplus}{\kern0pt}\ {\isadigit{2}}\ {\isacharasterisk}{\kern0pt}\ log\ {\isadigit{2}}\ {\isacharparenleft}{\kern0pt}real\ t\ {\isacharplus}{\kern0pt}\ {\isadigit{1}}{\isacharparenright}{\kern0pt}\ {\isacharplus}{\kern0pt}\ \isanewline
\ \ \ \ \ \ {\isadigit{2}}\ {\isacharasterisk}{\kern0pt}\ log\ {\isadigit{2}}\ {\isacharparenleft}{\kern0pt}{\isadigit{2}}\ {\isacharasterisk}{\kern0pt}\ {\isacharparenleft}{\kern0pt}{\isadigit{1}}{\isadigit{0}}\ {\isacharplus}{\kern0pt}\ real\ n{\isacharparenright}{\kern0pt}{\isacharparenright}{\kern0pt}\ {\isacharplus}{\kern0pt}\ {\isadigit{2}}\ {\isacharasterisk}{\kern0pt}\ log\ {\isadigit{2}}\ {\isacharparenleft}{\kern0pt}real\ r\ {\isacharplus}{\kern0pt}\ {\isadigit{1}}{\isacharparenright}{\kern0pt}\ {\isacharplus}{\kern0pt}\ real\ s\ {\isacharasterisk}{\kern0pt}\ {\isacharparenleft}{\kern0pt}{\isadigit{8}}\ {\isacharplus}{\kern0pt}\ {\isadigit{4}}\ {\isacharasterisk}{\kern0pt}\ log\ {\isadigit{2}}\ {\isacharparenleft}{\kern0pt}{\isadigit{2}}\ {\isacharasterisk}{\kern0pt}\ {\isacharparenleft}{\kern0pt}{\isadigit{1}}{\isadigit{0}}\ {\isacharplus}{\kern0pt}\ real\ n{\isacharparenright}{\kern0pt}{\isacharparenright}{\kern0pt}\ {\isacharplus}{\kern0pt}\ \isanewline
\ \ \ \ \ \ real\ t\ {\isacharasterisk}{\kern0pt}\ {\isacharparenleft}{\kern0pt}{\isadigit{1}}{\isadigit{1}}\ {\isacharplus}{\kern0pt}\ {\isacharparenleft}{\kern0pt}{\isadigit{4}}\ {\isacharasterisk}{\kern0pt}\ real\ r\ {\isacharplus}{\kern0pt}\ {\isadigit{2}}\ {\isacharasterisk}{\kern0pt}\ log\ {\isadigit{2}}\ {\isacharparenleft}{\kern0pt}log\ {\isadigit{2}}\ {\isacharparenleft}{\kern0pt}real\ n\ {\isacharplus}{\kern0pt}\ {\isadigit{9}}{\isacharparenright}{\kern0pt}{\isacharparenright}{\kern0pt}{\isacharparenright}{\kern0pt}{\isacharparenright}{\kern0pt}{\isacharparenright}{\kern0pt}{\isacharparenright}{\kern0pt}{\isachardoublequoteclose}\isanewline
\ \ \ \ \ \ \isacommand{apply}\isamarkupfalse%
\ {\isacharparenleft}{\kern0pt}simp{\isacharcomma}{\kern0pt}\ rule\ add{\isacharunderscore}{\kern0pt}mono{\isacharcomma}{\kern0pt}\ simp{\isacharparenright}{\kern0pt}\ \isacommand{using}\isamarkupfalse%
\ p{\isacharunderscore}{\kern0pt}le{\isacharunderscore}{\kern0pt}n\ \isacommand{apply}\isamarkupfalse%
\ simp\isanewline
\ \ \ \ \ \ \isacommand{apply}\isamarkupfalse%
\ {\isacharparenleft}{\kern0pt}rule\ mult{\isacharunderscore}{\kern0pt}left{\isacharunderscore}{\kern0pt}mono{\isacharcomma}{\kern0pt}\ simp{\isacharparenright}{\kern0pt}\isanewline
\ \ \ \ \ \ \ \isacommand{apply}\isamarkupfalse%
\ {\isacharparenleft}{\kern0pt}subst\ log{\isacharunderscore}{\kern0pt}le{\isacharunderscore}{\kern0pt}cancel{\isacharunderscore}{\kern0pt}iff{\isacharcomma}{\kern0pt}\ simp{\isacharcomma}{\kern0pt}\ simp\ add{\isacharcolon}{\kern0pt}p{\isacharunderscore}{\kern0pt}ge{\isacharunderscore}{\kern0pt}{\isadigit{0}}{\isacharcomma}{\kern0pt}\ simp{\isacharparenright}{\kern0pt}\isanewline
\ \ \ \ \ \ \ \isacommand{using}\isamarkupfalse%
\ p{\isacharunderscore}{\kern0pt}le{\isacharunderscore}{\kern0pt}n\ \isacommand{apply}\isamarkupfalse%
\ simp\isanewline
\ \ \ \ \ \ \isacommand{by}\isamarkupfalse%
\ simp\isanewline
\ \ \ \ \isacommand{also}\isamarkupfalse%
\ \isacommand{have}\isamarkupfalse%
\ {\isachardoublequoteopen}{\isachardot}{\kern0pt}{\isachardot}{\kern0pt}{\isachardot}{\kern0pt}\ {\isasymle}\ f{\isadigit{0}}{\isacharunderscore}{\kern0pt}space{\isacharunderscore}{\kern0pt}usage\ {\isacharparenleft}{\kern0pt}n{\isacharcomma}{\kern0pt}\ {\isasymepsilon}{\isacharcomma}{\kern0pt}\ {\isasymdelta}{\isacharparenright}{\kern0pt}{\isachardoublequoteclose}\isanewline
\ \ \ \ \ \ \isacommand{apply}\isamarkupfalse%
\ {\isacharparenleft}{\kern0pt}subst\ log{\isacharunderscore}{\kern0pt}mult{\isacharcomma}{\kern0pt}\ simp{\isacharcomma}{\kern0pt}\ simp{\isacharcomma}{\kern0pt}\ simp{\isacharparenright}{\kern0pt}\isanewline
\ \ \ \ \ \ \isacommand{apply}\isamarkupfalse%
\ {\isacharparenleft}{\kern0pt}subst\ log{\isacharunderscore}{\kern0pt}mult{\isacharcomma}{\kern0pt}\ simp{\isacharcomma}{\kern0pt}\ simp{\isacharcomma}{\kern0pt}\ simp{\isacharparenright}{\kern0pt}\isanewline
\ \ \ \ \ \ \isacommand{apply}\isamarkupfalse%
\ {\isacharparenleft}{\kern0pt}simp\ add{\isacharcolon}{\kern0pt}s{\isacharunderscore}{\kern0pt}def{\isacharbrackleft}{\kern0pt}symmetric{\isacharbrackright}{\kern0pt}\ r{\isacharunderscore}{\kern0pt}def{\isacharbrackleft}{\kern0pt}symmetric{\isacharbrackright}{\kern0pt}\ t{\isacharunderscore}{\kern0pt}def{\isacharbrackleft}{\kern0pt}symmetric{\isacharbrackright}{\kern0pt}{\isacharparenright}{\kern0pt}\isanewline
\ \ \ \ \ \ \isacommand{by}\isamarkupfalse%
\ {\isacharparenleft}{\kern0pt}simp\ add{\isacharcolon}{\kern0pt}algebra{\isacharunderscore}{\kern0pt}simps{\isacharparenright}{\kern0pt}\isanewline
\ \ \ \ \isacommand{finally}\isamarkupfalse%
\ \isacommand{show}\isamarkupfalse%
\ {\isachardoublequoteopen}bit{\isacharunderscore}{\kern0pt}count\ {\isacharparenleft}{\kern0pt}encode{\isacharunderscore}{\kern0pt}state\ {\isacharparenleft}{\kern0pt}s{\isacharcomma}{\kern0pt}\ t{\isacharcomma}{\kern0pt}\ p{\isacharcomma}{\kern0pt}\ r{\isacharcomma}{\kern0pt}\ x{\isacharcomma}{\kern0pt}\ {\isasymlambda}i{\isasymin}{\isacharbraceleft}{\kern0pt}{\isadigit{0}}{\isachardot}{\kern0pt}{\isachardot}{\kern0pt}{\isacharless}{\kern0pt}s{\isacharbraceright}{\kern0pt}{\isachardot}{\kern0pt}\ f{\isadigit{0}}{\isacharunderscore}{\kern0pt}sketch\ p\ r\ t\ {\isacharparenleft}{\kern0pt}x\ i{\isacharparenright}{\kern0pt}\ as{\isacharparenright}{\kern0pt}{\isacharparenright}{\kern0pt}\ {\isasymle}\ \isanewline
\ \ \ \ \ \ \ \ f{\isadigit{0}}{\isacharunderscore}{\kern0pt}space{\isacharunderscore}{\kern0pt}usage\ {\isacharparenleft}{\kern0pt}n{\isacharcomma}{\kern0pt}\ {\isasymepsilon}{\isacharcomma}{\kern0pt}\ {\isasymdelta}{\isacharparenright}{\kern0pt}{\isachardoublequoteclose}\ \isacommand{by}\isamarkupfalse%
\ simp\isanewline
\ \ \isacommand{qed}\isamarkupfalse%
\isanewline
\ \ \isanewline
\ \ \isacommand{have}\isamarkupfalse%
\ a{\isacharcolon}{\kern0pt}{\isachardoublequoteopen}{\isasymAnd}y{\isachardot}{\kern0pt}\ y\ {\isasymin}\ {\isacharparenleft}{\kern0pt}{\isasymlambda}x{\isachardot}{\kern0pt}\ {\isacharparenleft}{\kern0pt}s{\isacharcomma}{\kern0pt}\ t{\isacharcomma}{\kern0pt}\ p{\isacharcomma}{\kern0pt}\ r{\isacharcomma}{\kern0pt}\ x{\isacharcomma}{\kern0pt}\ {\isasymlambda}i{\isasymin}{\isacharbraceleft}{\kern0pt}{\isadigit{0}}{\isachardot}{\kern0pt}{\isachardot}{\kern0pt}{\isacharless}{\kern0pt}s{\isacharbraceright}{\kern0pt}{\isachardot}{\kern0pt}\ f{\isadigit{0}}{\isacharunderscore}{\kern0pt}sketch\ p\ r\ t\ {\isacharparenleft}{\kern0pt}x\ i{\isacharparenright}{\kern0pt}\ as{\isacharparenright}{\kern0pt}{\isacharparenright}{\kern0pt}\ {\isacharbackquote}{\kern0pt}\isanewline
\ \ \ \ \ \ \ \ \ \ \ \ \ {\isacharparenleft}{\kern0pt}{\isacharbraceleft}{\kern0pt}{\isadigit{0}}{\isachardot}{\kern0pt}{\isachardot}{\kern0pt}{\isacharless}{\kern0pt}s{\isacharbraceright}{\kern0pt}\ {\isasymrightarrow}\isactrlsub E\ bounded{\isacharunderscore}{\kern0pt}degree{\isacharunderscore}{\kern0pt}polynomials\ {\isacharparenleft}{\kern0pt}ZFact\ {\isacharparenleft}{\kern0pt}int\ p{\isacharparenright}{\kern0pt}{\isacharparenright}{\kern0pt}\ {\isadigit{2}}{\isacharparenright}{\kern0pt}\ {\isasymLongrightarrow}\isanewline
\ \ \ \ \ \ \ \ \ bit{\isacharunderscore}{\kern0pt}count\ {\isacharparenleft}{\kern0pt}encode{\isacharunderscore}{\kern0pt}state\ y{\isacharparenright}{\kern0pt}\ {\isasymle}\ f{\isadigit{0}}{\isacharunderscore}{\kern0pt}space{\isacharunderscore}{\kern0pt}usage\ {\isacharparenleft}{\kern0pt}n{\isacharcomma}{\kern0pt}\ {\isasymepsilon}{\isacharcomma}{\kern0pt}\ {\isasymdelta}{\isacharparenright}{\kern0pt}{\isachardoublequoteclose}\isanewline
\ \ \ \ \isacommand{using}\isamarkupfalse%
\ b\ \isacommand{apply}\isamarkupfalse%
\ {\isacharparenleft}{\kern0pt}simp\ add{\isacharcolon}{\kern0pt}image{\isacharunderscore}{\kern0pt}def\ del{\isacharcolon}{\kern0pt}f{\isadigit{0}}{\isacharunderscore}{\kern0pt}space{\isacharunderscore}{\kern0pt}usage{\isachardot}{\kern0pt}simps{\isacharparenright}{\kern0pt}\ \isacommand{by}\isamarkupfalse%
\ blast\isanewline
\isanewline
\ \ \isacommand{show}\isamarkupfalse%
\ {\isacharquery}{\kern0pt}thesis\isanewline
\ \ \ \ \isacommand{apply}\isamarkupfalse%
\ {\isacharparenleft}{\kern0pt}subst\ AE{\isacharunderscore}{\kern0pt}measure{\isacharunderscore}{\kern0pt}pmf{\isacharunderscore}{\kern0pt}iff{\isacharcomma}{\kern0pt}\ rule\ ballI{\isacharparenright}{\kern0pt}\isanewline
\ \ \ \ \isacommand{apply}\isamarkupfalse%
\ {\isacharparenleft}{\kern0pt}subst\ {\isacharparenleft}{\kern0pt}asm{\isacharparenright}{\kern0pt}\ M{\isacharunderscore}{\kern0pt}def{\isacharparenright}{\kern0pt}\isanewline
\ \ \ \ \isacommand{apply}\isamarkupfalse%
\ {\isacharparenleft}{\kern0pt}subst\ {\isacharparenleft}{\kern0pt}asm{\isacharparenright}{\kern0pt}\ f{\isadigit{0}}{\isacharunderscore}{\kern0pt}alg{\isacharunderscore}{\kern0pt}sketch{\isacharbrackleft}{\kern0pt}OF\ assms{\isacharparenleft}{\kern0pt}{\isadigit{1}}{\isacharparenright}{\kern0pt}\ assms{\isacharparenleft}{\kern0pt}{\isadigit{2}}{\isacharparenright}{\kern0pt}{\isacharbrackright}{\kern0pt}{\isacharcomma}{\kern0pt}\ simp{\isacharparenright}{\kern0pt}\isanewline
\ \ \ \ \isacommand{apply}\isamarkupfalse%
\ {\isacharparenleft}{\kern0pt}simp\ add{\isacharcolon}{\kern0pt}s{\isacharunderscore}{\kern0pt}def{\isacharbrackleft}{\kern0pt}symmetric{\isacharbrackright}{\kern0pt}\ t{\isacharunderscore}{\kern0pt}def{\isacharbrackleft}{\kern0pt}symmetric{\isacharbrackright}{\kern0pt}\ p{\isacharunderscore}{\kern0pt}def{\isacharbrackleft}{\kern0pt}symmetric{\isacharbrackright}{\kern0pt}\ r{\isacharunderscore}{\kern0pt}def{\isacharbrackleft}{\kern0pt}symmetric{\isacharbrackright}{\kern0pt}{\isacharparenright}{\kern0pt}\isanewline
\ \ \ \ \isacommand{apply}\isamarkupfalse%
\ {\isacharparenleft}{\kern0pt}subst\ {\isacharparenleft}{\kern0pt}asm{\isacharparenright}{\kern0pt}\ set{\isacharunderscore}{\kern0pt}prod{\isacharunderscore}{\kern0pt}pmf{\isacharcomma}{\kern0pt}\ simp{\isacharparenright}{\kern0pt}\isanewline
\ \ \ \ \isacommand{apply}\isamarkupfalse%
\ {\isacharparenleft}{\kern0pt}simp\ add{\isacharcolon}{\kern0pt}comp{\isacharunderscore}{\kern0pt}def{\isacharparenright}{\kern0pt}\isanewline
\ \ \ \ \isacommand{apply}\isamarkupfalse%
\ {\isacharparenleft}{\kern0pt}subst\ {\isacharparenleft}{\kern0pt}asm{\isacharparenright}{\kern0pt}\ set{\isacharunderscore}{\kern0pt}pmf{\isacharunderscore}{\kern0pt}of{\isacharunderscore}{\kern0pt}set{\isacharparenright}{\kern0pt}\isanewline
\ \ \ \ \ \ \isacommand{apply}\isamarkupfalse%
\ {\isacharparenleft}{\kern0pt}metis\ ne{\isacharunderscore}{\kern0pt}bounded{\isacharunderscore}{\kern0pt}degree{\isacharunderscore}{\kern0pt}polynomials{\isacharparenright}{\kern0pt}\isanewline
\ \ \ \ \ \isacommand{apply}\isamarkupfalse%
\ {\isacharparenleft}{\kern0pt}metis\ fin{\isacharunderscore}{\kern0pt}bounded{\isacharunderscore}{\kern0pt}degree{\isacharunderscore}{\kern0pt}polynomials{\isacharbrackleft}{\kern0pt}OF\ p{\isacharunderscore}{\kern0pt}ge{\isacharunderscore}{\kern0pt}{\isadigit{0}}{\isacharbrackright}{\kern0pt}{\isacharparenright}{\kern0pt}\isanewline
\ \ \ \ \isacommand{using}\isamarkupfalse%
\ a\isanewline
\ \ \ \ \isacommand{by}\isamarkupfalse%
\ {\isacharparenleft}{\kern0pt}simp\ add{\isacharcolon}{\kern0pt}s{\isacharunderscore}{\kern0pt}def{\isacharbrackleft}{\kern0pt}symmetric{\isacharbrackright}{\kern0pt}\ t{\isacharunderscore}{\kern0pt}def{\isacharbrackleft}{\kern0pt}symmetric{\isacharbrackright}{\kern0pt}\ p{\isacharunderscore}{\kern0pt}def{\isacharbrackleft}{\kern0pt}symmetric{\isacharbrackright}{\kern0pt}\ r{\isacharunderscore}{\kern0pt}def{\isacharbrackleft}{\kern0pt}symmetric{\isacharbrackright}{\kern0pt}{\isacharparenright}{\kern0pt}\isanewline
\isacommand{qed}\isamarkupfalse%
%
\endisatagproof
{\isafoldproof}%
%
\isadelimproof
\isanewline
%
\endisadelimproof
\isanewline
\isacommand{lemma}\isamarkupfalse%
\ f{\isadigit{0}}{\isacharunderscore}{\kern0pt}asympotic{\isacharunderscore}{\kern0pt}space{\isacharunderscore}{\kern0pt}complexity{\isacharcolon}{\kern0pt}\isanewline
\ \ {\isachardoublequoteopen}f{\isadigit{0}}{\isacharunderscore}{\kern0pt}space{\isacharunderscore}{\kern0pt}usage\ {\isasymin}\ O{\isacharbrackleft}{\kern0pt}at{\isacharunderscore}{\kern0pt}top\ {\isasymtimes}\isactrlsub F\ at{\isacharunderscore}{\kern0pt}right\ {\isadigit{0}}\ {\isasymtimes}\isactrlsub F\ at{\isacharunderscore}{\kern0pt}right\ {\isadigit{0}}{\isacharbrackright}{\kern0pt}{\isacharparenleft}{\kern0pt}{\isasymlambda}{\isacharparenleft}{\kern0pt}n{\isacharcomma}{\kern0pt}\ {\isasymepsilon}{\isacharcomma}{\kern0pt}\ {\isasymdelta}{\isacharparenright}{\kern0pt}{\isachardot}{\kern0pt}\ ln\ {\isacharparenleft}{\kern0pt}{\isadigit{1}}\ {\isacharslash}{\kern0pt}\ of{\isacharunderscore}{\kern0pt}rat\ {\isasymepsilon}{\isacharparenright}{\kern0pt}\ {\isacharasterisk}{\kern0pt}\ \isanewline
\ \ {\isacharparenleft}{\kern0pt}ln\ {\isacharparenleft}{\kern0pt}real\ n{\isacharparenright}{\kern0pt}\ {\isacharplus}{\kern0pt}\ {\isadigit{1}}\ {\isacharslash}{\kern0pt}\ {\isacharparenleft}{\kern0pt}of{\isacharunderscore}{\kern0pt}rat\ {\isasymdelta}{\isacharparenright}{\kern0pt}\isactrlsup {\isadigit{2}}\ {\isacharasterisk}{\kern0pt}\ {\isacharparenleft}{\kern0pt}ln\ {\isacharparenleft}{\kern0pt}ln\ {\isacharparenleft}{\kern0pt}real\ n{\isacharparenright}{\kern0pt}{\isacharparenright}{\kern0pt}\ {\isacharplus}{\kern0pt}\ ln\ {\isacharparenleft}{\kern0pt}{\isadigit{1}}\ {\isacharslash}{\kern0pt}\ of{\isacharunderscore}{\kern0pt}rat\ {\isasymdelta}{\isacharparenright}{\kern0pt}{\isacharparenright}{\kern0pt}{\isacharparenright}{\kern0pt}{\isacharparenright}{\kern0pt}{\isachardoublequoteclose}\isanewline
\ \ {\isacharparenleft}{\kern0pt}\isakeyword{is}\ {\isachardoublequoteopen}{\isacharunderscore}{\kern0pt}\ {\isasymin}\ O{\isacharbrackleft}{\kern0pt}{\isacharquery}{\kern0pt}F{\isacharbrackright}{\kern0pt}{\isacharparenleft}{\kern0pt}{\isacharquery}{\kern0pt}rhs{\isacharparenright}{\kern0pt}{\isachardoublequoteclose}{\isacharparenright}{\kern0pt}\isanewline
%
\isadelimproof
%
\endisadelimproof
%
\isatagproof
\isacommand{proof}\isamarkupfalse%
\ {\isacharminus}{\kern0pt}\isanewline
\ \ \isacommand{define}\isamarkupfalse%
\ n{\isacharunderscore}{\kern0pt}of\ {\isacharcolon}{\kern0pt}{\isacharcolon}{\kern0pt}\ {\isachardoublequoteopen}nat\ {\isasymtimes}\ rat\ {\isasymtimes}\ rat\ {\isasymRightarrow}\ nat{\isachardoublequoteclose}\ \isakeyword{where}\ {\isachardoublequoteopen}n{\isacharunderscore}{\kern0pt}of\ {\isacharequal}{\kern0pt}\ {\isacharparenleft}{\kern0pt}{\isasymlambda}{\isacharparenleft}{\kern0pt}n{\isacharcomma}{\kern0pt}\ {\isasymepsilon}{\isacharcomma}{\kern0pt}\ {\isasymdelta}{\isacharparenright}{\kern0pt}{\isachardot}{\kern0pt}\ n{\isacharparenright}{\kern0pt}{\isachardoublequoteclose}\isanewline
\ \ \isacommand{define}\isamarkupfalse%
\ {\isasymepsilon}{\isacharunderscore}{\kern0pt}of\ {\isacharcolon}{\kern0pt}{\isacharcolon}{\kern0pt}\ {\isachardoublequoteopen}nat\ {\isasymtimes}\ rat\ {\isasymtimes}\ rat\ {\isasymRightarrow}\ rat{\isachardoublequoteclose}\ \isakeyword{where}\ {\isachardoublequoteopen}{\isasymepsilon}{\isacharunderscore}{\kern0pt}of\ {\isacharequal}{\kern0pt}\ {\isacharparenleft}{\kern0pt}{\isasymlambda}{\isacharparenleft}{\kern0pt}n{\isacharcomma}{\kern0pt}\ {\isasymepsilon}{\isacharcomma}{\kern0pt}\ {\isasymdelta}{\isacharparenright}{\kern0pt}{\isachardot}{\kern0pt}\ {\isasymepsilon}{\isacharparenright}{\kern0pt}{\isachardoublequoteclose}\isanewline
\ \ \isacommand{define}\isamarkupfalse%
\ {\isasymdelta}{\isacharunderscore}{\kern0pt}of\ {\isacharcolon}{\kern0pt}{\isacharcolon}{\kern0pt}\ {\isachardoublequoteopen}nat\ {\isasymtimes}\ rat\ {\isasymtimes}\ rat\ {\isasymRightarrow}\ rat{\isachardoublequoteclose}\ \isakeyword{where}\ {\isachardoublequoteopen}{\isasymdelta}{\isacharunderscore}{\kern0pt}of\ {\isacharequal}{\kern0pt}\ {\isacharparenleft}{\kern0pt}{\isasymlambda}{\isacharparenleft}{\kern0pt}n{\isacharcomma}{\kern0pt}\ {\isasymepsilon}{\isacharcomma}{\kern0pt}\ {\isasymdelta}{\isacharparenright}{\kern0pt}{\isachardot}{\kern0pt}\ {\isasymdelta}{\isacharparenright}{\kern0pt}{\isachardoublequoteclose}\isanewline
\isanewline
\ \ \isacommand{define}\isamarkupfalse%
\ g\ \isakeyword{where}\ {\isachardoublequoteopen}g\ {\isacharequal}{\kern0pt}\ {\isacharparenleft}{\kern0pt}{\isasymlambda}x{\isachardot}{\kern0pt}\ ln\ {\isacharparenleft}{\kern0pt}{\isadigit{1}}\ {\isacharslash}{\kern0pt}\ of{\isacharunderscore}{\kern0pt}rat\ {\isacharparenleft}{\kern0pt}{\isasymepsilon}{\isacharunderscore}{\kern0pt}of\ x{\isacharparenright}{\kern0pt}{\isacharparenright}{\kern0pt}\ {\isacharasterisk}{\kern0pt}\ \isanewline
\ \ \ \ {\isacharparenleft}{\kern0pt}ln\ {\isacharparenleft}{\kern0pt}real\ {\isacharparenleft}{\kern0pt}n{\isacharunderscore}{\kern0pt}of\ x{\isacharparenright}{\kern0pt}{\isacharparenright}{\kern0pt}\ {\isacharplus}{\kern0pt}\ {\isadigit{1}}\ {\isacharslash}{\kern0pt}\ {\isacharparenleft}{\kern0pt}of{\isacharunderscore}{\kern0pt}rat\ {\isacharparenleft}{\kern0pt}{\isasymdelta}{\isacharunderscore}{\kern0pt}of\ x{\isacharparenright}{\kern0pt}{\isacharparenright}{\kern0pt}\isactrlsup {\isadigit{2}}\ {\isacharasterisk}{\kern0pt}\ {\isacharparenleft}{\kern0pt}ln\ {\isacharparenleft}{\kern0pt}ln\ {\isacharparenleft}{\kern0pt}real\ {\isacharparenleft}{\kern0pt}n{\isacharunderscore}{\kern0pt}of\ x{\isacharparenright}{\kern0pt}{\isacharparenright}{\kern0pt}{\isacharparenright}{\kern0pt}\ {\isacharplus}{\kern0pt}\ ln\ {\isacharparenleft}{\kern0pt}{\isadigit{1}}\ {\isacharslash}{\kern0pt}\ of{\isacharunderscore}{\kern0pt}rat\ {\isacharparenleft}{\kern0pt}{\isasymdelta}{\isacharunderscore}{\kern0pt}of\ x{\isacharparenright}{\kern0pt}{\isacharparenright}{\kern0pt}{\isacharparenright}{\kern0pt}{\isacharparenright}{\kern0pt}{\isacharparenright}{\kern0pt}{\isachardoublequoteclose}\isanewline
\isanewline
\ \ \isacommand{have}\isamarkupfalse%
\ n{\isacharunderscore}{\kern0pt}inf{\isacharcolon}{\kern0pt}\ {\isachardoublequoteopen}{\isasymAnd}c{\isachardot}{\kern0pt}\ eventually\ {\isacharparenleft}{\kern0pt}{\isasymlambda}x{\isachardot}{\kern0pt}\ c\ {\isasymle}\ {\isacharparenleft}{\kern0pt}real\ {\isacharparenleft}{\kern0pt}n{\isacharunderscore}{\kern0pt}of\ x{\isacharparenright}{\kern0pt}{\isacharparenright}{\kern0pt}{\isacharparenright}{\kern0pt}\ {\isacharquery}{\kern0pt}F{\isachardoublequoteclose}\ \isanewline
\ \ \ \ \isacommand{apply}\isamarkupfalse%
\ {\isacharparenleft}{\kern0pt}simp\ add{\isacharcolon}{\kern0pt}n{\isacharunderscore}{\kern0pt}of{\isacharunderscore}{\kern0pt}def\ case{\isacharunderscore}{\kern0pt}prod{\isacharunderscore}{\kern0pt}beta{\isacharprime}{\kern0pt}{\isacharparenright}{\kern0pt}\isanewline
\ \ \ \ \isacommand{apply}\isamarkupfalse%
\ {\isacharparenleft}{\kern0pt}subst\ eventually{\isacharunderscore}{\kern0pt}prod{\isadigit{1}}{\isacharprime}{\kern0pt}{\isacharcomma}{\kern0pt}\ simp\ add{\isacharcolon}{\kern0pt}prod{\isacharunderscore}{\kern0pt}filter{\isacharunderscore}{\kern0pt}eq{\isacharunderscore}{\kern0pt}bot{\isacharparenright}{\kern0pt}\isanewline
\ \ \ \ \isacommand{by}\isamarkupfalse%
\ {\isacharparenleft}{\kern0pt}meson\ eventually{\isacharunderscore}{\kern0pt}at{\isacharunderscore}{\kern0pt}top{\isacharunderscore}{\kern0pt}linorder\ nat{\isacharunderscore}{\kern0pt}ceiling{\isacharunderscore}{\kern0pt}le{\isacharunderscore}{\kern0pt}eq{\isacharparenright}{\kern0pt}\isanewline
\isanewline
\ \ \isacommand{have}\isamarkupfalse%
\ delta{\isacharunderscore}{\kern0pt}inf{\isacharcolon}{\kern0pt}\ {\isachardoublequoteopen}{\isasymAnd}c{\isachardot}{\kern0pt}\ eventually\ {\isacharparenleft}{\kern0pt}{\isasymlambda}x{\isachardot}{\kern0pt}\ c\ {\isasymle}\ {\isadigit{1}}\ {\isacharslash}{\kern0pt}\ {\isacharparenleft}{\kern0pt}real{\isacharunderscore}{\kern0pt}of{\isacharunderscore}{\kern0pt}rat\ {\isacharparenleft}{\kern0pt}{\isasymdelta}{\isacharunderscore}{\kern0pt}of\ x{\isacharparenright}{\kern0pt}{\isacharparenright}{\kern0pt}{\isacharparenright}{\kern0pt}\ {\isacharquery}{\kern0pt}F{\isachardoublequoteclose}\isanewline
\ \ \ \ \isacommand{apply}\isamarkupfalse%
\ {\isacharparenleft}{\kern0pt}simp\ add{\isacharcolon}{\kern0pt}{\isasymdelta}{\isacharunderscore}{\kern0pt}of{\isacharunderscore}{\kern0pt}def\ case{\isacharunderscore}{\kern0pt}prod{\isacharunderscore}{\kern0pt}beta{\isacharprime}{\kern0pt}{\isacharparenright}{\kern0pt}\isanewline
\ \ \ \ \isacommand{apply}\isamarkupfalse%
\ {\isacharparenleft}{\kern0pt}subst\ eventually{\isacharunderscore}{\kern0pt}prod{\isadigit{2}}{\isacharprime}{\kern0pt}{\isacharcomma}{\kern0pt}\ simp{\isacharparenright}{\kern0pt}\isanewline
\ \ \ \ \isacommand{apply}\isamarkupfalse%
\ {\isacharparenleft}{\kern0pt}subst\ eventually{\isacharunderscore}{\kern0pt}prod{\isadigit{2}}{\isacharprime}{\kern0pt}{\isacharcomma}{\kern0pt}\ simp{\isacharparenright}{\kern0pt}\isanewline
\ \ \ \ \isacommand{by}\isamarkupfalse%
\ {\isacharparenleft}{\kern0pt}rule\ inv{\isacharunderscore}{\kern0pt}at{\isacharunderscore}{\kern0pt}right{\isacharunderscore}{\kern0pt}{\isadigit{0}}{\isacharunderscore}{\kern0pt}inf{\isacharparenright}{\kern0pt}\isanewline
\isanewline
\ \ \isacommand{have}\isamarkupfalse%
\ eps{\isacharunderscore}{\kern0pt}inf{\isacharcolon}{\kern0pt}\ {\isachardoublequoteopen}{\isasymAnd}c{\isachardot}{\kern0pt}\ eventually\ {\isacharparenleft}{\kern0pt}{\isasymlambda}x{\isachardot}{\kern0pt}\ c\ {\isasymle}\ {\isadigit{1}}\ {\isacharslash}{\kern0pt}\ {\isacharparenleft}{\kern0pt}real{\isacharunderscore}{\kern0pt}of{\isacharunderscore}{\kern0pt}rat\ {\isacharparenleft}{\kern0pt}{\isasymepsilon}{\isacharunderscore}{\kern0pt}of\ x{\isacharparenright}{\kern0pt}{\isacharparenright}{\kern0pt}{\isacharparenright}{\kern0pt}\ {\isacharquery}{\kern0pt}F{\isachardoublequoteclose}\isanewline
\ \ \ \ \isacommand{apply}\isamarkupfalse%
\ {\isacharparenleft}{\kern0pt}simp\ add{\isacharcolon}{\kern0pt}{\isasymepsilon}{\isacharunderscore}{\kern0pt}of{\isacharunderscore}{\kern0pt}def\ case{\isacharunderscore}{\kern0pt}prod{\isacharunderscore}{\kern0pt}beta{\isacharprime}{\kern0pt}{\isacharparenright}{\kern0pt}\isanewline
\ \ \ \ \isacommand{apply}\isamarkupfalse%
\ {\isacharparenleft}{\kern0pt}subst\ eventually{\isacharunderscore}{\kern0pt}prod{\isadigit{2}}{\isacharprime}{\kern0pt}{\isacharcomma}{\kern0pt}\ simp{\isacharparenright}{\kern0pt}\isanewline
\ \ \ \ \isacommand{apply}\isamarkupfalse%
\ {\isacharparenleft}{\kern0pt}subst\ eventually{\isacharunderscore}{\kern0pt}prod{\isadigit{1}}{\isacharprime}{\kern0pt}{\isacharcomma}{\kern0pt}\ simp{\isacharparenright}{\kern0pt}\isanewline
\ \ \ \ \isacommand{by}\isamarkupfalse%
\ {\isacharparenleft}{\kern0pt}rule\ inv{\isacharunderscore}{\kern0pt}at{\isacharunderscore}{\kern0pt}right{\isacharunderscore}{\kern0pt}{\isadigit{0}}{\isacharunderscore}{\kern0pt}inf{\isacharparenright}{\kern0pt}\isanewline
\isanewline
\ \ \isacommand{have}\isamarkupfalse%
\ zero{\isacharunderscore}{\kern0pt}less{\isacharunderscore}{\kern0pt}eps{\isacharcolon}{\kern0pt}\ {\isachardoublequoteopen}eventually\ {\isacharparenleft}{\kern0pt}{\isasymlambda}x{\isachardot}{\kern0pt}\ {\isadigit{0}}\ {\isacharless}{\kern0pt}\ {\isacharparenleft}{\kern0pt}real{\isacharunderscore}{\kern0pt}of{\isacharunderscore}{\kern0pt}rat\ {\isacharparenleft}{\kern0pt}{\isasymepsilon}{\isacharunderscore}{\kern0pt}of\ x{\isacharparenright}{\kern0pt}{\isacharparenright}{\kern0pt}{\isacharparenright}{\kern0pt}\ {\isacharquery}{\kern0pt}F{\isachardoublequoteclose}\isanewline
\ \ \ \ \isacommand{apply}\isamarkupfalse%
\ {\isacharparenleft}{\kern0pt}simp\ add{\isacharcolon}{\kern0pt}{\isasymepsilon}{\isacharunderscore}{\kern0pt}of{\isacharunderscore}{\kern0pt}def\ case{\isacharunderscore}{\kern0pt}prod{\isacharunderscore}{\kern0pt}beta{\isacharprime}{\kern0pt}{\isacharparenright}{\kern0pt}\isanewline
\ \ \ \ \isacommand{apply}\isamarkupfalse%
\ {\isacharparenleft}{\kern0pt}subst\ eventually{\isacharunderscore}{\kern0pt}prod{\isadigit{2}}{\isacharprime}{\kern0pt}{\isacharcomma}{\kern0pt}\ simp{\isacharparenright}{\kern0pt}\isanewline
\ \ \ \ \isacommand{apply}\isamarkupfalse%
\ {\isacharparenleft}{\kern0pt}subst\ eventually{\isacharunderscore}{\kern0pt}prod{\isadigit{1}}{\isacharprime}{\kern0pt}{\isacharcomma}{\kern0pt}\ simp{\isacharparenright}{\kern0pt}\isanewline
\ \ \ \ \isacommand{by}\isamarkupfalse%
\ {\isacharparenleft}{\kern0pt}rule\ eventually{\isacharunderscore}{\kern0pt}at{\isacharunderscore}{\kern0pt}rightI{\isacharbrackleft}{\kern0pt}\isakeyword{where}\ b{\isacharequal}{\kern0pt}{\isachardoublequoteopen}{\isadigit{1}}{\isachardoublequoteclose}{\isacharbrackright}{\kern0pt}{\isacharcomma}{\kern0pt}\ simp{\isacharcomma}{\kern0pt}\ simp{\isacharparenright}{\kern0pt}\isanewline
\isanewline
\ \ \isacommand{have}\isamarkupfalse%
\ zero{\isacharunderscore}{\kern0pt}less{\isacharunderscore}{\kern0pt}delta{\isacharcolon}{\kern0pt}\ {\isachardoublequoteopen}eventually\ {\isacharparenleft}{\kern0pt}{\isasymlambda}x{\isachardot}{\kern0pt}\ {\isadigit{0}}\ {\isacharless}{\kern0pt}\ {\isacharparenleft}{\kern0pt}real{\isacharunderscore}{\kern0pt}of{\isacharunderscore}{\kern0pt}rat\ {\isacharparenleft}{\kern0pt}{\isasymdelta}{\isacharunderscore}{\kern0pt}of\ x{\isacharparenright}{\kern0pt}{\isacharparenright}{\kern0pt}{\isacharparenright}{\kern0pt}\ {\isacharquery}{\kern0pt}F{\isachardoublequoteclose}\isanewline
\ \ \ \ \isacommand{apply}\isamarkupfalse%
\ {\isacharparenleft}{\kern0pt}simp\ add{\isacharcolon}{\kern0pt}{\isasymdelta}{\isacharunderscore}{\kern0pt}of{\isacharunderscore}{\kern0pt}def\ case{\isacharunderscore}{\kern0pt}prod{\isacharunderscore}{\kern0pt}beta{\isacharprime}{\kern0pt}{\isacharparenright}{\kern0pt}\isanewline
\ \ \ \ \isacommand{apply}\isamarkupfalse%
\ {\isacharparenleft}{\kern0pt}subst\ eventually{\isacharunderscore}{\kern0pt}prod{\isadigit{2}}{\isacharprime}{\kern0pt}{\isacharcomma}{\kern0pt}\ simp{\isacharparenright}{\kern0pt}\isanewline
\ \ \ \ \isacommand{apply}\isamarkupfalse%
\ {\isacharparenleft}{\kern0pt}subst\ eventually{\isacharunderscore}{\kern0pt}prod{\isadigit{2}}{\isacharprime}{\kern0pt}{\isacharcomma}{\kern0pt}\ simp{\isacharparenright}{\kern0pt}\isanewline
\ \ \ \ \isacommand{by}\isamarkupfalse%
\ {\isacharparenleft}{\kern0pt}rule\ eventually{\isacharunderscore}{\kern0pt}at{\isacharunderscore}{\kern0pt}rightI{\isacharbrackleft}{\kern0pt}\isakeyword{where}\ b{\isacharequal}{\kern0pt}{\isachardoublequoteopen}{\isadigit{1}}{\isachardoublequoteclose}{\isacharbrackright}{\kern0pt}{\isacharcomma}{\kern0pt}\ simp{\isacharcomma}{\kern0pt}\ simp{\isacharparenright}{\kern0pt}\isanewline
\isanewline
\ \ \isacommand{have}\isamarkupfalse%
\ l{\isadigit{1}}{\isacharcolon}{\kern0pt}\ {\isachardoublequoteopen}{\isasymforall}\isactrlsub F\ x\ in\ {\isacharquery}{\kern0pt}F{\isachardot}{\kern0pt}\ {\isadigit{0}}\ {\isasymle}\ {\isacharparenleft}{\kern0pt}ln\ {\isacharparenleft}{\kern0pt}ln\ {\isacharparenleft}{\kern0pt}real\ {\isacharparenleft}{\kern0pt}n{\isacharunderscore}{\kern0pt}of\ x{\isacharparenright}{\kern0pt}{\isacharparenright}{\kern0pt}{\isacharparenright}{\kern0pt}\ {\isacharplus}{\kern0pt}\ ln\ {\isacharparenleft}{\kern0pt}{\isadigit{1}}\ {\isacharslash}{\kern0pt}\ real{\isacharunderscore}{\kern0pt}of{\isacharunderscore}{\kern0pt}rat\ {\isacharparenleft}{\kern0pt}{\isasymdelta}{\isacharunderscore}{\kern0pt}of\ x{\isacharparenright}{\kern0pt}{\isacharparenright}{\kern0pt}{\isacharparenright}{\kern0pt}\ {\isacharslash}{\kern0pt}\ {\isacharparenleft}{\kern0pt}real{\isacharunderscore}{\kern0pt}of{\isacharunderscore}{\kern0pt}rat\ {\isacharparenleft}{\kern0pt}{\isasymdelta}{\isacharunderscore}{\kern0pt}of\ x{\isacharparenright}{\kern0pt}{\isacharparenright}{\kern0pt}\isactrlsup {\isadigit{2}}{\isachardoublequoteclose}\isanewline
\ \ \ \ \isacommand{apply}\isamarkupfalse%
\ {\isacharparenleft}{\kern0pt}rule\ eventually{\isacharunderscore}{\kern0pt}nonneg{\isacharunderscore}{\kern0pt}div{\isacharparenright}{\kern0pt}\isanewline
\ \ \ \ \ \isacommand{apply}\isamarkupfalse%
\ {\isacharparenleft}{\kern0pt}rule\ eventually{\isacharunderscore}{\kern0pt}nonneg{\isacharunderscore}{\kern0pt}add{\isacharparenright}{\kern0pt}\isanewline
\ \ \ \ \ \isacommand{apply}\isamarkupfalse%
\ {\isacharparenleft}{\kern0pt}rule\ eventually{\isacharunderscore}{\kern0pt}ln{\isacharunderscore}{\kern0pt}ge{\isacharunderscore}{\kern0pt}iff{\isacharcomma}{\kern0pt}\ rule\ eventually{\isacharunderscore}{\kern0pt}ln{\isacharunderscore}{\kern0pt}ge{\isacharunderscore}{\kern0pt}iff{\isacharbrackleft}{\kern0pt}OF\ n{\isacharunderscore}{\kern0pt}inf{\isacharbrackright}{\kern0pt}{\isacharparenright}{\kern0pt}\isanewline
\ \ \ \ \isacommand{apply}\isamarkupfalse%
\ {\isacharparenleft}{\kern0pt}rule\ eventually{\isacharunderscore}{\kern0pt}ln{\isacharunderscore}{\kern0pt}ge{\isacharunderscore}{\kern0pt}iff{\isacharbrackleft}{\kern0pt}OF\ delta{\isacharunderscore}{\kern0pt}inf{\isacharbrackright}{\kern0pt}{\isacharparenright}{\kern0pt}\isanewline
\ \ \ \ \isacommand{by}\isamarkupfalse%
\ {\isacharparenleft}{\kern0pt}rule\ eventually{\isacharunderscore}{\kern0pt}mono{\isacharbrackleft}{\kern0pt}OF\ zero{\isacharunderscore}{\kern0pt}less{\isacharunderscore}{\kern0pt}delta{\isacharbrackright}{\kern0pt}{\isacharcomma}{\kern0pt}\ simp{\isacharparenright}{\kern0pt}\isanewline
\isanewline
\ \ \isacommand{have}\isamarkupfalse%
\ unit{\isacharunderscore}{\kern0pt}{\isadigit{1}}{\isacharcolon}{\kern0pt}\ {\isachardoublequoteopen}{\isacharparenleft}{\kern0pt}{\isasymlambda}{\isacharunderscore}{\kern0pt}{\isachardot}{\kern0pt}\ {\isadigit{1}}{\isacharparenright}{\kern0pt}\ {\isasymin}\ O{\isacharbrackleft}{\kern0pt}{\isacharquery}{\kern0pt}F{\isacharbrackright}{\kern0pt}{\isacharparenleft}{\kern0pt}{\isasymlambda}x{\isachardot}{\kern0pt}\ {\isadigit{1}}\ {\isacharslash}{\kern0pt}\ {\isacharparenleft}{\kern0pt}real{\isacharunderscore}{\kern0pt}of{\isacharunderscore}{\kern0pt}rat\ {\isacharparenleft}{\kern0pt}{\isasymdelta}{\isacharunderscore}{\kern0pt}of\ x{\isacharparenright}{\kern0pt}{\isacharparenright}{\kern0pt}\isactrlsup {\isadigit{2}}{\isacharparenright}{\kern0pt}{\isachardoublequoteclose}\ \isanewline
\ \ \ \ \isacommand{apply}\isamarkupfalse%
\ {\isacharparenleft}{\kern0pt}rule\ landau{\isacharunderscore}{\kern0pt}o{\isachardot}{\kern0pt}big{\isacharunderscore}{\kern0pt}mono{\isacharcomma}{\kern0pt}\ simp{\isacharparenright}{\kern0pt}\isanewline
\ \ \ \ \isacommand{apply}\isamarkupfalse%
\ {\isacharparenleft}{\kern0pt}rule\ eventually{\isacharunderscore}{\kern0pt}mono{\isacharbrackleft}{\kern0pt}OF\ eventually{\isacharunderscore}{\kern0pt}conj{\isacharbrackleft}{\kern0pt}OF\ delta{\isacharunderscore}{\kern0pt}inf{\isacharbrackleft}{\kern0pt}\isakeyword{where}\ c{\isacharequal}{\kern0pt}{\isachardoublequoteopen}{\isadigit{1}}{\isachardoublequoteclose}{\isacharbrackright}{\kern0pt}\ zero{\isacharunderscore}{\kern0pt}less{\isacharunderscore}{\kern0pt}delta{\isacharbrackright}{\kern0pt}{\isacharbrackright}{\kern0pt}{\isacharparenright}{\kern0pt}\isanewline
\ \ \ \ \isacommand{by}\isamarkupfalse%
\ {\isacharparenleft}{\kern0pt}metis\ one{\isacharunderscore}{\kern0pt}le{\isacharunderscore}{\kern0pt}power\ power{\isacharunderscore}{\kern0pt}one{\isacharunderscore}{\kern0pt}over{\isacharparenright}{\kern0pt}\isanewline
\isanewline
\ \ \isacommand{have}\isamarkupfalse%
\ unit{\isacharunderscore}{\kern0pt}{\isadigit{2}}{\isacharcolon}{\kern0pt}\ {\isachardoublequoteopen}{\isacharparenleft}{\kern0pt}{\isasymlambda}{\isacharunderscore}{\kern0pt}{\isachardot}{\kern0pt}\ {\isadigit{1}}{\isacharparenright}{\kern0pt}\ {\isasymin}\ O{\isacharbrackleft}{\kern0pt}{\isacharquery}{\kern0pt}F{\isacharbrackright}{\kern0pt}{\isacharparenleft}{\kern0pt}{\isasymlambda}x{\isachardot}{\kern0pt}\ ln\ {\isacharparenleft}{\kern0pt}{\isadigit{1}}\ {\isacharslash}{\kern0pt}\ real{\isacharunderscore}{\kern0pt}of{\isacharunderscore}{\kern0pt}rat\ {\isacharparenleft}{\kern0pt}{\isasymdelta}{\isacharunderscore}{\kern0pt}of\ x{\isacharparenright}{\kern0pt}{\isacharparenright}{\kern0pt}{\isacharparenright}{\kern0pt}{\isachardoublequoteclose}\isanewline
\ \ \ \ \isacommand{apply}\isamarkupfalse%
\ {\isacharparenleft}{\kern0pt}rule\ landau{\isacharunderscore}{\kern0pt}o{\isachardot}{\kern0pt}big{\isacharunderscore}{\kern0pt}mono{\isacharcomma}{\kern0pt}\ simp{\isacharparenright}{\kern0pt}\isanewline
\ \ \ \ \isacommand{apply}\isamarkupfalse%
\ {\isacharparenleft}{\kern0pt}rule\ eventually{\isacharunderscore}{\kern0pt}mono{\isacharbrackleft}{\kern0pt}OF\ eventually{\isacharunderscore}{\kern0pt}conj{\isacharbrackleft}{\kern0pt}OF\ delta{\isacharunderscore}{\kern0pt}inf{\isacharbrackleft}{\kern0pt}\isakeyword{where}\ c{\isacharequal}{\kern0pt}{\isachardoublequoteopen}exp\ {\isadigit{1}}{\isachardoublequoteclose}{\isacharbrackright}{\kern0pt}\ zero{\isacharunderscore}{\kern0pt}less{\isacharunderscore}{\kern0pt}delta{\isacharbrackright}{\kern0pt}{\isacharbrackright}{\kern0pt}{\isacharparenright}{\kern0pt}\isanewline
\ \ \ \ \isacommand{apply}\isamarkupfalse%
\ {\isacharparenleft}{\kern0pt}subst\ abs{\isacharunderscore}{\kern0pt}of{\isacharunderscore}{\kern0pt}nonneg{\isacharparenright}{\kern0pt}\isanewline
\ \ \ \ \ \isacommand{apply}\isamarkupfalse%
\ {\isacharparenleft}{\kern0pt}rule\ ln{\isacharunderscore}{\kern0pt}ge{\isacharunderscore}{\kern0pt}zero{\isacharparenright}{\kern0pt}\isanewline
\ \ \ \ \isacommand{apply}\isamarkupfalse%
\ {\isacharparenleft}{\kern0pt}meson\ dual{\isacharunderscore}{\kern0pt}order{\isachardot}{\kern0pt}trans\ one{\isacharunderscore}{\kern0pt}le{\isacharunderscore}{\kern0pt}exp{\isacharunderscore}{\kern0pt}iff\ rel{\isacharunderscore}{\kern0pt}simps{\isacharparenleft}{\kern0pt}{\isadigit{4}}{\isadigit{4}}{\isacharparenright}{\kern0pt}{\isacharparenright}{\kern0pt}\isanewline
\ \ \ \ \isacommand{by}\isamarkupfalse%
\ {\isacharparenleft}{\kern0pt}simp\ add{\isacharcolon}{\kern0pt}\ ln{\isacharunderscore}{\kern0pt}ge{\isacharunderscore}{\kern0pt}iff{\isacharparenright}{\kern0pt}\isanewline
\isanewline
\ \ \isacommand{have}\isamarkupfalse%
\ unit{\isacharunderscore}{\kern0pt}{\isadigit{3}}{\isacharcolon}{\kern0pt}\ {\isachardoublequoteopen}{\isacharparenleft}{\kern0pt}{\isasymlambda}{\isacharunderscore}{\kern0pt}{\isachardot}{\kern0pt}\ {\isadigit{1}}{\isacharparenright}{\kern0pt}\ {\isasymin}\ O{\isacharbrackleft}{\kern0pt}{\isacharquery}{\kern0pt}F{\isacharbrackright}{\kern0pt}{\isacharparenleft}{\kern0pt}{\isasymlambda}x{\isachardot}{\kern0pt}\ real\ {\isacharparenleft}{\kern0pt}n{\isacharunderscore}{\kern0pt}of\ x{\isacharparenright}{\kern0pt}{\isacharparenright}{\kern0pt}{\isachardoublequoteclose}\isanewline
\ \ \ \ \isacommand{by}\isamarkupfalse%
\ {\isacharparenleft}{\kern0pt}rule\ landau{\isacharunderscore}{\kern0pt}o{\isachardot}{\kern0pt}big{\isacharunderscore}{\kern0pt}mono{\isacharcomma}{\kern0pt}\ simp{\isacharcomma}{\kern0pt}\ rule\ n{\isacharunderscore}{\kern0pt}inf{\isacharparenright}{\kern0pt}\isanewline
\isanewline
\ \ \isacommand{have}\isamarkupfalse%
\ unit{\isacharunderscore}{\kern0pt}{\isadigit{4}}{\isacharcolon}{\kern0pt}\ {\isachardoublequoteopen}{\isacharparenleft}{\kern0pt}{\isasymlambda}{\isacharunderscore}{\kern0pt}{\isachardot}{\kern0pt}\ {\isadigit{1}}{\isacharparenright}{\kern0pt}\ {\isasymin}\ O{\isacharbrackleft}{\kern0pt}{\isacharquery}{\kern0pt}F{\isacharbrackright}{\kern0pt}{\isacharparenleft}{\kern0pt}{\isasymlambda}x{\isachardot}{\kern0pt}\ ln\ {\isacharparenleft}{\kern0pt}{\isadigit{1}}\ {\isacharslash}{\kern0pt}\ real{\isacharunderscore}{\kern0pt}of{\isacharunderscore}{\kern0pt}rat\ {\isacharparenleft}{\kern0pt}{\isasymepsilon}{\isacharunderscore}{\kern0pt}of\ x{\isacharparenright}{\kern0pt}{\isacharparenright}{\kern0pt}{\isacharparenright}{\kern0pt}{\isachardoublequoteclose}\isanewline
\ \ \ \ \isacommand{apply}\isamarkupfalse%
\ {\isacharparenleft}{\kern0pt}rule\ landau{\isacharunderscore}{\kern0pt}o{\isachardot}{\kern0pt}big{\isacharunderscore}{\kern0pt}mono{\isacharcomma}{\kern0pt}\ simp{\isacharparenright}{\kern0pt}\isanewline
\ \ \ \ \isacommand{apply}\isamarkupfalse%
\ {\isacharparenleft}{\kern0pt}rule\ eventually{\isacharunderscore}{\kern0pt}mono{\isacharbrackleft}{\kern0pt}OF\ eventually{\isacharunderscore}{\kern0pt}conj{\isacharbrackleft}{\kern0pt}OF\ eps{\isacharunderscore}{\kern0pt}inf{\isacharbrackleft}{\kern0pt}\isakeyword{where}\ c{\isacharequal}{\kern0pt}{\isachardoublequoteopen}exp\ {\isadigit{1}}{\isachardoublequoteclose}{\isacharbrackright}{\kern0pt}\ zero{\isacharunderscore}{\kern0pt}less{\isacharunderscore}{\kern0pt}eps{\isacharbrackright}{\kern0pt}{\isacharbrackright}{\kern0pt}{\isacharparenright}{\kern0pt}\isanewline
\ \ \ \ \isacommand{apply}\isamarkupfalse%
\ {\isacharparenleft}{\kern0pt}subst\ abs{\isacharunderscore}{\kern0pt}of{\isacharunderscore}{\kern0pt}nonneg{\isacharparenright}{\kern0pt}\isanewline
\ \ \ \ \ \isacommand{apply}\isamarkupfalse%
\ {\isacharparenleft}{\kern0pt}rule\ ln{\isacharunderscore}{\kern0pt}ge{\isacharunderscore}{\kern0pt}zero{\isacharparenright}{\kern0pt}\isanewline
\ \ \ \ \isacommand{using}\isamarkupfalse%
\ one{\isacharunderscore}{\kern0pt}le{\isacharunderscore}{\kern0pt}exp{\isacharunderscore}{\kern0pt}iff\ order{\isacharunderscore}{\kern0pt}trans{\isacharunderscore}{\kern0pt}rules{\isacharparenleft}{\kern0pt}{\isadigit{2}}{\isadigit{3}}{\isacharparenright}{\kern0pt}\ \isacommand{apply}\isamarkupfalse%
\ blast\isanewline
\ \ \ \ \isacommand{by}\isamarkupfalse%
\ {\isacharparenleft}{\kern0pt}simp\ add{\isacharcolon}{\kern0pt}\ ln{\isacharunderscore}{\kern0pt}ge{\isacharunderscore}{\kern0pt}iff{\isacharparenright}{\kern0pt}\isanewline
\isanewline
\ \ \isacommand{have}\isamarkupfalse%
\ unit{\isacharunderscore}{\kern0pt}{\isadigit{5}}{\isacharcolon}{\kern0pt}\ {\isachardoublequoteopen}{\isacharparenleft}{\kern0pt}{\isasymlambda}{\isacharunderscore}{\kern0pt}{\isachardot}{\kern0pt}\ {\isadigit{1}}{\isacharparenright}{\kern0pt}\ {\isasymin}\ O{\isacharbrackleft}{\kern0pt}{\isacharquery}{\kern0pt}F{\isacharbrackright}{\kern0pt}{\isacharparenleft}{\kern0pt}{\isasymlambda}x{\isachardot}{\kern0pt}\ {\isadigit{1}}\ {\isacharslash}{\kern0pt}\ real{\isacharunderscore}{\kern0pt}of{\isacharunderscore}{\kern0pt}rat\ {\isacharparenleft}{\kern0pt}{\isasymepsilon}{\isacharunderscore}{\kern0pt}of\ x{\isacharparenright}{\kern0pt}{\isacharparenright}{\kern0pt}{\isachardoublequoteclose}\isanewline
\ \ \ \ \isacommand{apply}\isamarkupfalse%
\ {\isacharparenleft}{\kern0pt}rule\ landau{\isacharunderscore}{\kern0pt}o{\isachardot}{\kern0pt}big{\isacharunderscore}{\kern0pt}mono{\isacharcomma}{\kern0pt}\ simp{\isacharparenright}{\kern0pt}\isanewline
\ \ \ \ \isacommand{apply}\isamarkupfalse%
\ {\isacharparenleft}{\kern0pt}rule\ eventually{\isacharunderscore}{\kern0pt}mono{\isacharbrackleft}{\kern0pt}OF\ eventually{\isacharunderscore}{\kern0pt}conj{\isacharbrackleft}{\kern0pt}OF\ eps{\isacharunderscore}{\kern0pt}inf{\isacharbrackleft}{\kern0pt}\isakeyword{where}\ c{\isacharequal}{\kern0pt}{\isachardoublequoteopen}{\isadigit{1}}{\isachardoublequoteclose}{\isacharbrackright}{\kern0pt}\ zero{\isacharunderscore}{\kern0pt}less{\isacharunderscore}{\kern0pt}eps{\isacharbrackright}{\kern0pt}{\isacharbrackright}{\kern0pt}{\isacharparenright}{\kern0pt}\isanewline
\ \ \ \ \isacommand{by}\isamarkupfalse%
\ simp\isanewline
\isanewline
\ \ \isacommand{have}\isamarkupfalse%
\ unit{\isacharunderscore}{\kern0pt}{\isadigit{6}}{\isacharcolon}{\kern0pt}\ {\isachardoublequoteopen}{\isacharparenleft}{\kern0pt}{\isasymlambda}{\isacharunderscore}{\kern0pt}{\isachardot}{\kern0pt}\ {\isadigit{1}}{\isacharparenright}{\kern0pt}\ {\isasymin}\ O{\isacharbrackleft}{\kern0pt}{\isacharquery}{\kern0pt}F{\isacharbrackright}{\kern0pt}{\isacharparenleft}{\kern0pt}{\isasymlambda}x{\isachardot}{\kern0pt}\ ln\ {\isacharparenleft}{\kern0pt}real\ {\isacharparenleft}{\kern0pt}n{\isacharunderscore}{\kern0pt}of\ x{\isacharparenright}{\kern0pt}{\isacharparenright}{\kern0pt}{\isacharparenright}{\kern0pt}{\isachardoublequoteclose}\ \isanewline
\ \ \ \ \isacommand{apply}\isamarkupfalse%
\ {\isacharparenleft}{\kern0pt}rule\ landau{\isacharunderscore}{\kern0pt}o{\isachardot}{\kern0pt}big{\isacharunderscore}{\kern0pt}mono{\isacharcomma}{\kern0pt}\ simp{\isacharparenright}{\kern0pt}\isanewline
\ \ \ \ \isacommand{apply}\isamarkupfalse%
\ {\isacharparenleft}{\kern0pt}rule\ eventually{\isacharunderscore}{\kern0pt}mono{\isacharbrackleft}{\kern0pt}OF\ n{\isacharunderscore}{\kern0pt}inf{\isacharbrackleft}{\kern0pt}\isakeyword{where}\ c{\isacharequal}{\kern0pt}{\isachardoublequoteopen}exp\ {\isadigit{1}}{\isachardoublequoteclose}{\isacharbrackright}{\kern0pt}{\isacharbrackright}{\kern0pt}{\isacharparenright}{\kern0pt}\isanewline
\ \ \ \ \isacommand{apply}\isamarkupfalse%
\ {\isacharparenleft}{\kern0pt}subst\ abs{\isacharunderscore}{\kern0pt}of{\isacharunderscore}{\kern0pt}nonneg{\isacharparenright}{\kern0pt}\isanewline
\ \ \ \ \isacommand{apply}\isamarkupfalse%
\ {\isacharparenleft}{\kern0pt}rule\ ln{\isacharunderscore}{\kern0pt}ge{\isacharunderscore}{\kern0pt}zero{\isacharparenright}{\kern0pt}\isanewline
\ \ \ \ \ \isacommand{apply}\isamarkupfalse%
\ {\isacharparenleft}{\kern0pt}metis\ less{\isacharunderscore}{\kern0pt}one\ not{\isacharunderscore}{\kern0pt}exp{\isacharunderscore}{\kern0pt}le{\isacharunderscore}{\kern0pt}zero\ not{\isacharunderscore}{\kern0pt}le\ of{\isacharunderscore}{\kern0pt}nat{\isacharunderscore}{\kern0pt}eq{\isacharunderscore}{\kern0pt}{\isadigit{0}}{\isacharunderscore}{\kern0pt}iff\ of{\isacharunderscore}{\kern0pt}nat{\isacharunderscore}{\kern0pt}ge{\isacharunderscore}{\kern0pt}{\isadigit{1}}{\isacharunderscore}{\kern0pt}iff{\isacharparenright}{\kern0pt}\isanewline
\ \ \ \ \isacommand{by}\isamarkupfalse%
\ {\isacharparenleft}{\kern0pt}metis\ less{\isacharunderscore}{\kern0pt}eq{\isacharunderscore}{\kern0pt}real{\isacharunderscore}{\kern0pt}def\ ln{\isacharunderscore}{\kern0pt}ge{\isacharunderscore}{\kern0pt}iff\ not{\isacharunderscore}{\kern0pt}exp{\isacharunderscore}{\kern0pt}le{\isacharunderscore}{\kern0pt}zero\ of{\isacharunderscore}{\kern0pt}nat{\isacharunderscore}{\kern0pt}{\isadigit{0}}{\isacharunderscore}{\kern0pt}le{\isacharunderscore}{\kern0pt}iff{\isacharparenright}{\kern0pt}\isanewline
\isanewline
\ \ \isacommand{have}\isamarkupfalse%
\ unit{\isacharunderscore}{\kern0pt}{\isadigit{7}}{\isacharcolon}{\kern0pt}\ {\isachardoublequoteopen}{\isacharparenleft}{\kern0pt}{\isasymlambda}{\isacharunderscore}{\kern0pt}{\isachardot}{\kern0pt}\ {\isadigit{1}}{\isacharparenright}{\kern0pt}\ {\isasymin}\ O{\isacharbrackleft}{\kern0pt}{\isacharquery}{\kern0pt}F{\isacharbrackright}{\kern0pt}{\isacharparenleft}{\kern0pt}{\isasymlambda}x{\isachardot}{\kern0pt}\ ln\ {\isacharparenleft}{\kern0pt}real\ {\isacharparenleft}{\kern0pt}n{\isacharunderscore}{\kern0pt}of\ x{\isacharparenright}{\kern0pt}{\isacharparenright}{\kern0pt}\ {\isacharplus}{\kern0pt}\ {\isacharparenleft}{\kern0pt}ln\ {\isacharparenleft}{\kern0pt}ln\ {\isacharparenleft}{\kern0pt}real\ {\isacharparenleft}{\kern0pt}n{\isacharunderscore}{\kern0pt}of\ x{\isacharparenright}{\kern0pt}{\isacharparenright}{\kern0pt}{\isacharparenright}{\kern0pt}\ {\isacharplus}{\kern0pt}\ ln\ {\isacharparenleft}{\kern0pt}{\isadigit{1}}\ {\isacharslash}{\kern0pt}\ real{\isacharunderscore}{\kern0pt}of{\isacharunderscore}{\kern0pt}rat\ {\isacharparenleft}{\kern0pt}{\isasymdelta}{\isacharunderscore}{\kern0pt}of\ x{\isacharparenright}{\kern0pt}{\isacharparenright}{\kern0pt}{\isacharparenright}{\kern0pt}\ {\isacharslash}{\kern0pt}\ {\isacharparenleft}{\kern0pt}real{\isacharunderscore}{\kern0pt}of{\isacharunderscore}{\kern0pt}rat\ {\isacharparenleft}{\kern0pt}{\isasymdelta}{\isacharunderscore}{\kern0pt}of\ x{\isacharparenright}{\kern0pt}{\isacharparenright}{\kern0pt}\isactrlsup {\isadigit{2}}{\isacharparenright}{\kern0pt}{\isachardoublequoteclose}\isanewline
\ \ \ \ \isacommand{apply}\isamarkupfalse%
\ {\isacharparenleft}{\kern0pt}rule\ landau{\isacharunderscore}{\kern0pt}sum{\isacharunderscore}{\kern0pt}{\isadigit{1}}{\isacharparenright}{\kern0pt}\isanewline
\ \ \ \ \ \ \isacommand{apply}\isamarkupfalse%
\ {\isacharparenleft}{\kern0pt}rule\ eventually{\isacharunderscore}{\kern0pt}ln{\isacharunderscore}{\kern0pt}ge{\isacharunderscore}{\kern0pt}iff{\isacharbrackleft}{\kern0pt}OF\ n{\isacharunderscore}{\kern0pt}inf{\isacharbrackright}{\kern0pt}{\isacharparenright}{\kern0pt}\isanewline
\ \ \ \ \ \isacommand{apply}\isamarkupfalse%
\ {\isacharparenleft}{\kern0pt}rule\ l{\isadigit{1}}{\isacharparenright}{\kern0pt}\isanewline
\ \ \ \ \isacommand{by}\isamarkupfalse%
\ {\isacharparenleft}{\kern0pt}rule\ unit{\isacharunderscore}{\kern0pt}{\isadigit{6}}{\isacharparenright}{\kern0pt}\isanewline
\isanewline
\ \ \isacommand{have}\isamarkupfalse%
\ unit{\isacharunderscore}{\kern0pt}{\isadigit{8}}{\isacharcolon}{\kern0pt}\ {\isachardoublequoteopen}{\isacharparenleft}{\kern0pt}{\isasymlambda}{\isacharunderscore}{\kern0pt}{\isachardot}{\kern0pt}\ {\isadigit{1}}{\isacharparenright}{\kern0pt}\ {\isasymin}\ O{\isacharbrackleft}{\kern0pt}{\isacharquery}{\kern0pt}F{\isacharbrackright}{\kern0pt}{\isacharparenleft}{\kern0pt}g{\isacharparenright}{\kern0pt}{\isachardoublequoteclose}\ \isanewline
\ \ \ \ \isacommand{apply}\isamarkupfalse%
\ {\isacharparenleft}{\kern0pt}simp\ add{\isacharcolon}{\kern0pt}g{\isacharunderscore}{\kern0pt}def{\isacharparenright}{\kern0pt}\isanewline
\ \ \ \ \isacommand{apply}\isamarkupfalse%
\ {\isacharparenleft}{\kern0pt}rule\ landau{\isacharunderscore}{\kern0pt}o{\isachardot}{\kern0pt}big{\isacharunderscore}{\kern0pt}mult{\isacharunderscore}{\kern0pt}{\isadigit{1}}{\isacharbrackleft}{\kern0pt}OF\ unit{\isacharunderscore}{\kern0pt}{\isadigit{4}}{\isacharbrackright}{\kern0pt}{\isacharparenright}{\kern0pt}\isanewline
\ \ \ \ \isacommand{by}\isamarkupfalse%
\ {\isacharparenleft}{\kern0pt}rule\ unit{\isacharunderscore}{\kern0pt}{\isadigit{7}}{\isacharparenright}{\kern0pt}\isanewline
\isanewline
\ \ \isacommand{have}\isamarkupfalse%
\ l{\isadigit{2}}{\isacharcolon}{\kern0pt}\ {\isachardoublequoteopen}{\isacharparenleft}{\kern0pt}{\isasymlambda}x{\isachardot}{\kern0pt}\ ln\ {\isacharparenleft}{\kern0pt}real\ {\isacharparenleft}{\kern0pt}nat\ {\isasymlceil}{\isacharminus}{\kern0pt}\ {\isacharparenleft}{\kern0pt}{\isadigit{1}}{\isadigit{8}}\ {\isacharasterisk}{\kern0pt}\ ln\ {\isacharparenleft}{\kern0pt}real{\isacharunderscore}{\kern0pt}of{\isacharunderscore}{\kern0pt}rat\ {\isacharparenleft}{\kern0pt}{\isasymepsilon}{\isacharunderscore}{\kern0pt}of\ x{\isacharparenright}{\kern0pt}{\isacharparenright}{\kern0pt}{\isacharparenright}{\kern0pt}{\isasymrceil}{\isacharparenright}{\kern0pt}\ {\isacharplus}{\kern0pt}\ {\isadigit{1}}{\isacharparenright}{\kern0pt}{\isacharparenright}{\kern0pt}\ {\isasymin}\ O{\isacharbrackleft}{\kern0pt}{\isacharquery}{\kern0pt}F{\isacharbrackright}{\kern0pt}{\isacharparenleft}{\kern0pt}g{\isacharparenright}{\kern0pt}{\isachardoublequoteclose}\ \isanewline
\ \ \ \ \isacommand{apply}\isamarkupfalse%
\ {\isacharparenleft}{\kern0pt}simp\ add{\isacharcolon}{\kern0pt}g{\isacharunderscore}{\kern0pt}def{\isacharparenright}{\kern0pt}\isanewline
\ \ \ \ \isacommand{apply}\isamarkupfalse%
\ {\isacharparenleft}{\kern0pt}rule\ landau{\isacharunderscore}{\kern0pt}o{\isachardot}{\kern0pt}big{\isacharunderscore}{\kern0pt}mult{\isacharunderscore}{\kern0pt}{\isadigit{1}}{\isacharparenright}{\kern0pt}\isanewline
\ \ \ \ \ \isacommand{apply}\isamarkupfalse%
\ {\isacharparenleft}{\kern0pt}rule\ landau{\isacharunderscore}{\kern0pt}ln{\isacharunderscore}{\kern0pt}{\isadigit{2}}{\isacharbrackleft}{\kern0pt}\isakeyword{where}\ a{\isacharequal}{\kern0pt}{\isachardoublequoteopen}{\isadigit{2}}{\isachardoublequoteclose}{\isacharbrackright}{\kern0pt}{\isacharcomma}{\kern0pt}\ simp{\isacharcomma}{\kern0pt}\ simp{\isacharparenright}{\kern0pt}\isanewline
\ \ \ \ \ \ \isacommand{apply}\isamarkupfalse%
\ {\isacharparenleft}{\kern0pt}rule\ eps{\isacharunderscore}{\kern0pt}inf{\isacharparenright}{\kern0pt}\isanewline
\ \ \ \ \isacommand{apply}\isamarkupfalse%
\ {\isacharparenleft}{\kern0pt}rule\ sum{\isacharunderscore}{\kern0pt}in{\isacharunderscore}{\kern0pt}bigo{\isacharparenright}{\kern0pt}\isanewline
\ \ \ \ \ \ \isacommand{apply}\isamarkupfalse%
\ {\isacharparenleft}{\kern0pt}rule\ landau{\isacharunderscore}{\kern0pt}nat{\isacharunderscore}{\kern0pt}ceil{\isacharbrackleft}{\kern0pt}OF\ unit{\isacharunderscore}{\kern0pt}{\isadigit{5}}{\isacharbrackright}{\kern0pt}{\isacharparenright}{\kern0pt}\isanewline
\ \ \ \ \isacommand{apply}\isamarkupfalse%
\ {\isacharparenleft}{\kern0pt}subst\ minus{\isacharunderscore}{\kern0pt}mult{\isacharunderscore}{\kern0pt}right{\isacharparenright}{\kern0pt}\isanewline
\ \ \ \ \ \ \isacommand{apply}\isamarkupfalse%
\ {\isacharparenleft}{\kern0pt}subst\ cmult{\isacharunderscore}{\kern0pt}in{\isacharunderscore}{\kern0pt}bigo{\isacharunderscore}{\kern0pt}iff{\isacharcomma}{\kern0pt}\ rule\ disjI{\isadigit{2}}{\isacharparenright}{\kern0pt}\isanewline
\ \ \ \ \ \ \isacommand{apply}\isamarkupfalse%
\ {\isacharparenleft}{\kern0pt}subst\ landau{\isacharunderscore}{\kern0pt}o{\isachardot}{\kern0pt}big{\isachardot}{\kern0pt}in{\isacharunderscore}{\kern0pt}cong{\isacharbrackleft}{\kern0pt}\isakeyword{where}\ f{\isacharequal}{\kern0pt}{\isachardoublequoteopen}{\isasymlambda}x{\isachardot}{\kern0pt}\ {\isacharminus}{\kern0pt}\ ln\ {\isacharparenleft}{\kern0pt}real{\isacharunderscore}{\kern0pt}of{\isacharunderscore}{\kern0pt}rat\ {\isacharparenleft}{\kern0pt}{\isasymepsilon}{\isacharunderscore}{\kern0pt}of\ x{\isacharparenright}{\kern0pt}{\isacharparenright}{\kern0pt}{\isachardoublequoteclose}\ \isakeyword{and}\ g{\isacharequal}{\kern0pt}{\isachardoublequoteopen}{\isasymlambda}x{\isachardot}{\kern0pt}\ ln\ {\isacharparenleft}{\kern0pt}{\isadigit{1}}\ {\isacharslash}{\kern0pt}\ real{\isacharunderscore}{\kern0pt}of{\isacharunderscore}{\kern0pt}rat\ {\isacharparenleft}{\kern0pt}{\isasymepsilon}{\isacharunderscore}{\kern0pt}of\ x{\isacharparenright}{\kern0pt}{\isacharparenright}{\kern0pt}{\isachardoublequoteclose}{\isacharbrackright}{\kern0pt}{\isacharparenright}{\kern0pt}\isanewline
\ \ \ \ \ \ \ \isacommand{apply}\isamarkupfalse%
\ {\isacharparenleft}{\kern0pt}rule\ eventually{\isacharunderscore}{\kern0pt}mono{\isacharbrackleft}{\kern0pt}OF\ zero{\isacharunderscore}{\kern0pt}less{\isacharunderscore}{\kern0pt}eps{\isacharbrackright}{\kern0pt}{\isacharcomma}{\kern0pt}\ simp\ add{\isacharcolon}{\kern0pt}ln{\isacharunderscore}{\kern0pt}div{\isacharparenright}{\kern0pt}\isanewline
\ \ \ \ \ \ \isacommand{apply}\isamarkupfalse%
\ {\isacharparenleft}{\kern0pt}rule\ landau{\isacharunderscore}{\kern0pt}ln{\isacharunderscore}{\kern0pt}{\isadigit{3}}{\isacharbrackleft}{\kern0pt}OF\ eps{\isacharunderscore}{\kern0pt}inf{\isacharbrackright}{\kern0pt}{\isacharcomma}{\kern0pt}\ simp{\isacharcomma}{\kern0pt}\ rule\ unit{\isacharunderscore}{\kern0pt}{\isadigit{5}}{\isacharparenright}{\kern0pt}\isanewline
\ \ \ \ \isacommand{by}\isamarkupfalse%
\ {\isacharparenleft}{\kern0pt}rule\ unit{\isacharunderscore}{\kern0pt}{\isadigit{7}}{\isacharparenright}{\kern0pt}\isanewline
\isanewline
\ \ \isacommand{have}\isamarkupfalse%
\ l{\isadigit{3}}{\isacharcolon}{\kern0pt}\ {\isachardoublequoteopen}{\isacharparenleft}{\kern0pt}{\isasymlambda}x{\isachardot}{\kern0pt}\ ln\ {\isacharparenleft}{\kern0pt}real\ {\isacharparenleft}{\kern0pt}nat\ {\isasymlceil}{\isadigit{8}}{\isadigit{0}}\ {\isacharslash}{\kern0pt}\ {\isacharparenleft}{\kern0pt}real{\isacharunderscore}{\kern0pt}of{\isacharunderscore}{\kern0pt}rat\ {\isacharparenleft}{\kern0pt}{\isasymdelta}{\isacharunderscore}{\kern0pt}of\ x{\isacharparenright}{\kern0pt}{\isacharparenright}{\kern0pt}\isactrlsup {\isadigit{2}}{\isasymrceil}{\isacharparenright}{\kern0pt}\ {\isacharplus}{\kern0pt}\ {\isadigit{1}}{\isacharparenright}{\kern0pt}{\isacharparenright}{\kern0pt}\ {\isasymin}\ O{\isacharbrackleft}{\kern0pt}{\isacharquery}{\kern0pt}F{\isacharbrackright}{\kern0pt}{\isacharparenleft}{\kern0pt}g{\isacharparenright}{\kern0pt}{\isachardoublequoteclose}\isanewline
\ \ \ \ \isacommand{apply}\isamarkupfalse%
\ {\isacharparenleft}{\kern0pt}simp\ add{\isacharcolon}{\kern0pt}g{\isacharunderscore}{\kern0pt}def{\isacharparenright}{\kern0pt}\isanewline
\ \ \ \ \isacommand{apply}\isamarkupfalse%
\ {\isacharparenleft}{\kern0pt}rule\ landau{\isacharunderscore}{\kern0pt}o{\isachardot}{\kern0pt}big{\isacharunderscore}{\kern0pt}mult{\isacharunderscore}{\kern0pt}{\isadigit{1}}{\isacharprime}{\kern0pt}{\isacharbrackleft}{\kern0pt}OF\ unit{\isacharunderscore}{\kern0pt}{\isadigit{4}}{\isacharbrackright}{\kern0pt}{\isacharparenright}{\kern0pt}\isanewline
\ \ \ \ \isacommand{apply}\isamarkupfalse%
\ {\isacharparenleft}{\kern0pt}rule\ landau{\isacharunderscore}{\kern0pt}sum{\isacharunderscore}{\kern0pt}{\isadigit{2}}{\isacharparenright}{\kern0pt}\isanewline
\ \ \ \ \ \ \isacommand{apply}\isamarkupfalse%
\ {\isacharparenleft}{\kern0pt}rule\ eventually{\isacharunderscore}{\kern0pt}ln{\isacharunderscore}{\kern0pt}ge{\isacharunderscore}{\kern0pt}iff{\isacharbrackleft}{\kern0pt}OF\ n{\isacharunderscore}{\kern0pt}inf{\isacharbrackright}{\kern0pt}{\isacharparenright}{\kern0pt}\isanewline
\ \ \ \ \isacommand{apply}\isamarkupfalse%
\ {\isacharparenleft}{\kern0pt}rule\ l{\isadigit{1}}{\isacharparenright}{\kern0pt}\isanewline
\ \ \ \ \isacommand{apply}\isamarkupfalse%
\ {\isacharparenleft}{\kern0pt}subst\ {\isacharparenleft}{\kern0pt}{\isadigit{3}}{\isacharparenright}{\kern0pt}\ div{\isacharunderscore}{\kern0pt}commute{\isacharparenright}{\kern0pt}\isanewline
\ \ \ \ \isacommand{apply}\isamarkupfalse%
\ {\isacharparenleft}{\kern0pt}rule\ landau{\isacharunderscore}{\kern0pt}o{\isachardot}{\kern0pt}big{\isacharunderscore}{\kern0pt}mult{\isacharunderscore}{\kern0pt}{\isadigit{1}}{\isacharparenright}{\kern0pt}\isanewline
\ \ \ \ \ \isacommand{apply}\isamarkupfalse%
\ {\isacharparenleft}{\kern0pt}rule\ landau{\isacharunderscore}{\kern0pt}ln{\isacharunderscore}{\kern0pt}{\isadigit{3}}{\isacharcomma}{\kern0pt}\ simp{\isacharparenright}{\kern0pt}\isanewline
\ \ \ \ \ \isacommand{apply}\isamarkupfalse%
\ {\isacharparenleft}{\kern0pt}rule\ sum{\isacharunderscore}{\kern0pt}in{\isacharunderscore}{\kern0pt}bigo{\isacharparenright}{\kern0pt}\isanewline
\ \ \ \ \ \ \isacommand{apply}\isamarkupfalse%
\ {\isacharparenleft}{\kern0pt}rule\ landau{\isacharunderscore}{\kern0pt}nat{\isacharunderscore}{\kern0pt}ceil{\isacharbrackleft}{\kern0pt}OF\ unit{\isacharunderscore}{\kern0pt}{\isadigit{1}}{\isacharbrackright}{\kern0pt}{\isacharparenright}{\kern0pt}\isanewline
\ \ \ \ \ \isacommand{apply}\isamarkupfalse%
\ {\isacharparenleft}{\kern0pt}rule\ landau{\isacharunderscore}{\kern0pt}const{\isacharunderscore}{\kern0pt}inv{\isacharcomma}{\kern0pt}\ simp{\isacharcomma}{\kern0pt}\ simp{\isacharcomma}{\kern0pt}\ rule\ unit{\isacharunderscore}{\kern0pt}{\isadigit{1}}{\isacharparenright}{\kern0pt}\isanewline
\ \ \ \ \isacommand{apply}\isamarkupfalse%
\ {\isacharparenleft}{\kern0pt}rule\ landau{\isacharunderscore}{\kern0pt}sum{\isacharunderscore}{\kern0pt}{\isadigit{2}}{\isacharparenright}{\kern0pt}\isanewline
\ \ \ \ \ \ \isacommand{apply}\isamarkupfalse%
\ {\isacharparenleft}{\kern0pt}rule\ eventually{\isacharunderscore}{\kern0pt}ln{\isacharunderscore}{\kern0pt}ge{\isacharunderscore}{\kern0pt}iff{\isacharbrackleft}{\kern0pt}OF\ eventually{\isacharunderscore}{\kern0pt}ln{\isacharunderscore}{\kern0pt}ge{\isacharunderscore}{\kern0pt}iff{\isacharbrackleft}{\kern0pt}OF\ n{\isacharunderscore}{\kern0pt}inf{\isacharbrackright}{\kern0pt}{\isacharbrackright}{\kern0pt}{\isacharparenright}{\kern0pt}\isanewline
\ \ \ \ \ \isacommand{apply}\isamarkupfalse%
\ {\isacharparenleft}{\kern0pt}rule\ eventually{\isacharunderscore}{\kern0pt}ln{\isacharunderscore}{\kern0pt}ge{\isacharunderscore}{\kern0pt}iff{\isacharbrackleft}{\kern0pt}OF\ delta{\isacharunderscore}{\kern0pt}inf{\isacharbrackright}{\kern0pt}{\isacharparenright}{\kern0pt}\isanewline
\ \ \ \ \isacommand{by}\isamarkupfalse%
\ {\isacharparenleft}{\kern0pt}rule\ unit{\isacharunderscore}{\kern0pt}{\isadigit{2}}{\isacharparenright}{\kern0pt}\isanewline
\isanewline
\ \ \isacommand{have}\isamarkupfalse%
\ unit{\isacharunderscore}{\kern0pt}{\isadigit{9}}{\isacharcolon}{\kern0pt}\ {\isachardoublequoteopen}{\isacharparenleft}{\kern0pt}{\isasymlambda}{\isacharunderscore}{\kern0pt}{\isachardot}{\kern0pt}\ {\isadigit{1}}{\isacharparenright}{\kern0pt}\ {\isasymin}\ O{\isacharbrackleft}{\kern0pt}{\isacharquery}{\kern0pt}F{\isacharbrackright}{\kern0pt}{\isacharparenleft}{\kern0pt}{\isasymlambda}x{\isachardot}{\kern0pt}\ ln\ {\isacharparenleft}{\kern0pt}real\ {\isacharparenleft}{\kern0pt}n{\isacharunderscore}{\kern0pt}of\ x{\isacharparenright}{\kern0pt}{\isacharparenright}{\kern0pt}{\isacharparenright}{\kern0pt}{\isachardoublequoteclose}\ \isanewline
\ \ \ \ \isacommand{apply}\isamarkupfalse%
\ {\isacharparenleft}{\kern0pt}rule\ landau{\isacharunderscore}{\kern0pt}o{\isachardot}{\kern0pt}big{\isacharunderscore}{\kern0pt}mono{\isacharcomma}{\kern0pt}\ simp{\isacharparenright}{\kern0pt}\isanewline
\ \ \ \ \isacommand{apply}\isamarkupfalse%
\ {\isacharparenleft}{\kern0pt}rule\ eventually{\isacharunderscore}{\kern0pt}mono{\isacharbrackleft}{\kern0pt}OF\ n{\isacharunderscore}{\kern0pt}inf{\isacharbrackleft}{\kern0pt}\isakeyword{where}\ c{\isacharequal}{\kern0pt}{\isachardoublequoteopen}exp\ {\isadigit{1}}{\isachardoublequoteclose}{\isacharbrackright}{\kern0pt}{\isacharbrackright}{\kern0pt}{\isacharparenright}{\kern0pt}\isanewline
\ \ \ \ \isacommand{by}\isamarkupfalse%
\ {\isacharparenleft}{\kern0pt}metis\ abs{\isacharunderscore}{\kern0pt}ge{\isacharunderscore}{\kern0pt}self\ less{\isacharunderscore}{\kern0pt}eq{\isacharunderscore}{\kern0pt}real{\isacharunderscore}{\kern0pt}def\ ln{\isacharunderscore}{\kern0pt}ge{\isacharunderscore}{\kern0pt}iff\ not{\isacharunderscore}{\kern0pt}exp{\isacharunderscore}{\kern0pt}le{\isacharunderscore}{\kern0pt}zero\ of{\isacharunderscore}{\kern0pt}nat{\isacharunderscore}{\kern0pt}{\isadigit{0}}{\isacharunderscore}{\kern0pt}le{\isacharunderscore}{\kern0pt}iff\ order{\isachardot}{\kern0pt}trans{\isacharparenright}{\kern0pt}\isanewline
\isanewline
\ \ \isacommand{have}\isamarkupfalse%
\ l{\isadigit{4}}{\isacharcolon}{\kern0pt}\ {\isachardoublequoteopen}{\isacharparenleft}{\kern0pt}{\isasymlambda}x{\isachardot}{\kern0pt}\ ln\ {\isacharparenleft}{\kern0pt}{\isadigit{1}}{\isadigit{0}}\ {\isacharplus}{\kern0pt}\ real\ {\isacharparenleft}{\kern0pt}n{\isacharunderscore}{\kern0pt}of\ x{\isacharparenright}{\kern0pt}{\isacharparenright}{\kern0pt}{\isacharparenright}{\kern0pt}\ {\isasymin}\ O{\isacharbrackleft}{\kern0pt}{\isacharquery}{\kern0pt}F{\isacharbrackright}{\kern0pt}{\isacharparenleft}{\kern0pt}{\isasymlambda}x{\isachardot}{\kern0pt}\ ln\ {\isacharparenleft}{\kern0pt}real\ {\isacharparenleft}{\kern0pt}n{\isacharunderscore}{\kern0pt}of\ x{\isacharparenright}{\kern0pt}{\isacharparenright}{\kern0pt}{\isacharparenright}{\kern0pt}{\isachardoublequoteclose}\isanewline
\ \ \ \ \isacommand{apply}\isamarkupfalse%
\ {\isacharparenleft}{\kern0pt}rule\ landau{\isacharunderscore}{\kern0pt}ln{\isacharunderscore}{\kern0pt}{\isadigit{2}}{\isacharbrackleft}{\kern0pt}\isakeyword{where}\ a{\isacharequal}{\kern0pt}{\isachardoublequoteopen}{\isadigit{2}}{\isachardoublequoteclose}{\isacharbrackright}{\kern0pt}{\isacharcomma}{\kern0pt}\ simp{\isacharcomma}{\kern0pt}\ simp{\isacharcomma}{\kern0pt}\ rule\ n{\isacharunderscore}{\kern0pt}inf{\isacharparenright}{\kern0pt}\isanewline
\ \ \ \ \isacommand{by}\isamarkupfalse%
\ {\isacharparenleft}{\kern0pt}rule\ sum{\isacharunderscore}{\kern0pt}in{\isacharunderscore}{\kern0pt}bigo{\isacharcomma}{\kern0pt}\ simp\ add{\isacharcolon}{\kern0pt}unit{\isacharunderscore}{\kern0pt}{\isadigit{3}}{\isacharcomma}{\kern0pt}\ simp{\isacharparenright}{\kern0pt}\isanewline
\isanewline
\ \ \isacommand{have}\isamarkupfalse%
\ l{\isadigit{5}}{\isacharcolon}{\kern0pt}\ {\isachardoublequoteopen}{\isacharparenleft}{\kern0pt}{\isasymlambda}x{\isachardot}{\kern0pt}\ ln\ {\isacharparenleft}{\kern0pt}real\ {\isacharparenleft}{\kern0pt}n{\isacharunderscore}{\kern0pt}of\ x{\isacharparenright}{\kern0pt}\ {\isacharplus}{\kern0pt}\ {\isadigit{1}}{\isadigit{0}}{\isacharparenright}{\kern0pt}{\isacharparenright}{\kern0pt}\ {\isasymin}\ O{\isacharbrackleft}{\kern0pt}{\isacharquery}{\kern0pt}F{\isacharbrackright}{\kern0pt}{\isacharparenleft}{\kern0pt}g{\isacharparenright}{\kern0pt}{\isachardoublequoteclose}\isanewline
\ \ \ \ \isacommand{apply}\isamarkupfalse%
\ {\isacharparenleft}{\kern0pt}simp\ add{\isacharcolon}{\kern0pt}g{\isacharunderscore}{\kern0pt}def{\isacharparenright}{\kern0pt}\isanewline
\ \ \ \ \isacommand{apply}\isamarkupfalse%
\ {\isacharparenleft}{\kern0pt}rule\ landau{\isacharunderscore}{\kern0pt}o{\isachardot}{\kern0pt}big{\isacharunderscore}{\kern0pt}mult{\isacharunderscore}{\kern0pt}{\isadigit{1}}{\isacharprime}{\kern0pt}{\isacharbrackleft}{\kern0pt}OF\ unit{\isacharunderscore}{\kern0pt}{\isadigit{4}}{\isacharbrackright}{\kern0pt}{\isacharparenright}{\kern0pt}\isanewline
\ \ \ \ \isacommand{apply}\isamarkupfalse%
\ {\isacharparenleft}{\kern0pt}rule\ landau{\isacharunderscore}{\kern0pt}sum{\isacharunderscore}{\kern0pt}{\isadigit{1}}{\isacharparenright}{\kern0pt}\isanewline
\ \ \ \ \ \ \isacommand{apply}\isamarkupfalse%
\ {\isacharparenleft}{\kern0pt}rule\ eventually{\isacharunderscore}{\kern0pt}ln{\isacharunderscore}{\kern0pt}ge{\isacharunderscore}{\kern0pt}iff{\isacharbrackleft}{\kern0pt}OF\ n{\isacharunderscore}{\kern0pt}inf{\isacharbrackright}{\kern0pt}{\isacharparenright}{\kern0pt}\isanewline
\ \ \ \ \ \isacommand{apply}\isamarkupfalse%
\ {\isacharparenleft}{\kern0pt}rule\ l{\isadigit{1}}{\isacharparenright}{\kern0pt}\isanewline
\ \ \ \ \isacommand{apply}\isamarkupfalse%
\ {\isacharparenleft}{\kern0pt}rule\ landau{\isacharunderscore}{\kern0pt}ln{\isacharunderscore}{\kern0pt}{\isadigit{2}}{\isacharbrackleft}{\kern0pt}\isakeyword{where}\ a{\isacharequal}{\kern0pt}{\isachardoublequoteopen}{\isadigit{2}}{\isachardoublequoteclose}{\isacharbrackright}{\kern0pt}{\isacharcomma}{\kern0pt}\ simp{\isacharcomma}{\kern0pt}\ simp{\isacharcomma}{\kern0pt}\ rule\ n{\isacharunderscore}{\kern0pt}inf{\isacharparenright}{\kern0pt}\isanewline
\ \ \ \ \isacommand{by}\isamarkupfalse%
\ {\isacharparenleft}{\kern0pt}rule\ sum{\isacharunderscore}{\kern0pt}in{\isacharunderscore}{\kern0pt}bigo{\isacharcomma}{\kern0pt}\ simp{\isacharcomma}{\kern0pt}\ simp\ add{\isacharcolon}{\kern0pt}unit{\isacharunderscore}{\kern0pt}{\isadigit{3}}{\isacharparenright}{\kern0pt}\isanewline
\ \ \isanewline
\ \ \isacommand{have}\isamarkupfalse%
\ l{\isadigit{6}}{\isacharcolon}{\kern0pt}\ {\isachardoublequoteopen}{\isacharparenleft}{\kern0pt}{\isasymlambda}x{\isachardot}{\kern0pt}\ log\ {\isadigit{2}}\ {\isacharparenleft}{\kern0pt}real\ {\isacharparenleft}{\kern0pt}nat\ {\isacharparenleft}{\kern0pt}{\isadigit{4}}\ {\isacharasterisk}{\kern0pt}\ {\isasymlceil}log\ {\isadigit{2}}\ {\isacharparenleft}{\kern0pt}{\isadigit{1}}\ {\isacharslash}{\kern0pt}\ real{\isacharunderscore}{\kern0pt}of{\isacharunderscore}{\kern0pt}rat\ {\isacharparenleft}{\kern0pt}{\isasymdelta}{\isacharunderscore}{\kern0pt}of\ x{\isacharparenright}{\kern0pt}{\isacharparenright}{\kern0pt}{\isasymrceil}\ {\isacharplus}{\kern0pt}\ {\isadigit{2}}{\isadigit{4}}{\isacharparenright}{\kern0pt}{\isacharparenright}{\kern0pt}\ {\isacharplus}{\kern0pt}\ {\isadigit{1}}{\isacharparenright}{\kern0pt}{\isacharparenright}{\kern0pt}\ {\isasymin}\ O{\isacharbrackleft}{\kern0pt}{\isacharquery}{\kern0pt}F{\isacharbrackright}{\kern0pt}{\isacharparenleft}{\kern0pt}g{\isacharparenright}{\kern0pt}{\isachardoublequoteclose}\isanewline
\ \ \ \ \isacommand{apply}\isamarkupfalse%
\ {\isacharparenleft}{\kern0pt}simp\ add{\isacharcolon}{\kern0pt}g{\isacharunderscore}{\kern0pt}def\ log{\isacharunderscore}{\kern0pt}def{\isacharcomma}{\kern0pt}\ rule\ landau{\isacharunderscore}{\kern0pt}o{\isachardot}{\kern0pt}big{\isacharunderscore}{\kern0pt}mult{\isacharunderscore}{\kern0pt}{\isadigit{1}}{\isacharprime}{\kern0pt}{\isacharbrackleft}{\kern0pt}OF\ unit{\isacharunderscore}{\kern0pt}{\isadigit{4}}{\isacharbrackright}{\kern0pt}{\isacharcomma}{\kern0pt}\ rule\ landau{\isacharunderscore}{\kern0pt}sum{\isacharunderscore}{\kern0pt}{\isadigit{2}}{\isacharparenright}{\kern0pt}\isanewline
\ \ \ \ \ \ \isacommand{apply}\isamarkupfalse%
\ {\isacharparenleft}{\kern0pt}rule\ eventually{\isacharunderscore}{\kern0pt}ln{\isacharunderscore}{\kern0pt}ge{\isacharunderscore}{\kern0pt}iff{\isacharbrackleft}{\kern0pt}OF\ n{\isacharunderscore}{\kern0pt}inf{\isacharbrackright}{\kern0pt}{\isacharparenright}{\kern0pt}\isanewline
\ \ \ \ \ \isacommand{apply}\isamarkupfalse%
\ {\isacharparenleft}{\kern0pt}rule\ l{\isadigit{1}}{\isacharparenright}{\kern0pt}\isanewline
\ \ \ \ \isacommand{apply}\isamarkupfalse%
\ {\isacharparenleft}{\kern0pt}subst\ {\isacharparenleft}{\kern0pt}{\isadigit{4}}{\isacharparenright}{\kern0pt}\ div{\isacharunderscore}{\kern0pt}commute{\isacharparenright}{\kern0pt}\isanewline
\ \ \ \ \isacommand{apply}\isamarkupfalse%
\ {\isacharparenleft}{\kern0pt}rule\ landau{\isacharunderscore}{\kern0pt}o{\isachardot}{\kern0pt}big{\isacharunderscore}{\kern0pt}mult{\isacharunderscore}{\kern0pt}{\isadigit{1}}{\isacharparenright}{\kern0pt}\isanewline
\ \ \ \ \ \isacommand{apply}\isamarkupfalse%
\ {\isacharparenleft}{\kern0pt}rule\ landau{\isacharunderscore}{\kern0pt}ln{\isacharunderscore}{\kern0pt}{\isadigit{3}}{\isacharcomma}{\kern0pt}\ simp{\isacharparenright}{\kern0pt}\isanewline
\ \ \ \ \ \isacommand{apply}\isamarkupfalse%
\ {\isacharparenleft}{\kern0pt}rule\ sum{\isacharunderscore}{\kern0pt}in{\isacharunderscore}{\kern0pt}bigo{\isacharparenright}{\kern0pt}\isanewline
\ \ \ \ \ \ \isacommand{apply}\isamarkupfalse%
\ {\isacharparenleft}{\kern0pt}rule\ landau{\isacharunderscore}{\kern0pt}real{\isacharunderscore}{\kern0pt}nat{\isacharcomma}{\kern0pt}\ simp{\isacharparenright}{\kern0pt}\isanewline
\ \ \ \ \ \ \isacommand{apply}\isamarkupfalse%
\ {\isacharparenleft}{\kern0pt}rule\ sum{\isacharunderscore}{\kern0pt}in{\isacharunderscore}{\kern0pt}bigo{\isacharparenright}{\kern0pt}\isanewline
\ \ \ \ \ \ \ \isacommand{apply}\isamarkupfalse%
\ {\isacharparenleft}{\kern0pt}simp{\isacharcomma}{\kern0pt}\ rule\ landau{\isacharunderscore}{\kern0pt}ceil{\isacharbrackleft}{\kern0pt}OF\ unit{\isacharunderscore}{\kern0pt}{\isadigit{1}}{\isacharbrackright}{\kern0pt}{\isacharcomma}{\kern0pt}\ simp{\isacharcomma}{\kern0pt}\ rule\ landau{\isacharunderscore}{\kern0pt}ln{\isacharunderscore}{\kern0pt}{\isadigit{3}}{\isacharbrackleft}{\kern0pt}OF\ delta{\isacharunderscore}{\kern0pt}inf{\isacharbrackright}{\kern0pt}{\isacharparenright}{\kern0pt}\isanewline
\ \ \ \ \ \ \ \isacommand{apply}\isamarkupfalse%
\ {\isacharparenleft}{\kern0pt}rule\ landau{\isacharunderscore}{\kern0pt}o{\isachardot}{\kern0pt}big{\isacharunderscore}{\kern0pt}mono{\isacharparenright}{\kern0pt}\isanewline
\ \ \ \ \ \ \ \isacommand{apply}\isamarkupfalse%
\ {\isacharparenleft}{\kern0pt}rule\ eventually{\isacharunderscore}{\kern0pt}mono{\isacharbrackleft}{\kern0pt}OF\ eventually{\isacharunderscore}{\kern0pt}conj{\isacharbrackleft}{\kern0pt}OF\ delta{\isacharunderscore}{\kern0pt}inf{\isacharbrackleft}{\kern0pt}\isakeyword{where}\ c{\isacharequal}{\kern0pt}{\isachardoublequoteopen}{\isadigit{1}}{\isachardoublequoteclose}{\isacharbrackright}{\kern0pt}\ zero{\isacharunderscore}{\kern0pt}less{\isacharunderscore}{\kern0pt}delta{\isacharbrackright}{\kern0pt}{\isacharbrackright}{\kern0pt}{\isacharparenright}{\kern0pt}\isanewline
\ \ \ \ \ \ \ \isacommand{apply}\isamarkupfalse%
\ {\isacharparenleft}{\kern0pt}simp{\isacharcomma}{\kern0pt}\ metis\ pos{\isadigit{2}}\ power{\isacharunderscore}{\kern0pt}one{\isacharunderscore}{\kern0pt}over\ self{\isacharunderscore}{\kern0pt}le{\isacharunderscore}{\kern0pt}power{\isacharparenright}{\kern0pt}\isanewline
\ \ \ \ \ \ \isacommand{apply}\isamarkupfalse%
\ {\isacharparenleft}{\kern0pt}simp\ add{\isacharcolon}{\kern0pt}unit{\isacharunderscore}{\kern0pt}{\isadigit{1}}{\isacharparenright}{\kern0pt}\isanewline
\ \ \ \ \ \isacommand{apply}\isamarkupfalse%
\ {\isacharparenleft}{\kern0pt}simp\ add{\isacharcolon}{\kern0pt}unit{\isacharunderscore}{\kern0pt}{\isadigit{1}}{\isacharparenright}{\kern0pt}\isanewline
\ \ \ \ \isacommand{apply}\isamarkupfalse%
\ {\isacharparenleft}{\kern0pt}rule\ landau{\isacharunderscore}{\kern0pt}sum{\isacharunderscore}{\kern0pt}{\isadigit{2}}{\isacharparenright}{\kern0pt}\isanewline
\ \ \ \ \ \ \isacommand{apply}\isamarkupfalse%
\ {\isacharparenleft}{\kern0pt}rule\ eventually{\isacharunderscore}{\kern0pt}ln{\isacharunderscore}{\kern0pt}ge{\isacharunderscore}{\kern0pt}iff{\isacharcomma}{\kern0pt}\ rule\ eventually{\isacharunderscore}{\kern0pt}ln{\isacharunderscore}{\kern0pt}ge{\isacharunderscore}{\kern0pt}iff{\isacharbrackleft}{\kern0pt}OF\ n{\isacharunderscore}{\kern0pt}inf{\isacharbrackright}{\kern0pt}{\isacharparenright}{\kern0pt}\isanewline
\ \ \ \ \ \isacommand{apply}\isamarkupfalse%
\ {\isacharparenleft}{\kern0pt}rule\ eventually{\isacharunderscore}{\kern0pt}ln{\isacharunderscore}{\kern0pt}ge{\isacharunderscore}{\kern0pt}iff{\isacharbrackleft}{\kern0pt}OF\ delta{\isacharunderscore}{\kern0pt}inf{\isacharbrackright}{\kern0pt}{\isacharparenright}{\kern0pt}\isanewline
\ \ \ \ \isacommand{by}\isamarkupfalse%
\ {\isacharparenleft}{\kern0pt}rule\ unit{\isacharunderscore}{\kern0pt}{\isadigit{2}}{\isacharparenright}{\kern0pt}\isanewline
\isanewline
\ \ \isacommand{have}\isamarkupfalse%
\ l{\isadigit{7}}{\isacharcolon}{\kern0pt}\ {\isachardoublequoteopen}{\isacharparenleft}{\kern0pt}{\isasymlambda}x{\isachardot}{\kern0pt}\ real\ {\isacharparenleft}{\kern0pt}nat\ {\isasymlceil}{\isacharminus}{\kern0pt}\ {\isacharparenleft}{\kern0pt}{\isadigit{1}}{\isadigit{8}}\ {\isacharasterisk}{\kern0pt}\ ln\ {\isacharparenleft}{\kern0pt}real{\isacharunderscore}{\kern0pt}of{\isacharunderscore}{\kern0pt}rat\ {\isacharparenleft}{\kern0pt}{\isasymepsilon}{\isacharunderscore}{\kern0pt}of\ x{\isacharparenright}{\kern0pt}{\isacharparenright}{\kern0pt}{\isacharparenright}{\kern0pt}{\isasymrceil}{\isacharparenright}{\kern0pt}{\isacharparenright}{\kern0pt}\ {\isasymin}\ O{\isacharbrackleft}{\kern0pt}{\isacharquery}{\kern0pt}F{\isacharbrackright}{\kern0pt}{\isacharparenleft}{\kern0pt}{\isasymlambda}x{\isachardot}{\kern0pt}\ ln\ {\isacharparenleft}{\kern0pt}{\isadigit{1}}\ {\isacharslash}{\kern0pt}\ real{\isacharunderscore}{\kern0pt}of{\isacharunderscore}{\kern0pt}rat\ {\isacharparenleft}{\kern0pt}{\isasymepsilon}{\isacharunderscore}{\kern0pt}of\ x{\isacharparenright}{\kern0pt}{\isacharparenright}{\kern0pt}{\isacharparenright}{\kern0pt}{\isachardoublequoteclose}\isanewline
\ \ \ \ \isacommand{apply}\isamarkupfalse%
\ {\isacharparenleft}{\kern0pt}rule\ landau{\isacharunderscore}{\kern0pt}nat{\isacharunderscore}{\kern0pt}ceil{\isacharcomma}{\kern0pt}\ rule\ unit{\isacharunderscore}{\kern0pt}{\isadigit{4}}{\isacharparenright}{\kern0pt}\isanewline
\ \ \ \ \isacommand{apply}\isamarkupfalse%
\ {\isacharparenleft}{\kern0pt}subst\ minus{\isacharunderscore}{\kern0pt}mult{\isacharunderscore}{\kern0pt}right{\isacharparenright}{\kern0pt}\isanewline
\ \ \ \ \isacommand{apply}\isamarkupfalse%
\ {\isacharparenleft}{\kern0pt}subst\ cmult{\isacharunderscore}{\kern0pt}in{\isacharunderscore}{\kern0pt}bigo{\isacharunderscore}{\kern0pt}iff{\isacharcomma}{\kern0pt}\ rule\ disjI{\isadigit{2}}{\isacharparenright}{\kern0pt}\isanewline
\ \ \ \ \isacommand{apply}\isamarkupfalse%
\ {\isacharparenleft}{\kern0pt}rule\ landau{\isacharunderscore}{\kern0pt}o{\isachardot}{\kern0pt}big{\isacharunderscore}{\kern0pt}mono{\isacharparenright}{\kern0pt}\isanewline
\ \ \ \ \isacommand{apply}\isamarkupfalse%
\ {\isacharparenleft}{\kern0pt}rule\ eventually{\isacharunderscore}{\kern0pt}mono{\isacharbrackleft}{\kern0pt}OF\ zero{\isacharunderscore}{\kern0pt}less{\isacharunderscore}{\kern0pt}eps{\isacharbrackright}{\kern0pt}{\isacharparenright}{\kern0pt}\isanewline
\ \ \ \ \isacommand{by}\isamarkupfalse%
\ {\isacharparenleft}{\kern0pt}subst\ ln{\isacharunderscore}{\kern0pt}div{\isacharcomma}{\kern0pt}\ simp{\isacharcomma}{\kern0pt}\ simp{\isacharcomma}{\kern0pt}\ simp{\isacharparenright}{\kern0pt}\isanewline
\isanewline
\ \ \isacommand{have}\isamarkupfalse%
\ l{\isadigit{8}}{\isacharcolon}{\kern0pt}\ {\isachardoublequoteopen}{\isacharparenleft}{\kern0pt}{\isasymlambda}x{\isachardot}{\kern0pt}\ real\ {\isacharparenleft}{\kern0pt}nat\ {\isasymlceil}{\isadigit{8}}{\isadigit{0}}\ {\isacharslash}{\kern0pt}\ {\isacharparenleft}{\kern0pt}real{\isacharunderscore}{\kern0pt}of{\isacharunderscore}{\kern0pt}rat\ {\isacharparenleft}{\kern0pt}{\isasymdelta}{\isacharunderscore}{\kern0pt}of\ x{\isacharparenright}{\kern0pt}{\isacharparenright}{\kern0pt}\isactrlsup {\isadigit{2}}{\isasymrceil}{\isacharparenright}{\kern0pt}\ {\isacharasterisk}{\kern0pt}\ \isanewline
\ \ \ \ {\isacharparenleft}{\kern0pt}{\isadigit{1}}{\isadigit{1}}\ {\isacharplus}{\kern0pt}\ {\isadigit{4}}\ {\isacharasterisk}{\kern0pt}\ real\ {\isacharparenleft}{\kern0pt}nat\ {\isacharparenleft}{\kern0pt}{\isadigit{4}}\ {\isacharasterisk}{\kern0pt}\ {\isasymlceil}log\ {\isadigit{2}}\ {\isacharparenleft}{\kern0pt}{\isadigit{1}}\ {\isacharslash}{\kern0pt}\ real{\isacharunderscore}{\kern0pt}of{\isacharunderscore}{\kern0pt}rat\ {\isacharparenleft}{\kern0pt}{\isasymdelta}{\isacharunderscore}{\kern0pt}of\ x{\isacharparenright}{\kern0pt}{\isacharparenright}{\kern0pt}{\isasymrceil}\ {\isacharplus}{\kern0pt}\ {\isadigit{2}}{\isadigit{4}}{\isacharparenright}{\kern0pt}{\isacharparenright}{\kern0pt}\ {\isacharplus}{\kern0pt}\ \isanewline
\ \ \ \ {\isadigit{2}}\ {\isacharasterisk}{\kern0pt}\ log\ {\isadigit{2}}\ {\isacharparenleft}{\kern0pt}log\ {\isadigit{2}}\ {\isacharparenleft}{\kern0pt}real\ {\isacharparenleft}{\kern0pt}n{\isacharunderscore}{\kern0pt}of\ x{\isacharparenright}{\kern0pt}\ {\isacharplus}{\kern0pt}\ {\isadigit{9}}{\isacharparenright}{\kern0pt}{\isacharparenright}{\kern0pt}{\isacharparenright}{\kern0pt}{\isacharparenright}{\kern0pt}\isanewline
\ \ \ \ {\isasymin}\ O{\isacharbrackleft}{\kern0pt}{\isacharquery}{\kern0pt}F{\isacharbrackright}{\kern0pt}{\isacharparenleft}{\kern0pt}{\isasymlambda}x{\isachardot}{\kern0pt}\ {\isacharparenleft}{\kern0pt}ln\ {\isacharparenleft}{\kern0pt}ln\ {\isacharparenleft}{\kern0pt}real\ {\isacharparenleft}{\kern0pt}n{\isacharunderscore}{\kern0pt}of\ x{\isacharparenright}{\kern0pt}{\isacharparenright}{\kern0pt}{\isacharparenright}{\kern0pt}\ {\isacharplus}{\kern0pt}\ ln\ {\isacharparenleft}{\kern0pt}{\isadigit{1}}\ {\isacharslash}{\kern0pt}\ real{\isacharunderscore}{\kern0pt}of{\isacharunderscore}{\kern0pt}rat\ {\isacharparenleft}{\kern0pt}{\isasymdelta}{\isacharunderscore}{\kern0pt}of\ x{\isacharparenright}{\kern0pt}{\isacharparenright}{\kern0pt}{\isacharparenright}{\kern0pt}\ {\isacharslash}{\kern0pt}\ {\isacharparenleft}{\kern0pt}real{\isacharunderscore}{\kern0pt}of{\isacharunderscore}{\kern0pt}rat\ {\isacharparenleft}{\kern0pt}{\isasymdelta}{\isacharunderscore}{\kern0pt}of\ x{\isacharparenright}{\kern0pt}{\isacharparenright}{\kern0pt}\isactrlsup {\isadigit{2}}{\isacharparenright}{\kern0pt}{\isachardoublequoteclose}\isanewline
\ \ \ \ \isacommand{apply}\isamarkupfalse%
\ {\isacharparenleft}{\kern0pt}subst\ {\isacharparenleft}{\kern0pt}{\isadigit{4}}{\isacharparenright}{\kern0pt}\ div{\isacharunderscore}{\kern0pt}commute{\isacharparenright}{\kern0pt}\isanewline
\ \ \ \ \isacommand{apply}\isamarkupfalse%
\ {\isacharparenleft}{\kern0pt}rule\ landau{\isacharunderscore}{\kern0pt}o{\isachardot}{\kern0pt}mult{\isacharparenright}{\kern0pt}\isanewline
\ \ \ \ \ \isacommand{apply}\isamarkupfalse%
\ {\isacharparenleft}{\kern0pt}rule\ landau{\isacharunderscore}{\kern0pt}nat{\isacharunderscore}{\kern0pt}ceil{\isacharbrackleft}{\kern0pt}OF\ unit{\isacharunderscore}{\kern0pt}{\isadigit{1}}{\isacharbrackright}{\kern0pt}{\isacharcomma}{\kern0pt}\ rule\ landau{\isacharunderscore}{\kern0pt}const{\isacharunderscore}{\kern0pt}inv{\isacharcomma}{\kern0pt}\ simp{\isacharcomma}{\kern0pt}\ simp{\isacharparenright}{\kern0pt}\isanewline
\ \ \ \ \isacommand{apply}\isamarkupfalse%
\ {\isacharparenleft}{\kern0pt}subst\ {\isacharparenleft}{\kern0pt}{\isadigit{3}}{\isacharparenright}{\kern0pt}\ add{\isachardot}{\kern0pt}commute{\isacharparenright}{\kern0pt}\isanewline
\ \ \ \ \isacommand{apply}\isamarkupfalse%
\ {\isacharparenleft}{\kern0pt}rule\ landau{\isacharunderscore}{\kern0pt}sum{\isacharparenright}{\kern0pt}\isanewline
\ \ \ \ \ \ \ \isacommand{apply}\isamarkupfalse%
\ {\isacharparenleft}{\kern0pt}rule\ eventually{\isacharunderscore}{\kern0pt}ln{\isacharunderscore}{\kern0pt}ge{\isacharunderscore}{\kern0pt}iff{\isacharcomma}{\kern0pt}\ rule\ eventually{\isacharunderscore}{\kern0pt}ln{\isacharunderscore}{\kern0pt}ge{\isacharunderscore}{\kern0pt}iff{\isacharcomma}{\kern0pt}\ rule\ n{\isacharunderscore}{\kern0pt}inf{\isacharparenright}{\kern0pt}\isanewline
\ \ \ \ \ \ \isacommand{apply}\isamarkupfalse%
\ {\isacharparenleft}{\kern0pt}rule\ eventually{\isacharunderscore}{\kern0pt}ln{\isacharunderscore}{\kern0pt}ge{\isacharunderscore}{\kern0pt}iff{\isacharcomma}{\kern0pt}\ rule\ delta{\isacharunderscore}{\kern0pt}inf{\isacharcomma}{\kern0pt}\ simp\ add{\isacharcolon}{\kern0pt}log{\isacharunderscore}{\kern0pt}def{\isacharparenright}{\kern0pt}\isanewline
\ \ \ \ \ \isacommand{apply}\isamarkupfalse%
\ {\isacharparenleft}{\kern0pt}rule\ landau{\isacharunderscore}{\kern0pt}ln{\isacharunderscore}{\kern0pt}{\isadigit{2}}{\isacharbrackleft}{\kern0pt}\isakeyword{where}\ a{\isacharequal}{\kern0pt}{\isachardoublequoteopen}{\isadigit{2}}{\isachardoublequoteclose}{\isacharbrackright}{\kern0pt}{\isacharcomma}{\kern0pt}\ simp{\isacharparenright}{\kern0pt}\isanewline
\ \ \ \ \ \ \ \isacommand{apply}\isamarkupfalse%
\ {\isacharparenleft}{\kern0pt}subst\ pos{\isacharunderscore}{\kern0pt}le{\isacharunderscore}{\kern0pt}divide{\isacharunderscore}{\kern0pt}eq{\isacharcomma}{\kern0pt}\ simp{\isacharcomma}{\kern0pt}\ simp{\isacharparenright}{\kern0pt}\isanewline
\ \ \ \ \ \ \isacommand{apply}\isamarkupfalse%
\ {\isacharparenleft}{\kern0pt}rule\ eventually{\isacharunderscore}{\kern0pt}mono{\isacharbrackleft}{\kern0pt}OF\ n{\isacharunderscore}{\kern0pt}inf{\isacharbrackleft}{\kern0pt}\isakeyword{where}\ c{\isacharequal}{\kern0pt}{\isachardoublequoteopen}exp\ {\isadigit{2}}{\isachardoublequoteclose}{\isacharbrackright}{\kern0pt}{\isacharbrackright}{\kern0pt}{\isacharparenright}{\kern0pt}\isanewline
\ \ \ \ \ \ \isacommand{apply}\isamarkupfalse%
\ {\isacharparenleft}{\kern0pt}subst\ ln{\isacharunderscore}{\kern0pt}ge{\isacharunderscore}{\kern0pt}iff{\isacharcomma}{\kern0pt}\ metis\ less{\isacharunderscore}{\kern0pt}eq{\isacharunderscore}{\kern0pt}real{\isacharunderscore}{\kern0pt}def\ not{\isacharunderscore}{\kern0pt}exp{\isacharunderscore}{\kern0pt}le{\isacharunderscore}{\kern0pt}zero\ of{\isacharunderscore}{\kern0pt}nat{\isacharunderscore}{\kern0pt}{\isadigit{0}}{\isacharunderscore}{\kern0pt}le{\isacharunderscore}{\kern0pt}iff{\isacharparenright}{\kern0pt}\isanewline
\ \ \ \ \ \ \isacommand{apply}\isamarkupfalse%
\ simp\isanewline
\ \ \ \ \ \isacommand{apply}\isamarkupfalse%
\ {\isacharparenleft}{\kern0pt}simp{\isacharcomma}{\kern0pt}\ rule\ landau{\isacharunderscore}{\kern0pt}ln{\isacharunderscore}{\kern0pt}{\isadigit{2}}{\isacharbrackleft}{\kern0pt}\isakeyword{where}\ a{\isacharequal}{\kern0pt}{\isachardoublequoteopen}{\isadigit{2}}{\isachardoublequoteclose}{\isacharbrackright}{\kern0pt}{\isacharcomma}{\kern0pt}\ simp{\isacharcomma}{\kern0pt}\ simp{\isacharcomma}{\kern0pt}\ rule\ n{\isacharunderscore}{\kern0pt}inf{\isacharparenright}{\kern0pt}\isanewline
\ \ \ \ \ \isacommand{apply}\isamarkupfalse%
\ {\isacharparenleft}{\kern0pt}rule\ sum{\isacharunderscore}{\kern0pt}in{\isacharunderscore}{\kern0pt}bigo{\isacharcomma}{\kern0pt}\ simp{\isacharcomma}{\kern0pt}\ simp\ add{\isacharcolon}{\kern0pt}unit{\isacharunderscore}{\kern0pt}{\isadigit{3}}{\isacharparenright}{\kern0pt}\isanewline
\ \ \ \ \isacommand{apply}\isamarkupfalse%
\ {\isacharparenleft}{\kern0pt}rule\ sum{\isacharunderscore}{\kern0pt}in{\isacharunderscore}{\kern0pt}bigo{\isacharcomma}{\kern0pt}\ simp\ add{\isacharcolon}{\kern0pt}unit{\isacharunderscore}{\kern0pt}{\isadigit{2}}{\isacharparenright}{\kern0pt}\isanewline
\ \ \ \ \isacommand{apply}\isamarkupfalse%
\ {\isacharparenleft}{\kern0pt}simp{\isacharcomma}{\kern0pt}\ rule\ landau{\isacharunderscore}{\kern0pt}real{\isacharunderscore}{\kern0pt}nat{\isacharcomma}{\kern0pt}\ simp{\isacharparenright}{\kern0pt}\isanewline
\ \ \ \ \isacommand{apply}\isamarkupfalse%
\ {\isacharparenleft}{\kern0pt}rule\ sum{\isacharunderscore}{\kern0pt}in{\isacharunderscore}{\kern0pt}bigo{\isacharcomma}{\kern0pt}\ simp{\isacharparenright}{\kern0pt}\isanewline
\ \ \ \ \isacommand{by}\isamarkupfalse%
\ {\isacharparenleft}{\kern0pt}rule\ landau{\isacharunderscore}{\kern0pt}ceil{\isacharbrackleft}{\kern0pt}OF\ unit{\isacharunderscore}{\kern0pt}{\isadigit{2}}{\isacharbrackright}{\kern0pt}{\isacharcomma}{\kern0pt}\ simp\ add{\isacharcolon}{\kern0pt}log{\isacharunderscore}{\kern0pt}def{\isacharcomma}{\kern0pt}\ simp\ add{\isacharcolon}{\kern0pt}unit{\isacharunderscore}{\kern0pt}{\isadigit{2}}{\isacharparenright}{\kern0pt}\isanewline
\isanewline
\ \ \isacommand{have}\isamarkupfalse%
\ {\isachardoublequoteopen}f{\isadigit{0}}{\isacharunderscore}{\kern0pt}space{\isacharunderscore}{\kern0pt}usage\ {\isacharequal}{\kern0pt}\ {\isacharparenleft}{\kern0pt}{\isasymlambda}x{\isachardot}{\kern0pt}\ f{\isadigit{0}}{\isacharunderscore}{\kern0pt}space{\isacharunderscore}{\kern0pt}usage\ {\isacharparenleft}{\kern0pt}n{\isacharunderscore}{\kern0pt}of\ x{\isacharcomma}{\kern0pt}\ {\isasymepsilon}{\isacharunderscore}{\kern0pt}of\ x{\isacharcomma}{\kern0pt}\ {\isasymdelta}{\isacharunderscore}{\kern0pt}of\ x{\isacharparenright}{\kern0pt}{\isacharparenright}{\kern0pt}{\isachardoublequoteclose}\isanewline
\ \ \ \ \isacommand{apply}\isamarkupfalse%
\ {\isacharparenleft}{\kern0pt}rule\ ext{\isacharparenright}{\kern0pt}\isanewline
\ \ \ \ \isacommand{by}\isamarkupfalse%
\ {\isacharparenleft}{\kern0pt}simp\ add{\isacharcolon}{\kern0pt}case{\isacharunderscore}{\kern0pt}prod{\isacharunderscore}{\kern0pt}beta{\isacharprime}{\kern0pt}\ n{\isacharunderscore}{\kern0pt}of{\isacharunderscore}{\kern0pt}def\ {\isasymepsilon}{\isacharunderscore}{\kern0pt}of{\isacharunderscore}{\kern0pt}def\ {\isasymdelta}{\isacharunderscore}{\kern0pt}of{\isacharunderscore}{\kern0pt}def{\isacharparenright}{\kern0pt}\isanewline
\ \ \isacommand{also}\isamarkupfalse%
\ \isacommand{have}\isamarkupfalse%
\ {\isachardoublequoteopen}{\isachardot}{\kern0pt}{\isachardot}{\kern0pt}{\isachardot}{\kern0pt}\ {\isasymin}\ \ O{\isacharbrackleft}{\kern0pt}{\isacharquery}{\kern0pt}F{\isacharbrackright}{\kern0pt}{\isacharparenleft}{\kern0pt}g{\isacharparenright}{\kern0pt}{\isachardoublequoteclose}\isanewline
\ \ \ \ \isacommand{apply}\isamarkupfalse%
\ {\isacharparenleft}{\kern0pt}simp\ add{\isacharcolon}{\kern0pt}Let{\isacharunderscore}{\kern0pt}def{\isacharparenright}{\kern0pt}\isanewline
\ \ \ \ \isacommand{apply}\isamarkupfalse%
\ {\isacharparenleft}{\kern0pt}rule\ sum{\isacharunderscore}{\kern0pt}in{\isacharunderscore}{\kern0pt}bigo{\isacharunderscore}{\kern0pt}r{\isacharparenright}{\kern0pt}\ \isanewline
\ \ \ \ \ \isacommand{apply}\isamarkupfalse%
\ {\isacharparenleft}{\kern0pt}simp\ add{\isacharcolon}{\kern0pt}g{\isacharunderscore}{\kern0pt}def{\isacharparenright}{\kern0pt}\isanewline
\ \ \ \ \ \isacommand{apply}\isamarkupfalse%
\ {\isacharparenleft}{\kern0pt}rule\ landau{\isacharunderscore}{\kern0pt}o{\isachardot}{\kern0pt}mult{\isacharcomma}{\kern0pt}\ simp\ add{\isacharcolon}{\kern0pt}l{\isadigit{7}}{\isacharparenright}{\kern0pt}\isanewline
\ \ \ \ \ \isacommand{apply}\isamarkupfalse%
\ {\isacharparenleft}{\kern0pt}rule\ landau{\isacharunderscore}{\kern0pt}sum{\isacharparenright}{\kern0pt}\isanewline
\ \ \ \ \ \ \ \ \isacommand{apply}\isamarkupfalse%
\ {\isacharparenleft}{\kern0pt}rule\ eventually{\isacharunderscore}{\kern0pt}ln{\isacharunderscore}{\kern0pt}ge{\isacharunderscore}{\kern0pt}iff{\isacharbrackleft}{\kern0pt}OF\ n{\isacharunderscore}{\kern0pt}inf{\isacharbrackright}{\kern0pt}{\isacharparenright}{\kern0pt}\isanewline
\ \ \ \ \ \ \ \isacommand{apply}\isamarkupfalse%
\ {\isacharparenleft}{\kern0pt}rule\ l{\isadigit{1}}{\isacharparenright}{\kern0pt}\isanewline
\ \ \ \ \ \ \isacommand{apply}\isamarkupfalse%
\ {\isacharparenleft}{\kern0pt}rule\ sum{\isacharunderscore}{\kern0pt}in{\isacharunderscore}{\kern0pt}bigo{\isacharunderscore}{\kern0pt}r{\isacharcomma}{\kern0pt}\ simp\ add{\isacharcolon}{\kern0pt}log{\isacharunderscore}{\kern0pt}def\ l{\isadigit{4}}{\isacharcomma}{\kern0pt}\ simp\ add{\isacharcolon}{\kern0pt}unit{\isacharunderscore}{\kern0pt}{\isadigit{9}}{\isacharparenright}{\kern0pt}\isanewline
\ \ \ \ \ \isacommand{apply}\isamarkupfalse%
\ {\isacharparenleft}{\kern0pt}simp\ add{\isacharcolon}{\kern0pt}l{\isadigit{8}}{\isacharparenright}{\kern0pt}\isanewline
\ \ \ \ \isacommand{apply}\isamarkupfalse%
\ {\isacharparenleft}{\kern0pt}rule\ sum{\isacharunderscore}{\kern0pt}in{\isacharunderscore}{\kern0pt}bigo{\isacharunderscore}{\kern0pt}r{\isacharcomma}{\kern0pt}\ simp\ add{\isacharcolon}{\kern0pt}l{\isadigit{6}}{\isacharparenright}{\kern0pt}\isanewline
\ \ \ \ \isacommand{apply}\isamarkupfalse%
\ {\isacharparenleft}{\kern0pt}rule\ sum{\isacharunderscore}{\kern0pt}in{\isacharunderscore}{\kern0pt}bigo{\isacharunderscore}{\kern0pt}r{\isacharcomma}{\kern0pt}\ simp\ add{\isacharcolon}{\kern0pt}log{\isacharunderscore}{\kern0pt}def\ l{\isadigit{5}}{\isacharparenright}{\kern0pt}\isanewline
\ \ \ \ \isacommand{apply}\isamarkupfalse%
\ {\isacharparenleft}{\kern0pt}rule\ sum{\isacharunderscore}{\kern0pt}in{\isacharunderscore}{\kern0pt}bigo{\isacharunderscore}{\kern0pt}r{\isacharcomma}{\kern0pt}\ simp\ add{\isacharcolon}{\kern0pt}log{\isacharunderscore}{\kern0pt}def\ l{\isadigit{3}}{\isacharparenright}{\kern0pt}\isanewline
\ \ \ \ \isacommand{apply}\isamarkupfalse%
\ {\isacharparenleft}{\kern0pt}rule\ sum{\isacharunderscore}{\kern0pt}in{\isacharunderscore}{\kern0pt}bigo{\isacharunderscore}{\kern0pt}r{\isacharcomma}{\kern0pt}\ simp\ add{\isacharcolon}{\kern0pt}log{\isacharunderscore}{\kern0pt}def\ l{\isadigit{2}}{\isacharparenright}{\kern0pt}\isanewline
\ \ \ \ \isacommand{by}\isamarkupfalse%
\ {\isacharparenleft}{\kern0pt}simp\ add{\isacharcolon}{\kern0pt}unit{\isacharunderscore}{\kern0pt}{\isadigit{8}}{\isacharparenright}{\kern0pt}\isanewline
\ \ \isacommand{also}\isamarkupfalse%
\ \isacommand{have}\isamarkupfalse%
\ {\isachardoublequoteopen}{\isachardot}{\kern0pt}{\isachardot}{\kern0pt}{\isachardot}{\kern0pt}\ {\isacharequal}{\kern0pt}\ O{\isacharbrackleft}{\kern0pt}{\isacharquery}{\kern0pt}F{\isacharbrackright}{\kern0pt}{\isacharparenleft}{\kern0pt}{\isacharquery}{\kern0pt}rhs{\isacharparenright}{\kern0pt}{\isachardoublequoteclose}\isanewline
\ \ \ \ \isacommand{apply}\isamarkupfalse%
\ {\isacharparenleft}{\kern0pt}rule\ arg{\isacharunderscore}{\kern0pt}cong{\isadigit{2}}{\isacharbrackleft}{\kern0pt}\isakeyword{where}\ f{\isacharequal}{\kern0pt}{\isachardoublequoteopen}bigo{\isachardoublequoteclose}{\isacharbrackright}{\kern0pt}{\isacharcomma}{\kern0pt}\ simp{\isacharparenright}{\kern0pt}\isanewline
\ \ \ \ \isacommand{apply}\isamarkupfalse%
\ {\isacharparenleft}{\kern0pt}rule\ ext{\isacharparenright}{\kern0pt}\isanewline
\ \ \ \ \isacommand{by}\isamarkupfalse%
\ {\isacharparenleft}{\kern0pt}simp\ add{\isacharcolon}{\kern0pt}case{\isacharunderscore}{\kern0pt}prod{\isacharunderscore}{\kern0pt}beta{\isacharprime}{\kern0pt}\ g{\isacharunderscore}{\kern0pt}def\ n{\isacharunderscore}{\kern0pt}of{\isacharunderscore}{\kern0pt}def\ {\isasymepsilon}{\isacharunderscore}{\kern0pt}of{\isacharunderscore}{\kern0pt}def\ {\isasymdelta}{\isacharunderscore}{\kern0pt}of{\isacharunderscore}{\kern0pt}def{\isacharparenright}{\kern0pt}\isanewline
\ \ \isacommand{finally}\isamarkupfalse%
\ \isacommand{show}\isamarkupfalse%
\ {\isacharquery}{\kern0pt}thesis\isanewline
\ \ \ \ \isacommand{by}\isamarkupfalse%
\ simp\isanewline
\isacommand{qed}\isamarkupfalse%
%
\endisatagproof
{\isafoldproof}%
%
\isadelimproof
\isanewline
%
\endisadelimproof
%
\isadelimtheory
\isanewline
%
\endisadelimtheory
%
\isatagtheory
\isacommand{end}\isamarkupfalse%
%
\endisatagtheory
{\isafoldtheory}%
%
\isadelimtheory
%
\endisadelimtheory
%
\end{isabellebody}%
\endinput
%:%file=Frequency_Moment_0.tex%:%
%:%11=1%:%
%:%27=3%:%
%:%28=3%:%
%:%29=4%:%
%:%30=5%:%
%:%31=6%:%
%:%40=8%:%
%:%41=9%:%
%:%42=10%:%
%:%46=12%:%
%:%47=13%:%
%:%49=15%:%
%:%50=15%:%
%:%51=16%:%
%:%52=17%:%
%:%53=17%:%
%:%54=18%:%
%:%62=26%:%
%:%63=27%:%
%:%64=28%:%
%:%65=28%:%
%:%66=29%:%
%:%68=31%:%
%:%69=32%:%
%:%70=33%:%
%:%71=33%:%
%:%72=34%:%
%:%75=37%:%
%:%76=38%:%
%:%77=39%:%
%:%78=39%:%
%:%79=40%:%
%:%80=41%:%
%:%81=42%:%
%:%82=42%:%
%:%83=43%:%
%:%84=44%:%
%:%85=45%:%
%:%86=46%:%
%:%87=47%:%
%:%88=48%:%
%:%89=49%:%
%:%90=50%:%
%:%91=51%:%
%:%92=52%:%
%:%93=53%:%
%:%100=54%:%
%:%101=54%:%
%:%102=55%:%
%:%103=55%:%
%:%104=56%:%
%:%105=56%:%
%:%106=56%:%
%:%107=57%:%
%:%108=57%:%
%:%109=58%:%
%:%110=58%:%
%:%111=59%:%
%:%112=59%:%
%:%113=60%:%
%:%114=60%:%
%:%115=61%:%
%:%116=61%:%
%:%117=62%:%
%:%118=62%:%
%:%119=63%:%
%:%120=63%:%
%:%121=64%:%
%:%122=64%:%
%:%123=64%:%
%:%124=65%:%
%:%125=65%:%
%:%126=66%:%
%:%127=66%:%
%:%128=67%:%
%:%129=67%:%
%:%130=68%:%
%:%131=68%:%
%:%132=69%:%
%:%133=69%:%
%:%134=70%:%
%:%135=70%:%
%:%136=71%:%
%:%137=71%:%
%:%138=72%:%
%:%139=72%:%
%:%140=73%:%
%:%141=73%:%
%:%142=74%:%
%:%148=74%:%
%:%151=75%:%
%:%152=76%:%
%:%153=76%:%
%:%154=77%:%
%:%155=78%:%
%:%156=79%:%
%:%163=80%:%
%:%164=80%:%
%:%165=81%:%
%:%166=81%:%
%:%167=82%:%
%:%168=82%:%
%:%169=83%:%
%:%170=83%:%
%:%171=84%:%
%:%172=84%:%
%:%173=84%:%
%:%174=85%:%
%:%175=85%:%
%:%176=86%:%
%:%177=86%:%
%:%178=87%:%
%:%179=87%:%
%:%180=88%:%
%:%181=88%:%
%:%182=89%:%
%:%183=89%:%
%:%184=89%:%
%:%185=89%:%
%:%186=90%:%
%:%192=90%:%
%:%195=91%:%
%:%196=92%:%
%:%197=92%:%
%:%198=93%:%
%:%199=94%:%
%:%200=95%:%
%:%201=96%:%
%:%202=97%:%
%:%209=98%:%
%:%210=98%:%
%:%211=99%:%
%:%212=99%:%
%:%213=100%:%
%:%214=100%:%
%:%215=101%:%
%:%216=101%:%
%:%217=101%:%
%:%218=102%:%
%:%219=102%:%
%:%220=103%:%
%:%221=103%:%
%:%222=103%:%
%:%223=103%:%
%:%224=104%:%
%:%230=104%:%
%:%233=105%:%
%:%234=106%:%
%:%235=106%:%
%:%236=107%:%
%:%237=108%:%
%:%238=109%:%
%:%241=110%:%
%:%245=110%:%
%:%246=110%:%
%:%247=111%:%
%:%248=111%:%
%:%249=112%:%
%:%250=112%:%
%:%251=113%:%
%:%252=113%:%
%:%257=113%:%
%:%260=114%:%
%:%261=115%:%
%:%262=115%:%
%:%265=116%:%
%:%269=116%:%
%:%270=116%:%
%:%275=116%:%
%:%278=117%:%
%:%279=118%:%
%:%280=118%:%
%:%281=119%:%
%:%282=120%:%
%:%289=121%:%
%:%290=121%:%
%:%291=122%:%
%:%292=122%:%
%:%293=123%:%
%:%294=123%:%
%:%295=124%:%
%:%296=124%:%
%:%297=124%:%
%:%298=125%:%
%:%299=125%:%
%:%300=125%:%
%:%301=126%:%
%:%302=126%:%
%:%303=127%:%
%:%304=127%:%
%:%305=127%:%
%:%306=127%:%
%:%307=128%:%
%:%313=128%:%
%:%316=129%:%
%:%317=130%:%
%:%318=130%:%
%:%319=131%:%
%:%320=132%:%
%:%323=133%:%
%:%327=133%:%
%:%328=133%:%
%:%329=133%:%
%:%330=133%:%
%:%335=133%:%
%:%338=134%:%
%:%339=135%:%
%:%340=135%:%
%:%343=136%:%
%:%347=136%:%
%:%348=136%:%
%:%353=136%:%
%:%356=137%:%
%:%357=138%:%
%:%358=138%:%
%:%361=139%:%
%:%365=139%:%
%:%366=139%:%
%:%371=139%:%
%:%374=140%:%
%:%375=141%:%
%:%376=141%:%
%:%377=142%:%
%:%378=143%:%
%:%379=144%:%
%:%380=145%:%
%:%381=146%:%
%:%388=147%:%
%:%389=147%:%
%:%390=148%:%
%:%391=148%:%
%:%392=149%:%
%:%393=149%:%
%:%394=150%:%
%:%395=150%:%
%:%396=151%:%
%:%397=152%:%
%:%398=152%:%
%:%399=153%:%
%:%400=153%:%
%:%401=154%:%
%:%402=154%:%
%:%403=154%:%
%:%404=155%:%
%:%405=155%:%
%:%406=156%:%
%:%407=156%:%
%:%408=157%:%
%:%409=157%:%
%:%410=158%:%
%:%411=158%:%
%:%412=159%:%
%:%413=159%:%
%:%414=159%:%
%:%415=160%:%
%:%416=160%:%
%:%417=161%:%
%:%418=161%:%
%:%419=161%:%
%:%420=162%:%
%:%421=162%:%
%:%422=163%:%
%:%423=163%:%
%:%424=164%:%
%:%425=164%:%
%:%426=164%:%
%:%427=165%:%
%:%428=165%:%
%:%429=166%:%
%:%430=166%:%
%:%431=167%:%
%:%432=167%:%
%:%433=168%:%
%:%434=168%:%
%:%435=168%:%
%:%436=169%:%
%:%437=169%:%
%:%438=170%:%
%:%439=170%:%
%:%440=170%:%
%:%441=171%:%
%:%442=171%:%
%:%443=172%:%
%:%444=172%:%
%:%445=173%:%
%:%446=173%:%
%:%447=174%:%
%:%448=174%:%
%:%449=175%:%
%:%450=175%:%
%:%451=176%:%
%:%452=176%:%
%:%453=176%:%
%:%454=176%:%
%:%455=177%:%
%:%461=177%:%
%:%464=178%:%
%:%465=179%:%
%:%466=179%:%
%:%467=180%:%
%:%468=181%:%
%:%469=182%:%
%:%470=183%:%
%:%471=184%:%
%:%472=185%:%
%:%473=186%:%
%:%478=191%:%
%:%485=192%:%
%:%486=192%:%
%:%487=193%:%
%:%488=193%:%
%:%489=194%:%
%:%490=194%:%
%:%491=194%:%
%:%492=195%:%
%:%493=196%:%
%:%494=196%:%
%:%495=196%:%
%:%496=196%:%
%:%497=197%:%
%:%498=198%:%
%:%499=198%:%
%:%500=199%:%
%:%501=199%:%
%:%502=200%:%
%:%503=201%:%
%:%504=201%:%
%:%505=202%:%
%:%506=202%:%
%:%507=203%:%
%:%508=204%:%
%:%509=204%:%
%:%510=205%:%
%:%511=205%:%
%:%512=205%:%
%:%513=206%:%
%:%514=206%:%
%:%515=207%:%
%:%516=207%:%
%:%517=208%:%
%:%518=208%:%
%:%519=209%:%
%:%520=209%:%
%:%522=211%:%
%:%523=212%:%
%:%524=212%:%
%:%525=213%:%
%:%526=213%:%
%:%527=214%:%
%:%528=215%:%
%:%529=215%:%
%:%530=216%:%
%:%533=219%:%
%:%534=220%:%
%:%535=220%:%
%:%536=221%:%
%:%537=221%:%
%:%538=222%:%
%:%539=222%:%
%:%540=223%:%
%:%541=223%:%
%:%542=224%:%
%:%543=224%:%
%:%544=225%:%
%:%545=226%:%
%:%546=226%:%
%:%548=228%:%
%:%549=229%:%
%:%550=229%:%
%:%551=230%:%
%:%552=230%:%
%:%553=231%:%
%:%554=231%:%
%:%555=232%:%
%:%556=232%:%
%:%557=233%:%
%:%558=233%:%
%:%559=234%:%
%:%560=234%:%
%:%561=235%:%
%:%562=236%:%
%:%563=236%:%
%:%564=237%:%
%:%565=237%:%
%:%566=238%:%
%:%567=238%:%
%:%568=239%:%
%:%569=239%:%
%:%570=239%:%
%:%571=240%:%
%:%572=241%:%
%:%573=241%:%
%:%574=242%:%
%:%575=242%:%
%:%576=243%:%
%:%577=243%:%
%:%578=244%:%
%:%579=244%:%
%:%580=244%:%
%:%581=244%:%
%:%582=245%:%
%:%583=245%:%
%:%584=246%:%
%:%585=246%:%
%:%586=247%:%
%:%587=247%:%
%:%588=248%:%
%:%589=249%:%
%:%590=249%:%
%:%591=250%:%
%:%592=250%:%
%:%593=251%:%
%:%594=251%:%
%:%595=252%:%
%:%596=252%:%
%:%597=253%:%
%:%598=253%:%
%:%599=253%:%
%:%600=253%:%
%:%601=254%:%
%:%602=254%:%
%:%603=255%:%
%:%604=255%:%
%:%605=255%:%
%:%606=255%:%
%:%607=256%:%
%:%608=257%:%
%:%609=257%:%
%:%610=258%:%
%:%611=258%:%
%:%612=258%:%
%:%613=259%:%
%:%614=259%:%
%:%615=260%:%
%:%616=260%:%
%:%617=260%:%
%:%618=261%:%
%:%619=261%:%
%:%620=262%:%
%:%621=262%:%
%:%622=262%:%
%:%623=262%:%
%:%624=263%:%
%:%625=263%:%
%:%626=263%:%
%:%627=264%:%
%:%628=264%:%
%:%629=264%:%
%:%630=264%:%
%:%631=265%:%
%:%632=265%:%
%:%633=265%:%
%:%634=266%:%
%:%635=266%:%
%:%636=267%:%
%:%637=268%:%
%:%638=268%:%
%:%639=269%:%
%:%640=269%:%
%:%641=269%:%
%:%642=270%:%
%:%643=270%:%
%:%644=271%:%
%:%645=272%:%
%:%646=272%:%
%:%650=276%:%
%:%651=277%:%
%:%652=277%:%
%:%653=278%:%
%:%654=278%:%
%:%655=279%:%
%:%656=279%:%
%:%657=280%:%
%:%658=280%:%
%:%659=281%:%
%:%660=281%:%
%:%661=282%:%
%:%662=282%:%
%:%663=283%:%
%:%664=283%:%
%:%665=283%:%
%:%667=285%:%
%:%668=286%:%
%:%669=286%:%
%:%670=287%:%
%:%671=287%:%
%:%672=288%:%
%:%673=288%:%
%:%674=289%:%
%:%675=289%:%
%:%676=289%:%
%:%678=291%:%
%:%679=292%:%
%:%680=292%:%
%:%681=293%:%
%:%682=293%:%
%:%683=294%:%
%:%684=294%:%
%:%685=295%:%
%:%686=295%:%
%:%687=295%:%
%:%688=296%:%
%:%689=297%:%
%:%690=297%:%
%:%691=298%:%
%:%692=298%:%
%:%693=299%:%
%:%694=299%:%
%:%695=300%:%
%:%696=300%:%
%:%697=300%:%
%:%698=301%:%
%:%699=301%:%
%:%700=301%:%
%:%701=302%:%
%:%702=302%:%
%:%703=303%:%
%:%704=303%:%
%:%705=303%:%
%:%706=304%:%
%:%707=305%:%
%:%708=305%:%
%:%709=306%:%
%:%710=306%:%
%:%711=306%:%
%:%712=307%:%
%:%713=308%:%
%:%714=308%:%
%:%715=309%:%
%:%716=309%:%
%:%717=310%:%
%:%718=310%:%
%:%719=311%:%
%:%720=311%:%
%:%721=312%:%
%:%722=312%:%
%:%723=312%:%
%:%724=313%:%
%:%725=314%:%
%:%726=314%:%
%:%727=315%:%
%:%728=315%:%
%:%729=316%:%
%:%730=316%:%
%:%731=317%:%
%:%732=317%:%
%:%733=318%:%
%:%734=318%:%
%:%735=319%:%
%:%736=319%:%
%:%737=320%:%
%:%738=320%:%
%:%739=321%:%
%:%740=321%:%
%:%741=321%:%
%:%742=322%:%
%:%743=323%:%
%:%744=323%:%
%:%745=324%:%
%:%746=324%:%
%:%747=325%:%
%:%748=325%:%
%:%749=326%:%
%:%750=326%:%
%:%751=326%:%
%:%752=327%:%
%:%753=328%:%
%:%754=328%:%
%:%755=329%:%
%:%756=329%:%
%:%757=330%:%
%:%758=330%:%
%:%759=331%:%
%:%760=331%:%
%:%761=331%:%
%:%762=332%:%
%:%763=333%:%
%:%764=333%:%
%:%765=334%:%
%:%766=334%:%
%:%767=335%:%
%:%768=335%:%
%:%769=336%:%
%:%770=336%:%
%:%771=336%:%
%:%772=337%:%
%:%773=338%:%
%:%774=338%:%
%:%775=339%:%
%:%776=339%:%
%:%777=340%:%
%:%778=340%:%
%:%779=341%:%
%:%780=341%:%
%:%781=342%:%
%:%782=342%:%
%:%783=342%:%
%:%784=343%:%
%:%785=344%:%
%:%786=344%:%
%:%787=345%:%
%:%788=345%:%
%:%789=346%:%
%:%790=346%:%
%:%791=347%:%
%:%792=347%:%
%:%793=348%:%
%:%794=348%:%
%:%795=349%:%
%:%796=349%:%
%:%797=349%:%
%:%798=350%:%
%:%799=350%:%
%:%800=351%:%
%:%801=351%:%
%:%802=352%:%
%:%803=352%:%
%:%804=353%:%
%:%805=353%:%
%:%806=354%:%
%:%807=354%:%
%:%808=354%:%
%:%809=354%:%
%:%810=355%:%
%:%811=355%:%
%:%812=356%:%
%:%813=356%:%
%:%814=357%:%
%:%815=357%:%
%:%816=357%:%
%:%817=358%:%
%:%818=358%:%
%:%819=359%:%
%:%820=359%:%
%:%821=360%:%
%:%822=360%:%
%:%823=360%:%
%:%824=360%:%
%:%825=361%:%
%:%826=361%:%
%:%827=362%:%
%:%828=362%:%
%:%829=362%:%
%:%830=363%:%
%:%831=363%:%
%:%832=364%:%
%:%833=364%:%
%:%834=364%:%
%:%835=365%:%
%:%836=366%:%
%:%837=366%:%
%:%838=367%:%
%:%839=367%:%
%:%840=368%:%
%:%841=369%:%
%:%842=369%:%
%:%845=372%:%
%:%846=373%:%
%:%847=373%:%
%:%848=374%:%
%:%849=374%:%
%:%850=375%:%
%:%851=375%:%
%:%852=376%:%
%:%853=376%:%
%:%854=376%:%
%:%856=378%:%
%:%857=379%:%
%:%858=379%:%
%:%859=380%:%
%:%860=380%:%
%:%861=381%:%
%:%862=381%:%
%:%863=382%:%
%:%864=382%:%
%:%865=383%:%
%:%866=383%:%
%:%867=383%:%
%:%868=384%:%
%:%869=384%:%
%:%870=385%:%
%:%871=385%:%
%:%872=385%:%
%:%873=386%:%
%:%874=386%:%
%:%875=386%:%
%:%876=387%:%
%:%877=387%:%
%:%878=388%:%
%:%879=388%:%
%:%880=388%:%
%:%881=389%:%
%:%882=389%:%
%:%883=390%:%
%:%884=390%:%
%:%885=391%:%
%:%886=391%:%
%:%887=392%:%
%:%888=392%:%
%:%889=393%:%
%:%890=393%:%
%:%891=394%:%
%:%892=394%:%
%:%893=395%:%
%:%894=395%:%
%:%895=396%:%
%:%896=396%:%
%:%897=397%:%
%:%898=397%:%
%:%899=397%:%
%:%900=398%:%
%:%901=398%:%
%:%902=399%:%
%:%903=399%:%
%:%904=399%:%
%:%905=399%:%
%:%906=400%:%
%:%907=401%:%
%:%908=401%:%
%:%909=402%:%
%:%910=403%:%
%:%911=403%:%
%:%912=404%:%
%:%913=404%:%
%:%914=405%:%
%:%915=405%:%
%:%916=406%:%
%:%917=407%:%
%:%918=407%:%
%:%919=408%:%
%:%920=408%:%
%:%921=409%:%
%:%922=409%:%
%:%923=410%:%
%:%924=410%:%
%:%925=411%:%
%:%926=411%:%
%:%927=412%:%
%:%928=412%:%
%:%929=413%:%
%:%930=413%:%
%:%931=414%:%
%:%932=414%:%
%:%933=415%:%
%:%934=415%:%
%:%935=416%:%
%:%936=417%:%
%:%937=417%:%
%:%938=418%:%
%:%939=419%:%
%:%940=419%:%
%:%941=420%:%
%:%942=420%:%
%:%943=420%:%
%:%944=420%:%
%:%945=421%:%
%:%946=421%:%
%:%947=421%:%
%:%948=421%:%
%:%949=422%:%
%:%955=422%:%
%:%958=423%:%
%:%959=424%:%
%:%960=424%:%
%:%963=425%:%
%:%967=425%:%
%:%968=425%:%
%:%973=425%:%
%:%976=426%:%
%:%977=427%:%
%:%978=427%:%
%:%981=428%:%
%:%985=428%:%
%:%986=428%:%
%:%991=428%:%
%:%994=429%:%
%:%995=430%:%
%:%996=430%:%
%:%997=431%:%
%:%998=432%:%
%:%999=433%:%
%:%1000=434%:%
%:%1001=435%:%
%:%1008=436%:%
%:%1009=436%:%
%:%1010=437%:%
%:%1011=437%:%
%:%1012=438%:%
%:%1013=438%:%
%:%1014=439%:%
%:%1015=439%:%
%:%1016=440%:%
%:%1017=440%:%
%:%1018=441%:%
%:%1019=441%:%
%:%1020=442%:%
%:%1021=442%:%
%:%1022=443%:%
%:%1023=443%:%
%:%1024=444%:%
%:%1025=444%:%
%:%1026=445%:%
%:%1027=445%:%
%:%1028=446%:%
%:%1029=446%:%
%:%1030=447%:%
%:%1031=448%:%
%:%1032=448%:%
%:%1033=449%:%
%:%1034=449%:%
%:%1035=450%:%
%:%1036=450%:%
%:%1037=451%:%
%:%1038=451%:%
%:%1039=452%:%
%:%1040=453%:%
%:%1041=453%:%
%:%1043=455%:%
%:%1044=456%:%
%:%1045=457%:%
%:%1046=457%:%
%:%1047=458%:%
%:%1048=458%:%
%:%1049=458%:%
%:%1050=459%:%
%:%1051=460%:%
%:%1052=460%:%
%:%1053=461%:%
%:%1054=461%:%
%:%1055=461%:%
%:%1056=462%:%
%:%1057=463%:%
%:%1058=463%:%
%:%1059=463%:%
%:%1060=463%:%
%:%1061=464%:%
%:%1062=464%:%
%:%1063=464%:%
%:%1064=464%:%
%:%1065=465%:%
%:%1066=466%:%
%:%1067=466%:%
%:%1068=467%:%
%:%1069=467%:%
%:%1070=468%:%
%:%1071=468%:%
%:%1072=469%:%
%:%1073=469%:%
%:%1074=470%:%
%:%1075=470%:%
%:%1076=471%:%
%:%1077=471%:%
%:%1078=472%:%
%:%1079=472%:%
%:%1080=473%:%
%:%1081=473%:%
%:%1082=474%:%
%:%1083=475%:%
%:%1084=475%:%
%:%1085=475%:%
%:%1086=476%:%
%:%1087=476%:%
%:%1088=476%:%
%:%1089=477%:%
%:%1090=477%:%
%:%1091=478%:%
%:%1092=478%:%
%:%1093=479%:%
%:%1094=479%:%
%:%1095=479%:%
%:%1096=480%:%
%:%1097=480%:%
%:%1098=480%:%
%:%1099=480%:%
%:%1100=480%:%
%:%1101=481%:%
%:%1102=481%:%
%:%1103=481%:%
%:%1104=481%:%
%:%1105=482%:%
%:%1106=483%:%
%:%1107=483%:%
%:%1108=484%:%
%:%1109=484%:%
%:%1110=484%:%
%:%1111=484%:%
%:%1112=484%:%
%:%1113=485%:%
%:%1114=485%:%
%:%1115=485%:%
%:%1116=486%:%
%:%1117=486%:%
%:%1118=487%:%
%:%1119=487%:%
%:%1120=487%:%
%:%1121=488%:%
%:%1122=488%:%
%:%1123=489%:%
%:%1124=489%:%
%:%1125=490%:%
%:%1126=490%:%
%:%1127=490%:%
%:%1128=491%:%
%:%1129=491%:%
%:%1130=491%:%
%:%1131=492%:%
%:%1132=492%:%
%:%1133=493%:%
%:%1134=493%:%
%:%1135=494%:%
%:%1136=494%:%
%:%1137=495%:%
%:%1138=495%:%
%:%1139=495%:%
%:%1140=496%:%
%:%1141=496%:%
%:%1142=497%:%
%:%1143=497%:%
%:%1144=497%:%
%:%1145=498%:%
%:%1146=498%:%
%:%1147=499%:%
%:%1148=500%:%
%:%1149=500%:%
%:%1150=501%:%
%:%1151=502%:%
%:%1152=502%:%
%:%1153=503%:%
%:%1154=503%:%
%:%1155=504%:%
%:%1156=504%:%
%:%1157=505%:%
%:%1158=505%:%
%:%1159=505%:%
%:%1160=505%:%
%:%1161=506%:%
%:%1162=506%:%
%:%1163=507%:%
%:%1164=507%:%
%:%1165=508%:%
%:%1166=508%:%
%:%1167=508%:%
%:%1168=508%:%
%:%1169=509%:%
%:%1170=509%:%
%:%1171=510%:%
%:%1172=510%:%
%:%1173=510%:%
%:%1174=511%:%
%:%1175=511%:%
%:%1176=512%:%
%:%1177=512%:%
%:%1178=513%:%
%:%1179=513%:%
%:%1180=514%:%
%:%1181=514%:%
%:%1182=514%:%
%:%1183=514%:%
%:%1184=515%:%
%:%1185=515%:%
%:%1186=516%:%
%:%1187=516%:%
%:%1188=516%:%
%:%1189=517%:%
%:%1190=517%:%
%:%1191=518%:%
%:%1192=518%:%
%:%1193=519%:%
%:%1194=519%:%
%:%1195=520%:%
%:%1196=520%:%
%:%1197=521%:%
%:%1198=521%:%
%:%1199=522%:%
%:%1200=522%:%
%:%1201=523%:%
%:%1202=523%:%
%:%1203=524%:%
%:%1204=524%:%
%:%1205=525%:%
%:%1206=525%:%
%:%1207=525%:%
%:%1208=526%:%
%:%1209=526%:%
%:%1210=526%:%
%:%1211=526%:%
%:%1212=527%:%
%:%1213=527%:%
%:%1214=527%:%
%:%1215=527%:%
%:%1216=528%:%
%:%1217=528%:%
%:%1218=528%:%
%:%1219=529%:%
%:%1220=529%:%
%:%1221=530%:%
%:%1222=530%:%
%:%1223=530%:%
%:%1224=531%:%
%:%1225=531%:%
%:%1226=532%:%
%:%1227=533%:%
%:%1228=533%:%
%:%1229=533%:%
%:%1230=533%:%
%:%1231=534%:%
%:%1232=534%:%
%:%1233=535%:%
%:%1234=535%:%
%:%1235=536%:%
%:%1236=536%:%
%:%1237=536%:%
%:%1238=537%:%
%:%1239=538%:%
%:%1240=538%:%
%:%1241=539%:%
%:%1242=539%:%
%:%1243=539%:%
%:%1244=540%:%
%:%1245=540%:%
%:%1246=540%:%
%:%1247=541%:%
%:%1248=541%:%
%:%1249=542%:%
%:%1250=542%:%
%:%1251=542%:%
%:%1252=543%:%
%:%1253=543%:%
%:%1254=544%:%
%:%1255=544%:%
%:%1256=545%:%
%:%1257=545%:%
%:%1258=545%:%
%:%1259=546%:%
%:%1260=546%:%
%:%1261=547%:%
%:%1262=547%:%
%:%1263=548%:%
%:%1264=548%:%
%:%1265=548%:%
%:%1266=549%:%
%:%1267=549%:%
%:%1268=550%:%
%:%1269=550%:%
%:%1270=551%:%
%:%1271=551%:%
%:%1272=551%:%
%:%1273=552%:%
%:%1274=552%:%
%:%1275=552%:%
%:%1276=553%:%
%:%1277=553%:%
%:%1278=554%:%
%:%1279=554%:%
%:%1280=554%:%
%:%1281=554%:%
%:%1282=555%:%
%:%1283=556%:%
%:%1284=556%:%
%:%1285=557%:%
%:%1286=557%:%
%:%1287=558%:%
%:%1288=558%:%
%:%1289=558%:%
%:%1290=559%:%
%:%1291=559%:%
%:%1292=559%:%
%:%1293=560%:%
%:%1294=560%:%
%:%1295=560%:%
%:%1296=560%:%
%:%1297=561%:%
%:%1298=562%:%
%:%1299=562%:%
%:%1300=563%:%
%:%1301=563%:%
%:%1302=563%:%
%:%1303=564%:%
%:%1304=564%:%
%:%1305=565%:%
%:%1306=565%:%
%:%1307=566%:%
%:%1308=566%:%
%:%1309=566%:%
%:%1310=567%:%
%:%1311=567%:%
%:%1312=567%:%
%:%1313=568%:%
%:%1314=569%:%
%:%1315=569%:%
%:%1316=570%:%
%:%1317=570%:%
%:%1318=571%:%
%:%1319=571%:%
%:%1320=572%:%
%:%1321=572%:%
%:%1322=572%:%
%:%1323=573%:%
%:%1324=573%:%
%:%1325=574%:%
%:%1326=574%:%
%:%1327=574%:%
%:%1328=574%:%
%:%1329=575%:%
%:%1330=576%:%
%:%1331=576%:%
%:%1332=577%:%
%:%1333=577%:%
%:%1334=578%:%
%:%1335=578%:%
%:%1336=578%:%
%:%1337=579%:%
%:%1338=579%:%
%:%1339=580%:%
%:%1340=581%:%
%:%1341=581%:%
%:%1342=582%:%
%:%1343=582%:%
%:%1344=583%:%
%:%1345=584%:%
%:%1346=584%:%
%:%1347=585%:%
%:%1348=585%:%
%:%1349=586%:%
%:%1350=586%:%
%:%1351=587%:%
%:%1352=588%:%
%:%1353=588%:%
%:%1355=590%:%
%:%1356=591%:%
%:%1357=591%:%
%:%1358=592%:%
%:%1359=592%:%
%:%1360=593%:%
%:%1361=593%:%
%:%1362=594%:%
%:%1363=594%:%
%:%1364=595%:%
%:%1365=595%:%
%:%1366=596%:%
%:%1367=596%:%
%:%1368=597%:%
%:%1369=597%:%
%:%1370=598%:%
%:%1371=599%:%
%:%1372=599%:%
%:%1373=600%:%
%:%1374=601%:%
%:%1375=601%:%
%:%1376=602%:%
%:%1377=602%:%
%:%1378=603%:%
%:%1379=603%:%
%:%1380=604%:%
%:%1381=604%:%
%:%1382=604%:%
%:%1383=604%:%
%:%1384=605%:%
%:%1385=605%:%
%:%1386=606%:%
%:%1387=607%:%
%:%1388=607%:%
%:%1389=608%:%
%:%1390=608%:%
%:%1391=609%:%
%:%1392=609%:%
%:%1393=610%:%
%:%1394=610%:%
%:%1395=610%:%
%:%1396=611%:%
%:%1397=611%:%
%:%1398=612%:%
%:%1399=612%:%
%:%1400=612%:%
%:%1401=613%:%
%:%1402=613%:%
%:%1403=614%:%
%:%1404=614%:%
%:%1405=615%:%
%:%1406=615%:%
%:%1407=615%:%
%:%1408=616%:%
%:%1409=616%:%
%:%1410=617%:%
%:%1411=617%:%
%:%1412=618%:%
%:%1413=618%:%
%:%1414=619%:%
%:%1415=619%:%
%:%1416=620%:%
%:%1417=620%:%
%:%1418=620%:%
%:%1419=621%:%
%:%1420=621%:%
%:%1421=622%:%
%:%1422=622%:%
%:%1423=622%:%
%:%1424=623%:%
%:%1425=623%:%
%:%1426=623%:%
%:%1427=624%:%
%:%1428=624%:%
%:%1429=624%:%
%:%1430=625%:%
%:%1431=626%:%
%:%1432=626%:%
%:%1433=627%:%
%:%1434=627%:%
%:%1435=628%:%
%:%1436=628%:%
%:%1437=629%:%
%:%1438=630%:%
%:%1439=630%:%
%:%1440=631%:%
%:%1441=631%:%
%:%1442=631%:%
%:%1443=632%:%
%:%1444=632%:%
%:%1445=633%:%
%:%1446=633%:%
%:%1447=634%:%
%:%1448=634%:%
%:%1449=634%:%
%:%1450=635%:%
%:%1451=635%:%
%:%1452=636%:%
%:%1453=636%:%
%:%1454=636%:%
%:%1455=637%:%
%:%1456=637%:%
%:%1457=638%:%
%:%1458=638%:%
%:%1459=638%:%
%:%1460=639%:%
%:%1461=639%:%
%:%1462=640%:%
%:%1463=641%:%
%:%1464=641%:%
%:%1465=642%:%
%:%1466=643%:%
%:%1467=643%:%
%:%1468=644%:%
%:%1469=644%:%
%:%1470=644%:%
%:%1471=645%:%
%:%1472=645%:%
%:%1473=646%:%
%:%1474=646%:%
%:%1475=647%:%
%:%1476=647%:%
%:%1477=648%:%
%:%1478=648%:%
%:%1479=649%:%
%:%1480=649%:%
%:%1481=650%:%
%:%1482=650%:%
%:%1483=651%:%
%:%1484=651%:%
%:%1485=652%:%
%:%1486=652%:%
%:%1487=653%:%
%:%1488=653%:%
%:%1489=653%:%
%:%1490=654%:%
%:%1491=654%:%
%:%1492=655%:%
%:%1493=655%:%
%:%1494=656%:%
%:%1495=656%:%
%:%1496=657%:%
%:%1497=657%:%
%:%1498=658%:%
%:%1499=658%:%
%:%1500=659%:%
%:%1501=659%:%
%:%1502=660%:%
%:%1503=660%:%
%:%1504=661%:%
%:%1505=661%:%
%:%1506=661%:%
%:%1507=662%:%
%:%1508=662%:%
%:%1509=663%:%
%:%1510=663%:%
%:%1511=663%:%
%:%1512=664%:%
%:%1513=664%:%
%:%1514=665%:%
%:%1515=665%:%
%:%1516=666%:%
%:%1517=667%:%
%:%1518=668%:%
%:%1519=668%:%
%:%1520=668%:%
%:%1521=669%:%
%:%1522=669%:%
%:%1523=670%:%
%:%1524=671%:%
%:%1525=671%:%
%:%1526=672%:%
%:%1527=672%:%
%:%1528=672%:%
%:%1529=673%:%
%:%1530=674%:%
%:%1531=674%:%
%:%1532=675%:%
%:%1533=675%:%
%:%1534=675%:%
%:%1535=676%:%
%:%1536=677%:%
%:%1537=677%:%
%:%1538=678%:%
%:%1539=679%:%
%:%1540=679%:%
%:%1541=680%:%
%:%1542=680%:%
%:%1543=681%:%
%:%1544=681%:%
%:%1545=681%:%
%:%1546=682%:%
%:%1547=682%:%
%:%1548=683%:%
%:%1549=683%:%
%:%1550=683%:%
%:%1551=684%:%
%:%1552=685%:%
%:%1553=685%:%
%:%1554=686%:%
%:%1555=686%:%
%:%1556=687%:%
%:%1557=687%:%
%:%1558=687%:%
%:%1559=688%:%
%:%1560=688%:%
%:%1561=689%:%
%:%1562=689%:%
%:%1563=690%:%
%:%1564=690%:%
%:%1565=690%:%
%:%1566=691%:%
%:%1567=691%:%
%:%1568=691%:%
%:%1569=692%:%
%:%1570=692%:%
%:%1571=692%:%
%:%1572=693%:%
%:%1573=693%:%
%:%1574=694%:%
%:%1575=695%:%
%:%1576=695%:%
%:%1577=696%:%
%:%1578=696%:%
%:%1579=697%:%
%:%1580=697%:%
%:%1581=698%:%
%:%1582=698%:%
%:%1583=698%:%
%:%1584=699%:%
%:%1585=699%:%
%:%1586=700%:%
%:%1587=700%:%
%:%1588=701%:%
%:%1589=701%:%
%:%1590=702%:%
%:%1591=702%:%
%:%1592=703%:%
%:%1593=703%:%
%:%1594=703%:%
%:%1595=704%:%
%:%1596=704%:%
%:%1597=704%:%
%:%1598=705%:%
%:%1599=705%:%
%:%1600=706%:%
%:%1601=706%:%
%:%1602=707%:%
%:%1603=707%:%
%:%1604=707%:%
%:%1605=707%:%
%:%1606=708%:%
%:%1607=708%:%
%:%1608=708%:%
%:%1609=708%:%
%:%1610=709%:%
%:%1611=709%:%
%:%1612=709%:%
%:%1613=709%:%
%:%1614=710%:%
%:%1615=710%:%
%:%1616=711%:%
%:%1617=711%:%
%:%1618=712%:%
%:%1619=712%:%
%:%1620=713%:%
%:%1621=714%:%
%:%1622=714%:%
%:%1623=715%:%
%:%1624=715%:%
%:%1625=716%:%
%:%1626=716%:%
%:%1627=717%:%
%:%1628=717%:%
%:%1629=717%:%
%:%1630=717%:%
%:%1631=718%:%
%:%1632=718%:%
%:%1633=719%:%
%:%1634=719%:%
%:%1635=720%:%
%:%1636=720%:%
%:%1637=720%:%
%:%1638=720%:%
%:%1639=721%:%
%:%1640=721%:%
%:%1641=721%:%
%:%1642=722%:%
%:%1643=722%:%
%:%1644=722%:%
%:%1645=723%:%
%:%1646=723%:%
%:%1647=724%:%
%:%1648=724%:%
%:%1649=725%:%
%:%1650=726%:%
%:%1651=726%:%
%:%1653=728%:%
%:%1654=729%:%
%:%1655=729%:%
%:%1656=730%:%
%:%1657=730%:%
%:%1658=731%:%
%:%1659=731%:%
%:%1660=732%:%
%:%1661=732%:%
%:%1662=733%:%
%:%1663=733%:%
%:%1664=734%:%
%:%1665=734%:%
%:%1666=735%:%
%:%1667=735%:%
%:%1668=735%:%
%:%1669=736%:%
%:%1670=736%:%
%:%1671=737%:%
%:%1672=737%:%
%:%1673=738%:%
%:%1674=738%:%
%:%1675=739%:%
%:%1676=739%:%
%:%1677=740%:%
%:%1678=740%:%
%:%1679=740%:%
%:%1680=741%:%
%:%1681=741%:%
%:%1682=741%:%
%:%1683=742%:%
%:%1684=742%:%
%:%1685=743%:%
%:%1686=743%:%
%:%1687=743%:%
%:%1688=743%:%
%:%1689=744%:%
%:%1690=744%:%
%:%1691=745%:%
%:%1692=745%:%
%:%1693=745%:%
%:%1694=746%:%
%:%1695=746%:%
%:%1696=747%:%
%:%1697=747%:%
%:%1698=748%:%
%:%1699=749%:%
%:%1700=749%:%
%:%1701=750%:%
%:%1702=750%:%
%:%1703=751%:%
%:%1704=751%:%
%:%1705=752%:%
%:%1706=752%:%
%:%1707=753%:%
%:%1708=753%:%
%:%1709=754%:%
%:%1710=754%:%
%:%1711=754%:%
%:%1712=755%:%
%:%1713=755%:%
%:%1714=756%:%
%:%1715=756%:%
%:%1716=757%:%
%:%1717=757%:%
%:%1718=758%:%
%:%1719=758%:%
%:%1720=758%:%
%:%1721=758%:%
%:%1722=759%:%
%:%1723=759%:%
%:%1724=759%:%
%:%1725=760%:%
%:%1726=760%:%
%:%1727=761%:%
%:%1728=761%:%
%:%1729=762%:%
%:%1730=762%:%
%:%1731=762%:%
%:%1732=762%:%
%:%1733=763%:%
%:%1734=763%:%
%:%1735=764%:%
%:%1736=764%:%
%:%1737=764%:%
%:%1738=765%:%
%:%1739=765%:%
%:%1740=765%:%
%:%1741=766%:%
%:%1742=766%:%
%:%1743=767%:%
%:%1744=767%:%
%:%1745=768%:%
%:%1746=768%:%
%:%1747=769%:%
%:%1748=769%:%
%:%1749=770%:%
%:%1750=770%:%
%:%1751=771%:%
%:%1752=771%:%
%:%1753=772%:%
%:%1754=772%:%
%:%1755=772%:%
%:%1756=772%:%
%:%1757=773%:%
%:%1758=773%:%
%:%1759=774%:%
%:%1760=774%:%
%:%1761=775%:%
%:%1762=775%:%
%:%1763=776%:%
%:%1764=776%:%
%:%1765=776%:%
%:%1766=776%:%
%:%1767=777%:%
%:%1768=777%:%
%:%1769=778%:%
%:%1770=778%:%
%:%1771=779%:%
%:%1772=779%:%
%:%1773=779%:%
%:%1774=780%:%
%:%1775=780%:%
%:%1776=780%:%
%:%1777=781%:%
%:%1778=781%:%
%:%1779=782%:%
%:%1780=782%:%
%:%1781=783%:%
%:%1782=783%:%
%:%1783=783%:%
%:%1784=783%:%
%:%1785=784%:%
%:%1786=784%:%
%:%1787=784%:%
%:%1788=785%:%
%:%1789=785%:%
%:%1790=786%:%
%:%1791=786%:%
%:%1792=787%:%
%:%1793=787%:%
%:%1794=788%:%
%:%1795=788%:%
%:%1796=789%:%
%:%1797=789%:%
%:%1798=789%:%
%:%1799=789%:%
%:%1800=789%:%
%:%1801=790%:%
%:%1802=790%:%
%:%1803=790%:%
%:%1804=790%:%
%:%1805=790%:%
%:%1806=791%:%
%:%1807=791%:%
%:%1808=791%:%
%:%1809=792%:%
%:%1810=792%:%
%:%1811=793%:%
%:%1812=793%:%
%:%1813=794%:%
%:%1814=795%:%
%:%1815=795%:%
%:%1816=795%:%
%:%1817=795%:%
%:%1818=796%:%
%:%1819=796%:%
%:%1820=797%:%
%:%1821=797%:%
%:%1822=797%:%
%:%1823=798%:%
%:%1824=798%:%
%:%1825=799%:%
%:%1826=799%:%
%:%1827=799%:%
%:%1828=800%:%
%:%1829=801%:%
%:%1830=801%:%
%:%1831=802%:%
%:%1832=802%:%
%:%1833=803%:%
%:%1834=803%:%
%:%1835=804%:%
%:%1836=804%:%
%:%1837=805%:%
%:%1838=805%:%
%:%1839=806%:%
%:%1840=806%:%
%:%1841=806%:%
%:%1842=807%:%
%:%1843=807%:%
%:%1844=807%:%
%:%1845=808%:%
%:%1846=808%:%
%:%1847=808%:%
%:%1848=808%:%
%:%1849=809%:%
%:%1850=809%:%
%:%1851=810%:%
%:%1852=810%:%
%:%1853=810%:%
%:%1854=810%:%
%:%1855=811%:%
%:%1856=811%:%
%:%1857=811%:%
%:%1858=811%:%
%:%1859=812%:%
%:%1860=812%:%
%:%1861=812%:%
%:%1862=813%:%
%:%1863=813%:%
%:%1864=813%:%
%:%1865=814%:%
%:%1866=814%:%
%:%1867=815%:%
%:%1868=816%:%
%:%1869=816%:%
%:%1870=817%:%
%:%1871=817%:%
%:%1872=818%:%
%:%1873=818%:%
%:%1874=819%:%
%:%1875=819%:%
%:%1877=821%:%
%:%1878=822%:%
%:%1879=822%:%
%:%1880=823%:%
%:%1881=823%:%
%:%1882=824%:%
%:%1883=824%:%
%:%1884=825%:%
%:%1885=825%:%
%:%1886=826%:%
%:%1887=827%:%
%:%1888=827%:%
%:%1889=828%:%
%:%1890=828%:%
%:%1891=828%:%
%:%1892=829%:%
%:%1893=829%:%
%:%1894=830%:%
%:%1895=830%:%
%:%1896=831%:%
%:%1897=831%:%
%:%1898=832%:%
%:%1899=832%:%
%:%1900=832%:%
%:%1901=833%:%
%:%1902=833%:%
%:%1903=833%:%
%:%1904=834%:%
%:%1905=834%:%
%:%1906=835%:%
%:%1907=835%:%
%:%1908=835%:%
%:%1909=836%:%
%:%1910=836%:%
%:%1911=837%:%
%:%1912=837%:%
%:%1913=838%:%
%:%1914=838%:%
%:%1915=838%:%
%:%1916=839%:%
%:%1917=839%:%
%:%1918=839%:%
%:%1919=840%:%
%:%1920=840%:%
%:%1921=841%:%
%:%1922=841%:%
%:%1923=842%:%
%:%1924=842%:%
%:%1925=842%:%
%:%1926=843%:%
%:%1927=843%:%
%:%1928=843%:%
%:%1929=844%:%
%:%1930=844%:%
%:%1931=845%:%
%:%1932=845%:%
%:%1933=846%:%
%:%1934=846%:%
%:%1935=847%:%
%:%1936=847%:%
%:%1937=848%:%
%:%1938=848%:%
%:%1939=849%:%
%:%1940=849%:%
%:%1941=850%:%
%:%1942=850%:%
%:%1943=850%:%
%:%1944=851%:%
%:%1945=851%:%
%:%1946=852%:%
%:%1947=852%:%
%:%1948=853%:%
%:%1949=853%:%
%:%1950=853%:%
%:%1951=854%:%
%:%1952=854%:%
%:%1953=854%:%
%:%1954=855%:%
%:%1955=855%:%
%:%1956=856%:%
%:%1957=856%:%
%:%1958=857%:%
%:%1959=857%:%
%:%1960=857%:%
%:%1961=858%:%
%:%1962=858%:%
%:%1963=859%:%
%:%1964=859%:%
%:%1965=860%:%
%:%1966=860%:%
%:%1967=860%:%
%:%1968=861%:%
%:%1969=861%:%
%:%1970=862%:%
%:%1971=862%:%
%:%1972=863%:%
%:%1973=863%:%
%:%1974=864%:%
%:%1975=865%:%
%:%1976=865%:%
%:%1977=866%:%
%:%1978=866%:%
%:%1979=866%:%
%:%1980=867%:%
%:%1981=867%:%
%:%1982=867%:%
%:%1983=868%:%
%:%1984=868%:%
%:%1985=868%:%
%:%1986=869%:%
%:%1987=869%:%
%:%1988=870%:%
%:%1989=870%:%
%:%1990=871%:%
%:%1991=871%:%
%:%1992=872%:%
%:%1993=872%:%
%:%1994=873%:%
%:%1995=873%:%
%:%1996=874%:%
%:%1997=874%:%
%:%1998=875%:%
%:%1999=875%:%
%:%2000=876%:%
%:%2001=876%:%
%:%2002=877%:%
%:%2003=877%:%
%:%2004=877%:%
%:%2005=878%:%
%:%2006=878%:%
%:%2007=879%:%
%:%2008=879%:%
%:%2009=879%:%
%:%2010=880%:%
%:%2011=880%:%
%:%2012=880%:%
%:%2013=881%:%
%:%2014=881%:%
%:%2015=881%:%
%:%2016=881%:%
%:%2017=882%:%
%:%2018=882%:%
%:%2019=882%:%
%:%2020=883%:%
%:%2021=883%:%
%:%2022=884%:%
%:%2023=884%:%
%:%2024=884%:%
%:%2025=885%:%
%:%2026=885%:%
%:%2027=885%:%
%:%2028=886%:%
%:%2029=886%:%
%:%2030=887%:%
%:%2031=887%:%
%:%2032=888%:%
%:%2033=888%:%
%:%2034=889%:%
%:%2035=889%:%
%:%2036=889%:%
%:%2037=890%:%
%:%2038=891%:%
%:%2039=891%:%
%:%2040=892%:%
%:%2041=892%:%
%:%2042=893%:%
%:%2043=894%:%
%:%2044=894%:%
%:%2045=894%:%
%:%2046=894%:%
%:%2047=895%:%
%:%2048=895%:%
%:%2049=896%:%
%:%2050=896%:%
%:%2051=896%:%
%:%2052=897%:%
%:%2053=897%:%
%:%2054=898%:%
%:%2055=898%:%
%:%2056=898%:%
%:%2057=899%:%
%:%2058=900%:%
%:%2059=900%:%
%:%2060=901%:%
%:%2061=901%:%
%:%2062=902%:%
%:%2063=902%:%
%:%2064=903%:%
%:%2065=903%:%
%:%2066=904%:%
%:%2067=904%:%
%:%2068=905%:%
%:%2069=905%:%
%:%2070=905%:%
%:%2071=906%:%
%:%2072=906%:%
%:%2073=906%:%
%:%2074=907%:%
%:%2075=907%:%
%:%2076=908%:%
%:%2077=908%:%
%:%2078=908%:%
%:%2079=909%:%
%:%2080=909%:%
%:%2081=910%:%
%:%2082=910%:%
%:%2083=911%:%
%:%2084=911%:%
%:%2085=911%:%
%:%2086=912%:%
%:%2087=912%:%
%:%2088=912%:%
%:%2089=912%:%
%:%2090=913%:%
%:%2091=913%:%
%:%2092=913%:%
%:%2093=914%:%
%:%2094=914%:%
%:%2095=914%:%
%:%2096=915%:%
%:%2097=915%:%
%:%2098=916%:%
%:%2099=916%:%
%:%2100=917%:%
%:%2101=917%:%
%:%2102=918%:%
%:%2103=918%:%
%:%2104=919%:%
%:%2105=919%:%
%:%2106=920%:%
%:%2107=920%:%
%:%2108=921%:%
%:%2109=921%:%
%:%2110=922%:%
%:%2111=922%:%
%:%2112=923%:%
%:%2113=923%:%
%:%2114=924%:%
%:%2115=924%:%
%:%2116=925%:%
%:%2117=925%:%
%:%2118=926%:%
%:%2119=926%:%
%:%2120=927%:%
%:%2121=927%:%
%:%2122=927%:%
%:%2123=928%:%
%:%2124=928%:%
%:%2125=928%:%
%:%2126=929%:%
%:%2127=929%:%
%:%2128=930%:%
%:%2129=930%:%
%:%2130=931%:%
%:%2131=931%:%
%:%2132=932%:%
%:%2133=932%:%
%:%2134=933%:%
%:%2135=933%:%
%:%2136=933%:%
%:%2137=933%:%
%:%2138=933%:%
%:%2139=934%:%
%:%2140=934%:%
%:%2141=934%:%
%:%2142=934%:%
%:%2143=935%:%
%:%2144=935%:%
%:%2145=936%:%
%:%2146=936%:%
%:%2147=937%:%
%:%2148=937%:%
%:%2149=938%:%
%:%2150=938%:%
%:%2151=938%:%
%:%2152=938%:%
%:%2153=939%:%
%:%2154=939%:%
%:%2155=939%:%
%:%2156=939%:%
%:%2157=940%:%
%:%2158=940%:%
%:%2159=941%:%
%:%2160=942%:%
%:%2161=942%:%
%:%2164=945%:%
%:%2165=946%:%
%:%2166=946%:%
%:%2167=947%:%
%:%2168=947%:%
%:%2169=948%:%
%:%2170=948%:%
%:%2171=949%:%
%:%2172=949%:%
%:%2173=949%:%
%:%2174=950%:%
%:%2175=951%:%
%:%2176=951%:%
%:%2177=952%:%
%:%2178=952%:%
%:%2179=953%:%
%:%2180=953%:%
%:%2181=954%:%
%:%2182=954%:%
%:%2183=955%:%
%:%2184=955%:%
%:%2185=956%:%
%:%2186=956%:%
%:%2187=957%:%
%:%2188=957%:%
%:%2189=957%:%
%:%2190=958%:%
%:%2191=958%:%
%:%2192=959%:%
%:%2193=959%:%
%:%2194=960%:%
%:%2195=960%:%
%:%2196=961%:%
%:%2197=961%:%
%:%2198=962%:%
%:%2199=962%:%
%:%2200=963%:%
%:%2201=963%:%
%:%2202=964%:%
%:%2203=964%:%
%:%2204=964%:%
%:%2205=965%:%
%:%2206=965%:%
%:%2207=965%:%
%:%2208=966%:%
%:%2209=966%:%
%:%2210=967%:%
%:%2211=967%:%
%:%2212=968%:%
%:%2213=968%:%
%:%2214=969%:%
%:%2215=969%:%
%:%2216=970%:%
%:%2217=970%:%
%:%2218=971%:%
%:%2219=971%:%
%:%2220=971%:%
%:%2221=972%:%
%:%2222=972%:%
%:%2223=973%:%
%:%2224=973%:%
%:%2225=974%:%
%:%2226=974%:%
%:%2227=975%:%
%:%2228=975%:%
%:%2229=976%:%
%:%2230=976%:%
%:%2231=976%:%
%:%2232=977%:%
%:%2233=977%:%
%:%2234=977%:%
%:%2235=978%:%
%:%2236=978%:%
%:%2237=978%:%
%:%2238=979%:%
%:%2239=979%:%
%:%2240=980%:%
%:%2241=980%:%
%:%2242=981%:%
%:%2243=981%:%
%:%2244=981%:%
%:%2245=982%:%
%:%2246=982%:%
%:%2247=982%:%
%:%2248=983%:%
%:%2249=983%:%
%:%2250=983%:%
%:%2251=983%:%
%:%2252=984%:%
%:%2253=984%:%
%:%2254=984%:%
%:%2255=985%:%
%:%2256=985%:%
%:%2257=986%:%
%:%2258=987%:%
%:%2259=987%:%
%:%2261=989%:%
%:%2262=990%:%
%:%2263=990%:%
%:%2264=991%:%
%:%2265=991%:%
%:%2266=992%:%
%:%2267=992%:%
%:%2268=993%:%
%:%2269=993%:%
%:%2270=994%:%
%:%2271=994%:%
%:%2272=995%:%
%:%2273=995%:%
%:%2274=995%:%
%:%2275=996%:%
%:%2276=997%:%
%:%2277=997%:%
%:%2278=998%:%
%:%2279=998%:%
%:%2280=999%:%
%:%2281=1000%:%
%:%2282=1000%:%
%:%2283=1001%:%
%:%2284=1001%:%
%:%2285=1002%:%
%:%2286=1002%:%
%:%2287=1003%:%
%:%2288=1003%:%
%:%2289=1003%:%
%:%2290=1004%:%
%:%2291=1004%:%
%:%2292=1005%:%
%:%2293=1005%:%
%:%2294=1006%:%
%:%2295=1006%:%
%:%2296=1007%:%
%:%2297=1007%:%
%:%2298=1008%:%
%:%2299=1009%:%
%:%2300=1009%:%
%:%2301=1010%:%
%:%2302=1010%:%
%:%2303=1011%:%
%:%2304=1011%:%
%:%2305=1011%:%
%:%2306=1011%:%
%:%2307=1012%:%
%:%2308=1012%:%
%:%2309=1013%:%
%:%2310=1013%:%
%:%2311=1014%:%
%:%2312=1015%:%
%:%2313=1015%:%
%:%2314=1015%:%
%:%2315=1015%:%
%:%2316=1016%:%
%:%2317=1016%:%
%:%2318=1017%:%
%:%2319=1017%:%
%:%2320=1018%:%
%:%2321=1018%:%
%:%2322=1019%:%
%:%2323=1019%:%
%:%2324=1019%:%
%:%2325=1019%:%
%:%2326=1020%:%
%:%2327=1020%:%
%:%2328=1021%:%
%:%2329=1021%:%
%:%2330=1022%:%
%:%2331=1022%:%
%:%2332=1023%:%
%:%2333=1023%:%
%:%2334=1023%:%
%:%2335=1024%:%
%:%2336=1024%:%
%:%2337=1025%:%
%:%2338=1025%:%
%:%2339=1026%:%
%:%2340=1026%:%
%:%2341=1027%:%
%:%2342=1027%:%
%:%2343=1027%:%
%:%2344=1028%:%
%:%2345=1028%:%
%:%2346=1029%:%
%:%2347=1029%:%
%:%2348=1029%:%
%:%2349=1030%:%
%:%2350=1030%:%
%:%2351=1030%:%
%:%2352=1030%:%
%:%2353=1031%:%
%:%2354=1032%:%
%:%2355=1032%:%
%:%2356=1033%:%
%:%2357=1033%:%
%:%2358=1034%:%
%:%2359=1034%:%
%:%2360=1034%:%
%:%2361=1035%:%
%:%2362=1035%:%
%:%2363=1035%:%
%:%2364=1036%:%
%:%2365=1036%:%
%:%2366=1036%:%
%:%2367=1037%:%
%:%2368=1037%:%
%:%2369=1038%:%
%:%2370=1039%:%
%:%2371=1039%:%
%:%2372=1040%:%
%:%2373=1040%:%
%:%2374=1041%:%
%:%2375=1041%:%
%:%2376=1041%:%
%:%2377=1042%:%
%:%2378=1042%:%
%:%2379=1043%:%
%:%2380=1043%:%
%:%2381=1044%:%
%:%2382=1044%:%
%:%2383=1045%:%
%:%2384=1045%:%
%:%2385=1045%:%
%:%2386=1046%:%
%:%2387=1046%:%
%:%2388=1046%:%
%:%2389=1046%:%
%:%2390=1047%:%
%:%2391=1048%:%
%:%2392=1048%:%
%:%2393=1049%:%
%:%2394=1049%:%
%:%2395=1050%:%
%:%2396=1050%:%
%:%2397=1050%:%
%:%2398=1051%:%
%:%2399=1051%:%
%:%2400=1052%:%
%:%2401=1053%:%
%:%2402=1053%:%
%:%2403=1054%:%
%:%2404=1054%:%
%:%2405=1055%:%
%:%2406=1055%:%
%:%2407=1056%:%
%:%2408=1056%:%
%:%2409=1056%:%
%:%2410=1057%:%
%:%2411=1057%:%
%:%2412=1057%:%
%:%2413=1058%:%
%:%2414=1058%:%
%:%2415=1059%:%
%:%2416=1059%:%
%:%2417=1060%:%
%:%2418=1061%:%
%:%2419=1061%:%
%:%2420=1062%:%
%:%2421=1062%:%
%:%2422=1063%:%
%:%2423=1063%:%
%:%2424=1064%:%
%:%2425=1064%:%
%:%2426=1064%:%
%:%2427=1065%:%
%:%2428=1065%:%
%:%2429=1066%:%
%:%2430=1066%:%
%:%2431=1066%:%
%:%2432=1067%:%
%:%2433=1067%:%
%:%2434=1067%:%
%:%2435=1068%:%
%:%2436=1068%:%
%:%2437=1068%:%
%:%2438=1068%:%
%:%2439=1068%:%
%:%2440=1069%:%
%:%2441=1069%:%
%:%2442=1069%:%
%:%2443=1070%:%
%:%2444=1070%:%
%:%2445=1071%:%
%:%2446=1071%:%
%:%2447=1072%:%
%:%2448=1072%:%
%:%2449=1072%:%
%:%2450=1073%:%
%:%2451=1074%:%
%:%2452=1074%:%
%:%2453=1075%:%
%:%2454=1075%:%
%:%2455=1076%:%
%:%2456=1076%:%
%:%2457=1076%:%
%:%2458=1077%:%
%:%2459=1077%:%
%:%2460=1077%:%
%:%2461=1078%:%
%:%2462=1078%:%
%:%2463=1079%:%
%:%2464=1079%:%
%:%2465=1080%:%
%:%2466=1080%:%
%:%2467=1080%:%
%:%2468=1081%:%
%:%2469=1081%:%
%:%2470=1081%:%
%:%2471=1082%:%
%:%2472=1082%:%
%:%2473=1083%:%
%:%2474=1083%:%
%:%2475=1084%:%
%:%2476=1084%:%
%:%2477=1085%:%
%:%2478=1085%:%
%:%2479=1085%:%
%:%2480=1086%:%
%:%2481=1086%:%
%:%2482=1086%:%
%:%2483=1087%:%
%:%2484=1087%:%
%:%2485=1088%:%
%:%2486=1088%:%
%:%2487=1089%:%
%:%2488=1089%:%
%:%2489=1090%:%
%:%2490=1090%:%
%:%2491=1090%:%
%:%2492=1091%:%
%:%2493=1091%:%
%:%2494=1091%:%
%:%2495=1092%:%
%:%2496=1092%:%
%:%2497=1093%:%
%:%2498=1093%:%
%:%2499=1093%:%
%:%2500=1094%:%
%:%2501=1094%:%
%:%2502=1095%:%
%:%2503=1095%:%
%:%2504=1095%:%
%:%2505=1096%:%
%:%2506=1096%:%
%:%2507=1096%:%
%:%2508=1096%:%
%:%2509=1097%:%
%:%2510=1097%:%
%:%2511=1098%:%
%:%2512=1098%:%
%:%2513=1099%:%
%:%2514=1099%:%
%:%2515=1100%:%
%:%2516=1100%:%
%:%2517=1101%:%
%:%2518=1101%:%
%:%2519=1102%:%
%:%2520=1102%:%
%:%2521=1102%:%
%:%2522=1103%:%
%:%2523=1103%:%
%:%2524=1103%:%
%:%2525=1104%:%
%:%2526=1104%:%
%:%2527=1105%:%
%:%2528=1105%:%
%:%2529=1105%:%
%:%2530=1106%:%
%:%2531=1106%:%
%:%2532=1107%:%
%:%2533=1107%:%
%:%2534=1108%:%
%:%2535=1108%:%
%:%2536=1108%:%
%:%2537=1109%:%
%:%2538=1109%:%
%:%2539=1109%:%
%:%2540=1110%:%
%:%2541=1110%:%
%:%2542=1110%:%
%:%2543=1111%:%
%:%2544=1111%:%
%:%2545=1112%:%
%:%2546=1112%:%
%:%2547=1113%:%
%:%2548=1113%:%
%:%2549=1114%:%
%:%2550=1114%:%
%:%2551=1114%:%
%:%2552=1115%:%
%:%2553=1115%:%
%:%2554=1115%:%
%:%2555=1116%:%
%:%2556=1116%:%
%:%2557=1117%:%
%:%2558=1117%:%
%:%2559=1117%:%
%:%2560=1118%:%
%:%2561=1118%:%
%:%2562=1118%:%
%:%2563=1119%:%
%:%2564=1119%:%
%:%2565=1119%:%
%:%2566=1119%:%
%:%2567=1120%:%
%:%2568=1120%:%
%:%2569=1121%:%
%:%2570=1121%:%
%:%2571=1122%:%
%:%2572=1122%:%
%:%2573=1123%:%
%:%2574=1123%:%
%:%2575=1124%:%
%:%2576=1124%:%
%:%2577=1124%:%
%:%2578=1125%:%
%:%2579=1125%:%
%:%2580=1126%:%
%:%2581=1126%:%
%:%2582=1127%:%
%:%2583=1127%:%
%:%2584=1127%:%
%:%2585=1128%:%
%:%2586=1128%:%
%:%2587=1129%:%
%:%2588=1129%:%
%:%2589=1129%:%
%:%2590=1130%:%
%:%2591=1130%:%
%:%2592=1131%:%
%:%2593=1131%:%
%:%2594=1132%:%
%:%2595=1132%:%
%:%2596=1133%:%
%:%2597=1133%:%
%:%2598=1134%:%
%:%2599=1134%:%
%:%2600=1134%:%
%:%2601=1134%:%
%:%2602=1135%:%
%:%2603=1135%:%
%:%2604=1135%:%
%:%2605=1135%:%
%:%2606=1136%:%
%:%2607=1136%:%
%:%2608=1137%:%
%:%2609=1137%:%
%:%2610=1138%:%
%:%2611=1138%:%
%:%2614=1141%:%
%:%2615=1142%:%
%:%2616=1142%:%
%:%2617=1143%:%
%:%2618=1143%:%
%:%2619=1144%:%
%:%2620=1144%:%
%:%2621=1145%:%
%:%2622=1146%:%
%:%2623=1146%:%
%:%2624=1147%:%
%:%2625=1147%:%
%:%2626=1147%:%
%:%2627=1147%:%
%:%2628=1148%:%
%:%2629=1148%:%
%:%2630=1149%:%
%:%2631=1149%:%
%:%2632=1150%:%
%:%2633=1150%:%
%:%2634=1151%:%
%:%2635=1151%:%
%:%2636=1152%:%
%:%2637=1152%:%
%:%2638=1152%:%
%:%2639=1153%:%
%:%2640=1153%:%
%:%2641=1154%:%
%:%2642=1154%:%
%:%2643=1155%:%
%:%2644=1155%:%
%:%2645=1155%:%
%:%2646=1156%:%
%:%2647=1156%:%
%:%2648=1157%:%
%:%2649=1157%:%
%:%2650=1157%:%
%:%2651=1158%:%
%:%2652=1158%:%
%:%2653=1159%:%
%:%2654=1159%:%
%:%2655=1160%:%
%:%2656=1160%:%
%:%2657=1160%:%
%:%2658=1161%:%
%:%2659=1161%:%
%:%2660=1162%:%
%:%2661=1162%:%
%:%2662=1163%:%
%:%2663=1163%:%
%:%2664=1163%:%
%:%2665=1164%:%
%:%2666=1164%:%
%:%2667=1164%:%
%:%2668=1165%:%
%:%2669=1165%:%
%:%2670=1166%:%
%:%2671=1166%:%
%:%2672=1167%:%
%:%2673=1167%:%
%:%2674=1168%:%
%:%2675=1168%:%
%:%2676=1169%:%
%:%2677=1169%:%
%:%2678=1170%:%
%:%2679=1170%:%
%:%2680=1171%:%
%:%2681=1171%:%
%:%2682=1172%:%
%:%2683=1172%:%
%:%2684=1172%:%
%:%2685=1173%:%
%:%2686=1173%:%
%:%2687=1174%:%
%:%2688=1174%:%
%:%2689=1174%:%
%:%2691=1176%:%
%:%2692=1177%:%
%:%2693=1177%:%
%:%2694=1177%:%
%:%2695=1178%:%
%:%2696=1178%:%
%:%2697=1179%:%
%:%2698=1179%:%
%:%2699=1179%:%
%:%2700=1180%:%
%:%2701=1180%:%
%:%2702=1181%:%
%:%2703=1181%:%
%:%2704=1182%:%
%:%2705=1182%:%
%:%2706=1183%:%
%:%2707=1183%:%
%:%2708=1184%:%
%:%2709=1184%:%
%:%2710=1184%:%
%:%2711=1185%:%
%:%2712=1185%:%
%:%2713=1185%:%
%:%2714=1186%:%
%:%2715=1186%:%
%:%2716=1187%:%
%:%2717=1187%:%
%:%2718=1187%:%
%:%2719=1188%:%
%:%2720=1188%:%
%:%2721=1188%:%
%:%2722=1189%:%
%:%2723=1189%:%
%:%2724=1190%:%
%:%2725=1190%:%
%:%2726=1191%:%
%:%2727=1191%:%
%:%2728=1192%:%
%:%2729=1192%:%
%:%2730=1192%:%
%:%2731=1192%:%
%:%2732=1193%:%
%:%2733=1193%:%
%:%2734=1194%:%
%:%2735=1194%:%
%:%2736=1194%:%
%:%2737=1195%:%
%:%2738=1195%:%
%:%2739=1196%:%
%:%2740=1196%:%
%:%2741=1197%:%
%:%2742=1197%:%
%:%2743=1197%:%
%:%2744=1198%:%
%:%2745=1198%:%
%:%2746=1198%:%
%:%2747=1199%:%
%:%2748=1199%:%
%:%2749=1199%:%
%:%2750=1199%:%
%:%2751=1200%:%
%:%2752=1200%:%
%:%2753=1200%:%
%:%2754=1200%:%
%:%2755=1201%:%
%:%2756=1201%:%
%:%2757=1202%:%
%:%2758=1203%:%
%:%2759=1203%:%
%:%2760=1204%:%
%:%2761=1205%:%
%:%2762=1205%:%
%:%2763=1206%:%
%:%2764=1206%:%
%:%2765=1207%:%
%:%2766=1207%:%
%:%2767=1208%:%
%:%2768=1209%:%
%:%2769=1209%:%
%:%2770=1210%:%
%:%2771=1210%:%
%:%2772=1211%:%
%:%2773=1211%:%
%:%2774=1212%:%
%:%2775=1212%:%
%:%2776=1213%:%
%:%2777=1213%:%
%:%2778=1214%:%
%:%2779=1215%:%
%:%2780=1215%:%
%:%2781=1216%:%
%:%2782=1216%:%
%:%2783=1217%:%
%:%2784=1217%:%
%:%2785=1218%:%
%:%2786=1218%:%
%:%2787=1218%:%
%:%2788=1219%:%
%:%2789=1220%:%
%:%2790=1220%:%
%:%2791=1221%:%
%:%2792=1221%:%
%:%2793=1222%:%
%:%2794=1222%:%
%:%2795=1223%:%
%:%2796=1223%:%
%:%2797=1224%:%
%:%2798=1224%:%
%:%2799=1224%:%
%:%2800=1225%:%
%:%2801=1225%:%
%:%2802=1226%:%
%:%2803=1226%:%
%:%2804=1226%:%
%:%2805=1227%:%
%:%2806=1227%:%
%:%2807=1228%:%
%:%2808=1229%:%
%:%2809=1229%:%
%:%2810=1230%:%
%:%2811=1231%:%
%:%2812=1231%:%
%:%2813=1232%:%
%:%2814=1232%:%
%:%2815=1232%:%
%:%2816=1233%:%
%:%2817=1233%:%
%:%2818=1234%:%
%:%2819=1234%:%
%:%2820=1235%:%
%:%2821=1235%:%
%:%2822=1235%:%
%:%2823=1235%:%
%:%2824=1236%:%
%:%2825=1236%:%
%:%2826=1237%:%
%:%2827=1237%:%
%:%2828=1238%:%
%:%2829=1238%:%
%:%2830=1239%:%
%:%2831=1239%:%
%:%2832=1239%:%
%:%2833=1240%:%
%:%2834=1240%:%
%:%2835=1240%:%
%:%2836=1241%:%
%:%2837=1242%:%
%:%2838=1242%:%
%:%2839=1243%:%
%:%2840=1243%:%
%:%2841=1244%:%
%:%2842=1244%:%
%:%2843=1245%:%
%:%2844=1245%:%
%:%2845=1246%:%
%:%2846=1246%:%
%:%2847=1246%:%
%:%2848=1247%:%
%:%2849=1247%:%
%:%2850=1247%:%
%:%2851=1248%:%
%:%2852=1249%:%
%:%2853=1249%:%
%:%2854=1250%:%
%:%2855=1251%:%
%:%2856=1251%:%
%:%2857=1252%:%
%:%2858=1252%:%
%:%2859=1253%:%
%:%2860=1253%:%
%:%2861=1254%:%
%:%2862=1254%:%
%:%2863=1255%:%
%:%2864=1255%:%
%:%2865=1256%:%
%:%2866=1256%:%
%:%2867=1257%:%
%:%2868=1257%:%
%:%2869=1258%:%
%:%2870=1258%:%
%:%2871=1259%:%
%:%2872=1259%:%
%:%2873=1259%:%
%:%2874=1260%:%
%:%2880=1260%:%
%:%2883=1261%:%
%:%2884=1262%:%
%:%2885=1262%:%
%:%2886=1263%:%
%:%2896=1273%:%
%:%2897=1274%:%
%:%2898=1275%:%
%:%2899=1275%:%
%:%2900=1276%:%
%:%2906=1282%:%
%:%2907=1283%:%
%:%2908=1284%:%
%:%2909=1284%:%
%:%2912=1285%:%
%:%2916=1285%:%
%:%2917=1285%:%
%:%2918=1286%:%
%:%2919=1286%:%
%:%2920=1287%:%
%:%2921=1287%:%
%:%2922=1288%:%
%:%2923=1288%:%
%:%2924=1289%:%
%:%2925=1289%:%
%:%2926=1290%:%
%:%2927=1290%:%
%:%2928=1291%:%
%:%2929=1291%:%
%:%2930=1292%:%
%:%2931=1292%:%
%:%2936=1292%:%
%:%2939=1293%:%
%:%2940=1294%:%
%:%2941=1294%:%
%:%2942=1295%:%
%:%2943=1296%:%
%:%2946=1297%:%
%:%2950=1297%:%
%:%2951=1297%:%
%:%2952=1297%:%
%:%2957=1297%:%
%:%2960=1298%:%
%:%2961=1299%:%
%:%2962=1299%:%
%:%2963=1300%:%
%:%2964=1301%:%
%:%2965=1302%:%
%:%2966=1303%:%
%:%2967=1304%:%
%:%2974=1305%:%
%:%2975=1305%:%
%:%2976=1306%:%
%:%2977=1306%:%
%:%2978=1307%:%
%:%2979=1307%:%
%:%2980=1308%:%
%:%2981=1308%:%
%:%2982=1309%:%
%:%2983=1309%:%
%:%2984=1310%:%
%:%2985=1311%:%
%:%2986=1311%:%
%:%2987=1312%:%
%:%2988=1312%:%
%:%2989=1313%:%
%:%2990=1313%:%
%:%2991=1314%:%
%:%2992=1314%:%
%:%2993=1315%:%
%:%2994=1316%:%
%:%2995=1316%:%
%:%2996=1317%:%
%:%2997=1317%:%
%:%2998=1318%:%
%:%2999=1318%:%
%:%3000=1319%:%
%:%3001=1320%:%
%:%3002=1320%:%
%:%3003=1321%:%
%:%3004=1321%:%
%:%3005=1322%:%
%:%3006=1322%:%
%:%3007=1323%:%
%:%3008=1323%:%
%:%3009=1324%:%
%:%3010=1324%:%
%:%3011=1325%:%
%:%3012=1326%:%
%:%3013=1326%:%
%:%3014=1327%:%
%:%3015=1327%:%
%:%3016=1328%:%
%:%3017=1329%:%
%:%3018=1329%:%
%:%3019=1330%:%
%:%3020=1331%:%
%:%3021=1331%:%
%:%3022=1332%:%
%:%3023=1332%:%
%:%3024=1333%:%
%:%3025=1333%:%
%:%3026=1334%:%
%:%3027=1335%:%
%:%3028=1335%:%
%:%3029=1336%:%
%:%3030=1336%:%
%:%3031=1337%:%
%:%3032=1337%:%
%:%3033=1338%:%
%:%3034=1339%:%
%:%3035=1339%:%
%:%3036=1340%:%
%:%3037=1340%:%
%:%3038=1341%:%
%:%3039=1341%:%
%:%3040=1341%:%
%:%3041=1342%:%
%:%3042=1343%:%
%:%3043=1343%:%
%:%3044=1344%:%
%:%3045=1344%:%
%:%3046=1345%:%
%:%3047=1345%:%
%:%3048=1345%:%
%:%3049=1346%:%
%:%3050=1346%:%
%:%3051=1347%:%
%:%3052=1347%:%
%:%3053=1348%:%
%:%3054=1348%:%
%:%3055=1348%:%
%:%3056=1348%:%
%:%3057=1349%:%
%:%3058=1349%:%
%:%3059=1349%:%
%:%3060=1349%:%
%:%3061=1350%:%
%:%3062=1350%:%
%:%3063=1351%:%
%:%3064=1351%:%
%:%3065=1351%:%
%:%3066=1352%:%
%:%3067=1352%:%
%:%3068=1352%:%
%:%3069=1353%:%
%:%3070=1353%:%
%:%3071=1354%:%
%:%3072=1354%:%
%:%3073=1355%:%
%:%3074=1355%:%
%:%3075=1356%:%
%:%3076=1356%:%
%:%3077=1357%:%
%:%3078=1357%:%
%:%3079=1358%:%
%:%3080=1358%:%
%:%3081=1359%:%
%:%3082=1359%:%
%:%3083=1359%:%
%:%3084=1360%:%
%:%3085=1360%:%
%:%3086=1360%:%
%:%3087=1361%:%
%:%3088=1361%:%
%:%3089=1362%:%
%:%3090=1362%:%
%:%3091=1363%:%
%:%3092=1363%:%
%:%3093=1364%:%
%:%3094=1364%:%
%:%3095=1365%:%
%:%3096=1365%:%
%:%3097=1365%:%
%:%3098=1366%:%
%:%3099=1366%:%
%:%3100=1367%:%
%:%3101=1367%:%
%:%3102=1368%:%
%:%3103=1368%:%
%:%3104=1369%:%
%:%3105=1369%:%
%:%3106=1370%:%
%:%3107=1370%:%
%:%3108=1370%:%
%:%3109=1370%:%
%:%3110=1371%:%
%:%3111=1371%:%
%:%3112=1372%:%
%:%3113=1372%:%
%:%3114=1373%:%
%:%3115=1373%:%
%:%3116=1373%:%
%:%3117=1373%:%
%:%3118=1374%:%
%:%3119=1374%:%
%:%3120=1374%:%
%:%3121=1374%:%
%:%3122=1375%:%
%:%3123=1375%:%
%:%3124=1376%:%
%:%3125=1376%:%
%:%3126=1377%:%
%:%3127=1378%:%
%:%3128=1378%:%
%:%3129=1379%:%
%:%3131=1381%:%
%:%3132=1382%:%
%:%3133=1382%:%
%:%3134=1383%:%
%:%3135=1383%:%
%:%3136=1384%:%
%:%3137=1384%:%
%:%3138=1385%:%
%:%3139=1385%:%
%:%3140=1385%:%
%:%3141=1385%:%
%:%3142=1386%:%
%:%3143=1387%:%
%:%3144=1387%:%
%:%3145=1388%:%
%:%3146=1388%:%
%:%3147=1389%:%
%:%3148=1389%:%
%:%3149=1390%:%
%:%3150=1390%:%
%:%3151=1391%:%
%:%3152=1392%:%
%:%3153=1392%:%
%:%3154=1393%:%
%:%3155=1393%:%
%:%3156=1394%:%
%:%3157=1395%:%
%:%3158=1395%:%
%:%3159=1396%:%
%:%3160=1396%:%
%:%3161=1397%:%
%:%3162=1397%:%
%:%3163=1398%:%
%:%3164=1398%:%
%:%3165=1399%:%
%:%3166=1399%:%
%:%3167=1400%:%
%:%3168=1400%:%
%:%3169=1401%:%
%:%3170=1401%:%
%:%3171=1401%:%
%:%3172=1402%:%
%:%3173=1402%:%
%:%3174=1403%:%
%:%3175=1403%:%
%:%3176=1404%:%
%:%3177=1405%:%
%:%3178=1405%:%
%:%3179=1406%:%
%:%3180=1407%:%
%:%3181=1407%:%
%:%3182=1408%:%
%:%3183=1409%:%
%:%3184=1409%:%
%:%3185=1409%:%
%:%3186=1410%:%
%:%3187=1411%:%
%:%3188=1411%:%
%:%3189=1412%:%
%:%3190=1412%:%
%:%3191=1413%:%
%:%3192=1413%:%
%:%3193=1414%:%
%:%3194=1414%:%
%:%3195=1414%:%
%:%3196=1415%:%
%:%3197=1416%:%
%:%3198=1416%:%
%:%3201=1419%:%
%:%3202=1420%:%
%:%3203=1420%:%
%:%3204=1421%:%
%:%3205=1422%:%
%:%3206=1423%:%
%:%3207=1423%:%
%:%3208=1424%:%
%:%3209=1424%:%
%:%3210=1424%:%
%:%3214=1428%:%
%:%3215=1429%:%
%:%3216=1429%:%
%:%3217=1430%:%
%:%3218=1430%:%
%:%3219=1431%:%
%:%3220=1431%:%
%:%3221=1432%:%
%:%3222=1432%:%
%:%3223=1433%:%
%:%3224=1433%:%
%:%3225=1434%:%
%:%3226=1434%:%
%:%3227=1435%:%
%:%3228=1435%:%
%:%3229=1435%:%
%:%3230=1435%:%
%:%3231=1436%:%
%:%3232=1436%:%
%:%3233=1437%:%
%:%3234=1437%:%
%:%3235=1438%:%
%:%3236=1438%:%
%:%3237=1439%:%
%:%3238=1439%:%
%:%3239=1440%:%
%:%3240=1440%:%
%:%3241=1440%:%
%:%3243=1442%:%
%:%3244=1443%:%
%:%3245=1443%:%
%:%3246=1444%:%
%:%3247=1444%:%
%:%3248=1445%:%
%:%3249=1445%:%
%:%3250=1445%:%
%:%3252=1447%:%
%:%3253=1448%:%
%:%3254=1448%:%
%:%3255=1448%:%
%:%3256=1448%:%
%:%3257=1449%:%
%:%3258=1449%:%
%:%3259=1450%:%
%:%3260=1450%:%
%:%3261=1451%:%
%:%3262=1451%:%
%:%3263=1451%:%
%:%3264=1452%:%
%:%3265=1452%:%
%:%3266=1453%:%
%:%3267=1453%:%
%:%3268=1453%:%
%:%3269=1454%:%
%:%3270=1454%:%
%:%3271=1455%:%
%:%3272=1455%:%
%:%3273=1456%:%
%:%3274=1456%:%
%:%3275=1457%:%
%:%3276=1457%:%
%:%3277=1458%:%
%:%3278=1458%:%
%:%3279=1458%:%
%:%3280=1459%:%
%:%3281=1459%:%
%:%3282=1460%:%
%:%3283=1460%:%
%:%3284=1461%:%
%:%3285=1462%:%
%:%3286=1462%:%
%:%3288=1464%:%
%:%3289=1465%:%
%:%3290=1465%:%
%:%3291=1465%:%
%:%3292=1465%:%
%:%3293=1466%:%
%:%3294=1467%:%
%:%3295=1467%:%
%:%3296=1468%:%
%:%3297=1468%:%
%:%3298=1469%:%
%:%3299=1469%:%
%:%3300=1470%:%
%:%3301=1470%:%
%:%3302=1471%:%
%:%3303=1471%:%
%:%3304=1472%:%
%:%3305=1472%:%
%:%3306=1473%:%
%:%3307=1473%:%
%:%3308=1474%:%
%:%3309=1474%:%
%:%3310=1475%:%
%:%3311=1475%:%
%:%3312=1476%:%
%:%3313=1476%:%
%:%3314=1477%:%
%:%3315=1477%:%
%:%3316=1478%:%
%:%3317=1478%:%
%:%3318=1479%:%
%:%3324=1479%:%
%:%3327=1480%:%
%:%3328=1481%:%
%:%3329=1481%:%
%:%3330=1482%:%
%:%3331=1483%:%
%:%3332=1484%:%
%:%3339=1485%:%
%:%3340=1485%:%
%:%3341=1486%:%
%:%3342=1486%:%
%:%3343=1487%:%
%:%3344=1487%:%
%:%3345=1488%:%
%:%3346=1488%:%
%:%3347=1489%:%
%:%3348=1490%:%
%:%3349=1490%:%
%:%3350=1491%:%
%:%3351=1492%:%
%:%3352=1493%:%
%:%3353=1493%:%
%:%3354=1494%:%
%:%3355=1494%:%
%:%3356=1495%:%
%:%3357=1495%:%
%:%3358=1496%:%
%:%3359=1496%:%
%:%3360=1497%:%
%:%3361=1498%:%
%:%3362=1498%:%
%:%3363=1499%:%
%:%3364=1499%:%
%:%3365=1500%:%
%:%3366=1500%:%
%:%3367=1501%:%
%:%3368=1501%:%
%:%3369=1502%:%
%:%3370=1502%:%
%:%3371=1503%:%
%:%3372=1504%:%
%:%3373=1504%:%
%:%3374=1505%:%
%:%3375=1505%:%
%:%3376=1506%:%
%:%3377=1506%:%
%:%3378=1507%:%
%:%3379=1507%:%
%:%3380=1508%:%
%:%3381=1508%:%
%:%3382=1509%:%
%:%3383=1510%:%
%:%3384=1510%:%
%:%3385=1511%:%
%:%3386=1511%:%
%:%3387=1512%:%
%:%3388=1512%:%
%:%3389=1513%:%
%:%3390=1513%:%
%:%3391=1514%:%
%:%3392=1514%:%
%:%3393=1515%:%
%:%3394=1516%:%
%:%3395=1516%:%
%:%3396=1517%:%
%:%3397=1517%:%
%:%3398=1518%:%
%:%3399=1518%:%
%:%3400=1519%:%
%:%3401=1519%:%
%:%3402=1520%:%
%:%3403=1520%:%
%:%3404=1521%:%
%:%3405=1522%:%
%:%3406=1522%:%
%:%3407=1523%:%
%:%3408=1523%:%
%:%3409=1524%:%
%:%3410=1524%:%
%:%3411=1525%:%
%:%3412=1525%:%
%:%3413=1526%:%
%:%3414=1526%:%
%:%3415=1527%:%
%:%3416=1527%:%
%:%3417=1528%:%
%:%3418=1529%:%
%:%3419=1529%:%
%:%3420=1530%:%
%:%3421=1530%:%
%:%3422=1531%:%
%:%3423=1531%:%
%:%3424=1532%:%
%:%3425=1532%:%
%:%3426=1533%:%
%:%3427=1534%:%
%:%3428=1534%:%
%:%3429=1535%:%
%:%3430=1535%:%
%:%3431=1536%:%
%:%3432=1536%:%
%:%3433=1537%:%
%:%3434=1537%:%
%:%3435=1538%:%
%:%3436=1538%:%
%:%3437=1539%:%
%:%3438=1539%:%
%:%3439=1540%:%
%:%3440=1540%:%
%:%3441=1541%:%
%:%3442=1542%:%
%:%3443=1542%:%
%:%3444=1543%:%
%:%3445=1543%:%
%:%3446=1544%:%
%:%3447=1545%:%
%:%3448=1545%:%
%:%3449=1546%:%
%:%3450=1546%:%
%:%3451=1547%:%
%:%3452=1547%:%
%:%3453=1548%:%
%:%3454=1548%:%
%:%3455=1549%:%
%:%3456=1549%:%
%:%3457=1550%:%
%:%3458=1550%:%
%:%3459=1550%:%
%:%3460=1551%:%
%:%3461=1551%:%
%:%3462=1552%:%
%:%3463=1553%:%
%:%3464=1553%:%
%:%3465=1554%:%
%:%3466=1554%:%
%:%3467=1555%:%
%:%3468=1555%:%
%:%3469=1556%:%
%:%3470=1556%:%
%:%3471=1557%:%
%:%3472=1558%:%
%:%3473=1558%:%
%:%3474=1559%:%
%:%3475=1559%:%
%:%3476=1560%:%
%:%3477=1560%:%
%:%3478=1561%:%
%:%3479=1561%:%
%:%3480=1562%:%
%:%3481=1562%:%
%:%3482=1563%:%
%:%3483=1563%:%
%:%3484=1564%:%
%:%3485=1564%:%
%:%3486=1565%:%
%:%3487=1566%:%
%:%3488=1566%:%
%:%3489=1567%:%
%:%3490=1567%:%
%:%3491=1568%:%
%:%3492=1568%:%
%:%3493=1569%:%
%:%3494=1569%:%
%:%3495=1570%:%
%:%3496=1570%:%
%:%3497=1571%:%
%:%3498=1572%:%
%:%3499=1572%:%
%:%3500=1573%:%
%:%3501=1573%:%
%:%3502=1574%:%
%:%3503=1574%:%
%:%3504=1575%:%
%:%3505=1575%:%
%:%3506=1576%:%
%:%3507=1577%:%
%:%3508=1577%:%
%:%3509=1578%:%
%:%3510=1578%:%
%:%3511=1579%:%
%:%3512=1579%:%
%:%3513=1580%:%
%:%3514=1580%:%
%:%3515=1581%:%
%:%3516=1581%:%
%:%3517=1582%:%
%:%3518=1582%:%
%:%3519=1583%:%
%:%3520=1583%:%
%:%3521=1584%:%
%:%3522=1584%:%
%:%3523=1585%:%
%:%3524=1585%:%
%:%3525=1586%:%
%:%3526=1586%:%
%:%3527=1587%:%
%:%3528=1587%:%
%:%3529=1588%:%
%:%3530=1588%:%
%:%3531=1589%:%
%:%3532=1589%:%
%:%3533=1590%:%
%:%3534=1591%:%
%:%3535=1591%:%
%:%3536=1592%:%
%:%3537=1592%:%
%:%3538=1593%:%
%:%3539=1593%:%
%:%3540=1594%:%
%:%3541=1594%:%
%:%3542=1595%:%
%:%3543=1595%:%
%:%3544=1596%:%
%:%3545=1596%:%
%:%3546=1597%:%
%:%3547=1597%:%
%:%3548=1598%:%
%:%3549=1598%:%
%:%3550=1599%:%
%:%3551=1599%:%
%:%3552=1600%:%
%:%3553=1600%:%
%:%3554=1601%:%
%:%3555=1601%:%
%:%3556=1602%:%
%:%3557=1602%:%
%:%3558=1603%:%
%:%3559=1603%:%
%:%3560=1604%:%
%:%3561=1604%:%
%:%3562=1605%:%
%:%3563=1605%:%
%:%3564=1606%:%
%:%3565=1606%:%
%:%3566=1607%:%
%:%3567=1608%:%
%:%3568=1608%:%
%:%3569=1609%:%
%:%3570=1609%:%
%:%3571=1610%:%
%:%3572=1610%:%
%:%3573=1611%:%
%:%3574=1611%:%
%:%3575=1612%:%
%:%3576=1613%:%
%:%3577=1613%:%
%:%3578=1614%:%
%:%3579=1614%:%
%:%3580=1615%:%
%:%3581=1615%:%
%:%3582=1616%:%
%:%3583=1617%:%
%:%3584=1617%:%
%:%3585=1618%:%
%:%3586=1618%:%
%:%3587=1619%:%
%:%3588=1619%:%
%:%3589=1620%:%
%:%3590=1620%:%
%:%3591=1621%:%
%:%3592=1621%:%
%:%3593=1622%:%
%:%3594=1622%:%
%:%3595=1623%:%
%:%3596=1623%:%
%:%3597=1624%:%
%:%3598=1624%:%
%:%3599=1625%:%
%:%3600=1626%:%
%:%3601=1626%:%
%:%3602=1627%:%
%:%3603=1627%:%
%:%3604=1628%:%
%:%3605=1628%:%
%:%3606=1629%:%
%:%3607=1629%:%
%:%3608=1630%:%
%:%3609=1630%:%
%:%3610=1631%:%
%:%3611=1631%:%
%:%3612=1632%:%
%:%3613=1632%:%
%:%3614=1633%:%
%:%3615=1633%:%
%:%3616=1634%:%
%:%3617=1634%:%
%:%3618=1635%:%
%:%3619=1635%:%
%:%3620=1636%:%
%:%3621=1636%:%
%:%3622=1637%:%
%:%3623=1637%:%
%:%3624=1638%:%
%:%3625=1638%:%
%:%3626=1639%:%
%:%3627=1639%:%
%:%3628=1640%:%
%:%3629=1640%:%
%:%3630=1641%:%
%:%3631=1641%:%
%:%3632=1642%:%
%:%3633=1642%:%
%:%3634=1643%:%
%:%3635=1643%:%
%:%3636=1644%:%
%:%3637=1644%:%
%:%3638=1645%:%
%:%3639=1645%:%
%:%3640=1646%:%
%:%3641=1647%:%
%:%3642=1647%:%
%:%3643=1648%:%
%:%3644=1648%:%
%:%3645=1649%:%
%:%3646=1649%:%
%:%3647=1650%:%
%:%3648=1650%:%
%:%3649=1651%:%
%:%3650=1651%:%
%:%3651=1652%:%
%:%3652=1652%:%
%:%3653=1653%:%
%:%3654=1653%:%
%:%3655=1654%:%
%:%3656=1655%:%
%:%3657=1655%:%
%:%3660=1658%:%
%:%3661=1659%:%
%:%3662=1659%:%
%:%3663=1660%:%
%:%3664=1660%:%
%:%3665=1661%:%
%:%3666=1661%:%
%:%3667=1662%:%
%:%3668=1662%:%
%:%3669=1663%:%
%:%3670=1663%:%
%:%3671=1664%:%
%:%3672=1664%:%
%:%3673=1665%:%
%:%3674=1665%:%
%:%3675=1666%:%
%:%3676=1666%:%
%:%3677=1667%:%
%:%3678=1667%:%
%:%3679=1668%:%
%:%3680=1668%:%
%:%3681=1669%:%
%:%3682=1669%:%
%:%3683=1670%:%
%:%3684=1670%:%
%:%3685=1671%:%
%:%3686=1671%:%
%:%3687=1672%:%
%:%3688=1672%:%
%:%3689=1673%:%
%:%3690=1673%:%
%:%3691=1674%:%
%:%3692=1674%:%
%:%3693=1675%:%
%:%3694=1675%:%
%:%3695=1676%:%
%:%3696=1676%:%
%:%3697=1677%:%
%:%3698=1678%:%
%:%3699=1678%:%
%:%3700=1679%:%
%:%3701=1679%:%
%:%3702=1680%:%
%:%3703=1680%:%
%:%3704=1681%:%
%:%3705=1681%:%
%:%3706=1681%:%
%:%3707=1682%:%
%:%3708=1682%:%
%:%3709=1683%:%
%:%3710=1683%:%
%:%3711=1684%:%
%:%3712=1684%:%
%:%3713=1685%:%
%:%3714=1685%:%
%:%3715=1686%:%
%:%3716=1686%:%
%:%3717=1687%:%
%:%3718=1687%:%
%:%3719=1688%:%
%:%3720=1688%:%
%:%3721=1689%:%
%:%3722=1689%:%
%:%3723=1690%:%
%:%3724=1690%:%
%:%3725=1691%:%
%:%3726=1691%:%
%:%3727=1692%:%
%:%3728=1692%:%
%:%3729=1693%:%
%:%3730=1693%:%
%:%3731=1694%:%
%:%3732=1694%:%
%:%3733=1695%:%
%:%3734=1695%:%
%:%3735=1696%:%
%:%3736=1696%:%
%:%3737=1696%:%
%:%3738=1697%:%
%:%3739=1697%:%
%:%3740=1698%:%
%:%3741=1698%:%
%:%3742=1699%:%
%:%3743=1699%:%
%:%3744=1700%:%
%:%3745=1700%:%
%:%3746=1700%:%
%:%3747=1701%:%
%:%3748=1701%:%
%:%3749=1702%:%
%:%3755=1702%:%
%:%3760=1703%:%
%:%3765=1704%:%

%
\begin{isabellebody}%
\setisabellecontext{Partitions}%
%
\isadelimdocument
%
\endisadelimdocument
%
\isatagdocument
%
\isamarkupsection{Partitions%
}
\isamarkuptrue%
%
\endisatagdocument
{\isafolddocument}%
%
\isadelimdocument
%
\endisadelimdocument
%
\isadelimtheory
%
\endisadelimtheory
%
\isatagtheory
\isacommand{theory}\isamarkupfalse%
\ Partitions\isanewline
\ \ \isakeyword{imports}\ Main\ {\isachardoublequoteopen}HOL{\isacharminus}{\kern0pt}Library{\isachardot}{\kern0pt}Multiset{\isachardoublequoteclose}\ {\isachardoublequoteopen}HOL{\isachardot}{\kern0pt}Real{\isachardoublequoteclose}\ List{\isacharunderscore}{\kern0pt}Ext\isanewline
\isakeyword{begin}%
\endisatagtheory
{\isafoldtheory}%
%
\isadelimtheory
%
\endisadelimtheory
%
\begin{isamarkuptext}%
This section introduces a function that enumerates all the partitions of
\isa{{\isacharbraceleft}{\kern0pt}{\isadigit{0}}{\isachardot}{\kern0pt}{\isachardot}{\kern0pt}{\isacharless}{\kern0pt}n{\isacharbraceright}{\kern0pt}}. The partitions are represented as lists with \isa{n} elements. If the element
at index \isa{i} and \isa{j} have the same value, then \isa{i} and \isa{j} are in
the same partition.%
\end{isamarkuptext}\isamarkuptrue%
\isacommand{fun}\isamarkupfalse%
\ enum{\isacharunderscore}{\kern0pt}partitions{\isacharunderscore}{\kern0pt}aux\ {\isacharcolon}{\kern0pt}{\isacharcolon}{\kern0pt}\ {\isachardoublequoteopen}nat\ {\isasymRightarrow}\ {\isacharparenleft}{\kern0pt}nat\ {\isasymtimes}\ nat\ list{\isacharparenright}{\kern0pt}\ list{\isachardoublequoteclose}\isanewline
\ \ \isakeyword{where}\isanewline
\ \ \ \ {\isachardoublequoteopen}enum{\isacharunderscore}{\kern0pt}partitions{\isacharunderscore}{\kern0pt}aux\ {\isadigit{0}}\ {\isacharequal}{\kern0pt}\ {\isacharbrackleft}{\kern0pt}{\isacharparenleft}{\kern0pt}{\isadigit{0}}{\isacharcomma}{\kern0pt}\ {\isacharbrackleft}{\kern0pt}{\isacharbrackright}{\kern0pt}{\isacharparenright}{\kern0pt}{\isacharbrackright}{\kern0pt}{\isachardoublequoteclose}\ {\isacharbar}{\kern0pt}\isanewline
\ \ \ \ {\isachardoublequoteopen}enum{\isacharunderscore}{\kern0pt}partitions{\isacharunderscore}{\kern0pt}aux\ {\isacharparenleft}{\kern0pt}Suc\ n{\isacharparenright}{\kern0pt}\ {\isacharequal}{\kern0pt}\ \isanewline
\ \ \ \ \ \ {\isacharbrackleft}{\kern0pt}{\isacharparenleft}{\kern0pt}c{\isacharplus}{\kern0pt}{\isadigit{1}}{\isacharcomma}{\kern0pt}\ c{\isacharhash}{\kern0pt}x{\isacharparenright}{\kern0pt}{\isachardot}{\kern0pt}\ {\isacharparenleft}{\kern0pt}c{\isacharcomma}{\kern0pt}x{\isacharparenright}{\kern0pt}\ {\isasymleftarrow}\ enum{\isacharunderscore}{\kern0pt}partitions{\isacharunderscore}{\kern0pt}aux\ n{\isacharbrackright}{\kern0pt}{\isacharat}{\kern0pt}\isanewline
\ \ \ \ \ \ {\isacharbrackleft}{\kern0pt}{\isacharparenleft}{\kern0pt}c{\isacharcomma}{\kern0pt}\ y{\isacharhash}{\kern0pt}x{\isacharparenright}{\kern0pt}{\isachardot}{\kern0pt}\ {\isacharparenleft}{\kern0pt}c{\isacharcomma}{\kern0pt}x{\isacharparenright}{\kern0pt}\ {\isasymleftarrow}\ enum{\isacharunderscore}{\kern0pt}partitions{\isacharunderscore}{\kern0pt}aux\ n{\isacharcomma}{\kern0pt}\ \ y\ {\isasymleftarrow}\ {\isacharbrackleft}{\kern0pt}{\isadigit{0}}{\isachardot}{\kern0pt}{\isachardot}{\kern0pt}{\isacharless}{\kern0pt}c{\isacharbrackright}{\kern0pt}{\isacharbrackright}{\kern0pt}{\isachardoublequoteclose}\isanewline
\isanewline
\isacommand{fun}\isamarkupfalse%
\ enum{\isacharunderscore}{\kern0pt}partitions\ \isakeyword{where}\ {\isachardoublequoteopen}enum{\isacharunderscore}{\kern0pt}partitions\ n\ {\isacharequal}{\kern0pt}\ map\ snd\ {\isacharparenleft}{\kern0pt}enum{\isacharunderscore}{\kern0pt}partitions{\isacharunderscore}{\kern0pt}aux\ n{\isacharparenright}{\kern0pt}{\isachardoublequoteclose}\isanewline
\isanewline
\isacommand{definition}\isamarkupfalse%
\ has{\isacharunderscore}{\kern0pt}eq{\isacharunderscore}{\kern0pt}relation\ {\isacharcolon}{\kern0pt}{\isacharcolon}{\kern0pt}\ {\isachardoublequoteopen}nat\ list\ {\isasymRightarrow}\ {\isacharprime}{\kern0pt}a\ list\ {\isasymRightarrow}\ bool{\isachardoublequoteclose}\ \ \isakeyword{where}\isanewline
\ \ {\isachardoublequoteopen}has{\isacharunderscore}{\kern0pt}eq{\isacharunderscore}{\kern0pt}relation\ r\ xs\ {\isacharequal}{\kern0pt}\ {\isacharparenleft}{\kern0pt}length\ xs\ {\isacharequal}{\kern0pt}\ length\ r\ {\isasymand}\ {\isacharparenleft}{\kern0pt}{\isasymforall}i\ {\isacharless}{\kern0pt}\ length\ xs{\isachardot}{\kern0pt}\ {\isasymforall}j\ {\isacharless}{\kern0pt}\ length\ xs{\isachardot}{\kern0pt}\ {\isacharparenleft}{\kern0pt}xs\ {\isacharbang}{\kern0pt}\ i\ {\isacharequal}{\kern0pt}\ xs\ {\isacharbang}{\kern0pt}\ j{\isacharparenright}{\kern0pt}\ {\isacharequal}{\kern0pt}\ {\isacharparenleft}{\kern0pt}r\ {\isacharbang}{\kern0pt}\ i\ {\isacharequal}{\kern0pt}\ r\ {\isacharbang}{\kern0pt}\ j{\isacharparenright}{\kern0pt}{\isacharparenright}{\kern0pt}{\isacharparenright}{\kern0pt}{\isachardoublequoteclose}\isanewline
\isanewline
\isacommand{lemma}\isamarkupfalse%
\ filter{\isacharunderscore}{\kern0pt}one{\isacharunderscore}{\kern0pt}elim{\isacharcolon}{\kern0pt}\isanewline
\ \ {\isachardoublequoteopen}length\ {\isacharparenleft}{\kern0pt}filter\ p\ xs{\isacharparenright}{\kern0pt}\ {\isacharequal}{\kern0pt}\ {\isadigit{1}}\ {\isasymLongrightarrow}\ {\isacharparenleft}{\kern0pt}{\isasymexists}u\ v\ w{\isachardot}{\kern0pt}\ xs\ {\isacharequal}{\kern0pt}\ u{\isacharat}{\kern0pt}v{\isacharhash}{\kern0pt}w\ {\isasymand}\ p\ v\ {\isasymand}\ length\ {\isacharparenleft}{\kern0pt}filter\ p\ u{\isacharparenright}{\kern0pt}\ {\isacharequal}{\kern0pt}\ {\isadigit{0}}\ {\isasymand}\ length\ {\isacharparenleft}{\kern0pt}filter\ p\ w{\isacharparenright}{\kern0pt}\ {\isacharequal}{\kern0pt}\ {\isadigit{0}}{\isacharparenright}{\kern0pt}{\isachardoublequoteclose}\isanewline
\ \ {\isacharparenleft}{\kern0pt}\isakeyword{is}\ {\isachardoublequoteopen}{\isacharquery}{\kern0pt}A\ xs\ {\isasymLongrightarrow}\ {\isacharquery}{\kern0pt}B\ xs{\isachardoublequoteclose}{\isacharparenright}{\kern0pt}\isanewline
%
\isadelimproof
%
\endisadelimproof
%
\isatagproof
\isacommand{proof}\isamarkupfalse%
\ {\isacharparenleft}{\kern0pt}induction\ xs{\isacharparenright}{\kern0pt}\isanewline
\ \ \isacommand{case}\isamarkupfalse%
\ Nil\isanewline
\ \ \isacommand{then}\isamarkupfalse%
\ \isacommand{show}\isamarkupfalse%
\ {\isacharquery}{\kern0pt}case\ \isacommand{by}\isamarkupfalse%
\ simp\isanewline
\isacommand{next}\isamarkupfalse%
\isanewline
\ \ \isacommand{case}\isamarkupfalse%
\ {\isacharparenleft}{\kern0pt}Cons\ a\ xs{\isacharparenright}{\kern0pt}\isanewline
\ \ \isacommand{then}\isamarkupfalse%
\ \isacommand{show}\isamarkupfalse%
\ {\isacharquery}{\kern0pt}case\isanewline
\ \ \ \ \isacommand{apply}\isamarkupfalse%
\ {\isacharparenleft}{\kern0pt}cases\ {\isachardoublequoteopen}p\ a{\isachardoublequoteclose}{\isacharparenright}{\kern0pt}\ \isanewline
\ \ \ \ \ \isacommand{apply}\isamarkupfalse%
\ {\isacharparenleft}{\kern0pt}simp{\isacharcomma}{\kern0pt}\ metis\ append{\isachardot}{\kern0pt}left{\isacharunderscore}{\kern0pt}neutral\ filter{\isachardot}{\kern0pt}simps{\isacharparenleft}{\kern0pt}{\isadigit{1}}{\isacharparenright}{\kern0pt}{\isacharparenright}{\kern0pt}\isanewline
\ \ \ \ \isacommand{by}\isamarkupfalse%
\ {\isacharparenleft}{\kern0pt}simp{\isacharcomma}{\kern0pt}\ metis\ append{\isacharunderscore}{\kern0pt}Cons\ filter{\isachardot}{\kern0pt}simps{\isacharparenleft}{\kern0pt}{\isadigit{2}}{\isacharparenright}{\kern0pt}{\isacharparenright}{\kern0pt}\isanewline
\isacommand{qed}\isamarkupfalse%
%
\endisatagproof
{\isafoldproof}%
%
\isadelimproof
\isanewline
%
\endisadelimproof
\isanewline
\isacommand{lemma}\isamarkupfalse%
\ has{\isacharunderscore}{\kern0pt}eq{\isacharunderscore}{\kern0pt}elim{\isacharcolon}{\kern0pt}\isanewline
\ \ {\isachardoublequoteopen}has{\isacharunderscore}{\kern0pt}eq{\isacharunderscore}{\kern0pt}relation\ {\isacharparenleft}{\kern0pt}r{\isacharhash}{\kern0pt}rs{\isacharparenright}{\kern0pt}\ {\isacharparenleft}{\kern0pt}x{\isacharhash}{\kern0pt}xs{\isacharparenright}{\kern0pt}\ {\isacharequal}{\kern0pt}\ {\isacharparenleft}{\kern0pt}\isanewline
\ \ \ \ {\isacharparenleft}{\kern0pt}{\isasymforall}i\ {\isacharless}{\kern0pt}\ length\ xs{\isachardot}{\kern0pt}\ {\isacharparenleft}{\kern0pt}r\ {\isacharequal}{\kern0pt}\ rs\ {\isacharbang}{\kern0pt}\ i{\isacharparenright}{\kern0pt}\ {\isacharequal}{\kern0pt}\ {\isacharparenleft}{\kern0pt}x\ {\isacharequal}{\kern0pt}\ xs\ {\isacharbang}{\kern0pt}\ i{\isacharparenright}{\kern0pt}{\isacharparenright}{\kern0pt}\ {\isasymand}\isanewline
\ \ \ \ has{\isacharunderscore}{\kern0pt}eq{\isacharunderscore}{\kern0pt}relation\ rs\ xs{\isacharparenright}{\kern0pt}{\isachardoublequoteclose}\isanewline
%
\isadelimproof
%
\endisadelimproof
%
\isatagproof
\isacommand{proof}\isamarkupfalse%
\isanewline
\ \ \isacommand{assume}\isamarkupfalse%
\ a{\isacharcolon}{\kern0pt}{\isachardoublequoteopen}has{\isacharunderscore}{\kern0pt}eq{\isacharunderscore}{\kern0pt}relation\ {\isacharparenleft}{\kern0pt}r{\isacharhash}{\kern0pt}rs{\isacharparenright}{\kern0pt}\ {\isacharparenleft}{\kern0pt}x{\isacharhash}{\kern0pt}xs{\isacharparenright}{\kern0pt}{\isachardoublequoteclose}\isanewline
\ \ \isacommand{have}\isamarkupfalse%
\ {\isachardoublequoteopen}{\isasymAnd}i\ j{\isachardot}{\kern0pt}\ i\ {\isacharless}{\kern0pt}\ length\ xs\ {\isasymLongrightarrow}\ j\ {\isacharless}{\kern0pt}\ length\ xs\ {\isasymLongrightarrow}\ {\isacharparenleft}{\kern0pt}xs\ {\isacharbang}{\kern0pt}\ i\ {\isacharequal}{\kern0pt}\ xs\ {\isacharbang}{\kern0pt}\ j{\isacharparenright}{\kern0pt}\ {\isacharequal}{\kern0pt}\ {\isacharparenleft}{\kern0pt}rs\ {\isacharbang}{\kern0pt}\ i\ {\isacharequal}{\kern0pt}\ rs\ {\isacharbang}{\kern0pt}\ j{\isacharparenright}{\kern0pt}{\isachardoublequoteclose}\isanewline
\ \ \ \ {\isacharparenleft}{\kern0pt}\isakeyword{is}\ {\isachardoublequoteopen}{\isasymAnd}i\ j{\isachardot}{\kern0pt}\ {\isacharquery}{\kern0pt}l{\isadigit{1}}\ i\ {\isasymLongrightarrow}\ {\isacharquery}{\kern0pt}l{\isadigit{2}}\ j\ {\isasymLongrightarrow}\ {\isacharquery}{\kern0pt}rhs\ i\ j{\isachardoublequoteclose}{\isacharparenright}{\kern0pt}\isanewline
\ \ \isacommand{proof}\isamarkupfalse%
\ {\isacharminus}{\kern0pt}\isanewline
\ \ \ \ \isacommand{fix}\isamarkupfalse%
\ i\ j\isanewline
\ \ \ \ \isacommand{assume}\isamarkupfalse%
\ {\isachardoublequoteopen}i\ {\isacharless}{\kern0pt}\ length\ xs{\isachardoublequoteclose}\isanewline
\ \ \ \ \isacommand{hence}\isamarkupfalse%
\ {\isachardoublequoteopen}Suc\ i\ {\isacharless}{\kern0pt}\ length\ {\isacharparenleft}{\kern0pt}x{\isacharhash}{\kern0pt}xs{\isacharparenright}{\kern0pt}{\isachardoublequoteclose}\ \isacommand{by}\isamarkupfalse%
\ auto\isanewline
\ \ \ \ \isacommand{moreover}\isamarkupfalse%
\ \isacommand{assume}\isamarkupfalse%
\ {\isachardoublequoteopen}j\ {\isacharless}{\kern0pt}\ length\ xs{\isachardoublequoteclose}\isanewline
\ \ \ \ \isacommand{hence}\isamarkupfalse%
\ {\isachardoublequoteopen}Suc\ j\ {\isacharless}{\kern0pt}\ length\ {\isacharparenleft}{\kern0pt}x{\isacharhash}{\kern0pt}xs{\isacharparenright}{\kern0pt}{\isachardoublequoteclose}\ \isacommand{by}\isamarkupfalse%
\ auto\isanewline
\ \ \ \ \isacommand{ultimately}\isamarkupfalse%
\ \isacommand{show}\isamarkupfalse%
\ {\isachardoublequoteopen}{\isacharquery}{\kern0pt}rhs\ i\ j{\isachardoublequoteclose}\ \isacommand{using}\isamarkupfalse%
\ a\ \isacommand{apply}\isamarkupfalse%
\ {\isacharparenleft}{\kern0pt}simp\ only{\isacharcolon}{\kern0pt}has{\isacharunderscore}{\kern0pt}eq{\isacharunderscore}{\kern0pt}relation{\isacharunderscore}{\kern0pt}def{\isacharparenright}{\kern0pt}\ \isanewline
\ \ \ \ \ \ \isacommand{by}\isamarkupfalse%
\ {\isacharparenleft}{\kern0pt}metis\ nth{\isacharunderscore}{\kern0pt}Cons{\isacharunderscore}{\kern0pt}Suc{\isacharparenright}{\kern0pt}\isanewline
\ \ \isacommand{qed}\isamarkupfalse%
\isanewline
\ \ \isacommand{hence}\isamarkupfalse%
\ {\isachardoublequoteopen}has{\isacharunderscore}{\kern0pt}eq{\isacharunderscore}{\kern0pt}relation\ rs\ xs{\isachardoublequoteclose}\ \isacommand{using}\isamarkupfalse%
\ a\ \isacommand{by}\isamarkupfalse%
\ {\isacharparenleft}{\kern0pt}simp\ add{\isacharcolon}{\kern0pt}has{\isacharunderscore}{\kern0pt}eq{\isacharunderscore}{\kern0pt}relation{\isacharunderscore}{\kern0pt}def{\isacharparenright}{\kern0pt}\isanewline
\ \ \isacommand{thus}\isamarkupfalse%
\ {\isachardoublequoteopen}{\isacharparenleft}{\kern0pt}{\isasymforall}i{\isacharless}{\kern0pt}length\ xs{\isachardot}{\kern0pt}\ {\isacharparenleft}{\kern0pt}r\ {\isacharequal}{\kern0pt}\ rs\ {\isacharbang}{\kern0pt}\ i{\isacharparenright}{\kern0pt}\ {\isacharequal}{\kern0pt}\ {\isacharparenleft}{\kern0pt}x\ {\isacharequal}{\kern0pt}\ xs\ {\isacharbang}{\kern0pt}\ i{\isacharparenright}{\kern0pt}{\isacharparenright}{\kern0pt}\ {\isasymand}\ has{\isacharunderscore}{\kern0pt}eq{\isacharunderscore}{\kern0pt}relation\ rs\ xs{\isachardoublequoteclose}\isanewline
\ \ \ \ \isacommand{apply}\isamarkupfalse%
\ simp\isanewline
\ \ \ \ \isacommand{using}\isamarkupfalse%
\ a\ \isacommand{apply}\isamarkupfalse%
\ {\isacharparenleft}{\kern0pt}simp\ only{\isacharcolon}{\kern0pt}has{\isacharunderscore}{\kern0pt}eq{\isacharunderscore}{\kern0pt}relation{\isacharunderscore}{\kern0pt}def{\isacharparenright}{\kern0pt}\isanewline
\ \ \ \ \isacommand{by}\isamarkupfalse%
\ {\isacharparenleft}{\kern0pt}metis\ Suc{\isacharunderscore}{\kern0pt}less{\isacharunderscore}{\kern0pt}eq\ length{\isacharunderscore}{\kern0pt}Cons\ nth{\isacharunderscore}{\kern0pt}Cons{\isacharunderscore}{\kern0pt}{\isadigit{0}}\ nth{\isacharunderscore}{\kern0pt}Cons{\isacharunderscore}{\kern0pt}Suc\ zero{\isacharunderscore}{\kern0pt}less{\isacharunderscore}{\kern0pt}Suc{\isacharparenright}{\kern0pt}\isanewline
\isacommand{next}\isamarkupfalse%
\isanewline
\ \ \isacommand{assume}\isamarkupfalse%
\ a{\isacharcolon}{\kern0pt}{\isachardoublequoteopen}{\isacharparenleft}{\kern0pt}{\isasymforall}i{\isacharless}{\kern0pt}length\ xs{\isachardot}{\kern0pt}\ {\isacharparenleft}{\kern0pt}r\ {\isacharequal}{\kern0pt}\ rs\ {\isacharbang}{\kern0pt}\ i{\isacharparenright}{\kern0pt}\ {\isacharequal}{\kern0pt}\ {\isacharparenleft}{\kern0pt}x\ {\isacharequal}{\kern0pt}\ xs\ {\isacharbang}{\kern0pt}\ i{\isacharparenright}{\kern0pt}{\isacharparenright}{\kern0pt}\ {\isasymand}\ has{\isacharunderscore}{\kern0pt}eq{\isacharunderscore}{\kern0pt}relation\ rs\ xs{\isachardoublequoteclose}\isanewline
\ \ \isacommand{have}\isamarkupfalse%
\ {\isachardoublequoteopen}{\isasymAnd}i\ j{\isachardot}{\kern0pt}\ i{\isacharless}{\kern0pt}Suc\ {\isacharparenleft}{\kern0pt}length\ rs{\isacharparenright}{\kern0pt}\ {\isasymLongrightarrow}\ j{\isacharless}{\kern0pt}Suc\ {\isacharparenleft}{\kern0pt}length\ rs{\isacharparenright}{\kern0pt}\ {\isasymLongrightarrow}\ {\isacharparenleft}{\kern0pt}{\isacharparenleft}{\kern0pt}x\ {\isacharhash}{\kern0pt}\ xs{\isacharparenright}{\kern0pt}\ {\isacharbang}{\kern0pt}\ i\ {\isacharequal}{\kern0pt}\ {\isacharparenleft}{\kern0pt}x\ {\isacharhash}{\kern0pt}\ xs{\isacharparenright}{\kern0pt}\ {\isacharbang}{\kern0pt}\ j{\isacharparenright}{\kern0pt}\ {\isacharequal}{\kern0pt}\ {\isacharparenleft}{\kern0pt}{\isacharparenleft}{\kern0pt}r\ {\isacharhash}{\kern0pt}\ rs{\isacharparenright}{\kern0pt}\ {\isacharbang}{\kern0pt}\ i\ {\isacharequal}{\kern0pt}\ {\isacharparenleft}{\kern0pt}r\ {\isacharhash}{\kern0pt}\ rs{\isacharparenright}{\kern0pt}\ {\isacharbang}{\kern0pt}\ j{\isacharparenright}{\kern0pt}{\isachardoublequoteclose}\isanewline
\ \ \ \ {\isacharparenleft}{\kern0pt}\isakeyword{is}\ {\isachardoublequoteopen}{\isasymAnd}i\ j{\isachardot}{\kern0pt}\ {\isacharquery}{\kern0pt}l{\isadigit{1}}\ i\ {\isasymLongrightarrow}\ {\isacharquery}{\kern0pt}l{\isadigit{2}}\ j\ {\isasymLongrightarrow}\ {\isacharquery}{\kern0pt}rhs\ i\ j{\isachardoublequoteclose}{\isacharparenright}{\kern0pt}\isanewline
\ \ \isacommand{proof}\isamarkupfalse%
\ {\isacharminus}{\kern0pt}\isanewline
\ \ \ \ \isacommand{fix}\isamarkupfalse%
\ i\ j\isanewline
\ \ \ \ \isacommand{assume}\isamarkupfalse%
\ {\isachardoublequoteopen}i\ {\isacharless}{\kern0pt}\ Suc\ {\isacharparenleft}{\kern0pt}length\ rs{\isacharparenright}{\kern0pt}{\isachardoublequoteclose}\isanewline
\ \ \ \ \isacommand{moreover}\isamarkupfalse%
\ \isacommand{assume}\isamarkupfalse%
\ {\isachardoublequoteopen}j\ {\isacharless}{\kern0pt}\ Suc\ {\isacharparenleft}{\kern0pt}length\ rs{\isacharparenright}{\kern0pt}{\isachardoublequoteclose}\ \isanewline
\ \ \ \ \isacommand{ultimately}\isamarkupfalse%
\ \isacommand{show}\isamarkupfalse%
\ {\isachardoublequoteopen}{\isacharquery}{\kern0pt}rhs\ i\ j{\isachardoublequoteclose}\ \isacommand{using}\isamarkupfalse%
\ a\isanewline
\ \ \ \ \ \ \isacommand{apply}\isamarkupfalse%
\ {\isacharparenleft}{\kern0pt}cases\ i{\isacharcomma}{\kern0pt}\ cases\ j{\isacharparenright}{\kern0pt}\isanewline
\ \ \ \ \ \ \isacommand{apply}\isamarkupfalse%
\ {\isacharparenleft}{\kern0pt}simp\ add{\isacharcolon}{\kern0pt}\ has{\isacharunderscore}{\kern0pt}eq{\isacharunderscore}{\kern0pt}relation{\isacharunderscore}{\kern0pt}def{\isacharparenright}{\kern0pt}\isanewline
\ \ \ \ \ \ \isacommand{apply}\isamarkupfalse%
\ {\isacharparenleft}{\kern0pt}cases\ j{\isacharparenright}{\kern0pt}\isanewline
\ \ \ \ \ \ \isacommand{apply}\isamarkupfalse%
\ {\isacharparenleft}{\kern0pt}simp\ add{\isacharcolon}{\kern0pt}\ has{\isacharunderscore}{\kern0pt}eq{\isacharunderscore}{\kern0pt}relation{\isacharunderscore}{\kern0pt}def{\isacharparenright}{\kern0pt}{\isacharplus}{\kern0pt}\isanewline
\ \ \ \ \ \ \isacommand{by}\isamarkupfalse%
\ {\isacharparenleft}{\kern0pt}metis\ less{\isacharunderscore}{\kern0pt}Suc{\isacharunderscore}{\kern0pt}eq{\isacharunderscore}{\kern0pt}{\isadigit{0}}{\isacharunderscore}{\kern0pt}disj\ nth{\isacharunderscore}{\kern0pt}Cons{\isacharprime}{\kern0pt}\ nth{\isacharunderscore}{\kern0pt}Cons{\isacharunderscore}{\kern0pt}Suc{\isacharparenright}{\kern0pt}\isanewline
\ \ \isacommand{qed}\isamarkupfalse%
\isanewline
\ \ \isacommand{then}\isamarkupfalse%
\ \isacommand{show}\isamarkupfalse%
\ {\isachardoublequoteopen}has{\isacharunderscore}{\kern0pt}eq{\isacharunderscore}{\kern0pt}relation\ {\isacharparenleft}{\kern0pt}r\ {\isacharhash}{\kern0pt}\ rs{\isacharparenright}{\kern0pt}\ {\isacharparenleft}{\kern0pt}x\ {\isacharhash}{\kern0pt}\ xs{\isacharparenright}{\kern0pt}{\isachardoublequoteclose}\isanewline
\ \ \ \ \isacommand{using}\isamarkupfalse%
\ a\ \isacommand{by}\isamarkupfalse%
\ {\isacharparenleft}{\kern0pt}simp\ add{\isacharcolon}{\kern0pt}has{\isacharunderscore}{\kern0pt}eq{\isacharunderscore}{\kern0pt}relation{\isacharunderscore}{\kern0pt}def{\isacharparenright}{\kern0pt}\isanewline
\isacommand{qed}\isamarkupfalse%
%
\endisatagproof
{\isafoldproof}%
%
\isadelimproof
\isanewline
%
\endisadelimproof
\isanewline
\isacommand{lemma}\isamarkupfalse%
\ enum{\isacharunderscore}{\kern0pt}partitions{\isacharunderscore}{\kern0pt}aux{\isacharunderscore}{\kern0pt}range{\isacharcolon}{\kern0pt}\isanewline
\ \ {\isachardoublequoteopen}x\ {\isasymin}\ set\ {\isacharparenleft}{\kern0pt}enum{\isacharunderscore}{\kern0pt}partitions{\isacharunderscore}{\kern0pt}aux\ n{\isacharparenright}{\kern0pt}\ {\isasymLongrightarrow}\ set\ {\isacharparenleft}{\kern0pt}snd\ x{\isacharparenright}{\kern0pt}\ {\isacharequal}{\kern0pt}\ {\isacharbraceleft}{\kern0pt}k{\isachardot}{\kern0pt}\ k\ {\isacharless}{\kern0pt}\ fst\ x{\isacharbraceright}{\kern0pt}{\isachardoublequoteclose}\isanewline
%
\isadelimproof
\ \ %
\endisadelimproof
%
\isatagproof
\isacommand{by}\isamarkupfalse%
\ {\isacharparenleft}{\kern0pt}induction\ n\ arbitrary{\isacharcolon}{\kern0pt}x{\isacharcomma}{\kern0pt}\ simp{\isacharcomma}{\kern0pt}\ simp{\isacharcomma}{\kern0pt}\ force{\isacharparenright}{\kern0pt}%
\endisatagproof
{\isafoldproof}%
%
\isadelimproof
\isanewline
%
\endisadelimproof
\isanewline
\isacommand{lemma}\isamarkupfalse%
\ enum{\isacharunderscore}{\kern0pt}partitions{\isacharunderscore}{\kern0pt}aux{\isacharunderscore}{\kern0pt}len{\isacharcolon}{\kern0pt}\isanewline
\ \ {\isachardoublequoteopen}x\ {\isasymin}\ set\ {\isacharparenleft}{\kern0pt}enum{\isacharunderscore}{\kern0pt}partitions{\isacharunderscore}{\kern0pt}aux\ n{\isacharparenright}{\kern0pt}\ {\isasymLongrightarrow}\ length\ {\isacharparenleft}{\kern0pt}snd\ x{\isacharparenright}{\kern0pt}\ {\isacharequal}{\kern0pt}\ n{\isachardoublequoteclose}\isanewline
%
\isadelimproof
\ \ %
\endisadelimproof
%
\isatagproof
\isacommand{by}\isamarkupfalse%
\ {\isacharparenleft}{\kern0pt}induction\ n\ arbitrary{\isacharcolon}{\kern0pt}x{\isacharcomma}{\kern0pt}\ simp{\isacharcomma}{\kern0pt}\ simp{\isacharcomma}{\kern0pt}\ force{\isacharparenright}{\kern0pt}%
\endisatagproof
{\isafoldproof}%
%
\isadelimproof
\isanewline
%
\endisadelimproof
\isanewline
\isacommand{lemma}\isamarkupfalse%
\ enum{\isacharunderscore}{\kern0pt}partitions{\isacharunderscore}{\kern0pt}complete{\isacharunderscore}{\kern0pt}aux{\isacharcolon}{\kern0pt}\ {\isachardoublequoteopen}k\ {\isacharless}{\kern0pt}\ n\ {\isasymLongrightarrow}\ length\ {\isacharparenleft}{\kern0pt}filter\ {\isacharparenleft}{\kern0pt}{\isasymlambda}x{\isachardot}{\kern0pt}\ x\ {\isacharequal}{\kern0pt}\ k{\isacharparenright}{\kern0pt}\ {\isacharbrackleft}{\kern0pt}{\isadigit{0}}{\isachardot}{\kern0pt}{\isachardot}{\kern0pt}{\isacharless}{\kern0pt}n{\isacharbrackright}{\kern0pt}{\isacharparenright}{\kern0pt}\ {\isacharequal}{\kern0pt}\ Suc\ {\isadigit{0}}{\isachardoublequoteclose}\isanewline
%
\isadelimproof
\ \ %
\endisadelimproof
%
\isatagproof
\isacommand{by}\isamarkupfalse%
\ {\isacharparenleft}{\kern0pt}induction\ n{\isacharcomma}{\kern0pt}\ simp{\isacharcomma}{\kern0pt}\ simp{\isacharparenright}{\kern0pt}%
\endisatagproof
{\isafoldproof}%
%
\isadelimproof
\isanewline
%
\endisadelimproof
\isanewline
\isacommand{lemma}\isamarkupfalse%
\ enum{\isacharunderscore}{\kern0pt}partitions{\isacharunderscore}{\kern0pt}complete{\isacharcolon}{\kern0pt}\isanewline
\ \ {\isachardoublequoteopen}length\ {\isacharparenleft}{\kern0pt}filter\ {\isacharparenleft}{\kern0pt}{\isasymlambda}p{\isachardot}{\kern0pt}\ has{\isacharunderscore}{\kern0pt}eq{\isacharunderscore}{\kern0pt}relation\ p\ x{\isacharparenright}{\kern0pt}\ {\isacharparenleft}{\kern0pt}enum{\isacharunderscore}{\kern0pt}partitions\ {\isacharparenleft}{\kern0pt}length\ x{\isacharparenright}{\kern0pt}{\isacharparenright}{\kern0pt}{\isacharparenright}{\kern0pt}\ {\isacharequal}{\kern0pt}\ {\isadigit{1}}{\isachardoublequoteclose}\isanewline
%
\isadelimproof
%
\endisadelimproof
%
\isatagproof
\isacommand{proof}\isamarkupfalse%
\ {\isacharparenleft}{\kern0pt}induction\ {\isachardoublequoteopen}x{\isachardoublequoteclose}{\isacharparenright}{\kern0pt}\isanewline
\ \ \isacommand{case}\isamarkupfalse%
\ Nil\isanewline
\ \ \isacommand{then}\isamarkupfalse%
\ \isacommand{show}\isamarkupfalse%
\ {\isacharquery}{\kern0pt}case\ \isacommand{by}\isamarkupfalse%
\ {\isacharparenleft}{\kern0pt}simp\ add{\isacharcolon}{\kern0pt}has{\isacharunderscore}{\kern0pt}eq{\isacharunderscore}{\kern0pt}relation{\isacharunderscore}{\kern0pt}def{\isacharparenright}{\kern0pt}\isanewline
\isacommand{next}\isamarkupfalse%
\isanewline
\ \ \isacommand{case}\isamarkupfalse%
\ {\isacharparenleft}{\kern0pt}Cons\ a\ y{\isacharparenright}{\kern0pt}\isanewline
\ \ \isacommand{have}\isamarkupfalse%
\ {\isachardoublequoteopen}length\ {\isacharparenleft}{\kern0pt}filter\ {\isacharparenleft}{\kern0pt}{\isasymlambda}x{\isachardot}{\kern0pt}\ has{\isacharunderscore}{\kern0pt}eq{\isacharunderscore}{\kern0pt}relation\ {\isacharparenleft}{\kern0pt}snd\ x{\isacharparenright}{\kern0pt}\ y{\isacharparenright}{\kern0pt}\ {\isacharparenleft}{\kern0pt}enum{\isacharunderscore}{\kern0pt}partitions{\isacharunderscore}{\kern0pt}aux\ {\isacharparenleft}{\kern0pt}length\ y{\isacharparenright}{\kern0pt}{\isacharparenright}{\kern0pt}{\isacharparenright}{\kern0pt}\ {\isacharequal}{\kern0pt}\ {\isadigit{1}}{\isachardoublequoteclose}\isanewline
\ \ \ \ \isacommand{using}\isamarkupfalse%
\ Cons\ \isacommand{by}\isamarkupfalse%
\ {\isacharparenleft}{\kern0pt}simp\ add{\isacharcolon}{\kern0pt}comp{\isacharunderscore}{\kern0pt}def{\isacharparenright}{\kern0pt}\ \isanewline
\ \ \isacommand{then}\isamarkupfalse%
\ \isacommand{obtain}\isamarkupfalse%
\ p{\isadigit{1}}\ p{\isadigit{2}}\ p{\isadigit{3}}\ \isakeyword{where}\ pi{\isacharunderscore}{\kern0pt}def{\isacharcolon}{\kern0pt}\ {\isachardoublequoteopen}enum{\isacharunderscore}{\kern0pt}partitions{\isacharunderscore}{\kern0pt}aux\ {\isacharparenleft}{\kern0pt}length\ y{\isacharparenright}{\kern0pt}\ {\isacharequal}{\kern0pt}\ p{\isadigit{1}}{\isacharat}{\kern0pt}p{\isadigit{2}}{\isacharhash}{\kern0pt}p{\isadigit{3}}{\isachardoublequoteclose}\ \isakeyword{and}\isanewline
\ \ \ p{\isadigit{2}}{\isacharunderscore}{\kern0pt}t{\isacharcolon}{\kern0pt}\ {\isachardoublequoteopen}has{\isacharunderscore}{\kern0pt}eq{\isacharunderscore}{\kern0pt}relation\ {\isacharparenleft}{\kern0pt}snd\ p{\isadigit{2}}{\isacharparenright}{\kern0pt}\ y{\isachardoublequoteclose}\ \isakeyword{and}\isanewline
\ \ \ p{\isadigit{1}}{\isacharunderscore}{\kern0pt}f{\isadigit{1}}{\isacharcolon}{\kern0pt}\ {\isachardoublequoteopen}filter\ {\isacharparenleft}{\kern0pt}{\isasymlambda}x{\isachardot}{\kern0pt}\ has{\isacharunderscore}{\kern0pt}eq{\isacharunderscore}{\kern0pt}relation\ {\isacharparenleft}{\kern0pt}snd\ x{\isacharparenright}{\kern0pt}\ y{\isacharparenright}{\kern0pt}\ p{\isadigit{1}}\ {\isacharequal}{\kern0pt}\ {\isacharbrackleft}{\kern0pt}{\isacharbrackright}{\kern0pt}{\isachardoublequoteclose}\ \isakeyword{and}\isanewline
\ \ \ p{\isadigit{3}}{\isacharunderscore}{\kern0pt}f{\isadigit{1}}{\isacharcolon}{\kern0pt}\ {\isachardoublequoteopen}filter\ {\isacharparenleft}{\kern0pt}{\isasymlambda}x{\isachardot}{\kern0pt}\ has{\isacharunderscore}{\kern0pt}eq{\isacharunderscore}{\kern0pt}relation\ {\isacharparenleft}{\kern0pt}snd\ x{\isacharparenright}{\kern0pt}\ y{\isacharparenright}{\kern0pt}\ p{\isadigit{3}}\ {\isacharequal}{\kern0pt}\ {\isacharbrackleft}{\kern0pt}{\isacharbrackright}{\kern0pt}{\isachardoublequoteclose}\isanewline
\ \ \ \ \isacommand{using}\isamarkupfalse%
\ Cons\ filter{\isacharunderscore}{\kern0pt}one{\isacharunderscore}{\kern0pt}elim\ \isacommand{by}\isamarkupfalse%
\ {\isacharparenleft}{\kern0pt}metis\ {\isacharparenleft}{\kern0pt}no{\isacharunderscore}{\kern0pt}types{\isacharcomma}{\kern0pt}\ lifting{\isacharparenright}{\kern0pt}\ length{\isacharunderscore}{\kern0pt}{\isadigit{0}}{\isacharunderscore}{\kern0pt}conv{\isacharparenright}{\kern0pt}\isanewline
\ \ \isacommand{have}\isamarkupfalse%
\ p{\isadigit{2}}{\isacharunderscore}{\kern0pt}e{\isacharcolon}{\kern0pt}\ {\isachardoublequoteopen}p{\isadigit{2}}\ {\isasymin}\ set{\isacharparenleft}{\kern0pt}enum{\isacharunderscore}{\kern0pt}partitions{\isacharunderscore}{\kern0pt}aux\ {\isacharparenleft}{\kern0pt}length\ y{\isacharparenright}{\kern0pt}{\isacharparenright}{\kern0pt}{\isachardoublequoteclose}\ \isanewline
\ \ \ \ \isacommand{using}\isamarkupfalse%
\ pi{\isacharunderscore}{\kern0pt}def\ \isacommand{by}\isamarkupfalse%
\ auto\isanewline
\ \ \isacommand{have}\isamarkupfalse%
\ p{\isadigit{1}}{\isacharunderscore}{\kern0pt}f{\isacharcolon}{\kern0pt}\ {\isachardoublequoteopen}{\isasymAnd}x\ p{\isachardot}{\kern0pt}\ x\ {\isasymin}\ set\ p{\isadigit{1}}\ {\isasymLongrightarrow}\ has{\isacharunderscore}{\kern0pt}eq{\isacharunderscore}{\kern0pt}relation\ {\isacharparenleft}{\kern0pt}p{\isacharhash}{\kern0pt}{\isacharparenleft}{\kern0pt}snd\ x{\isacharparenright}{\kern0pt}{\isacharparenright}{\kern0pt}\ {\isacharparenleft}{\kern0pt}a{\isacharhash}{\kern0pt}y{\isacharparenright}{\kern0pt}\ {\isacharequal}{\kern0pt}\ False{\isachardoublequoteclose}\isanewline
\ \ \ \ \isacommand{by}\isamarkupfalse%
\ {\isacharparenleft}{\kern0pt}metis\ p{\isadigit{1}}{\isacharunderscore}{\kern0pt}f{\isadigit{1}}\ filter{\isacharunderscore}{\kern0pt}empty{\isacharunderscore}{\kern0pt}conv\ has{\isacharunderscore}{\kern0pt}eq{\isacharunderscore}{\kern0pt}elim{\isacharparenright}{\kern0pt}\isanewline
\ \ \isacommand{have}\isamarkupfalse%
\ p{\isadigit{3}}{\isacharunderscore}{\kern0pt}f{\isacharcolon}{\kern0pt}\ {\isachardoublequoteopen}{\isasymAnd}x\ p{\isachardot}{\kern0pt}\ x\ {\isasymin}\ set\ p{\isadigit{3}}\ {\isasymLongrightarrow}\ has{\isacharunderscore}{\kern0pt}eq{\isacharunderscore}{\kern0pt}relation\ {\isacharparenleft}{\kern0pt}p{\isacharhash}{\kern0pt}{\isacharparenleft}{\kern0pt}snd\ x{\isacharparenright}{\kern0pt}{\isacharparenright}{\kern0pt}\ {\isacharparenleft}{\kern0pt}a{\isacharhash}{\kern0pt}y{\isacharparenright}{\kern0pt}\ {\isacharequal}{\kern0pt}\ False{\isachardoublequoteclose}\isanewline
\ \ \ \ \isacommand{by}\isamarkupfalse%
\ {\isacharparenleft}{\kern0pt}metis\ p{\isadigit{3}}{\isacharunderscore}{\kern0pt}f{\isadigit{1}}\ filter{\isacharunderscore}{\kern0pt}empty{\isacharunderscore}{\kern0pt}conv\ has{\isacharunderscore}{\kern0pt}eq{\isacharunderscore}{\kern0pt}elim{\isacharparenright}{\kern0pt}\isanewline
\ \ \isacommand{show}\isamarkupfalse%
\ {\isacharquery}{\kern0pt}case\isanewline
\ \ \isacommand{proof}\isamarkupfalse%
\ {\isacharparenleft}{\kern0pt}cases\ {\isachardoublequoteopen}a\ {\isasymin}\ set\ y{\isachardoublequoteclose}{\isacharparenright}{\kern0pt}\isanewline
\ \ \ \ \isacommand{case}\isamarkupfalse%
\ True\isanewline
\ \ \ \ \isacommand{then}\isamarkupfalse%
\ \isacommand{obtain}\isamarkupfalse%
\ h\ \isakeyword{where}\ h{\isacharunderscore}{\kern0pt}def{\isacharcolon}{\kern0pt}\ {\isachardoublequoteopen}h\ {\isacharless}{\kern0pt}\ length\ y\ {\isasymand}\ a\ {\isacharequal}{\kern0pt}\ y\ {\isacharbang}{\kern0pt}\ h{\isachardoublequoteclose}\ \isacommand{by}\isamarkupfalse%
\ {\isacharparenleft}{\kern0pt}metis\ in{\isacharunderscore}{\kern0pt}set{\isacharunderscore}{\kern0pt}conv{\isacharunderscore}{\kern0pt}nth{\isacharparenright}{\kern0pt}\isanewline
\ \ \ \ \isacommand{define}\isamarkupfalse%
\ k\ \isakeyword{where}\ {\isachardoublequoteopen}k\ {\isacharequal}{\kern0pt}\ snd\ p{\isadigit{2}}\ {\isacharbang}{\kern0pt}\ h{\isachardoublequoteclose}\isanewline
\ \ \ \ \isacommand{have}\isamarkupfalse%
\ k{\isacharunderscore}{\kern0pt}bound{\isacharcolon}{\kern0pt}\ {\isachardoublequoteopen}k\ {\isacharless}{\kern0pt}\ fst\ p{\isadigit{2}}{\isachardoublequoteclose}\isanewline
\ \ \ \ \ \ \isacommand{using}\isamarkupfalse%
\ enum{\isacharunderscore}{\kern0pt}partitions{\isacharunderscore}{\kern0pt}aux{\isacharunderscore}{\kern0pt}len\ enum{\isacharunderscore}{\kern0pt}partitions{\isacharunderscore}{\kern0pt}aux{\isacharunderscore}{\kern0pt}range\ p{\isadigit{2}}{\isacharunderscore}{\kern0pt}e\ k{\isacharunderscore}{\kern0pt}def\ h{\isacharunderscore}{\kern0pt}def\ \isanewline
\ \ \ \ \ \ \isacommand{by}\isamarkupfalse%
\ {\isacharparenleft}{\kern0pt}metis\ mem{\isacharunderscore}{\kern0pt}Collect{\isacharunderscore}{\kern0pt}eq\ nth{\isacharunderscore}{\kern0pt}mem{\isacharparenright}{\kern0pt}\isanewline
\ \ \ \ \isacommand{have}\isamarkupfalse%
\ k{\isacharunderscore}{\kern0pt}eq{\isacharcolon}{\kern0pt}\ {\isachardoublequoteopen}{\isasymAnd}i{\isachardot}{\kern0pt}\ has{\isacharunderscore}{\kern0pt}eq{\isacharunderscore}{\kern0pt}relation\ {\isacharparenleft}{\kern0pt}i\ {\isacharhash}{\kern0pt}\ snd\ p{\isadigit{2}}{\isacharparenright}{\kern0pt}\ {\isacharparenleft}{\kern0pt}a\ {\isacharhash}{\kern0pt}\ y{\isacharparenright}{\kern0pt}\ {\isacharequal}{\kern0pt}\ {\isacharparenleft}{\kern0pt}i\ {\isacharequal}{\kern0pt}\ k{\isacharparenright}{\kern0pt}{\isachardoublequoteclose}\ \isanewline
\ \ \ \ \ \ \isacommand{apply}\isamarkupfalse%
\ {\isacharparenleft}{\kern0pt}simp\ add{\isacharcolon}{\kern0pt}has{\isacharunderscore}{\kern0pt}eq{\isacharunderscore}{\kern0pt}elim\ p{\isadigit{2}}{\isacharunderscore}{\kern0pt}t\ k{\isacharunderscore}{\kern0pt}def{\isacharparenright}{\kern0pt}\isanewline
\ \ \ \ \ \ \isacommand{using}\isamarkupfalse%
\ h{\isacharunderscore}{\kern0pt}def\ has{\isacharunderscore}{\kern0pt}eq{\isacharunderscore}{\kern0pt}relation{\isacharunderscore}{\kern0pt}def\ p{\isadigit{2}}{\isacharunderscore}{\kern0pt}t\ \isacommand{by}\isamarkupfalse%
\ auto\isanewline
\ \ \ \ \isacommand{show}\isamarkupfalse%
\ {\isacharquery}{\kern0pt}thesis\ \isanewline
\ \ \ \ \ \ \isacommand{apply}\isamarkupfalse%
\ {\isacharparenleft}{\kern0pt}simp\ add{\isacharcolon}{\kern0pt}\ filter{\isacharunderscore}{\kern0pt}concat\ length{\isacharunderscore}{\kern0pt}concat\ case{\isacharunderscore}{\kern0pt}prod{\isacharunderscore}{\kern0pt}beta{\isacharprime}{\kern0pt}\ comp{\isacharunderscore}{\kern0pt}def{\isacharparenright}{\kern0pt}\isanewline
\ \ \ \ \ \ \isacommand{apply}\isamarkupfalse%
\ {\isacharparenleft}{\kern0pt}simp\ add{\isacharcolon}{\kern0pt}\ pi{\isacharunderscore}{\kern0pt}def\ p{\isadigit{1}}{\isacharunderscore}{\kern0pt}f\ p{\isadigit{3}}{\isacharunderscore}{\kern0pt}f\ cong{\isacharcolon}{\kern0pt}map{\isacharunderscore}{\kern0pt}cong{\isacharparenright}{\kern0pt}\isanewline
\ \ \ \ \ \ \isacommand{by}\isamarkupfalse%
\ {\isacharparenleft}{\kern0pt}simp\ add{\isacharcolon}{\kern0pt}\ k{\isacharunderscore}{\kern0pt}eq\ k{\isacharunderscore}{\kern0pt}bound\ enum{\isacharunderscore}{\kern0pt}partitions{\isacharunderscore}{\kern0pt}complete{\isacharunderscore}{\kern0pt}aux{\isacharparenright}{\kern0pt}\isanewline
\ \ \isacommand{next}\isamarkupfalse%
\isanewline
\ \ \ \ \isacommand{case}\isamarkupfalse%
\ False\isanewline
\ \ \ \ \isacommand{hence}\isamarkupfalse%
\ {\isachardoublequoteopen}has{\isacharunderscore}{\kern0pt}eq{\isacharunderscore}{\kern0pt}relation\ {\isacharparenleft}{\kern0pt}fst\ p{\isadigit{2}}\ {\isacharhash}{\kern0pt}\ snd\ p{\isadigit{2}}{\isacharparenright}{\kern0pt}\ {\isacharparenleft}{\kern0pt}a\ {\isacharhash}{\kern0pt}\ y{\isacharparenright}{\kern0pt}{\isachardoublequoteclose}\isanewline
\ \ \ \ \ \ \isacommand{apply}\isamarkupfalse%
\ {\isacharparenleft}{\kern0pt}simp\ add{\isacharcolon}{\kern0pt}has{\isacharunderscore}{\kern0pt}eq{\isacharunderscore}{\kern0pt}elim\ p{\isadigit{2}}{\isacharunderscore}{\kern0pt}t{\isacharparenright}{\kern0pt}\isanewline
\ \ \ \ \ \ \isacommand{using}\isamarkupfalse%
\ enum{\isacharunderscore}{\kern0pt}partitions{\isacharunderscore}{\kern0pt}aux{\isacharunderscore}{\kern0pt}range\ p{\isadigit{2}}{\isacharunderscore}{\kern0pt}e\isanewline
\ \ \ \ \ \ \isacommand{by}\isamarkupfalse%
\ {\isacharparenleft}{\kern0pt}metis\ enum{\isacharunderscore}{\kern0pt}partitions{\isacharunderscore}{\kern0pt}aux{\isacharunderscore}{\kern0pt}len\ mem{\isacharunderscore}{\kern0pt}Collect{\isacharunderscore}{\kern0pt}eq\ nat{\isacharunderscore}{\kern0pt}neq{\isacharunderscore}{\kern0pt}iff\ nth{\isacharunderscore}{\kern0pt}mem{\isacharparenright}{\kern0pt}\isanewline
\ \ \ \ \isacommand{moreover}\isamarkupfalse%
\ \isacommand{have}\isamarkupfalse%
\ {\isachardoublequoteopen}{\isasymAnd}i{\isachardot}{\kern0pt}\ i\ {\isacharless}{\kern0pt}\ fst\ p{\isadigit{2}}\ {\isasymLongrightarrow}\ {\isasymnot}{\isacharparenleft}{\kern0pt}has{\isacharunderscore}{\kern0pt}eq{\isacharunderscore}{\kern0pt}relation\ {\isacharparenleft}{\kern0pt}i\ {\isacharhash}{\kern0pt}\ snd\ p{\isadigit{2}}{\isacharparenright}{\kern0pt}\ {\isacharparenleft}{\kern0pt}a\ {\isacharhash}{\kern0pt}\ y{\isacharparenright}{\kern0pt}{\isacharparenright}{\kern0pt}{\isachardoublequoteclose}\ \isanewline
\ \ \ \ \ \ \isacommand{apply}\isamarkupfalse%
\ {\isacharparenleft}{\kern0pt}simp\ add{\isacharcolon}{\kern0pt}has{\isacharunderscore}{\kern0pt}eq{\isacharunderscore}{\kern0pt}elim\ p{\isadigit{2}}{\isacharunderscore}{\kern0pt}t{\isacharparenright}{\kern0pt}\isanewline
\ \ \ \ \ \ \isacommand{by}\isamarkupfalse%
\ {\isacharparenleft}{\kern0pt}metis\ False\ enum{\isacharunderscore}{\kern0pt}partitions{\isacharunderscore}{\kern0pt}aux{\isacharunderscore}{\kern0pt}range\ p{\isadigit{2}}{\isacharunderscore}{\kern0pt}e\ has{\isacharunderscore}{\kern0pt}eq{\isacharunderscore}{\kern0pt}relation{\isacharunderscore}{\kern0pt}def\ in{\isacharunderscore}{\kern0pt}set{\isacharunderscore}{\kern0pt}conv{\isacharunderscore}{\kern0pt}nth\ mem{\isacharunderscore}{\kern0pt}Collect{\isacharunderscore}{\kern0pt}eq\ p{\isadigit{2}}{\isacharunderscore}{\kern0pt}t{\isacharparenright}{\kern0pt}\isanewline
\ \ \ \ \isacommand{ultimately}\isamarkupfalse%
\ \isacommand{show}\isamarkupfalse%
\ {\isacharquery}{\kern0pt}thesis\isanewline
\ \ \ \ \ \ \isacommand{apply}\isamarkupfalse%
\ {\isacharparenleft}{\kern0pt}simp\ add{\isacharcolon}{\kern0pt}\ filter{\isacharunderscore}{\kern0pt}concat\ length{\isacharunderscore}{\kern0pt}concat\ case{\isacharunderscore}{\kern0pt}prod{\isacharunderscore}{\kern0pt}beta{\isacharprime}{\kern0pt}\ comp{\isacharunderscore}{\kern0pt}def{\isacharparenright}{\kern0pt}\isanewline
\ \ \ \ \ \ \isacommand{by}\isamarkupfalse%
\ {\isacharparenleft}{\kern0pt}simp\ add{\isacharcolon}{\kern0pt}\ pi{\isacharunderscore}{\kern0pt}def\ p{\isadigit{1}}{\isacharunderscore}{\kern0pt}f\ p{\isadigit{3}}{\isacharunderscore}{\kern0pt}f\ cong{\isacharcolon}{\kern0pt}map{\isacharunderscore}{\kern0pt}cong{\isacharparenright}{\kern0pt}\isanewline
\ \ \isacommand{qed}\isamarkupfalse%
\isanewline
\isacommand{qed}\isamarkupfalse%
%
\endisatagproof
{\isafoldproof}%
%
\isadelimproof
\isanewline
%
\endisadelimproof
\isanewline
\isacommand{fun}\isamarkupfalse%
\ verify\ \isakeyword{where}\isanewline
\ \ {\isachardoublequoteopen}verify\ r\ x\ {\isadigit{0}}\ {\isacharunderscore}{\kern0pt}\ {\isacharequal}{\kern0pt}\ True{\isachardoublequoteclose}\ {\isacharbar}{\kern0pt}\isanewline
\ \ {\isachardoublequoteopen}verify\ r\ x\ {\isacharparenleft}{\kern0pt}Suc\ n{\isacharparenright}{\kern0pt}\ {\isadigit{0}}\ {\isacharequal}{\kern0pt}\ verify\ r\ x\ n\ n{\isachardoublequoteclose}\ {\isacharbar}{\kern0pt}\isanewline
\ \ {\isachardoublequoteopen}verify\ r\ x\ {\isacharparenleft}{\kern0pt}Suc\ n{\isacharparenright}{\kern0pt}\ {\isacharparenleft}{\kern0pt}Suc\ m{\isacharparenright}{\kern0pt}\ {\isacharequal}{\kern0pt}\ {\isacharparenleft}{\kern0pt}{\isacharparenleft}{\kern0pt}{\isacharparenleft}{\kern0pt}r\ {\isacharbang}{\kern0pt}\ n\ {\isacharequal}{\kern0pt}\ r\ {\isacharbang}{\kern0pt}\ m{\isacharparenright}{\kern0pt}\ {\isacharequal}{\kern0pt}\ {\isacharparenleft}{\kern0pt}x\ {\isacharbang}{\kern0pt}\ n\ {\isacharequal}{\kern0pt}\ x\ {\isacharbang}{\kern0pt}\ m{\isacharparenright}{\kern0pt}{\isacharparenright}{\kern0pt}\ {\isasymand}\ {\isacharparenleft}{\kern0pt}verify\ r\ x\ {\isacharparenleft}{\kern0pt}Suc\ n{\isacharparenright}{\kern0pt}\ m{\isacharparenright}{\kern0pt}{\isacharparenright}{\kern0pt}{\isachardoublequoteclose}\isanewline
\isanewline
\isacommand{lemma}\isamarkupfalse%
\ verify{\isacharunderscore}{\kern0pt}elim{\isacharunderscore}{\kern0pt}{\isadigit{1}}{\isacharcolon}{\kern0pt}\isanewline
\ \ {\isachardoublequoteopen}verify\ r\ x\ {\isacharparenleft}{\kern0pt}Suc\ n{\isacharparenright}{\kern0pt}\ m\ {\isacharequal}{\kern0pt}\ {\isacharparenleft}{\kern0pt}verify\ r\ x\ n\ n\ {\isasymand}\ \ {\isacharparenleft}{\kern0pt}{\isasymforall}i\ {\isacharless}{\kern0pt}\ m{\isachardot}{\kern0pt}\ {\isacharparenleft}{\kern0pt}r\ {\isacharbang}{\kern0pt}\ n\ {\isacharequal}{\kern0pt}\ r\ {\isacharbang}{\kern0pt}\ i{\isacharparenright}{\kern0pt}\ {\isacharequal}{\kern0pt}\ {\isacharparenleft}{\kern0pt}x\ {\isacharbang}{\kern0pt}\ n\ {\isacharequal}{\kern0pt}\ x\ {\isacharbang}{\kern0pt}\ i{\isacharparenright}{\kern0pt}{\isacharparenright}{\kern0pt}{\isacharparenright}{\kern0pt}{\isachardoublequoteclose}\isanewline
%
\isadelimproof
\ \ %
\endisadelimproof
%
\isatagproof
\isacommand{apply}\isamarkupfalse%
\ {\isacharparenleft}{\kern0pt}induction\ m{\isacharcomma}{\kern0pt}\ simp{\isacharcomma}{\kern0pt}\ simp{\isacharparenright}{\kern0pt}\ \isanewline
\ \ \isacommand{using}\isamarkupfalse%
\ less{\isacharunderscore}{\kern0pt}Suc{\isacharunderscore}{\kern0pt}eq\ \isacommand{by}\isamarkupfalse%
\ auto%
\endisatagproof
{\isafoldproof}%
%
\isadelimproof
\isanewline
%
\endisadelimproof
\isanewline
\isacommand{lemma}\isamarkupfalse%
\ verify{\isacharunderscore}{\kern0pt}elim{\isacharcolon}{\kern0pt}\isanewline
\ \ {\isachardoublequoteopen}verify\ r\ x\ m\ m\ {\isacharequal}{\kern0pt}\ {\isacharparenleft}{\kern0pt}{\isasymforall}i\ {\isacharless}{\kern0pt}\ m{\isachardot}{\kern0pt}\ {\isasymforall}j\ {\isacharless}{\kern0pt}\ i{\isachardot}{\kern0pt}\ {\isacharparenleft}{\kern0pt}r\ {\isacharbang}{\kern0pt}\ i\ {\isacharequal}{\kern0pt}\ r\ {\isacharbang}{\kern0pt}\ j{\isacharparenright}{\kern0pt}\ {\isacharequal}{\kern0pt}\ {\isacharparenleft}{\kern0pt}x\ {\isacharbang}{\kern0pt}\ i\ {\isacharequal}{\kern0pt}\ x\ {\isacharbang}{\kern0pt}\ j{\isacharparenright}{\kern0pt}{\isacharparenright}{\kern0pt}{\isachardoublequoteclose}\isanewline
%
\isadelimproof
\ \ %
\endisadelimproof
%
\isatagproof
\isacommand{apply}\isamarkupfalse%
\ {\isacharparenleft}{\kern0pt}induction\ m{\isacharcomma}{\kern0pt}\ simp{\isacharcomma}{\kern0pt}\ simp\ add{\isacharcolon}{\kern0pt}verify{\isacharunderscore}{\kern0pt}elim{\isacharunderscore}{\kern0pt}{\isadigit{1}}{\isacharparenright}{\kern0pt}\isanewline
\ \ \isacommand{apply}\isamarkupfalse%
\ {\isacharparenleft}{\kern0pt}rule\ order{\isacharunderscore}{\kern0pt}antisym{\isacharcomma}{\kern0pt}\ simp{\isacharcomma}{\kern0pt}\ metis\ less{\isacharunderscore}{\kern0pt}antisym\ less{\isacharunderscore}{\kern0pt}trans{\isacharparenright}{\kern0pt}\isanewline
\ \ \isacommand{apply}\isamarkupfalse%
\ {\isacharparenleft}{\kern0pt}simp{\isacharparenright}{\kern0pt}\ \isanewline
\ \ \isacommand{using}\isamarkupfalse%
\ less{\isacharunderscore}{\kern0pt}Suc{\isacharunderscore}{\kern0pt}eq\ \isacommand{by}\isamarkupfalse%
\ presburger%
\endisatagproof
{\isafoldproof}%
%
\isadelimproof
\isanewline
%
\endisadelimproof
\isanewline
\isacommand{lemma}\isamarkupfalse%
\ has{\isacharunderscore}{\kern0pt}eq{\isacharunderscore}{\kern0pt}relation{\isacharunderscore}{\kern0pt}elim{\isacharcolon}{\kern0pt}\isanewline
\ \ {\isachardoublequoteopen}has{\isacharunderscore}{\kern0pt}eq{\isacharunderscore}{\kern0pt}relation\ r\ xs\ {\isacharequal}{\kern0pt}\ {\isacharparenleft}{\kern0pt}length\ r\ {\isacharequal}{\kern0pt}\ length\ xs\ {\isasymand}\ verify\ r\ xs\ {\isacharparenleft}{\kern0pt}length\ xs{\isacharparenright}{\kern0pt}\ {\isacharparenleft}{\kern0pt}length\ xs{\isacharparenright}{\kern0pt}{\isacharparenright}{\kern0pt}{\isachardoublequoteclose}\isanewline
%
\isadelimproof
\ \ %
\endisadelimproof
%
\isatagproof
\isacommand{apply}\isamarkupfalse%
\ {\isacharparenleft}{\kern0pt}simp\ add{\isacharcolon}{\kern0pt}\ has{\isacharunderscore}{\kern0pt}eq{\isacharunderscore}{\kern0pt}relation{\isacharunderscore}{\kern0pt}def\ verify{\isacharunderscore}{\kern0pt}elim{\isacharparenright}{\kern0pt}\ \isanewline
\ \ \isacommand{by}\isamarkupfalse%
\ {\isacharparenleft}{\kern0pt}metis\ {\isacharparenleft}{\kern0pt}mono{\isacharunderscore}{\kern0pt}tags{\isacharcomma}{\kern0pt}\ lifting{\isacharparenright}{\kern0pt}\ less{\isacharunderscore}{\kern0pt}trans\ nat{\isacharunderscore}{\kern0pt}neq{\isacharunderscore}{\kern0pt}iff{\isacharparenright}{\kern0pt}%
\endisatagproof
{\isafoldproof}%
%
\isadelimproof
\isanewline
%
\endisadelimproof
\isanewline
\isacommand{lemma}\isamarkupfalse%
\ sum{\isacharunderscore}{\kern0pt}filter{\isacharcolon}{\kern0pt}\ {\isachardoublequoteopen}sum{\isacharunderscore}{\kern0pt}list\ {\isacharparenleft}{\kern0pt}map\ {\isacharparenleft}{\kern0pt}{\isasymlambda}p{\isachardot}{\kern0pt}\ if\ f\ p\ then\ {\isacharparenleft}{\kern0pt}r{\isacharcolon}{\kern0pt}{\isacharcolon}{\kern0pt}real{\isacharparenright}{\kern0pt}\ else\ {\isadigit{0}}{\isacharparenright}{\kern0pt}\ y{\isacharparenright}{\kern0pt}\ {\isacharequal}{\kern0pt}\ r{\isacharasterisk}{\kern0pt}{\isacharparenleft}{\kern0pt}length\ {\isacharparenleft}{\kern0pt}filter\ f\ y{\isacharparenright}{\kern0pt}{\isacharparenright}{\kern0pt}{\isachardoublequoteclose}\isanewline
%
\isadelimproof
\ \ %
\endisadelimproof
%
\isatagproof
\isacommand{by}\isamarkupfalse%
\ {\isacharparenleft}{\kern0pt}induction\ y{\isacharcomma}{\kern0pt}\ simp{\isacharcomma}{\kern0pt}\ simp\ add{\isacharcolon}{\kern0pt}algebra{\isacharunderscore}{\kern0pt}simps{\isacharparenright}{\kern0pt}%
\endisatagproof
{\isafoldproof}%
%
\isadelimproof
\isanewline
%
\endisadelimproof
\isanewline
\isacommand{lemma}\isamarkupfalse%
\ sum{\isacharunderscore}{\kern0pt}partitions{\isacharcolon}{\kern0pt}\ {\isachardoublequoteopen}sum{\isacharunderscore}{\kern0pt}list\ {\isacharparenleft}{\kern0pt}map\ {\isacharparenleft}{\kern0pt}{\isasymlambda}p{\isachardot}{\kern0pt}\ if\ has{\isacharunderscore}{\kern0pt}eq{\isacharunderscore}{\kern0pt}relation\ p\ x\ then\ {\isacharparenleft}{\kern0pt}r{\isacharcolon}{\kern0pt}{\isacharcolon}{\kern0pt}real{\isacharparenright}{\kern0pt}\ else\ {\isadigit{0}}{\isacharparenright}{\kern0pt}\ {\isacharparenleft}{\kern0pt}enum{\isacharunderscore}{\kern0pt}partitions\ {\isacharparenleft}{\kern0pt}length\ x{\isacharparenright}{\kern0pt}{\isacharparenright}{\kern0pt}{\isacharparenright}{\kern0pt}\ {\isacharequal}{\kern0pt}\ r{\isachardoublequoteclose}\isanewline
%
\isadelimproof
\ \ %
\endisadelimproof
%
\isatagproof
\isacommand{by}\isamarkupfalse%
\ {\isacharparenleft}{\kern0pt}metis\ mult{\isachardot}{\kern0pt}right{\isacharunderscore}{\kern0pt}neutral\ of{\isacharunderscore}{\kern0pt}nat{\isacharunderscore}{\kern0pt}{\isadigit{1}}\ enum{\isacharunderscore}{\kern0pt}partitions{\isacharunderscore}{\kern0pt}complete\ sum{\isacharunderscore}{\kern0pt}filter{\isacharparenright}{\kern0pt}%
\endisatagproof
{\isafoldproof}%
%
\isadelimproof
\isanewline
%
\endisadelimproof
\isanewline
\isacommand{lemma}\isamarkupfalse%
\ sum{\isacharunderscore}{\kern0pt}partitions{\isacharprime}{\kern0pt}{\isacharcolon}{\kern0pt}\ \isanewline
\ \ \isakeyword{assumes}\ {\isachardoublequoteopen}n\ {\isacharequal}{\kern0pt}\ length\ x{\isachardoublequoteclose}\isanewline
\ \ \isakeyword{shows}\ {\isachardoublequoteopen}sum{\isacharunderscore}{\kern0pt}list\ {\isacharparenleft}{\kern0pt}map\ {\isacharparenleft}{\kern0pt}{\isasymlambda}p{\isachardot}{\kern0pt}\ of{\isacharunderscore}{\kern0pt}bool\ {\isacharparenleft}{\kern0pt}has{\isacharunderscore}{\kern0pt}eq{\isacharunderscore}{\kern0pt}relation\ p\ x{\isacharparenright}{\kern0pt}\ {\isacharasterisk}{\kern0pt}\ {\isacharparenleft}{\kern0pt}r{\isacharcolon}{\kern0pt}{\isacharcolon}{\kern0pt}real{\isacharparenright}{\kern0pt}{\isacharparenright}{\kern0pt}\ {\isacharparenleft}{\kern0pt}enum{\isacharunderscore}{\kern0pt}partitions\ n{\isacharparenright}{\kern0pt}{\isacharparenright}{\kern0pt}\ {\isacharequal}{\kern0pt}\ r{\isachardoublequoteclose}\isanewline
%
\isadelimproof
\ \ %
\endisadelimproof
%
\isatagproof
\isacommand{apply}\isamarkupfalse%
\ {\isacharparenleft}{\kern0pt}simp\ add{\isacharcolon}{\kern0pt}of{\isacharunderscore}{\kern0pt}bool{\isacharunderscore}{\kern0pt}def\ comp{\isacharunderscore}{\kern0pt}def\ assms\ del{\isacharcolon}{\kern0pt}enum{\isacharunderscore}{\kern0pt}partitions{\isachardot}{\kern0pt}simps{\isacharparenright}{\kern0pt}\isanewline
\ \ \isacommand{apply}\isamarkupfalse%
\ {\isacharparenleft}{\kern0pt}subst\ {\isacharparenleft}{\kern0pt}{\isadigit{2}}{\isacharparenright}{\kern0pt}\ sum{\isacharunderscore}{\kern0pt}partitions{\isacharbrackleft}{\kern0pt}\isakeyword{where}\ x{\isacharequal}{\kern0pt}{\isachardoublequoteopen}x{\isachardoublequoteclose}\ \isakeyword{and}\ r{\isacharequal}{\kern0pt}{\isachardoublequoteopen}r{\isachardoublequoteclose}{\isacharcomma}{\kern0pt}\ symmetric{\isacharbrackright}{\kern0pt}{\isacharparenright}{\kern0pt}\ \isanewline
\ \ \isacommand{apply}\isamarkupfalse%
\ {\isacharparenleft}{\kern0pt}rule\ arg{\isacharunderscore}{\kern0pt}cong{\isacharbrackleft}{\kern0pt}\isakeyword{where}\ f{\isacharequal}{\kern0pt}{\isachardoublequoteopen}sum{\isacharunderscore}{\kern0pt}list{\isachardoublequoteclose}{\isacharbrackright}{\kern0pt}{\isacharparenright}{\kern0pt}\isanewline
\ \ \isacommand{apply}\isamarkupfalse%
\ {\isacharparenleft}{\kern0pt}rule\ map{\isacharunderscore}{\kern0pt}cong{\isacharcomma}{\kern0pt}\ simp{\isacharparenright}{\kern0pt}\isanewline
\ \ \isacommand{by}\isamarkupfalse%
\ simp%
\endisatagproof
{\isafoldproof}%
%
\isadelimproof
\isanewline
%
\endisadelimproof
\isanewline
\isacommand{lemma}\isamarkupfalse%
\ eq{\isacharunderscore}{\kern0pt}rel{\isacharunderscore}{\kern0pt}obtain{\isacharunderscore}{\kern0pt}bij{\isacharcolon}{\kern0pt}\isanewline
\ \ \isakeyword{assumes}\ {\isachardoublequoteopen}has{\isacharunderscore}{\kern0pt}eq{\isacharunderscore}{\kern0pt}relation\ u\ v{\isachardoublequoteclose}\isanewline
\ \ \isakeyword{obtains}\ f\ \isakeyword{where}\ {\isachardoublequoteopen}bij{\isacharunderscore}{\kern0pt}betw\ f\ {\isacharparenleft}{\kern0pt}set\ u{\isacharparenright}{\kern0pt}\ {\isacharparenleft}{\kern0pt}set\ v{\isacharparenright}{\kern0pt}{\isachardoublequoteclose}\ {\isachardoublequoteopen}{\isasymAnd}y{\isachardot}{\kern0pt}\ y\ {\isasymin}\ set\ u\ {\isasymLongrightarrow}\ count{\isacharunderscore}{\kern0pt}list\ u\ y\ {\isacharequal}{\kern0pt}\ count{\isacharunderscore}{\kern0pt}list\ v\ {\isacharparenleft}{\kern0pt}f\ y{\isacharparenright}{\kern0pt}{\isachardoublequoteclose}\isanewline
%
\isadelimproof
%
\endisadelimproof
%
\isatagproof
\isacommand{proof}\isamarkupfalse%
\ {\isacharminus}{\kern0pt}\isanewline
\ \ \isacommand{define}\isamarkupfalse%
\ A\ \isakeyword{where}\ {\isachardoublequoteopen}A\ {\isacharequal}{\kern0pt}\ {\isacharparenleft}{\kern0pt}{\isasymlambda}x{\isachardot}{\kern0pt}\ {\isacharbraceleft}{\kern0pt}k{\isachardot}{\kern0pt}\ k\ {\isacharless}{\kern0pt}\ length\ u\ {\isasymand}\ u\ {\isacharbang}{\kern0pt}\ k\ {\isacharequal}{\kern0pt}\ x{\isacharbraceright}{\kern0pt}{\isacharparenright}{\kern0pt}{\isachardoublequoteclose}\isanewline
\ \ \isacommand{define}\isamarkupfalse%
\ q\ \isakeyword{where}\ {\isachardoublequoteopen}q\ {\isacharequal}{\kern0pt}\ {\isacharparenleft}{\kern0pt}{\isasymlambda}x{\isachardot}{\kern0pt}\ v\ {\isacharbang}{\kern0pt}\ {\isacharparenleft}{\kern0pt}Min\ {\isacharparenleft}{\kern0pt}A\ x{\isacharparenright}{\kern0pt}{\isacharparenright}{\kern0pt}{\isacharparenright}{\kern0pt}{\isachardoublequoteclose}\isanewline
\isanewline
\ \ \isacommand{have}\isamarkupfalse%
\ A{\isacharunderscore}{\kern0pt}ne{\isacharunderscore}{\kern0pt}iff{\isacharcolon}{\kern0pt}\ {\isachardoublequoteopen}{\isasymAnd}x{\isachardot}{\kern0pt}\ x\ {\isasymin}\ set\ u\ {\isasymLongrightarrow}\ A\ x\ {\isasymnoteq}\ {\isacharbraceleft}{\kern0pt}{\isacharbraceright}{\kern0pt}{\isachardoublequoteclose}\ \isacommand{by}\isamarkupfalse%
\ {\isacharparenleft}{\kern0pt}simp\ add{\isacharcolon}{\kern0pt}A{\isacharunderscore}{\kern0pt}def\ in{\isacharunderscore}{\kern0pt}set{\isacharunderscore}{\kern0pt}conv{\isacharunderscore}{\kern0pt}nth{\isacharparenright}{\kern0pt}\ \isanewline
\ \ \isacommand{have}\isamarkupfalse%
\ f{\isacharunderscore}{\kern0pt}A{\isacharcolon}{\kern0pt}\ {\isachardoublequoteopen}{\isasymAnd}x{\isachardot}{\kern0pt}\ finite\ {\isacharparenleft}{\kern0pt}A\ x{\isacharparenright}{\kern0pt}{\isachardoublequoteclose}\ \isacommand{by}\isamarkupfalse%
\ {\isacharparenleft}{\kern0pt}simp\ add{\isacharcolon}{\kern0pt}A{\isacharunderscore}{\kern0pt}def{\isacharparenright}{\kern0pt}\isanewline
\isanewline
\ \ \isacommand{have}\isamarkupfalse%
\ a{\isacharcolon}{\kern0pt}{\isachardoublequoteopen}inj{\isacharunderscore}{\kern0pt}on\ q\ {\isacharparenleft}{\kern0pt}set\ u{\isacharparenright}{\kern0pt}{\isachardoublequoteclose}\isanewline
\ \ \isacommand{proof}\isamarkupfalse%
\ {\isacharparenleft}{\kern0pt}rule\ inj{\isacharunderscore}{\kern0pt}onI{\isacharparenright}{\kern0pt}\isanewline
\ \ \ \ \isacommand{fix}\isamarkupfalse%
\ x\ y\isanewline
\ \ \ \ \isacommand{assume}\isamarkupfalse%
\ a{\isacharunderscore}{\kern0pt}{\isadigit{1}}{\isacharcolon}{\kern0pt}{\isachardoublequoteopen}x\ {\isasymin}\ set\ u{\isachardoublequoteclose}\ {\isachardoublequoteopen}y\ {\isasymin}\ set\ u{\isachardoublequoteclose}\isanewline
\ \ \ \ \isacommand{have}\isamarkupfalse%
\ {\isachardoublequoteopen}length\ u\ {\isachargreater}{\kern0pt}\ {\isadigit{0}}{\isachardoublequoteclose}\ \isacommand{using}\isamarkupfalse%
\ a{\isacharunderscore}{\kern0pt}{\isadigit{1}}\ \isacommand{by}\isamarkupfalse%
\ force\isanewline
\ \ \ \ \isacommand{define}\isamarkupfalse%
\ xi\ \isakeyword{where}\ {\isachardoublequoteopen}xi\ {\isacharequal}{\kern0pt}\ Min\ {\isacharparenleft}{\kern0pt}A\ x{\isacharparenright}{\kern0pt}{\isachardoublequoteclose}\isanewline
\ \ \ \ \isacommand{have}\isamarkupfalse%
\ xi{\isacharunderscore}{\kern0pt}l{\isacharcolon}{\kern0pt}\ {\isachardoublequoteopen}xi\ {\isacharless}{\kern0pt}\ length\ u{\isachardoublequoteclose}\isanewline
\ \ \ \ \ \ \isacommand{using}\isamarkupfalse%
\ Min{\isacharunderscore}{\kern0pt}in{\isacharbrackleft}{\kern0pt}OF\ f{\isacharunderscore}{\kern0pt}A\ A{\isacharunderscore}{\kern0pt}ne{\isacharunderscore}{\kern0pt}iff{\isacharbrackleft}{\kern0pt}OF\ a{\isacharunderscore}{\kern0pt}{\isadigit{1}}{\isacharparenleft}{\kern0pt}{\isadigit{1}}{\isacharparenright}{\kern0pt}{\isacharbrackright}{\kern0pt}{\isacharbrackright}{\kern0pt}\isanewline
\ \ \ \ \ \ \isacommand{by}\isamarkupfalse%
\ {\isacharparenleft}{\kern0pt}simp\ add{\isacharcolon}{\kern0pt}xi{\isacharunderscore}{\kern0pt}def\ A{\isacharunderscore}{\kern0pt}def{\isacharparenright}{\kern0pt}\ \isanewline
\ \ \ \ \isacommand{have}\isamarkupfalse%
\ xi{\isacharunderscore}{\kern0pt}v{\isacharcolon}{\kern0pt}\ {\isachardoublequoteopen}u\ {\isacharbang}{\kern0pt}\ xi\ {\isacharequal}{\kern0pt}\ x{\isachardoublequoteclose}\ \isanewline
\ \ \ \ \ \ \isacommand{using}\isamarkupfalse%
\ Min{\isacharunderscore}{\kern0pt}in{\isacharbrackleft}{\kern0pt}OF\ f{\isacharunderscore}{\kern0pt}A\ A{\isacharunderscore}{\kern0pt}ne{\isacharunderscore}{\kern0pt}iff{\isacharbrackleft}{\kern0pt}OF\ a{\isacharunderscore}{\kern0pt}{\isadigit{1}}{\isacharparenleft}{\kern0pt}{\isadigit{1}}{\isacharparenright}{\kern0pt}{\isacharbrackright}{\kern0pt}{\isacharbrackright}{\kern0pt}\isanewline
\ \ \ \ \ \ \isacommand{by}\isamarkupfalse%
\ {\isacharparenleft}{\kern0pt}simp\ add{\isacharcolon}{\kern0pt}xi{\isacharunderscore}{\kern0pt}def\ A{\isacharunderscore}{\kern0pt}def{\isacharparenright}{\kern0pt}\ \isanewline
\ \ \ \ \isacommand{define}\isamarkupfalse%
\ yi\ \isakeyword{where}\ {\isachardoublequoteopen}yi\ {\isacharequal}{\kern0pt}\ Min\ {\isacharparenleft}{\kern0pt}A\ y{\isacharparenright}{\kern0pt}{\isachardoublequoteclose}\isanewline
\ \ \ \ \isacommand{have}\isamarkupfalse%
\ yi{\isacharunderscore}{\kern0pt}l{\isacharcolon}{\kern0pt}\ {\isachardoublequoteopen}yi\ {\isacharless}{\kern0pt}\ length\ u{\isachardoublequoteclose}\ \isanewline
\ \ \ \ \ \ \isacommand{using}\isamarkupfalse%
\ Min{\isacharunderscore}{\kern0pt}in{\isacharbrackleft}{\kern0pt}OF\ f{\isacharunderscore}{\kern0pt}A\ A{\isacharunderscore}{\kern0pt}ne{\isacharunderscore}{\kern0pt}iff{\isacharbrackleft}{\kern0pt}OF\ a{\isacharunderscore}{\kern0pt}{\isadigit{1}}{\isacharparenleft}{\kern0pt}{\isadigit{2}}{\isacharparenright}{\kern0pt}{\isacharbrackright}{\kern0pt}{\isacharbrackright}{\kern0pt}\isanewline
\ \ \ \ \ \ \isacommand{by}\isamarkupfalse%
\ {\isacharparenleft}{\kern0pt}simp\ add{\isacharcolon}{\kern0pt}yi{\isacharunderscore}{\kern0pt}def\ A{\isacharunderscore}{\kern0pt}def{\isacharparenright}{\kern0pt}\ \isanewline
\ \ \ \ \isacommand{have}\isamarkupfalse%
\ yi{\isacharunderscore}{\kern0pt}v{\isacharcolon}{\kern0pt}\ {\isachardoublequoteopen}u\ {\isacharbang}{\kern0pt}\ yi\ {\isacharequal}{\kern0pt}\ y{\isachardoublequoteclose}\ \isanewline
\ \ \ \ \ \ \isacommand{using}\isamarkupfalse%
\ Min{\isacharunderscore}{\kern0pt}in{\isacharbrackleft}{\kern0pt}OF\ f{\isacharunderscore}{\kern0pt}A\ A{\isacharunderscore}{\kern0pt}ne{\isacharunderscore}{\kern0pt}iff{\isacharbrackleft}{\kern0pt}OF\ a{\isacharunderscore}{\kern0pt}{\isadigit{1}}{\isacharparenleft}{\kern0pt}{\isadigit{2}}{\isacharparenright}{\kern0pt}{\isacharbrackright}{\kern0pt}{\isacharbrackright}{\kern0pt}\isanewline
\ \ \ \ \ \ \isacommand{by}\isamarkupfalse%
\ {\isacharparenleft}{\kern0pt}simp\ add{\isacharcolon}{\kern0pt}yi{\isacharunderscore}{\kern0pt}def\ A{\isacharunderscore}{\kern0pt}def{\isacharparenright}{\kern0pt}\ \isanewline
\isanewline
\ \ \ \ \isacommand{assume}\isamarkupfalse%
\ {\isachardoublequoteopen}q\ x\ {\isacharequal}{\kern0pt}\ q\ y{\isachardoublequoteclose}\isanewline
\ \ \ \ \isacommand{hence}\isamarkupfalse%
\ {\isachardoublequoteopen}v\ {\isacharbang}{\kern0pt}\ xi\ {\isacharequal}{\kern0pt}\ v\ {\isacharbang}{\kern0pt}\ yi{\isachardoublequoteclose}\isanewline
\ \ \ \ \ \ \isacommand{by}\isamarkupfalse%
\ {\isacharparenleft}{\kern0pt}simp\ add{\isacharcolon}{\kern0pt}q{\isacharunderscore}{\kern0pt}def\ xi{\isacharunderscore}{\kern0pt}def\ yi{\isacharunderscore}{\kern0pt}def{\isacharparenright}{\kern0pt}\isanewline
\ \ \ \ \isacommand{hence}\isamarkupfalse%
\ {\isachardoublequoteopen}u\ {\isacharbang}{\kern0pt}\ xi\ {\isacharequal}{\kern0pt}\ u\ {\isacharbang}{\kern0pt}\ yi{\isachardoublequoteclose}\isanewline
\ \ \ \ \ \ \isacommand{by}\isamarkupfalse%
\ {\isacharparenleft}{\kern0pt}metis\ {\isacharparenleft}{\kern0pt}no{\isacharunderscore}{\kern0pt}types{\isacharcomma}{\kern0pt}\ lifting{\isacharparenright}{\kern0pt}\ has{\isacharunderscore}{\kern0pt}eq{\isacharunderscore}{\kern0pt}relation{\isacharunderscore}{\kern0pt}def\ assms{\isacharparenleft}{\kern0pt}{\isadigit{1}}{\isacharparenright}{\kern0pt}\ xi{\isacharunderscore}{\kern0pt}l\ yi{\isacharunderscore}{\kern0pt}l{\isacharparenright}{\kern0pt}\isanewline
\ \ \ \ \isacommand{thus}\isamarkupfalse%
\ {\isachardoublequoteopen}x\ {\isacharequal}{\kern0pt}\ y{\isachardoublequoteclose}\isanewline
\ \ \ \ \ \ \isacommand{using}\isamarkupfalse%
\ yi{\isacharunderscore}{\kern0pt}v\ xi{\isacharunderscore}{\kern0pt}v\ \isacommand{by}\isamarkupfalse%
\ blast\isanewline
\ \ \isacommand{qed}\isamarkupfalse%
\isanewline
\isanewline
\ \ \isacommand{have}\isamarkupfalse%
\ b{\isacharcolon}{\kern0pt}{\isachardoublequoteopen}{\isasymAnd}y{\isachardot}{\kern0pt}\ y\ {\isasymin}\ set\ u\ {\isasymLongrightarrow}\ count{\isacharunderscore}{\kern0pt}list\ u\ y\ {\isacharequal}{\kern0pt}\ count{\isacharunderscore}{\kern0pt}list\ v\ {\isacharparenleft}{\kern0pt}q\ y{\isacharparenright}{\kern0pt}{\isachardoublequoteclose}\isanewline
\ \ \isacommand{proof}\isamarkupfalse%
\ {\isacharminus}{\kern0pt}\isanewline
\ \ \ \ \isacommand{fix}\isamarkupfalse%
\ y\isanewline
\ \ \ \ \isacommand{assume}\isamarkupfalse%
\ b{\isacharunderscore}{\kern0pt}{\isadigit{1}}{\isacharcolon}{\kern0pt}{\isachardoublequoteopen}y\ {\isasymin}\ set\ u{\isachardoublequoteclose}\isanewline
\ \ \ \ \isacommand{define}\isamarkupfalse%
\ i\ \isakeyword{where}\ {\isachardoublequoteopen}i\ {\isacharequal}{\kern0pt}\ Min\ {\isacharparenleft}{\kern0pt}A\ y{\isacharparenright}{\kern0pt}{\isachardoublequoteclose}\isanewline
\ \ \ \ \isacommand{have}\isamarkupfalse%
\ i{\isacharunderscore}{\kern0pt}bound{\isacharcolon}{\kern0pt}\ {\isachardoublequoteopen}i\ {\isacharless}{\kern0pt}\ length\ u{\isachardoublequoteclose}\ \isanewline
\ \ \ \ \ \ \isacommand{using}\isamarkupfalse%
\ Min{\isacharunderscore}{\kern0pt}in{\isacharbrackleft}{\kern0pt}OF\ f{\isacharunderscore}{\kern0pt}A\ A{\isacharunderscore}{\kern0pt}ne{\isacharunderscore}{\kern0pt}iff{\isacharbrackleft}{\kern0pt}OF\ b{\isacharunderscore}{\kern0pt}{\isadigit{1}}{\isacharbrackright}{\kern0pt}{\isacharbrackright}{\kern0pt}\isanewline
\ \ \ \ \ \ \isacommand{by}\isamarkupfalse%
\ {\isacharparenleft}{\kern0pt}simp\ add{\isacharcolon}{\kern0pt}i{\isacharunderscore}{\kern0pt}def\ A{\isacharunderscore}{\kern0pt}def{\isacharparenright}{\kern0pt}\ \isanewline
\ \ \ \ \isacommand{have}\isamarkupfalse%
\ y{\isacharunderscore}{\kern0pt}def{\isacharcolon}{\kern0pt}\ {\isachardoublequoteopen}y\ {\isacharequal}{\kern0pt}\ u\ {\isacharbang}{\kern0pt}\ i{\isachardoublequoteclose}\isanewline
\ \ \ \ \ \ \isacommand{using}\isamarkupfalse%
\ Min{\isacharunderscore}{\kern0pt}in{\isacharbrackleft}{\kern0pt}OF\ f{\isacharunderscore}{\kern0pt}A\ A{\isacharunderscore}{\kern0pt}ne{\isacharunderscore}{\kern0pt}iff{\isacharbrackleft}{\kern0pt}OF\ b{\isacharunderscore}{\kern0pt}{\isadigit{1}}{\isacharbrackright}{\kern0pt}{\isacharbrackright}{\kern0pt}\isanewline
\ \ \ \ \ \ \isacommand{by}\isamarkupfalse%
\ {\isacharparenleft}{\kern0pt}simp\ add{\isacharcolon}{\kern0pt}i{\isacharunderscore}{\kern0pt}def\ A{\isacharunderscore}{\kern0pt}def{\isacharparenright}{\kern0pt}\ \isanewline
\isanewline
\ \ \ \ \isacommand{have}\isamarkupfalse%
\ {\isachardoublequoteopen}count{\isacharunderscore}{\kern0pt}list\ u\ y\ {\isacharequal}{\kern0pt}\ \ card\ {\isacharbraceleft}{\kern0pt}k{\isachardot}{\kern0pt}\ k\ {\isacharless}{\kern0pt}\ length\ u\ {\isasymand}\ u\ {\isacharbang}{\kern0pt}\ k\ {\isacharequal}{\kern0pt}\ u\ {\isacharbang}{\kern0pt}\ i{\isacharbraceright}{\kern0pt}{\isachardoublequoteclose}\isanewline
\ \ \ \ \ \ \isacommand{by}\isamarkupfalse%
\ {\isacharparenleft}{\kern0pt}simp\ add{\isacharcolon}{\kern0pt}count{\isacharunderscore}{\kern0pt}list{\isacharunderscore}{\kern0pt}card\ y{\isacharunderscore}{\kern0pt}def{\isacharparenright}{\kern0pt}\isanewline
\ \ \ \ \isacommand{also}\isamarkupfalse%
\ \isacommand{have}\isamarkupfalse%
\ {\isachardoublequoteopen}{\isachardot}{\kern0pt}{\isachardot}{\kern0pt}{\isachardot}{\kern0pt}\ {\isacharequal}{\kern0pt}\ card\ {\isacharbraceleft}{\kern0pt}k{\isachardot}{\kern0pt}\ k\ {\isacharless}{\kern0pt}\ length\ v\ {\isasymand}\ v\ {\isacharbang}{\kern0pt}\ k\ {\isacharequal}{\kern0pt}\ v\ {\isacharbang}{\kern0pt}\ i{\isacharbraceright}{\kern0pt}{\isachardoublequoteclose}\isanewline
\ \ \ \ \ \ \isacommand{apply}\isamarkupfalse%
\ {\isacharparenleft}{\kern0pt}rule\ arg{\isacharunderscore}{\kern0pt}cong{\isacharbrackleft}{\kern0pt}\isakeyword{where}\ f{\isacharequal}{\kern0pt}{\isachardoublequoteopen}card{\isachardoublequoteclose}{\isacharbrackright}{\kern0pt}{\isacharparenright}{\kern0pt}\isanewline
\ \ \ \ \ \ \isacommand{apply}\isamarkupfalse%
\ {\isacharparenleft}{\kern0pt}rule\ set{\isacharunderscore}{\kern0pt}eqI{\isacharcomma}{\kern0pt}\ simp{\isacharparenright}{\kern0pt}\isanewline
\ \ \ \ \ \ \isacommand{by}\isamarkupfalse%
\ {\isacharparenleft}{\kern0pt}metis\ {\isacharparenleft}{\kern0pt}no{\isacharunderscore}{\kern0pt}types{\isacharcomma}{\kern0pt}\ lifting{\isacharparenright}{\kern0pt}\ assms{\isacharparenleft}{\kern0pt}{\isadigit{1}}{\isacharparenright}{\kern0pt}\ has{\isacharunderscore}{\kern0pt}eq{\isacharunderscore}{\kern0pt}relation{\isacharunderscore}{\kern0pt}def\ i{\isacharunderscore}{\kern0pt}bound{\isacharparenright}{\kern0pt}\isanewline
\ \ \ \ \isacommand{also}\isamarkupfalse%
\ \isacommand{have}\isamarkupfalse%
\ {\isachardoublequoteopen}{\isachardot}{\kern0pt}{\isachardot}{\kern0pt}{\isachardot}{\kern0pt}\ {\isacharequal}{\kern0pt}\ card\ {\isacharbraceleft}{\kern0pt}k{\isachardot}{\kern0pt}\ k\ {\isacharless}{\kern0pt}\ length\ v\ {\isasymand}\ v\ {\isacharbang}{\kern0pt}\ k\ {\isacharequal}{\kern0pt}\ q\ y{\isacharbraceright}{\kern0pt}{\isachardoublequoteclose}\isanewline
\ \ \ \ \ \ \isacommand{by}\isamarkupfalse%
\ {\isacharparenleft}{\kern0pt}simp\ add{\isacharcolon}{\kern0pt}q{\isacharunderscore}{\kern0pt}def\ i{\isacharunderscore}{\kern0pt}def{\isacharparenright}{\kern0pt}\isanewline
\ \ \ \ \isacommand{also}\isamarkupfalse%
\ \isacommand{have}\isamarkupfalse%
\ {\isachardoublequoteopen}{\isachardot}{\kern0pt}{\isachardot}{\kern0pt}{\isachardot}{\kern0pt}\ {\isacharequal}{\kern0pt}\ count{\isacharunderscore}{\kern0pt}list\ v\ {\isacharparenleft}{\kern0pt}q\ y{\isacharparenright}{\kern0pt}{\isachardoublequoteclose}\isanewline
\ \ \ \ \ \ \isacommand{by}\isamarkupfalse%
\ {\isacharparenleft}{\kern0pt}simp\ add{\isacharcolon}{\kern0pt}count{\isacharunderscore}{\kern0pt}list{\isacharunderscore}{\kern0pt}card{\isacharparenright}{\kern0pt}\isanewline
\ \ \ \ \isacommand{finally}\isamarkupfalse%
\ \isacommand{show}\isamarkupfalse%
\ {\isachardoublequoteopen}count{\isacharunderscore}{\kern0pt}list\ u\ y\ {\isacharequal}{\kern0pt}\ count{\isacharunderscore}{\kern0pt}list\ v\ {\isacharparenleft}{\kern0pt}q\ y{\isacharparenright}{\kern0pt}{\isachardoublequoteclose}\isanewline
\ \ \ \ \ \ \isacommand{by}\isamarkupfalse%
\ simp\isanewline
\ \ \isacommand{qed}\isamarkupfalse%
\isanewline
\isanewline
\ \ \isacommand{have}\isamarkupfalse%
\ c{\isacharcolon}{\kern0pt}{\isachardoublequoteopen}q\ {\isacharbackquote}{\kern0pt}\ set\ u\ {\isasymsubseteq}\ set\ v{\isachardoublequoteclose}\ \isanewline
\ \ \ \ \isacommand{apply}\isamarkupfalse%
\ {\isacharparenleft}{\kern0pt}rule\ image{\isacharunderscore}{\kern0pt}subsetI{\isacharparenright}{\kern0pt}\isanewline
\ \ \ \ \isacommand{by}\isamarkupfalse%
\ {\isacharparenleft}{\kern0pt}metis\ b\ count{\isacharunderscore}{\kern0pt}list{\isacharunderscore}{\kern0pt}gr{\isacharunderscore}{\kern0pt}{\isadigit{1}}{\isacharparenright}{\kern0pt}\isanewline
\isanewline
\ \ \isacommand{have}\isamarkupfalse%
\ d{\isacharunderscore}{\kern0pt}{\isadigit{1}}{\isacharcolon}{\kern0pt}{\isachardoublequoteopen}length\ v\ {\isacharequal}{\kern0pt}\ length\ u{\isachardoublequoteclose}\ \isacommand{using}\isamarkupfalse%
\ assms\ has{\isacharunderscore}{\kern0pt}eq{\isacharunderscore}{\kern0pt}relation{\isacharunderscore}{\kern0pt}def\ \isacommand{by}\isamarkupfalse%
\ blast\isanewline
\ \ \isacommand{also}\isamarkupfalse%
\ \isacommand{have}\isamarkupfalse%
\ {\isachardoublequoteopen}{\isachardot}{\kern0pt}{\isachardot}{\kern0pt}{\isachardot}{\kern0pt}\ {\isacharequal}{\kern0pt}\ sum\ {\isacharparenleft}{\kern0pt}count{\isacharunderscore}{\kern0pt}list\ u{\isacharparenright}{\kern0pt}\ {\isacharparenleft}{\kern0pt}set\ u{\isacharparenright}{\kern0pt}{\isachardoublequoteclose}\isanewline
\ \ \ \ \isacommand{by}\isamarkupfalse%
\ {\isacharparenleft}{\kern0pt}simp\ add{\isacharcolon}{\kern0pt}sum{\isacharunderscore}{\kern0pt}count{\isacharunderscore}{\kern0pt}set{\isacharparenright}{\kern0pt}\isanewline
\ \ \isacommand{also}\isamarkupfalse%
\ \isacommand{have}\isamarkupfalse%
\ {\isachardoublequoteopen}{\isachardot}{\kern0pt}{\isachardot}{\kern0pt}{\isachardot}{\kern0pt}\ {\isacharequal}{\kern0pt}\ sum\ {\isacharparenleft}{\kern0pt}{\isacharparenleft}{\kern0pt}count{\isacharunderscore}{\kern0pt}list\ v{\isacharparenright}{\kern0pt}\ {\isasymcirc}\ q{\isacharparenright}{\kern0pt}\ {\isacharparenleft}{\kern0pt}set\ u{\isacharparenright}{\kern0pt}{\isachardoublequoteclose}\isanewline
\ \ \ \ \isacommand{by}\isamarkupfalse%
\ {\isacharparenleft}{\kern0pt}rule\ sum{\isachardot}{\kern0pt}cong{\isacharcomma}{\kern0pt}\ simp{\isacharcomma}{\kern0pt}\ simp\ add{\isacharcolon}{\kern0pt}comp{\isacharunderscore}{\kern0pt}def\ b{\isacharparenright}{\kern0pt}\isanewline
\ \ \isacommand{also}\isamarkupfalse%
\ \isacommand{have}\isamarkupfalse%
\ {\isachardoublequoteopen}{\isachardot}{\kern0pt}{\isachardot}{\kern0pt}{\isachardot}{\kern0pt}\ {\isacharequal}{\kern0pt}\ sum\ {\isacharparenleft}{\kern0pt}count{\isacharunderscore}{\kern0pt}list\ v{\isacharparenright}{\kern0pt}\ {\isacharparenleft}{\kern0pt}q\ {\isacharbackquote}{\kern0pt}\ set\ u{\isacharparenright}{\kern0pt}{\isachardoublequoteclose}\isanewline
\ \ \ \ \isacommand{by}\isamarkupfalse%
\ {\isacharparenleft}{\kern0pt}rule\ sum{\isachardot}{\kern0pt}reindex{\isacharbrackleft}{\kern0pt}OF\ a{\isacharcomma}{\kern0pt}\ symmetric{\isacharbrackright}{\kern0pt}{\isacharparenright}{\kern0pt}\isanewline
\ \ \isacommand{finally}\isamarkupfalse%
\ \isacommand{have}\isamarkupfalse%
\ d{\isacharunderscore}{\kern0pt}{\isadigit{1}}{\isacharcolon}{\kern0pt}{\isachardoublequoteopen}sum\ {\isacharparenleft}{\kern0pt}count{\isacharunderscore}{\kern0pt}list\ v{\isacharparenright}{\kern0pt}\ {\isacharparenleft}{\kern0pt}q\ {\isacharbackquote}{\kern0pt}\ set\ u{\isacharparenright}{\kern0pt}\ {\isacharequal}{\kern0pt}\ length\ v{\isachardoublequoteclose}\isanewline
\ \ \ \ \isacommand{by}\isamarkupfalse%
\ simp\isanewline
\isanewline
\ \ \isacommand{have}\isamarkupfalse%
\ {\isachardoublequoteopen}sum\ {\isacharparenleft}{\kern0pt}count{\isacharunderscore}{\kern0pt}list\ v{\isacharparenright}{\kern0pt}\ {\isacharparenleft}{\kern0pt}q\ {\isacharbackquote}{\kern0pt}\ set\ u{\isacharparenright}{\kern0pt}\ {\isacharplus}{\kern0pt}\ sum\ {\isacharparenleft}{\kern0pt}count{\isacharunderscore}{\kern0pt}list\ v{\isacharparenright}{\kern0pt}\ {\isacharparenleft}{\kern0pt}set\ v\ {\isacharminus}{\kern0pt}\ {\isacharparenleft}{\kern0pt}q\ {\isacharbackquote}{\kern0pt}\ set\ u{\isacharparenright}{\kern0pt}{\isacharparenright}{\kern0pt}\ {\isacharequal}{\kern0pt}\ sum\ {\isacharparenleft}{\kern0pt}count{\isacharunderscore}{\kern0pt}list\ v{\isacharparenright}{\kern0pt}\ {\isacharparenleft}{\kern0pt}set\ v{\isacharparenright}{\kern0pt}{\isachardoublequoteclose}\isanewline
\ \ \ \ \isacommand{apply}\isamarkupfalse%
\ {\isacharparenleft}{\kern0pt}subst\ sum{\isachardot}{\kern0pt}union{\isacharunderscore}{\kern0pt}disjoint{\isacharbrackleft}{\kern0pt}symmetric{\isacharbrackright}{\kern0pt}{\isacharcomma}{\kern0pt}\ simp{\isacharcomma}{\kern0pt}\ simp{\isacharcomma}{\kern0pt}\ simp{\isacharparenright}{\kern0pt}\isanewline
\ \ \ \ \isacommand{apply}\isamarkupfalse%
\ {\isacharparenleft}{\kern0pt}rule\ sum{\isachardot}{\kern0pt}cong{\isacharparenright}{\kern0pt}\isanewline
\ \ \ \ \isacommand{using}\isamarkupfalse%
\ c\ \isacommand{apply}\isamarkupfalse%
\ blast\isanewline
\ \ \ \ \isacommand{by}\isamarkupfalse%
\ simp\isanewline
\ \ \isacommand{also}\isamarkupfalse%
\ \isacommand{have}\isamarkupfalse%
\ {\isachardoublequoteopen}{\isachardot}{\kern0pt}{\isachardot}{\kern0pt}{\isachardot}{\kern0pt}\ {\isacharequal}{\kern0pt}\ length\ v{\isachardoublequoteclose}\isanewline
\ \ \ \ \isacommand{by}\isamarkupfalse%
\ {\isacharparenleft}{\kern0pt}simp\ add{\isacharcolon}{\kern0pt}sum{\isacharunderscore}{\kern0pt}count{\isacharunderscore}{\kern0pt}set{\isacharparenright}{\kern0pt}\isanewline
\ \ \isacommand{finally}\isamarkupfalse%
\ \isacommand{have}\isamarkupfalse%
\ d{\isacharunderscore}{\kern0pt}{\isadigit{2}}{\isacharcolon}{\kern0pt}{\isachardoublequoteopen}sum\ {\isacharparenleft}{\kern0pt}count{\isacharunderscore}{\kern0pt}list\ v{\isacharparenright}{\kern0pt}\ {\isacharparenleft}{\kern0pt}q\ {\isacharbackquote}{\kern0pt}\ set\ u{\isacharparenright}{\kern0pt}\ {\isacharplus}{\kern0pt}\ sum\ {\isacharparenleft}{\kern0pt}count{\isacharunderscore}{\kern0pt}list\ v{\isacharparenright}{\kern0pt}\ {\isacharparenleft}{\kern0pt}set\ v\ {\isacharminus}{\kern0pt}\ {\isacharparenleft}{\kern0pt}q\ {\isacharbackquote}{\kern0pt}\ set\ u{\isacharparenright}{\kern0pt}{\isacharparenright}{\kern0pt}\ {\isacharequal}{\kern0pt}\ length\ v{\isachardoublequoteclose}\ \isacommand{by}\isamarkupfalse%
\ simp\isanewline
\isanewline
\ \ \isacommand{have}\isamarkupfalse%
\ {\isachardoublequoteopen}sum\ {\isacharparenleft}{\kern0pt}count{\isacharunderscore}{\kern0pt}list\ v{\isacharparenright}{\kern0pt}\ {\isacharparenleft}{\kern0pt}set\ v\ {\isacharminus}{\kern0pt}\ {\isacharparenleft}{\kern0pt}q\ {\isacharbackquote}{\kern0pt}\ set\ u{\isacharparenright}{\kern0pt}{\isacharparenright}{\kern0pt}\ {\isacharequal}{\kern0pt}\ {\isadigit{0}}{\isachardoublequoteclose}\isanewline
\ \ \ \ \isacommand{using}\isamarkupfalse%
\ d{\isacharunderscore}{\kern0pt}{\isadigit{1}}\ d{\isacharunderscore}{\kern0pt}{\isadigit{2}}\ \isacommand{by}\isamarkupfalse%
\ linarith\isanewline
\isanewline
\ \ \isacommand{hence}\isamarkupfalse%
\ {\isachardoublequoteopen}{\isasymAnd}x{\isachardot}{\kern0pt}\ x\ {\isasymin}\ {\isacharparenleft}{\kern0pt}set\ v\ {\isacharminus}{\kern0pt}\ {\isacharparenleft}{\kern0pt}q\ {\isacharbackquote}{\kern0pt}\ set\ u{\isacharparenright}{\kern0pt}{\isacharparenright}{\kern0pt}\ {\isasymLongrightarrow}\ count{\isacharunderscore}{\kern0pt}list\ v\ x\ {\isasymle}\ {\isadigit{0}}{\isachardoublequoteclose}\isanewline
\ \ \ \ \isacommand{using}\isamarkupfalse%
\ member{\isacharunderscore}{\kern0pt}le{\isacharunderscore}{\kern0pt}sum\ \isacommand{by}\isamarkupfalse%
\ simp\isanewline
\ \ \isacommand{hence}\isamarkupfalse%
\ {\isachardoublequoteopen}{\isasymAnd}x{\isachardot}{\kern0pt}\ x\ {\isasymin}\ {\isacharparenleft}{\kern0pt}set\ v\ {\isacharminus}{\kern0pt}\ {\isacharparenleft}{\kern0pt}q\ {\isacharbackquote}{\kern0pt}\ set\ u{\isacharparenright}{\kern0pt}{\isacharparenright}{\kern0pt}\ {\isasymLongrightarrow}\ False{\isachardoublequoteclose}\isanewline
\ \ \ \ \isacommand{by}\isamarkupfalse%
\ {\isacharparenleft}{\kern0pt}metis\ count{\isacharunderscore}{\kern0pt}list{\isacharunderscore}{\kern0pt}gr{\isacharunderscore}{\kern0pt}{\isadigit{1}}\ Diff{\isacharunderscore}{\kern0pt}iff\ le{\isacharunderscore}{\kern0pt}{\isadigit{0}}{\isacharunderscore}{\kern0pt}eq\ not{\isacharunderscore}{\kern0pt}one{\isacharunderscore}{\kern0pt}le{\isacharunderscore}{\kern0pt}zero{\isacharparenright}{\kern0pt}\isanewline
\ \ \isacommand{hence}\isamarkupfalse%
\ {\isachardoublequoteopen}set\ v\ {\isacharminus}{\kern0pt}\ {\isacharparenleft}{\kern0pt}q\ {\isacharbackquote}{\kern0pt}\ set\ u{\isacharparenright}{\kern0pt}\ {\isacharequal}{\kern0pt}\ {\isacharbraceleft}{\kern0pt}{\isacharbraceright}{\kern0pt}{\isachardoublequoteclose}\isanewline
\ \ \ \ \isacommand{by}\isamarkupfalse%
\ blast\isanewline
\isanewline
\ \ \isacommand{hence}\isamarkupfalse%
\ e{\isacharcolon}{\kern0pt}\ {\isachardoublequoteopen}q\ {\isacharbackquote}{\kern0pt}\ set\ u\ {\isacharequal}{\kern0pt}\ set\ v{\isachardoublequoteclose}\isanewline
\ \ \ \ \isacommand{using}\isamarkupfalse%
\ c\ \isacommand{by}\isamarkupfalse%
\ blast\isanewline
\isanewline
\ \ \isacommand{have}\isamarkupfalse%
\ d{\isacharcolon}{\kern0pt}{\isachardoublequoteopen}bij{\isacharunderscore}{\kern0pt}betw\ q\ {\isacharparenleft}{\kern0pt}set\ u{\isacharparenright}{\kern0pt}\ {\isacharparenleft}{\kern0pt}set\ v{\isacharparenright}{\kern0pt}{\isachardoublequoteclose}\isanewline
\ \ \ \ \isacommand{apply}\isamarkupfalse%
\ {\isacharparenleft}{\kern0pt}simp\ add{\isacharcolon}{\kern0pt}\ bij{\isacharunderscore}{\kern0pt}betw{\isacharunderscore}{\kern0pt}def{\isacharparenright}{\kern0pt}\isanewline
\ \ \ \ \isacommand{using}\isamarkupfalse%
\ c\ e\ a\ \isacommand{by}\isamarkupfalse%
\ blast\isanewline
\ \ \isacommand{have}\isamarkupfalse%
\ {\isachardoublequoteopen}{\isasymexists}f{\isachardot}{\kern0pt}\ bij{\isacharunderscore}{\kern0pt}betw\ f\ {\isacharparenleft}{\kern0pt}set\ u{\isacharparenright}{\kern0pt}\ {\isacharparenleft}{\kern0pt}set\ v{\isacharparenright}{\kern0pt}\ {\isasymand}\ {\isacharparenleft}{\kern0pt}{\isasymforall}y\ {\isasymin}\ set\ u{\isachardot}{\kern0pt}\ count{\isacharunderscore}{\kern0pt}list\ u\ y\ {\isacharequal}{\kern0pt}\ count{\isacharunderscore}{\kern0pt}list\ v\ {\isacharparenleft}{\kern0pt}f\ y{\isacharparenright}{\kern0pt}{\isacharparenright}{\kern0pt}{\isachardoublequoteclose}\isanewline
\ \ \ \ \isacommand{using}\isamarkupfalse%
\ b\ d\ \isacommand{by}\isamarkupfalse%
\ blast\isanewline
\ \ \isacommand{with}\isamarkupfalse%
\ that\ \isacommand{show}\isamarkupfalse%
\ {\isacharquery}{\kern0pt}thesis\ \isacommand{by}\isamarkupfalse%
\ blast\isanewline
\isacommand{qed}\isamarkupfalse%
%
\endisatagproof
{\isafoldproof}%
%
\isadelimproof
\isanewline
%
\endisadelimproof
%
\isadelimtheory
\isanewline
%
\endisadelimtheory
%
\isatagtheory
\isacommand{end}\isamarkupfalse%
%
\endisatagtheory
{\isafoldtheory}%
%
\isadelimtheory
%
\endisadelimtheory
%
\end{isabellebody}%
\endinput
%:%file=Partitions.tex%:%
%:%11=1%:%
%:%27=3%:%
%:%28=3%:%
%:%29=4%:%
%:%30=5%:%
%:%39=7%:%
%:%40=8%:%
%:%41=9%:%
%:%42=10%:%
%:%44=12%:%
%:%45=12%:%
%:%46=13%:%
%:%47=14%:%
%:%48=15%:%
%:%50=17%:%
%:%51=18%:%
%:%52=19%:%
%:%53=19%:%
%:%54=20%:%
%:%55=21%:%
%:%56=21%:%
%:%57=22%:%
%:%58=23%:%
%:%59=24%:%
%:%60=24%:%
%:%61=25%:%
%:%62=26%:%
%:%69=27%:%
%:%70=27%:%
%:%71=28%:%
%:%72=28%:%
%:%73=29%:%
%:%74=29%:%
%:%75=29%:%
%:%76=29%:%
%:%77=30%:%
%:%78=30%:%
%:%79=31%:%
%:%80=31%:%
%:%81=32%:%
%:%82=32%:%
%:%83=32%:%
%:%84=33%:%
%:%85=33%:%
%:%86=34%:%
%:%87=34%:%
%:%88=35%:%
%:%89=35%:%
%:%90=36%:%
%:%96=36%:%
%:%99=37%:%
%:%100=38%:%
%:%101=38%:%
%:%102=39%:%
%:%104=41%:%
%:%111=42%:%
%:%112=42%:%
%:%113=43%:%
%:%114=43%:%
%:%115=44%:%
%:%116=44%:%
%:%117=45%:%
%:%118=46%:%
%:%119=46%:%
%:%120=47%:%
%:%121=47%:%
%:%122=48%:%
%:%123=48%:%
%:%124=49%:%
%:%125=49%:%
%:%126=49%:%
%:%127=50%:%
%:%128=50%:%
%:%129=50%:%
%:%130=51%:%
%:%131=51%:%
%:%132=51%:%
%:%133=52%:%
%:%134=52%:%
%:%135=52%:%
%:%136=52%:%
%:%137=52%:%
%:%138=53%:%
%:%139=53%:%
%:%140=54%:%
%:%141=54%:%
%:%142=55%:%
%:%143=55%:%
%:%144=55%:%
%:%145=55%:%
%:%146=56%:%
%:%147=56%:%
%:%148=57%:%
%:%149=57%:%
%:%150=58%:%
%:%151=58%:%
%:%152=58%:%
%:%153=59%:%
%:%154=59%:%
%:%155=60%:%
%:%156=60%:%
%:%157=61%:%
%:%158=61%:%
%:%159=62%:%
%:%160=62%:%
%:%161=63%:%
%:%162=64%:%
%:%163=64%:%
%:%164=65%:%
%:%165=65%:%
%:%166=66%:%
%:%167=66%:%
%:%168=67%:%
%:%169=67%:%
%:%170=67%:%
%:%171=68%:%
%:%172=68%:%
%:%173=68%:%
%:%174=68%:%
%:%175=69%:%
%:%176=69%:%
%:%177=70%:%
%:%178=70%:%
%:%179=71%:%
%:%180=71%:%
%:%181=72%:%
%:%182=72%:%
%:%183=73%:%
%:%184=73%:%
%:%185=74%:%
%:%186=74%:%
%:%187=75%:%
%:%188=75%:%
%:%189=75%:%
%:%190=76%:%
%:%191=76%:%
%:%192=76%:%
%:%193=77%:%
%:%199=77%:%
%:%202=78%:%
%:%203=79%:%
%:%204=79%:%
%:%205=80%:%
%:%208=81%:%
%:%212=81%:%
%:%213=81%:%
%:%218=81%:%
%:%221=82%:%
%:%222=83%:%
%:%223=83%:%
%:%224=84%:%
%:%227=85%:%
%:%231=85%:%
%:%232=85%:%
%:%237=85%:%
%:%240=86%:%
%:%241=87%:%
%:%242=87%:%
%:%245=88%:%
%:%249=88%:%
%:%250=88%:%
%:%255=88%:%
%:%258=89%:%
%:%259=90%:%
%:%260=90%:%
%:%261=91%:%
%:%268=92%:%
%:%269=92%:%
%:%270=93%:%
%:%271=93%:%
%:%272=94%:%
%:%273=94%:%
%:%274=94%:%
%:%275=94%:%
%:%276=95%:%
%:%277=95%:%
%:%278=96%:%
%:%279=96%:%
%:%280=97%:%
%:%281=97%:%
%:%282=98%:%
%:%283=98%:%
%:%284=98%:%
%:%285=99%:%
%:%286=99%:%
%:%287=99%:%
%:%288=100%:%
%:%289=101%:%
%:%290=102%:%
%:%291=103%:%
%:%292=103%:%
%:%293=103%:%
%:%294=104%:%
%:%295=104%:%
%:%296=105%:%
%:%297=105%:%
%:%298=105%:%
%:%299=106%:%
%:%300=106%:%
%:%301=107%:%
%:%302=107%:%
%:%303=108%:%
%:%304=108%:%
%:%305=109%:%
%:%306=109%:%
%:%307=110%:%
%:%308=110%:%
%:%309=111%:%
%:%310=111%:%
%:%311=112%:%
%:%312=112%:%
%:%313=113%:%
%:%314=113%:%
%:%315=113%:%
%:%316=113%:%
%:%317=114%:%
%:%318=114%:%
%:%319=115%:%
%:%320=115%:%
%:%321=116%:%
%:%322=116%:%
%:%323=117%:%
%:%324=117%:%
%:%325=118%:%
%:%326=118%:%
%:%327=119%:%
%:%328=119%:%
%:%329=120%:%
%:%330=120%:%
%:%331=120%:%
%:%332=121%:%
%:%333=121%:%
%:%334=122%:%
%:%335=122%:%
%:%336=123%:%
%:%337=123%:%
%:%338=124%:%
%:%339=124%:%
%:%340=125%:%
%:%341=125%:%
%:%342=126%:%
%:%343=126%:%
%:%344=127%:%
%:%345=127%:%
%:%346=128%:%
%:%347=128%:%
%:%348=129%:%
%:%349=129%:%
%:%350=130%:%
%:%351=130%:%
%:%352=131%:%
%:%353=131%:%
%:%354=131%:%
%:%355=132%:%
%:%356=132%:%
%:%357=133%:%
%:%358=133%:%
%:%359=134%:%
%:%360=134%:%
%:%361=134%:%
%:%362=135%:%
%:%363=135%:%
%:%364=136%:%
%:%365=136%:%
%:%366=137%:%
%:%367=137%:%
%:%368=138%:%
%:%374=138%:%
%:%377=139%:%
%:%378=140%:%
%:%379=140%:%
%:%380=141%:%
%:%381=142%:%
%:%382=143%:%
%:%383=144%:%
%:%384=145%:%
%:%385=145%:%
%:%386=146%:%
%:%389=147%:%
%:%393=147%:%
%:%394=147%:%
%:%395=148%:%
%:%396=148%:%
%:%397=148%:%
%:%402=148%:%
%:%405=149%:%
%:%406=150%:%
%:%407=150%:%
%:%408=151%:%
%:%411=152%:%
%:%415=152%:%
%:%416=152%:%
%:%417=153%:%
%:%418=153%:%
%:%419=154%:%
%:%420=154%:%
%:%421=155%:%
%:%422=155%:%
%:%423=155%:%
%:%428=155%:%
%:%431=156%:%
%:%432=157%:%
%:%433=157%:%
%:%434=158%:%
%:%437=159%:%
%:%441=159%:%
%:%442=159%:%
%:%443=160%:%
%:%444=160%:%
%:%449=160%:%
%:%452=161%:%
%:%453=162%:%
%:%454=162%:%
%:%457=163%:%
%:%461=163%:%
%:%462=163%:%
%:%467=163%:%
%:%470=164%:%
%:%471=165%:%
%:%472=165%:%
%:%475=166%:%
%:%479=166%:%
%:%480=166%:%
%:%485=166%:%
%:%488=167%:%
%:%489=168%:%
%:%490=168%:%
%:%491=169%:%
%:%492=170%:%
%:%495=171%:%
%:%499=171%:%
%:%500=171%:%
%:%501=172%:%
%:%502=172%:%
%:%503=173%:%
%:%504=173%:%
%:%505=174%:%
%:%506=174%:%
%:%507=175%:%
%:%508=175%:%
%:%513=175%:%
%:%516=176%:%
%:%517=177%:%
%:%518=177%:%
%:%519=178%:%
%:%520=179%:%
%:%527=180%:%
%:%528=180%:%
%:%529=181%:%
%:%530=181%:%
%:%531=182%:%
%:%532=182%:%
%:%533=183%:%
%:%534=184%:%
%:%535=184%:%
%:%536=184%:%
%:%537=185%:%
%:%538=185%:%
%:%539=185%:%
%:%540=186%:%
%:%541=187%:%
%:%542=187%:%
%:%543=188%:%
%:%544=188%:%
%:%545=189%:%
%:%546=189%:%
%:%547=190%:%
%:%548=190%:%
%:%549=191%:%
%:%550=191%:%
%:%551=191%:%
%:%552=191%:%
%:%553=192%:%
%:%554=192%:%
%:%555=193%:%
%:%556=193%:%
%:%557=194%:%
%:%558=194%:%
%:%559=195%:%
%:%560=195%:%
%:%561=196%:%
%:%562=196%:%
%:%563=197%:%
%:%564=197%:%
%:%565=198%:%
%:%566=198%:%
%:%567=199%:%
%:%568=199%:%
%:%569=200%:%
%:%570=200%:%
%:%571=201%:%
%:%572=201%:%
%:%573=202%:%
%:%574=202%:%
%:%575=203%:%
%:%576=203%:%
%:%577=204%:%
%:%578=204%:%
%:%579=205%:%
%:%580=205%:%
%:%581=206%:%
%:%582=207%:%
%:%583=207%:%
%:%584=208%:%
%:%585=208%:%
%:%586=209%:%
%:%587=209%:%
%:%588=210%:%
%:%589=210%:%
%:%590=211%:%
%:%591=211%:%
%:%592=212%:%
%:%593=212%:%
%:%594=213%:%
%:%595=213%:%
%:%596=213%:%
%:%597=214%:%
%:%598=214%:%
%:%599=215%:%
%:%600=216%:%
%:%601=216%:%
%:%602=217%:%
%:%603=217%:%
%:%604=218%:%
%:%605=218%:%
%:%606=219%:%
%:%607=219%:%
%:%608=220%:%
%:%609=220%:%
%:%610=221%:%
%:%611=221%:%
%:%612=222%:%
%:%613=222%:%
%:%614=223%:%
%:%615=223%:%
%:%616=224%:%
%:%617=224%:%
%:%618=225%:%
%:%619=225%:%
%:%620=226%:%
%:%621=226%:%
%:%622=227%:%
%:%623=228%:%
%:%624=228%:%
%:%625=229%:%
%:%626=229%:%
%:%627=230%:%
%:%628=230%:%
%:%629=230%:%
%:%630=231%:%
%:%631=231%:%
%:%632=232%:%
%:%633=232%:%
%:%634=233%:%
%:%635=233%:%
%:%636=234%:%
%:%637=234%:%
%:%638=234%:%
%:%639=235%:%
%:%640=235%:%
%:%641=236%:%
%:%642=236%:%
%:%643=236%:%
%:%644=237%:%
%:%645=237%:%
%:%646=238%:%
%:%647=238%:%
%:%648=238%:%
%:%649=239%:%
%:%650=239%:%
%:%651=240%:%
%:%652=240%:%
%:%653=241%:%
%:%654=242%:%
%:%655=242%:%
%:%656=243%:%
%:%657=243%:%
%:%658=244%:%
%:%659=244%:%
%:%660=245%:%
%:%661=246%:%
%:%662=246%:%
%:%663=246%:%
%:%664=246%:%
%:%665=247%:%
%:%666=247%:%
%:%667=247%:%
%:%668=248%:%
%:%669=248%:%
%:%670=249%:%
%:%671=249%:%
%:%672=249%:%
%:%673=250%:%
%:%674=250%:%
%:%675=251%:%
%:%676=251%:%
%:%677=251%:%
%:%678=252%:%
%:%679=252%:%
%:%680=253%:%
%:%681=253%:%
%:%682=253%:%
%:%683=254%:%
%:%684=254%:%
%:%685=255%:%
%:%686=256%:%
%:%687=256%:%
%:%688=257%:%
%:%689=257%:%
%:%690=258%:%
%:%691=258%:%
%:%692=259%:%
%:%693=259%:%
%:%694=259%:%
%:%695=260%:%
%:%696=260%:%
%:%697=261%:%
%:%698=261%:%
%:%699=261%:%
%:%700=262%:%
%:%701=262%:%
%:%702=263%:%
%:%703=263%:%
%:%704=263%:%
%:%705=263%:%
%:%706=264%:%
%:%707=265%:%
%:%708=265%:%
%:%709=266%:%
%:%710=266%:%
%:%711=266%:%
%:%712=267%:%
%:%713=268%:%
%:%714=268%:%
%:%715=269%:%
%:%716=269%:%
%:%717=269%:%
%:%718=270%:%
%:%719=270%:%
%:%720=271%:%
%:%721=271%:%
%:%722=272%:%
%:%723=272%:%
%:%724=273%:%
%:%725=273%:%
%:%726=274%:%
%:%727=275%:%
%:%728=275%:%
%:%729=276%:%
%:%730=276%:%
%:%731=276%:%
%:%732=277%:%
%:%733=278%:%
%:%734=278%:%
%:%735=279%:%
%:%736=279%:%
%:%737=280%:%
%:%738=280%:%
%:%739=280%:%
%:%740=281%:%
%:%741=281%:%
%:%742=282%:%
%:%743=282%:%
%:%744=282%:%
%:%745=283%:%
%:%746=283%:%
%:%747=283%:%
%:%748=283%:%
%:%749=284%:%
%:%755=284%:%
%:%760=285%:%
%:%765=286%:%

%
\begin{isabellebody}%
\setisabellecontext{Frequency{\isacharunderscore}{\kern0pt}Moment{\isacharunderscore}{\kern0pt}{\isadigit{2}}}%
%
\isadelimdocument
%
\endisadelimdocument
%
\isatagdocument
%
\isamarkupsection{Frequency Moment $2$%
}
\isamarkuptrue%
%
\endisatagdocument
{\isafolddocument}%
%
\isadelimdocument
%
\endisadelimdocument
%
\isadelimtheory
%
\endisadelimtheory
%
\isatagtheory
\isacommand{theory}\isamarkupfalse%
\ Frequency{\isacharunderscore}{\kern0pt}Moment{\isacharunderscore}{\kern0pt}{\isadigit{2}}\isanewline
\ \ \isakeyword{imports}\ Main\ Median\ Partitions\ Primes{\isacharunderscore}{\kern0pt}Ext\ Encoding\ List{\isacharunderscore}{\kern0pt}Ext\ \isanewline
\ \ \ \ UniversalHashFamilyOfPrime\ Frequency{\isacharunderscore}{\kern0pt}Moments\ Landau{\isacharunderscore}{\kern0pt}Ext\isanewline
\isakeyword{begin}%
\endisatagtheory
{\isafoldtheory}%
%
\isadelimtheory
%
\endisadelimtheory
%
\begin{isamarkuptext}%
This section contains a formalization of the algorithm for the second frequency moment.
It is based on the algorithm described in \cite[\textsection 2.2]{alon1999}.
The only difference is that the algorithm is adapted to work with prime field of odd order, which
greatly reduces the implementation complexity.%
\end{isamarkuptext}\isamarkuptrue%
\isacommand{fun}\isamarkupfalse%
\ f{\isadigit{2}}{\isacharunderscore}{\kern0pt}hash\ \isakeyword{where}\isanewline
\ \ {\isachardoublequoteopen}f{\isadigit{2}}{\isacharunderscore}{\kern0pt}hash\ p\ h\ k\ {\isacharequal}{\kern0pt}\ {\isacharparenleft}{\kern0pt}if\ hash\ p\ k\ h\ {\isasymin}\ {\isacharbraceleft}{\kern0pt}k{\isachardot}{\kern0pt}\ {\isadigit{2}}{\isacharasterisk}{\kern0pt}k\ {\isacharless}{\kern0pt}\ p{\isacharbraceright}{\kern0pt}\ then\ int\ p\ {\isacharminus}{\kern0pt}\ {\isadigit{1}}\ else\ {\isacharminus}{\kern0pt}\ int\ p\ {\isacharminus}{\kern0pt}\ {\isadigit{1}}{\isacharparenright}{\kern0pt}{\isachardoublequoteclose}\isanewline
\isanewline
\isacommand{type{\isacharunderscore}{\kern0pt}synonym}\isamarkupfalse%
\ f{\isadigit{2}}{\isacharunderscore}{\kern0pt}state\ {\isacharequal}{\kern0pt}\ {\isachardoublequoteopen}nat\ {\isasymtimes}\ nat\ {\isasymtimes}\ nat\ {\isasymtimes}\ {\isacharparenleft}{\kern0pt}nat\ {\isasymtimes}\ nat\ {\isasymRightarrow}\ int\ set\ list{\isacharparenright}{\kern0pt}\ {\isasymtimes}\ {\isacharparenleft}{\kern0pt}nat\ {\isasymtimes}\ nat\ {\isasymRightarrow}\ int{\isacharparenright}{\kern0pt}{\isachardoublequoteclose}\isanewline
\isanewline
\isacommand{fun}\isamarkupfalse%
\ f{\isadigit{2}}{\isacharunderscore}{\kern0pt}init\ {\isacharcolon}{\kern0pt}{\isacharcolon}{\kern0pt}\ {\isachardoublequoteopen}rat\ {\isasymRightarrow}\ rat\ {\isasymRightarrow}\ nat\ {\isasymRightarrow}\ f{\isadigit{2}}{\isacharunderscore}{\kern0pt}state\ pmf{\isachardoublequoteclose}\ \isakeyword{where}\isanewline
\ \ {\isachardoublequoteopen}f{\isadigit{2}}{\isacharunderscore}{\kern0pt}init\ {\isasymdelta}\ {\isasymepsilon}\ n\ {\isacharequal}{\kern0pt}\isanewline
\ \ \ \ do\ {\isacharbraceleft}{\kern0pt}\isanewline
\ \ \ \ \ \ let\ s\isactrlsub {\isadigit{1}}\ {\isacharequal}{\kern0pt}\ nat\ {\isasymlceil}{\isadigit{6}}\ {\isacharslash}{\kern0pt}\ {\isasymdelta}\isactrlsup {\isadigit{2}}{\isasymrceil}{\isacharsemicolon}{\kern0pt}\isanewline
\ \ \ \ \ \ let\ s\isactrlsub {\isadigit{2}}\ {\isacharequal}{\kern0pt}\ nat\ {\isasymlceil}{\isacharminus}{\kern0pt}{\isacharparenleft}{\kern0pt}{\isadigit{1}}{\isadigit{8}}\ {\isacharasterisk}{\kern0pt}\ ln\ {\isacharparenleft}{\kern0pt}real{\isacharunderscore}{\kern0pt}of{\isacharunderscore}{\kern0pt}rat\ {\isasymepsilon}{\isacharparenright}{\kern0pt}{\isacharparenright}{\kern0pt}{\isasymrceil}{\isacharsemicolon}{\kern0pt}\isanewline
\ \ \ \ \ \ let\ p\ {\isacharequal}{\kern0pt}\ find{\isacharunderscore}{\kern0pt}prime{\isacharunderscore}{\kern0pt}above\ {\isacharparenleft}{\kern0pt}max\ n\ {\isadigit{3}}{\isacharparenright}{\kern0pt}{\isacharsemicolon}{\kern0pt}\isanewline
\ \ \ \ \ \ h\ {\isasymleftarrow}\ prod{\isacharunderscore}{\kern0pt}pmf\ {\isacharparenleft}{\kern0pt}{\isacharbraceleft}{\kern0pt}{\isadigit{0}}{\isachardot}{\kern0pt}{\isachardot}{\kern0pt}{\isacharless}{\kern0pt}s\isactrlsub {\isadigit{1}}{\isacharbraceright}{\kern0pt}\ {\isasymtimes}\ {\isacharbraceleft}{\kern0pt}{\isadigit{0}}{\isachardot}{\kern0pt}{\isachardot}{\kern0pt}{\isacharless}{\kern0pt}s\isactrlsub {\isadigit{2}}{\isacharbraceright}{\kern0pt}{\isacharparenright}{\kern0pt}\ {\isacharparenleft}{\kern0pt}{\isasymlambda}{\isacharunderscore}{\kern0pt}{\isachardot}{\kern0pt}\ pmf{\isacharunderscore}{\kern0pt}of{\isacharunderscore}{\kern0pt}set\ {\isacharparenleft}{\kern0pt}bounded{\isacharunderscore}{\kern0pt}degree{\isacharunderscore}{\kern0pt}polynomials\ {\isacharparenleft}{\kern0pt}ZFact\ {\isacharparenleft}{\kern0pt}int\ p{\isacharparenright}{\kern0pt}{\isacharparenright}{\kern0pt}\ {\isadigit{4}}{\isacharparenright}{\kern0pt}{\isacharparenright}{\kern0pt}{\isacharsemicolon}{\kern0pt}\isanewline
\ \ \ \ \ \ return{\isacharunderscore}{\kern0pt}pmf\ {\isacharparenleft}{\kern0pt}s\isactrlsub {\isadigit{1}}{\isacharcomma}{\kern0pt}\ s\isactrlsub {\isadigit{2}}{\isacharcomma}{\kern0pt}\ p{\isacharcomma}{\kern0pt}\ h{\isacharcomma}{\kern0pt}\ {\isacharparenleft}{\kern0pt}{\isasymlambda}{\isacharunderscore}{\kern0pt}\ {\isasymin}\ {\isacharbraceleft}{\kern0pt}{\isadigit{0}}{\isachardot}{\kern0pt}{\isachardot}{\kern0pt}{\isacharless}{\kern0pt}s\isactrlsub {\isadigit{1}}{\isacharbraceright}{\kern0pt}\ {\isasymtimes}\ {\isacharbraceleft}{\kern0pt}{\isadigit{0}}{\isachardot}{\kern0pt}{\isachardot}{\kern0pt}{\isacharless}{\kern0pt}s\isactrlsub {\isadigit{2}}{\isacharbraceright}{\kern0pt}{\isachardot}{\kern0pt}\ {\isacharparenleft}{\kern0pt}{\isadigit{0}}\ {\isacharcolon}{\kern0pt}{\isacharcolon}{\kern0pt}\ int{\isacharparenright}{\kern0pt}{\isacharparenright}{\kern0pt}{\isacharparenright}{\kern0pt}\isanewline
\ \ \ \ {\isacharbraceright}{\kern0pt}{\isachardoublequoteclose}\isanewline
\isanewline
\isacommand{fun}\isamarkupfalse%
\ f{\isadigit{2}}{\isacharunderscore}{\kern0pt}update\ {\isacharcolon}{\kern0pt}{\isacharcolon}{\kern0pt}\ {\isachardoublequoteopen}nat\ {\isasymRightarrow}\ f{\isadigit{2}}{\isacharunderscore}{\kern0pt}state\ {\isasymRightarrow}\ f{\isadigit{2}}{\isacharunderscore}{\kern0pt}state\ pmf{\isachardoublequoteclose}\ \isakeyword{where}\isanewline
\ \ {\isachardoublequoteopen}f{\isadigit{2}}{\isacharunderscore}{\kern0pt}update\ x\ {\isacharparenleft}{\kern0pt}s\isactrlsub {\isadigit{1}}{\isacharcomma}{\kern0pt}\ s\isactrlsub {\isadigit{2}}{\isacharcomma}{\kern0pt}\ p{\isacharcomma}{\kern0pt}\ h{\isacharcomma}{\kern0pt}\ sketch{\isacharparenright}{\kern0pt}\ {\isacharequal}{\kern0pt}\ \isanewline
\ \ \ \ return{\isacharunderscore}{\kern0pt}pmf\ {\isacharparenleft}{\kern0pt}s\isactrlsub {\isadigit{1}}{\isacharcomma}{\kern0pt}\ s\isactrlsub {\isadigit{2}}{\isacharcomma}{\kern0pt}\ p{\isacharcomma}{\kern0pt}\ h{\isacharcomma}{\kern0pt}\ {\isasymlambda}i\ {\isasymin}\ {\isacharbraceleft}{\kern0pt}{\isadigit{0}}{\isachardot}{\kern0pt}{\isachardot}{\kern0pt}{\isacharless}{\kern0pt}s\isactrlsub {\isadigit{1}}{\isacharbraceright}{\kern0pt}\ {\isasymtimes}\ {\isacharbraceleft}{\kern0pt}{\isadigit{0}}{\isachardot}{\kern0pt}{\isachardot}{\kern0pt}{\isacharless}{\kern0pt}s\isactrlsub {\isadigit{2}}{\isacharbraceright}{\kern0pt}{\isachardot}{\kern0pt}\ f{\isadigit{2}}{\isacharunderscore}{\kern0pt}hash\ p\ {\isacharparenleft}{\kern0pt}h\ i{\isacharparenright}{\kern0pt}\ x\ {\isacharplus}{\kern0pt}\ sketch\ i{\isacharparenright}{\kern0pt}{\isachardoublequoteclose}\isanewline
\isanewline
\isacommand{fun}\isamarkupfalse%
\ f{\isadigit{2}}{\isacharunderscore}{\kern0pt}result\ {\isacharcolon}{\kern0pt}{\isacharcolon}{\kern0pt}\ {\isachardoublequoteopen}f{\isadigit{2}}{\isacharunderscore}{\kern0pt}state\ {\isasymRightarrow}\ rat\ pmf{\isachardoublequoteclose}\ \isakeyword{where}\isanewline
\ \ {\isachardoublequoteopen}f{\isadigit{2}}{\isacharunderscore}{\kern0pt}result\ {\isacharparenleft}{\kern0pt}s\isactrlsub {\isadigit{1}}{\isacharcomma}{\kern0pt}\ s\isactrlsub {\isadigit{2}}{\isacharcomma}{\kern0pt}\ p{\isacharcomma}{\kern0pt}\ h{\isacharcomma}{\kern0pt}\ sketch{\isacharparenright}{\kern0pt}\ {\isacharequal}{\kern0pt}\ \isanewline
\ \ \ \ return{\isacharunderscore}{\kern0pt}pmf\ {\isacharparenleft}{\kern0pt}median\ {\isacharparenleft}{\kern0pt}{\isasymlambda}i\isactrlsub {\isadigit{2}}\ {\isasymin}\ {\isacharbraceleft}{\kern0pt}{\isadigit{0}}{\isachardot}{\kern0pt}{\isachardot}{\kern0pt}{\isacharless}{\kern0pt}s\isactrlsub {\isadigit{2}}{\isacharbraceright}{\kern0pt}{\isachardot}{\kern0pt}\ \isanewline
\ \ \ \ \ \ {\isacharparenleft}{\kern0pt}{\isasymSum}i\isactrlsub {\isadigit{1}}{\isasymin}{\isacharbraceleft}{\kern0pt}{\isadigit{0}}{\isachardot}{\kern0pt}{\isachardot}{\kern0pt}{\isacharless}{\kern0pt}s\isactrlsub {\isadigit{1}}{\isacharbraceright}{\kern0pt}\ {\isachardot}{\kern0pt}\ {\isacharparenleft}{\kern0pt}rat{\isacharunderscore}{\kern0pt}of{\isacharunderscore}{\kern0pt}int\ {\isacharparenleft}{\kern0pt}sketch\ {\isacharparenleft}{\kern0pt}i\isactrlsub {\isadigit{1}}{\isacharcomma}{\kern0pt}\ i\isactrlsub {\isadigit{2}}{\isacharparenright}{\kern0pt}{\isacharparenright}{\kern0pt}{\isacharparenright}{\kern0pt}\isactrlsup {\isadigit{2}}{\isacharparenright}{\kern0pt}\ {\isacharslash}{\kern0pt}\ {\isacharparenleft}{\kern0pt}{\isacharparenleft}{\kern0pt}{\isacharparenleft}{\kern0pt}rat{\isacharunderscore}{\kern0pt}of{\isacharunderscore}{\kern0pt}nat\ p{\isacharparenright}{\kern0pt}\isactrlsup {\isadigit{2}}{\isacharminus}{\kern0pt}{\isadigit{1}}{\isacharparenright}{\kern0pt}\ {\isacharasterisk}{\kern0pt}\ rat{\isacharunderscore}{\kern0pt}of{\isacharunderscore}{\kern0pt}nat\ s\isactrlsub {\isadigit{1}}{\isacharparenright}{\kern0pt}{\isacharparenright}{\kern0pt}\ s\isactrlsub {\isadigit{2}}\isanewline
\ \ \ \ {\isacharparenright}{\kern0pt}{\isachardoublequoteclose}\isanewline
\isanewline
\isacommand{lemma}\isamarkupfalse%
\ f{\isadigit{2}}{\isacharunderscore}{\kern0pt}hash{\isacharunderscore}{\kern0pt}exp{\isacharcolon}{\kern0pt}\isanewline
\ \ \isakeyword{assumes}\ {\isachardoublequoteopen}Factorial{\isacharunderscore}{\kern0pt}Ring{\isachardot}{\kern0pt}prime\ p{\isachardoublequoteclose}\isanewline
\ \ \isakeyword{assumes}\ {\isachardoublequoteopen}k\ {\isacharless}{\kern0pt}\ p{\isachardoublequoteclose}\isanewline
\ \ \isakeyword{assumes}\ {\isachardoublequoteopen}p\ {\isachargreater}{\kern0pt}\ {\isadigit{2}}{\isachardoublequoteclose}\ \isanewline
\ \ \isakeyword{shows}\isanewline
\ \ \ \ {\isachardoublequoteopen}prob{\isacharunderscore}{\kern0pt}space{\isachardot}{\kern0pt}expectation\ {\isacharparenleft}{\kern0pt}pmf{\isacharunderscore}{\kern0pt}of{\isacharunderscore}{\kern0pt}set\ {\isacharparenleft}{\kern0pt}bounded{\isacharunderscore}{\kern0pt}degree{\isacharunderscore}{\kern0pt}polynomials\ {\isacharparenleft}{\kern0pt}ZFact\ {\isacharparenleft}{\kern0pt}int\ p{\isacharparenright}{\kern0pt}{\isacharparenright}{\kern0pt}\ {\isadigit{4}}{\isacharparenright}{\kern0pt}{\isacharparenright}{\kern0pt}\ \isanewline
\ \ \ \ {\isacharparenleft}{\kern0pt}{\isasymlambda}{\isasymomega}{\isachardot}{\kern0pt}\ real{\isacharunderscore}{\kern0pt}of{\isacharunderscore}{\kern0pt}int\ {\isacharparenleft}{\kern0pt}f{\isadigit{2}}{\isacharunderscore}{\kern0pt}hash\ p\ {\isasymomega}\ k{\isacharparenright}{\kern0pt}\ {\isacharcircum}{\kern0pt}m{\isacharparenright}{\kern0pt}\ {\isacharequal}{\kern0pt}\ \isanewline
\ \ \ \ \ {\isacharparenleft}{\kern0pt}{\isacharparenleft}{\kern0pt}{\isacharparenleft}{\kern0pt}real\ p\ {\isacharminus}{\kern0pt}\ {\isadigit{1}}{\isacharparenright}{\kern0pt}\ {\isacharcircum}{\kern0pt}\ m\ {\isacharasterisk}{\kern0pt}\ {\isacharparenleft}{\kern0pt}real\ p\ {\isacharplus}{\kern0pt}\ {\isadigit{1}}{\isacharparenright}{\kern0pt}\ {\isacharplus}{\kern0pt}\ {\isacharparenleft}{\kern0pt}{\isacharminus}{\kern0pt}\ real\ p\ {\isacharminus}{\kern0pt}\ {\isadigit{1}}{\isacharparenright}{\kern0pt}\ {\isacharcircum}{\kern0pt}\ m\ {\isacharasterisk}{\kern0pt}\ {\isacharparenleft}{\kern0pt}real\ p\ {\isacharminus}{\kern0pt}\ {\isadigit{1}}{\isacharparenright}{\kern0pt}{\isacharparenright}{\kern0pt}\ {\isacharslash}{\kern0pt}\ {\isacharparenleft}{\kern0pt}{\isadigit{2}}\ {\isacharasterisk}{\kern0pt}\ real\ p{\isacharparenright}{\kern0pt}{\isacharparenright}{\kern0pt}{\isachardoublequoteclose}\isanewline
%
\isadelimproof
%
\endisadelimproof
%
\isatagproof
\isacommand{proof}\isamarkupfalse%
\ {\isacharminus}{\kern0pt}\isanewline
\ \ \isacommand{have}\isamarkupfalse%
\ g{\isacharcolon}{\kern0pt}{\isachardoublequoteopen}p\ {\isachargreater}{\kern0pt}\ {\isadigit{0}}{\isachardoublequoteclose}\ \isacommand{using}\isamarkupfalse%
\ assms{\isacharparenleft}{\kern0pt}{\isadigit{1}}{\isacharparenright}{\kern0pt}\ prime{\isacharunderscore}{\kern0pt}gt{\isacharunderscore}{\kern0pt}{\isadigit{0}}{\isacharunderscore}{\kern0pt}nat\ \isacommand{by}\isamarkupfalse%
\ auto\isanewline
\isanewline
\ \ \isacommand{have}\isamarkupfalse%
\ {\isachardoublequoteopen}odd\ p{\isachardoublequoteclose}\ \isacommand{using}\isamarkupfalse%
\ assms\ prime{\isacharunderscore}{\kern0pt}odd{\isacharunderscore}{\kern0pt}nat\ \isacommand{by}\isamarkupfalse%
\ blast\isanewline
\ \ \isacommand{then}\isamarkupfalse%
\ \isacommand{obtain}\isamarkupfalse%
\ t\ \isakeyword{where}\ t{\isacharunderscore}{\kern0pt}def{\isacharcolon}{\kern0pt}\ {\isachardoublequoteopen}p{\isacharequal}{\kern0pt}{\isadigit{2}}{\isacharasterisk}{\kern0pt}t{\isacharplus}{\kern0pt}{\isadigit{1}}{\isachardoublequoteclose}\ \isanewline
\ \ \ \ \isacommand{using}\isamarkupfalse%
\ oddE\ \isacommand{by}\isamarkupfalse%
\ blast\isanewline
\isanewline
\ \ \isacommand{define}\isamarkupfalse%
\ {\isasymOmega}\ \isakeyword{where}\ {\isachardoublequoteopen}{\isasymOmega}\ {\isacharequal}{\kern0pt}\ pmf{\isacharunderscore}{\kern0pt}of{\isacharunderscore}{\kern0pt}set\ {\isacharparenleft}{\kern0pt}bounded{\isacharunderscore}{\kern0pt}degree{\isacharunderscore}{\kern0pt}polynomials\ {\isacharparenleft}{\kern0pt}ZFact\ {\isacharparenleft}{\kern0pt}int\ p{\isacharparenright}{\kern0pt}{\isacharparenright}{\kern0pt}\ {\isadigit{4}}{\isacharparenright}{\kern0pt}{\isachardoublequoteclose}\isanewline
\isanewline
\ \ \isacommand{have}\isamarkupfalse%
\ b{\isacharcolon}{\kern0pt}\ {\isachardoublequoteopen}finite\ {\isacharparenleft}{\kern0pt}set{\isacharunderscore}{\kern0pt}pmf\ {\isasymOmega}{\isacharparenright}{\kern0pt}{\isachardoublequoteclose}\isanewline
\ \ \ \ \isacommand{apply}\isamarkupfalse%
\ {\isacharparenleft}{\kern0pt}simp\ add{\isacharcolon}{\kern0pt}{\isasymOmega}{\isacharunderscore}{\kern0pt}def{\isacharparenright}{\kern0pt}\isanewline
\ \ \ \ \isacommand{by}\isamarkupfalse%
\ {\isacharparenleft}{\kern0pt}metis\ fin{\isacharunderscore}{\kern0pt}bounded{\isacharunderscore}{\kern0pt}degree{\isacharunderscore}{\kern0pt}polynomials{\isacharbrackleft}{\kern0pt}OF\ g{\isacharbrackright}{\kern0pt}\ ne{\isacharunderscore}{\kern0pt}bounded{\isacharunderscore}{\kern0pt}degree{\isacharunderscore}{\kern0pt}polynomials\ set{\isacharunderscore}{\kern0pt}pmf{\isacharunderscore}{\kern0pt}of{\isacharunderscore}{\kern0pt}set{\isacharparenright}{\kern0pt}\isanewline
\isanewline
\ \ \isacommand{have}\isamarkupfalse%
\ zero{\isacharunderscore}{\kern0pt}le{\isacharunderscore}{\kern0pt}{\isadigit{4}}{\isacharcolon}{\kern0pt}\ {\isachardoublequoteopen}{\isadigit{0}}\ {\isacharless}{\kern0pt}\ {\isacharparenleft}{\kern0pt}{\isadigit{4}}{\isacharcolon}{\kern0pt}{\isacharcolon}{\kern0pt}nat{\isacharparenright}{\kern0pt}{\isachardoublequoteclose}\ \isacommand{by}\isamarkupfalse%
\ simp\isanewline
\isanewline
\ \ \isacommand{have}\isamarkupfalse%
\ {\isachardoublequoteopen}card\ {\isacharparenleft}{\kern0pt}{\isacharbraceleft}{\kern0pt}k{\isachardot}{\kern0pt}\ {\isadigit{2}}\ {\isacharasterisk}{\kern0pt}\ k\ {\isacharless}{\kern0pt}\ p{\isacharbraceright}{\kern0pt}\ {\isasyminter}\ {\isacharbraceleft}{\kern0pt}{\isadigit{0}}{\isachardot}{\kern0pt}{\isachardot}{\kern0pt}{\isacharless}{\kern0pt}p{\isacharbraceright}{\kern0pt}{\isacharparenright}{\kern0pt}\ {\isacharequal}{\kern0pt}\ card\ {\isacharparenleft}{\kern0pt}{\isacharbraceleft}{\kern0pt}{\isadigit{0}}{\isachardot}{\kern0pt}{\isachardot}{\kern0pt}t{\isacharbraceright}{\kern0pt}{\isacharparenright}{\kern0pt}{\isachardoublequoteclose}\isanewline
\ \ \ \ \isacommand{apply}\isamarkupfalse%
\ {\isacharparenleft}{\kern0pt}subst\ Int{\isacharunderscore}{\kern0pt}absorb{\isadigit{2}}{\isacharcomma}{\kern0pt}\ rule\ subsetI{\isacharcomma}{\kern0pt}\ simp{\isacharparenright}{\kern0pt}\isanewline
\ \ \ \ \isacommand{apply}\isamarkupfalse%
\ {\isacharparenleft}{\kern0pt}rule\ arg{\isacharunderscore}{\kern0pt}cong{\isacharbrackleft}{\kern0pt}\isakeyword{where}\ f{\isacharequal}{\kern0pt}{\isachardoublequoteopen}card{\isachardoublequoteclose}{\isacharbrackright}{\kern0pt}{\isacharparenright}{\kern0pt}\isanewline
\ \ \ \ \isacommand{apply}\isamarkupfalse%
\ {\isacharparenleft}{\kern0pt}rule\ order{\isacharunderscore}{\kern0pt}antisym{\isacharcomma}{\kern0pt}\ rule\ subsetI{\isacharcomma}{\kern0pt}\ simp\ add{\isacharcolon}{\kern0pt}t{\isacharunderscore}{\kern0pt}def{\isacharparenright}{\kern0pt}\ \isanewline
\ \ \ \ \isacommand{by}\isamarkupfalse%
\ {\isacharparenleft}{\kern0pt}rule\ subsetI{\isacharcomma}{\kern0pt}\ simp\ add{\isacharcolon}{\kern0pt}t{\isacharunderscore}{\kern0pt}def{\isacharparenright}{\kern0pt}\isanewline
\ \ \isacommand{also}\isamarkupfalse%
\ \isacommand{have}\isamarkupfalse%
\ {\isachardoublequoteopen}{\isachardot}{\kern0pt}{\isachardot}{\kern0pt}{\isachardot}{\kern0pt}\ {\isacharequal}{\kern0pt}\ t{\isacharplus}{\kern0pt}{\isadigit{1}}{\isachardoublequoteclose}\isanewline
\ \ \ \ \isacommand{by}\isamarkupfalse%
\ simp\isanewline
\ \ \isacommand{also}\isamarkupfalse%
\ \isacommand{have}\isamarkupfalse%
\ {\isachardoublequoteopen}{\isachardot}{\kern0pt}{\isachardot}{\kern0pt}{\isachardot}{\kern0pt}\ {\isacharequal}{\kern0pt}\ {\isacharparenleft}{\kern0pt}real\ p\ {\isacharplus}{\kern0pt}\ {\isadigit{1}}{\isacharparenright}{\kern0pt}{\isacharslash}{\kern0pt}{\isadigit{2}}{\isachardoublequoteclose}\isanewline
\ \ \ \ \isacommand{by}\isamarkupfalse%
\ {\isacharparenleft}{\kern0pt}simp\ add{\isacharcolon}{\kern0pt}t{\isacharunderscore}{\kern0pt}def{\isacharparenright}{\kern0pt}\isanewline
\ \ \isacommand{finally}\isamarkupfalse%
\ \isacommand{have}\isamarkupfalse%
\ c{\isacharunderscore}{\kern0pt}{\isadigit{1}}{\isacharcolon}{\kern0pt}\ {\isachardoublequoteopen}card\ {\isacharparenleft}{\kern0pt}{\isacharbraceleft}{\kern0pt}k{\isachardot}{\kern0pt}\ {\isadigit{2}}\ {\isacharasterisk}{\kern0pt}\ k\ {\isacharless}{\kern0pt}\ p{\isacharbraceright}{\kern0pt}\ {\isasyminter}\ {\isacharbraceleft}{\kern0pt}{\isadigit{0}}{\isachardot}{\kern0pt}{\isachardot}{\kern0pt}{\isacharless}{\kern0pt}p{\isacharbraceright}{\kern0pt}{\isacharparenright}{\kern0pt}\ {\isacharequal}{\kern0pt}\ {\isacharparenleft}{\kern0pt}real\ p{\isacharplus}{\kern0pt}{\isadigit{1}}{\isacharparenright}{\kern0pt}{\isacharslash}{\kern0pt}{\isadigit{2}}{\isachardoublequoteclose}\ \isacommand{by}\isamarkupfalse%
\ simp\isanewline
\isanewline
\ \ \isacommand{have}\isamarkupfalse%
\ {\isachardoublequoteopen}card\ {\isacharparenleft}{\kern0pt}{\isacharbraceleft}{\kern0pt}k{\isachardot}{\kern0pt}\ p\ {\isasymle}\ {\isadigit{2}}\ {\isacharasterisk}{\kern0pt}\ k{\isacharbraceright}{\kern0pt}\ {\isasyminter}\ {\isacharbraceleft}{\kern0pt}{\isadigit{0}}{\isachardot}{\kern0pt}{\isachardot}{\kern0pt}{\isacharless}{\kern0pt}p{\isacharbraceright}{\kern0pt}{\isacharparenright}{\kern0pt}\ {\isacharequal}{\kern0pt}\ card\ {\isacharbraceleft}{\kern0pt}t{\isacharplus}{\kern0pt}{\isadigit{1}}{\isachardot}{\kern0pt}{\isachardot}{\kern0pt}{\isacharless}{\kern0pt}p{\isacharbraceright}{\kern0pt}{\isachardoublequoteclose}\isanewline
\ \ \ \ \isacommand{apply}\isamarkupfalse%
\ {\isacharparenleft}{\kern0pt}rule\ arg{\isacharunderscore}{\kern0pt}cong{\isacharbrackleft}{\kern0pt}\isakeyword{where}\ f{\isacharequal}{\kern0pt}{\isachardoublequoteopen}card{\isachardoublequoteclose}{\isacharbrackright}{\kern0pt}{\isacharparenright}{\kern0pt}\isanewline
\ \ \ \ \isacommand{apply}\isamarkupfalse%
\ {\isacharparenleft}{\kern0pt}rule\ order{\isacharunderscore}{\kern0pt}antisym{\isacharcomma}{\kern0pt}\ rule\ subsetI{\isacharcomma}{\kern0pt}\ simp\ add{\isacharcolon}{\kern0pt}t{\isacharunderscore}{\kern0pt}def{\isacharparenright}{\kern0pt}\ \isanewline
\ \ \ \ \isacommand{by}\isamarkupfalse%
\ {\isacharparenleft}{\kern0pt}rule\ subsetI{\isacharcomma}{\kern0pt}\ simp\ add{\isacharcolon}{\kern0pt}t{\isacharunderscore}{\kern0pt}def{\isacharparenright}{\kern0pt}\isanewline
\ \ \isacommand{also}\isamarkupfalse%
\ \isacommand{have}\isamarkupfalse%
\ {\isachardoublequoteopen}{\isachardot}{\kern0pt}{\isachardot}{\kern0pt}{\isachardot}{\kern0pt}\ {\isacharequal}{\kern0pt}\ p\ {\isacharminus}{\kern0pt}\ {\isacharparenleft}{\kern0pt}t{\isacharplus}{\kern0pt}{\isadigit{1}}{\isacharparenright}{\kern0pt}{\isachardoublequoteclose}\ \isacommand{by}\isamarkupfalse%
\ simp\isanewline
\ \ \isacommand{also}\isamarkupfalse%
\ \isacommand{have}\isamarkupfalse%
\ {\isachardoublequoteopen}{\isachardot}{\kern0pt}{\isachardot}{\kern0pt}{\isachardot}{\kern0pt}\ {\isacharequal}{\kern0pt}\ {\isacharparenleft}{\kern0pt}real\ p{\isacharminus}{\kern0pt}{\isadigit{1}}{\isacharparenright}{\kern0pt}{\isacharslash}{\kern0pt}{\isadigit{2}}{\isachardoublequoteclose}\isanewline
\ \ \ \ \isacommand{by}\isamarkupfalse%
\ {\isacharparenleft}{\kern0pt}simp\ add{\isacharcolon}{\kern0pt}t{\isacharunderscore}{\kern0pt}def{\isacharparenright}{\kern0pt}\isanewline
\ \ \isacommand{finally}\isamarkupfalse%
\ \isacommand{have}\isamarkupfalse%
\ c{\isacharunderscore}{\kern0pt}{\isadigit{2}}{\isacharcolon}{\kern0pt}\ {\isachardoublequoteopen}card\ {\isacharparenleft}{\kern0pt}{\isacharbraceleft}{\kern0pt}k{\isachardot}{\kern0pt}\ p\ {\isasymle}\ {\isadigit{2}}\ {\isacharasterisk}{\kern0pt}\ k{\isacharbraceright}{\kern0pt}\ {\isasyminter}\ {\isacharbraceleft}{\kern0pt}{\isadigit{0}}{\isachardot}{\kern0pt}{\isachardot}{\kern0pt}{\isacharless}{\kern0pt}p{\isacharbraceright}{\kern0pt}{\isacharparenright}{\kern0pt}\ {\isacharequal}{\kern0pt}\ {\isacharparenleft}{\kern0pt}real\ p{\isacharminus}{\kern0pt}{\isadigit{1}}{\isacharparenright}{\kern0pt}{\isacharslash}{\kern0pt}{\isadigit{2}}{\isachardoublequoteclose}\ \isacommand{by}\isamarkupfalse%
\ simp\isanewline
\isanewline
\ \ \isacommand{have}\isamarkupfalse%
\ {\isachardoublequoteopen}integral\isactrlsup L\ {\isasymOmega}\ {\isacharparenleft}{\kern0pt}{\isasymlambda}x{\isachardot}{\kern0pt}\ real{\isacharunderscore}{\kern0pt}of{\isacharunderscore}{\kern0pt}int\ {\isacharparenleft}{\kern0pt}f{\isadigit{2}}{\isacharunderscore}{\kern0pt}hash\ p\ x\ k{\isacharparenright}{\kern0pt}\ {\isacharcircum}{\kern0pt}\ m{\isacharparenright}{\kern0pt}\ {\isacharequal}{\kern0pt}\isanewline
\ \ \ \ integral\isactrlsup L\ {\isasymOmega}\ {\isacharparenleft}{\kern0pt}{\isasymlambda}{\isasymomega}{\isachardot}{\kern0pt}\ indicator\ {\isacharbraceleft}{\kern0pt}{\isasymomega}{\isachardot}{\kern0pt}\ {\isadigit{2}}\ {\isacharasterisk}{\kern0pt}\ hash\ p\ k\ {\isasymomega}\ {\isacharless}{\kern0pt}\ p{\isacharbraceright}{\kern0pt}\ {\isasymomega}\ {\isacharasterisk}{\kern0pt}\ {\isacharparenleft}{\kern0pt}real\ p\ {\isacharminus}{\kern0pt}\ {\isadigit{1}}{\isacharparenright}{\kern0pt}{\isacharcircum}{\kern0pt}m\ {\isacharplus}{\kern0pt}\ \isanewline
\ \ \ \ \ \ indicator\ {\isacharbraceleft}{\kern0pt}{\isasymomega}{\isachardot}{\kern0pt}\ {\isadigit{2}}\ {\isacharasterisk}{\kern0pt}\ hash\ p\ k\ {\isasymomega}\ {\isasymge}\ p{\isacharbraceright}{\kern0pt}\ {\isasymomega}\ {\isacharasterisk}{\kern0pt}\ {\isacharparenleft}{\kern0pt}{\isacharminus}{\kern0pt}real\ p\ {\isacharminus}{\kern0pt}\ {\isadigit{1}}{\isacharparenright}{\kern0pt}{\isacharcircum}{\kern0pt}m{\isacharparenright}{\kern0pt}{\isachardoublequoteclose}\ \isanewline
\ \ \ \ \isacommand{by}\isamarkupfalse%
\ {\isacharparenleft}{\kern0pt}rule\ Bochner{\isacharunderscore}{\kern0pt}Integration{\isachardot}{\kern0pt}integral{\isacharunderscore}{\kern0pt}cong{\isacharcomma}{\kern0pt}\ simp{\isacharcomma}{\kern0pt}\ simp{\isacharparenright}{\kern0pt}\isanewline
\ \ \isacommand{also}\isamarkupfalse%
\ \isacommand{have}\isamarkupfalse%
\ {\isachardoublequoteopen}{\isachardot}{\kern0pt}{\isachardot}{\kern0pt}{\isachardot}{\kern0pt}\ {\isacharequal}{\kern0pt}\ \isanewline
\ \ \ \ \ {\isasymP}{\isacharparenleft}{\kern0pt}{\isasymomega}\ in\ measure{\isacharunderscore}{\kern0pt}pmf\ {\isasymOmega}{\isachardot}{\kern0pt}\ hash\ p\ k\ {\isasymomega}\ {\isasymin}\ {\isacharbraceleft}{\kern0pt}k{\isachardot}{\kern0pt}\ {\isadigit{2}}\ {\isacharasterisk}{\kern0pt}\ k\ {\isacharless}{\kern0pt}\ p{\isacharbraceright}{\kern0pt}{\isacharparenright}{\kern0pt}\ \ {\isacharasterisk}{\kern0pt}\ {\isacharparenleft}{\kern0pt}real\ p\ {\isacharminus}{\kern0pt}\ {\isadigit{1}}{\isacharparenright}{\kern0pt}\ {\isacharcircum}{\kern0pt}\ m\ \ {\isacharplus}{\kern0pt}\ \isanewline
\ \ \ \ \ {\isasymP}{\isacharparenleft}{\kern0pt}{\isasymomega}\ in\ measure{\isacharunderscore}{\kern0pt}pmf\ {\isasymOmega}{\isachardot}{\kern0pt}\ hash\ p\ k\ {\isasymomega}\ {\isasymin}\ {\isacharbraceleft}{\kern0pt}k{\isachardot}{\kern0pt}\ {\isadigit{2}}\ {\isacharasterisk}{\kern0pt}\ k\ {\isasymge}\ p{\isacharbraceright}{\kern0pt}{\isacharparenright}{\kern0pt}\ \ {\isacharasterisk}{\kern0pt}\ {\isacharparenleft}{\kern0pt}{\isacharminus}{\kern0pt}real\ p\ {\isacharminus}{\kern0pt}\ {\isadigit{1}}{\isacharparenright}{\kern0pt}\ {\isacharcircum}{\kern0pt}\ m\ {\isachardoublequoteclose}\isanewline
\ \ \ \ \isacommand{apply}\isamarkupfalse%
\ {\isacharparenleft}{\kern0pt}subst\ Bochner{\isacharunderscore}{\kern0pt}Integration{\isachardot}{\kern0pt}integral{\isacharunderscore}{\kern0pt}add{\isacharparenright}{\kern0pt}\isanewline
\ \ \ \ \isacommand{apply}\isamarkupfalse%
\ {\isacharparenleft}{\kern0pt}rule\ integrable{\isacharunderscore}{\kern0pt}measure{\isacharunderscore}{\kern0pt}pmf{\isacharunderscore}{\kern0pt}finite{\isacharbrackleft}{\kern0pt}OF\ b{\isacharbrackright}{\kern0pt}{\isacharparenright}{\kern0pt}\isanewline
\ \ \ \ \isacommand{apply}\isamarkupfalse%
\ {\isacharparenleft}{\kern0pt}rule\ integrable{\isacharunderscore}{\kern0pt}measure{\isacharunderscore}{\kern0pt}pmf{\isacharunderscore}{\kern0pt}finite{\isacharbrackleft}{\kern0pt}OF\ b{\isacharbrackright}{\kern0pt}{\isacharparenright}{\kern0pt}\isanewline
\ \ \ \ \isacommand{by}\isamarkupfalse%
\ simp\isanewline
\ \ \isacommand{also}\isamarkupfalse%
\ \isacommand{have}\isamarkupfalse%
\ {\isachardoublequoteopen}{\isachardot}{\kern0pt}{\isachardot}{\kern0pt}{\isachardot}{\kern0pt}\ {\isacharequal}{\kern0pt}\ {\isacharparenleft}{\kern0pt}real\ p\ {\isacharplus}{\kern0pt}\ {\isadigit{1}}{\isacharparenright}{\kern0pt}\ {\isacharasterisk}{\kern0pt}\ {\isacharparenleft}{\kern0pt}real\ p\ {\isacharminus}{\kern0pt}\ {\isadigit{1}}{\isacharparenright}{\kern0pt}\ {\isacharcircum}{\kern0pt}\ m\ {\isacharslash}{\kern0pt}\ {\isacharparenleft}{\kern0pt}{\isadigit{2}}\ {\isacharasterisk}{\kern0pt}\ real\ p{\isacharparenright}{\kern0pt}\ {\isacharplus}{\kern0pt}\ {\isacharparenleft}{\kern0pt}real\ p\ {\isacharminus}{\kern0pt}\ {\isadigit{1}}{\isacharparenright}{\kern0pt}\ {\isacharasterisk}{\kern0pt}\ {\isacharparenleft}{\kern0pt}{\isacharminus}{\kern0pt}\ real\ p\ {\isacharminus}{\kern0pt}\ {\isadigit{1}}{\isacharparenright}{\kern0pt}\ {\isacharcircum}{\kern0pt}\ m\ {\isacharslash}{\kern0pt}\ {\isacharparenleft}{\kern0pt}{\isadigit{2}}\ {\isacharasterisk}{\kern0pt}\ real\ p{\isacharparenright}{\kern0pt}{\isachardoublequoteclose}\isanewline
\ \ \ \ \isacommand{apply}\isamarkupfalse%
\ {\isacharparenleft}{\kern0pt}simp\ only{\isacharcolon}{\kern0pt}{\isasymOmega}{\isacharunderscore}{\kern0pt}def\ hash{\isacharunderscore}{\kern0pt}prob{\isacharunderscore}{\kern0pt}range{\isacharbrackleft}{\kern0pt}OF\ assms{\isacharparenleft}{\kern0pt}{\isadigit{1}}{\isacharparenright}{\kern0pt}\ assms{\isacharparenleft}{\kern0pt}{\isadigit{2}}{\isacharparenright}{\kern0pt}\ zero{\isacharunderscore}{\kern0pt}le{\isacharunderscore}{\kern0pt}{\isadigit{4}}{\isacharbrackright}{\kern0pt}\ c{\isacharunderscore}{\kern0pt}{\isadigit{1}}\ c{\isacharunderscore}{\kern0pt}{\isadigit{2}}{\isacharparenright}{\kern0pt}\isanewline
\ \ \ \ \isacommand{by}\isamarkupfalse%
\ simp\isanewline
\ \ \isacommand{also}\isamarkupfalse%
\ \isacommand{have}\isamarkupfalse%
\ {\isachardoublequoteopen}{\isachardot}{\kern0pt}{\isachardot}{\kern0pt}{\isachardot}{\kern0pt}\ {\isacharequal}{\kern0pt}\ \ \isanewline
\ \ \ \ {\isacharparenleft}{\kern0pt}{\isacharparenleft}{\kern0pt}real\ p\ {\isacharminus}{\kern0pt}\ {\isadigit{1}}{\isacharparenright}{\kern0pt}\ {\isacharcircum}{\kern0pt}\ m\ {\isacharasterisk}{\kern0pt}\ {\isacharparenleft}{\kern0pt}real\ p\ {\isacharplus}{\kern0pt}\ {\isadigit{1}}{\isacharparenright}{\kern0pt}\ {\isacharplus}{\kern0pt}\ {\isacharparenleft}{\kern0pt}{\isacharminus}{\kern0pt}\ real\ p\ {\isacharminus}{\kern0pt}\ {\isadigit{1}}{\isacharparenright}{\kern0pt}\ {\isacharcircum}{\kern0pt}\ m\ {\isacharasterisk}{\kern0pt}\ {\isacharparenleft}{\kern0pt}real\ p\ {\isacharminus}{\kern0pt}\ {\isadigit{1}}{\isacharparenright}{\kern0pt}{\isacharparenright}{\kern0pt}\ {\isacharslash}{\kern0pt}\ {\isacharparenleft}{\kern0pt}{\isadigit{2}}\ {\isacharasterisk}{\kern0pt}\ real\ p{\isacharparenright}{\kern0pt}{\isachardoublequoteclose}\isanewline
\ \ \ \ \isacommand{by}\isamarkupfalse%
\ {\isacharparenleft}{\kern0pt}simp\ add{\isacharcolon}{\kern0pt}add{\isacharunderscore}{\kern0pt}divide{\isacharunderscore}{\kern0pt}distrib\ ac{\isacharunderscore}{\kern0pt}simps{\isacharparenright}{\kern0pt}\isanewline
\ \ \isacommand{finally}\isamarkupfalse%
\ \isacommand{have}\isamarkupfalse%
\ a{\isacharcolon}{\kern0pt}{\isachardoublequoteopen}integral\isactrlsup L\ {\isasymOmega}\ {\isacharparenleft}{\kern0pt}{\isasymlambda}x{\isachardot}{\kern0pt}\ real{\isacharunderscore}{\kern0pt}of{\isacharunderscore}{\kern0pt}int\ {\isacharparenleft}{\kern0pt}f{\isadigit{2}}{\isacharunderscore}{\kern0pt}hash\ p\ x\ k{\isacharparenright}{\kern0pt}\ {\isacharcircum}{\kern0pt}\ m{\isacharparenright}{\kern0pt}\ {\isacharequal}{\kern0pt}\ \isanewline
\ \ \ \ {\isacharparenleft}{\kern0pt}{\isacharparenleft}{\kern0pt}real\ p\ {\isacharminus}{\kern0pt}\ {\isadigit{1}}{\isacharparenright}{\kern0pt}\ {\isacharcircum}{\kern0pt}\ m\ {\isacharasterisk}{\kern0pt}\ {\isacharparenleft}{\kern0pt}real\ p\ {\isacharplus}{\kern0pt}\ {\isadigit{1}}{\isacharparenright}{\kern0pt}\ {\isacharplus}{\kern0pt}\ {\isacharparenleft}{\kern0pt}{\isacharminus}{\kern0pt}\ real\ p\ {\isacharminus}{\kern0pt}\ {\isadigit{1}}{\isacharparenright}{\kern0pt}\ {\isacharcircum}{\kern0pt}\ m\ {\isacharasterisk}{\kern0pt}\ {\isacharparenleft}{\kern0pt}real\ p\ {\isacharminus}{\kern0pt}\ {\isadigit{1}}{\isacharparenright}{\kern0pt}{\isacharparenright}{\kern0pt}\ {\isacharslash}{\kern0pt}\ {\isacharparenleft}{\kern0pt}{\isadigit{2}}\ {\isacharasterisk}{\kern0pt}\ real\ p{\isacharparenright}{\kern0pt}{\isachardoublequoteclose}\ \isacommand{by}\isamarkupfalse%
\ simp\isanewline
\isanewline
\ \ \isacommand{show}\isamarkupfalse%
\ {\isacharquery}{\kern0pt}thesis\isanewline
\ \ \ \ \isacommand{apply}\isamarkupfalse%
\ {\isacharparenleft}{\kern0pt}subst\ {\isasymOmega}{\isacharunderscore}{\kern0pt}def{\isacharbrackleft}{\kern0pt}symmetric{\isacharbrackright}{\kern0pt}{\isacharparenright}{\kern0pt}\isanewline
\ \ \ \ \isacommand{by}\isamarkupfalse%
\ {\isacharparenleft}{\kern0pt}metis\ a{\isacharparenright}{\kern0pt}\isanewline
\isacommand{qed}\isamarkupfalse%
%
\endisatagproof
{\isafoldproof}%
%
\isadelimproof
\isanewline
%
\endisadelimproof
\isanewline
\isacommand{lemma}\isamarkupfalse%
\ \isanewline
\ \ \isakeyword{assumes}\ {\isachardoublequoteopen}Factorial{\isacharunderscore}{\kern0pt}Ring{\isachardot}{\kern0pt}prime\ p{\isachardoublequoteclose}\isanewline
\ \ \isakeyword{assumes}\ {\isachardoublequoteopen}p\ {\isachargreater}{\kern0pt}\ {\isadigit{2}}{\isachardoublequoteclose}\isanewline
\ \ \isakeyword{assumes}\ {\isachardoublequoteopen}{\isasymAnd}a{\isachardot}{\kern0pt}\ a\ {\isasymin}\ set\ as\ {\isasymLongrightarrow}\ a\ {\isacharless}{\kern0pt}\ p{\isachardoublequoteclose}\isanewline
\ \ \isakeyword{defines}\ {\isachardoublequoteopen}M\ {\isasymequiv}\ measure{\isacharunderscore}{\kern0pt}pmf\ {\isacharparenleft}{\kern0pt}pmf{\isacharunderscore}{\kern0pt}of{\isacharunderscore}{\kern0pt}set\ {\isacharparenleft}{\kern0pt}bounded{\isacharunderscore}{\kern0pt}degree{\isacharunderscore}{\kern0pt}polynomials\ {\isacharparenleft}{\kern0pt}ZFact\ {\isacharparenleft}{\kern0pt}int\ p{\isacharparenright}{\kern0pt}{\isacharparenright}{\kern0pt}\ {\isadigit{4}}{\isacharparenright}{\kern0pt}{\isacharparenright}{\kern0pt}{\isachardoublequoteclose}\isanewline
\ \ \isakeyword{defines}\ {\isachardoublequoteopen}f\ {\isasymequiv}\ {\isacharparenleft}{\kern0pt}{\isasymlambda}{\isasymomega}{\isachardot}{\kern0pt}\ real{\isacharunderscore}{\kern0pt}of{\isacharunderscore}{\kern0pt}int\ {\isacharparenleft}{\kern0pt}sum{\isacharunderscore}{\kern0pt}list\ {\isacharparenleft}{\kern0pt}map\ {\isacharparenleft}{\kern0pt}f{\isadigit{2}}{\isacharunderscore}{\kern0pt}hash\ p\ {\isasymomega}{\isacharparenright}{\kern0pt}\ as{\isacharparenright}{\kern0pt}{\isacharparenright}{\kern0pt}{\isacharcircum}{\kern0pt}{\isadigit{2}}{\isacharparenright}{\kern0pt}{\isachardoublequoteclose}\isanewline
\ \ \isakeyword{shows}\ var{\isacharunderscore}{\kern0pt}f{\isadigit{2}}{\isacharcolon}{\kern0pt}{\isachardoublequoteopen}prob{\isacharunderscore}{\kern0pt}space{\isachardot}{\kern0pt}variance\ M\ f\ {\isasymle}\ {\isadigit{2}}{\isacharasterisk}{\kern0pt}{\isacharparenleft}{\kern0pt}real{\isacharunderscore}{\kern0pt}of{\isacharunderscore}{\kern0pt}rat\ {\isacharparenleft}{\kern0pt}F\ {\isadigit{2}}\ as{\isacharparenright}{\kern0pt}{\isacharcircum}{\kern0pt}{\isadigit{2}}{\isacharparenright}{\kern0pt}\ {\isacharasterisk}{\kern0pt}\ {\isacharparenleft}{\kern0pt}{\isacharparenleft}{\kern0pt}real\ p{\isacharparenright}{\kern0pt}\isactrlsup {\isadigit{2}}{\isacharminus}{\kern0pt}{\isadigit{1}}{\isacharparenright}{\kern0pt}\isactrlsup {\isadigit{2}}{\isachardoublequoteclose}\ {\isacharparenleft}{\kern0pt}\isakeyword{is}\ {\isachardoublequoteopen}{\isacharquery}{\kern0pt}A{\isachardoublequoteclose}{\isacharparenright}{\kern0pt}\isanewline
\ \ \isakeyword{and}\ exp{\isacharunderscore}{\kern0pt}f{\isadigit{2}}{\isacharcolon}{\kern0pt}{\isachardoublequoteopen}prob{\isacharunderscore}{\kern0pt}space{\isachardot}{\kern0pt}expectation\ M\ f\ {\isacharequal}{\kern0pt}\ real{\isacharunderscore}{\kern0pt}of{\isacharunderscore}{\kern0pt}rat\ {\isacharparenleft}{\kern0pt}F\ {\isadigit{2}}\ as{\isacharparenright}{\kern0pt}\ {\isacharasterisk}{\kern0pt}\ {\isacharparenleft}{\kern0pt}{\isacharparenleft}{\kern0pt}real\ p{\isacharparenright}{\kern0pt}\isactrlsup {\isadigit{2}}{\isacharminus}{\kern0pt}{\isadigit{1}}{\isacharparenright}{\kern0pt}{\isachardoublequoteclose}\ {\isacharparenleft}{\kern0pt}\isakeyword{is}\ {\isachardoublequoteopen}{\isacharquery}{\kern0pt}B{\isachardoublequoteclose}{\isacharparenright}{\kern0pt}\isanewline
%
\isadelimproof
%
\endisadelimproof
%
\isatagproof
\isacommand{proof}\isamarkupfalse%
\ {\isacharminus}{\kern0pt}\isanewline
\ \ \isacommand{define}\isamarkupfalse%
\ h\ \isakeyword{where}\ {\isachardoublequoteopen}h\ {\isacharequal}{\kern0pt}\ {\isacharparenleft}{\kern0pt}{\isasymlambda}{\isasymomega}\ x{\isachardot}{\kern0pt}\ real{\isacharunderscore}{\kern0pt}of{\isacharunderscore}{\kern0pt}int\ {\isacharparenleft}{\kern0pt}f{\isadigit{2}}{\isacharunderscore}{\kern0pt}hash\ p\ {\isasymomega}\ x{\isacharparenright}{\kern0pt}{\isacharparenright}{\kern0pt}{\isachardoublequoteclose}\isanewline
\ \ \isacommand{define}\isamarkupfalse%
\ c\ \isakeyword{where}\ {\isachardoublequoteopen}c\ {\isacharequal}{\kern0pt}\ {\isacharparenleft}{\kern0pt}{\isasymlambda}x{\isachardot}{\kern0pt}\ real\ {\isacharparenleft}{\kern0pt}count{\isacharunderscore}{\kern0pt}list\ as\ x{\isacharparenright}{\kern0pt}{\isacharparenright}{\kern0pt}{\isachardoublequoteclose}\isanewline
\ \ \isacommand{define}\isamarkupfalse%
\ r\ \isakeyword{where}\ {\isachardoublequoteopen}r\ {\isacharequal}{\kern0pt}\ {\isacharparenleft}{\kern0pt}{\isasymlambda}{\isacharparenleft}{\kern0pt}m{\isacharcolon}{\kern0pt}{\isacharcolon}{\kern0pt}nat{\isacharparenright}{\kern0pt}{\isachardot}{\kern0pt}\ {\isacharparenleft}{\kern0pt}{\isacharparenleft}{\kern0pt}real\ p\ {\isacharminus}{\kern0pt}\ {\isadigit{1}}{\isacharparenright}{\kern0pt}\ {\isacharcircum}{\kern0pt}\ m\ {\isacharasterisk}{\kern0pt}\ {\isacharparenleft}{\kern0pt}real\ p\ {\isacharplus}{\kern0pt}\ {\isadigit{1}}{\isacharparenright}{\kern0pt}\ {\isacharplus}{\kern0pt}\ {\isacharparenleft}{\kern0pt}{\isacharminus}{\kern0pt}\ real\ p\ {\isacharminus}{\kern0pt}\ {\isadigit{1}}{\isacharparenright}{\kern0pt}\ {\isacharcircum}{\kern0pt}\ m\ {\isacharasterisk}{\kern0pt}\ {\isacharparenleft}{\kern0pt}real\ p\ {\isacharminus}{\kern0pt}\ {\isadigit{1}}{\isacharparenright}{\kern0pt}{\isacharparenright}{\kern0pt}\ {\isacharslash}{\kern0pt}\ {\isacharparenleft}{\kern0pt}{\isadigit{2}}\ {\isacharasterisk}{\kern0pt}\ real\ p{\isacharparenright}{\kern0pt}{\isacharparenright}{\kern0pt}{\isachardoublequoteclose}\isanewline
\ \ \isacommand{define}\isamarkupfalse%
\ h{\isacharunderscore}{\kern0pt}prod\ \isakeyword{where}\ {\isachardoublequoteopen}h{\isacharunderscore}{\kern0pt}prod\ {\isacharequal}{\kern0pt}\ {\isacharparenleft}{\kern0pt}{\isasymlambda}as\ {\isasymomega}{\isachardot}{\kern0pt}\ prod{\isacharunderscore}{\kern0pt}list\ {\isacharparenleft}{\kern0pt}map\ {\isacharparenleft}{\kern0pt}h\ {\isasymomega}{\isacharparenright}{\kern0pt}\ as{\isacharparenright}{\kern0pt}{\isacharparenright}{\kern0pt}{\isachardoublequoteclose}\ \isanewline
\isanewline
\ \ \isacommand{define}\isamarkupfalse%
\ exp{\isacharunderscore}{\kern0pt}h{\isacharunderscore}{\kern0pt}prod\ {\isacharcolon}{\kern0pt}{\isacharcolon}{\kern0pt}\ {\isachardoublequoteopen}nat\ list\ {\isasymRightarrow}\ real{\isachardoublequoteclose}\ \isakeyword{where}\ {\isachardoublequoteopen}exp{\isacharunderscore}{\kern0pt}h{\isacharunderscore}{\kern0pt}prod\ {\isacharequal}{\kern0pt}\ {\isacharparenleft}{\kern0pt}{\isasymlambda}as{\isachardot}{\kern0pt}\ {\isacharparenleft}{\kern0pt}{\isasymProd}i\ {\isasymin}\ set\ as{\isachardot}{\kern0pt}\ r\ {\isacharparenleft}{\kern0pt}count{\isacharunderscore}{\kern0pt}list\ as\ i{\isacharparenright}{\kern0pt}{\isacharparenright}{\kern0pt}{\isacharparenright}{\kern0pt}{\isachardoublequoteclose}\isanewline
\isanewline
\ \ \isacommand{interpret}\isamarkupfalse%
\ prob{\isacharunderscore}{\kern0pt}space\ M\isanewline
\ \ \ \ \isacommand{using}\isamarkupfalse%
\ prob{\isacharunderscore}{\kern0pt}space{\isacharunderscore}{\kern0pt}measure{\isacharunderscore}{\kern0pt}pmf\ M{\isacharunderscore}{\kern0pt}def\ \isacommand{by}\isamarkupfalse%
\ auto\isanewline
\isanewline
\ \ \isacommand{have}\isamarkupfalse%
\ f{\isacharunderscore}{\kern0pt}eq{\isacharcolon}{\kern0pt}\ {\isachardoublequoteopen}f\ {\isacharequal}{\kern0pt}\ {\isacharparenleft}{\kern0pt}{\isasymlambda}{\isasymomega}{\isachardot}{\kern0pt}\ {\isacharparenleft}{\kern0pt}{\isasymSum}x\ {\isasymin}\ set\ as{\isachardot}{\kern0pt}\ c\ x\ {\isacharasterisk}{\kern0pt}\ h\ {\isasymomega}\ x{\isacharparenright}{\kern0pt}{\isacharcircum}{\kern0pt}{\isadigit{2}}{\isacharparenright}{\kern0pt}{\isachardoublequoteclose}\isanewline
\ \ \ \ \isacommand{by}\isamarkupfalse%
\ {\isacharparenleft}{\kern0pt}simp\ add{\isacharcolon}{\kern0pt}f{\isacharunderscore}{\kern0pt}def\ c{\isacharunderscore}{\kern0pt}def\ h{\isacharunderscore}{\kern0pt}def\ sum{\isacharunderscore}{\kern0pt}list{\isacharunderscore}{\kern0pt}eval\ del{\isacharcolon}{\kern0pt}f{\isadigit{2}}{\isacharunderscore}{\kern0pt}hash{\isachardot}{\kern0pt}simps{\isacharparenright}{\kern0pt}\isanewline
\isanewline
\ \ \isacommand{have}\isamarkupfalse%
\ p{\isacharunderscore}{\kern0pt}ge{\isacharunderscore}{\kern0pt}{\isadigit{0}}{\isacharcolon}{\kern0pt}\ {\isachardoublequoteopen}p\ {\isachargreater}{\kern0pt}\ {\isadigit{0}}{\isachardoublequoteclose}\ \isacommand{using}\isamarkupfalse%
\ assms{\isacharparenleft}{\kern0pt}{\isadigit{2}}{\isacharparenright}{\kern0pt}\ \isacommand{by}\isamarkupfalse%
\ simp\isanewline
\isanewline
\ \ \isacommand{have}\isamarkupfalse%
\ int{\isacharunderscore}{\kern0pt}M{\isacharcolon}{\kern0pt}\ {\isachardoublequoteopen}{\isasymAnd}f{\isachardot}{\kern0pt}\ integrable\ M\ {\isacharparenleft}{\kern0pt}{\isasymlambda}{\isasymomega}{\isachardot}{\kern0pt}\ {\isacharparenleft}{\kern0pt}{\isacharparenleft}{\kern0pt}f\ {\isasymomega}{\isacharparenright}{\kern0pt}{\isacharcolon}{\kern0pt}{\isacharcolon}{\kern0pt}real{\isacharparenright}{\kern0pt}{\isacharparenright}{\kern0pt}{\isachardoublequoteclose}\isanewline
\ \ \ \ \isacommand{apply}\isamarkupfalse%
\ {\isacharparenleft}{\kern0pt}simp\ add{\isacharcolon}{\kern0pt}M{\isacharunderscore}{\kern0pt}def{\isacharparenright}{\kern0pt}\isanewline
\ \ \ \ \isacommand{apply}\isamarkupfalse%
\ {\isacharparenleft}{\kern0pt}rule\ integrable{\isacharunderscore}{\kern0pt}measure{\isacharunderscore}{\kern0pt}pmf{\isacharunderscore}{\kern0pt}finite{\isacharparenright}{\kern0pt}\isanewline
\ \ \ \ \isacommand{by}\isamarkupfalse%
\ {\isacharparenleft}{\kern0pt}metis\ p{\isacharunderscore}{\kern0pt}ge{\isacharunderscore}{\kern0pt}{\isadigit{0}}\ set{\isacharunderscore}{\kern0pt}pmf{\isacharunderscore}{\kern0pt}of{\isacharunderscore}{\kern0pt}set\ ne{\isacharunderscore}{\kern0pt}bounded{\isacharunderscore}{\kern0pt}degree{\isacharunderscore}{\kern0pt}polynomials\ fin{\isacharunderscore}{\kern0pt}bounded{\isacharunderscore}{\kern0pt}degree{\isacharunderscore}{\kern0pt}polynomials{\isacharparenright}{\kern0pt}\isanewline
\isanewline
\ \ \isacommand{have}\isamarkupfalse%
\ r{\isacharunderscore}{\kern0pt}one{\isacharcolon}{\kern0pt}\ {\isachardoublequoteopen}r\ {\isacharparenleft}{\kern0pt}Suc\ {\isadigit{0}}{\isacharparenright}{\kern0pt}\ {\isacharequal}{\kern0pt}\ {\isadigit{0}}{\isachardoublequoteclose}\ \isacommand{by}\isamarkupfalse%
\ {\isacharparenleft}{\kern0pt}simp\ add{\isacharcolon}{\kern0pt}r{\isacharunderscore}{\kern0pt}def\ algebra{\isacharunderscore}{\kern0pt}simps{\isacharparenright}{\kern0pt}\isanewline
\isanewline
\ \ \isacommand{have}\isamarkupfalse%
\ r{\isacharunderscore}{\kern0pt}two{\isacharcolon}{\kern0pt}\ {\isachardoublequoteopen}r\ {\isadigit{2}}\ {\isacharequal}{\kern0pt}\ {\isacharparenleft}{\kern0pt}real\ p{\isacharcircum}{\kern0pt}{\isadigit{2}}{\isacharminus}{\kern0pt}{\isadigit{1}}{\isacharparenright}{\kern0pt}{\isachardoublequoteclose}\isanewline
\ \ \ \ \isacommand{apply}\isamarkupfalse%
\ {\isacharparenleft}{\kern0pt}simp\ add{\isacharcolon}{\kern0pt}r{\isacharunderscore}{\kern0pt}def{\isacharparenright}{\kern0pt}\isanewline
\ \ \ \ \isacommand{apply}\isamarkupfalse%
\ {\isacharparenleft}{\kern0pt}subst\ nonzero{\isacharunderscore}{\kern0pt}divide{\isacharunderscore}{\kern0pt}eq{\isacharunderscore}{\kern0pt}eq{\isacharparenright}{\kern0pt}\ \isacommand{using}\isamarkupfalse%
\ assms\ \isacommand{apply}\isamarkupfalse%
\ simp\isanewline
\ \ \ \ \isacommand{by}\isamarkupfalse%
\ {\isacharparenleft}{\kern0pt}simp\ add{\isacharcolon}{\kern0pt}algebra{\isacharunderscore}{\kern0pt}simps\ power{\isadigit{2}}{\isacharunderscore}{\kern0pt}eq{\isacharunderscore}{\kern0pt}square{\isacharparenright}{\kern0pt}\isanewline
\isanewline
\ \ \isanewline
\ \ \isacommand{have}\isamarkupfalse%
\ r{\isacharunderscore}{\kern0pt}four{\isacharunderscore}{\kern0pt}est{\isacharcolon}{\kern0pt}\ {\isachardoublequoteopen}r\ {\isadigit{4}}\ {\isasymle}\ {\isadigit{3}}\ {\isacharasterisk}{\kern0pt}\ r\ {\isadigit{2}}\ {\isacharasterisk}{\kern0pt}\ r\ {\isadigit{2}}{\isachardoublequoteclose}\ \isanewline
\ \ \ \ \isacommand{apply}\isamarkupfalse%
\ {\isacharparenleft}{\kern0pt}simp\ add{\isacharcolon}{\kern0pt}r{\isacharunderscore}{\kern0pt}two{\isacharparenright}{\kern0pt}\isanewline
\ \ \ \ \isacommand{apply}\isamarkupfalse%
\ {\isacharparenleft}{\kern0pt}simp\ add{\isacharcolon}{\kern0pt}r{\isacharunderscore}{\kern0pt}def{\isacharparenright}{\kern0pt}\isanewline
\ \ \ \ \isacommand{apply}\isamarkupfalse%
\ {\isacharparenleft}{\kern0pt}subst\ pos{\isacharunderscore}{\kern0pt}divide{\isacharunderscore}{\kern0pt}le{\isacharunderscore}{\kern0pt}eq{\isacharparenright}{\kern0pt}\ \isacommand{using}\isamarkupfalse%
\ assms\ \isacommand{apply}\isamarkupfalse%
\ simp\isanewline
\ \ \ \ \isacommand{apply}\isamarkupfalse%
\ {\isacharparenleft}{\kern0pt}simp\ add{\isacharcolon}{\kern0pt}algebra{\isacharunderscore}{\kern0pt}simps\ power{\isadigit{2}}{\isacharunderscore}{\kern0pt}eq{\isacharunderscore}{\kern0pt}square\ power{\isadigit{4}}{\isacharunderscore}{\kern0pt}eq{\isacharunderscore}{\kern0pt}xxxx{\isacharparenright}{\kern0pt}\isanewline
\ \ \ \ \isacommand{apply}\isamarkupfalse%
\ {\isacharparenleft}{\kern0pt}rule\ order{\isacharunderscore}{\kern0pt}trans{\isacharbrackleft}{\kern0pt}\isakeyword{where}\ y{\isacharequal}{\kern0pt}{\isachardoublequoteopen}real\ p\ {\isacharasterisk}{\kern0pt}\ {\isadigit{1}}{\isadigit{2}}\ {\isacharplus}{\kern0pt}\ real\ p\ {\isacharasterisk}{\kern0pt}\ {\isacharparenleft}{\kern0pt}real\ p\ {\isacharasterisk}{\kern0pt}\ {\isacharparenleft}{\kern0pt}real\ p\ {\isacharasterisk}{\kern0pt}\ {\isadigit{1}}{\isadigit{6}}{\isacharparenright}{\kern0pt}{\isacharparenright}{\kern0pt}{\isachardoublequoteclose}{\isacharbrackright}{\kern0pt}{\isacharparenright}{\kern0pt}\isanewline
\ \ \ \ \ \isacommand{apply}\isamarkupfalse%
\ simp\isanewline
\ \ \ \ \isacommand{apply}\isamarkupfalse%
\ {\isacharparenleft}{\kern0pt}rule\ add{\isacharunderscore}{\kern0pt}mono{\isacharcomma}{\kern0pt}\ simp{\isacharparenright}{\kern0pt}\isanewline
\ \ \ \ \isacommand{apply}\isamarkupfalse%
\ {\isacharparenleft}{\kern0pt}rule\ mult{\isacharunderscore}{\kern0pt}left{\isacharunderscore}{\kern0pt}mono{\isacharparenright}{\kern0pt}\isanewline
\ \ \ \ \isacommand{apply}\isamarkupfalse%
\ {\isacharparenleft}{\kern0pt}rule\ mult{\isacharunderscore}{\kern0pt}left{\isacharunderscore}{\kern0pt}mono{\isacharparenright}{\kern0pt}\isanewline
\ \ \ \ \ \ \isacommand{apply}\isamarkupfalse%
\ {\isacharparenleft}{\kern0pt}rule\ mult{\isacharunderscore}{\kern0pt}left{\isacharunderscore}{\kern0pt}mono{\isacharparenright}{\kern0pt}\isanewline
\ \ \ \ \isacommand{apply}\isamarkupfalse%
\ simp\isanewline
\ \ \ \ \isacommand{using}\isamarkupfalse%
\ assms{\isacharparenleft}{\kern0pt}{\isadigit{2}}{\isacharparenright}{\kern0pt}\ \isanewline
\ \ \ \ \ \ \ \isacommand{apply}\isamarkupfalse%
\ {\isacharparenleft}{\kern0pt}metis\ assms{\isacharparenleft}{\kern0pt}{\isadigit{1}}{\isacharparenright}{\kern0pt}\ linorder{\isacharunderscore}{\kern0pt}not{\isacharunderscore}{\kern0pt}less\ num{\isacharunderscore}{\kern0pt}double\ numeral{\isacharunderscore}{\kern0pt}mult\ of{\isacharunderscore}{\kern0pt}nat{\isacharunderscore}{\kern0pt}power\ power{\isadigit{2}}{\isacharunderscore}{\kern0pt}eq{\isacharunderscore}{\kern0pt}square\ power{\isadigit{2}}{\isacharunderscore}{\kern0pt}nat{\isacharunderscore}{\kern0pt}le{\isacharunderscore}{\kern0pt}eq{\isacharunderscore}{\kern0pt}le\ prime{\isacharunderscore}{\kern0pt}ge{\isacharunderscore}{\kern0pt}{\isadigit{2}}{\isacharunderscore}{\kern0pt}nat\ real{\isacharunderscore}{\kern0pt}of{\isacharunderscore}{\kern0pt}nat{\isacharunderscore}{\kern0pt}less{\isacharunderscore}{\kern0pt}numeral{\isacharunderscore}{\kern0pt}iff{\isacharparenright}{\kern0pt}\isanewline
\ \ \ \ \isacommand{by}\isamarkupfalse%
\ simp{\isacharplus}{\kern0pt}\isanewline
\isanewline
\ \ \isacommand{have}\isamarkupfalse%
\ fold{\isacharunderscore}{\kern0pt}sym{\isacharcolon}{\kern0pt}\ {\isachardoublequoteopen}{\isasymAnd}x\ y{\isachardot}{\kern0pt}\ {\isacharparenleft}{\kern0pt}x\ {\isasymnoteq}\ y\ {\isasymand}\ y\ {\isasymnoteq}\ x{\isacharparenright}{\kern0pt}\ {\isacharequal}{\kern0pt}\ {\isacharparenleft}{\kern0pt}x\ {\isasymnoteq}\ y{\isacharparenright}{\kern0pt}{\isachardoublequoteclose}\ \isacommand{by}\isamarkupfalse%
\ auto\isanewline
\isanewline
\ \ \isacommand{have}\isamarkupfalse%
\ exp{\isacharunderscore}{\kern0pt}h{\isacharunderscore}{\kern0pt}prod{\isacharunderscore}{\kern0pt}elim{\isacharcolon}{\kern0pt}\ {\isachardoublequoteopen}exp{\isacharunderscore}{\kern0pt}h{\isacharunderscore}{\kern0pt}prod\ {\isacharequal}{\kern0pt}\ {\isacharparenleft}{\kern0pt}{\isasymlambda}as{\isachardot}{\kern0pt}\ prod{\isacharunderscore}{\kern0pt}list\ {\isacharparenleft}{\kern0pt}map\ {\isacharparenleft}{\kern0pt}r\ {\isasymcirc}\ count{\isacharunderscore}{\kern0pt}list\ as{\isacharparenright}{\kern0pt}\ {\isacharparenleft}{\kern0pt}remdups\ as{\isacharparenright}{\kern0pt}{\isacharparenright}{\kern0pt}{\isacharparenright}{\kern0pt}{\isachardoublequoteclose}\ \isanewline
\ \ \ \ \isacommand{apply}\isamarkupfalse%
\ {\isacharparenleft}{\kern0pt}simp\ add{\isacharcolon}{\kern0pt}exp{\isacharunderscore}{\kern0pt}h{\isacharunderscore}{\kern0pt}prod{\isacharunderscore}{\kern0pt}def{\isacharparenright}{\kern0pt}\isanewline
\ \ \ \ \isacommand{apply}\isamarkupfalse%
\ {\isacharparenleft}{\kern0pt}rule\ ext{\isacharparenright}{\kern0pt}\isanewline
\ \ \ \ \isacommand{apply}\isamarkupfalse%
\ {\isacharparenleft}{\kern0pt}subst\ prod{\isachardot}{\kern0pt}set{\isacharunderscore}{\kern0pt}conv{\isacharunderscore}{\kern0pt}list{\isacharbrackleft}{\kern0pt}symmetric{\isacharbrackright}{\kern0pt}{\isacharparenright}{\kern0pt}\isanewline
\ \ \ \ \isacommand{by}\isamarkupfalse%
\ {\isacharparenleft}{\kern0pt}rule\ prod{\isachardot}{\kern0pt}cong{\isacharcomma}{\kern0pt}\ simp{\isacharcomma}{\kern0pt}\ simp\ add{\isacharcolon}{\kern0pt}comp{\isacharunderscore}{\kern0pt}def{\isacharparenright}{\kern0pt}\isanewline
\isanewline
\ \ \isacommand{have}\isamarkupfalse%
\ exp{\isacharunderscore}{\kern0pt}h{\isacharunderscore}{\kern0pt}prod{\isacharcolon}{\kern0pt}\ {\isachardoublequoteopen}{\isasymAnd}x{\isachardot}{\kern0pt}\ set\ x\ {\isasymsubseteq}\ set\ as\ {\isasymLongrightarrow}\ length\ x\ {\isasymle}\ {\isadigit{4}}\ {\isasymLongrightarrow}\ expectation\ {\isacharparenleft}{\kern0pt}h{\isacharunderscore}{\kern0pt}prod\ x{\isacharparenright}{\kern0pt}\ {\isacharequal}{\kern0pt}\ exp{\isacharunderscore}{\kern0pt}h{\isacharunderscore}{\kern0pt}prod\ x{\isachardoublequoteclose}\isanewline
\ \ \isacommand{proof}\isamarkupfalse%
\ {\isacharminus}{\kern0pt}\isanewline
\ \ \ \ \isacommand{fix}\isamarkupfalse%
\ x\ \isanewline
\ \ \ \ \isacommand{assume}\isamarkupfalse%
\ {\isachardoublequoteopen}set\ x\ {\isasymsubseteq}\ set\ as{\isachardoublequoteclose}\isanewline
\ \ \ \ \isacommand{hence}\isamarkupfalse%
\ x{\isacharunderscore}{\kern0pt}sub{\isacharunderscore}{\kern0pt}p{\isacharcolon}{\kern0pt}\ {\isachardoublequoteopen}set\ x\ {\isasymsubseteq}\ {\isacharbraceleft}{\kern0pt}{\isadigit{0}}{\isachardot}{\kern0pt}{\isachardot}{\kern0pt}{\isacharless}{\kern0pt}p{\isacharbraceright}{\kern0pt}{\isachardoublequoteclose}\ \isacommand{using}\isamarkupfalse%
\ assms{\isacharparenleft}{\kern0pt}{\isadigit{3}}{\isacharparenright}{\kern0pt}\ atLeastLessThan{\isacharunderscore}{\kern0pt}iff\ \isacommand{by}\isamarkupfalse%
\ blast\isanewline
\ \ \ \ \isacommand{hence}\isamarkupfalse%
\ x{\isacharunderscore}{\kern0pt}le{\isacharunderscore}{\kern0pt}p{\isacharcolon}{\kern0pt}\ {\isachardoublequoteopen}{\isasymAnd}k{\isachardot}{\kern0pt}\ k\ {\isasymin}\ set\ x\ {\isasymLongrightarrow}\ k\ {\isacharless}{\kern0pt}\ p{\isachardoublequoteclose}\ \isacommand{by}\isamarkupfalse%
\ auto\isanewline
\ \ \ \ \isacommand{assume}\isamarkupfalse%
\ {\isachardoublequoteopen}length\ x\ {\isasymle}\ {\isadigit{4}}{\isachardoublequoteclose}\isanewline
\ \ \ \ \isacommand{hence}\isamarkupfalse%
\ card{\isacharunderscore}{\kern0pt}x{\isacharcolon}{\kern0pt}\ {\isachardoublequoteopen}card\ {\isacharparenleft}{\kern0pt}set\ x{\isacharparenright}{\kern0pt}\ {\isasymle}\ {\isadigit{4}}{\isachardoublequoteclose}\ \isacommand{using}\isamarkupfalse%
\ card{\isacharunderscore}{\kern0pt}length\ dual{\isacharunderscore}{\kern0pt}order{\isachardot}{\kern0pt}trans\ \isacommand{by}\isamarkupfalse%
\ blast\isanewline
\isanewline
\ \ \ \ \isacommand{have}\isamarkupfalse%
\ {\isachardoublequoteopen}expectation\ {\isacharparenleft}{\kern0pt}h{\isacharunderscore}{\kern0pt}prod\ x{\isacharparenright}{\kern0pt}\ {\isacharequal}{\kern0pt}\ expectation\ {\isacharparenleft}{\kern0pt}{\isasymlambda}{\isasymomega}{\isachardot}{\kern0pt}\ {\isasymProd}\ i\ {\isasymin}\ set\ x{\isachardot}{\kern0pt}\ h\ {\isasymomega}\ i{\isacharcircum}{\kern0pt}{\isacharparenleft}{\kern0pt}count{\isacharunderscore}{\kern0pt}list\ x\ i{\isacharparenright}{\kern0pt}{\isacharparenright}{\kern0pt}{\isachardoublequoteclose}\isanewline
\ \ \ \ \ \ \isacommand{apply}\isamarkupfalse%
\ {\isacharparenleft}{\kern0pt}rule\ arg{\isacharunderscore}{\kern0pt}cong{\isacharbrackleft}{\kern0pt}\isakeyword{where}\ f{\isacharequal}{\kern0pt}{\isachardoublequoteopen}expectation{\isachardoublequoteclose}{\isacharbrackright}{\kern0pt}{\isacharparenright}{\kern0pt}\isanewline
\ \ \ \ \ \ \isacommand{by}\isamarkupfalse%
\ {\isacharparenleft}{\kern0pt}simp\ add{\isacharcolon}{\kern0pt}h{\isacharunderscore}{\kern0pt}prod{\isacharunderscore}{\kern0pt}def\ prod{\isacharunderscore}{\kern0pt}list{\isacharunderscore}{\kern0pt}eval{\isacharparenright}{\kern0pt}\isanewline
\ \ \ \ \isacommand{also}\isamarkupfalse%
\ \isacommand{have}\isamarkupfalse%
\ {\isachardoublequoteopen}{\isachardot}{\kern0pt}{\isachardot}{\kern0pt}{\isachardot}{\kern0pt}\ {\isacharequal}{\kern0pt}\ {\isacharparenleft}{\kern0pt}{\isasymProd}i\ {\isasymin}\ set\ x{\isachardot}{\kern0pt}\ expectation\ {\isacharparenleft}{\kern0pt}{\isasymlambda}{\isasymomega}{\isachardot}{\kern0pt}\ h\ {\isasymomega}\ i{\isacharcircum}{\kern0pt}{\isacharparenleft}{\kern0pt}count{\isacharunderscore}{\kern0pt}list\ x\ i{\isacharparenright}{\kern0pt}{\isacharparenright}{\kern0pt}{\isacharparenright}{\kern0pt}{\isachardoublequoteclose}\isanewline
\ \ \ \ \ \ \isacommand{apply}\isamarkupfalse%
\ {\isacharparenleft}{\kern0pt}subst\ indep{\isacharunderscore}{\kern0pt}vars{\isacharunderscore}{\kern0pt}lebesgue{\isacharunderscore}{\kern0pt}integral{\isacharcomma}{\kern0pt}\ simp{\isacharparenright}{\kern0pt}\isanewline
\ \ \ \ \ \ \ \ \isacommand{apply}\isamarkupfalse%
\ {\isacharparenleft}{\kern0pt}simp\ add{\isacharcolon}{\kern0pt}h{\isacharunderscore}{\kern0pt}def{\isacharparenright}{\kern0pt}\isanewline
\ \ \ \ \ \ \ \ \isacommand{apply}\isamarkupfalse%
\ {\isacharparenleft}{\kern0pt}rule\ indep{\isacharunderscore}{\kern0pt}vars{\isacharunderscore}{\kern0pt}compose{\isadigit{2}}{\isacharbrackleft}{\kern0pt}\isakeyword{where}\ X{\isacharequal}{\kern0pt}{\isachardoublequoteopen}hash\ p{\isachardoublequoteclose}\ \isakeyword{and}\ M{\isacharprime}{\kern0pt}{\isacharequal}{\kern0pt}{\isachardoublequoteopen}\ {\isacharparenleft}{\kern0pt}{\isasymlambda}{\isacharunderscore}{\kern0pt}{\isachardot}{\kern0pt}\ pmf{\isacharunderscore}{\kern0pt}of{\isacharunderscore}{\kern0pt}set\ {\isacharbraceleft}{\kern0pt}{\isadigit{0}}{\isachardot}{\kern0pt}{\isachardot}{\kern0pt}{\isacharless}{\kern0pt}p{\isacharbraceright}{\kern0pt}{\isacharparenright}{\kern0pt}{\isachardoublequoteclose}{\isacharbrackright}{\kern0pt}{\isacharparenright}{\kern0pt}\isanewline
\ \ \ \ \ \ \ \ \ \isacommand{using}\isamarkupfalse%
\ hash{\isacharunderscore}{\kern0pt}k{\isacharunderscore}{\kern0pt}wise{\isacharunderscore}{\kern0pt}indep{\isacharbrackleft}{\kern0pt}\isakeyword{where}\ n{\isacharequal}{\kern0pt}{\isachardoublequoteopen}{\isadigit{4}}{\isachardoublequoteclose}\ \isakeyword{and}\ p{\isacharequal}{\kern0pt}{\isachardoublequoteopen}p{\isachardoublequoteclose}{\isacharbrackright}{\kern0pt}\ card{\isacharunderscore}{\kern0pt}x\ x{\isacharunderscore}{\kern0pt}sub{\isacharunderscore}{\kern0pt}p\ assms{\isacharparenleft}{\kern0pt}{\isadigit{1}}{\isacharparenright}{\kern0pt}\isanewline
\ \ \ \ \ \ \ \ \ \isacommand{apply}\isamarkupfalse%
\ {\isacharparenleft}{\kern0pt}simp\ add{\isacharcolon}{\kern0pt}k{\isacharunderscore}{\kern0pt}wise{\isacharunderscore}{\kern0pt}indep{\isacharunderscore}{\kern0pt}vars{\isacharunderscore}{\kern0pt}def\ M{\isacharunderscore}{\kern0pt}def{\isacharbrackleft}{\kern0pt}symmetric{\isacharbrackright}{\kern0pt}{\isacharparenright}{\kern0pt}\isanewline
\ \ \ \ \ \ \ \ \isacommand{apply}\isamarkupfalse%
\ simp\isanewline
\ \ \ \ \ \ \ \isacommand{apply}\isamarkupfalse%
\ {\isacharparenleft}{\kern0pt}rule\ int{\isacharunderscore}{\kern0pt}M{\isacharparenright}{\kern0pt}\isanewline
\ \ \ \ \ \ \isacommand{by}\isamarkupfalse%
\ simp\isanewline
\ \ \ \ \isacommand{also}\isamarkupfalse%
\ \isacommand{have}\isamarkupfalse%
\ {\isachardoublequoteopen}{\isachardot}{\kern0pt}{\isachardot}{\kern0pt}{\isachardot}{\kern0pt}\ {\isacharequal}{\kern0pt}\ {\isacharparenleft}{\kern0pt}{\isasymProd}i\ {\isasymin}\ set\ x{\isachardot}{\kern0pt}\ r\ {\isacharparenleft}{\kern0pt}count{\isacharunderscore}{\kern0pt}list\ x\ i{\isacharparenright}{\kern0pt}{\isacharparenright}{\kern0pt}{\isachardoublequoteclose}\isanewline
\ \ \ \ \ \ \isacommand{apply}\isamarkupfalse%
\ {\isacharparenleft}{\kern0pt}rule\ prod{\isachardot}{\kern0pt}cong{\isacharcomma}{\kern0pt}\ simp{\isacharparenright}{\kern0pt}\isanewline
\ \ \ \ \ \ \isacommand{using}\isamarkupfalse%
\ f{\isadigit{2}}{\isacharunderscore}{\kern0pt}hash{\isacharunderscore}{\kern0pt}exp{\isacharbrackleft}{\kern0pt}OF\ assms{\isacharparenleft}{\kern0pt}{\isadigit{1}}{\isacharparenright}{\kern0pt}\ x{\isacharunderscore}{\kern0pt}le{\isacharunderscore}{\kern0pt}p\ assms{\isacharparenleft}{\kern0pt}{\isadigit{2}}{\isacharparenright}{\kern0pt}{\isacharbrackright}{\kern0pt}\ \isanewline
\ \ \ \ \ \ \isacommand{by}\isamarkupfalse%
\ {\isacharparenleft}{\kern0pt}simp\ add{\isacharcolon}{\kern0pt}h{\isacharunderscore}{\kern0pt}def\ r{\isacharunderscore}{\kern0pt}def\ M{\isacharunderscore}{\kern0pt}def{\isacharbrackleft}{\kern0pt}symmetric{\isacharbrackright}{\kern0pt}\ del{\isacharcolon}{\kern0pt}f{\isadigit{2}}{\isacharunderscore}{\kern0pt}hash{\isachardot}{\kern0pt}simps{\isacharparenright}{\kern0pt}\isanewline
\ \ \ \ \isacommand{also}\isamarkupfalse%
\ \isacommand{have}\isamarkupfalse%
\ {\isachardoublequoteopen}{\isachardot}{\kern0pt}{\isachardot}{\kern0pt}{\isachardot}{\kern0pt}\ {\isacharequal}{\kern0pt}\ exp{\isacharunderscore}{\kern0pt}h{\isacharunderscore}{\kern0pt}prod\ x{\isachardoublequoteclose}\isanewline
\ \ \ \ \ \ \isacommand{by}\isamarkupfalse%
\ {\isacharparenleft}{\kern0pt}simp\ add{\isacharcolon}{\kern0pt}exp{\isacharunderscore}{\kern0pt}h{\isacharunderscore}{\kern0pt}prod{\isacharunderscore}{\kern0pt}def{\isacharparenright}{\kern0pt}\isanewline
\ \ \ \ \isacommand{finally}\isamarkupfalse%
\ \isacommand{show}\isamarkupfalse%
\ {\isachardoublequoteopen}expectation\ {\isacharparenleft}{\kern0pt}h{\isacharunderscore}{\kern0pt}prod\ x{\isacharparenright}{\kern0pt}\ {\isacharequal}{\kern0pt}\ exp{\isacharunderscore}{\kern0pt}h{\isacharunderscore}{\kern0pt}prod\ x{\isachardoublequoteclose}\ \isacommand{by}\isamarkupfalse%
\ simp\isanewline
\ \ \isacommand{qed}\isamarkupfalse%
\isanewline
\isanewline
\ \ \isacommand{have}\isamarkupfalse%
\ exp{\isacharunderscore}{\kern0pt}h{\isacharunderscore}{\kern0pt}prod{\isacharunderscore}{\kern0pt}cong{\isacharcolon}{\kern0pt}\ {\isachardoublequoteopen}{\isasymAnd}x\ y{\isachardot}{\kern0pt}\ has{\isacharunderscore}{\kern0pt}eq{\isacharunderscore}{\kern0pt}relation\ x\ y\ {\isasymLongrightarrow}\ exp{\isacharunderscore}{\kern0pt}h{\isacharunderscore}{\kern0pt}prod\ x\ {\isacharequal}{\kern0pt}\ exp{\isacharunderscore}{\kern0pt}h{\isacharunderscore}{\kern0pt}prod\ y{\isachardoublequoteclose}\ \isanewline
\ \ \isacommand{proof}\isamarkupfalse%
\ {\isacharminus}{\kern0pt}\isanewline
\ \ \ \ \isacommand{fix}\isamarkupfalse%
\ x\ y\ {\isacharcolon}{\kern0pt}{\isacharcolon}{\kern0pt}\ {\isachardoublequoteopen}nat\ list{\isachardoublequoteclose}\isanewline
\ \ \ \ \isacommand{assume}\isamarkupfalse%
\ a{\isacharcolon}{\kern0pt}{\isachardoublequoteopen}has{\isacharunderscore}{\kern0pt}eq{\isacharunderscore}{\kern0pt}relation\ x\ y{\isachardoublequoteclose}\isanewline
\ \ \ \ \isacommand{then}\isamarkupfalse%
\ \isacommand{obtain}\isamarkupfalse%
\ f\ \isakeyword{where}\ b{\isacharcolon}{\kern0pt}{\isachardoublequoteopen}bij{\isacharunderscore}{\kern0pt}betw\ f\ {\isacharparenleft}{\kern0pt}set\ x{\isacharparenright}{\kern0pt}\ {\isacharparenleft}{\kern0pt}set\ y{\isacharparenright}{\kern0pt}{\isachardoublequoteclose}\ \isakeyword{and}\ c{\isacharcolon}{\kern0pt}{\isachardoublequoteopen}{\isasymAnd}z{\isachardot}{\kern0pt}\ z\ {\isasymin}\ set\ x\ {\isasymLongrightarrow}\ count{\isacharunderscore}{\kern0pt}list\ x\ z\ {\isacharequal}{\kern0pt}\ count{\isacharunderscore}{\kern0pt}list\ y\ {\isacharparenleft}{\kern0pt}f\ z{\isacharparenright}{\kern0pt}{\isachardoublequoteclose}\isanewline
\ \ \ \ \ \ \isacommand{using}\isamarkupfalse%
\ eq{\isacharunderscore}{\kern0pt}rel{\isacharunderscore}{\kern0pt}obtain{\isacharunderscore}{\kern0pt}bij{\isacharbrackleft}{\kern0pt}OF\ a{\isacharbrackright}{\kern0pt}\ \isacommand{by}\isamarkupfalse%
\ blast\isanewline
\ \ \ \ \isacommand{have}\isamarkupfalse%
\ {\isachardoublequoteopen}exp{\isacharunderscore}{\kern0pt}h{\isacharunderscore}{\kern0pt}prod\ x\ {\isacharequal}{\kern0pt}\ prod\ {\isacharparenleft}{\kern0pt}\ {\isacharparenleft}{\kern0pt}{\isasymlambda}i{\isachardot}{\kern0pt}\ r{\isacharparenleft}{\kern0pt}count{\isacharunderscore}{\kern0pt}list\ y\ i{\isacharparenright}{\kern0pt}{\isacharparenright}{\kern0pt}\ {\isasymcirc}\ f{\isacharparenright}{\kern0pt}\ {\isacharparenleft}{\kern0pt}set\ x{\isacharparenright}{\kern0pt}{\isachardoublequoteclose}\isanewline
\ \ \ \ \ \ \isacommand{by}\isamarkupfalse%
\ {\isacharparenleft}{\kern0pt}simp\ add{\isacharcolon}{\kern0pt}exp{\isacharunderscore}{\kern0pt}h{\isacharunderscore}{\kern0pt}prod{\isacharunderscore}{\kern0pt}def\ c{\isacharparenright}{\kern0pt}\isanewline
\ \ \ \ \isacommand{also}\isamarkupfalse%
\ \isacommand{have}\isamarkupfalse%
\ {\isachardoublequoteopen}{\isachardot}{\kern0pt}{\isachardot}{\kern0pt}{\isachardot}{\kern0pt}\ {\isacharequal}{\kern0pt}\ {\isacharparenleft}{\kern0pt}{\isasymProd}i\ {\isasymin}\ f\ {\isacharbackquote}{\kern0pt}\ {\isacharparenleft}{\kern0pt}set\ x{\isacharparenright}{\kern0pt}{\isachardot}{\kern0pt}\ r{\isacharparenleft}{\kern0pt}count{\isacharunderscore}{\kern0pt}list\ y\ i{\isacharparenright}{\kern0pt}{\isacharparenright}{\kern0pt}{\isachardoublequoteclose}\isanewline
\ \ \ \ \ \ \isacommand{apply}\isamarkupfalse%
\ {\isacharparenleft}{\kern0pt}rule\ prod{\isachardot}{\kern0pt}reindex{\isacharbrackleft}{\kern0pt}symmetric{\isacharbrackright}{\kern0pt}{\isacharparenright}{\kern0pt}\isanewline
\ \ \ \ \ \ \isacommand{using}\isamarkupfalse%
\ b\ bij{\isacharunderscore}{\kern0pt}betw{\isacharunderscore}{\kern0pt}def\ \isacommand{by}\isamarkupfalse%
\ blast\isanewline
\ \ \ \ \isacommand{also}\isamarkupfalse%
\ \isacommand{have}\isamarkupfalse%
\ {\isachardoublequoteopen}{\isachardot}{\kern0pt}{\isachardot}{\kern0pt}{\isachardot}{\kern0pt}\ {\isacharequal}{\kern0pt}\ exp{\isacharunderscore}{\kern0pt}h{\isacharunderscore}{\kern0pt}prod\ y{\isachardoublequoteclose}\isanewline
\ \ \ \ \ \ \isacommand{apply}\isamarkupfalse%
\ {\isacharparenleft}{\kern0pt}simp\ add{\isacharcolon}{\kern0pt}exp{\isacharunderscore}{\kern0pt}h{\isacharunderscore}{\kern0pt}prod{\isacharunderscore}{\kern0pt}def{\isacharparenright}{\kern0pt}\isanewline
\ \ \ \ \ \ \isacommand{apply}\isamarkupfalse%
\ {\isacharparenleft}{\kern0pt}rule\ prod{\isachardot}{\kern0pt}cong{\isacharparenright}{\kern0pt}\isanewline
\ \ \ \ \ \ \ \isacommand{apply}\isamarkupfalse%
\ {\isacharparenleft}{\kern0pt}metis\ b\ bij{\isacharunderscore}{\kern0pt}betw{\isacharunderscore}{\kern0pt}def{\isacharparenright}{\kern0pt}\isanewline
\ \ \ \ \ \ \isacommand{by}\isamarkupfalse%
\ simp\isanewline
\ \ \ \ \isanewline
\ \ \ \ \isacommand{finally}\isamarkupfalse%
\ \isacommand{show}\isamarkupfalse%
\ {\isachardoublequoteopen}exp{\isacharunderscore}{\kern0pt}h{\isacharunderscore}{\kern0pt}prod\ x\ {\isacharequal}{\kern0pt}\ exp{\isacharunderscore}{\kern0pt}h{\isacharunderscore}{\kern0pt}prod\ y{\isachardoublequoteclose}\ \isacommand{by}\isamarkupfalse%
\ simp\isanewline
\ \ \isacommand{qed}\isamarkupfalse%
\isanewline
\isanewline
\ \ \isacommand{hence}\isamarkupfalse%
\ exp{\isacharunderscore}{\kern0pt}h{\isacharunderscore}{\kern0pt}prod{\isacharunderscore}{\kern0pt}cong{\isacharcolon}{\kern0pt}\ {\isachardoublequoteopen}{\isasymAnd}p\ x{\isachardot}{\kern0pt}\ of{\isacharunderscore}{\kern0pt}bool\ {\isacharparenleft}{\kern0pt}has{\isacharunderscore}{\kern0pt}eq{\isacharunderscore}{\kern0pt}relation\ p\ x{\isacharparenright}{\kern0pt}\ {\isacharasterisk}{\kern0pt}\ exp{\isacharunderscore}{\kern0pt}h{\isacharunderscore}{\kern0pt}prod\ p\ {\isacharequal}{\kern0pt}\ of{\isacharunderscore}{\kern0pt}bool\ {\isacharparenleft}{\kern0pt}has{\isacharunderscore}{\kern0pt}eq{\isacharunderscore}{\kern0pt}relation\ p\ x{\isacharparenright}{\kern0pt}\ {\isacharasterisk}{\kern0pt}\ exp{\isacharunderscore}{\kern0pt}h{\isacharunderscore}{\kern0pt}prod\ x{\isachardoublequoteclose}\ \isanewline
\ \ \ \ \isacommand{by}\isamarkupfalse%
\ simp\isanewline
\isanewline
\ \ \isacommand{have}\isamarkupfalse%
\ {\isachardoublequoteopen}expectation\ f\ {\isacharequal}{\kern0pt}\ {\isacharparenleft}{\kern0pt}{\isasymSum}i{\isasymin}set\ as{\isachardot}{\kern0pt}\ {\isacharparenleft}{\kern0pt}{\isasymSum}j{\isasymin}set\ as{\isachardot}{\kern0pt}\ c\ i\ {\isacharasterisk}{\kern0pt}\ c\ j\ {\isacharasterisk}{\kern0pt}\ expectation\ {\isacharparenleft}{\kern0pt}h{\isacharunderscore}{\kern0pt}prod\ {\isacharbrackleft}{\kern0pt}i{\isacharcomma}{\kern0pt}j{\isacharbrackright}{\kern0pt}{\isacharparenright}{\kern0pt}{\isacharparenright}{\kern0pt}{\isacharparenright}{\kern0pt}{\isachardoublequoteclose}\isanewline
\ \ \ \ \isacommand{by}\isamarkupfalse%
\ {\isacharparenleft}{\kern0pt}simp\ add{\isacharcolon}{\kern0pt}f{\isacharunderscore}{\kern0pt}eq\ h{\isacharunderscore}{\kern0pt}prod{\isacharunderscore}{\kern0pt}def\ power{\isadigit{2}}{\isacharunderscore}{\kern0pt}eq{\isacharunderscore}{\kern0pt}square\ sum{\isacharunderscore}{\kern0pt}distrib{\isacharunderscore}{\kern0pt}left\ sum{\isacharunderscore}{\kern0pt}distrib{\isacharunderscore}{\kern0pt}right\ Bochner{\isacharunderscore}{\kern0pt}Integration{\isachardot}{\kern0pt}integral{\isacharunderscore}{\kern0pt}sum{\isacharbrackleft}{\kern0pt}OF\ int{\isacharunderscore}{\kern0pt}M{\isacharbrackright}{\kern0pt}\ algebra{\isacharunderscore}{\kern0pt}simps{\isacharparenright}{\kern0pt}\isanewline
\ \ \isacommand{also}\isamarkupfalse%
\ \isacommand{have}\isamarkupfalse%
\ {\isachardoublequoteopen}{\isachardot}{\kern0pt}{\isachardot}{\kern0pt}{\isachardot}{\kern0pt}\ {\isacharequal}{\kern0pt}\ {\isacharparenleft}{\kern0pt}{\isasymSum}i{\isasymin}set\ as{\isachardot}{\kern0pt}\ {\isacharparenleft}{\kern0pt}{\isasymSum}j{\isasymin}set\ as{\isachardot}{\kern0pt}\ c\ i\ {\isacharasterisk}{\kern0pt}\ c\ j\ {\isacharasterisk}{\kern0pt}\ exp{\isacharunderscore}{\kern0pt}h{\isacharunderscore}{\kern0pt}prod\ {\isacharbrackleft}{\kern0pt}i{\isacharcomma}{\kern0pt}j{\isacharbrackright}{\kern0pt}{\isacharparenright}{\kern0pt}{\isacharparenright}{\kern0pt}{\isachardoublequoteclose}\isanewline
\ \ \ \ \isacommand{apply}\isamarkupfalse%
\ {\isacharparenleft}{\kern0pt}rule\ sum{\isachardot}{\kern0pt}cong{\isacharcomma}{\kern0pt}\ simp{\isacharparenright}{\kern0pt}\isanewline
\ \ \ \ \isacommand{apply}\isamarkupfalse%
\ {\isacharparenleft}{\kern0pt}rule\ sum{\isachardot}{\kern0pt}cong{\isacharcomma}{\kern0pt}\ simp{\isacharparenright}{\kern0pt}\isanewline
\ \ \ \ \isacommand{apply}\isamarkupfalse%
\ {\isacharparenleft}{\kern0pt}subst\ exp{\isacharunderscore}{\kern0pt}h{\isacharunderscore}{\kern0pt}prod{\isacharcomma}{\kern0pt}\ simp{\isacharcomma}{\kern0pt}\ simp{\isacharparenright}{\kern0pt}\isanewline
\ \ \ \ \isacommand{by}\isamarkupfalse%
\ simp\isanewline
\ \ \isacommand{also}\isamarkupfalse%
\ \isacommand{have}\isamarkupfalse%
\ {\isachardoublequoteopen}{\isachardot}{\kern0pt}{\isachardot}{\kern0pt}{\isachardot}{\kern0pt}\ {\isacharequal}{\kern0pt}\ {\isacharparenleft}{\kern0pt}{\isasymSum}i\ {\isasymin}\ set\ as{\isachardot}{\kern0pt}\ {\isacharparenleft}{\kern0pt}{\isasymSum}j\ {\isasymin}\ set\ as{\isachardot}{\kern0pt}\ \ \isanewline
\ \ \ \ c\ i\ {\isacharasterisk}{\kern0pt}\ c\ j\ {\isacharasterisk}{\kern0pt}\ {\isacharparenleft}{\kern0pt}sum{\isacharunderscore}{\kern0pt}list\ {\isacharparenleft}{\kern0pt}map\ {\isacharparenleft}{\kern0pt}{\isasymlambda}p{\isachardot}{\kern0pt}\ of{\isacharunderscore}{\kern0pt}bool\ {\isacharparenleft}{\kern0pt}has{\isacharunderscore}{\kern0pt}eq{\isacharunderscore}{\kern0pt}relation\ p\ {\isacharbrackleft}{\kern0pt}i{\isacharcomma}{\kern0pt}j{\isacharbrackright}{\kern0pt}{\isacharparenright}{\kern0pt}\ {\isacharasterisk}{\kern0pt}\ exp{\isacharunderscore}{\kern0pt}h{\isacharunderscore}{\kern0pt}prod\ p{\isacharparenright}{\kern0pt}\ {\isacharparenleft}{\kern0pt}enum{\isacharunderscore}{\kern0pt}partitions\ {\isadigit{2}}{\isacharparenright}{\kern0pt}{\isacharparenright}{\kern0pt}{\isacharparenright}{\kern0pt}{\isacharparenright}{\kern0pt}{\isacharparenright}{\kern0pt}{\isachardoublequoteclose}\isanewline
\ \ \ \ \isacommand{apply}\isamarkupfalse%
\ {\isacharparenleft}{\kern0pt}subst\ exp{\isacharunderscore}{\kern0pt}h{\isacharunderscore}{\kern0pt}prod{\isacharunderscore}{\kern0pt}cong{\isacharparenright}{\kern0pt}\isanewline
\ \ \ \ \isacommand{apply}\isamarkupfalse%
\ {\isacharparenleft}{\kern0pt}subst\ sum{\isacharunderscore}{\kern0pt}partitions{\isacharprime}{\kern0pt}{\isacharcomma}{\kern0pt}\ simp{\isacharparenright}{\kern0pt}\isanewline
\ \ \ \ \isacommand{by}\isamarkupfalse%
\ simp\isanewline
\ \ \isacommand{also}\isamarkupfalse%
\ \isacommand{have}\isamarkupfalse%
\ {\isachardoublequoteopen}{\isachardot}{\kern0pt}{\isachardot}{\kern0pt}{\isachardot}{\kern0pt}\ {\isacharequal}{\kern0pt}\ {\isacharparenleft}{\kern0pt}{\isasymSum}i{\isasymin}set\ as{\isachardot}{\kern0pt}\ c\ i\ {\isacharasterisk}{\kern0pt}\ c\ i\ {\isacharasterisk}{\kern0pt}\ r\ {\isadigit{2}}{\isacharparenright}{\kern0pt}{\isachardoublequoteclose}\isanewline
\ \ \ \ \isacommand{apply}\isamarkupfalse%
\ {\isacharparenleft}{\kern0pt}simp\ add{\isacharcolon}{\kern0pt}numeral{\isacharunderscore}{\kern0pt}eq{\isacharunderscore}{\kern0pt}Suc\ exp{\isacharunderscore}{\kern0pt}h{\isacharunderscore}{\kern0pt}prod{\isacharunderscore}{\kern0pt}elim\ r{\isacharunderscore}{\kern0pt}one{\isacharparenright}{\kern0pt}\ \isanewline
\ \ \ \ \isacommand{by}\isamarkupfalse%
\ {\isacharparenleft}{\kern0pt}simp\ add{\isacharcolon}{\kern0pt}\ has{\isacharunderscore}{\kern0pt}eq{\isacharunderscore}{\kern0pt}relation{\isacharunderscore}{\kern0pt}elim\ distrib{\isacharunderscore}{\kern0pt}left\ sum{\isachardot}{\kern0pt}distrib\ sum{\isacharunderscore}{\kern0pt}collapse\ fold{\isacharunderscore}{\kern0pt}sym{\isacharparenright}{\kern0pt}\isanewline
\ \ \isacommand{also}\isamarkupfalse%
\ \isacommand{have}\isamarkupfalse%
\ {\isachardoublequoteopen}{\isachardot}{\kern0pt}{\isachardot}{\kern0pt}{\isachardot}{\kern0pt}\ {\isacharequal}{\kern0pt}\ real{\isacharunderscore}{\kern0pt}of{\isacharunderscore}{\kern0pt}rat\ {\isacharparenleft}{\kern0pt}F\ {\isadigit{2}}\ as{\isacharparenright}{\kern0pt}\ {\isacharasterisk}{\kern0pt}\ {\isacharparenleft}{\kern0pt}{\isacharparenleft}{\kern0pt}real\ p{\isacharparenright}{\kern0pt}{\isacharcircum}{\kern0pt}{\isadigit{2}}{\isacharminus}{\kern0pt}{\isadigit{1}}{\isacharparenright}{\kern0pt}{\isachardoublequoteclose}\isanewline
\ \ \ \ \isacommand{apply}\isamarkupfalse%
\ {\isacharparenleft}{\kern0pt}subst\ sum{\isacharunderscore}{\kern0pt}distrib{\isacharunderscore}{\kern0pt}right{\isacharbrackleft}{\kern0pt}symmetric{\isacharbrackright}{\kern0pt}{\isacharparenright}{\kern0pt}\isanewline
\ \ \ \ \isacommand{by}\isamarkupfalse%
\ {\isacharparenleft}{\kern0pt}simp\ add{\isacharcolon}{\kern0pt}c{\isacharunderscore}{\kern0pt}def\ F{\isacharunderscore}{\kern0pt}def\ power{\isadigit{2}}{\isacharunderscore}{\kern0pt}eq{\isacharunderscore}{\kern0pt}square\ of{\isacharunderscore}{\kern0pt}rat{\isacharunderscore}{\kern0pt}sum\ of{\isacharunderscore}{\kern0pt}rat{\isacharunderscore}{\kern0pt}mult\ r{\isacharunderscore}{\kern0pt}two{\isacharparenright}{\kern0pt}\isanewline
\ \ \isacommand{finally}\isamarkupfalse%
\ \isacommand{show}\isamarkupfalse%
\ b{\isacharcolon}{\kern0pt}{\isacharquery}{\kern0pt}B\ \isacommand{by}\isamarkupfalse%
\ simp\isanewline
\isanewline
\ \ \isacommand{have}\isamarkupfalse%
\ {\isachardoublequoteopen}expectation\ {\isacharparenleft}{\kern0pt}{\isasymlambda}x{\isachardot}{\kern0pt}\ {\isacharparenleft}{\kern0pt}f\ x{\isacharparenright}{\kern0pt}\isactrlsup {\isadigit{2}}{\isacharparenright}{\kern0pt}\ {\isacharequal}{\kern0pt}\ {\isacharparenleft}{\kern0pt}{\isasymSum}i{\isadigit{1}}\ {\isasymin}\ set\ as{\isachardot}{\kern0pt}\ {\isacharparenleft}{\kern0pt}{\isasymSum}i{\isadigit{2}}\ {\isasymin}\ set\ as{\isachardot}{\kern0pt}\ {\isacharparenleft}{\kern0pt}{\isasymSum}i{\isadigit{3}}\ {\isasymin}\ set\ as{\isachardot}{\kern0pt}\ {\isacharparenleft}{\kern0pt}{\isasymSum}i{\isadigit{4}}\ {\isasymin}\ set\ as{\isachardot}{\kern0pt}\isanewline
\ \ \ \ c\ i{\isadigit{1}}\ {\isacharasterisk}{\kern0pt}\ c\ i{\isadigit{2}}\ {\isacharasterisk}{\kern0pt}\ c\ i{\isadigit{3}}\ {\isacharasterisk}{\kern0pt}\ c\ i{\isadigit{4}}\ {\isacharasterisk}{\kern0pt}\ expectation\ {\isacharparenleft}{\kern0pt}h{\isacharunderscore}{\kern0pt}prod\ {\isacharbrackleft}{\kern0pt}i{\isadigit{1}}{\isacharcomma}{\kern0pt}\ i{\isadigit{2}}{\isacharcomma}{\kern0pt}\ i{\isadigit{3}}{\isacharcomma}{\kern0pt}\ i{\isadigit{4}}{\isacharbrackright}{\kern0pt}{\isacharparenright}{\kern0pt}{\isacharparenright}{\kern0pt}{\isacharparenright}{\kern0pt}{\isacharparenright}{\kern0pt}{\isacharparenright}{\kern0pt}{\isachardoublequoteclose}\isanewline
\ \ \ \ \isacommand{apply}\isamarkupfalse%
\ {\isacharparenleft}{\kern0pt}simp\ add{\isacharcolon}{\kern0pt}f{\isacharunderscore}{\kern0pt}eq\ h{\isacharunderscore}{\kern0pt}prod{\isacharunderscore}{\kern0pt}def\ power{\isadigit{4}}{\isacharunderscore}{\kern0pt}eq{\isacharunderscore}{\kern0pt}xxxx\ sum{\isacharunderscore}{\kern0pt}distrib{\isacharunderscore}{\kern0pt}left\ sum{\isacharunderscore}{\kern0pt}distrib{\isacharunderscore}{\kern0pt}right\ Bochner{\isacharunderscore}{\kern0pt}Integration{\isachardot}{\kern0pt}integral{\isacharunderscore}{\kern0pt}sum{\isacharbrackleft}{\kern0pt}OF\ int{\isacharunderscore}{\kern0pt}M{\isacharbrackright}{\kern0pt}{\isacharparenright}{\kern0pt}\isanewline
\ \ \ \ \isacommand{by}\isamarkupfalse%
\ {\isacharparenleft}{\kern0pt}simp\ add{\isacharcolon}{\kern0pt}algebra{\isacharunderscore}{\kern0pt}simps{\isacharparenright}{\kern0pt}\isanewline
\ \ \isacommand{also}\isamarkupfalse%
\ \isacommand{have}\isamarkupfalse%
\ {\isachardoublequoteopen}{\isachardot}{\kern0pt}{\isachardot}{\kern0pt}{\isachardot}{\kern0pt}\ {\isacharequal}{\kern0pt}\ {\isacharparenleft}{\kern0pt}{\isasymSum}i{\isadigit{1}}\ {\isasymin}\ set\ as{\isachardot}{\kern0pt}\ {\isacharparenleft}{\kern0pt}{\isasymSum}i{\isadigit{2}}\ {\isasymin}\ set\ as{\isachardot}{\kern0pt}\ {\isacharparenleft}{\kern0pt}{\isasymSum}i{\isadigit{3}}\ {\isasymin}\ set\ as{\isachardot}{\kern0pt}\ {\isacharparenleft}{\kern0pt}{\isasymSum}i{\isadigit{4}}\ {\isasymin}\ set\ as{\isachardot}{\kern0pt}\ \isanewline
\ \ \ \ c\ i{\isadigit{1}}\ {\isacharasterisk}{\kern0pt}\ c\ i{\isadigit{2}}\ {\isacharasterisk}{\kern0pt}\ c\ i{\isadigit{3}}\ {\isacharasterisk}{\kern0pt}\ c\ i{\isadigit{4}}\ {\isacharasterisk}{\kern0pt}\ exp{\isacharunderscore}{\kern0pt}h{\isacharunderscore}{\kern0pt}prod\ {\isacharbrackleft}{\kern0pt}i{\isadigit{1}}{\isacharcomma}{\kern0pt}i{\isadigit{2}}{\isacharcomma}{\kern0pt}i{\isadigit{3}}{\isacharcomma}{\kern0pt}i{\isadigit{4}}{\isacharbrackright}{\kern0pt}{\isacharparenright}{\kern0pt}{\isacharparenright}{\kern0pt}{\isacharparenright}{\kern0pt}{\isacharparenright}{\kern0pt}{\isachardoublequoteclose}\isanewline
\ \ \ \ \isacommand{apply}\isamarkupfalse%
\ {\isacharparenleft}{\kern0pt}rule\ sum{\isachardot}{\kern0pt}cong{\isacharcomma}{\kern0pt}\ simp{\isacharparenright}{\kern0pt}\isanewline
\ \ \ \ \isacommand{apply}\isamarkupfalse%
\ {\isacharparenleft}{\kern0pt}rule\ sum{\isachardot}{\kern0pt}cong{\isacharcomma}{\kern0pt}\ simp{\isacharparenright}{\kern0pt}\isanewline
\ \ \ \ \isacommand{apply}\isamarkupfalse%
\ {\isacharparenleft}{\kern0pt}rule\ sum{\isachardot}{\kern0pt}cong{\isacharcomma}{\kern0pt}\ simp{\isacharparenright}{\kern0pt}\isanewline
\ \ \ \ \isacommand{apply}\isamarkupfalse%
\ {\isacharparenleft}{\kern0pt}rule\ sum{\isachardot}{\kern0pt}cong{\isacharcomma}{\kern0pt}\ simp{\isacharparenright}{\kern0pt}\isanewline
\ \ \ \ \isacommand{apply}\isamarkupfalse%
\ {\isacharparenleft}{\kern0pt}subst\ exp{\isacharunderscore}{\kern0pt}h{\isacharunderscore}{\kern0pt}prod{\isacharcomma}{\kern0pt}\ simp{\isacharcomma}{\kern0pt}\ simp{\isacharparenright}{\kern0pt}\isanewline
\ \ \ \ \isacommand{by}\isamarkupfalse%
\ simp\isanewline
\ \ \isacommand{also}\isamarkupfalse%
\ \isacommand{have}\isamarkupfalse%
\ {\isachardoublequoteopen}{\isachardot}{\kern0pt}{\isachardot}{\kern0pt}{\isachardot}{\kern0pt}\ {\isacharequal}{\kern0pt}\ {\isacharparenleft}{\kern0pt}{\isasymSum}i{\isadigit{1}}\ {\isasymin}\ set\ as{\isachardot}{\kern0pt}\ {\isacharparenleft}{\kern0pt}{\isasymSum}i{\isadigit{2}}\ {\isasymin}\ set\ as{\isachardot}{\kern0pt}\ {\isacharparenleft}{\kern0pt}{\isasymSum}i{\isadigit{3}}\ {\isasymin}\ set\ as{\isachardot}{\kern0pt}\ {\isacharparenleft}{\kern0pt}{\isasymSum}i{\isadigit{4}}\ {\isasymin}\ set\ as{\isachardot}{\kern0pt}\ \isanewline
\ \ \ \ c\ i{\isadigit{1}}\ {\isacharasterisk}{\kern0pt}\ c\ i{\isadigit{2}}\ {\isacharasterisk}{\kern0pt}\ c\ i{\isadigit{3}}\ {\isacharasterisk}{\kern0pt}\ c\ i{\isadigit{4}}\ {\isacharasterisk}{\kern0pt}\ \isanewline
\ \ \ \ {\isacharparenleft}{\kern0pt}sum{\isacharunderscore}{\kern0pt}list\ {\isacharparenleft}{\kern0pt}map\ {\isacharparenleft}{\kern0pt}{\isasymlambda}p{\isachardot}{\kern0pt}\ of{\isacharunderscore}{\kern0pt}bool\ {\isacharparenleft}{\kern0pt}has{\isacharunderscore}{\kern0pt}eq{\isacharunderscore}{\kern0pt}relation\ p\ {\isacharbrackleft}{\kern0pt}i{\isadigit{1}}{\isacharcomma}{\kern0pt}i{\isadigit{2}}{\isacharcomma}{\kern0pt}i{\isadigit{3}}{\isacharcomma}{\kern0pt}i{\isadigit{4}}{\isacharbrackright}{\kern0pt}{\isacharparenright}{\kern0pt}\ {\isacharasterisk}{\kern0pt}\ exp{\isacharunderscore}{\kern0pt}h{\isacharunderscore}{\kern0pt}prod\ p{\isacharparenright}{\kern0pt}\ {\isacharparenleft}{\kern0pt}enum{\isacharunderscore}{\kern0pt}partitions\ {\isadigit{4}}{\isacharparenright}{\kern0pt}{\isacharparenright}{\kern0pt}{\isacharparenright}{\kern0pt}{\isacharparenright}{\kern0pt}{\isacharparenright}{\kern0pt}{\isacharparenright}{\kern0pt}{\isacharparenright}{\kern0pt}{\isachardoublequoteclose}\isanewline
\ \ \ \ \isacommand{apply}\isamarkupfalse%
\ {\isacharparenleft}{\kern0pt}subst\ exp{\isacharunderscore}{\kern0pt}h{\isacharunderscore}{\kern0pt}prod{\isacharunderscore}{\kern0pt}cong{\isacharparenright}{\kern0pt}\isanewline
\ \ \ \ \isacommand{apply}\isamarkupfalse%
\ {\isacharparenleft}{\kern0pt}subst\ sum{\isacharunderscore}{\kern0pt}partitions{\isacharprime}{\kern0pt}{\isacharcomma}{\kern0pt}\ simp{\isacharparenright}{\kern0pt}\isanewline
\ \ \ \ \isacommand{by}\isamarkupfalse%
\ simp\isanewline
\ \ \isacommand{also}\isamarkupfalse%
\ \isacommand{have}\isamarkupfalse%
\ {\isachardoublequoteopen}{\isachardot}{\kern0pt}{\isachardot}{\kern0pt}{\isachardot}{\kern0pt}\ {\isacharequal}{\kern0pt}\ \isanewline
\ \ \ \ {\isadigit{3}}\ {\isacharasterisk}{\kern0pt}\ {\isacharparenleft}{\kern0pt}{\isasymSum}i\ {\isasymin}\ set\ as{\isachardot}{\kern0pt}\ {\isacharparenleft}{\kern0pt}{\isasymSum}j\ {\isasymin}\ set\ as{\isachardot}{\kern0pt}\ c\ i{\isacharcircum}{\kern0pt}{\isadigit{2}}\ {\isacharasterisk}{\kern0pt}\ c\ j{\isacharcircum}{\kern0pt}{\isadigit{2}}\ {\isacharasterisk}{\kern0pt}\ r\ {\isadigit{2}}\ {\isacharasterisk}{\kern0pt}\ r\ {\isadigit{2}}{\isacharparenright}{\kern0pt}{\isacharparenright}{\kern0pt}\ {\isacharplus}{\kern0pt}\ {\isacharparenleft}{\kern0pt}{\isacharparenleft}{\kern0pt}{\isasymSum}\ i\ {\isasymin}\ set\ as{\isachardot}{\kern0pt}\ c\ i{\isacharcircum}{\kern0pt}{\isadigit{4}}\ {\isacharasterisk}{\kern0pt}\ r\ {\isadigit{4}}{\isacharparenright}{\kern0pt}\ {\isacharminus}{\kern0pt}\ {\isadigit{3}}\ {\isacharasterisk}{\kern0pt}\ \ {\isacharparenleft}{\kern0pt}{\isasymSum}\ i\ {\isasymin}\ set\ as{\isachardot}{\kern0pt}\ c\ i\ {\isacharcircum}{\kern0pt}\ {\isadigit{4}}\ {\isacharasterisk}{\kern0pt}\ r\ {\isadigit{2}}\ {\isacharasterisk}{\kern0pt}\ r\ {\isadigit{2}}{\isacharparenright}{\kern0pt}{\isacharparenright}{\kern0pt}{\isachardoublequoteclose}\isanewline
\ \ \ \ \isacommand{apply}\isamarkupfalse%
\ {\isacharparenleft}{\kern0pt}simp\ add{\isacharcolon}{\kern0pt}numeral{\isacharunderscore}{\kern0pt}eq{\isacharunderscore}{\kern0pt}Suc\ exp{\isacharunderscore}{\kern0pt}h{\isacharunderscore}{\kern0pt}prod{\isacharunderscore}{\kern0pt}elim\ r{\isacharunderscore}{\kern0pt}one{\isacharparenright}{\kern0pt}\ \isanewline
\ \ \ \ \isacommand{apply}\isamarkupfalse%
\ {\isacharparenleft}{\kern0pt}simp\ add{\isacharcolon}{\kern0pt}\ has{\isacharunderscore}{\kern0pt}eq{\isacharunderscore}{\kern0pt}relation{\isacharunderscore}{\kern0pt}elim\ distrib{\isacharunderscore}{\kern0pt}left\ sum{\isachardot}{\kern0pt}distrib\ sum{\isacharunderscore}{\kern0pt}collapse\ fold{\isacharunderscore}{\kern0pt}sym{\isacharparenright}{\kern0pt}\isanewline
\ \ \ \ \isacommand{by}\isamarkupfalse%
\ {\isacharparenleft}{\kern0pt}simp\ add{\isacharcolon}{\kern0pt}\ algebra{\isacharunderscore}{\kern0pt}simps\ sum{\isacharunderscore}{\kern0pt}subtractf\ sum{\isacharunderscore}{\kern0pt}collapse{\isacharparenright}{\kern0pt}\isanewline
\ \ \isacommand{also}\isamarkupfalse%
\ \isacommand{have}\isamarkupfalse%
\ {\isachardoublequoteopen}{\isachardot}{\kern0pt}{\isachardot}{\kern0pt}{\isachardot}{\kern0pt}\ {\isacharequal}{\kern0pt}\ {\isadigit{3}}\ {\isacharasterisk}{\kern0pt}\ {\isacharparenleft}{\kern0pt}{\isasymSum}i\ {\isasymin}\ set\ as{\isachardot}{\kern0pt}\ c\ i{\isacharcircum}{\kern0pt}{\isadigit{2}}\ {\isacharasterisk}{\kern0pt}\ r\ {\isadigit{2}}{\isacharparenright}{\kern0pt}{\isacharcircum}{\kern0pt}{\isadigit{2}}\ {\isacharplus}{\kern0pt}\ {\isacharparenleft}{\kern0pt}{\isasymSum}\ i\ {\isasymin}\ set\ as{\isachardot}{\kern0pt}\ c\ i\ {\isacharcircum}{\kern0pt}\ {\isadigit{4}}\ {\isacharasterisk}{\kern0pt}\ {\isacharparenleft}{\kern0pt}r\ {\isadigit{4}}\ {\isacharminus}{\kern0pt}\ {\isadigit{3}}\ {\isacharasterisk}{\kern0pt}\ r\ {\isadigit{2}}\ {\isacharasterisk}{\kern0pt}\ r\ {\isadigit{2}}{\isacharparenright}{\kern0pt}{\isacharparenright}{\kern0pt}{\isachardoublequoteclose}\isanewline
\ \ \ \ \isacommand{apply}\isamarkupfalse%
\ {\isacharparenleft}{\kern0pt}rule\ arg{\isacharunderscore}{\kern0pt}cong{\isadigit{2}}{\isacharbrackleft}{\kern0pt}\isakeyword{where}\ f{\isacharequal}{\kern0pt}{\isachardoublequoteopen}{\isacharparenleft}{\kern0pt}{\isacharplus}{\kern0pt}{\isacharparenright}{\kern0pt}{\isachardoublequoteclose}{\isacharbrackright}{\kern0pt}{\isacharparenright}{\kern0pt}\isanewline
\ \ \ \ \ \isacommand{apply}\isamarkupfalse%
\ {\isacharparenleft}{\kern0pt}simp\ add{\isacharcolon}{\kern0pt}power{\isadigit{2}}{\isacharunderscore}{\kern0pt}eq{\isacharunderscore}{\kern0pt}square\ sum{\isacharunderscore}{\kern0pt}distrib{\isacharunderscore}{\kern0pt}left\ sum{\isacharunderscore}{\kern0pt}distrib{\isacharunderscore}{\kern0pt}right\ algebra{\isacharunderscore}{\kern0pt}simps{\isacharparenright}{\kern0pt}\isanewline
\ \ \ \ \isacommand{apply}\isamarkupfalse%
\ {\isacharparenleft}{\kern0pt}simp\ add{\isacharcolon}{\kern0pt}sum{\isacharunderscore}{\kern0pt}distrib{\isacharunderscore}{\kern0pt}left\ sum{\isacharunderscore}{\kern0pt}subtractf{\isacharbrackleft}{\kern0pt}symmetric{\isacharbrackright}{\kern0pt}{\isacharparenright}{\kern0pt}\isanewline
\ \ \ \ \isacommand{apply}\isamarkupfalse%
\ {\isacharparenleft}{\kern0pt}rule\ sum{\isachardot}{\kern0pt}cong{\isacharcomma}{\kern0pt}\ simp{\isacharparenright}{\kern0pt}\isanewline
\ \ \ \ \isacommand{by}\isamarkupfalse%
\ {\isacharparenleft}{\kern0pt}simp\ add{\isacharcolon}{\kern0pt}algebra{\isacharunderscore}{\kern0pt}simps{\isacharparenright}{\kern0pt}\isanewline
\ \ \isacommand{also}\isamarkupfalse%
\ \isacommand{have}\isamarkupfalse%
\ {\isachardoublequoteopen}{\isachardot}{\kern0pt}{\isachardot}{\kern0pt}{\isachardot}{\kern0pt}\ {\isasymle}\ {\isadigit{3}}\ {\isacharasterisk}{\kern0pt}\ {\isacharparenleft}{\kern0pt}{\isasymSum}i\ {\isasymin}\ set\ as{\isachardot}{\kern0pt}\ c\ i{\isacharcircum}{\kern0pt}{\isadigit{2}}{\isacharparenright}{\kern0pt}{\isacharcircum}{\kern0pt}{\isadigit{2}}\ {\isacharasterisk}{\kern0pt}\ {\isacharparenleft}{\kern0pt}r\ {\isadigit{2}}{\isacharparenright}{\kern0pt}{\isacharcircum}{\kern0pt}{\isadigit{2}}\ {\isacharplus}{\kern0pt}\ \ {\isacharparenleft}{\kern0pt}{\isasymSum}i\ {\isasymin}\ set\ as{\isachardot}{\kern0pt}\ c\ i\ {\isacharcircum}{\kern0pt}\ {\isadigit{4}}\ {\isacharasterisk}{\kern0pt}\ {\isadigit{0}}{\isacharparenright}{\kern0pt}{\isachardoublequoteclose}\isanewline
\ \ \ \ \isacommand{apply}\isamarkupfalse%
\ {\isacharparenleft}{\kern0pt}rule\ add{\isacharunderscore}{\kern0pt}mono{\isacharparenright}{\kern0pt}\isanewline
\ \ \ \ \ \isacommand{apply}\isamarkupfalse%
\ {\isacharparenleft}{\kern0pt}simp\ add{\isacharcolon}{\kern0pt}power{\isacharunderscore}{\kern0pt}mult{\isacharunderscore}{\kern0pt}distrib\ sum{\isacharunderscore}{\kern0pt}distrib{\isacharunderscore}{\kern0pt}right{\isacharbrackleft}{\kern0pt}symmetric{\isacharbrackright}{\kern0pt}{\isacharparenright}{\kern0pt}\isanewline
\ \ \ \ \isacommand{apply}\isamarkupfalse%
\ {\isacharparenleft}{\kern0pt}rule\ sum{\isacharunderscore}{\kern0pt}mono{\isacharcomma}{\kern0pt}\ rule\ mult{\isacharunderscore}{\kern0pt}left{\isacharunderscore}{\kern0pt}mono{\isacharparenright}{\kern0pt}\isanewline
\ \ \ \ \isacommand{using}\isamarkupfalse%
\ r{\isacharunderscore}{\kern0pt}four{\isacharunderscore}{\kern0pt}est\ \isacommand{by}\isamarkupfalse%
\ simp{\isacharplus}{\kern0pt}\isanewline
\ \ \isacommand{also}\isamarkupfalse%
\ \isacommand{have}\isamarkupfalse%
\ {\isachardoublequoteopen}{\isachardot}{\kern0pt}{\isachardot}{\kern0pt}{\isachardot}{\kern0pt}\ {\isacharequal}{\kern0pt}\ {\isadigit{3}}\ {\isacharasterisk}{\kern0pt}\ {\isacharparenleft}{\kern0pt}real{\isacharunderscore}{\kern0pt}of{\isacharunderscore}{\kern0pt}rat\ {\isacharparenleft}{\kern0pt}F\ {\isadigit{2}}\ as{\isacharparenright}{\kern0pt}{\isacharcircum}{\kern0pt}{\isadigit{2}}{\isacharparenright}{\kern0pt}\ {\isacharasterisk}{\kern0pt}\ {\isacharparenleft}{\kern0pt}{\isacharparenleft}{\kern0pt}real\ p{\isacharparenright}{\kern0pt}\isactrlsup {\isadigit{2}}{\isacharminus}{\kern0pt}{\isadigit{1}}{\isacharparenright}{\kern0pt}\isactrlsup {\isadigit{2}}{\isachardoublequoteclose}\ \isanewline
\ \ \ \ \isacommand{by}\isamarkupfalse%
\ {\isacharparenleft}{\kern0pt}simp\ add{\isacharcolon}{\kern0pt}c{\isacharunderscore}{\kern0pt}def\ r{\isacharunderscore}{\kern0pt}two\ F{\isacharunderscore}{\kern0pt}def\ of{\isacharunderscore}{\kern0pt}rat{\isacharunderscore}{\kern0pt}sum\ of{\isacharunderscore}{\kern0pt}rat{\isacharunderscore}{\kern0pt}power{\isacharparenright}{\kern0pt}\isanewline
\isanewline
\ \ \isacommand{finally}\isamarkupfalse%
\ \isacommand{have}\isamarkupfalse%
\ v{\isacharunderscore}{\kern0pt}{\isadigit{1}}{\isacharcolon}{\kern0pt}\ {\isachardoublequoteopen}expectation\ {\isacharparenleft}{\kern0pt}{\isasymlambda}x{\isachardot}{\kern0pt}\ {\isacharparenleft}{\kern0pt}f\ x{\isacharparenright}{\kern0pt}\isactrlsup {\isadigit{2}}{\isacharparenright}{\kern0pt}\ {\isasymle}\ {\isadigit{3}}\ {\isacharasterisk}{\kern0pt}\ {\isacharparenleft}{\kern0pt}real{\isacharunderscore}{\kern0pt}of{\isacharunderscore}{\kern0pt}rat\ {\isacharparenleft}{\kern0pt}F\ {\isadigit{2}}\ as{\isacharparenright}{\kern0pt}{\isacharcircum}{\kern0pt}{\isadigit{2}}{\isacharparenright}{\kern0pt}\ {\isacharasterisk}{\kern0pt}\ {\isacharparenleft}{\kern0pt}{\isacharparenleft}{\kern0pt}real\ p{\isacharparenright}{\kern0pt}\isactrlsup {\isadigit{2}}{\isacharminus}{\kern0pt}{\isadigit{1}}{\isacharparenright}{\kern0pt}\isactrlsup {\isadigit{2}}{\isachardoublequoteclose}\isanewline
\ \ \ \ \isacommand{by}\isamarkupfalse%
\ simp\isanewline
\ \ \isanewline
\ \ \isacommand{have}\isamarkupfalse%
\ {\isachardoublequoteopen}variance\ f\ {\isasymle}\ {\isadigit{2}}{\isacharasterisk}{\kern0pt}{\isacharparenleft}{\kern0pt}real{\isacharunderscore}{\kern0pt}of{\isacharunderscore}{\kern0pt}rat\ {\isacharparenleft}{\kern0pt}F\ {\isadigit{2}}\ as{\isacharparenright}{\kern0pt}{\isacharcircum}{\kern0pt}{\isadigit{2}}{\isacharparenright}{\kern0pt}\ {\isacharasterisk}{\kern0pt}\ {\isacharparenleft}{\kern0pt}{\isacharparenleft}{\kern0pt}real\ p{\isacharparenright}{\kern0pt}\isactrlsup {\isadigit{2}}{\isacharminus}{\kern0pt}{\isadigit{1}}{\isacharparenright}{\kern0pt}\isactrlsup {\isadigit{2}}{\isachardoublequoteclose}\isanewline
\ \ \ \ \isacommand{apply}\isamarkupfalse%
\ {\isacharparenleft}{\kern0pt}subst\ variance{\isacharunderscore}{\kern0pt}eq{\isacharbrackleft}{\kern0pt}OF\ int{\isacharunderscore}{\kern0pt}M\ int{\isacharunderscore}{\kern0pt}M{\isacharbrackright}{\kern0pt}{\isacharcomma}{\kern0pt}\ subst\ b{\isacharparenright}{\kern0pt}\isanewline
\ \ \ \ \isacommand{apply}\isamarkupfalse%
\ {\isacharparenleft}{\kern0pt}simp\ add{\isacharcolon}{\kern0pt}power{\isacharunderscore}{\kern0pt}mult{\isacharunderscore}{\kern0pt}distrib{\isacharparenright}{\kern0pt}\isanewline
\ \ \ \ \isacommand{using}\isamarkupfalse%
\ v{\isacharunderscore}{\kern0pt}{\isadigit{1}}\ \isacommand{by}\isamarkupfalse%
\ simp\isanewline
\isanewline
\ \ \isacommand{thus}\isamarkupfalse%
\ {\isacharquery}{\kern0pt}A\ \isacommand{by}\isamarkupfalse%
\ simp\isanewline
\isacommand{qed}\isamarkupfalse%
%
\endisatagproof
{\isafoldproof}%
%
\isadelimproof
\isanewline
%
\endisadelimproof
\isanewline
\isacommand{lemma}\isamarkupfalse%
\ f{\isadigit{2}}{\isacharunderscore}{\kern0pt}alg{\isacharunderscore}{\kern0pt}sketch{\isacharcolon}{\kern0pt}\isanewline
\ \ \isakeyword{fixes}\ n\ {\isacharcolon}{\kern0pt}{\isacharcolon}{\kern0pt}\ nat\isanewline
\ \ \isakeyword{fixes}\ as\ {\isacharcolon}{\kern0pt}{\isacharcolon}{\kern0pt}\ {\isachardoublequoteopen}nat\ list{\isachardoublequoteclose}\isanewline
\ \ \isakeyword{assumes}\ {\isachardoublequoteopen}{\isasymepsilon}\ {\isasymin}\ {\isacharbraceleft}{\kern0pt}{\isadigit{0}}{\isacharless}{\kern0pt}{\isachardot}{\kern0pt}{\isachardot}{\kern0pt}{\isacharless}{\kern0pt}{\isadigit{1}}{\isacharbraceright}{\kern0pt}{\isachardoublequoteclose}\isanewline
\ \ \isakeyword{assumes}\ {\isachardoublequoteopen}{\isasymdelta}\ {\isachargreater}{\kern0pt}\ {\isadigit{0}}{\isachardoublequoteclose}\isanewline
\ \ \isakeyword{defines}\ {\isachardoublequoteopen}s\isactrlsub {\isadigit{1}}\ {\isasymequiv}\ nat\ {\isasymlceil}{\isadigit{6}}\ {\isacharslash}{\kern0pt}\ {\isasymdelta}\isactrlsup {\isadigit{2}}{\isasymrceil}{\isachardoublequoteclose}\isanewline
\ \ \isakeyword{defines}\ {\isachardoublequoteopen}s\isactrlsub {\isadigit{2}}\ {\isasymequiv}\ nat\ {\isasymlceil}{\isacharminus}{\kern0pt}{\isacharparenleft}{\kern0pt}{\isadigit{1}}{\isadigit{8}}{\isacharasterisk}{\kern0pt}\ ln\ {\isacharparenleft}{\kern0pt}real{\isacharunderscore}{\kern0pt}of{\isacharunderscore}{\kern0pt}rat\ {\isasymepsilon}{\isacharparenright}{\kern0pt}{\isacharparenright}{\kern0pt}{\isasymrceil}{\isachardoublequoteclose}\isanewline
\ \ \isakeyword{defines}\ {\isachardoublequoteopen}p\ {\isasymequiv}\ find{\isacharunderscore}{\kern0pt}prime{\isacharunderscore}{\kern0pt}above\ {\isacharparenleft}{\kern0pt}max\ n\ {\isadigit{3}}{\isacharparenright}{\kern0pt}{\isachardoublequoteclose}\isanewline
\ \ \isakeyword{defines}\ {\isachardoublequoteopen}sketch\ {\isasymequiv}\ fold\ {\isacharparenleft}{\kern0pt}{\isasymlambda}a\ state{\isachardot}{\kern0pt}\ state\ {\isasymbind}\ f{\isadigit{2}}{\isacharunderscore}{\kern0pt}update\ a{\isacharparenright}{\kern0pt}\ as\ {\isacharparenleft}{\kern0pt}f{\isadigit{2}}{\isacharunderscore}{\kern0pt}init\ {\isasymdelta}\ {\isasymepsilon}\ n{\isacharparenright}{\kern0pt}{\isachardoublequoteclose}\isanewline
\ \ \isakeyword{defines}\ {\isachardoublequoteopen}{\isasymOmega}\ {\isasymequiv}\ prod{\isacharunderscore}{\kern0pt}pmf\ {\isacharparenleft}{\kern0pt}{\isacharbraceleft}{\kern0pt}{\isadigit{0}}{\isachardot}{\kern0pt}{\isachardot}{\kern0pt}{\isacharless}{\kern0pt}s\isactrlsub {\isadigit{1}}{\isacharbraceright}{\kern0pt}\ {\isasymtimes}\ {\isacharbraceleft}{\kern0pt}{\isadigit{0}}{\isachardot}{\kern0pt}{\isachardot}{\kern0pt}{\isacharless}{\kern0pt}s\isactrlsub {\isadigit{2}}{\isacharbraceright}{\kern0pt}{\isacharparenright}{\kern0pt}\ {\isacharparenleft}{\kern0pt}{\isasymlambda}{\isacharunderscore}{\kern0pt}{\isachardot}{\kern0pt}\ pmf{\isacharunderscore}{\kern0pt}of{\isacharunderscore}{\kern0pt}set\ {\isacharparenleft}{\kern0pt}bounded{\isacharunderscore}{\kern0pt}degree{\isacharunderscore}{\kern0pt}polynomials\ {\isacharparenleft}{\kern0pt}ZFact\ {\isacharparenleft}{\kern0pt}int\ p{\isacharparenright}{\kern0pt}{\isacharparenright}{\kern0pt}\ {\isadigit{4}}{\isacharparenright}{\kern0pt}{\isacharparenright}{\kern0pt}{\isachardoublequoteclose}\ \isanewline
\ \ \isakeyword{shows}\ {\isachardoublequoteopen}sketch\ {\isacharequal}{\kern0pt}\ {\isasymOmega}\ {\isasymbind}\ {\isacharparenleft}{\kern0pt}{\isasymlambda}h{\isachardot}{\kern0pt}\ return{\isacharunderscore}{\kern0pt}pmf\ {\isacharparenleft}{\kern0pt}s\isactrlsub {\isadigit{1}}{\isacharcomma}{\kern0pt}\ s\isactrlsub {\isadigit{2}}{\isacharcomma}{\kern0pt}\ p{\isacharcomma}{\kern0pt}\ h{\isacharcomma}{\kern0pt}\ \isanewline
\ \ \ \ \ \ {\isasymlambda}i\ {\isasymin}\ {\isacharbraceleft}{\kern0pt}{\isadigit{0}}{\isachardot}{\kern0pt}{\isachardot}{\kern0pt}{\isacharless}{\kern0pt}s\isactrlsub {\isadigit{1}}{\isacharbraceright}{\kern0pt}\ {\isasymtimes}\ {\isacharbraceleft}{\kern0pt}{\isadigit{0}}{\isachardot}{\kern0pt}{\isachardot}{\kern0pt}{\isacharless}{\kern0pt}s\isactrlsub {\isadigit{2}}{\isacharbraceright}{\kern0pt}{\isachardot}{\kern0pt}\ sum{\isacharunderscore}{\kern0pt}list\ {\isacharparenleft}{\kern0pt}map\ {\isacharparenleft}{\kern0pt}f{\isadigit{2}}{\isacharunderscore}{\kern0pt}hash\ p\ {\isacharparenleft}{\kern0pt}h\ i{\isacharparenright}{\kern0pt}{\isacharparenright}{\kern0pt}\ as{\isacharparenright}{\kern0pt}{\isacharparenright}{\kern0pt}{\isacharparenright}{\kern0pt}{\isachardoublequoteclose}\isanewline
%
\isadelimproof
%
\endisadelimproof
%
\isatagproof
\isacommand{proof}\isamarkupfalse%
\ {\isacharminus}{\kern0pt}\isanewline
\ \ \isacommand{define}\isamarkupfalse%
\ ys\ \isakeyword{where}\ {\isachardoublequoteopen}ys\ {\isacharequal}{\kern0pt}\ rev\ as{\isachardoublequoteclose}\isanewline
\ \ \isacommand{have}\isamarkupfalse%
\ b{\isacharcolon}{\kern0pt}{\isachardoublequoteopen}sketch\ {\isacharequal}{\kern0pt}\ foldr\ {\isacharparenleft}{\kern0pt}{\isasymlambda}x\ state{\isachardot}{\kern0pt}\ state\ {\isasymbind}\ f{\isadigit{2}}{\isacharunderscore}{\kern0pt}update\ x{\isacharparenright}{\kern0pt}\ ys\ {\isacharparenleft}{\kern0pt}f{\isadigit{2}}{\isacharunderscore}{\kern0pt}init\ {\isasymdelta}\ {\isasymepsilon}\ n{\isacharparenright}{\kern0pt}{\isachardoublequoteclose}\isanewline
\ \ \ \ \isacommand{by}\isamarkupfalse%
\ {\isacharparenleft}{\kern0pt}simp\ add{\isacharcolon}{\kern0pt}\ foldr{\isacharunderscore}{\kern0pt}conv{\isacharunderscore}{\kern0pt}fold\ ys{\isacharunderscore}{\kern0pt}def\ sketch{\isacharunderscore}{\kern0pt}def{\isacharparenright}{\kern0pt}\isanewline
\ \ \isacommand{also}\isamarkupfalse%
\ \isacommand{have}\isamarkupfalse%
\ {\isachardoublequoteopen}{\isachardot}{\kern0pt}{\isachardot}{\kern0pt}{\isachardot}{\kern0pt}\ {\isacharequal}{\kern0pt}\ {\isasymOmega}\ {\isasymbind}\ {\isacharparenleft}{\kern0pt}{\isasymlambda}h{\isachardot}{\kern0pt}\ return{\isacharunderscore}{\kern0pt}pmf\ {\isacharparenleft}{\kern0pt}s\isactrlsub {\isadigit{1}}{\isacharcomma}{\kern0pt}\ s\isactrlsub {\isadigit{2}}{\isacharcomma}{\kern0pt}\ p{\isacharcomma}{\kern0pt}\ h{\isacharcomma}{\kern0pt}\ \isanewline
\ \ \ \ \ \ {\isasymlambda}i\ {\isasymin}\ {\isacharbraceleft}{\kern0pt}{\isadigit{0}}{\isachardot}{\kern0pt}{\isachardot}{\kern0pt}{\isacharless}{\kern0pt}s\isactrlsub {\isadigit{1}}{\isacharbraceright}{\kern0pt}\ {\isasymtimes}\ {\isacharbraceleft}{\kern0pt}{\isadigit{0}}{\isachardot}{\kern0pt}{\isachardot}{\kern0pt}{\isacharless}{\kern0pt}s\isactrlsub {\isadigit{2}}{\isacharbraceright}{\kern0pt}{\isachardot}{\kern0pt}\ sum{\isacharunderscore}{\kern0pt}list\ {\isacharparenleft}{\kern0pt}map\ {\isacharparenleft}{\kern0pt}f{\isadigit{2}}{\isacharunderscore}{\kern0pt}hash\ p\ {\isacharparenleft}{\kern0pt}h\ i{\isacharparenright}{\kern0pt}{\isacharparenright}{\kern0pt}\ ys{\isacharparenright}{\kern0pt}{\isacharparenright}{\kern0pt}{\isacharparenright}{\kern0pt}{\isachardoublequoteclose}\isanewline
\ \ \isacommand{proof}\isamarkupfalse%
\ {\isacharparenleft}{\kern0pt}induction\ ys{\isacharparenright}{\kern0pt}\isanewline
\ \ \ \ \isacommand{case}\isamarkupfalse%
\ Nil\isanewline
\ \ \ \ \isacommand{then}\isamarkupfalse%
\ \isacommand{show}\isamarkupfalse%
\ {\isacharquery}{\kern0pt}case\ \isanewline
\ \ \ \ \ \ \isacommand{by}\isamarkupfalse%
\ {\isacharparenleft}{\kern0pt}simp\ add{\isacharcolon}{\kern0pt}s\isactrlsub {\isadigit{1}}{\isacharunderscore}{\kern0pt}def\ {\isacharbrackleft}{\kern0pt}symmetric{\isacharbrackright}{\kern0pt}\ s\isactrlsub {\isadigit{2}}{\isacharunderscore}{\kern0pt}def{\isacharbrackleft}{\kern0pt}symmetric{\isacharbrackright}{\kern0pt}\ p{\isacharunderscore}{\kern0pt}def{\isacharbrackleft}{\kern0pt}symmetric{\isacharbrackright}{\kern0pt}\ {\isasymOmega}{\isacharunderscore}{\kern0pt}def\ restrict{\isacharunderscore}{\kern0pt}def{\isacharparenright}{\kern0pt}\ \isanewline
\ \ \isacommand{next}\isamarkupfalse%
\isanewline
\ \ \ \ \isacommand{case}\isamarkupfalse%
\ {\isacharparenleft}{\kern0pt}Cons\ a\ as{\isacharparenright}{\kern0pt}\isanewline
\ \ \ \ \isacommand{have}\isamarkupfalse%
\ a{\isacharcolon}{\kern0pt}{\isachardoublequoteopen}f{\isadigit{2}}{\isacharunderscore}{\kern0pt}update\ a\ {\isacharequal}{\kern0pt}\ {\isacharparenleft}{\kern0pt}{\isasymlambda}x{\isachardot}{\kern0pt}\ f{\isadigit{2}}{\isacharunderscore}{\kern0pt}update\ a\ {\isacharparenleft}{\kern0pt}fst\ x{\isacharcomma}{\kern0pt}\ fst\ {\isacharparenleft}{\kern0pt}snd\ x{\isacharparenright}{\kern0pt}{\isacharcomma}{\kern0pt}\ fst\ {\isacharparenleft}{\kern0pt}snd\ {\isacharparenleft}{\kern0pt}snd\ x{\isacharparenright}{\kern0pt}{\isacharparenright}{\kern0pt}{\isacharcomma}{\kern0pt}\ fst\ {\isacharparenleft}{\kern0pt}snd\ {\isacharparenleft}{\kern0pt}snd\ {\isacharparenleft}{\kern0pt}snd\ x{\isacharparenright}{\kern0pt}{\isacharparenright}{\kern0pt}{\isacharparenright}{\kern0pt}{\isacharcomma}{\kern0pt}\ \isanewline
\ \ \ \ \ \ \ \ snd\ {\isacharparenleft}{\kern0pt}snd\ {\isacharparenleft}{\kern0pt}snd\ {\isacharparenleft}{\kern0pt}snd\ x{\isacharparenright}{\kern0pt}{\isacharparenright}{\kern0pt}{\isacharparenright}{\kern0pt}{\isacharparenright}{\kern0pt}{\isacharparenright}{\kern0pt}{\isachardoublequoteclose}\ \isacommand{by}\isamarkupfalse%
\ simp\isanewline
\ \ \ \ \isacommand{show}\isamarkupfalse%
\ {\isacharquery}{\kern0pt}case\isanewline
\ \ \ \ \ \ \isacommand{using}\isamarkupfalse%
\ Cons\ \isacommand{apply}\isamarkupfalse%
\ {\isacharparenleft}{\kern0pt}simp\ del{\isacharcolon}{\kern0pt}f{\isadigit{2}}{\isacharunderscore}{\kern0pt}hash{\isachardot}{\kern0pt}simps\ f{\isadigit{2}}{\isacharunderscore}{\kern0pt}init{\isachardot}{\kern0pt}simps{\isacharparenright}{\kern0pt}\isanewline
\ \ \ \ \ \ \isacommand{apply}\isamarkupfalse%
\ {\isacharparenleft}{\kern0pt}subst\ a{\isacharparenright}{\kern0pt}\isanewline
\ \ \ \ \ \ \isacommand{apply}\isamarkupfalse%
\ {\isacharparenleft}{\kern0pt}subst\ bind{\isacharunderscore}{\kern0pt}assoc{\isacharunderscore}{\kern0pt}pmf{\isacharparenright}{\kern0pt}\isanewline
\ \ \ \ \ \ \isacommand{apply}\isamarkupfalse%
\ {\isacharparenleft}{\kern0pt}subst\ bind{\isacharunderscore}{\kern0pt}return{\isacharunderscore}{\kern0pt}pmf{\isacharparenright}{\kern0pt}\isanewline
\ \ \ \ \ \ \isacommand{by}\isamarkupfalse%
\ {\isacharparenleft}{\kern0pt}simp\ add{\isacharcolon}{\kern0pt}restrict{\isacharunderscore}{\kern0pt}def\ del{\isacharcolon}{\kern0pt}f{\isadigit{2}}{\isacharunderscore}{\kern0pt}hash{\isachardot}{\kern0pt}simps\ f{\isadigit{2}}{\isacharunderscore}{\kern0pt}init{\isachardot}{\kern0pt}simps\ cong{\isacharcolon}{\kern0pt}restrict{\isacharunderscore}{\kern0pt}cong{\isacharparenright}{\kern0pt}\isanewline
\ \ \isacommand{qed}\isamarkupfalse%
\isanewline
\ \ \isacommand{also}\isamarkupfalse%
\ \isacommand{have}\isamarkupfalse%
\ {\isachardoublequoteopen}{\isachardot}{\kern0pt}{\isachardot}{\kern0pt}{\isachardot}{\kern0pt}\ {\isacharequal}{\kern0pt}\ {\isasymOmega}\ {\isasymbind}\ {\isacharparenleft}{\kern0pt}{\isasymlambda}h{\isachardot}{\kern0pt}\ return{\isacharunderscore}{\kern0pt}pmf\ {\isacharparenleft}{\kern0pt}s\isactrlsub {\isadigit{1}}{\isacharcomma}{\kern0pt}\ s\isactrlsub {\isadigit{2}}{\isacharcomma}{\kern0pt}\ p{\isacharcomma}{\kern0pt}\ h{\isacharcomma}{\kern0pt}\ \isanewline
\ \ \ \ \ \ {\isasymlambda}i\ {\isasymin}\ {\isacharbraceleft}{\kern0pt}{\isadigit{0}}{\isachardot}{\kern0pt}{\isachardot}{\kern0pt}{\isacharless}{\kern0pt}s\isactrlsub {\isadigit{1}}{\isacharbraceright}{\kern0pt}\ {\isasymtimes}\ {\isacharbraceleft}{\kern0pt}{\isadigit{0}}{\isachardot}{\kern0pt}{\isachardot}{\kern0pt}{\isacharless}{\kern0pt}s\isactrlsub {\isadigit{2}}{\isacharbraceright}{\kern0pt}{\isachardot}{\kern0pt}\ sum{\isacharunderscore}{\kern0pt}list\ {\isacharparenleft}{\kern0pt}map\ {\isacharparenleft}{\kern0pt}f{\isadigit{2}}{\isacharunderscore}{\kern0pt}hash\ p\ {\isacharparenleft}{\kern0pt}h\ i{\isacharparenright}{\kern0pt}{\isacharparenright}{\kern0pt}\ as{\isacharparenright}{\kern0pt}{\isacharparenright}{\kern0pt}{\isacharparenright}{\kern0pt}{\isachardoublequoteclose}\isanewline
\ \ \ \ \isacommand{by}\isamarkupfalse%
\ {\isacharparenleft}{\kern0pt}simp\ add{\isacharcolon}{\kern0pt}\ ys{\isacharunderscore}{\kern0pt}def\ rev{\isacharunderscore}{\kern0pt}map{\isacharbrackleft}{\kern0pt}symmetric{\isacharbrackright}{\kern0pt}{\isacharparenright}{\kern0pt}\isanewline
\ \ \isacommand{finally}\isamarkupfalse%
\ \isacommand{show}\isamarkupfalse%
\ {\isacharquery}{\kern0pt}thesis\ \isacommand{by}\isamarkupfalse%
\ auto\isanewline
\isacommand{qed}\isamarkupfalse%
%
\endisatagproof
{\isafoldproof}%
%
\isadelimproof
\isanewline
%
\endisadelimproof
\isanewline
\isacommand{theorem}\isamarkupfalse%
\ f{\isadigit{2}}{\isacharunderscore}{\kern0pt}alg{\isacharunderscore}{\kern0pt}correct{\isacharcolon}{\kern0pt}\isanewline
\ \ \isakeyword{assumes}\ {\isachardoublequoteopen}{\isasymepsilon}\ {\isasymin}\ {\isacharbraceleft}{\kern0pt}{\isadigit{0}}{\isacharless}{\kern0pt}{\isachardot}{\kern0pt}{\isachardot}{\kern0pt}{\isacharless}{\kern0pt}{\isadigit{1}}{\isacharbraceright}{\kern0pt}{\isachardoublequoteclose}\isanewline
\ \ \isakeyword{assumes}\ {\isachardoublequoteopen}{\isasymdelta}\ {\isachargreater}{\kern0pt}\ {\isadigit{0}}{\isachardoublequoteclose}\isanewline
\ \ \isakeyword{assumes}\ {\isachardoublequoteopen}set\ as\ {\isasymsubseteq}\ {\isacharbraceleft}{\kern0pt}{\isadigit{0}}{\isachardot}{\kern0pt}{\isachardot}{\kern0pt}{\isacharless}{\kern0pt}n{\isacharbraceright}{\kern0pt}{\isachardoublequoteclose}\isanewline
\ \ \isakeyword{defines}\ {\isachardoublequoteopen}M\ {\isasymequiv}\ fold\ {\isacharparenleft}{\kern0pt}{\isasymlambda}a\ state{\isachardot}{\kern0pt}\ state\ {\isasymbind}\ f{\isadigit{2}}{\isacharunderscore}{\kern0pt}update\ a{\isacharparenright}{\kern0pt}\ as\ {\isacharparenleft}{\kern0pt}f{\isadigit{2}}{\isacharunderscore}{\kern0pt}init\ {\isasymdelta}\ {\isasymepsilon}\ n{\isacharparenright}{\kern0pt}\ {\isasymbind}\ f{\isadigit{2}}{\isacharunderscore}{\kern0pt}result{\isachardoublequoteclose}\isanewline
\ \ \isakeyword{shows}\ {\isachardoublequoteopen}{\isasymP}{\isacharparenleft}{\kern0pt}{\isasymomega}\ in\ measure{\isacharunderscore}{\kern0pt}pmf\ M{\isachardot}{\kern0pt}\ {\isasymbar}{\isasymomega}\ {\isacharminus}{\kern0pt}\ F\ {\isadigit{2}}\ as{\isasymbar}\ {\isasymle}\ {\isasymdelta}\ {\isacharasterisk}{\kern0pt}\ F\ {\isadigit{2}}\ as{\isacharparenright}{\kern0pt}\ {\isasymge}\ {\isadigit{1}}{\isacharminus}{\kern0pt}of{\isacharunderscore}{\kern0pt}rat\ {\isasymepsilon}{\isachardoublequoteclose}\isanewline
%
\isadelimproof
%
\endisadelimproof
%
\isatagproof
\isacommand{proof}\isamarkupfalse%
\ {\isacharminus}{\kern0pt}\isanewline
\ \ \isacommand{define}\isamarkupfalse%
\ s\isactrlsub {\isadigit{1}}\ \isakeyword{where}\ {\isachardoublequoteopen}s\isactrlsub {\isadigit{1}}\ {\isacharequal}{\kern0pt}\ nat\ {\isasymlceil}{\isadigit{6}}\ {\isacharslash}{\kern0pt}\ {\isasymdelta}\isactrlsup {\isadigit{2}}{\isasymrceil}{\isachardoublequoteclose}\isanewline
\ \ \isacommand{define}\isamarkupfalse%
\ s\isactrlsub {\isadigit{2}}\ \isakeyword{where}\ {\isachardoublequoteopen}s\isactrlsub {\isadigit{2}}\ {\isacharequal}{\kern0pt}\ nat\ {\isasymlceil}{\isacharminus}{\kern0pt}{\isacharparenleft}{\kern0pt}{\isadigit{1}}{\isadigit{8}}{\isacharasterisk}{\kern0pt}\ ln\ {\isacharparenleft}{\kern0pt}real{\isacharunderscore}{\kern0pt}of{\isacharunderscore}{\kern0pt}rat\ {\isasymepsilon}{\isacharparenright}{\kern0pt}{\isacharparenright}{\kern0pt}{\isasymrceil}{\isachardoublequoteclose}\isanewline
\ \ \isacommand{define}\isamarkupfalse%
\ p\ \isakeyword{where}\ {\isachardoublequoteopen}p\ {\isacharequal}{\kern0pt}\ find{\isacharunderscore}{\kern0pt}prime{\isacharunderscore}{\kern0pt}above\ {\isacharparenleft}{\kern0pt}max\ n\ {\isadigit{3}}{\isacharparenright}{\kern0pt}{\isachardoublequoteclose}\isanewline
\ \ \isacommand{define}\isamarkupfalse%
\ {\isasymOmega}\isactrlsub {\isadigit{0}}\ \isakeyword{where}\ {\isachardoublequoteopen}{\isasymOmega}\isactrlsub {\isadigit{0}}\ {\isacharequal}{\kern0pt}\ \isanewline
\ \ \ \ prod{\isacharunderscore}{\kern0pt}pmf\ {\isacharparenleft}{\kern0pt}{\isacharbraceleft}{\kern0pt}{\isadigit{0}}{\isachardot}{\kern0pt}{\isachardot}{\kern0pt}{\isacharless}{\kern0pt}s\isactrlsub {\isadigit{1}}{\isacharbraceright}{\kern0pt}\ {\isasymtimes}\ {\isacharbraceleft}{\kern0pt}{\isadigit{0}}{\isachardot}{\kern0pt}{\isachardot}{\kern0pt}{\isacharless}{\kern0pt}s\isactrlsub {\isadigit{2}}{\isacharbraceright}{\kern0pt}{\isacharparenright}{\kern0pt}\ {\isacharparenleft}{\kern0pt}{\isasymlambda}{\isacharunderscore}{\kern0pt}{\isachardot}{\kern0pt}\ pmf{\isacharunderscore}{\kern0pt}of{\isacharunderscore}{\kern0pt}set\ {\isacharparenleft}{\kern0pt}bounded{\isacharunderscore}{\kern0pt}degree{\isacharunderscore}{\kern0pt}polynomials\ {\isacharparenleft}{\kern0pt}ZFact\ {\isacharparenleft}{\kern0pt}int\ p{\isacharparenright}{\kern0pt}{\isacharparenright}{\kern0pt}\ {\isadigit{4}}{\isacharparenright}{\kern0pt}{\isacharparenright}{\kern0pt}{\isachardoublequoteclose}\isanewline
\isanewline
\ \ \isacommand{define}\isamarkupfalse%
\ s\isactrlsub {\isadigit{1}}{\isacharunderscore}{\kern0pt}from\ {\isacharcolon}{\kern0pt}{\isacharcolon}{\kern0pt}\ {\isachardoublequoteopen}f{\isadigit{2}}{\isacharunderscore}{\kern0pt}state\ {\isasymRightarrow}\ nat{\isachardoublequoteclose}\ \isakeyword{where}\ {\isachardoublequoteopen}s\isactrlsub {\isadigit{1}}{\isacharunderscore}{\kern0pt}from\ {\isacharequal}{\kern0pt}\ fst{\isachardoublequoteclose}\isanewline
\ \ \isacommand{define}\isamarkupfalse%
\ s\isactrlsub {\isadigit{2}}{\isacharunderscore}{\kern0pt}from\ {\isacharcolon}{\kern0pt}{\isacharcolon}{\kern0pt}\ {\isachardoublequoteopen}f{\isadigit{2}}{\isacharunderscore}{\kern0pt}state\ {\isasymRightarrow}\ nat{\isachardoublequoteclose}\ \isakeyword{where}\ {\isachardoublequoteopen}s\isactrlsub {\isadigit{2}}{\isacharunderscore}{\kern0pt}from\ {\isacharequal}{\kern0pt}\ fst\ {\isasymcirc}\ snd{\isachardoublequoteclose}\isanewline
\ \ \isacommand{define}\isamarkupfalse%
\ p{\isacharunderscore}{\kern0pt}from\ {\isacharcolon}{\kern0pt}{\isacharcolon}{\kern0pt}\ {\isachardoublequoteopen}f{\isadigit{2}}{\isacharunderscore}{\kern0pt}state\ {\isasymRightarrow}\ nat{\isachardoublequoteclose}\ \isakeyword{where}\ {\isachardoublequoteopen}p{\isacharunderscore}{\kern0pt}from\ {\isacharequal}{\kern0pt}\ fst\ {\isasymcirc}\ snd\ {\isasymcirc}\ snd{\isachardoublequoteclose}\isanewline
\ \ \isacommand{define}\isamarkupfalse%
\ h{\isacharunderscore}{\kern0pt}from\ {\isacharcolon}{\kern0pt}{\isacharcolon}{\kern0pt}\ {\isachardoublequoteopen}f{\isadigit{2}}{\isacharunderscore}{\kern0pt}state\ {\isasymRightarrow}\ {\isacharparenleft}{\kern0pt}nat\ {\isasymtimes}\ nat\ {\isasymRightarrow}\ int\ set\ list{\isacharparenright}{\kern0pt}{\isachardoublequoteclose}\ \isakeyword{where}\ {\isachardoublequoteopen}h{\isacharunderscore}{\kern0pt}from\ {\isacharequal}{\kern0pt}\ fst\ {\isasymcirc}\ snd\ {\isasymcirc}\ snd\ {\isasymcirc}\ snd{\isachardoublequoteclose}\isanewline
\ \ \isacommand{define}\isamarkupfalse%
\ sketch{\isacharunderscore}{\kern0pt}from\ {\isacharcolon}{\kern0pt}{\isacharcolon}{\kern0pt}\ {\isachardoublequoteopen}f{\isadigit{2}}{\isacharunderscore}{\kern0pt}state\ {\isasymRightarrow}\ {\isacharparenleft}{\kern0pt}nat\ {\isasymtimes}\ nat\ {\isasymRightarrow}\ int{\isacharparenright}{\kern0pt}{\isachardoublequoteclose}\ \isakeyword{where}\ {\isachardoublequoteopen}sketch{\isacharunderscore}{\kern0pt}from\ {\isacharequal}{\kern0pt}\ snd\ {\isasymcirc}\ snd\ {\isasymcirc}\ snd\ {\isasymcirc}\ snd{\isachardoublequoteclose}\isanewline
\isanewline
\ \ \isacommand{have}\isamarkupfalse%
\ p{\isacharunderscore}{\kern0pt}prime{\isacharcolon}{\kern0pt}\ {\isachardoublequoteopen}Factorial{\isacharunderscore}{\kern0pt}Ring{\isachardot}{\kern0pt}prime\ p{\isachardoublequoteclose}\ \isanewline
\ \ \ \ \isacommand{apply}\isamarkupfalse%
\ {\isacharparenleft}{\kern0pt}simp\ add{\isacharcolon}{\kern0pt}p{\isacharunderscore}{\kern0pt}def{\isacharparenright}{\kern0pt}\ \isanewline
\ \ \ \ \isacommand{using}\isamarkupfalse%
\ find{\isacharunderscore}{\kern0pt}prime{\isacharunderscore}{\kern0pt}above{\isacharunderscore}{\kern0pt}is{\isacharunderscore}{\kern0pt}prime\ \isacommand{by}\isamarkupfalse%
\ blast\isanewline
\isanewline
\ \ \isacommand{have}\isamarkupfalse%
\ p{\isacharunderscore}{\kern0pt}ge{\isacharunderscore}{\kern0pt}{\isadigit{3}}{\isacharcolon}{\kern0pt}\ {\isachardoublequoteopen}p\ {\isasymge}\ {\isadigit{3}}{\isachardoublequoteclose}\isanewline
\ \ \ \ \isacommand{apply}\isamarkupfalse%
\ {\isacharparenleft}{\kern0pt}simp\ add{\isacharcolon}{\kern0pt}p{\isacharunderscore}{\kern0pt}def{\isacharparenright}{\kern0pt}\isanewline
\ \ \ \ \isacommand{by}\isamarkupfalse%
\ {\isacharparenleft}{\kern0pt}meson\ find{\isacharunderscore}{\kern0pt}prime{\isacharunderscore}{\kern0pt}above{\isacharunderscore}{\kern0pt}lower{\isacharunderscore}{\kern0pt}bound\ dual{\isacharunderscore}{\kern0pt}order{\isachardot}{\kern0pt}trans\ max{\isachardot}{\kern0pt}cobounded{\isadigit{2}}{\isacharparenright}{\kern0pt}\isanewline
\isanewline
\ \ \isacommand{hence}\isamarkupfalse%
\ p{\isacharunderscore}{\kern0pt}ge{\isacharunderscore}{\kern0pt}{\isadigit{2}}{\isacharcolon}{\kern0pt}\ {\isachardoublequoteopen}p\ {\isachargreater}{\kern0pt}\ {\isadigit{2}}{\isachardoublequoteclose}\ \isacommand{by}\isamarkupfalse%
\ simp\isanewline
\isanewline
\ \ \isacommand{hence}\isamarkupfalse%
\ p{\isacharunderscore}{\kern0pt}sq{\isacharunderscore}{\kern0pt}ne{\isacharunderscore}{\kern0pt}{\isadigit{1}}{\isacharcolon}{\kern0pt}\ {\isachardoublequoteopen}{\isacharparenleft}{\kern0pt}real\ p{\isacharparenright}{\kern0pt}{\isacharcircum}{\kern0pt}{\isadigit{2}}\ {\isasymnoteq}\ {\isadigit{1}}{\isachardoublequoteclose}\ \isanewline
\ \ \ \ \isacommand{by}\isamarkupfalse%
\ {\isacharparenleft}{\kern0pt}metis\ Num{\isachardot}{\kern0pt}of{\isacharunderscore}{\kern0pt}nat{\isacharunderscore}{\kern0pt}simps{\isacharparenleft}{\kern0pt}{\isadigit{2}}{\isacharparenright}{\kern0pt}\ nat{\isacharunderscore}{\kern0pt}{\isadigit{1}}\ nat{\isacharunderscore}{\kern0pt}one{\isacharunderscore}{\kern0pt}as{\isacharunderscore}{\kern0pt}int\ nat{\isacharunderscore}{\kern0pt}power{\isacharunderscore}{\kern0pt}eq{\isacharunderscore}{\kern0pt}Suc{\isacharunderscore}{\kern0pt}{\isadigit{0}}{\isacharunderscore}{\kern0pt}iff\ not{\isacharunderscore}{\kern0pt}numeral{\isacharunderscore}{\kern0pt}less{\isacharunderscore}{\kern0pt}one\ of{\isacharunderscore}{\kern0pt}nat{\isacharunderscore}{\kern0pt}eq{\isacharunderscore}{\kern0pt}iff\ of{\isacharunderscore}{\kern0pt}nat{\isacharunderscore}{\kern0pt}power\ zero{\isacharunderscore}{\kern0pt}neq{\isacharunderscore}{\kern0pt}numeral{\isacharparenright}{\kern0pt}\isanewline
\isanewline
\ \ \isacommand{have}\isamarkupfalse%
\ p{\isacharunderscore}{\kern0pt}ge{\isacharunderscore}{\kern0pt}{\isadigit{0}}{\isacharcolon}{\kern0pt}\ {\isachardoublequoteopen}p\ {\isachargreater}{\kern0pt}\ {\isadigit{0}}{\isachardoublequoteclose}\ \isacommand{using}\isamarkupfalse%
\ p{\isacharunderscore}{\kern0pt}ge{\isacharunderscore}{\kern0pt}{\isadigit{2}}\ \isacommand{by}\isamarkupfalse%
\ simp\isanewline
\isanewline
\ \ \isacommand{have}\isamarkupfalse%
\ fin{\isacharunderscore}{\kern0pt}omega{\isacharunderscore}{\kern0pt}{\isadigit{2}}{\isacharcolon}{\kern0pt}\ {\isachardoublequoteopen}finite\ {\isacharparenleft}{\kern0pt}set{\isacharunderscore}{\kern0pt}pmf\ {\isacharparenleft}{\kern0pt}\ pmf{\isacharunderscore}{\kern0pt}of{\isacharunderscore}{\kern0pt}set\ {\isacharparenleft}{\kern0pt}bounded{\isacharunderscore}{\kern0pt}degree{\isacharunderscore}{\kern0pt}polynomials\ {\isacharparenleft}{\kern0pt}ZFact\ {\isacharparenleft}{\kern0pt}int\ p{\isacharparenright}{\kern0pt}{\isacharparenright}{\kern0pt}\ {\isadigit{4}}{\isacharparenright}{\kern0pt}{\isacharparenright}{\kern0pt}{\isacharparenright}{\kern0pt}{\isachardoublequoteclose}\isanewline
\ \ \ \ \isacommand{by}\isamarkupfalse%
\ {\isacharparenleft}{\kern0pt}metis\ fin{\isacharunderscore}{\kern0pt}bounded{\isacharunderscore}{\kern0pt}degree{\isacharunderscore}{\kern0pt}polynomials{\isacharbrackleft}{\kern0pt}OF\ p{\isacharunderscore}{\kern0pt}ge{\isacharunderscore}{\kern0pt}{\isadigit{0}}{\isacharbrackright}{\kern0pt}\ ne{\isacharunderscore}{\kern0pt}bounded{\isacharunderscore}{\kern0pt}degree{\isacharunderscore}{\kern0pt}polynomials\ set{\isacharunderscore}{\kern0pt}pmf{\isacharunderscore}{\kern0pt}of{\isacharunderscore}{\kern0pt}set{\isacharparenright}{\kern0pt}\isanewline
\isanewline
\ \ \isacommand{have}\isamarkupfalse%
\ fin{\isacharunderscore}{\kern0pt}omega{\isacharunderscore}{\kern0pt}{\isadigit{1}}{\isacharcolon}{\kern0pt}\ {\isachardoublequoteopen}finite\ {\isacharparenleft}{\kern0pt}set{\isacharunderscore}{\kern0pt}pmf\ {\isasymOmega}\isactrlsub {\isadigit{0}}{\isacharparenright}{\kern0pt}{\isachardoublequoteclose}\isanewline
\ \ \ \ \isacommand{apply}\isamarkupfalse%
\ {\isacharparenleft}{\kern0pt}simp\ add{\isacharcolon}{\kern0pt}{\isasymOmega}\isactrlsub {\isadigit{0}}{\isacharunderscore}{\kern0pt}def\ set{\isacharunderscore}{\kern0pt}prod{\isacharunderscore}{\kern0pt}pmf{\isacharparenright}{\kern0pt}\isanewline
\ \ \ \ \isacommand{apply}\isamarkupfalse%
\ {\isacharparenleft}{\kern0pt}rule\ finite{\isacharunderscore}{\kern0pt}PiE{\isacharcomma}{\kern0pt}\ simp{\isacharparenright}{\kern0pt}\isanewline
\ \ \ \ \isacommand{by}\isamarkupfalse%
\ {\isacharparenleft}{\kern0pt}metis\ fin{\isacharunderscore}{\kern0pt}omega{\isacharunderscore}{\kern0pt}{\isadigit{2}}{\isacharparenright}{\kern0pt}\isanewline
\isanewline
\ \ \isacommand{have}\isamarkupfalse%
\ as{\isacharunderscore}{\kern0pt}le{\isacharunderscore}{\kern0pt}p{\isacharcolon}{\kern0pt}\ {\isachardoublequoteopen}{\isasymAnd}x{\isachardot}{\kern0pt}\ x\ {\isasymin}\ set\ as\ {\isasymLongrightarrow}\ x\ {\isacharless}{\kern0pt}\ p{\isachardoublequoteclose}\ \isanewline
\ \ \ \ \isacommand{apply}\isamarkupfalse%
\ {\isacharparenleft}{\kern0pt}rule\ order{\isacharunderscore}{\kern0pt}less{\isacharunderscore}{\kern0pt}le{\isacharunderscore}{\kern0pt}trans{\isacharbrackleft}{\kern0pt}\isakeyword{where}\ y{\isacharequal}{\kern0pt}{\isachardoublequoteopen}n{\isachardoublequoteclose}{\isacharbrackright}{\kern0pt}{\isacharparenright}{\kern0pt}\isanewline
\ \ \ \ \ \isacommand{using}\isamarkupfalse%
\ assms{\isacharparenleft}{\kern0pt}{\isadigit{3}}{\isacharparenright}{\kern0pt}\ atLeastLessThan{\isacharunderscore}{\kern0pt}iff\ \isacommand{apply}\isamarkupfalse%
\ blast\isanewline
\ \ \ \ \isacommand{apply}\isamarkupfalse%
\ {\isacharparenleft}{\kern0pt}simp\ add{\isacharcolon}{\kern0pt}p{\isacharunderscore}{\kern0pt}def{\isacharparenright}{\kern0pt}\ \isanewline
\ \ \ \ \isacommand{by}\isamarkupfalse%
\ {\isacharparenleft}{\kern0pt}meson\ find{\isacharunderscore}{\kern0pt}prime{\isacharunderscore}{\kern0pt}above{\isacharunderscore}{\kern0pt}lower{\isacharunderscore}{\kern0pt}bound\ max{\isachardot}{\kern0pt}boundedE{\isacharparenright}{\kern0pt}\isanewline
\isanewline
\ \ \isacommand{have}\isamarkupfalse%
\ fin{\isacharunderscore}{\kern0pt}poly{\isacharprime}{\kern0pt}{\isacharcolon}{\kern0pt}\ {\isachardoublequoteopen}finite\ {\isacharparenleft}{\kern0pt}bounded{\isacharunderscore}{\kern0pt}degree{\isacharunderscore}{\kern0pt}polynomials\ {\isacharparenleft}{\kern0pt}ZFact\ {\isacharparenleft}{\kern0pt}int\ p{\isacharparenright}{\kern0pt}{\isacharparenright}{\kern0pt}\ {\isadigit{4}}{\isacharparenright}{\kern0pt}{\isachardoublequoteclose}\isanewline
\ \ \ \ \isacommand{apply}\isamarkupfalse%
\ {\isacharparenleft}{\kern0pt}rule\ fin{\isacharunderscore}{\kern0pt}bounded{\isacharunderscore}{\kern0pt}degree{\isacharunderscore}{\kern0pt}polynomials{\isacharparenright}{\kern0pt}\isanewline
\ \ \ \ \isacommand{using}\isamarkupfalse%
\ p{\isacharunderscore}{\kern0pt}ge{\isacharunderscore}{\kern0pt}{\isadigit{3}}\ \isacommand{by}\isamarkupfalse%
\ auto\isanewline
\isanewline
\ \ \isacommand{have}\isamarkupfalse%
\ s{\isadigit{2}}{\isacharunderscore}{\kern0pt}nonzero{\isacharcolon}{\kern0pt}\ {\isachardoublequoteopen}s\isactrlsub {\isadigit{2}}\ {\isachargreater}{\kern0pt}\ {\isadigit{0}}{\isachardoublequoteclose}\isanewline
\ \ \ \ \isacommand{using}\isamarkupfalse%
\ assms\ \isacommand{by}\isamarkupfalse%
\ {\isacharparenleft}{\kern0pt}simp\ add{\isacharcolon}{\kern0pt}s\isactrlsub {\isadigit{2}}{\isacharunderscore}{\kern0pt}def{\isacharparenright}{\kern0pt}\isanewline
\isanewline
\ \ \isacommand{have}\isamarkupfalse%
\ s{\isadigit{1}}{\isacharunderscore}{\kern0pt}nonzero{\isacharcolon}{\kern0pt}\ {\isachardoublequoteopen}s\isactrlsub {\isadigit{1}}\ {\isachargreater}{\kern0pt}\ {\isadigit{0}}{\isachardoublequoteclose}\ \ \isanewline
\ \ \ \ \isacommand{using}\isamarkupfalse%
\ assms\ \isacommand{by}\isamarkupfalse%
\ {\isacharparenleft}{\kern0pt}simp\ add{\isacharcolon}{\kern0pt}s\isactrlsub {\isadigit{1}}{\isacharunderscore}{\kern0pt}def{\isacharparenright}{\kern0pt}\isanewline
\isanewline
\ \ \isacommand{have}\isamarkupfalse%
\ split{\isacharunderscore}{\kern0pt}f{\isadigit{2}}{\isacharunderscore}{\kern0pt}space{\isacharcolon}{\kern0pt}\ {\isachardoublequoteopen}{\isasymAnd}x{\isachardot}{\kern0pt}\ x\ {\isacharequal}{\kern0pt}\ {\isacharparenleft}{\kern0pt}s\isactrlsub {\isadigit{1}}{\isacharunderscore}{\kern0pt}from\ x{\isacharcomma}{\kern0pt}\ s\isactrlsub {\isadigit{2}}{\isacharunderscore}{\kern0pt}from\ x{\isacharcomma}{\kern0pt}\ p{\isacharunderscore}{\kern0pt}from\ x{\isacharcomma}{\kern0pt}\ h{\isacharunderscore}{\kern0pt}from\ x{\isacharcomma}{\kern0pt}\ sketch{\isacharunderscore}{\kern0pt}from\ x{\isacharparenright}{\kern0pt}{\isachardoublequoteclose}\isanewline
\ \ \ \ \isacommand{by}\isamarkupfalse%
\ {\isacharparenleft}{\kern0pt}simp\ add{\isacharcolon}{\kern0pt}prod{\isacharunderscore}{\kern0pt}eq{\isacharunderscore}{\kern0pt}iff\ s\isactrlsub {\isadigit{1}}{\isacharunderscore}{\kern0pt}from{\isacharunderscore}{\kern0pt}def\ s\isactrlsub {\isadigit{2}}{\isacharunderscore}{\kern0pt}from{\isacharunderscore}{\kern0pt}def\ p{\isacharunderscore}{\kern0pt}from{\isacharunderscore}{\kern0pt}def\ h{\isacharunderscore}{\kern0pt}from{\isacharunderscore}{\kern0pt}def\ sketch{\isacharunderscore}{\kern0pt}from{\isacharunderscore}{\kern0pt}def{\isacharparenright}{\kern0pt}\isanewline
\isanewline
\ \ \isacommand{have}\isamarkupfalse%
\ f{\isadigit{2}}{\isacharunderscore}{\kern0pt}result{\isacharunderscore}{\kern0pt}conv{\isacharcolon}{\kern0pt}\ {\isachardoublequoteopen}f{\isadigit{2}}{\isacharunderscore}{\kern0pt}result\ {\isacharequal}{\kern0pt}\ {\isacharparenleft}{\kern0pt}{\isasymlambda}x{\isachardot}{\kern0pt}\ f{\isadigit{2}}{\isacharunderscore}{\kern0pt}result\ {\isacharparenleft}{\kern0pt}s\isactrlsub {\isadigit{1}}{\isacharunderscore}{\kern0pt}from\ x{\isacharcomma}{\kern0pt}\ s\isactrlsub {\isadigit{2}}{\isacharunderscore}{\kern0pt}from\ x{\isacharcomma}{\kern0pt}\ p{\isacharunderscore}{\kern0pt}from\ x{\isacharcomma}{\kern0pt}\ h{\isacharunderscore}{\kern0pt}from\ x{\isacharcomma}{\kern0pt}\ sketch{\isacharunderscore}{\kern0pt}from\ x{\isacharparenright}{\kern0pt}{\isacharparenright}{\kern0pt}{\isachardoublequoteclose}\isanewline
\ \ \ \ \isacommand{by}\isamarkupfalse%
\ {\isacharparenleft}{\kern0pt}simp\ add{\isacharcolon}{\kern0pt}split{\isacharunderscore}{\kern0pt}f{\isadigit{2}}{\isacharunderscore}{\kern0pt}space{\isacharbrackleft}{\kern0pt}symmetric{\isacharbrackright}{\kern0pt}\ del{\isacharcolon}{\kern0pt}f{\isadigit{2}}{\isacharunderscore}{\kern0pt}result{\isachardot}{\kern0pt}simps{\isacharparenright}{\kern0pt}\isanewline
\isanewline
\ \ \isacommand{define}\isamarkupfalse%
\ f\ \isakeyword{where}\ {\isachardoublequoteopen}f\ {\isacharequal}{\kern0pt}\ {\isacharparenleft}{\kern0pt}{\isasymlambda}x{\isachardot}{\kern0pt}\ median\isanewline
\ \ \ \ \ \ \ \ \ \ \ {\isacharparenleft}{\kern0pt}{\isasymlambda}i{\isasymin}{\isacharbraceleft}{\kern0pt}{\isadigit{0}}{\isachardot}{\kern0pt}{\isachardot}{\kern0pt}{\isacharless}{\kern0pt}s\isactrlsub {\isadigit{2}}{\isacharbraceright}{\kern0pt}{\isachardot}{\kern0pt}\isanewline
\ \ \ \ \ \ \ \ \ \ \ \ \ \ \ {\isacharparenleft}{\kern0pt}{\isasymSum}i\isactrlsub {\isadigit{1}}\ {\isacharequal}{\kern0pt}\ {\isadigit{0}}{\isachardot}{\kern0pt}{\isachardot}{\kern0pt}{\isacharless}{\kern0pt}s\isactrlsub {\isadigit{1}}{\isachardot}{\kern0pt}\ {\isacharparenleft}{\kern0pt}rat{\isacharunderscore}{\kern0pt}of{\isacharunderscore}{\kern0pt}int\ {\isacharparenleft}{\kern0pt}sum{\isacharunderscore}{\kern0pt}list\ {\isacharparenleft}{\kern0pt}map\ {\isacharparenleft}{\kern0pt}f{\isadigit{2}}{\isacharunderscore}{\kern0pt}hash\ p\ {\isacharparenleft}{\kern0pt}x\ {\isacharparenleft}{\kern0pt}i\isactrlsub {\isadigit{1}}{\isacharcomma}{\kern0pt}\ i{\isacharparenright}{\kern0pt}{\isacharparenright}{\kern0pt}{\isacharparenright}{\kern0pt}\ as{\isacharparenright}{\kern0pt}{\isacharparenright}{\kern0pt}{\isacharparenright}{\kern0pt}\isactrlsup {\isadigit{2}}{\isacharparenright}{\kern0pt}\ {\isacharslash}{\kern0pt}\isanewline
\ \ \ \ \ \ \ \ \ \ \ \ \ \ \ {\isacharparenleft}{\kern0pt}{\isacharparenleft}{\kern0pt}{\isacharparenleft}{\kern0pt}rat{\isacharunderscore}{\kern0pt}of{\isacharunderscore}{\kern0pt}nat\ p{\isacharparenright}{\kern0pt}\isactrlsup {\isadigit{2}}\ {\isacharminus}{\kern0pt}\ {\isadigit{1}}{\isacharparenright}{\kern0pt}\ {\isacharasterisk}{\kern0pt}\ rat{\isacharunderscore}{\kern0pt}of{\isacharunderscore}{\kern0pt}nat\ s\isactrlsub {\isadigit{1}}{\isacharparenright}{\kern0pt}{\isacharparenright}{\kern0pt}\isanewline
\ \ \ \ \ \ \ \ \ \ \ s\isactrlsub {\isadigit{2}}{\isacharparenright}{\kern0pt}{\isachardoublequoteclose}\isanewline
\isanewline
\ \ \isacommand{define}\isamarkupfalse%
\ f{\isadigit{3}}\ \isakeyword{where}\ \isanewline
\ \ \ \ {\isachardoublequoteopen}f{\isadigit{3}}\ {\isacharequal}{\kern0pt}\ {\isacharparenleft}{\kern0pt}{\isasymlambda}x\ {\isacharparenleft}{\kern0pt}i\isactrlsub {\isadigit{1}}{\isacharcolon}{\kern0pt}{\isacharcolon}{\kern0pt}nat{\isacharparenright}{\kern0pt}\ {\isacharparenleft}{\kern0pt}i\isactrlsub {\isadigit{2}}{\isacharcolon}{\kern0pt}{\isacharcolon}{\kern0pt}nat{\isacharparenright}{\kern0pt}{\isachardot}{\kern0pt}\ {\isacharparenleft}{\kern0pt}real{\isacharunderscore}{\kern0pt}of{\isacharunderscore}{\kern0pt}int\ {\isacharparenleft}{\kern0pt}sum{\isacharunderscore}{\kern0pt}list\ {\isacharparenleft}{\kern0pt}map\ {\isacharparenleft}{\kern0pt}f{\isadigit{2}}{\isacharunderscore}{\kern0pt}hash\ p\ {\isacharparenleft}{\kern0pt}x\ {\isacharparenleft}{\kern0pt}i\isactrlsub {\isadigit{1}}{\isacharcomma}{\kern0pt}\ i\isactrlsub {\isadigit{2}}{\isacharparenright}{\kern0pt}{\isacharparenright}{\kern0pt}{\isacharparenright}{\kern0pt}\ as{\isacharparenright}{\kern0pt}{\isacharparenright}{\kern0pt}{\isacharparenright}{\kern0pt}\isactrlsup {\isadigit{2}}{\isacharparenright}{\kern0pt}{\isachardoublequoteclose}\isanewline
\isanewline
\ \ \isacommand{define}\isamarkupfalse%
\ f{\isadigit{2}}\ \isakeyword{where}\ {\isachardoublequoteopen}f{\isadigit{2}}\ {\isacharequal}{\kern0pt}\ {\isacharparenleft}{\kern0pt}{\isasymlambda}x{\isachardot}{\kern0pt}\ {\isasymlambda}i{\isasymin}{\isacharbraceleft}{\kern0pt}{\isadigit{0}}{\isachardot}{\kern0pt}{\isachardot}{\kern0pt}{\isacharless}{\kern0pt}s\isactrlsub {\isadigit{2}}{\isacharbraceright}{\kern0pt}{\isachardot}{\kern0pt}\ {\isacharparenleft}{\kern0pt}{\isasymSum}i\isactrlsub {\isadigit{1}}\ {\isacharequal}{\kern0pt}\ {\isadigit{0}}{\isachardot}{\kern0pt}{\isachardot}{\kern0pt}{\isacharless}{\kern0pt}s\isactrlsub {\isadigit{1}}{\isachardot}{\kern0pt}\ f{\isadigit{3}}\ x\ i\isactrlsub {\isadigit{1}}\ i{\isacharparenright}{\kern0pt}\ {\isacharslash}{\kern0pt}\ {\isacharparenleft}{\kern0pt}{\isacharparenleft}{\kern0pt}{\isacharparenleft}{\kern0pt}real\ p{\isacharparenright}{\kern0pt}\isactrlsup {\isadigit{2}}\ {\isacharminus}{\kern0pt}\ {\isadigit{1}}{\isacharparenright}{\kern0pt}\ {\isacharasterisk}{\kern0pt}\ real\ s\isactrlsub {\isadigit{1}}{\isacharparenright}{\kern0pt}{\isacharparenright}{\kern0pt}{\isachardoublequoteclose}\isanewline
\ \ \isanewline
\ \ \isacommand{have}\isamarkupfalse%
\ f{\isadigit{2}}{\isacharunderscore}{\kern0pt}var{\isacharprime}{\kern0pt}{\isacharprime}{\kern0pt}{\isacharcolon}{\kern0pt}\ {\isachardoublequoteopen}{\isasymAnd}i{\isachardot}{\kern0pt}\ i\ {\isacharless}{\kern0pt}\ s\isactrlsub {\isadigit{2}}\ {\isasymLongrightarrow}\ prob{\isacharunderscore}{\kern0pt}space{\isachardot}{\kern0pt}variance\ {\isasymOmega}\isactrlsub {\isadigit{0}}\ {\isacharparenleft}{\kern0pt}{\isasymlambda}{\isasymomega}{\isachardot}{\kern0pt}\ f{\isadigit{2}}\ {\isasymomega}\ i{\isacharparenright}{\kern0pt}\ {\isasymle}\ {\isacharparenleft}{\kern0pt}real{\isacharunderscore}{\kern0pt}of{\isacharunderscore}{\kern0pt}rat\ {\isacharparenleft}{\kern0pt}{\isasymdelta}\ {\isacharasterisk}{\kern0pt}\ F\ {\isadigit{2}}\ as{\isacharparenright}{\kern0pt}{\isacharparenright}{\kern0pt}\isactrlsup {\isadigit{2}}\ {\isacharslash}{\kern0pt}\ {\isadigit{3}}{\isachardoublequoteclose}\isanewline
\ \ \isacommand{proof}\isamarkupfalse%
\ {\isacharminus}{\kern0pt}\isanewline
\ \ \ \ \isacommand{fix}\isamarkupfalse%
\ i\isanewline
\ \ \ \ \isacommand{assume}\isamarkupfalse%
\ a{\isacharcolon}{\kern0pt}{\isachardoublequoteopen}i\ {\isacharless}{\kern0pt}\ s\isactrlsub {\isadigit{2}}{\isachardoublequoteclose}\isanewline
\ \ \ \ \isacommand{have}\isamarkupfalse%
\ b{\isacharcolon}{\kern0pt}\ {\isachardoublequoteopen}prob{\isacharunderscore}{\kern0pt}space{\isachardot}{\kern0pt}indep{\isacharunderscore}{\kern0pt}vars\ {\isacharparenleft}{\kern0pt}measure{\isacharunderscore}{\kern0pt}pmf\ {\isasymOmega}\isactrlsub {\isadigit{0}}{\isacharparenright}{\kern0pt}\ {\isacharparenleft}{\kern0pt}{\isasymlambda}{\isacharunderscore}{\kern0pt}{\isachardot}{\kern0pt}\ borel{\isacharparenright}{\kern0pt}\ {\isacharparenleft}{\kern0pt}{\isasymlambda}i\isactrlsub {\isadigit{1}}\ x{\isachardot}{\kern0pt}\ f{\isadigit{3}}\ x\ i\isactrlsub {\isadigit{1}}\ i{\isacharparenright}{\kern0pt}\ {\isacharbraceleft}{\kern0pt}{\isadigit{0}}{\isachardot}{\kern0pt}{\isachardot}{\kern0pt}{\isacharless}{\kern0pt}s\isactrlsub {\isadigit{1}}{\isacharbraceright}{\kern0pt}{\isachardoublequoteclose}\isanewline
\ \ \ \ \ \ \isacommand{apply}\isamarkupfalse%
\ {\isacharparenleft}{\kern0pt}simp\ add{\isacharcolon}{\kern0pt}{\isasymOmega}\isactrlsub {\isadigit{0}}{\isacharunderscore}{\kern0pt}def{\isacharcomma}{\kern0pt}\ rule\ indep{\isacharunderscore}{\kern0pt}vars{\isacharunderscore}{\kern0pt}restrict{\isacharunderscore}{\kern0pt}intro\ {\isacharbrackleft}{\kern0pt}\isakeyword{where}\ f{\isacharequal}{\kern0pt}{\isachardoublequoteopen}{\isasymlambda}j{\isachardot}{\kern0pt}\ {\isacharbraceleft}{\kern0pt}{\isacharparenleft}{\kern0pt}j{\isacharcomma}{\kern0pt}i{\isacharparenright}{\kern0pt}{\isacharbraceright}{\kern0pt}{\isachardoublequoteclose}{\isacharbrackright}{\kern0pt}{\isacharparenright}{\kern0pt}\isanewline
\ \ \ \ \ \ \isacommand{using}\isamarkupfalse%
\ a\ f{\isadigit{3}}{\isacharunderscore}{\kern0pt}def\ disjoint{\isacharunderscore}{\kern0pt}family{\isacharunderscore}{\kern0pt}on{\isacharunderscore}{\kern0pt}def\ s{\isadigit{1}}{\isacharunderscore}{\kern0pt}nonzero\ s{\isadigit{2}}{\isacharunderscore}{\kern0pt}nonzero\ \isacommand{by}\isamarkupfalse%
\ auto\isanewline
\isanewline
\ \ \ \ \isacommand{have}\isamarkupfalse%
\ {\isachardoublequoteopen}prob{\isacharunderscore}{\kern0pt}space{\isachardot}{\kern0pt}variance\ {\isasymOmega}\isactrlsub {\isadigit{0}}\ {\isacharparenleft}{\kern0pt}{\isasymlambda}{\isasymomega}{\isachardot}{\kern0pt}\ f{\isadigit{2}}\ {\isasymomega}\ i{\isacharparenright}{\kern0pt}\ {\isacharequal}{\kern0pt}\ {\isacharparenleft}{\kern0pt}{\isasymSum}j\ {\isacharequal}{\kern0pt}\ {\isadigit{0}}{\isachardot}{\kern0pt}{\isachardot}{\kern0pt}{\isacharless}{\kern0pt}s\isactrlsub {\isadigit{1}}{\isachardot}{\kern0pt}\ prob{\isacharunderscore}{\kern0pt}space{\isachardot}{\kern0pt}variance\ {\isasymOmega}\isactrlsub {\isadigit{0}}\ {\isacharparenleft}{\kern0pt}{\isasymlambda}{\isasymomega}{\isachardot}{\kern0pt}\ f{\isadigit{3}}\ {\isasymomega}\ j\ i{\isacharparenright}{\kern0pt}{\isacharparenright}{\kern0pt}\ {\isacharslash}{\kern0pt}\ {\isacharparenleft}{\kern0pt}{\isacharparenleft}{\kern0pt}{\isacharparenleft}{\kern0pt}real\ p{\isacharparenright}{\kern0pt}\isactrlsup {\isadigit{2}}\ {\isacharminus}{\kern0pt}\ {\isadigit{1}}{\isacharparenright}{\kern0pt}\ {\isacharasterisk}{\kern0pt}\ real\ s\isactrlsub {\isadigit{1}}{\isacharparenright}{\kern0pt}\isactrlsup {\isadigit{2}}{\isachardoublequoteclose}\isanewline
\ \ \ \ \ \ \isacommand{apply}\isamarkupfalse%
\ {\isacharparenleft}{\kern0pt}simp\ add{\isacharcolon}{\kern0pt}\ a\ f{\isadigit{2}}{\isacharunderscore}{\kern0pt}def\ del{\isacharcolon}{\kern0pt}Bochner{\isacharunderscore}{\kern0pt}Integration{\isachardot}{\kern0pt}integral{\isacharunderscore}{\kern0pt}divide{\isacharunderscore}{\kern0pt}zero{\isacharparenright}{\kern0pt}\isanewline
\ \ \ \ \ \ \isacommand{apply}\isamarkupfalse%
\ {\isacharparenleft}{\kern0pt}subst\ prob{\isacharunderscore}{\kern0pt}space{\isachardot}{\kern0pt}variance{\isacharunderscore}{\kern0pt}divide{\isacharbrackleft}{\kern0pt}OF\ prob{\isacharunderscore}{\kern0pt}space{\isacharunderscore}{\kern0pt}measure{\isacharunderscore}{\kern0pt}pmf{\isacharbrackright}{\kern0pt}{\isacharparenright}{\kern0pt}\isanewline
\ \ \ \ \ \ \ \isacommand{apply}\isamarkupfalse%
\ {\isacharparenleft}{\kern0pt}rule\ integrable{\isacharunderscore}{\kern0pt}measure{\isacharunderscore}{\kern0pt}pmf{\isacharunderscore}{\kern0pt}finite{\isacharbrackleft}{\kern0pt}OF\ fin{\isacharunderscore}{\kern0pt}omega{\isacharunderscore}{\kern0pt}{\isadigit{1}}{\isacharbrackright}{\kern0pt}{\isacharparenright}{\kern0pt}\isanewline
\ \ \ \ \ \ \isacommand{apply}\isamarkupfalse%
\ {\isacharparenleft}{\kern0pt}subst\ prob{\isacharunderscore}{\kern0pt}space{\isachardot}{\kern0pt}var{\isacharunderscore}{\kern0pt}sum{\isacharunderscore}{\kern0pt}all{\isacharunderscore}{\kern0pt}indep{\isacharbrackleft}{\kern0pt}OF\ prob{\isacharunderscore}{\kern0pt}space{\isacharunderscore}{\kern0pt}measure{\isacharunderscore}{\kern0pt}pmf{\isacharbrackright}{\kern0pt}{\isacharparenright}{\kern0pt}\isanewline
\ \ \ \ \ \ \ \ \ \ \isacommand{apply}\isamarkupfalse%
\ {\isacharparenleft}{\kern0pt}simp{\isacharparenright}{\kern0pt}\isanewline
\ \ \ \ \ \ \ \ \ \isacommand{apply}\isamarkupfalse%
\ {\isacharparenleft}{\kern0pt}simp{\isacharparenright}{\kern0pt}\isanewline
\ \ \ \ \ \ \ \ \isacommand{apply}\isamarkupfalse%
\ {\isacharparenleft}{\kern0pt}rule\ integrable{\isacharunderscore}{\kern0pt}measure{\isacharunderscore}{\kern0pt}pmf{\isacharunderscore}{\kern0pt}finite{\isacharbrackleft}{\kern0pt}OF\ fin{\isacharunderscore}{\kern0pt}omega{\isacharunderscore}{\kern0pt}{\isadigit{1}}{\isacharbrackright}{\kern0pt}{\isacharparenright}{\kern0pt}\isanewline
\ \ \ \ \ \ \ \isacommand{apply}\isamarkupfalse%
\ {\isacharparenleft}{\kern0pt}metis\ b{\isacharparenright}{\kern0pt}\isanewline
\ \ \ \ \ \ \isacommand{by}\isamarkupfalse%
\ simp\isanewline
\ \ \ \ \isacommand{also}\isamarkupfalse%
\ \isacommand{have}\isamarkupfalse%
\ {\isachardoublequoteopen}{\isachardot}{\kern0pt}{\isachardot}{\kern0pt}{\isachardot}{\kern0pt}\ {\isasymle}\ {\isacharparenleft}{\kern0pt}{\isasymSum}j\ {\isacharequal}{\kern0pt}\ {\isadigit{0}}{\isachardot}{\kern0pt}{\isachardot}{\kern0pt}{\isacharless}{\kern0pt}s\isactrlsub {\isadigit{1}}{\isachardot}{\kern0pt}\ {\isadigit{2}}{\isacharasterisk}{\kern0pt}{\isacharparenleft}{\kern0pt}real{\isacharunderscore}{\kern0pt}of{\isacharunderscore}{\kern0pt}rat\ {\isacharparenleft}{\kern0pt}F\ {\isadigit{2}}\ as{\isacharparenright}{\kern0pt}{\isacharcircum}{\kern0pt}{\isadigit{2}}{\isacharparenright}{\kern0pt}\ {\isacharasterisk}{\kern0pt}\ {\isacharparenleft}{\kern0pt}{\isacharparenleft}{\kern0pt}real\ p{\isacharparenright}{\kern0pt}\isactrlsup {\isadigit{2}}{\isacharminus}{\kern0pt}{\isadigit{1}}{\isacharparenright}{\kern0pt}\isactrlsup {\isadigit{2}}{\isacharparenright}{\kern0pt}\ {\isacharslash}{\kern0pt}\ {\isacharparenleft}{\kern0pt}{\isacharparenleft}{\kern0pt}{\isacharparenleft}{\kern0pt}real\ p{\isacharparenright}{\kern0pt}\isactrlsup {\isadigit{2}}\ {\isacharminus}{\kern0pt}\ {\isadigit{1}}{\isacharparenright}{\kern0pt}\ {\isacharasterisk}{\kern0pt}\ real\ s\isactrlsub {\isadigit{1}}{\isacharparenright}{\kern0pt}\isactrlsup {\isadigit{2}}{\isachardoublequoteclose}\isanewline
\ \ \ \ \ \ \isacommand{apply}\isamarkupfalse%
\ {\isacharparenleft}{\kern0pt}rule\ divide{\isacharunderscore}{\kern0pt}right{\isacharunderscore}{\kern0pt}mono{\isacharparenright}{\kern0pt}\isanewline
\ \ \ \ \ \ \ \isacommand{apply}\isamarkupfalse%
\ {\isacharparenleft}{\kern0pt}rule\ sum{\isacharunderscore}{\kern0pt}mono{\isacharparenright}{\kern0pt}\isanewline
\ \ \ \ \ \ \ \isacommand{apply}\isamarkupfalse%
\ {\isacharparenleft}{\kern0pt}simp\ add{\isacharcolon}{\kern0pt}f{\isadigit{3}}{\isacharunderscore}{\kern0pt}def\ {\isasymOmega}\isactrlsub {\isadigit{0}}{\isacharunderscore}{\kern0pt}def{\isacharparenright}{\kern0pt}\isanewline
\ \ \ \ \ \ \ \isacommand{apply}\isamarkupfalse%
\ {\isacharparenleft}{\kern0pt}subst\ variance{\isacharunderscore}{\kern0pt}prod{\isacharunderscore}{\kern0pt}pmf{\isacharunderscore}{\kern0pt}slice{\isacharcomma}{\kern0pt}\ simp\ add{\isacharcolon}{\kern0pt}a{\isacharcomma}{\kern0pt}\ simp{\isacharparenright}{\kern0pt}\isanewline
\ \ \ \ \ \ \ \isacommand{apply}\isamarkupfalse%
\ {\isacharparenleft}{\kern0pt}rule\ integrable{\isacharunderscore}{\kern0pt}measure{\isacharunderscore}{\kern0pt}pmf{\isacharunderscore}{\kern0pt}finite{\isacharbrackleft}{\kern0pt}OF\ fin{\isacharunderscore}{\kern0pt}omega{\isacharunderscore}{\kern0pt}{\isadigit{2}}{\isacharbrackright}{\kern0pt}{\isacharparenright}{\kern0pt}\isanewline
\ \ \ \ \ \ \ \isacommand{apply}\isamarkupfalse%
\ {\isacharparenleft}{\kern0pt}rule\ var{\isacharunderscore}{\kern0pt}f{\isadigit{2}}{\isacharbrackleft}{\kern0pt}OF\ p{\isacharunderscore}{\kern0pt}prime\ p{\isacharunderscore}{\kern0pt}ge{\isacharunderscore}{\kern0pt}{\isadigit{2}}\ as{\isacharunderscore}{\kern0pt}le{\isacharunderscore}{\kern0pt}p{\isacharbrackright}{\kern0pt}{\isacharcomma}{\kern0pt}\ simp{\isacharparenright}{\kern0pt}\isanewline
\ \ \ \ \ \ \isacommand{by}\isamarkupfalse%
\ simp\isanewline
\ \ \ \ \isacommand{also}\isamarkupfalse%
\ \isacommand{have}\isamarkupfalse%
\ {\isachardoublequoteopen}{\isachardot}{\kern0pt}{\isachardot}{\kern0pt}{\isachardot}{\kern0pt}\ {\isacharequal}{\kern0pt}\ {\isadigit{2}}\ {\isacharasterisk}{\kern0pt}\ {\isacharparenleft}{\kern0pt}real{\isacharunderscore}{\kern0pt}of{\isacharunderscore}{\kern0pt}rat\ {\isacharparenleft}{\kern0pt}F\ {\isadigit{2}}\ as{\isacharparenright}{\kern0pt}{\isacharcircum}{\kern0pt}{\isadigit{2}}{\isacharparenright}{\kern0pt}\ {\isacharslash}{\kern0pt}\ real\ s\isactrlsub {\isadigit{1}}{\isachardoublequoteclose}\isanewline
\ \ \ \ \ \ \isacommand{apply}\isamarkupfalse%
\ {\isacharparenleft}{\kern0pt}simp{\isacharparenright}{\kern0pt}\isanewline
\ \ \ \ \ \ \isacommand{apply}\isamarkupfalse%
\ {\isacharparenleft}{\kern0pt}subst\ frac{\isacharunderscore}{\kern0pt}eq{\isacharunderscore}{\kern0pt}eq{\isacharcomma}{\kern0pt}\ simp\ add{\isacharcolon}{\kern0pt}s{\isadigit{1}}{\isacharunderscore}{\kern0pt}nonzero{\isacharcomma}{\kern0pt}\ metis\ p{\isacharunderscore}{\kern0pt}sq{\isacharunderscore}{\kern0pt}ne{\isacharunderscore}{\kern0pt}{\isadigit{1}}{\isacharcomma}{\kern0pt}\ simp\ add{\isacharcolon}{\kern0pt}s{\isadigit{1}}{\isacharunderscore}{\kern0pt}nonzero{\isacharparenright}{\kern0pt}\isanewline
\ \ \ \ \ \ \isacommand{by}\isamarkupfalse%
\ {\isacharparenleft}{\kern0pt}simp\ add{\isacharcolon}{\kern0pt}power{\isadigit{2}}{\isacharunderscore}{\kern0pt}eq{\isacharunderscore}{\kern0pt}square{\isacharparenright}{\kern0pt}\isanewline
\ \ \ \ \isacommand{also}\isamarkupfalse%
\ \isacommand{have}\isamarkupfalse%
\ {\isachardoublequoteopen}{\isachardot}{\kern0pt}{\isachardot}{\kern0pt}{\isachardot}{\kern0pt}\ {\isasymle}\ {\isadigit{2}}\ {\isacharasterisk}{\kern0pt}\ {\isacharparenleft}{\kern0pt}real{\isacharunderscore}{\kern0pt}of{\isacharunderscore}{\kern0pt}rat\ {\isacharparenleft}{\kern0pt}F\ {\isadigit{2}}\ as{\isacharparenright}{\kern0pt}{\isacharcircum}{\kern0pt}{\isadigit{2}}{\isacharparenright}{\kern0pt}\ {\isacharslash}{\kern0pt}\ {\isacharparenleft}{\kern0pt}{\isadigit{6}}\ {\isacharslash}{\kern0pt}\ {\isacharparenleft}{\kern0pt}real{\isacharunderscore}{\kern0pt}of{\isacharunderscore}{\kern0pt}rat\ {\isasymdelta}{\isacharparenright}{\kern0pt}\isactrlsup {\isadigit{2}}{\isacharparenright}{\kern0pt}{\isachardoublequoteclose}\isanewline
\ \ \ \ \ \ \isacommand{apply}\isamarkupfalse%
\ {\isacharparenleft}{\kern0pt}rule\ divide{\isacharunderscore}{\kern0pt}left{\isacharunderscore}{\kern0pt}mono{\isacharparenright}{\kern0pt}\isanewline
\ \ \ \ \ \ \isacommand{apply}\isamarkupfalse%
\ {\isacharparenleft}{\kern0pt}simp\ add{\isacharcolon}{\kern0pt}s\isactrlsub {\isadigit{1}}{\isacharunderscore}{\kern0pt}def{\isacharparenright}{\kern0pt}\ \isanewline
\ \ \ \ \ \ \ \ \isacommand{apply}\isamarkupfalse%
\ {\isacharparenleft}{\kern0pt}metis\ {\isacharparenleft}{\kern0pt}mono{\isacharunderscore}{\kern0pt}tags{\isacharcomma}{\kern0pt}\ opaque{\isacharunderscore}{\kern0pt}lifting{\isacharparenright}{\kern0pt}\ of{\isacharunderscore}{\kern0pt}rat{\isacharunderscore}{\kern0pt}ceiling\ of{\isacharunderscore}{\kern0pt}rat{\isacharunderscore}{\kern0pt}divide\ of{\isacharunderscore}{\kern0pt}rat{\isacharunderscore}{\kern0pt}numeral{\isacharunderscore}{\kern0pt}eq\ of{\isacharunderscore}{\kern0pt}rat{\isacharunderscore}{\kern0pt}power\ real{\isacharunderscore}{\kern0pt}nat{\isacharunderscore}{\kern0pt}ceiling{\isacharunderscore}{\kern0pt}ge{\isacharparenright}{\kern0pt}\isanewline
\ \ \ \ \ \ \ \isacommand{apply}\isamarkupfalse%
\ simp\isanewline
\ \ \ \ \ \ \isacommand{apply}\isamarkupfalse%
\ {\isacharparenleft}{\kern0pt}rule\ mult{\isacharunderscore}{\kern0pt}pos{\isacharunderscore}{\kern0pt}pos{\isacharparenright}{\kern0pt}\isanewline
\ \ \ \ \ \ \isacommand{using}\isamarkupfalse%
\ s{\isadigit{1}}{\isacharunderscore}{\kern0pt}nonzero\ \isacommand{apply}\isamarkupfalse%
\ simp\isanewline
\ \ \ \ \ \ \isacommand{using}\isamarkupfalse%
\ assms{\isacharparenleft}{\kern0pt}{\isadigit{2}}{\isacharparenright}{\kern0pt}\ \isacommand{by}\isamarkupfalse%
\ simp\isanewline
\ \ \ \ \isacommand{also}\isamarkupfalse%
\ \isacommand{have}\isamarkupfalse%
\ {\isachardoublequoteopen}{\isachardot}{\kern0pt}{\isachardot}{\kern0pt}{\isachardot}{\kern0pt}\ {\isacharequal}{\kern0pt}\ {\isacharparenleft}{\kern0pt}real{\isacharunderscore}{\kern0pt}of{\isacharunderscore}{\kern0pt}rat\ {\isacharparenleft}{\kern0pt}{\isasymdelta}\ {\isacharasterisk}{\kern0pt}\ F\ {\isadigit{2}}\ as{\isacharparenright}{\kern0pt}{\isacharparenright}{\kern0pt}\isactrlsup {\isadigit{2}}\ {\isacharslash}{\kern0pt}\ {\isadigit{3}}{\isachardoublequoteclose}\isanewline
\ \ \ \ \ \ \isacommand{by}\isamarkupfalse%
\ {\isacharparenleft}{\kern0pt}simp\ add{\isacharcolon}{\kern0pt}of{\isacharunderscore}{\kern0pt}rat{\isacharunderscore}{\kern0pt}mult\ algebra{\isacharunderscore}{\kern0pt}simps{\isacharparenright}{\kern0pt}\isanewline
\ \ \ \ \isacommand{finally}\isamarkupfalse%
\ \isacommand{show}\isamarkupfalse%
\ {\isachardoublequoteopen}prob{\isacharunderscore}{\kern0pt}space{\isachardot}{\kern0pt}variance\ {\isasymOmega}\isactrlsub {\isadigit{0}}\ {\isacharparenleft}{\kern0pt}{\isasymlambda}{\isasymomega}{\isachardot}{\kern0pt}\ f{\isadigit{2}}\ {\isasymomega}\ i{\isacharparenright}{\kern0pt}\ {\isasymle}\ {\isacharparenleft}{\kern0pt}real{\isacharunderscore}{\kern0pt}of{\isacharunderscore}{\kern0pt}rat\ {\isacharparenleft}{\kern0pt}{\isasymdelta}\ {\isacharasterisk}{\kern0pt}\ F\ {\isadigit{2}}\ as{\isacharparenright}{\kern0pt}{\isacharparenright}{\kern0pt}\isactrlsup {\isadigit{2}}\ {\isacharslash}{\kern0pt}\ {\isadigit{3}}{\isachardoublequoteclose}\isanewline
\ \ \ \ \ \ \isacommand{by}\isamarkupfalse%
\ simp\isanewline
\ \ \isacommand{qed}\isamarkupfalse%
\isanewline
\isanewline
\ \ \isacommand{have}\isamarkupfalse%
\ f{\isadigit{2}}{\isacharunderscore}{\kern0pt}exp{\isacharprime}{\kern0pt}{\isacharprime}{\kern0pt}{\isacharcolon}{\kern0pt}\ {\isachardoublequoteopen}{\isasymAnd}i{\isachardot}{\kern0pt}\ i\ {\isacharless}{\kern0pt}\ s\isactrlsub {\isadigit{2}}\ {\isasymLongrightarrow}\ prob{\isacharunderscore}{\kern0pt}space{\isachardot}{\kern0pt}expectation\ {\isasymOmega}\isactrlsub {\isadigit{0}}\ {\isacharparenleft}{\kern0pt}{\isasymlambda}{\isasymomega}{\isachardot}{\kern0pt}\ f{\isadigit{2}}\ {\isasymomega}\ i{\isacharparenright}{\kern0pt}\ {\isacharequal}{\kern0pt}\ real{\isacharunderscore}{\kern0pt}of{\isacharunderscore}{\kern0pt}rat\ {\isacharparenleft}{\kern0pt}F\ {\isadigit{2}}\ as{\isacharparenright}{\kern0pt}{\isachardoublequoteclose}\isanewline
\ \ \isacommand{proof}\isamarkupfalse%
\ {\isacharminus}{\kern0pt}\isanewline
\ \ \ \ \isacommand{fix}\isamarkupfalse%
\ i\isanewline
\ \ \ \ \isacommand{assume}\isamarkupfalse%
\ a{\isacharcolon}{\kern0pt}{\isachardoublequoteopen}i\ {\isacharless}{\kern0pt}\ s\isactrlsub {\isadigit{2}}{\isachardoublequoteclose}\isanewline
\ \ \ \ \isacommand{have}\isamarkupfalse%
\ {\isachardoublequoteopen}prob{\isacharunderscore}{\kern0pt}space{\isachardot}{\kern0pt}expectation\ {\isasymOmega}\isactrlsub {\isadigit{0}}\ {\isacharparenleft}{\kern0pt}{\isasymlambda}{\isasymomega}{\isachardot}{\kern0pt}\ f{\isadigit{2}}\ {\isasymomega}\ i{\isacharparenright}{\kern0pt}\ {\isacharequal}{\kern0pt}\ {\isacharparenleft}{\kern0pt}{\isasymSum}j\ {\isacharequal}{\kern0pt}\ {\isadigit{0}}{\isachardot}{\kern0pt}{\isachardot}{\kern0pt}{\isacharless}{\kern0pt}s\isactrlsub {\isadigit{1}}{\isachardot}{\kern0pt}\ prob{\isacharunderscore}{\kern0pt}space{\isachardot}{\kern0pt}expectation\ {\isasymOmega}\isactrlsub {\isadigit{0}}\ {\isacharparenleft}{\kern0pt}{\isasymlambda}{\isasymomega}{\isachardot}{\kern0pt}\ f{\isadigit{3}}\ {\isasymomega}\ j\ i{\isacharparenright}{\kern0pt}{\isacharparenright}{\kern0pt}\ {\isacharslash}{\kern0pt}\ {\isacharparenleft}{\kern0pt}{\isacharparenleft}{\kern0pt}{\isacharparenleft}{\kern0pt}real\ p{\isacharparenright}{\kern0pt}\isactrlsup {\isadigit{2}}\ {\isacharminus}{\kern0pt}\ {\isadigit{1}}{\isacharparenright}{\kern0pt}\ {\isacharasterisk}{\kern0pt}\ real\ s\isactrlsub {\isadigit{1}}{\isacharparenright}{\kern0pt}{\isachardoublequoteclose}\isanewline
\ \ \ \ \ \ \isacommand{apply}\isamarkupfalse%
\ {\isacharparenleft}{\kern0pt}simp\ add{\isacharcolon}{\kern0pt}\ a\ f{\isadigit{2}}{\isacharunderscore}{\kern0pt}def{\isacharparenright}{\kern0pt}\isanewline
\ \ \ \ \ \ \isacommand{apply}\isamarkupfalse%
\ {\isacharparenleft}{\kern0pt}subst\ Bochner{\isacharunderscore}{\kern0pt}Integration{\isachardot}{\kern0pt}integral{\isacharunderscore}{\kern0pt}sum{\isacharparenright}{\kern0pt}\isanewline
\ \ \ \ \ \ \ \isacommand{apply}\isamarkupfalse%
\ {\isacharparenleft}{\kern0pt}rule\ integrable{\isacharunderscore}{\kern0pt}measure{\isacharunderscore}{\kern0pt}pmf{\isacharunderscore}{\kern0pt}finite{\isacharbrackleft}{\kern0pt}OF\ fin{\isacharunderscore}{\kern0pt}omega{\isacharunderscore}{\kern0pt}{\isadigit{1}}{\isacharbrackright}{\kern0pt}{\isacharparenright}{\kern0pt}\isanewline
\ \ \ \ \ \ \isacommand{by}\isamarkupfalse%
\ simp\isanewline
\ \ \ \ \isacommand{also}\isamarkupfalse%
\ \isacommand{have}\isamarkupfalse%
\ {\isachardoublequoteopen}{\isachardot}{\kern0pt}{\isachardot}{\kern0pt}{\isachardot}{\kern0pt}\ {\isacharequal}{\kern0pt}\ {\isacharparenleft}{\kern0pt}{\isasymSum}j\ {\isacharequal}{\kern0pt}\ {\isadigit{0}}{\isachardot}{\kern0pt}{\isachardot}{\kern0pt}{\isacharless}{\kern0pt}s\isactrlsub {\isadigit{1}}{\isachardot}{\kern0pt}\ real{\isacharunderscore}{\kern0pt}of{\isacharunderscore}{\kern0pt}rat\ {\isacharparenleft}{\kern0pt}F\ {\isadigit{2}}\ as{\isacharparenright}{\kern0pt}\ {\isacharasterisk}{\kern0pt}\ {\isacharparenleft}{\kern0pt}{\isacharparenleft}{\kern0pt}real\ p{\isacharparenright}{\kern0pt}\isactrlsup {\isadigit{2}}{\isacharminus}{\kern0pt}{\isadigit{1}}{\isacharparenright}{\kern0pt}{\isacharparenright}{\kern0pt}\ {\isacharslash}{\kern0pt}\ {\isacharparenleft}{\kern0pt}{\isacharparenleft}{\kern0pt}{\isacharparenleft}{\kern0pt}real\ p{\isacharparenright}{\kern0pt}\isactrlsup {\isadigit{2}}\ {\isacharminus}{\kern0pt}\ {\isadigit{1}}{\isacharparenright}{\kern0pt}\ {\isacharasterisk}{\kern0pt}\ real\ s\isactrlsub {\isadigit{1}}{\isacharparenright}{\kern0pt}{\isachardoublequoteclose}\isanewline
\ \ \ \ \ \ \isacommand{apply}\isamarkupfalse%
\ {\isacharparenleft}{\kern0pt}rule\ arg{\isacharunderscore}{\kern0pt}cong{\isadigit{2}}{\isacharbrackleft}{\kern0pt}\isakeyword{where}\ f{\isacharequal}{\kern0pt}{\isachardoublequoteopen}{\isacharparenleft}{\kern0pt}{\isacharslash}{\kern0pt}{\isacharparenright}{\kern0pt}{\isachardoublequoteclose}{\isacharbrackright}{\kern0pt}{\isacharparenright}{\kern0pt}\isanewline
\ \ \ \ \ \ \ \isacommand{apply}\isamarkupfalse%
\ {\isacharparenleft}{\kern0pt}rule\ sum{\isachardot}{\kern0pt}cong{\isacharcomma}{\kern0pt}\ simp{\isacharparenright}{\kern0pt}\isanewline
\ \ \ \ \ \ \ \isacommand{apply}\isamarkupfalse%
\ {\isacharparenleft}{\kern0pt}simp\ add{\isacharcolon}{\kern0pt}f{\isadigit{3}}{\isacharunderscore}{\kern0pt}def\ {\isasymOmega}\isactrlsub {\isadigit{0}}{\isacharunderscore}{\kern0pt}def{\isacharparenright}{\kern0pt}\isanewline
\ \ \ \ \ \ \ \isacommand{apply}\isamarkupfalse%
\ {\isacharparenleft}{\kern0pt}subst\ integral{\isacharunderscore}{\kern0pt}prod{\isacharunderscore}{\kern0pt}pmf{\isacharunderscore}{\kern0pt}slice{\isacharcomma}{\kern0pt}\ simp{\isacharcomma}{\kern0pt}\ simp\ add{\isacharcolon}{\kern0pt}a{\isacharparenright}{\kern0pt}\isanewline
\ \ \ \ \ \ \ \ \isacommand{apply}\isamarkupfalse%
\ {\isacharparenleft}{\kern0pt}rule\ integrable{\isacharunderscore}{\kern0pt}measure{\isacharunderscore}{\kern0pt}pmf{\isacharunderscore}{\kern0pt}finite{\isacharbrackleft}{\kern0pt}OF\ fin{\isacharunderscore}{\kern0pt}omega{\isacharunderscore}{\kern0pt}{\isadigit{2}}{\isacharbrackright}{\kern0pt}{\isacharparenright}{\kern0pt}\isanewline
\ \ \ \ \ \ \ \isacommand{apply}\isamarkupfalse%
\ {\isacharparenleft}{\kern0pt}subst\ exp{\isacharunderscore}{\kern0pt}f{\isadigit{2}}{\isacharbrackleft}{\kern0pt}OF\ p{\isacharunderscore}{\kern0pt}prime\ p{\isacharunderscore}{\kern0pt}ge{\isacharunderscore}{\kern0pt}{\isadigit{2}}\ as{\isacharunderscore}{\kern0pt}le{\isacharunderscore}{\kern0pt}p{\isacharbrackright}{\kern0pt}{\isacharcomma}{\kern0pt}\ simp{\isacharcomma}{\kern0pt}\ simp{\isacharparenright}{\kern0pt}\isanewline
\ \ \ \ \ \ \isacommand{by}\isamarkupfalse%
\ simp\isanewline
\ \ \ \ \isacommand{also}\isamarkupfalse%
\ \isacommand{have}\isamarkupfalse%
\ {\isachardoublequoteopen}{\isachardot}{\kern0pt}{\isachardot}{\kern0pt}{\isachardot}{\kern0pt}\ {\isacharequal}{\kern0pt}\ \ real{\isacharunderscore}{\kern0pt}of{\isacharunderscore}{\kern0pt}rat\ {\isacharparenleft}{\kern0pt}F\ {\isadigit{2}}\ as{\isacharparenright}{\kern0pt}{\isachardoublequoteclose}\isanewline
\ \ \ \ \ \ \isacommand{by}\isamarkupfalse%
\ {\isacharparenleft}{\kern0pt}simp\ add{\isacharcolon}{\kern0pt}s{\isadigit{1}}{\isacharunderscore}{\kern0pt}nonzero\ p{\isacharunderscore}{\kern0pt}sq{\isacharunderscore}{\kern0pt}ne{\isacharunderscore}{\kern0pt}{\isadigit{1}}{\isacharparenright}{\kern0pt}\isanewline
\ \ \ \ \isacommand{finally}\isamarkupfalse%
\ \isacommand{show}\isamarkupfalse%
\ {\isachardoublequoteopen}\ prob{\isacharunderscore}{\kern0pt}space{\isachardot}{\kern0pt}expectation\ {\isasymOmega}\isactrlsub {\isadigit{0}}\ {\isacharparenleft}{\kern0pt}{\isasymlambda}{\isasymomega}{\isachardot}{\kern0pt}\ f{\isadigit{2}}\ {\isasymomega}\ i{\isacharparenright}{\kern0pt}\ {\isacharequal}{\kern0pt}\ real{\isacharunderscore}{\kern0pt}of{\isacharunderscore}{\kern0pt}rat\ {\isacharparenleft}{\kern0pt}F\ {\isadigit{2}}\ as{\isacharparenright}{\kern0pt}{\isachardoublequoteclose}\isanewline
\ \ \ \ \ \ \isacommand{by}\isamarkupfalse%
\ simp\isanewline
\ \ \isacommand{qed}\isamarkupfalse%
\isanewline
\isanewline
\ \ \isacommand{define}\isamarkupfalse%
\ f{\isacharprime}{\kern0pt}\ \isakeyword{where}\ {\isachardoublequoteopen}f{\isacharprime}{\kern0pt}\ {\isacharequal}{\kern0pt}\ {\isacharparenleft}{\kern0pt}{\isasymlambda}x{\isachardot}{\kern0pt}\ median\ {\isacharparenleft}{\kern0pt}f{\isadigit{2}}\ x{\isacharparenright}{\kern0pt}\ s\isactrlsub {\isadigit{2}}{\isacharparenright}{\kern0pt}{\isachardoublequoteclose}\isanewline
\ \ \isacommand{have}\isamarkupfalse%
\ real{\isacharunderscore}{\kern0pt}f{\isacharcolon}{\kern0pt}\ {\isachardoublequoteopen}{\isasymAnd}x{\isachardot}{\kern0pt}\ real{\isacharunderscore}{\kern0pt}of{\isacharunderscore}{\kern0pt}rat\ {\isacharparenleft}{\kern0pt}f\ x{\isacharparenright}{\kern0pt}\ {\isacharequal}{\kern0pt}\ f{\isacharprime}{\kern0pt}\ x{\isachardoublequoteclose}\isanewline
\ \ \ \ \isacommand{using}\isamarkupfalse%
\ s{\isadigit{2}}{\isacharunderscore}{\kern0pt}nonzero\ \isacommand{apply}\isamarkupfalse%
\ {\isacharparenleft}{\kern0pt}simp\ add{\isacharcolon}{\kern0pt}f{\isacharprime}{\kern0pt}{\isacharunderscore}{\kern0pt}def\ f{\isadigit{2}}{\isacharunderscore}{\kern0pt}def\ f{\isadigit{3}}{\isacharunderscore}{\kern0pt}def\ f{\isacharunderscore}{\kern0pt}def\ median{\isacharunderscore}{\kern0pt}rat\ median{\isacharunderscore}{\kern0pt}restrict\ cong{\isacharcolon}{\kern0pt}restrict{\isacharunderscore}{\kern0pt}cong{\isacharparenright}{\kern0pt}\isanewline
\ \ \ \ \isacommand{by}\isamarkupfalse%
\ {\isacharparenleft}{\kern0pt}simp\ add{\isacharcolon}{\kern0pt}of{\isacharunderscore}{\kern0pt}rat{\isacharunderscore}{\kern0pt}divide\ of{\isacharunderscore}{\kern0pt}rat{\isacharunderscore}{\kern0pt}sum\ of{\isacharunderscore}{\kern0pt}rat{\isacharunderscore}{\kern0pt}power\ of{\isacharunderscore}{\kern0pt}rat{\isacharunderscore}{\kern0pt}mult\ of{\isacharunderscore}{\kern0pt}rat{\isacharunderscore}{\kern0pt}diff{\isacharparenright}{\kern0pt}\isanewline
\isanewline
\ \ \isacommand{have}\isamarkupfalse%
\ distr{\isacharprime}{\kern0pt}{\isacharcolon}{\kern0pt}\ {\isachardoublequoteopen}M\ {\isacharequal}{\kern0pt}\ map{\isacharunderscore}{\kern0pt}pmf\ f\ {\isacharparenleft}{\kern0pt}prod{\isacharunderscore}{\kern0pt}pmf\ \ {\isacharparenleft}{\kern0pt}{\isacharbraceleft}{\kern0pt}{\isadigit{0}}{\isachardot}{\kern0pt}{\isachardot}{\kern0pt}{\isacharless}{\kern0pt}s\isactrlsub {\isadigit{1}}{\isacharbraceright}{\kern0pt}\ {\isasymtimes}\ {\isacharbraceleft}{\kern0pt}{\isadigit{0}}{\isachardot}{\kern0pt}{\isachardot}{\kern0pt}{\isacharless}{\kern0pt}s\isactrlsub {\isadigit{2}}{\isacharbraceright}{\kern0pt}{\isacharparenright}{\kern0pt}\ {\isacharparenleft}{\kern0pt}{\isasymlambda}{\isacharunderscore}{\kern0pt}{\isachardot}{\kern0pt}\ pmf{\isacharunderscore}{\kern0pt}of{\isacharunderscore}{\kern0pt}set\ {\isacharparenleft}{\kern0pt}bounded{\isacharunderscore}{\kern0pt}degree{\isacharunderscore}{\kern0pt}polynomials\ {\isacharparenleft}{\kern0pt}ZFact\ {\isacharparenleft}{\kern0pt}int\ p{\isacharparenright}{\kern0pt}{\isacharparenright}{\kern0pt}\ {\isadigit{4}}{\isacharparenright}{\kern0pt}{\isacharparenright}{\kern0pt}{\isacharparenright}{\kern0pt}{\isachardoublequoteclose}\isanewline
\ \ \ \ \isacommand{using}\isamarkupfalse%
\ f{\isadigit{2}}{\isacharunderscore}{\kern0pt}alg{\isacharunderscore}{\kern0pt}sketch{\isacharbrackleft}{\kern0pt}OF\ assms{\isacharparenleft}{\kern0pt}{\isadigit{1}}{\isacharparenright}{\kern0pt}\ assms{\isacharparenleft}{\kern0pt}{\isadigit{2}}{\isacharparenright}{\kern0pt}{\isacharcomma}{\kern0pt}\ \isakeyword{where}\ as{\isacharequal}{\kern0pt}{\isachardoublequoteopen}as{\isachardoublequoteclose}\ \isakeyword{and}\ n{\isacharequal}{\kern0pt}{\isachardoublequoteopen}n{\isachardoublequoteclose}{\isacharbrackright}{\kern0pt}\isanewline
\ \ \ \ \isacommand{apply}\isamarkupfalse%
\ {\isacharparenleft}{\kern0pt}simp\ add{\isacharcolon}{\kern0pt}M{\isacharunderscore}{\kern0pt}def\ Let{\isacharunderscore}{\kern0pt}def\ s\isactrlsub {\isadigit{1}}{\isacharunderscore}{\kern0pt}def\ {\isacharbrackleft}{\kern0pt}symmetric{\isacharbrackright}{\kern0pt}\ s\isactrlsub {\isadigit{2}}{\isacharunderscore}{\kern0pt}def{\isacharbrackleft}{\kern0pt}symmetric{\isacharbrackright}{\kern0pt}\ p{\isacharunderscore}{\kern0pt}def{\isacharbrackleft}{\kern0pt}symmetric{\isacharbrackright}{\kern0pt}{\isacharparenright}{\kern0pt}\isanewline
\ \ \ \ \isacommand{apply}\isamarkupfalse%
\ {\isacharparenleft}{\kern0pt}subst\ bind{\isacharunderscore}{\kern0pt}assoc{\isacharunderscore}{\kern0pt}pmf{\isacharparenright}{\kern0pt}\isanewline
\ \ \ \ \isacommand{apply}\isamarkupfalse%
\ {\isacharparenleft}{\kern0pt}subst\ bind{\isacharunderscore}{\kern0pt}return{\isacharunderscore}{\kern0pt}pmf{\isacharparenright}{\kern0pt}\isanewline
\ \ \ \ \isacommand{apply}\isamarkupfalse%
\ {\isacharparenleft}{\kern0pt}subst\ f{\isadigit{2}}{\isacharunderscore}{\kern0pt}result{\isacharunderscore}{\kern0pt}conv{\isacharcomma}{\kern0pt}\ simp{\isacharparenright}{\kern0pt}\isanewline
\ \ \ \ \isacommand{apply}\isamarkupfalse%
\ {\isacharparenleft}{\kern0pt}simp\ add{\isacharcolon}{\kern0pt}s\isactrlsub {\isadigit{2}}{\isacharunderscore}{\kern0pt}from{\isacharunderscore}{\kern0pt}def\ s\isactrlsub {\isadigit{1}}{\isacharunderscore}{\kern0pt}from{\isacharunderscore}{\kern0pt}def\ p{\isacharunderscore}{\kern0pt}from{\isacharunderscore}{\kern0pt}def\ h{\isacharunderscore}{\kern0pt}from{\isacharunderscore}{\kern0pt}def\ sketch{\isacharunderscore}{\kern0pt}from{\isacharunderscore}{\kern0pt}def\ cong{\isacharcolon}{\kern0pt}restrict{\isacharunderscore}{\kern0pt}cong{\isacharparenright}{\kern0pt}\isanewline
\ \ \ \ \isacommand{by}\isamarkupfalse%
\ {\isacharparenleft}{\kern0pt}simp\ add{\isacharcolon}{\kern0pt}map{\isacharunderscore}{\kern0pt}pmf{\isacharunderscore}{\kern0pt}def{\isacharbrackleft}{\kern0pt}symmetric{\isacharbrackright}{\kern0pt}\ f{\isacharunderscore}{\kern0pt}def{\isacharparenright}{\kern0pt}\isanewline
\isanewline
\ \ \isacommand{define}\isamarkupfalse%
\ g\ \isakeyword{where}\ {\isachardoublequoteopen}g\ {\isacharequal}{\kern0pt}\ {\isacharparenleft}{\kern0pt}{\isasymlambda}{\isasymomega}{\isachardot}{\kern0pt}\ real{\isacharunderscore}{\kern0pt}of{\isacharunderscore}{\kern0pt}rat\ {\isacharparenleft}{\kern0pt}{\isasymdelta}\ {\isacharasterisk}{\kern0pt}\ F\ {\isadigit{2}}\ as{\isacharparenright}{\kern0pt}\ {\isasymge}\ {\isasymbar}{\isasymomega}\ {\isacharminus}{\kern0pt}\ real{\isacharunderscore}{\kern0pt}of{\isacharunderscore}{\kern0pt}rat\ {\isacharparenleft}{\kern0pt}F\ {\isadigit{2}}\ as{\isacharparenright}{\kern0pt}{\isasymbar}{\isacharparenright}{\kern0pt}{\isachardoublequoteclose}\isanewline
\ \ \isacommand{have}\isamarkupfalse%
\ e{\isacharcolon}{\kern0pt}\ {\isachardoublequoteopen}{\isacharbraceleft}{\kern0pt}{\isasymomega}{\isachardot}{\kern0pt}\ {\isasymdelta}\ {\isacharasterisk}{\kern0pt}\ F\ {\isadigit{2}}\ as\ {\isasymge}\ {\isasymbar}{\isasymomega}\ {\isacharminus}{\kern0pt}\ F\ {\isadigit{2}}\ as{\isasymbar}{\isacharbraceright}{\kern0pt}\ {\isacharequal}{\kern0pt}\ {\isacharbraceleft}{\kern0pt}{\isasymomega}{\isachardot}{\kern0pt}\ {\isacharparenleft}{\kern0pt}g\ {\isasymcirc}\ real{\isacharunderscore}{\kern0pt}of{\isacharunderscore}{\kern0pt}rat{\isacharparenright}{\kern0pt}\ {\isasymomega}{\isacharbraceright}{\kern0pt}{\isachardoublequoteclose}\isanewline
\ \ \ \ \isacommand{apply}\isamarkupfalse%
\ {\isacharparenleft}{\kern0pt}simp\ add{\isacharcolon}{\kern0pt}g{\isacharunderscore}{\kern0pt}def{\isacharparenright}{\kern0pt}\isanewline
\ \ \ \ \isacommand{apply}\isamarkupfalse%
\ {\isacharparenleft}{\kern0pt}rule\ order{\isacharunderscore}{\kern0pt}antisym{\isacharcomma}{\kern0pt}\ rule\ subsetI{\isacharcomma}{\kern0pt}\ simp{\isacharparenright}{\kern0pt}\ \isanewline
\ \ \ \ \isacommand{apply}\isamarkupfalse%
\ {\isacharparenleft}{\kern0pt}metis\ abs{\isacharunderscore}{\kern0pt}of{\isacharunderscore}{\kern0pt}rat\ of{\isacharunderscore}{\kern0pt}rat{\isacharunderscore}{\kern0pt}diff\ of{\isacharunderscore}{\kern0pt}rat{\isacharunderscore}{\kern0pt}less{\isacharunderscore}{\kern0pt}eq{\isacharparenright}{\kern0pt}\isanewline
\ \ \ \ \isacommand{apply}\isamarkupfalse%
\ {\isacharparenleft}{\kern0pt}rule\ subsetI{\isacharcomma}{\kern0pt}\ simp{\isacharparenright}{\kern0pt}\isanewline
\ \ \ \ \isacommand{by}\isamarkupfalse%
\ {\isacharparenleft}{\kern0pt}metis\ abs{\isacharunderscore}{\kern0pt}of{\isacharunderscore}{\kern0pt}rat\ of{\isacharunderscore}{\kern0pt}rat{\isacharunderscore}{\kern0pt}diff\ of{\isacharunderscore}{\kern0pt}rat{\isacharunderscore}{\kern0pt}less{\isacharunderscore}{\kern0pt}eq{\isacharparenright}{\kern0pt}\isanewline
\isanewline
\ \ \isacommand{have}\isamarkupfalse%
\ median{\isacharunderscore}{\kern0pt}bound{\isacharunderscore}{\kern0pt}{\isadigit{2}}{\isacharprime}{\kern0pt}{\isacharcolon}{\kern0pt}\ {\isachardoublequoteopen}prob{\isacharunderscore}{\kern0pt}space{\isachardot}{\kern0pt}indep{\isacharunderscore}{\kern0pt}vars\ {\isasymOmega}\isactrlsub {\isadigit{0}}\ {\isacharparenleft}{\kern0pt}{\isasymlambda}{\isacharunderscore}{\kern0pt}{\isachardot}{\kern0pt}\ borel{\isacharparenright}{\kern0pt}\ {\isacharparenleft}{\kern0pt}{\isasymlambda}i\ {\isasymomega}{\isachardot}{\kern0pt}\ f{\isadigit{2}}\ {\isasymomega}\ i{\isacharparenright}{\kern0pt}\ {\isacharbraceleft}{\kern0pt}{\isadigit{0}}{\isachardot}{\kern0pt}{\isachardot}{\kern0pt}{\isacharless}{\kern0pt}s\isactrlsub {\isadigit{2}}{\isacharbraceright}{\kern0pt}{\isachardoublequoteclose}\isanewline
\ \ \ \ \isacommand{apply}\isamarkupfalse%
\ {\isacharparenleft}{\kern0pt}subst\ {\isasymOmega}\isactrlsub {\isadigit{0}}{\isacharunderscore}{\kern0pt}def{\isacharparenright}{\kern0pt}\isanewline
\ \ \ \ \isacommand{apply}\isamarkupfalse%
\ {\isacharparenleft}{\kern0pt}rule\ indep{\isacharunderscore}{\kern0pt}vars{\isacharunderscore}{\kern0pt}restrict{\isacharunderscore}{\kern0pt}intro\ {\isacharbrackleft}{\kern0pt}\isakeyword{where}\ f{\isacharequal}{\kern0pt}{\isachardoublequoteopen}{\isasymlambda}j{\isachardot}{\kern0pt}\ {\isacharbraceleft}{\kern0pt}{\isadigit{0}}{\isachardot}{\kern0pt}{\isachardot}{\kern0pt}{\isacharless}{\kern0pt}s\isactrlsub {\isadigit{1}}{\isacharbraceright}{\kern0pt}\ {\isasymtimes}\ {\isacharbraceleft}{\kern0pt}j{\isacharbraceright}{\kern0pt}{\isachardoublequoteclose}{\isacharbrackright}{\kern0pt}{\isacharparenright}{\kern0pt}\isanewline
\ \ \ \ \ \ \ \ \ \isacommand{apply}\isamarkupfalse%
\ {\isacharparenleft}{\kern0pt}simp\ add{\isacharcolon}{\kern0pt}f{\isadigit{2}}{\isacharunderscore}{\kern0pt}def\ f{\isadigit{3}}{\isacharunderscore}{\kern0pt}def{\isacharparenright}{\kern0pt}\isanewline
\ \ \ \ \ \ \ \ \isacommand{apply}\isamarkupfalse%
\ {\isacharparenleft}{\kern0pt}simp\ add{\isacharcolon}{\kern0pt}disjoint{\isacharunderscore}{\kern0pt}family{\isacharunderscore}{\kern0pt}on{\isacharunderscore}{\kern0pt}def{\isacharcomma}{\kern0pt}\ fastforce{\isacharparenright}{\kern0pt}\isanewline
\ \ \ \ \ \ \ \isacommand{apply}\isamarkupfalse%
\ {\isacharparenleft}{\kern0pt}simp\ add{\isacharcolon}{\kern0pt}s{\isadigit{2}}{\isacharunderscore}{\kern0pt}nonzero{\isacharparenright}{\kern0pt}\isanewline
\ \ \ \ \ \ \isacommand{apply}\isamarkupfalse%
\ {\isacharparenleft}{\kern0pt}rule\ subsetI{\isacharcomma}{\kern0pt}\ simp\ add{\isacharcolon}{\kern0pt}mem{\isacharunderscore}{\kern0pt}Times{\isacharunderscore}{\kern0pt}iff{\isacharparenright}{\kern0pt}\isanewline
\ \ \ \ \ \isacommand{apply}\isamarkupfalse%
\ simp\isanewline
\ \ \ \ \isacommand{by}\isamarkupfalse%
\ simp\isanewline
\isanewline
\ \ \isacommand{have}\isamarkupfalse%
\ median{\isacharunderscore}{\kern0pt}bound{\isacharunderscore}{\kern0pt}{\isadigit{3}}{\isacharcolon}{\kern0pt}\ {\isachardoublequoteopen}\ {\isacharminus}{\kern0pt}\ {\isacharparenleft}{\kern0pt}{\isadigit{1}}{\isadigit{8}}\ {\isacharasterisk}{\kern0pt}\ ln\ {\isacharparenleft}{\kern0pt}real{\isacharunderscore}{\kern0pt}of{\isacharunderscore}{\kern0pt}rat\ {\isasymepsilon}{\isacharparenright}{\kern0pt}{\isacharparenright}{\kern0pt}\ {\isasymle}\ real\ s\isactrlsub {\isadigit{2}}{\isachardoublequoteclose}\isanewline
\ \ \ \ \isacommand{apply}\isamarkupfalse%
\ {\isacharparenleft}{\kern0pt}simp\ add{\isacharcolon}{\kern0pt}s\isactrlsub {\isadigit{2}}{\isacharunderscore}{\kern0pt}def{\isacharparenright}{\kern0pt}\isanewline
\ \ \ \ \isacommand{using}\isamarkupfalse%
\ of{\isacharunderscore}{\kern0pt}nat{\isacharunderscore}{\kern0pt}ceiling\ \isacommand{by}\isamarkupfalse%
\ blast\isanewline
\isanewline
\ \ \isacommand{have}\isamarkupfalse%
\ median{\isacharunderscore}{\kern0pt}bound{\isacharunderscore}{\kern0pt}{\isadigit{4}}{\isacharcolon}{\kern0pt}\ {\isachardoublequoteopen}{\isasymAnd}i{\isachardot}{\kern0pt}\ i\ {\isacharless}{\kern0pt}\ s\isactrlsub {\isadigit{2}}\ {\isasymLongrightarrow}\isanewline
\ \ \ \ {\isasymP}{\isacharparenleft}{\kern0pt}{\isasymomega}\ in\ {\isasymOmega}\isactrlsub {\isadigit{0}}{\isachardot}{\kern0pt}\ real{\isacharunderscore}{\kern0pt}of{\isacharunderscore}{\kern0pt}rat\ {\isacharparenleft}{\kern0pt}{\isasymdelta}\ {\isacharasterisk}{\kern0pt}\ F\ {\isadigit{2}}\ as{\isacharparenright}{\kern0pt}\ {\isacharless}{\kern0pt}\ {\isasymbar}f{\isadigit{2}}\ {\isasymomega}\ i\ {\isacharminus}{\kern0pt}\ real{\isacharunderscore}{\kern0pt}of{\isacharunderscore}{\kern0pt}rat\ {\isacharparenleft}{\kern0pt}F\ {\isadigit{2}}\ as{\isacharparenright}{\kern0pt}{\isasymbar}{\isacharparenright}{\kern0pt}\ {\isasymle}\ {\isadigit{1}}{\isacharslash}{\kern0pt}{\isadigit{3}}{\isachardoublequoteclose}\isanewline
\ \ \isacommand{proof}\isamarkupfalse%
\ {\isacharminus}{\kern0pt}\isanewline
\ \ \ \ \isacommand{fix}\isamarkupfalse%
\ i\isanewline
\ \ \ \ \isacommand{assume}\isamarkupfalse%
\ a{\isacharcolon}{\kern0pt}{\isachardoublequoteopen}i\ {\isacharless}{\kern0pt}\ s\isactrlsub {\isadigit{2}}{\isachardoublequoteclose}\isanewline
\ \ \ \ \isacommand{show}\isamarkupfalse%
\ {\isachardoublequoteopen}{\isasymP}{\isacharparenleft}{\kern0pt}{\isasymomega}\ in\ {\isasymOmega}\isactrlsub {\isadigit{0}}{\isachardot}{\kern0pt}\ real{\isacharunderscore}{\kern0pt}of{\isacharunderscore}{\kern0pt}rat\ {\isacharparenleft}{\kern0pt}{\isasymdelta}\ {\isacharasterisk}{\kern0pt}\ F\ {\isadigit{2}}\ as{\isacharparenright}{\kern0pt}\ {\isacharless}{\kern0pt}\ {\isasymbar}f{\isadigit{2}}\ {\isasymomega}\ i\ {\isacharminus}{\kern0pt}\ real{\isacharunderscore}{\kern0pt}of{\isacharunderscore}{\kern0pt}rat\ {\isacharparenleft}{\kern0pt}F\ {\isadigit{2}}\ as{\isacharparenright}{\kern0pt}{\isasymbar}{\isacharparenright}{\kern0pt}\ {\isasymle}\ {\isadigit{1}}{\isacharslash}{\kern0pt}{\isadigit{3}}{\isachardoublequoteclose}\isanewline
\ \ \ \ \isacommand{proof}\isamarkupfalse%
\ {\isacharparenleft}{\kern0pt}cases\ {\isachardoublequoteopen}as\ {\isacharequal}{\kern0pt}\ {\isacharbrackleft}{\kern0pt}{\isacharbrackright}{\kern0pt}{\isachardoublequoteclose}{\isacharparenright}{\kern0pt}\isanewline
\ \ \ \ \ \ \isacommand{case}\isamarkupfalse%
\ True\isanewline
\ \ \ \ \ \ \isacommand{then}\isamarkupfalse%
\ \isacommand{show}\isamarkupfalse%
\ {\isacharquery}{\kern0pt}thesis\ \isacommand{using}\isamarkupfalse%
\ a\ \isacommand{by}\isamarkupfalse%
\ {\isacharparenleft}{\kern0pt}simp\ add{\isacharcolon}{\kern0pt}f{\isadigit{2}}{\isacharunderscore}{\kern0pt}def\ F{\isacharunderscore}{\kern0pt}def\ f{\isadigit{3}}{\isacharunderscore}{\kern0pt}def{\isacharparenright}{\kern0pt}\isanewline
\ \ \ \ \isacommand{next}\isamarkupfalse%
\isanewline
\ \ \ \ \ \ \isacommand{case}\isamarkupfalse%
\ False\isanewline
\ \ \ \ \ \ \isacommand{have}\isamarkupfalse%
\ F{\isacharunderscore}{\kern0pt}{\isadigit{2}}{\isacharunderscore}{\kern0pt}nonzero{\isacharcolon}{\kern0pt}\ {\isachardoublequoteopen}F\ {\isadigit{2}}\ as\ {\isachargreater}{\kern0pt}\ {\isadigit{0}}{\isachardoublequoteclose}\ \isacommand{using}\isamarkupfalse%
\ F{\isacharunderscore}{\kern0pt}gr{\isacharunderscore}{\kern0pt}{\isadigit{0}}{\isacharbrackleft}{\kern0pt}OF\ False{\isacharbrackright}{\kern0pt}\ \isacommand{by}\isamarkupfalse%
\ simp\isanewline
\isanewline
\ \ \ \ \ \ \isacommand{define}\isamarkupfalse%
\ var\ \isakeyword{where}\ \ {\isachardoublequoteopen}var\ {\isacharequal}{\kern0pt}\ prob{\isacharunderscore}{\kern0pt}space{\isachardot}{\kern0pt}variance\ {\isasymOmega}\isactrlsub {\isadigit{0}}\ {\isacharparenleft}{\kern0pt}{\isasymlambda}{\isasymomega}{\isachardot}{\kern0pt}\ f{\isadigit{2}}\ {\isasymomega}\ i{\isacharparenright}{\kern0pt}{\isachardoublequoteclose}\isanewline
\ \ \ \ \ \ \isacommand{have}\isamarkupfalse%
\ b{\isacharunderscore}{\kern0pt}{\isadigit{1}}{\isacharcolon}{\kern0pt}\ {\isachardoublequoteopen}real{\isacharunderscore}{\kern0pt}of{\isacharunderscore}{\kern0pt}rat\ {\isacharparenleft}{\kern0pt}F\ {\isadigit{2}}\ as{\isacharparenright}{\kern0pt}\ {\isacharequal}{\kern0pt}\ prob{\isacharunderscore}{\kern0pt}space{\isachardot}{\kern0pt}expectation\ {\isasymOmega}\isactrlsub {\isadigit{0}}\ {\isacharparenleft}{\kern0pt}{\isasymlambda}{\isasymomega}{\isachardot}{\kern0pt}\ f{\isadigit{2}}\ {\isasymomega}\ i{\isacharparenright}{\kern0pt}{\isachardoublequoteclose}\isanewline
\ \ \ \ \ \ \ \ \isacommand{using}\isamarkupfalse%
\ f{\isadigit{2}}{\isacharunderscore}{\kern0pt}exp{\isacharprime}{\kern0pt}{\isacharprime}{\kern0pt}\ a\ \isacommand{by}\isamarkupfalse%
\ metis\isanewline
\ \ \ \ \ \ \isacommand{have}\isamarkupfalse%
\ b{\isacharunderscore}{\kern0pt}{\isadigit{2}}{\isacharcolon}{\kern0pt}\ {\isachardoublequoteopen}{\isadigit{0}}\ {\isacharless}{\kern0pt}\ real{\isacharunderscore}{\kern0pt}of{\isacharunderscore}{\kern0pt}rat\ {\isacharparenleft}{\kern0pt}{\isasymdelta}\ {\isacharasterisk}{\kern0pt}\ F\ {\isadigit{2}}\ as{\isacharparenright}{\kern0pt}{\isachardoublequoteclose}\isanewline
\ \ \ \ \ \ \ \ \isacommand{using}\isamarkupfalse%
\ assms{\isacharparenleft}{\kern0pt}{\isadigit{2}}{\isacharparenright}{\kern0pt}\ F{\isacharunderscore}{\kern0pt}{\isadigit{2}}{\isacharunderscore}{\kern0pt}nonzero\ \isacommand{by}\isamarkupfalse%
\ simp\isanewline
\ \ \ \ \ \ \isacommand{have}\isamarkupfalse%
\ b{\isacharunderscore}{\kern0pt}{\isadigit{3}}{\isacharcolon}{\kern0pt}\ {\isachardoublequoteopen}integrable\ {\isasymOmega}\isactrlsub {\isadigit{0}}\ {\isacharparenleft}{\kern0pt}{\isasymlambda}{\isasymomega}{\isachardot}{\kern0pt}\ f{\isadigit{2}}\ {\isasymomega}\ i{\isacharcircum}{\kern0pt}{\isadigit{2}}{\isacharparenright}{\kern0pt}{\isachardoublequoteclose}\isanewline
\ \ \ \ \ \ \ \ \isacommand{by}\isamarkupfalse%
\ {\isacharparenleft}{\kern0pt}rule\ integrable{\isacharunderscore}{\kern0pt}measure{\isacharunderscore}{\kern0pt}pmf{\isacharunderscore}{\kern0pt}finite{\isacharbrackleft}{\kern0pt}OF\ fin{\isacharunderscore}{\kern0pt}omega{\isacharunderscore}{\kern0pt}{\isadigit{1}}{\isacharbrackright}{\kern0pt}{\isacharparenright}{\kern0pt}\isanewline
\ \ \ \ \ \ \isacommand{have}\isamarkupfalse%
\ b{\isacharunderscore}{\kern0pt}{\isadigit{4}}{\isacharcolon}{\kern0pt}\ {\isachardoublequoteopen}{\isacharparenleft}{\kern0pt}{\isasymlambda}{\isasymomega}{\isachardot}{\kern0pt}\ f{\isadigit{2}}\ {\isasymomega}\ i{\isacharparenright}{\kern0pt}\ {\isasymin}\ borel{\isacharunderscore}{\kern0pt}measurable\ {\isasymOmega}\isactrlsub {\isadigit{0}}{\isachardoublequoteclose}\isanewline
\ \ \ \ \ \ \ \ \isacommand{by}\isamarkupfalse%
\ {\isacharparenleft}{\kern0pt}simp\ add{\isacharcolon}{\kern0pt}{\isasymOmega}\isactrlsub {\isadigit{0}}{\isacharunderscore}{\kern0pt}def{\isacharparenright}{\kern0pt}\isanewline
\ \ \ \ \ \ \isacommand{have}\isamarkupfalse%
\ {\isachardoublequoteopen}{\isasymP}{\isacharparenleft}{\kern0pt}{\isasymomega}\ in\ {\isasymOmega}\isactrlsub {\isadigit{0}}{\isachardot}{\kern0pt}\ real{\isacharunderscore}{\kern0pt}of{\isacharunderscore}{\kern0pt}rat\ {\isacharparenleft}{\kern0pt}{\isasymdelta}\ {\isacharasterisk}{\kern0pt}\ F\ {\isadigit{2}}\ as{\isacharparenright}{\kern0pt}\ {\isacharless}{\kern0pt}\ {\isasymbar}f{\isadigit{2}}\ {\isasymomega}\ i\ {\isacharminus}{\kern0pt}\ real{\isacharunderscore}{\kern0pt}of{\isacharunderscore}{\kern0pt}rat\ {\isacharparenleft}{\kern0pt}F\ {\isadigit{2}}\ as{\isacharparenright}{\kern0pt}{\isasymbar}{\isacharparenright}{\kern0pt}\ {\isasymle}\ \isanewline
\ \ \ \ \ \ \ \ \ \ {\isasymP}{\isacharparenleft}{\kern0pt}{\isasymomega}\ in\ {\isasymOmega}\isactrlsub {\isadigit{0}}{\isachardot}{\kern0pt}\ real{\isacharunderscore}{\kern0pt}of{\isacharunderscore}{\kern0pt}rat\ {\isacharparenleft}{\kern0pt}{\isasymdelta}\ {\isacharasterisk}{\kern0pt}\ F\ {\isadigit{2}}\ as{\isacharparenright}{\kern0pt}\ {\isasymle}\ {\isasymbar}f{\isadigit{2}}\ {\isasymomega}\ i\ {\isacharminus}{\kern0pt}\ real{\isacharunderscore}{\kern0pt}of{\isacharunderscore}{\kern0pt}rat\ {\isacharparenleft}{\kern0pt}F\ {\isadigit{2}}\ as{\isacharparenright}{\kern0pt}{\isasymbar}{\isacharparenright}{\kern0pt}{\isachardoublequoteclose}\isanewline
\ \ \ \ \ \ \ \ \ \ \isacommand{apply}\isamarkupfalse%
\ {\isacharparenleft}{\kern0pt}simp\ add{\isacharcolon}{\kern0pt}{\isasymOmega}\isactrlsub {\isadigit{0}}{\isacharunderscore}{\kern0pt}def{\isacharparenright}{\kern0pt}\isanewline
\ \ \ \ \ \ \ \ \ \ \isacommand{apply}\isamarkupfalse%
\ {\isacharparenleft}{\kern0pt}rule\ pmf{\isacharunderscore}{\kern0pt}mono{\isacharunderscore}{\kern0pt}{\isadigit{1}}{\isacharparenright}{\kern0pt}\isanewline
\ \ \ \ \ \ \ \ \isacommand{by}\isamarkupfalse%
\ simp\ \isanewline
\ \ \ \ \ \ \isacommand{also}\isamarkupfalse%
\ \isacommand{have}\isamarkupfalse%
\ {\isachardoublequoteopen}{\isachardot}{\kern0pt}{\isachardot}{\kern0pt}{\isachardot}{\kern0pt}\ {\isasymle}\ var\ {\isacharslash}{\kern0pt}\ {\isacharparenleft}{\kern0pt}real{\isacharunderscore}{\kern0pt}of{\isacharunderscore}{\kern0pt}rat\ {\isacharparenleft}{\kern0pt}{\isasymdelta}\ {\isacharasterisk}{\kern0pt}\ F\ {\isadigit{2}}\ as{\isacharparenright}{\kern0pt}{\isacharparenright}{\kern0pt}\isactrlsup {\isadigit{2}}{\isachardoublequoteclose}\isanewline
\ \ \ \ \ \ \ \ \isacommand{using}\isamarkupfalse%
\ prob{\isacharunderscore}{\kern0pt}space{\isachardot}{\kern0pt}Chebyshev{\isacharunderscore}{\kern0pt}inequality{\isacharbrackleft}{\kern0pt}\isakeyword{where}\ M{\isacharequal}{\kern0pt}{\isachardoublequoteopen}{\isasymOmega}\isactrlsub {\isadigit{0}}{\isachardoublequoteclose}\ \isakeyword{and}\ a{\isacharequal}{\kern0pt}{\isachardoublequoteopen}real{\isacharunderscore}{\kern0pt}of{\isacharunderscore}{\kern0pt}rat\ {\isacharparenleft}{\kern0pt}{\isasymdelta}\ {\isacharasterisk}{\kern0pt}\ F\ {\isadigit{2}}\ as{\isacharparenright}{\kern0pt}{\isachardoublequoteclose}\isanewline
\ \ \ \ \ \ \ \ \ \ \ \ \isakeyword{and}\ f{\isacharequal}{\kern0pt}{\isachardoublequoteopen}{\isasymlambda}{\isasymomega}{\isachardot}{\kern0pt}\ f{\isadigit{2}}\ {\isasymomega}\ i{\isachardoublequoteclose}{\isacharcomma}{\kern0pt}simplified{\isacharbrackright}{\kern0pt}\ assms{\isacharparenleft}{\kern0pt}{\isadigit{2}}{\isacharparenright}{\kern0pt}\ prob{\isacharunderscore}{\kern0pt}space{\isacharunderscore}{\kern0pt}measure{\isacharunderscore}{\kern0pt}pmf{\isacharbrackleft}{\kern0pt}\isakeyword{where}\ p{\isacharequal}{\kern0pt}{\isachardoublequoteopen}{\isasymOmega}\isactrlsub {\isadigit{0}}{\isachardoublequoteclose}{\isacharbrackright}{\kern0pt}\ F{\isacharunderscore}{\kern0pt}{\isadigit{2}}{\isacharunderscore}{\kern0pt}nonzero\isanewline
\ \ \ \ \ \ \ \ \ \ b{\isacharunderscore}{\kern0pt}{\isadigit{1}}\ b{\isacharunderscore}{\kern0pt}{\isadigit{2}}\ b{\isacharunderscore}{\kern0pt}{\isadigit{3}}\ b{\isacharunderscore}{\kern0pt}{\isadigit{4}}\ \isacommand{by}\isamarkupfalse%
\ {\isacharparenleft}{\kern0pt}simp\ add{\isacharcolon}{\kern0pt}var{\isacharunderscore}{\kern0pt}def{\isacharparenright}{\kern0pt}\isanewline
\ \ \ \ \ \ \isacommand{also}\isamarkupfalse%
\ \ \isacommand{have}\isamarkupfalse%
\ {\isachardoublequoteopen}{\isachardot}{\kern0pt}{\isachardot}{\kern0pt}{\isachardot}{\kern0pt}\ {\isasymle}\ {\isadigit{1}}{\isacharslash}{\kern0pt}{\isadigit{3}}{\isachardoublequoteclose}\ {\isacharparenleft}{\kern0pt}\isakeyword{is}\ {\isacharquery}{\kern0pt}ths{\isacharparenright}{\kern0pt}\isanewline
\ \ \ \ \ \ \ \ \isacommand{apply}\isamarkupfalse%
\ {\isacharparenleft}{\kern0pt}subst\ pos{\isacharunderscore}{\kern0pt}divide{\isacharunderscore}{\kern0pt}le{\isacharunderscore}{\kern0pt}eq\ {\isacharparenright}{\kern0pt}\isanewline
\ \ \ \ \ \ \ \ \isacommand{using}\isamarkupfalse%
\ F{\isacharunderscore}{\kern0pt}{\isadigit{2}}{\isacharunderscore}{\kern0pt}nonzero\ assms{\isacharparenleft}{\kern0pt}{\isadigit{2}}{\isacharparenright}{\kern0pt}\ \isacommand{apply}\isamarkupfalse%
\ simp\isanewline
\ \ \ \ \ \ \ \ \isacommand{apply}\isamarkupfalse%
\ {\isacharparenleft}{\kern0pt}simp\ add{\isacharcolon}{\kern0pt}var{\isacharunderscore}{\kern0pt}def{\isacharparenright}{\kern0pt}\isanewline
\ \ \ \ \ \ \ \ \isacommand{using}\isamarkupfalse%
\ f{\isadigit{2}}{\isacharunderscore}{\kern0pt}var{\isacharprime}{\kern0pt}{\isacharprime}{\kern0pt}\ a\ \isacommand{by}\isamarkupfalse%
\ fastforce\isanewline
\ \ \ \ \ \ \isacommand{finally}\isamarkupfalse%
\ \isacommand{show}\isamarkupfalse%
\ {\isacharquery}{\kern0pt}thesis\isanewline
\ \ \ \ \ \ \ \ \isacommand{by}\isamarkupfalse%
\ blast\isanewline
\ \ \ \ \isacommand{qed}\isamarkupfalse%
\isanewline
\ \ \isacommand{qed}\isamarkupfalse%
\isanewline
\isanewline
\ \ \isacommand{show}\isamarkupfalse%
\ {\isacharquery}{\kern0pt}thesis\isanewline
\ \ \ \ \isacommand{apply}\isamarkupfalse%
\ {\isacharparenleft}{\kern0pt}simp\ add{\isacharcolon}{\kern0pt}\ distr{\isacharprime}{\kern0pt}\ e\ real{\isacharunderscore}{\kern0pt}f\ f{\isacharprime}{\kern0pt}{\isacharunderscore}{\kern0pt}def\ g{\isacharunderscore}{\kern0pt}def\ {\isasymOmega}\isactrlsub {\isadigit{0}}{\isacharunderscore}{\kern0pt}def{\isacharbrackleft}{\kern0pt}symmetric{\isacharbrackright}{\kern0pt}{\isacharparenright}{\kern0pt}\isanewline
\ \ \ \ \isacommand{apply}\isamarkupfalse%
\ {\isacharparenleft}{\kern0pt}rule\ prob{\isacharunderscore}{\kern0pt}space{\isachardot}{\kern0pt}median{\isacharunderscore}{\kern0pt}bound{\isacharunderscore}{\kern0pt}{\isadigit{2}}{\isacharbrackleft}{\kern0pt}\isakeyword{where}\ M{\isacharequal}{\kern0pt}{\isachardoublequoteopen}{\isasymOmega}\isactrlsub {\isadigit{0}}{\isachardoublequoteclose}\ \isakeyword{and}\ {\isasymepsilon}{\isacharequal}{\kern0pt}{\isachardoublequoteopen}real{\isacharunderscore}{\kern0pt}of{\isacharunderscore}{\kern0pt}rat\ {\isasymepsilon}{\isachardoublequoteclose}\ \isakeyword{and}\ X{\isacharequal}{\kern0pt}{\isachardoublequoteopen}{\isacharparenleft}{\kern0pt}{\isasymlambda}i\ {\isasymomega}{\isachardot}{\kern0pt}\ f{\isadigit{2}}\ {\isasymomega}\ i{\isacharparenright}{\kern0pt}{\isachardoublequoteclose}{\isacharcomma}{\kern0pt}\ simplified{\isacharbrackright}{\kern0pt}{\isacharparenright}{\kern0pt}\isanewline
\ \ \ \ \ \ \ \ \isacommand{apply}\isamarkupfalse%
\ {\isacharparenleft}{\kern0pt}metis\ prob{\isacharunderscore}{\kern0pt}space{\isacharunderscore}{\kern0pt}measure{\isacharunderscore}{\kern0pt}pmf{\isacharparenright}{\kern0pt}\isanewline
\ \ \ \ \ \ \ \isacommand{using}\isamarkupfalse%
\ assms\ \isacommand{apply}\isamarkupfalse%
\ simp\ \isanewline
\ \ \ \ \ \ \isacommand{apply}\isamarkupfalse%
\ {\isacharparenleft}{\kern0pt}metis\ median{\isacharunderscore}{\kern0pt}bound{\isacharunderscore}{\kern0pt}{\isadigit{2}}{\isacharprime}{\kern0pt}{\isacharparenright}{\kern0pt}\isanewline
\ \ \ \ \ \isacommand{apply}\isamarkupfalse%
\ {\isacharparenleft}{\kern0pt}metis\ median{\isacharunderscore}{\kern0pt}bound{\isacharunderscore}{\kern0pt}{\isadigit{3}}{\isacharparenright}{\kern0pt}\isanewline
\ \ \ \ \isacommand{using}\isamarkupfalse%
\ median{\isacharunderscore}{\kern0pt}bound{\isacharunderscore}{\kern0pt}{\isadigit{4}}\ \isacommand{by}\isamarkupfalse%
\ simp\isanewline
\isacommand{qed}\isamarkupfalse%
%
\endisatagproof
{\isafoldproof}%
%
\isadelimproof
\isanewline
%
\endisadelimproof
\isanewline
\isacommand{fun}\isamarkupfalse%
\ f{\isadigit{2}}{\isacharunderscore}{\kern0pt}space{\isacharunderscore}{\kern0pt}usage\ {\isacharcolon}{\kern0pt}{\isacharcolon}{\kern0pt}\ {\isachardoublequoteopen}{\isacharparenleft}{\kern0pt}nat\ {\isasymtimes}\ nat\ {\isasymtimes}\ rat\ {\isasymtimes}\ rat{\isacharparenright}{\kern0pt}\ {\isasymRightarrow}\ real{\isachardoublequoteclose}\ \isakeyword{where}\isanewline
\ \ {\isachardoublequoteopen}f{\isadigit{2}}{\isacharunderscore}{\kern0pt}space{\isacharunderscore}{\kern0pt}usage\ {\isacharparenleft}{\kern0pt}n{\isacharcomma}{\kern0pt}\ m{\isacharcomma}{\kern0pt}\ {\isasymepsilon}{\isacharcomma}{\kern0pt}\ {\isasymdelta}{\isacharparenright}{\kern0pt}\ {\isacharequal}{\kern0pt}\ {\isacharparenleft}{\kern0pt}\isanewline
\ \ \ \ let\ s\isactrlsub {\isadigit{1}}\ {\isacharequal}{\kern0pt}\ nat\ {\isasymlceil}{\isadigit{6}}\ {\isacharslash}{\kern0pt}\ {\isasymdelta}\isactrlsup {\isadigit{2}}\ {\isasymrceil}\ in\isanewline
\ \ \ \ let\ s\isactrlsub {\isadigit{2}}\ {\isacharequal}{\kern0pt}\ nat\ {\isasymlceil}{\isacharminus}{\kern0pt}{\isacharparenleft}{\kern0pt}{\isadigit{1}}{\isadigit{8}}\ {\isacharasterisk}{\kern0pt}\ ln\ {\isacharparenleft}{\kern0pt}real{\isacharunderscore}{\kern0pt}of{\isacharunderscore}{\kern0pt}rat\ {\isasymepsilon}{\isacharparenright}{\kern0pt}{\isacharparenright}{\kern0pt}{\isasymrceil}\ in\ \isanewline
\ \ \ \ {\isadigit{5}}\ {\isacharplus}{\kern0pt}\isanewline
\ \ \ \ {\isadigit{2}}\ {\isacharasterisk}{\kern0pt}\ log\ {\isadigit{2}}\ {\isacharparenleft}{\kern0pt}s\isactrlsub {\isadigit{1}}\ {\isacharplus}{\kern0pt}\ {\isadigit{1}}{\isacharparenright}{\kern0pt}\ {\isacharplus}{\kern0pt}\isanewline
\ \ \ \ {\isadigit{2}}\ {\isacharasterisk}{\kern0pt}\ log\ {\isadigit{2}}\ {\isacharparenleft}{\kern0pt}s\isactrlsub {\isadigit{2}}\ {\isacharplus}{\kern0pt}\ {\isadigit{1}}{\isacharparenright}{\kern0pt}\ {\isacharplus}{\kern0pt}\isanewline
\ \ \ \ {\isadigit{2}}\ {\isacharasterisk}{\kern0pt}\ log\ {\isadigit{2}}\ {\isacharparenleft}{\kern0pt}{\isadigit{4}}\ {\isacharplus}{\kern0pt}\ {\isadigit{2}}\ {\isacharasterisk}{\kern0pt}\ real\ n{\isacharparenright}{\kern0pt}\ {\isacharplus}{\kern0pt}\isanewline
\ \ \ \ s\isactrlsub {\isadigit{1}}\ {\isacharasterisk}{\kern0pt}\ s\isactrlsub {\isadigit{2}}\ {\isacharasterisk}{\kern0pt}\ {\isacharparenleft}{\kern0pt}{\isadigit{1}}{\isadigit{3}}\ {\isacharplus}{\kern0pt}\ {\isadigit{8}}\ {\isacharasterisk}{\kern0pt}\ log\ {\isadigit{2}}\ {\isacharparenleft}{\kern0pt}{\isadigit{4}}\ {\isacharplus}{\kern0pt}\ {\isadigit{2}}\ {\isacharasterisk}{\kern0pt}\ real\ n{\isacharparenright}{\kern0pt}\ {\isacharplus}{\kern0pt}\ {\isadigit{2}}\ {\isacharasterisk}{\kern0pt}\ log\ {\isadigit{2}}\ {\isacharparenleft}{\kern0pt}real\ m\ {\isacharasterisk}{\kern0pt}\ {\isacharparenleft}{\kern0pt}{\isadigit{4}}\ {\isacharplus}{\kern0pt}\ {\isadigit{2}}\ {\isacharasterisk}{\kern0pt}\ real\ n{\isacharparenright}{\kern0pt}\ {\isacharplus}{\kern0pt}\ {\isadigit{1}}\ {\isacharparenright}{\kern0pt}{\isacharparenright}{\kern0pt}{\isacharparenright}{\kern0pt}{\isachardoublequoteclose}\isanewline
\isanewline
\isacommand{definition}\isamarkupfalse%
\ encode{\isacharunderscore}{\kern0pt}state\ \isakeyword{where}\isanewline
\ \ {\isachardoublequoteopen}encode{\isacharunderscore}{\kern0pt}state\ {\isacharequal}{\kern0pt}\ \isanewline
\ \ \ \ N\isactrlsub S\ {\isasymtimes}\isactrlsub D\ {\isacharparenleft}{\kern0pt}{\isasymlambda}s\isactrlsub {\isadigit{1}}{\isachardot}{\kern0pt}\ \isanewline
\ \ \ \ N\isactrlsub S\ {\isasymtimes}\isactrlsub D\ {\isacharparenleft}{\kern0pt}{\isasymlambda}s\isactrlsub {\isadigit{2}}{\isachardot}{\kern0pt}\ \isanewline
\ \ \ \ N\isactrlsub S\ {\isasymtimes}\isactrlsub D\ {\isacharparenleft}{\kern0pt}{\isasymlambda}p{\isachardot}{\kern0pt}\ \isanewline
\ \ \ \ {\isacharparenleft}{\kern0pt}List{\isachardot}{\kern0pt}product\ {\isacharbrackleft}{\kern0pt}{\isadigit{0}}{\isachardot}{\kern0pt}{\isachardot}{\kern0pt}{\isacharless}{\kern0pt}s\isactrlsub {\isadigit{1}}{\isacharbrackright}{\kern0pt}\ {\isacharbrackleft}{\kern0pt}{\isadigit{0}}{\isachardot}{\kern0pt}{\isachardot}{\kern0pt}{\isacharless}{\kern0pt}s\isactrlsub {\isadigit{2}}{\isacharbrackright}{\kern0pt}\ {\isasymrightarrow}\isactrlsub S\ {\isacharparenleft}{\kern0pt}list\isactrlsub S\ {\isacharparenleft}{\kern0pt}zfact\isactrlsub S\ p{\isacharparenright}{\kern0pt}{\isacharparenright}{\kern0pt}{\isacharparenright}{\kern0pt}\ {\isasymtimes}\isactrlsub S\isanewline
\ \ \ \ {\isacharparenleft}{\kern0pt}List{\isachardot}{\kern0pt}product\ {\isacharbrackleft}{\kern0pt}{\isadigit{0}}{\isachardot}{\kern0pt}{\isachardot}{\kern0pt}{\isacharless}{\kern0pt}s\isactrlsub {\isadigit{1}}{\isacharbrackright}{\kern0pt}\ {\isacharbrackleft}{\kern0pt}{\isadigit{0}}{\isachardot}{\kern0pt}{\isachardot}{\kern0pt}{\isacharless}{\kern0pt}s\isactrlsub {\isadigit{2}}{\isacharbrackright}{\kern0pt}\ {\isasymrightarrow}\isactrlsub S\ I\isactrlsub S{\isacharparenright}{\kern0pt}{\isacharparenright}{\kern0pt}{\isacharparenright}{\kern0pt}{\isacharparenright}{\kern0pt}{\isachardoublequoteclose}\isanewline
\isanewline
\isacommand{lemma}\isamarkupfalse%
\ {\isachardoublequoteopen}inj{\isacharunderscore}{\kern0pt}on\ encode{\isacharunderscore}{\kern0pt}state\ {\isacharparenleft}{\kern0pt}dom\ encode{\isacharunderscore}{\kern0pt}state{\isacharparenright}{\kern0pt}{\isachardoublequoteclose}\isanewline
%
\isadelimproof
\ \ %
\endisadelimproof
%
\isatagproof
\isacommand{apply}\isamarkupfalse%
\ {\isacharparenleft}{\kern0pt}rule\ encoding{\isacharunderscore}{\kern0pt}imp{\isacharunderscore}{\kern0pt}inj{\isacharparenright}{\kern0pt}\isanewline
\ \ \isacommand{apply}\isamarkupfalse%
\ {\isacharparenleft}{\kern0pt}simp\ add{\isacharcolon}{\kern0pt}encode{\isacharunderscore}{\kern0pt}state{\isacharunderscore}{\kern0pt}def{\isacharparenright}{\kern0pt}\isanewline
\ \ \isacommand{apply}\isamarkupfalse%
\ {\isacharparenleft}{\kern0pt}rule\ dependent{\isacharunderscore}{\kern0pt}encoding{\isacharcomma}{\kern0pt}\ metis\ nat{\isacharunderscore}{\kern0pt}encoding{\isacharparenright}{\kern0pt}\isanewline
\ \ \isacommand{apply}\isamarkupfalse%
\ {\isacharparenleft}{\kern0pt}rule\ dependent{\isacharunderscore}{\kern0pt}encoding{\isacharcomma}{\kern0pt}\ metis\ nat{\isacharunderscore}{\kern0pt}encoding{\isacharparenright}{\kern0pt}\isanewline
\ \ \isacommand{apply}\isamarkupfalse%
\ {\isacharparenleft}{\kern0pt}rule\ dependent{\isacharunderscore}{\kern0pt}encoding{\isacharcomma}{\kern0pt}\ metis\ nat{\isacharunderscore}{\kern0pt}encoding{\isacharparenright}{\kern0pt}\isanewline
\ \ \isacommand{apply}\isamarkupfalse%
\ {\isacharparenleft}{\kern0pt}rule\ prod{\isacharunderscore}{\kern0pt}encoding{\isacharcomma}{\kern0pt}\ metis\ encode{\isacharunderscore}{\kern0pt}extensional\ list{\isacharunderscore}{\kern0pt}encoding\ zfact{\isacharunderscore}{\kern0pt}encoding{\isacharparenright}{\kern0pt}\isanewline
\ \ \isacommand{by}\isamarkupfalse%
\ {\isacharparenleft}{\kern0pt}metis\ encode{\isacharunderscore}{\kern0pt}extensional\ int{\isacharunderscore}{\kern0pt}encoding{\isacharparenright}{\kern0pt}%
\endisatagproof
{\isafoldproof}%
%
\isadelimproof
\isanewline
%
\endisadelimproof
\isanewline
\isacommand{theorem}\isamarkupfalse%
\ f{\isadigit{2}}{\isacharunderscore}{\kern0pt}exact{\isacharunderscore}{\kern0pt}space{\isacharunderscore}{\kern0pt}usage{\isacharcolon}{\kern0pt}\isanewline
\ \ \isakeyword{assumes}\ {\isachardoublequoteopen}{\isasymepsilon}\ {\isasymin}\ {\isacharbraceleft}{\kern0pt}{\isadigit{0}}{\isacharless}{\kern0pt}{\isachardot}{\kern0pt}{\isachardot}{\kern0pt}{\isacharless}{\kern0pt}{\isadigit{1}}{\isacharbraceright}{\kern0pt}{\isachardoublequoteclose}\isanewline
\ \ \isakeyword{assumes}\ {\isachardoublequoteopen}{\isasymdelta}\ {\isachargreater}{\kern0pt}\ {\isadigit{0}}{\isachardoublequoteclose}\isanewline
\ \ \isakeyword{assumes}\ {\isachardoublequoteopen}set\ as\ {\isasymsubseteq}\ {\isacharbraceleft}{\kern0pt}{\isadigit{0}}{\isachardot}{\kern0pt}{\isachardot}{\kern0pt}{\isacharless}{\kern0pt}n{\isacharbraceright}{\kern0pt}{\isachardoublequoteclose}\isanewline
\ \ \isakeyword{defines}\ {\isachardoublequoteopen}M\ {\isasymequiv}\ fold\ {\isacharparenleft}{\kern0pt}{\isasymlambda}a\ state{\isachardot}{\kern0pt}\ state\ {\isasymbind}\ f{\isadigit{2}}{\isacharunderscore}{\kern0pt}update\ a{\isacharparenright}{\kern0pt}\ as\ {\isacharparenleft}{\kern0pt}f{\isadigit{2}}{\isacharunderscore}{\kern0pt}init\ {\isasymdelta}\ {\isasymepsilon}\ n{\isacharparenright}{\kern0pt}{\isachardoublequoteclose}\isanewline
\ \ \isakeyword{shows}\ {\isachardoublequoteopen}AE\ {\isasymomega}\ in\ M{\isachardot}{\kern0pt}\ bit{\isacharunderscore}{\kern0pt}count\ {\isacharparenleft}{\kern0pt}encode{\isacharunderscore}{\kern0pt}state\ {\isasymomega}{\isacharparenright}{\kern0pt}\ {\isasymle}\ f{\isadigit{2}}{\isacharunderscore}{\kern0pt}space{\isacharunderscore}{\kern0pt}usage\ {\isacharparenleft}{\kern0pt}n{\isacharcomma}{\kern0pt}\ length\ as{\isacharcomma}{\kern0pt}\ {\isasymepsilon}{\isacharcomma}{\kern0pt}\ {\isasymdelta}{\isacharparenright}{\kern0pt}{\isachardoublequoteclose}\isanewline
%
\isadelimproof
%
\endisadelimproof
%
\isatagproof
\isacommand{proof}\isamarkupfalse%
\ {\isacharminus}{\kern0pt}\isanewline
\ \ \isacommand{define}\isamarkupfalse%
\ s\isactrlsub {\isadigit{1}}\ \isakeyword{where}\ {\isachardoublequoteopen}s\isactrlsub {\isadigit{1}}\ {\isacharequal}{\kern0pt}\ nat\ {\isasymlceil}{\isadigit{6}}\ {\isacharslash}{\kern0pt}\ {\isasymdelta}\isactrlsup {\isadigit{2}}{\isasymrceil}{\isachardoublequoteclose}\isanewline
\ \ \isacommand{define}\isamarkupfalse%
\ s\isactrlsub {\isadigit{2}}\ \isakeyword{where}\ {\isachardoublequoteopen}s\isactrlsub {\isadigit{2}}\ {\isacharequal}{\kern0pt}\ nat\ {\isasymlceil}{\isacharminus}{\kern0pt}{\isacharparenleft}{\kern0pt}{\isadigit{1}}{\isadigit{8}}\ {\isacharasterisk}{\kern0pt}\ ln\ {\isacharparenleft}{\kern0pt}real{\isacharunderscore}{\kern0pt}of{\isacharunderscore}{\kern0pt}rat\ {\isasymepsilon}{\isacharparenright}{\kern0pt}{\isacharparenright}{\kern0pt}{\isasymrceil}{\isachardoublequoteclose}\isanewline
\ \ \isacommand{define}\isamarkupfalse%
\ p\ \isakeyword{where}\ {\isachardoublequoteopen}p\ {\isacharequal}{\kern0pt}\ find{\isacharunderscore}{\kern0pt}prime{\isacharunderscore}{\kern0pt}above\ {\isacharparenleft}{\kern0pt}max\ n\ {\isadigit{3}}{\isacharparenright}{\kern0pt}{\isachardoublequoteclose}\isanewline
\isanewline
\ \ \isacommand{have}\isamarkupfalse%
\ find{\isacharunderscore}{\kern0pt}prime{\isacharunderscore}{\kern0pt}above{\isacharunderscore}{\kern0pt}{\isadigit{3}}{\isacharcolon}{\kern0pt}\ {\isachardoublequoteopen}find{\isacharunderscore}{\kern0pt}prime{\isacharunderscore}{\kern0pt}above\ {\isadigit{3}}\ {\isacharequal}{\kern0pt}\ {\isadigit{3}}{\isachardoublequoteclose}\ \isanewline
\ \ \ \ \isacommand{by}\isamarkupfalse%
\ {\isacharparenleft}{\kern0pt}simp\ add{\isacharcolon}{\kern0pt}find{\isacharunderscore}{\kern0pt}prime{\isacharunderscore}{\kern0pt}above{\isachardot}{\kern0pt}simps{\isacharparenright}{\kern0pt}\isanewline
\isanewline
\ \ \isacommand{have}\isamarkupfalse%
\ p{\isacharunderscore}{\kern0pt}ge{\isacharunderscore}{\kern0pt}{\isadigit{0}}{\isacharcolon}{\kern0pt}\ {\isachardoublequoteopen}p\ {\isachargreater}{\kern0pt}\ {\isadigit{0}}{\isachardoublequoteclose}\ \isanewline
\ \ \ \ \isacommand{by}\isamarkupfalse%
\ {\isacharparenleft}{\kern0pt}metis\ find{\isacharunderscore}{\kern0pt}prime{\isacharunderscore}{\kern0pt}above{\isacharunderscore}{\kern0pt}min\ p{\isacharunderscore}{\kern0pt}def\ gr{\isadigit{0}}I\ not{\isacharunderscore}{\kern0pt}numeral{\isacharunderscore}{\kern0pt}le{\isacharunderscore}{\kern0pt}zero{\isacharparenright}{\kern0pt}\isanewline
\ \ \isacommand{have}\isamarkupfalse%
\ p{\isacharunderscore}{\kern0pt}le{\isacharunderscore}{\kern0pt}n{\isacharcolon}{\kern0pt}\ {\isachardoublequoteopen}p\ {\isasymle}\ {\isadigit{2}}\ {\isacharasterisk}{\kern0pt}\ n\ {\isacharplus}{\kern0pt}\ {\isadigit{3}}{\isachardoublequoteclose}\ \isanewline
\ \ \ \ \isacommand{apply}\isamarkupfalse%
\ {\isacharparenleft}{\kern0pt}cases\ {\isachardoublequoteopen}n\ {\isasymle}\ {\isadigit{3}}{\isachardoublequoteclose}{\isacharparenright}{\kern0pt}\isanewline
\ \ \ \ \isacommand{apply}\isamarkupfalse%
\ {\isacharparenleft}{\kern0pt}simp\ add{\isacharcolon}{\kern0pt}\ p{\isacharunderscore}{\kern0pt}def\ find{\isacharunderscore}{\kern0pt}prime{\isacharunderscore}{\kern0pt}above{\isacharunderscore}{\kern0pt}{\isadigit{3}}{\isacharparenright}{\kern0pt}\ \isanewline
\ \ \ \ \isacommand{apply}\isamarkupfalse%
\ {\isacharparenleft}{\kern0pt}simp\ add{\isacharcolon}{\kern0pt}\ p{\isacharunderscore}{\kern0pt}def{\isacharparenright}{\kern0pt}\ \isanewline
\ \ \ \ \isacommand{by}\isamarkupfalse%
\ {\isacharparenleft}{\kern0pt}metis\ One{\isacharunderscore}{\kern0pt}nat{\isacharunderscore}{\kern0pt}def\ find{\isacharunderscore}{\kern0pt}prime{\isacharunderscore}{\kern0pt}above{\isacharunderscore}{\kern0pt}upper{\isacharunderscore}{\kern0pt}bound\ Suc{\isacharunderscore}{\kern0pt}{\isadigit{1}}\ add{\isacharunderscore}{\kern0pt}Suc{\isacharunderscore}{\kern0pt}right\ linear\ not{\isacharunderscore}{\kern0pt}less{\isacharunderscore}{\kern0pt}eq{\isacharunderscore}{\kern0pt}eq\ numeral{\isacharunderscore}{\kern0pt}{\isadigit{3}}{\isacharunderscore}{\kern0pt}eq{\isacharunderscore}{\kern0pt}{\isadigit{3}}{\isacharparenright}{\kern0pt}\isanewline
\isanewline
\ \ \isacommand{have}\isamarkupfalse%
\ a{\isacharcolon}{\kern0pt}\ {\isachardoublequoteopen}{\isasymAnd}y{\isachardot}{\kern0pt}\ y{\isasymin}{\isacharbraceleft}{\kern0pt}{\isadigit{0}}{\isachardot}{\kern0pt}{\isachardot}{\kern0pt}{\isacharless}{\kern0pt}s\isactrlsub {\isadigit{1}}{\isacharbraceright}{\kern0pt}\ {\isasymtimes}\ {\isacharbraceleft}{\kern0pt}{\isadigit{0}}{\isachardot}{\kern0pt}{\isachardot}{\kern0pt}{\isacharless}{\kern0pt}s\isactrlsub {\isadigit{2}}{\isacharbraceright}{\kern0pt}\ {\isasymrightarrow}\isactrlsub E\ bounded{\isacharunderscore}{\kern0pt}degree{\isacharunderscore}{\kern0pt}polynomials\ {\isacharparenleft}{\kern0pt}ZFact\ {\isacharparenleft}{\kern0pt}int\ p{\isacharparenright}{\kern0pt}{\isacharparenright}{\kern0pt}\ {\isadigit{4}}\ {\isasymLongrightarrow}\isanewline
\ \ \ \ \ \ \ bit{\isacharunderscore}{\kern0pt}count\ {\isacharparenleft}{\kern0pt}encode{\isacharunderscore}{\kern0pt}state\ {\isacharparenleft}{\kern0pt}s\isactrlsub {\isadigit{1}}{\isacharcomma}{\kern0pt}\ s\isactrlsub {\isadigit{2}}{\isacharcomma}{\kern0pt}\ p{\isacharcomma}{\kern0pt}\ y{\isacharcomma}{\kern0pt}\ {\isasymlambda}i{\isasymin}{\isacharbraceleft}{\kern0pt}{\isadigit{0}}{\isachardot}{\kern0pt}{\isachardot}{\kern0pt}{\isacharless}{\kern0pt}s\isactrlsub {\isadigit{1}}{\isacharbraceright}{\kern0pt}\ {\isasymtimes}\ {\isacharbraceleft}{\kern0pt}{\isadigit{0}}{\isachardot}{\kern0pt}{\isachardot}{\kern0pt}{\isacharless}{\kern0pt}s\isactrlsub {\isadigit{2}}{\isacharbraceright}{\kern0pt}{\isachardot}{\kern0pt}\ \isanewline
\ \ \ \ \ \ sum{\isacharunderscore}{\kern0pt}list\ {\isacharparenleft}{\kern0pt}map\ {\isacharparenleft}{\kern0pt}f{\isadigit{2}}{\isacharunderscore}{\kern0pt}hash\ p\ {\isacharparenleft}{\kern0pt}y\ i{\isacharparenright}{\kern0pt}{\isacharparenright}{\kern0pt}\ as{\isacharparenright}{\kern0pt}{\isacharparenright}{\kern0pt}{\isacharparenright}{\kern0pt}\isanewline
\ \ \ \ \ \ \ {\isasymle}\ ereal\ {\isacharparenleft}{\kern0pt}f{\isadigit{2}}{\isacharunderscore}{\kern0pt}space{\isacharunderscore}{\kern0pt}usage\ {\isacharparenleft}{\kern0pt}n{\isacharcomma}{\kern0pt}\ length\ as{\isacharcomma}{\kern0pt}\ {\isasymepsilon}{\isacharcomma}{\kern0pt}\ {\isasymdelta}{\isacharparenright}{\kern0pt}{\isacharparenright}{\kern0pt}{\isachardoublequoteclose}\isanewline
\ \ \isacommand{proof}\isamarkupfalse%
\ {\isacharminus}{\kern0pt}\isanewline
\ \ \ \ \isacommand{fix}\isamarkupfalse%
\ y\isanewline
\ \ \ \ \isacommand{assume}\isamarkupfalse%
\ a{\isacharunderscore}{\kern0pt}{\isadigit{1}}{\isacharcolon}{\kern0pt}{\isachardoublequoteopen}y\ {\isasymin}\ {\isacharbraceleft}{\kern0pt}{\isadigit{0}}{\isachardot}{\kern0pt}{\isachardot}{\kern0pt}{\isacharless}{\kern0pt}s\isactrlsub {\isadigit{1}}{\isacharbraceright}{\kern0pt}\ {\isasymtimes}\ {\isacharbraceleft}{\kern0pt}{\isadigit{0}}{\isachardot}{\kern0pt}{\isachardot}{\kern0pt}{\isacharless}{\kern0pt}s\isactrlsub {\isadigit{2}}{\isacharbraceright}{\kern0pt}\ {\isasymrightarrow}\isactrlsub E\ bounded{\isacharunderscore}{\kern0pt}degree{\isacharunderscore}{\kern0pt}polynomials\ {\isacharparenleft}{\kern0pt}ZFact\ {\isacharparenleft}{\kern0pt}int\ p{\isacharparenright}{\kern0pt}{\isacharparenright}{\kern0pt}\ {\isadigit{4}}{\isachardoublequoteclose}\isanewline
\isanewline
\ \ \ \ \isacommand{have}\isamarkupfalse%
\ a{\isacharunderscore}{\kern0pt}{\isadigit{2}}{\isacharcolon}{\kern0pt}\ {\isachardoublequoteopen}y\ {\isasymin}\ extensional\ {\isacharparenleft}{\kern0pt}{\isacharbraceleft}{\kern0pt}{\isadigit{0}}{\isachardot}{\kern0pt}{\isachardot}{\kern0pt}{\isacharless}{\kern0pt}s\isactrlsub {\isadigit{1}}{\isacharbraceright}{\kern0pt}\ {\isasymtimes}\ {\isacharbraceleft}{\kern0pt}{\isadigit{0}}{\isachardot}{\kern0pt}{\isachardot}{\kern0pt}{\isacharless}{\kern0pt}s\isactrlsub {\isadigit{2}}{\isacharbraceright}{\kern0pt}{\isacharparenright}{\kern0pt}{\isachardoublequoteclose}\ \isacommand{using}\isamarkupfalse%
\ a{\isacharunderscore}{\kern0pt}{\isadigit{1}}\ \ PiE{\isacharunderscore}{\kern0pt}iff\ \isacommand{by}\isamarkupfalse%
\ blast\isanewline
\isanewline
\ \ \ \ \isacommand{have}\isamarkupfalse%
\ a{\isacharunderscore}{\kern0pt}{\isadigit{3}}{\isacharcolon}{\kern0pt}\ {\isachardoublequoteopen}{\isasymAnd}x{\isachardot}{\kern0pt}\ x\ {\isasymin}\ y\ {\isacharbackquote}{\kern0pt}\ {\isacharparenleft}{\kern0pt}{\isacharbraceleft}{\kern0pt}{\isadigit{0}}{\isachardot}{\kern0pt}{\isachardot}{\kern0pt}{\isacharless}{\kern0pt}s\isactrlsub {\isadigit{1}}{\isacharbraceright}{\kern0pt}\ {\isasymtimes}\ {\isacharbraceleft}{\kern0pt}{\isadigit{0}}{\isachardot}{\kern0pt}{\isachardot}{\kern0pt}{\isacharless}{\kern0pt}s\isactrlsub {\isadigit{2}}{\isacharbraceright}{\kern0pt}{\isacharparenright}{\kern0pt}\ {\isasymLongrightarrow}\ bit{\isacharunderscore}{\kern0pt}count\ {\isacharparenleft}{\kern0pt}list\isactrlsub S\ {\isacharparenleft}{\kern0pt}zfact\isactrlsub S\ p{\isacharparenright}{\kern0pt}\ x{\isacharparenright}{\kern0pt}\ \isanewline
\ \ \ \ \ \ {\isasymle}\ ereal\ {\isacharparenleft}{\kern0pt}{\isadigit{9}}\ {\isacharplus}{\kern0pt}\ {\isadigit{8}}\ {\isacharasterisk}{\kern0pt}\ log\ {\isadigit{2}}\ {\isacharparenleft}{\kern0pt}{\isadigit{4}}\ {\isacharplus}{\kern0pt}\ {\isadigit{2}}\ {\isacharasterisk}{\kern0pt}\ real\ n{\isacharparenright}{\kern0pt}{\isacharparenright}{\kern0pt}{\isachardoublequoteclose}\isanewline
\ \ \ \ \isacommand{proof}\isamarkupfalse%
\ {\isacharminus}{\kern0pt}\isanewline
\ \ \ \ \ \ \isacommand{fix}\isamarkupfalse%
\ x\ \isanewline
\ \ \ \ \ \ \isacommand{assume}\isamarkupfalse%
\ a{\isacharunderscore}{\kern0pt}{\isadigit{5}}{\isacharcolon}{\kern0pt}\ {\isachardoublequoteopen}x\ {\isasymin}\ y\ {\isacharbackquote}{\kern0pt}\ {\isacharparenleft}{\kern0pt}{\isacharbraceleft}{\kern0pt}{\isadigit{0}}{\isachardot}{\kern0pt}{\isachardot}{\kern0pt}{\isacharless}{\kern0pt}s\isactrlsub {\isadigit{1}}{\isacharbraceright}{\kern0pt}\ {\isasymtimes}\ {\isacharbraceleft}{\kern0pt}{\isadigit{0}}{\isachardot}{\kern0pt}{\isachardot}{\kern0pt}{\isacharless}{\kern0pt}s\isactrlsub {\isadigit{2}}{\isacharbraceright}{\kern0pt}{\isacharparenright}{\kern0pt}{\isachardoublequoteclose}\isanewline
\ \ \ \ \ \ \isacommand{have}\isamarkupfalse%
\ {\isachardoublequoteopen}bit{\isacharunderscore}{\kern0pt}count\ {\isacharparenleft}{\kern0pt}list\isactrlsub S\ {\isacharparenleft}{\kern0pt}zfact\isactrlsub S\ p{\isacharparenright}{\kern0pt}\ x{\isacharparenright}{\kern0pt}\ {\isasymle}\ ereal\ {\isacharparenleft}{\kern0pt}\ real\ {\isadigit{4}}\ {\isacharasterisk}{\kern0pt}\ {\isacharparenleft}{\kern0pt}{\isadigit{2}}\ {\isacharasterisk}{\kern0pt}\ log\ {\isadigit{2}}\ {\isacharparenleft}{\kern0pt}real\ p{\isacharparenright}{\kern0pt}\ {\isacharplus}{\kern0pt}\ {\isadigit{2}}{\isacharparenright}{\kern0pt}\ {\isacharplus}{\kern0pt}\ {\isadigit{1}}{\isacharparenright}{\kern0pt}{\isachardoublequoteclose}\isanewline
\ \ \ \ \ \ \ \ \isacommand{apply}\isamarkupfalse%
\ {\isacharparenleft}{\kern0pt}rule\ bounded{\isacharunderscore}{\kern0pt}degree{\isacharunderscore}{\kern0pt}polynomial{\isacharunderscore}{\kern0pt}bit{\isacharunderscore}{\kern0pt}count{\isacharbrackleft}{\kern0pt}OF\ p{\isacharunderscore}{\kern0pt}ge{\isacharunderscore}{\kern0pt}{\isadigit{0}}{\isacharbrackright}{\kern0pt}{\isacharparenright}{\kern0pt}\isanewline
\ \ \ \ \ \ \ \ \isacommand{using}\isamarkupfalse%
\ a{\isacharunderscore}{\kern0pt}{\isadigit{1}}\ a{\isacharunderscore}{\kern0pt}{\isadigit{5}}\ \isacommand{by}\isamarkupfalse%
\ blast\isanewline
\ \ \ \ \ \ \isacommand{also}\isamarkupfalse%
\ \isacommand{have}\isamarkupfalse%
\ {\isachardoublequoteopen}{\isachardot}{\kern0pt}{\isachardot}{\kern0pt}{\isachardot}{\kern0pt}\ {\isasymle}\ ereal\ {\isacharparenleft}{\kern0pt}real\ {\isadigit{4}}\ {\isacharasterisk}{\kern0pt}\ {\isacharparenleft}{\kern0pt}{\isadigit{2}}\ {\isacharasterisk}{\kern0pt}\ log\ {\isadigit{2}}\ {\isacharparenleft}{\kern0pt}{\isadigit{3}}\ {\isacharplus}{\kern0pt}\ {\isadigit{2}}\ {\isacharasterisk}{\kern0pt}\ real\ n{\isacharparenright}{\kern0pt}\ {\isacharplus}{\kern0pt}\ {\isadigit{2}}{\isacharparenright}{\kern0pt}\ {\isacharplus}{\kern0pt}\ {\isadigit{1}}{\isacharparenright}{\kern0pt}{\isachardoublequoteclose}\isanewline
\ \ \ \ \ \ \ \ \isacommand{apply}\isamarkupfalse%
\ simp\isanewline
\ \ \ \ \ \ \ \ \isacommand{apply}\isamarkupfalse%
\ {\isacharparenleft}{\kern0pt}subst\ log{\isacharunderscore}{\kern0pt}le{\isacharunderscore}{\kern0pt}cancel{\isacharunderscore}{\kern0pt}iff{\isacharcomma}{\kern0pt}\ simp{\isacharcomma}{\kern0pt}\ simp\ add{\isacharcolon}{\kern0pt}p{\isacharunderscore}{\kern0pt}ge{\isacharunderscore}{\kern0pt}{\isadigit{0}}{\isacharcomma}{\kern0pt}\ simp{\isacharparenright}{\kern0pt}\isanewline
\ \ \ \ \ \ \ \ \isacommand{using}\isamarkupfalse%
\ p{\isacharunderscore}{\kern0pt}le{\isacharunderscore}{\kern0pt}n\ \isacommand{by}\isamarkupfalse%
\ simp\isanewline
\ \ \ \ \ \ \isacommand{also}\isamarkupfalse%
\ \isacommand{have}\isamarkupfalse%
\ {\isachardoublequoteopen}{\isachardot}{\kern0pt}{\isachardot}{\kern0pt}{\isachardot}{\kern0pt}\ {\isasymle}\ ereal\ {\isacharparenleft}{\kern0pt}{\isadigit{9}}\ {\isacharplus}{\kern0pt}\ {\isadigit{8}}\ {\isacharasterisk}{\kern0pt}\ log\ {\isadigit{2}}\ {\isacharparenleft}{\kern0pt}{\isadigit{4}}\ {\isacharplus}{\kern0pt}\ {\isadigit{2}}\ {\isacharasterisk}{\kern0pt}\ real\ n{\isacharparenright}{\kern0pt}{\isacharparenright}{\kern0pt}{\isachardoublequoteclose}\isanewline
\ \ \ \ \ \ \ \ \isacommand{by}\isamarkupfalse%
\ simp\isanewline
\ \ \ \ \ \ \isacommand{finally}\isamarkupfalse%
\ \isacommand{show}\isamarkupfalse%
\ {\isachardoublequoteopen}bit{\isacharunderscore}{\kern0pt}count\ {\isacharparenleft}{\kern0pt}list\isactrlsub S\ {\isacharparenleft}{\kern0pt}zfact\isactrlsub S\ p{\isacharparenright}{\kern0pt}\ x{\isacharparenright}{\kern0pt}\ {\isasymle}\ ereal\ {\isacharparenleft}{\kern0pt}{\isadigit{9}}\ {\isacharplus}{\kern0pt}\ {\isadigit{8}}\ {\isacharasterisk}{\kern0pt}\ log\ {\isadigit{2}}\ {\isacharparenleft}{\kern0pt}{\isadigit{4}}\ {\isacharplus}{\kern0pt}\ {\isadigit{2}}\ {\isacharasterisk}{\kern0pt}\ real\ n{\isacharparenright}{\kern0pt}{\isacharparenright}{\kern0pt}{\isachardoublequoteclose}\isanewline
\ \ \ \ \ \ \ \ \isacommand{by}\isamarkupfalse%
\ blast\isanewline
\ \ \ \ \isacommand{qed}\isamarkupfalse%
\isanewline
\isanewline
\ \ \ \ \isacommand{have}\isamarkupfalse%
\ a{\isacharunderscore}{\kern0pt}{\isadigit{7}}{\isacharcolon}{\kern0pt}\ {\isachardoublequoteopen}{\isasymAnd}x{\isachardot}{\kern0pt}\ \isanewline
\ \ \ \ \ \ x\ {\isasymin}\ {\isacharparenleft}{\kern0pt}{\isasymlambda}x{\isachardot}{\kern0pt}\ sum{\isacharunderscore}{\kern0pt}list\ {\isacharparenleft}{\kern0pt}map\ {\isacharparenleft}{\kern0pt}f{\isadigit{2}}{\isacharunderscore}{\kern0pt}hash\ p\ {\isacharparenleft}{\kern0pt}y\ x{\isacharparenright}{\kern0pt}{\isacharparenright}{\kern0pt}\ as{\isacharparenright}{\kern0pt}{\isacharparenright}{\kern0pt}\ {\isacharbackquote}{\kern0pt}\ {\isacharparenleft}{\kern0pt}{\isacharbraceleft}{\kern0pt}{\isadigit{0}}{\isachardot}{\kern0pt}{\isachardot}{\kern0pt}{\isacharless}{\kern0pt}s\isactrlsub {\isadigit{1}}{\isacharbraceright}{\kern0pt}\ {\isasymtimes}\ {\isacharbraceleft}{\kern0pt}{\isadigit{0}}{\isachardot}{\kern0pt}{\isachardot}{\kern0pt}{\isacharless}{\kern0pt}s\isactrlsub {\isadigit{2}}{\isacharbraceright}{\kern0pt}{\isacharparenright}{\kern0pt}\ {\isasymLongrightarrow}\isanewline
\ \ \ \ \ \ \ \ \ {\isasymbar}x{\isasymbar}\ {\isasymle}\ {\isacharparenleft}{\kern0pt}{\isadigit{4}}\ {\isacharplus}{\kern0pt}\ {\isadigit{2}}\ {\isacharasterisk}{\kern0pt}\ int\ n{\isacharparenright}{\kern0pt}\ {\isacharasterisk}{\kern0pt}\ int\ {\isacharparenleft}{\kern0pt}length\ as{\isacharparenright}{\kern0pt}{\isachardoublequoteclose}\isanewline
\ \ \ \ \isacommand{proof}\isamarkupfalse%
\ {\isacharminus}{\kern0pt}\isanewline
\ \ \ \ \ \ \isacommand{fix}\isamarkupfalse%
\ x\isanewline
\ \ \ \ \ \ \isacommand{assume}\isamarkupfalse%
\ {\isachardoublequoteopen}x\ {\isasymin}\ {\isacharparenleft}{\kern0pt}{\isasymlambda}x{\isachardot}{\kern0pt}\ sum{\isacharunderscore}{\kern0pt}list\ {\isacharparenleft}{\kern0pt}map\ {\isacharparenleft}{\kern0pt}f{\isadigit{2}}{\isacharunderscore}{\kern0pt}hash\ p\ {\isacharparenleft}{\kern0pt}y\ x{\isacharparenright}{\kern0pt}{\isacharparenright}{\kern0pt}\ as{\isacharparenright}{\kern0pt}{\isacharparenright}{\kern0pt}\ {\isacharbackquote}{\kern0pt}\ {\isacharparenleft}{\kern0pt}{\isacharbraceleft}{\kern0pt}{\isadigit{0}}{\isachardot}{\kern0pt}{\isachardot}{\kern0pt}{\isacharless}{\kern0pt}s\isactrlsub {\isadigit{1}}{\isacharbraceright}{\kern0pt}\ {\isasymtimes}\ {\isacharbraceleft}{\kern0pt}{\isadigit{0}}{\isachardot}{\kern0pt}{\isachardot}{\kern0pt}{\isacharless}{\kern0pt}s\isactrlsub {\isadigit{2}}{\isacharbraceright}{\kern0pt}{\isacharparenright}{\kern0pt}{\isachardoublequoteclose}\isanewline
\ \ \ \ \ \ \isacommand{then}\isamarkupfalse%
\ \isacommand{obtain}\isamarkupfalse%
\ i\ \isakeyword{where}\ {\isachardoublequoteopen}i\ {\isasymin}\ {\isacharbraceleft}{\kern0pt}{\isadigit{0}}{\isachardot}{\kern0pt}{\isachardot}{\kern0pt}{\isacharless}{\kern0pt}s\isactrlsub {\isadigit{1}}{\isacharbraceright}{\kern0pt}\ {\isasymtimes}\ {\isacharbraceleft}{\kern0pt}{\isadigit{0}}{\isachardot}{\kern0pt}{\isachardot}{\kern0pt}{\isacharless}{\kern0pt}s\isactrlsub {\isadigit{2}}{\isacharbraceright}{\kern0pt}{\isachardoublequoteclose}\ \isakeyword{and}\ x{\isacharunderscore}{\kern0pt}def{\isacharcolon}{\kern0pt}\ {\isachardoublequoteopen}x\ {\isacharequal}{\kern0pt}\ sum{\isacharunderscore}{\kern0pt}list\ {\isacharparenleft}{\kern0pt}map\ {\isacharparenleft}{\kern0pt}f{\isadigit{2}}{\isacharunderscore}{\kern0pt}hash\ p\ {\isacharparenleft}{\kern0pt}y\ i{\isacharparenright}{\kern0pt}{\isacharparenright}{\kern0pt}\ as{\isacharparenright}{\kern0pt}{\isachardoublequoteclose}\isanewline
\ \ \ \ \ \ \ \ \isacommand{by}\isamarkupfalse%
\ blast\isanewline
\ \ \ \ \ \ \isacommand{have}\isamarkupfalse%
\ {\isachardoublequoteopen}abs\ x\ {\isasymle}\ sum{\isacharunderscore}{\kern0pt}list\ {\isacharparenleft}{\kern0pt}map\ abs\ {\isacharparenleft}{\kern0pt}map\ {\isacharparenleft}{\kern0pt}f{\isadigit{2}}{\isacharunderscore}{\kern0pt}hash\ p\ {\isacharparenleft}{\kern0pt}y\ i{\isacharparenright}{\kern0pt}{\isacharparenright}{\kern0pt}\ as{\isacharparenright}{\kern0pt}{\isacharparenright}{\kern0pt}{\isachardoublequoteclose}\isanewline
\ \ \ \ \ \ \ \ \isacommand{by}\isamarkupfalse%
\ {\isacharparenleft}{\kern0pt}subst\ x{\isacharunderscore}{\kern0pt}def{\isacharcomma}{\kern0pt}\ rule\ sum{\isacharunderscore}{\kern0pt}list{\isacharunderscore}{\kern0pt}abs{\isacharparenright}{\kern0pt}\isanewline
\ \ \ \ \ \ \isacommand{also}\isamarkupfalse%
\ \isacommand{have}\isamarkupfalse%
\ {\isachardoublequoteopen}{\isachardot}{\kern0pt}{\isachardot}{\kern0pt}{\isachardot}{\kern0pt}\ {\isasymle}\ sum{\isacharunderscore}{\kern0pt}list\ {\isacharparenleft}{\kern0pt}map\ {\isacharparenleft}{\kern0pt}{\isasymlambda}{\isacharunderscore}{\kern0pt}{\isachardot}{\kern0pt}\ {\isacharparenleft}{\kern0pt}int\ p{\isacharplus}{\kern0pt}{\isadigit{1}}{\isacharparenright}{\kern0pt}{\isacharparenright}{\kern0pt}\ as{\isacharparenright}{\kern0pt}{\isachardoublequoteclose}\isanewline
\ \ \ \ \ \ \ \ \isacommand{apply}\isamarkupfalse%
\ {\isacharparenleft}{\kern0pt}simp\ add{\isacharcolon}{\kern0pt}comp{\isacharunderscore}{\kern0pt}def\ del{\isacharcolon}{\kern0pt}f{\isadigit{2}}{\isacharunderscore}{\kern0pt}hash{\isachardot}{\kern0pt}simps{\isacharparenright}{\kern0pt}\isanewline
\ \ \ \ \ \ \ \ \isacommand{apply}\isamarkupfalse%
\ {\isacharparenleft}{\kern0pt}rule\ sum{\isacharunderscore}{\kern0pt}list{\isacharunderscore}{\kern0pt}mono{\isacharparenright}{\kern0pt}\isanewline
\ \ \ \ \ \ \ \ \isacommand{using}\isamarkupfalse%
\ p{\isacharunderscore}{\kern0pt}ge{\isacharunderscore}{\kern0pt}{\isadigit{0}}\ \isacommand{by}\isamarkupfalse%
\ simp\ \isanewline
\ \ \ \ \ \ \isacommand{also}\isamarkupfalse%
\ \isacommand{have}\isamarkupfalse%
\ {\isachardoublequoteopen}{\isachardot}{\kern0pt}{\isachardot}{\kern0pt}{\isachardot}{\kern0pt}\ {\isacharequal}{\kern0pt}\ int\ {\isacharparenleft}{\kern0pt}length\ as{\isacharparenright}{\kern0pt}\ {\isacharasterisk}{\kern0pt}\ {\isacharparenleft}{\kern0pt}int\ p{\isacharplus}{\kern0pt}{\isadigit{1}}{\isacharparenright}{\kern0pt}{\isachardoublequoteclose}\isanewline
\ \ \ \ \ \ \ \ \isacommand{by}\isamarkupfalse%
\ {\isacharparenleft}{\kern0pt}simp\ add{\isacharcolon}{\kern0pt}\ sum{\isacharunderscore}{\kern0pt}list{\isacharunderscore}{\kern0pt}triv{\isacharparenright}{\kern0pt}\isanewline
\ \ \ \ \ \ \isacommand{also}\isamarkupfalse%
\ \isacommand{have}\isamarkupfalse%
\ {\isachardoublequoteopen}{\isachardot}{\kern0pt}{\isachardot}{\kern0pt}{\isachardot}{\kern0pt}\ {\isasymle}\ int\ {\isacharparenleft}{\kern0pt}length\ as{\isacharparenright}{\kern0pt}\ {\isacharasterisk}{\kern0pt}\ {\isacharparenleft}{\kern0pt}{\isadigit{4}}{\isacharplus}{\kern0pt}{\isadigit{2}}{\isacharasterisk}{\kern0pt}{\isacharparenleft}{\kern0pt}int\ n{\isacharparenright}{\kern0pt}{\isacharparenright}{\kern0pt}{\isachardoublequoteclose}\isanewline
\ \ \ \ \ \ \ \ \isacommand{apply}\isamarkupfalse%
\ {\isacharparenleft}{\kern0pt}rule\ mult{\isacharunderscore}{\kern0pt}mono{\isacharcomma}{\kern0pt}\ simp{\isacharparenright}{\kern0pt}\isanewline
\ \ \ \ \ \ \ \ \isacommand{using}\isamarkupfalse%
\ p{\isacharunderscore}{\kern0pt}le{\isacharunderscore}{\kern0pt}n\ \isacommand{apply}\isamarkupfalse%
\ linarith\isanewline
\ \ \ \ \ \ \ \ \isacommand{by}\isamarkupfalse%
\ simp{\isacharplus}{\kern0pt}\isanewline
\ \ \ \ \ \ \isacommand{finally}\isamarkupfalse%
\ \isacommand{show}\isamarkupfalse%
\ {\isachardoublequoteopen}abs\ x\ {\isasymle}\ \ {\isacharparenleft}{\kern0pt}{\isadigit{4}}\ {\isacharplus}{\kern0pt}\ {\isadigit{2}}\ {\isacharasterisk}{\kern0pt}\ int\ n{\isacharparenright}{\kern0pt}\ {\isacharasterisk}{\kern0pt}\ int\ {\isacharparenleft}{\kern0pt}length\ as{\isacharparenright}{\kern0pt}{\isachardoublequoteclose}\isanewline
\ \ \ \ \ \ \ \ \isacommand{by}\isamarkupfalse%
\ {\isacharparenleft}{\kern0pt}simp\ add{\isacharcolon}{\kern0pt}\ mult{\isachardot}{\kern0pt}commute{\isacharparenright}{\kern0pt}\isanewline
\ \ \ \ \isacommand{qed}\isamarkupfalse%
\isanewline
\ \ \ \ \isanewline
\ \ \ \ \isacommand{have}\isamarkupfalse%
\ {\isachardoublequoteopen}bit{\isacharunderscore}{\kern0pt}count\ {\isacharparenleft}{\kern0pt}encode{\isacharunderscore}{\kern0pt}state\ {\isacharparenleft}{\kern0pt}s\isactrlsub {\isadigit{1}}{\isacharcomma}{\kern0pt}\ s\isactrlsub {\isadigit{2}}{\isacharcomma}{\kern0pt}\ p{\isacharcomma}{\kern0pt}\ y{\isacharcomma}{\kern0pt}\ {\isasymlambda}i{\isasymin}{\isacharbraceleft}{\kern0pt}{\isadigit{0}}{\isachardot}{\kern0pt}{\isachardot}{\kern0pt}{\isacharless}{\kern0pt}s\isactrlsub {\isadigit{1}}{\isacharbraceright}{\kern0pt}\ {\isasymtimes}\ {\isacharbraceleft}{\kern0pt}{\isadigit{0}}{\isachardot}{\kern0pt}{\isachardot}{\kern0pt}{\isacharless}{\kern0pt}s\isactrlsub {\isadigit{2}}{\isacharbraceright}{\kern0pt}{\isachardot}{\kern0pt}\isanewline
\ \ \ \ \ \ sum{\isacharunderscore}{\kern0pt}list\ {\isacharparenleft}{\kern0pt}map\ {\isacharparenleft}{\kern0pt}f{\isadigit{2}}{\isacharunderscore}{\kern0pt}hash\ p\ {\isacharparenleft}{\kern0pt}y\ i{\isacharparenright}{\kern0pt}{\isacharparenright}{\kern0pt}\ as{\isacharparenright}{\kern0pt}{\isacharparenright}{\kern0pt}{\isacharparenright}{\kern0pt}\isanewline
\ \ \ \ \ \ \ {\isasymle}\ ereal\ {\isacharparenleft}{\kern0pt}{\isadigit{2}}\ {\isacharasterisk}{\kern0pt}\ {\isacharparenleft}{\kern0pt}log\ {\isadigit{2}}\ {\isacharparenleft}{\kern0pt}real\ s\isactrlsub {\isadigit{1}}\ {\isacharplus}{\kern0pt}\ {\isadigit{1}}{\isacharparenright}{\kern0pt}{\isacharparenright}{\kern0pt}\ {\isacharplus}{\kern0pt}\ {\isadigit{1}}{\isacharparenright}{\kern0pt}\ \isanewline
\ \ \ \ \ \ \ {\isacharplus}{\kern0pt}\ {\isacharparenleft}{\kern0pt}ereal\ {\isacharparenleft}{\kern0pt}{\isadigit{2}}\ {\isacharasterisk}{\kern0pt}\ {\isacharparenleft}{\kern0pt}log\ {\isadigit{2}}\ {\isacharparenleft}{\kern0pt}real\ s\isactrlsub {\isadigit{2}}\ {\isacharplus}{\kern0pt}\ {\isadigit{1}}{\isacharparenright}{\kern0pt}{\isacharparenright}{\kern0pt}\ {\isacharplus}{\kern0pt}\ {\isadigit{1}}{\isacharparenright}{\kern0pt}\isanewline
\ \ \ \ \ \ \ {\isacharplus}{\kern0pt}\ {\isacharparenleft}{\kern0pt}ereal\ {\isacharparenleft}{\kern0pt}{\isadigit{2}}\ {\isacharasterisk}{\kern0pt}\ {\isacharparenleft}{\kern0pt}log\ {\isadigit{2}}\ {\isacharparenleft}{\kern0pt}{\isadigit{1}}\ {\isacharplus}{\kern0pt}\ real\ {\isacharparenleft}{\kern0pt}{\isadigit{2}}{\isacharasterisk}{\kern0pt}n{\isacharplus}{\kern0pt}{\isadigit{3}}{\isacharparenright}{\kern0pt}{\isacharparenright}{\kern0pt}{\isacharparenright}{\kern0pt}\ {\isacharplus}{\kern0pt}\ {\isadigit{1}}{\isacharparenright}{\kern0pt}\ \isanewline
\ \ \ \ \ \ \ {\isacharplus}{\kern0pt}\ {\isacharparenleft}{\kern0pt}{\isacharparenleft}{\kern0pt}ereal\ {\isacharparenleft}{\kern0pt}real\ s\isactrlsub {\isadigit{1}}\ {\isacharasterisk}{\kern0pt}\ real\ s\isactrlsub {\isadigit{2}}{\isacharparenright}{\kern0pt}\ {\isacharasterisk}{\kern0pt}\ {\isacharparenleft}{\kern0pt}{\isadigit{1}}{\isadigit{0}}\ {\isacharplus}{\kern0pt}\ {\isadigit{8}}\ {\isacharasterisk}{\kern0pt}\ log\ {\isadigit{2}}\ {\isacharparenleft}{\kern0pt}{\isadigit{4}}\ {\isacharplus}{\kern0pt}\ {\isadigit{2}}\ {\isacharasterisk}{\kern0pt}\ real\ n{\isacharparenright}{\kern0pt}{\isacharparenright}{\kern0pt}\ {\isacharplus}{\kern0pt}\ {\isadigit{1}}{\isacharparenright}{\kern0pt}\ \isanewline
\ \ \ \ \ \ \ {\isacharplus}{\kern0pt}\ {\isacharparenleft}{\kern0pt}ereal\ {\isacharparenleft}{\kern0pt}real\ s\isactrlsub {\isadigit{1}}\ {\isacharasterisk}{\kern0pt}\ real\ s\isactrlsub {\isadigit{2}}{\isacharparenright}{\kern0pt}\ {\isacharasterisk}{\kern0pt}\ {\isacharparenleft}{\kern0pt}{\isadigit{3}}\ {\isacharplus}{\kern0pt}\ {\isadigit{2}}\ {\isacharasterisk}{\kern0pt}\ log\ {\isadigit{2}}\ {\isacharparenleft}{\kern0pt}real\ {\isacharparenleft}{\kern0pt}length\ as{\isacharparenright}{\kern0pt}\ {\isacharasterisk}{\kern0pt}\ {\isacharparenleft}{\kern0pt}{\isadigit{4}}\ {\isacharplus}{\kern0pt}\ {\isadigit{2}}\ {\isacharasterisk}{\kern0pt}\ real\ n{\isacharparenright}{\kern0pt}\ {\isacharplus}{\kern0pt}\ {\isadigit{1}}{\isacharparenright}{\kern0pt}\ {\isacharparenright}{\kern0pt}\ {\isacharplus}{\kern0pt}\ {\isadigit{1}}{\isacharparenright}{\kern0pt}{\isacharparenright}{\kern0pt}{\isacharparenright}{\kern0pt}{\isacharparenright}{\kern0pt}{\isachardoublequoteclose}\isanewline
\ \ \ \ \ \ \isacommand{using}\isamarkupfalse%
\ a{\isacharunderscore}{\kern0pt}{\isadigit{2}}\isanewline
\ \ \ \ \ \ \isacommand{apply}\isamarkupfalse%
\ {\isacharparenleft}{\kern0pt}simp\ add{\isacharcolon}{\kern0pt}\ encode{\isacharunderscore}{\kern0pt}state{\isacharunderscore}{\kern0pt}def\ s\isactrlsub {\isadigit{1}}{\isacharunderscore}{\kern0pt}def{\isacharbrackleft}{\kern0pt}symmetric{\isacharbrackright}{\kern0pt}\ s\isactrlsub {\isadigit{2}}{\isacharunderscore}{\kern0pt}def{\isacharbrackleft}{\kern0pt}symmetric{\isacharbrackright}{\kern0pt}\ p{\isacharunderscore}{\kern0pt}def{\isacharbrackleft}{\kern0pt}symmetric{\isacharbrackright}{\kern0pt}\ \isanewline
\ \ \ \ \ \ \ \ dependent{\isacharunderscore}{\kern0pt}bit{\isacharunderscore}{\kern0pt}count\ prod{\isacharunderscore}{\kern0pt}bit{\isacharunderscore}{\kern0pt}count\ fun\isactrlsub S{\isacharunderscore}{\kern0pt}def\isanewline
\ \ \ \ \ \ \ \ \ \ del{\isacharcolon}{\kern0pt}encode{\isacharunderscore}{\kern0pt}dependent{\isacharunderscore}{\kern0pt}sum{\isachardot}{\kern0pt}simps\ encode{\isacharunderscore}{\kern0pt}prod{\isachardot}{\kern0pt}simps\ N\isactrlsub S{\isachardot}{\kern0pt}simps\ plus{\isacharunderscore}{\kern0pt}ereal{\isachardot}{\kern0pt}simps\ of{\isacharunderscore}{\kern0pt}nat{\isacharunderscore}{\kern0pt}add{\isacharparenright}{\kern0pt}\isanewline
\ \ \ \ \ \ \isacommand{apply}\isamarkupfalse%
\ {\isacharparenleft}{\kern0pt}rule\ add{\isacharunderscore}{\kern0pt}mono{\isacharcomma}{\kern0pt}\ rule\ nat{\isacharunderscore}{\kern0pt}bit{\isacharunderscore}{\kern0pt}count{\isacharparenright}{\kern0pt}\isanewline
\ \ \ \ \ \ \isacommand{apply}\isamarkupfalse%
\ {\isacharparenleft}{\kern0pt}rule\ add{\isacharunderscore}{\kern0pt}mono{\isacharcomma}{\kern0pt}\ rule\ nat{\isacharunderscore}{\kern0pt}bit{\isacharunderscore}{\kern0pt}count{\isacharparenright}{\kern0pt}\isanewline
\ \ \ \ \ \ \isacommand{apply}\isamarkupfalse%
\ {\isacharparenleft}{\kern0pt}rule\ add{\isacharunderscore}{\kern0pt}mono{\isacharcomma}{\kern0pt}\ rule\ nat{\isacharunderscore}{\kern0pt}bit{\isacharunderscore}{\kern0pt}count{\isacharunderscore}{\kern0pt}est{\isacharcomma}{\kern0pt}\ metis\ p{\isacharunderscore}{\kern0pt}le{\isacharunderscore}{\kern0pt}n{\isacharparenright}{\kern0pt}\isanewline
\ \ \ \ \ \ \isacommand{apply}\isamarkupfalse%
\ {\isacharparenleft}{\kern0pt}rule\ add{\isacharunderscore}{\kern0pt}mono{\isacharparenright}{\kern0pt}\isanewline
\ \ \ \ \ \ \ \isacommand{apply}\isamarkupfalse%
\ {\isacharparenleft}{\kern0pt}rule\ list{\isacharunderscore}{\kern0pt}bit{\isacharunderscore}{\kern0pt}count{\isacharunderscore}{\kern0pt}estI{\isacharbrackleft}{\kern0pt}\isakeyword{where}\ a{\isacharequal}{\kern0pt}{\isachardoublequoteopen}{\isadigit{9}}\ {\isacharplus}{\kern0pt}\ {\isadigit{8}}\ {\isacharasterisk}{\kern0pt}\ log\ {\isadigit{2}}\ {\isacharparenleft}{\kern0pt}{\isadigit{4}}\ {\isacharplus}{\kern0pt}\ {\isadigit{2}}\ {\isacharasterisk}{\kern0pt}\ real\ n{\isacharparenright}{\kern0pt}{\isachardoublequoteclose}{\isacharbrackright}{\kern0pt}{\isacharcomma}{\kern0pt}\ rule\ a{\isacharunderscore}{\kern0pt}{\isadigit{3}}{\isacharcomma}{\kern0pt}\ simp{\isacharcomma}{\kern0pt}\ simp{\isacharparenright}{\kern0pt}\isanewline
\ \ \ \ \ \ \isacommand{apply}\isamarkupfalse%
\ {\isacharparenleft}{\kern0pt}rule\ list{\isacharunderscore}{\kern0pt}bit{\isacharunderscore}{\kern0pt}count{\isacharunderscore}{\kern0pt}estI{\isacharbrackleft}{\kern0pt}\isakeyword{where}\ a{\isacharequal}{\kern0pt}{\isachardoublequoteopen}{\isadigit{2}}{\isacharasterisk}{\kern0pt}\ log\ {\isadigit{2}}\ {\isacharparenleft}{\kern0pt}real{\isacharunderscore}{\kern0pt}of{\isacharunderscore}{\kern0pt}int\ {\isacharparenleft}{\kern0pt}int\ {\isacharparenleft}{\kern0pt}{\isacharparenleft}{\kern0pt}{\isadigit{4}}{\isacharplus}{\kern0pt}{\isadigit{2}}{\isacharasterisk}{\kern0pt}n{\isacharparenright}{\kern0pt}\ {\isacharasterisk}{\kern0pt}\ length\ as{\isacharparenright}{\kern0pt}{\isacharplus}{\kern0pt}{\isadigit{1}}{\isacharparenright}{\kern0pt}{\isacharparenright}{\kern0pt}{\isacharplus}{\kern0pt}{\isadigit{2}}{\isachardoublequoteclose}{\isacharbrackright}{\kern0pt}{\isacharparenright}{\kern0pt}\isanewline
\ \ \ \ \ \ \ \isacommand{apply}\isamarkupfalse%
\ {\isacharparenleft}{\kern0pt}rule\ int{\isacharunderscore}{\kern0pt}bit{\isacharunderscore}{\kern0pt}count{\isacharunderscore}{\kern0pt}est{\isacharparenright}{\kern0pt}\isanewline
\ \ \ \ \ \ \ \isacommand{apply}\isamarkupfalse%
\ {\isacharparenleft}{\kern0pt}simp\ add{\isacharcolon}{\kern0pt}a{\isacharunderscore}{\kern0pt}{\isadigit{7}}{\isacharparenright}{\kern0pt}\isanewline
\ \ \ \ \ \ \isacommand{by}\isamarkupfalse%
\ {\isacharparenleft}{\kern0pt}simp\ add{\isacharcolon}{\kern0pt}algebra{\isacharunderscore}{\kern0pt}simps{\isacharparenright}{\kern0pt}\isanewline
\ \ \ \ \isacommand{also}\isamarkupfalse%
\ \isacommand{have}\isamarkupfalse%
\ {\isachardoublequoteopen}{\isachardot}{\kern0pt}{\isachardot}{\kern0pt}{\isachardot}{\kern0pt}\ {\isacharequal}{\kern0pt}\ ereal\ {\isacharparenleft}{\kern0pt}f{\isadigit{2}}{\isacharunderscore}{\kern0pt}space{\isacharunderscore}{\kern0pt}usage\ {\isacharparenleft}{\kern0pt}n{\isacharcomma}{\kern0pt}\ length\ as{\isacharcomma}{\kern0pt}\ {\isasymepsilon}{\isacharcomma}{\kern0pt}\ {\isasymdelta}{\isacharparenright}{\kern0pt}{\isacharparenright}{\kern0pt}{\isachardoublequoteclose}\isanewline
\ \ \ \ \ \ \isacommand{by}\isamarkupfalse%
\ {\isacharparenleft}{\kern0pt}simp\ add{\isacharcolon}{\kern0pt}distrib{\isacharunderscore}{\kern0pt}left{\isacharbrackleft}{\kern0pt}symmetric{\isacharbrackright}{\kern0pt}\ s\isactrlsub {\isadigit{1}}{\isacharunderscore}{\kern0pt}def{\isacharbrackleft}{\kern0pt}symmetric{\isacharbrackright}{\kern0pt}\ s\isactrlsub {\isadigit{2}}{\isacharunderscore}{\kern0pt}def{\isacharbrackleft}{\kern0pt}symmetric{\isacharbrackright}{\kern0pt}\ p{\isacharunderscore}{\kern0pt}def{\isacharbrackleft}{\kern0pt}symmetric{\isacharbrackright}{\kern0pt}{\isacharparenright}{\kern0pt}\isanewline
\ \ \ \ \isacommand{finally}\isamarkupfalse%
\ \isacommand{show}\isamarkupfalse%
\ {\isachardoublequoteopen}bit{\isacharunderscore}{\kern0pt}count\ {\isacharparenleft}{\kern0pt}encode{\isacharunderscore}{\kern0pt}state\ {\isacharparenleft}{\kern0pt}s\isactrlsub {\isadigit{1}}{\isacharcomma}{\kern0pt}\ s\isactrlsub {\isadigit{2}}{\isacharcomma}{\kern0pt}\ p{\isacharcomma}{\kern0pt}\ y{\isacharcomma}{\kern0pt}\ {\isasymlambda}i{\isasymin}{\isacharbraceleft}{\kern0pt}{\isadigit{0}}{\isachardot}{\kern0pt}{\isachardot}{\kern0pt}{\isacharless}{\kern0pt}s\isactrlsub {\isadigit{1}}{\isacharbraceright}{\kern0pt}\ {\isasymtimes}\ {\isacharbraceleft}{\kern0pt}{\isadigit{0}}{\isachardot}{\kern0pt}{\isachardot}{\kern0pt}{\isacharless}{\kern0pt}s\isactrlsub {\isadigit{2}}{\isacharbraceright}{\kern0pt}{\isachardot}{\kern0pt}\isanewline
\ \ \ \ \ \ sum{\isacharunderscore}{\kern0pt}list\ {\isacharparenleft}{\kern0pt}map\ {\isacharparenleft}{\kern0pt}f{\isadigit{2}}{\isacharunderscore}{\kern0pt}hash\ p\ {\isacharparenleft}{\kern0pt}y\ i{\isacharparenright}{\kern0pt}{\isacharparenright}{\kern0pt}\ as{\isacharparenright}{\kern0pt}{\isacharparenright}{\kern0pt}{\isacharparenright}{\kern0pt}\isanewline
\ \ \ \ \ \ \ {\isasymle}\ ereal\ {\isacharparenleft}{\kern0pt}f{\isadigit{2}}{\isacharunderscore}{\kern0pt}space{\isacharunderscore}{\kern0pt}usage\ {\isacharparenleft}{\kern0pt}n{\isacharcomma}{\kern0pt}\ length\ as{\isacharcomma}{\kern0pt}\ {\isasymepsilon}{\isacharcomma}{\kern0pt}\ {\isasymdelta}{\isacharparenright}{\kern0pt}{\isacharparenright}{\kern0pt}{\isachardoublequoteclose}\ \isacommand{by}\isamarkupfalse%
\ blast\isanewline
\ \ \isacommand{qed}\isamarkupfalse%
\isanewline
\isanewline
\ \ \isacommand{show}\isamarkupfalse%
\ {\isacharquery}{\kern0pt}thesis\isanewline
\ \ \ \ \isacommand{apply}\isamarkupfalse%
\ {\isacharparenleft}{\kern0pt}subst\ AE{\isacharunderscore}{\kern0pt}measure{\isacharunderscore}{\kern0pt}pmf{\isacharunderscore}{\kern0pt}iff{\isacharparenright}{\kern0pt}\isanewline
\ \ \ \ \isacommand{apply}\isamarkupfalse%
\ {\isacharparenleft}{\kern0pt}subst\ M{\isacharunderscore}{\kern0pt}def{\isacharparenright}{\kern0pt}\isanewline
\ \ \ \ \isacommand{apply}\isamarkupfalse%
\ {\isacharparenleft}{\kern0pt}subst\ f{\isadigit{2}}{\isacharunderscore}{\kern0pt}alg{\isacharunderscore}{\kern0pt}sketch{\isacharbrackleft}{\kern0pt}OF\ assms{\isacharparenleft}{\kern0pt}{\isadigit{1}}{\isacharparenright}{\kern0pt}\ assms{\isacharparenleft}{\kern0pt}{\isadigit{2}}{\isacharparenright}{\kern0pt}{\isacharcomma}{\kern0pt}\ \isakeyword{where}\ n{\isacharequal}{\kern0pt}{\isachardoublequoteopen}n{\isachardoublequoteclose}\ \isakeyword{and}\ as{\isacharequal}{\kern0pt}{\isachardoublequoteopen}as{\isachardoublequoteclose}{\isacharbrackright}{\kern0pt}{\isacharparenright}{\kern0pt}\isanewline
\ \ \ \ \isacommand{apply}\isamarkupfalse%
\ {\isacharparenleft}{\kern0pt}simp\ add{\isacharcolon}{\kern0pt}\ s\isactrlsub {\isadigit{1}}{\isacharunderscore}{\kern0pt}def{\isacharbrackleft}{\kern0pt}symmetric{\isacharbrackright}{\kern0pt}\ s\isactrlsub {\isadigit{2}}{\isacharunderscore}{\kern0pt}def{\isacharbrackleft}{\kern0pt}symmetric{\isacharbrackright}{\kern0pt}\ p{\isacharunderscore}{\kern0pt}def{\isacharbrackleft}{\kern0pt}symmetric{\isacharbrackright}{\kern0pt}\ del{\isacharcolon}{\kern0pt}f{\isadigit{2}}{\isacharunderscore}{\kern0pt}space{\isacharunderscore}{\kern0pt}usage{\isachardot}{\kern0pt}simps{\isacharparenright}{\kern0pt}\isanewline
\ \ \ \ \isacommand{apply}\isamarkupfalse%
\ {\isacharparenleft}{\kern0pt}subst\ set{\isacharunderscore}{\kern0pt}prod{\isacharunderscore}{\kern0pt}pmf{\isacharcomma}{\kern0pt}\ simp{\isacharparenright}{\kern0pt}\isanewline
\ \ \ \ \isacommand{apply}\isamarkupfalse%
\ {\isacharparenleft}{\kern0pt}simp\ add{\isacharcolon}{\kern0pt}\ PiE{\isacharunderscore}{\kern0pt}iff\ \ del{\isacharcolon}{\kern0pt}f{\isadigit{2}}{\isacharunderscore}{\kern0pt}space{\isacharunderscore}{\kern0pt}usage{\isachardot}{\kern0pt}simps{\isacharparenright}{\kern0pt}\isanewline
\ \ \ \ \isacommand{apply}\isamarkupfalse%
\ {\isacharparenleft}{\kern0pt}subst\ set{\isacharunderscore}{\kern0pt}pmf{\isacharunderscore}{\kern0pt}of{\isacharunderscore}{\kern0pt}set{\isacharcomma}{\kern0pt}\ metis\ ne{\isacharunderscore}{\kern0pt}bounded{\isacharunderscore}{\kern0pt}degree{\isacharunderscore}{\kern0pt}polynomials{\isacharcomma}{\kern0pt}\ metis\ fin{\isacharunderscore}{\kern0pt}bounded{\isacharunderscore}{\kern0pt}degree{\isacharunderscore}{\kern0pt}polynomials{\isacharbrackleft}{\kern0pt}OF\ p{\isacharunderscore}{\kern0pt}ge{\isacharunderscore}{\kern0pt}{\isadigit{0}}{\isacharbrackright}{\kern0pt}{\isacharparenright}{\kern0pt}\isanewline
\ \ \ \ \isacommand{by}\isamarkupfalse%
\ {\isacharparenleft}{\kern0pt}metis\ a{\isacharparenright}{\kern0pt}\isanewline
\isacommand{qed}\isamarkupfalse%
%
\endisatagproof
{\isafoldproof}%
%
\isadelimproof
\isanewline
%
\endisadelimproof
\isanewline
\isacommand{theorem}\isamarkupfalse%
\ f{\isadigit{2}}{\isacharunderscore}{\kern0pt}asympotic{\isacharunderscore}{\kern0pt}space{\isacharunderscore}{\kern0pt}complexity{\isacharcolon}{\kern0pt}\isanewline
\ \ {\isachardoublequoteopen}f{\isadigit{2}}{\isacharunderscore}{\kern0pt}space{\isacharunderscore}{\kern0pt}usage\ {\isasymin}\ O{\isacharbrackleft}{\kern0pt}at{\isacharunderscore}{\kern0pt}top\ {\isasymtimes}\isactrlsub F\ at{\isacharunderscore}{\kern0pt}top\ {\isasymtimes}\isactrlsub F\ at{\isacharunderscore}{\kern0pt}right\ {\isadigit{0}}\ {\isasymtimes}\isactrlsub F\ at{\isacharunderscore}{\kern0pt}right\ {\isadigit{0}}{\isacharbrackright}{\kern0pt}{\isacharparenleft}{\kern0pt}{\isasymlambda}\ {\isacharparenleft}{\kern0pt}n{\isacharcomma}{\kern0pt}\ m{\isacharcomma}{\kern0pt}\ {\isasymepsilon}{\isacharcomma}{\kern0pt}\ {\isasymdelta}{\isacharparenright}{\kern0pt}{\isachardot}{\kern0pt}\ \isanewline
\ \ {\isacharparenleft}{\kern0pt}ln\ {\isacharparenleft}{\kern0pt}{\isadigit{1}}\ {\isacharslash}{\kern0pt}\ of{\isacharunderscore}{\kern0pt}rat\ {\isasymepsilon}{\isacharparenright}{\kern0pt}{\isacharparenright}{\kern0pt}\ {\isacharslash}{\kern0pt}\ {\isacharparenleft}{\kern0pt}of{\isacharunderscore}{\kern0pt}rat\ {\isasymdelta}{\isacharparenright}{\kern0pt}\isactrlsup {\isadigit{2}}\ {\isacharasterisk}{\kern0pt}\ {\isacharparenleft}{\kern0pt}ln\ {\isacharparenleft}{\kern0pt}real\ n{\isacharparenright}{\kern0pt}\ {\isacharplus}{\kern0pt}\ ln\ {\isacharparenleft}{\kern0pt}real\ m{\isacharparenright}{\kern0pt}{\isacharparenright}{\kern0pt}{\isacharparenright}{\kern0pt}{\isachardoublequoteclose}\isanewline
\ \ {\isacharparenleft}{\kern0pt}\isakeyword{is}\ {\isachardoublequoteopen}{\isacharunderscore}{\kern0pt}\ {\isasymin}\ O{\isacharbrackleft}{\kern0pt}{\isacharquery}{\kern0pt}F{\isacharbrackright}{\kern0pt}{\isacharparenleft}{\kern0pt}{\isacharquery}{\kern0pt}rhs{\isacharparenright}{\kern0pt}{\isachardoublequoteclose}{\isacharparenright}{\kern0pt}\isanewline
%
\isadelimproof
%
\endisadelimproof
%
\isatagproof
\isacommand{proof}\isamarkupfalse%
\ {\isacharminus}{\kern0pt}\isanewline
\ \ \isacommand{define}\isamarkupfalse%
\ n{\isacharunderscore}{\kern0pt}of\ {\isacharcolon}{\kern0pt}{\isacharcolon}{\kern0pt}\ {\isachardoublequoteopen}nat\ {\isasymtimes}\ nat\ {\isasymtimes}\ rat\ {\isasymtimes}\ rat\ {\isasymRightarrow}\ nat{\isachardoublequoteclose}\ \isakeyword{where}\ {\isachardoublequoteopen}n{\isacharunderscore}{\kern0pt}of\ {\isacharequal}{\kern0pt}\ {\isacharparenleft}{\kern0pt}{\isasymlambda}{\isacharparenleft}{\kern0pt}n{\isacharcomma}{\kern0pt}\ m{\isacharcomma}{\kern0pt}\ {\isasymepsilon}{\isacharcomma}{\kern0pt}\ {\isasymdelta}{\isacharparenright}{\kern0pt}{\isachardot}{\kern0pt}\ n{\isacharparenright}{\kern0pt}{\isachardoublequoteclose}\isanewline
\ \ \isacommand{define}\isamarkupfalse%
\ m{\isacharunderscore}{\kern0pt}of\ {\isacharcolon}{\kern0pt}{\isacharcolon}{\kern0pt}\ {\isachardoublequoteopen}nat\ {\isasymtimes}\ nat\ {\isasymtimes}\ rat\ {\isasymtimes}\ rat\ {\isasymRightarrow}\ nat{\isachardoublequoteclose}\ \isakeyword{where}\ {\isachardoublequoteopen}m{\isacharunderscore}{\kern0pt}of\ {\isacharequal}{\kern0pt}\ {\isacharparenleft}{\kern0pt}{\isasymlambda}{\isacharparenleft}{\kern0pt}n{\isacharcomma}{\kern0pt}\ m{\isacharcomma}{\kern0pt}\ {\isasymepsilon}{\isacharcomma}{\kern0pt}\ {\isasymdelta}{\isacharparenright}{\kern0pt}{\isachardot}{\kern0pt}\ m{\isacharparenright}{\kern0pt}{\isachardoublequoteclose}\isanewline
\ \ \isacommand{define}\isamarkupfalse%
\ {\isasymepsilon}{\isacharunderscore}{\kern0pt}of\ {\isacharcolon}{\kern0pt}{\isacharcolon}{\kern0pt}\ {\isachardoublequoteopen}nat\ {\isasymtimes}\ nat\ {\isasymtimes}\ rat\ {\isasymtimes}\ rat\ {\isasymRightarrow}\ rat{\isachardoublequoteclose}\ \isakeyword{where}\ {\isachardoublequoteopen}{\isasymepsilon}{\isacharunderscore}{\kern0pt}of\ {\isacharequal}{\kern0pt}\ {\isacharparenleft}{\kern0pt}{\isasymlambda}{\isacharparenleft}{\kern0pt}n{\isacharcomma}{\kern0pt}\ m{\isacharcomma}{\kern0pt}\ {\isasymepsilon}{\isacharcomma}{\kern0pt}\ {\isasymdelta}{\isacharparenright}{\kern0pt}{\isachardot}{\kern0pt}\ {\isasymepsilon}{\isacharparenright}{\kern0pt}{\isachardoublequoteclose}\isanewline
\ \ \isacommand{define}\isamarkupfalse%
\ {\isasymdelta}{\isacharunderscore}{\kern0pt}of\ {\isacharcolon}{\kern0pt}{\isacharcolon}{\kern0pt}\ {\isachardoublequoteopen}nat\ {\isasymtimes}\ nat\ {\isasymtimes}\ rat\ {\isasymtimes}\ rat\ {\isasymRightarrow}\ rat{\isachardoublequoteclose}\ \isakeyword{where}\ {\isachardoublequoteopen}{\isasymdelta}{\isacharunderscore}{\kern0pt}of\ {\isacharequal}{\kern0pt}\ {\isacharparenleft}{\kern0pt}{\isasymlambda}{\isacharparenleft}{\kern0pt}n{\isacharcomma}{\kern0pt}\ m{\isacharcomma}{\kern0pt}\ {\isasymepsilon}{\isacharcomma}{\kern0pt}\ {\isasymdelta}{\isacharparenright}{\kern0pt}{\isachardot}{\kern0pt}\ {\isasymdelta}{\isacharparenright}{\kern0pt}{\isachardoublequoteclose}\isanewline
\isanewline
\ \ \isacommand{define}\isamarkupfalse%
\ g\ \isakeyword{where}\ {\isachardoublequoteopen}g\ {\isacharequal}{\kern0pt}\ {\isacharparenleft}{\kern0pt}{\isasymlambda}x{\isachardot}{\kern0pt}\ {\isacharparenleft}{\kern0pt}ln\ {\isacharparenleft}{\kern0pt}{\isadigit{1}}\ {\isacharslash}{\kern0pt}\ of{\isacharunderscore}{\kern0pt}rat\ {\isacharparenleft}{\kern0pt}{\isasymepsilon}{\isacharunderscore}{\kern0pt}of\ x{\isacharparenright}{\kern0pt}{\isacharparenright}{\kern0pt}{\isacharparenright}{\kern0pt}\ {\isacharslash}{\kern0pt}\ {\isacharparenleft}{\kern0pt}of{\isacharunderscore}{\kern0pt}rat\ {\isacharparenleft}{\kern0pt}{\isasymdelta}{\isacharunderscore}{\kern0pt}of\ x{\isacharparenright}{\kern0pt}{\isacharparenright}{\kern0pt}\isactrlsup {\isadigit{2}}\ {\isacharasterisk}{\kern0pt}\ {\isacharparenleft}{\kern0pt}ln\ {\isacharparenleft}{\kern0pt}real\ {\isacharparenleft}{\kern0pt}n{\isacharunderscore}{\kern0pt}of\ x{\isacharparenright}{\kern0pt}{\isacharparenright}{\kern0pt}\ {\isacharplus}{\kern0pt}\ ln\ {\isacharparenleft}{\kern0pt}real\ {\isacharparenleft}{\kern0pt}m{\isacharunderscore}{\kern0pt}of\ x{\isacharparenright}{\kern0pt}{\isacharparenright}{\kern0pt}{\isacharparenright}{\kern0pt}{\isacharparenright}{\kern0pt}{\isachardoublequoteclose}\isanewline
\isanewline
\ \ \isacommand{have}\isamarkupfalse%
\ n{\isacharunderscore}{\kern0pt}inf{\isacharcolon}{\kern0pt}\ {\isachardoublequoteopen}{\isasymAnd}c{\isachardot}{\kern0pt}\ eventually\ {\isacharparenleft}{\kern0pt}{\isasymlambda}x{\isachardot}{\kern0pt}\ c\ {\isasymle}\ {\isacharparenleft}{\kern0pt}real\ {\isacharparenleft}{\kern0pt}n{\isacharunderscore}{\kern0pt}of\ x{\isacharparenright}{\kern0pt}{\isacharparenright}{\kern0pt}{\isacharparenright}{\kern0pt}\ {\isacharquery}{\kern0pt}F{\isachardoublequoteclose}\ \isanewline
\ \ \ \ \isacommand{apply}\isamarkupfalse%
\ {\isacharparenleft}{\kern0pt}simp\ add{\isacharcolon}{\kern0pt}n{\isacharunderscore}{\kern0pt}of{\isacharunderscore}{\kern0pt}def\ case{\isacharunderscore}{\kern0pt}prod{\isacharunderscore}{\kern0pt}beta{\isacharprime}{\kern0pt}{\isacharparenright}{\kern0pt}\isanewline
\ \ \ \ \isacommand{apply}\isamarkupfalse%
\ {\isacharparenleft}{\kern0pt}subst\ eventually{\isacharunderscore}{\kern0pt}prod{\isadigit{1}}{\isacharprime}{\kern0pt}{\isacharcomma}{\kern0pt}\ simp\ add{\isacharcolon}{\kern0pt}prod{\isacharunderscore}{\kern0pt}filter{\isacharunderscore}{\kern0pt}eq{\isacharunderscore}{\kern0pt}bot{\isacharparenright}{\kern0pt}\isanewline
\ \ \ \ \isacommand{by}\isamarkupfalse%
\ {\isacharparenleft}{\kern0pt}meson\ eventually{\isacharunderscore}{\kern0pt}at{\isacharunderscore}{\kern0pt}top{\isacharunderscore}{\kern0pt}linorder\ nat{\isacharunderscore}{\kern0pt}ceiling{\isacharunderscore}{\kern0pt}le{\isacharunderscore}{\kern0pt}eq{\isacharparenright}{\kern0pt}\isanewline
\isanewline
\ \ \isacommand{have}\isamarkupfalse%
\ m{\isacharunderscore}{\kern0pt}inf{\isacharcolon}{\kern0pt}\ {\isachardoublequoteopen}{\isasymAnd}c{\isachardot}{\kern0pt}\ eventually\ {\isacharparenleft}{\kern0pt}{\isasymlambda}x{\isachardot}{\kern0pt}\ c\ {\isasymle}\ {\isacharparenleft}{\kern0pt}real\ {\isacharparenleft}{\kern0pt}m{\isacharunderscore}{\kern0pt}of\ x{\isacharparenright}{\kern0pt}{\isacharparenright}{\kern0pt}{\isacharparenright}{\kern0pt}\ {\isacharquery}{\kern0pt}F{\isachardoublequoteclose}\ \isanewline
\ \ \ \ \isacommand{apply}\isamarkupfalse%
\ {\isacharparenleft}{\kern0pt}simp\ add{\isacharcolon}{\kern0pt}m{\isacharunderscore}{\kern0pt}of{\isacharunderscore}{\kern0pt}def\ case{\isacharunderscore}{\kern0pt}prod{\isacharunderscore}{\kern0pt}beta{\isacharprime}{\kern0pt}{\isacharparenright}{\kern0pt}\isanewline
\ \ \ \ \isacommand{apply}\isamarkupfalse%
\ {\isacharparenleft}{\kern0pt}subst\ eventually{\isacharunderscore}{\kern0pt}prod{\isadigit{2}}{\isacharprime}{\kern0pt}{\isacharcomma}{\kern0pt}\ simp\ add{\isacharcolon}{\kern0pt}prod{\isacharunderscore}{\kern0pt}filter{\isacharunderscore}{\kern0pt}eq{\isacharunderscore}{\kern0pt}bot{\isacharparenright}{\kern0pt}\isanewline
\ \ \ \ \isacommand{apply}\isamarkupfalse%
\ {\isacharparenleft}{\kern0pt}subst\ eventually{\isacharunderscore}{\kern0pt}prod{\isadigit{1}}{\isacharprime}{\kern0pt}{\isacharcomma}{\kern0pt}\ simp\ add{\isacharcolon}{\kern0pt}prod{\isacharunderscore}{\kern0pt}filter{\isacharunderscore}{\kern0pt}eq{\isacharunderscore}{\kern0pt}bot{\isacharparenright}{\kern0pt}\isanewline
\ \ \ \ \isacommand{by}\isamarkupfalse%
\ {\isacharparenleft}{\kern0pt}meson\ eventually{\isacharunderscore}{\kern0pt}at{\isacharunderscore}{\kern0pt}top{\isacharunderscore}{\kern0pt}linorder\ nat{\isacharunderscore}{\kern0pt}ceiling{\isacharunderscore}{\kern0pt}le{\isacharunderscore}{\kern0pt}eq{\isacharparenright}{\kern0pt}\isanewline
\isanewline
\ \ \isacommand{have}\isamarkupfalse%
\ eps{\isacharunderscore}{\kern0pt}inf{\isacharcolon}{\kern0pt}\ {\isachardoublequoteopen}{\isasymAnd}c{\isachardot}{\kern0pt}\ eventually\ {\isacharparenleft}{\kern0pt}{\isasymlambda}x{\isachardot}{\kern0pt}\ c\ {\isasymle}\ {\isadigit{1}}\ {\isacharslash}{\kern0pt}\ {\isacharparenleft}{\kern0pt}real{\isacharunderscore}{\kern0pt}of{\isacharunderscore}{\kern0pt}rat\ {\isacharparenleft}{\kern0pt}{\isasymepsilon}{\isacharunderscore}{\kern0pt}of\ x{\isacharparenright}{\kern0pt}{\isacharparenright}{\kern0pt}{\isacharparenright}{\kern0pt}\ {\isacharquery}{\kern0pt}F{\isachardoublequoteclose}\isanewline
\ \ \ \ \isacommand{apply}\isamarkupfalse%
\ {\isacharparenleft}{\kern0pt}simp\ add{\isacharcolon}{\kern0pt}{\isasymepsilon}{\isacharunderscore}{\kern0pt}of{\isacharunderscore}{\kern0pt}def\ case{\isacharunderscore}{\kern0pt}prod{\isacharunderscore}{\kern0pt}beta{\isacharprime}{\kern0pt}{\isacharparenright}{\kern0pt}\isanewline
\ \ \ \ \isacommand{apply}\isamarkupfalse%
\ {\isacharparenleft}{\kern0pt}subst\ eventually{\isacharunderscore}{\kern0pt}prod{\isadigit{2}}{\isacharprime}{\kern0pt}{\isacharcomma}{\kern0pt}\ simp{\isacharparenright}{\kern0pt}\isanewline
\ \ \ \ \isacommand{apply}\isamarkupfalse%
\ {\isacharparenleft}{\kern0pt}subst\ eventually{\isacharunderscore}{\kern0pt}prod{\isadigit{2}}{\isacharprime}{\kern0pt}{\isacharcomma}{\kern0pt}\ simp{\isacharparenright}{\kern0pt}\isanewline
\ \ \ \ \isacommand{apply}\isamarkupfalse%
\ {\isacharparenleft}{\kern0pt}subst\ eventually{\isacharunderscore}{\kern0pt}prod{\isadigit{1}}{\isacharprime}{\kern0pt}{\isacharcomma}{\kern0pt}\ simp{\isacharparenright}{\kern0pt}\isanewline
\ \ \ \ \isacommand{by}\isamarkupfalse%
\ {\isacharparenleft}{\kern0pt}rule\ inv{\isacharunderscore}{\kern0pt}at{\isacharunderscore}{\kern0pt}right{\isacharunderscore}{\kern0pt}{\isadigit{0}}{\isacharunderscore}{\kern0pt}inf{\isacharparenright}{\kern0pt}\isanewline
\isanewline
\ \ \isacommand{have}\isamarkupfalse%
\ delta{\isacharunderscore}{\kern0pt}inf{\isacharcolon}{\kern0pt}\ {\isachardoublequoteopen}{\isasymAnd}c{\isachardot}{\kern0pt}\ eventually\ {\isacharparenleft}{\kern0pt}{\isasymlambda}x{\isachardot}{\kern0pt}\ c\ {\isasymle}\ {\isadigit{1}}\ {\isacharslash}{\kern0pt}\ {\isacharparenleft}{\kern0pt}real{\isacharunderscore}{\kern0pt}of{\isacharunderscore}{\kern0pt}rat\ {\isacharparenleft}{\kern0pt}{\isasymdelta}{\isacharunderscore}{\kern0pt}of\ x{\isacharparenright}{\kern0pt}{\isacharparenright}{\kern0pt}{\isacharparenright}{\kern0pt}\ {\isacharquery}{\kern0pt}F{\isachardoublequoteclose}\isanewline
\ \ \ \ \isacommand{apply}\isamarkupfalse%
\ {\isacharparenleft}{\kern0pt}simp\ add{\isacharcolon}{\kern0pt}{\isasymdelta}{\isacharunderscore}{\kern0pt}of{\isacharunderscore}{\kern0pt}def\ case{\isacharunderscore}{\kern0pt}prod{\isacharunderscore}{\kern0pt}beta{\isacharprime}{\kern0pt}{\isacharparenright}{\kern0pt}\isanewline
\ \ \ \ \isacommand{apply}\isamarkupfalse%
\ {\isacharparenleft}{\kern0pt}subst\ eventually{\isacharunderscore}{\kern0pt}prod{\isadigit{2}}{\isacharprime}{\kern0pt}{\isacharcomma}{\kern0pt}\ simp{\isacharparenright}{\kern0pt}\isanewline
\ \ \ \ \isacommand{apply}\isamarkupfalse%
\ {\isacharparenleft}{\kern0pt}subst\ eventually{\isacharunderscore}{\kern0pt}prod{\isadigit{2}}{\isacharprime}{\kern0pt}{\isacharcomma}{\kern0pt}\ simp{\isacharparenright}{\kern0pt}\isanewline
\ \ \ \ \isacommand{apply}\isamarkupfalse%
\ {\isacharparenleft}{\kern0pt}subst\ eventually{\isacharunderscore}{\kern0pt}prod{\isadigit{2}}{\isacharprime}{\kern0pt}{\isacharcomma}{\kern0pt}\ simp{\isacharparenright}{\kern0pt}\isanewline
\ \ \ \ \isacommand{by}\isamarkupfalse%
\ {\isacharparenleft}{\kern0pt}rule\ inv{\isacharunderscore}{\kern0pt}at{\isacharunderscore}{\kern0pt}right{\isacharunderscore}{\kern0pt}{\isadigit{0}}{\isacharunderscore}{\kern0pt}inf{\isacharparenright}{\kern0pt}\isanewline
\isanewline
\ \ \isacommand{have}\isamarkupfalse%
\ zero{\isacharunderscore}{\kern0pt}less{\isacharunderscore}{\kern0pt}eps{\isacharcolon}{\kern0pt}\ {\isachardoublequoteopen}eventually\ {\isacharparenleft}{\kern0pt}{\isasymlambda}x{\isachardot}{\kern0pt}\ {\isadigit{0}}\ {\isacharless}{\kern0pt}\ {\isacharparenleft}{\kern0pt}real{\isacharunderscore}{\kern0pt}of{\isacharunderscore}{\kern0pt}rat\ {\isacharparenleft}{\kern0pt}{\isasymepsilon}{\isacharunderscore}{\kern0pt}of\ x{\isacharparenright}{\kern0pt}{\isacharparenright}{\kern0pt}{\isacharparenright}{\kern0pt}\ {\isacharquery}{\kern0pt}F{\isachardoublequoteclose}\isanewline
\ \ \ \ \isacommand{apply}\isamarkupfalse%
\ {\isacharparenleft}{\kern0pt}simp\ add{\isacharcolon}{\kern0pt}{\isasymepsilon}{\isacharunderscore}{\kern0pt}of{\isacharunderscore}{\kern0pt}def\ case{\isacharunderscore}{\kern0pt}prod{\isacharunderscore}{\kern0pt}beta{\isacharprime}{\kern0pt}{\isacharparenright}{\kern0pt}\isanewline
\ \ \ \ \isacommand{apply}\isamarkupfalse%
\ {\isacharparenleft}{\kern0pt}subst\ eventually{\isacharunderscore}{\kern0pt}prod{\isadigit{2}}{\isacharprime}{\kern0pt}{\isacharcomma}{\kern0pt}\ simp{\isacharparenright}{\kern0pt}\isanewline
\ \ \ \ \isacommand{apply}\isamarkupfalse%
\ {\isacharparenleft}{\kern0pt}subst\ eventually{\isacharunderscore}{\kern0pt}prod{\isadigit{2}}{\isacharprime}{\kern0pt}{\isacharcomma}{\kern0pt}\ simp{\isacharparenright}{\kern0pt}\isanewline
\ \ \ \ \isacommand{apply}\isamarkupfalse%
\ {\isacharparenleft}{\kern0pt}subst\ eventually{\isacharunderscore}{\kern0pt}prod{\isadigit{1}}{\isacharprime}{\kern0pt}{\isacharcomma}{\kern0pt}\ simp{\isacharparenright}{\kern0pt}\isanewline
\ \ \ \ \isacommand{by}\isamarkupfalse%
\ {\isacharparenleft}{\kern0pt}rule\ eventually{\isacharunderscore}{\kern0pt}at{\isacharunderscore}{\kern0pt}rightI{\isacharbrackleft}{\kern0pt}\isakeyword{where}\ b{\isacharequal}{\kern0pt}{\isachardoublequoteopen}{\isadigit{1}}{\isachardoublequoteclose}{\isacharbrackright}{\kern0pt}{\isacharcomma}{\kern0pt}\ simp{\isacharcomma}{\kern0pt}\ simp{\isacharparenright}{\kern0pt}\isanewline
\isanewline
\ \ \isacommand{have}\isamarkupfalse%
\ zero{\isacharunderscore}{\kern0pt}less{\isacharunderscore}{\kern0pt}delta{\isacharcolon}{\kern0pt}\ {\isachardoublequoteopen}eventually\ {\isacharparenleft}{\kern0pt}{\isasymlambda}x{\isachardot}{\kern0pt}\ {\isadigit{0}}\ {\isacharless}{\kern0pt}\ {\isacharparenleft}{\kern0pt}real{\isacharunderscore}{\kern0pt}of{\isacharunderscore}{\kern0pt}rat\ {\isacharparenleft}{\kern0pt}{\isasymdelta}{\isacharunderscore}{\kern0pt}of\ x{\isacharparenright}{\kern0pt}{\isacharparenright}{\kern0pt}{\isacharparenright}{\kern0pt}\ {\isacharquery}{\kern0pt}F{\isachardoublequoteclose}\isanewline
\ \ \ \ \isacommand{apply}\isamarkupfalse%
\ {\isacharparenleft}{\kern0pt}simp\ add{\isacharcolon}{\kern0pt}{\isasymdelta}{\isacharunderscore}{\kern0pt}of{\isacharunderscore}{\kern0pt}def\ case{\isacharunderscore}{\kern0pt}prod{\isacharunderscore}{\kern0pt}beta{\isacharprime}{\kern0pt}{\isacharparenright}{\kern0pt}\isanewline
\ \ \ \ \isacommand{apply}\isamarkupfalse%
\ {\isacharparenleft}{\kern0pt}subst\ eventually{\isacharunderscore}{\kern0pt}prod{\isadigit{2}}{\isacharprime}{\kern0pt}{\isacharcomma}{\kern0pt}\ simp{\isacharparenright}{\kern0pt}\isanewline
\ \ \ \ \isacommand{apply}\isamarkupfalse%
\ {\isacharparenleft}{\kern0pt}subst\ eventually{\isacharunderscore}{\kern0pt}prod{\isadigit{2}}{\isacharprime}{\kern0pt}{\isacharcomma}{\kern0pt}\ simp{\isacharparenright}{\kern0pt}\isanewline
\ \ \ \ \isacommand{apply}\isamarkupfalse%
\ {\isacharparenleft}{\kern0pt}subst\ eventually{\isacharunderscore}{\kern0pt}prod{\isadigit{2}}{\isacharprime}{\kern0pt}{\isacharcomma}{\kern0pt}\ simp{\isacharparenright}{\kern0pt}\isanewline
\ \ \ \ \isacommand{by}\isamarkupfalse%
\ {\isacharparenleft}{\kern0pt}rule\ eventually{\isacharunderscore}{\kern0pt}at{\isacharunderscore}{\kern0pt}rightI{\isacharbrackleft}{\kern0pt}\isakeyword{where}\ b{\isacharequal}{\kern0pt}{\isachardoublequoteopen}{\isadigit{1}}{\isachardoublequoteclose}{\isacharbrackright}{\kern0pt}{\isacharcomma}{\kern0pt}\ simp{\isacharcomma}{\kern0pt}\ simp{\isacharparenright}{\kern0pt}\isanewline
\isanewline
\ \ \isacommand{have}\isamarkupfalse%
\ unit{\isacharunderscore}{\kern0pt}{\isadigit{1}}{\isacharcolon}{\kern0pt}\ {\isachardoublequoteopen}{\isacharparenleft}{\kern0pt}{\isasymlambda}{\isacharunderscore}{\kern0pt}{\isachardot}{\kern0pt}\ {\isadigit{1}}{\isacharparenright}{\kern0pt}\ {\isasymin}\ O{\isacharbrackleft}{\kern0pt}{\isacharquery}{\kern0pt}F{\isacharbrackright}{\kern0pt}{\isacharparenleft}{\kern0pt}{\isasymlambda}x{\isachardot}{\kern0pt}\ {\isadigit{1}}\ {\isacharslash}{\kern0pt}\ {\isacharparenleft}{\kern0pt}real{\isacharunderscore}{\kern0pt}of{\isacharunderscore}{\kern0pt}rat\ {\isacharparenleft}{\kern0pt}{\isasymdelta}{\isacharunderscore}{\kern0pt}of\ x{\isacharparenright}{\kern0pt}{\isacharparenright}{\kern0pt}\isactrlsup {\isadigit{2}}{\isacharparenright}{\kern0pt}{\isachardoublequoteclose}\isanewline
\ \ \ \ \isacommand{apply}\isamarkupfalse%
\ {\isacharparenleft}{\kern0pt}rule\ landau{\isacharunderscore}{\kern0pt}o{\isachardot}{\kern0pt}big{\isacharunderscore}{\kern0pt}mono{\isacharcomma}{\kern0pt}\ simp{\isacharparenright}{\kern0pt}\isanewline
\ \ \ \ \isacommand{apply}\isamarkupfalse%
\ {\isacharparenleft}{\kern0pt}rule\ eventually{\isacharunderscore}{\kern0pt}mono{\isacharbrackleft}{\kern0pt}OF\ eventually{\isacharunderscore}{\kern0pt}conj{\isacharbrackleft}{\kern0pt}OF\ zero{\isacharunderscore}{\kern0pt}less{\isacharunderscore}{\kern0pt}delta\ delta{\isacharunderscore}{\kern0pt}inf{\isacharbrackleft}{\kern0pt}\isakeyword{where}\ c{\isacharequal}{\kern0pt}{\isachardoublequoteopen}{\isadigit{1}}{\isachardoublequoteclose}{\isacharbrackright}{\kern0pt}{\isacharbrackright}{\kern0pt}{\isacharbrackright}{\kern0pt}{\isacharparenright}{\kern0pt}\isanewline
\ \ \ \ \isacommand{by}\isamarkupfalse%
\ {\isacharparenleft}{\kern0pt}metis\ one{\isacharunderscore}{\kern0pt}le{\isacharunderscore}{\kern0pt}power\ power{\isacharunderscore}{\kern0pt}one{\isacharunderscore}{\kern0pt}over{\isacharparenright}{\kern0pt}\isanewline
\isanewline
\ \ \isacommand{have}\isamarkupfalse%
\ unit{\isacharunderscore}{\kern0pt}{\isadigit{2}}{\isacharcolon}{\kern0pt}\ {\isachardoublequoteopen}{\isacharparenleft}{\kern0pt}{\isasymlambda}{\isacharunderscore}{\kern0pt}{\isachardot}{\kern0pt}\ {\isadigit{1}}{\isacharparenright}{\kern0pt}\ {\isasymin}\ O{\isacharbrackleft}{\kern0pt}{\isacharquery}{\kern0pt}F{\isacharbrackright}{\kern0pt}{\isacharparenleft}{\kern0pt}{\isasymlambda}x{\isachardot}{\kern0pt}\ ln\ {\isacharparenleft}{\kern0pt}{\isadigit{1}}\ {\isacharslash}{\kern0pt}\ real{\isacharunderscore}{\kern0pt}of{\isacharunderscore}{\kern0pt}rat\ {\isacharparenleft}{\kern0pt}{\isasymepsilon}{\isacharunderscore}{\kern0pt}of\ x{\isacharparenright}{\kern0pt}{\isacharparenright}{\kern0pt}{\isacharparenright}{\kern0pt}{\isachardoublequoteclose}\isanewline
\ \ \ \ \isacommand{apply}\isamarkupfalse%
\ {\isacharparenleft}{\kern0pt}rule\ landau{\isacharunderscore}{\kern0pt}o{\isachardot}{\kern0pt}big{\isacharunderscore}{\kern0pt}mono{\isacharcomma}{\kern0pt}\ simp{\isacharparenright}{\kern0pt}\isanewline
\ \ \ \ \isacommand{apply}\isamarkupfalse%
\ {\isacharparenleft}{\kern0pt}rule\ eventually{\isacharunderscore}{\kern0pt}mono{\isacharbrackleft}{\kern0pt}OF\ eventually{\isacharunderscore}{\kern0pt}conj{\isacharbrackleft}{\kern0pt}OF\ zero{\isacharunderscore}{\kern0pt}less{\isacharunderscore}{\kern0pt}eps\ eps{\isacharunderscore}{\kern0pt}inf{\isacharbrackleft}{\kern0pt}\isakeyword{where}\ c{\isacharequal}{\kern0pt}{\isachardoublequoteopen}exp\ {\isadigit{1}}{\isachardoublequoteclose}{\isacharbrackright}{\kern0pt}{\isacharbrackright}{\kern0pt}{\isacharbrackright}{\kern0pt}{\isacharparenright}{\kern0pt}\isanewline
\ \ \ \ \isacommand{by}\isamarkupfalse%
\ {\isacharparenleft}{\kern0pt}meson\ abs{\isacharunderscore}{\kern0pt}ge{\isacharunderscore}{\kern0pt}self\ dual{\isacharunderscore}{\kern0pt}order{\isachardot}{\kern0pt}trans\ exp{\isacharunderscore}{\kern0pt}gt{\isacharunderscore}{\kern0pt}zero\ ln{\isacharunderscore}{\kern0pt}ge{\isacharunderscore}{\kern0pt}iff\ order{\isacharunderscore}{\kern0pt}trans{\isacharunderscore}{\kern0pt}rules{\isacharparenleft}{\kern0pt}{\isadigit{2}}{\isadigit{2}}{\isacharparenright}{\kern0pt}{\isacharparenright}{\kern0pt}\isanewline
\isanewline
\ \ \isacommand{have}\isamarkupfalse%
\ unit{\isacharunderscore}{\kern0pt}{\isadigit{3}}{\isacharcolon}{\kern0pt}\ {\isachardoublequoteopen}{\isacharparenleft}{\kern0pt}{\isasymlambda}{\isacharunderscore}{\kern0pt}{\isachardot}{\kern0pt}\ {\isadigit{1}}{\isacharparenright}{\kern0pt}\ {\isasymin}\ O{\isacharbrackleft}{\kern0pt}{\isacharquery}{\kern0pt}F{\isacharbrackright}{\kern0pt}{\isacharparenleft}{\kern0pt}{\isasymlambda}x{\isachardot}{\kern0pt}\ real\ {\isacharparenleft}{\kern0pt}n{\isacharunderscore}{\kern0pt}of\ x{\isacharparenright}{\kern0pt}{\isacharparenright}{\kern0pt}{\isachardoublequoteclose}\isanewline
\ \ \ \ \isacommand{by}\isamarkupfalse%
\ {\isacharparenleft}{\kern0pt}rule\ landau{\isacharunderscore}{\kern0pt}o{\isachardot}{\kern0pt}big{\isacharunderscore}{\kern0pt}mono{\isacharcomma}{\kern0pt}\ simp{\isacharcomma}{\kern0pt}\ rule\ n{\isacharunderscore}{\kern0pt}inf{\isacharparenright}{\kern0pt}\isanewline
\isanewline
\ \ \isacommand{have}\isamarkupfalse%
\ unit{\isacharunderscore}{\kern0pt}{\isadigit{4}}{\isacharcolon}{\kern0pt}\ {\isachardoublequoteopen}{\isacharparenleft}{\kern0pt}{\isasymlambda}{\isacharunderscore}{\kern0pt}{\isachardot}{\kern0pt}\ {\isadigit{1}}{\isacharparenright}{\kern0pt}\ {\isasymin}\ O{\isacharbrackleft}{\kern0pt}{\isacharquery}{\kern0pt}F{\isacharbrackright}{\kern0pt}{\isacharparenleft}{\kern0pt}{\isasymlambda}x{\isachardot}{\kern0pt}\ real\ {\isacharparenleft}{\kern0pt}m{\isacharunderscore}{\kern0pt}of\ x{\isacharparenright}{\kern0pt}{\isacharparenright}{\kern0pt}{\isachardoublequoteclose}\isanewline
\ \ \ \ \isacommand{by}\isamarkupfalse%
\ {\isacharparenleft}{\kern0pt}rule\ landau{\isacharunderscore}{\kern0pt}o{\isachardot}{\kern0pt}big{\isacharunderscore}{\kern0pt}mono{\isacharcomma}{\kern0pt}\ simp{\isacharcomma}{\kern0pt}\ rule\ m{\isacharunderscore}{\kern0pt}inf{\isacharparenright}{\kern0pt}\isanewline
\isanewline
\ \ \isacommand{have}\isamarkupfalse%
\ unit{\isacharunderscore}{\kern0pt}{\isadigit{5}}{\isacharcolon}{\kern0pt}\ {\isachardoublequoteopen}{\isacharparenleft}{\kern0pt}{\isasymlambda}{\isacharunderscore}{\kern0pt}{\isachardot}{\kern0pt}\ {\isadigit{1}}{\isacharparenright}{\kern0pt}\ {\isasymin}\ O{\isacharbrackleft}{\kern0pt}{\isacharquery}{\kern0pt}F{\isacharbrackright}{\kern0pt}{\isacharparenleft}{\kern0pt}{\isasymlambda}x{\isachardot}{\kern0pt}\ ln\ {\isacharparenleft}{\kern0pt}real\ {\isacharparenleft}{\kern0pt}n{\isacharunderscore}{\kern0pt}of\ x{\isacharparenright}{\kern0pt}{\isacharparenright}{\kern0pt}{\isacharparenright}{\kern0pt}{\isachardoublequoteclose}\isanewline
\ \ \ \ \isacommand{apply}\isamarkupfalse%
\ {\isacharparenleft}{\kern0pt}rule\ landau{\isacharunderscore}{\kern0pt}o{\isachardot}{\kern0pt}big{\isacharunderscore}{\kern0pt}mono{\isacharcomma}{\kern0pt}\ simp{\isacharparenright}{\kern0pt}\isanewline
\ \ \ \ \isacommand{apply}\isamarkupfalse%
\ {\isacharparenleft}{\kern0pt}rule\ eventually{\isacharunderscore}{\kern0pt}mono\ {\isacharbrackleft}{\kern0pt}OF\ n{\isacharunderscore}{\kern0pt}inf{\isacharbrackleft}{\kern0pt}\isakeyword{where}\ c{\isacharequal}{\kern0pt}{\isachardoublequoteopen}exp\ {\isadigit{1}}{\isachardoublequoteclose}{\isacharbrackright}{\kern0pt}{\isacharbrackright}{\kern0pt}{\isacharparenright}{\kern0pt}\ \isanewline
\ \ \ \ \isacommand{by}\isamarkupfalse%
\ {\isacharparenleft}{\kern0pt}metis\ abs{\isacharunderscore}{\kern0pt}ge{\isacharunderscore}{\kern0pt}self\ linorder{\isacharunderscore}{\kern0pt}not{\isacharunderscore}{\kern0pt}le\ ln{\isacharunderscore}{\kern0pt}ge{\isacharunderscore}{\kern0pt}iff\ not{\isacharunderscore}{\kern0pt}exp{\isacharunderscore}{\kern0pt}le{\isacharunderscore}{\kern0pt}zero\ order{\isachardot}{\kern0pt}trans{\isacharparenright}{\kern0pt}\isanewline
\isanewline
\ \ \isacommand{have}\isamarkupfalse%
\ unit{\isacharunderscore}{\kern0pt}{\isadigit{6}}{\isacharcolon}{\kern0pt}\ {\isachardoublequoteopen}{\isacharparenleft}{\kern0pt}{\isasymlambda}{\isacharunderscore}{\kern0pt}{\isachardot}{\kern0pt}\ {\isadigit{1}}{\isacharparenright}{\kern0pt}\ {\isasymin}\ O{\isacharbrackleft}{\kern0pt}{\isacharquery}{\kern0pt}F{\isacharbrackright}{\kern0pt}{\isacharparenleft}{\kern0pt}{\isasymlambda}x{\isachardot}{\kern0pt}\ ln\ {\isacharparenleft}{\kern0pt}real\ {\isacharparenleft}{\kern0pt}n{\isacharunderscore}{\kern0pt}of\ x{\isacharparenright}{\kern0pt}{\isacharparenright}{\kern0pt}\ {\isacharplus}{\kern0pt}\ ln\ {\isacharparenleft}{\kern0pt}real\ {\isacharparenleft}{\kern0pt}m{\isacharunderscore}{\kern0pt}of\ x{\isacharparenright}{\kern0pt}{\isacharparenright}{\kern0pt}{\isacharparenright}{\kern0pt}{\isachardoublequoteclose}\isanewline
\ \ \ \ \isacommand{apply}\isamarkupfalse%
\ {\isacharparenleft}{\kern0pt}rule\ landau{\isacharunderscore}{\kern0pt}sum{\isacharunderscore}{\kern0pt}{\isadigit{1}}{\isacharparenright}{\kern0pt}\isanewline
\ \ \ \ \ \ \isacommand{apply}\isamarkupfalse%
\ {\isacharparenleft}{\kern0pt}rule\ eventually{\isacharunderscore}{\kern0pt}ln{\isacharunderscore}{\kern0pt}ge{\isacharunderscore}{\kern0pt}iff{\isacharbrackleft}{\kern0pt}OF\ n{\isacharunderscore}{\kern0pt}inf{\isacharbrackright}{\kern0pt}{\isacharparenright}{\kern0pt}\isanewline
\ \ \ \ \ \isacommand{apply}\isamarkupfalse%
\ {\isacharparenleft}{\kern0pt}rule\ eventually{\isacharunderscore}{\kern0pt}ln{\isacharunderscore}{\kern0pt}ge{\isacharunderscore}{\kern0pt}iff{\isacharbrackleft}{\kern0pt}OF\ m{\isacharunderscore}{\kern0pt}inf{\isacharbrackright}{\kern0pt}{\isacharparenright}{\kern0pt}\isanewline
\ \ \ \ \isacommand{by}\isamarkupfalse%
\ {\isacharparenleft}{\kern0pt}rule\ unit{\isacharunderscore}{\kern0pt}{\isadigit{5}}{\isacharparenright}{\kern0pt}\isanewline
\isanewline
\ \ \isacommand{have}\isamarkupfalse%
\ unit{\isacharunderscore}{\kern0pt}{\isadigit{7}}{\isacharcolon}{\kern0pt}\ {\isachardoublequoteopen}{\isacharparenleft}{\kern0pt}{\isasymlambda}{\isacharunderscore}{\kern0pt}{\isachardot}{\kern0pt}\ {\isadigit{1}}{\isacharparenright}{\kern0pt}\ {\isasymin}\ O{\isacharbrackleft}{\kern0pt}{\isacharquery}{\kern0pt}F{\isacharbrackright}{\kern0pt}{\isacharparenleft}{\kern0pt}{\isasymlambda}x{\isachardot}{\kern0pt}\ {\isadigit{1}}\ {\isacharslash}{\kern0pt}\ real{\isacharunderscore}{\kern0pt}of{\isacharunderscore}{\kern0pt}rat\ {\isacharparenleft}{\kern0pt}{\isasymepsilon}{\isacharunderscore}{\kern0pt}of\ x{\isacharparenright}{\kern0pt}{\isacharparenright}{\kern0pt}{\isachardoublequoteclose}\isanewline
\ \ \ \ \isacommand{apply}\isamarkupfalse%
\ {\isacharparenleft}{\kern0pt}rule\ landau{\isacharunderscore}{\kern0pt}o{\isachardot}{\kern0pt}big{\isacharunderscore}{\kern0pt}mono{\isacharcomma}{\kern0pt}\ simp{\isacharparenright}{\kern0pt}\isanewline
\ \ \ \ \isacommand{apply}\isamarkupfalse%
\ {\isacharparenleft}{\kern0pt}rule\ eventually{\isacharunderscore}{\kern0pt}mono\ {\isacharbrackleft}{\kern0pt}OF\ eventually{\isacharunderscore}{\kern0pt}conj{\isacharbrackleft}{\kern0pt}OF\ zero{\isacharunderscore}{\kern0pt}less{\isacharunderscore}{\kern0pt}eps\ eps{\isacharunderscore}{\kern0pt}inf{\isacharbrackleft}{\kern0pt}\isakeyword{where}\ c{\isacharequal}{\kern0pt}{\isachardoublequoteopen}{\isadigit{1}}{\isachardoublequoteclose}{\isacharbrackright}{\kern0pt}{\isacharbrackright}{\kern0pt}{\isacharbrackright}{\kern0pt}{\isacharparenright}{\kern0pt}\isanewline
\ \ \ \ \isacommand{by}\isamarkupfalse%
\ simp\isanewline
\isanewline
\ \ \isacommand{have}\isamarkupfalse%
\ unit{\isacharunderscore}{\kern0pt}{\isadigit{8}}{\isacharcolon}{\kern0pt}\ {\isachardoublequoteopen}{\isacharparenleft}{\kern0pt}{\isasymlambda}{\isacharunderscore}{\kern0pt}{\isachardot}{\kern0pt}\ {\isadigit{1}}{\isacharparenright}{\kern0pt}\ {\isasymin}\ O{\isacharbrackleft}{\kern0pt}{\isacharquery}{\kern0pt}F{\isacharbrackright}{\kern0pt}{\isacharparenleft}{\kern0pt}{\isasymlambda}x{\isachardot}{\kern0pt}\ ln\ {\isacharparenleft}{\kern0pt}{\isadigit{1}}\ {\isacharslash}{\kern0pt}\ real{\isacharunderscore}{\kern0pt}of{\isacharunderscore}{\kern0pt}rat\ {\isacharparenleft}{\kern0pt}{\isasymepsilon}{\isacharunderscore}{\kern0pt}of\ x{\isacharparenright}{\kern0pt}{\isacharparenright}{\kern0pt}\ {\isacharasterisk}{\kern0pt}\isanewline
\ \ \ \ {\isacharparenleft}{\kern0pt}ln\ {\isacharparenleft}{\kern0pt}real\ {\isacharparenleft}{\kern0pt}n{\isacharunderscore}{\kern0pt}of\ x{\isacharparenright}{\kern0pt}{\isacharparenright}{\kern0pt}\ {\isacharplus}{\kern0pt}\ ln\ {\isacharparenleft}{\kern0pt}real\ {\isacharparenleft}{\kern0pt}m{\isacharunderscore}{\kern0pt}of\ x{\isacharparenright}{\kern0pt}{\isacharparenright}{\kern0pt}{\isacharparenright}{\kern0pt}\ {\isacharslash}{\kern0pt}\ {\isacharparenleft}{\kern0pt}real{\isacharunderscore}{\kern0pt}of{\isacharunderscore}{\kern0pt}rat\ {\isacharparenleft}{\kern0pt}{\isasymdelta}{\isacharunderscore}{\kern0pt}of\ x{\isacharparenright}{\kern0pt}{\isacharparenright}{\kern0pt}\isactrlsup {\isadigit{2}}{\isacharparenright}{\kern0pt}{\isachardoublequoteclose}\ \isanewline
\ \ \ \ \isacommand{apply}\isamarkupfalse%
\ {\isacharparenleft}{\kern0pt}subst\ {\isacharparenleft}{\kern0pt}{\isadigit{2}}{\isacharparenright}{\kern0pt}\ div{\isacharunderscore}{\kern0pt}commute{\isacharparenright}{\kern0pt}\isanewline
\ \ \ \ \isacommand{apply}\isamarkupfalse%
\ {\isacharparenleft}{\kern0pt}rule\ landau{\isacharunderscore}{\kern0pt}o{\isachardot}{\kern0pt}big{\isacharunderscore}{\kern0pt}mult{\isacharunderscore}{\kern0pt}{\isadigit{1}}{\isacharbrackleft}{\kern0pt}OF\ unit{\isacharunderscore}{\kern0pt}{\isadigit{1}}{\isacharbrackright}{\kern0pt}{\isacharparenright}{\kern0pt}\isanewline
\ \ \ \ \isacommand{by}\isamarkupfalse%
\ {\isacharparenleft}{\kern0pt}rule\ landau{\isacharunderscore}{\kern0pt}o{\isachardot}{\kern0pt}big{\isacharunderscore}{\kern0pt}mult{\isacharunderscore}{\kern0pt}{\isadigit{1}}{\isacharbrackleft}{\kern0pt}OF\ unit{\isacharunderscore}{\kern0pt}{\isadigit{2}}\ unit{\isacharunderscore}{\kern0pt}{\isadigit{6}}{\isacharbrackright}{\kern0pt}{\isacharparenright}{\kern0pt}\ \isanewline
\isanewline
\ \ \isacommand{have}\isamarkupfalse%
\ unit{\isacharunderscore}{\kern0pt}{\isadigit{9}}{\isacharcolon}{\kern0pt}\ {\isachardoublequoteopen}{\isacharparenleft}{\kern0pt}{\isasymlambda}{\isacharunderscore}{\kern0pt}{\isachardot}{\kern0pt}\ {\isadigit{1}}{\isacharparenright}{\kern0pt}\ {\isasymin}\ O{\isacharbrackleft}{\kern0pt}{\isacharquery}{\kern0pt}F{\isacharbrackright}{\kern0pt}{\isacharparenleft}{\kern0pt}{\isasymlambda}x{\isachardot}{\kern0pt}\ real\ {\isacharparenleft}{\kern0pt}n{\isacharunderscore}{\kern0pt}of\ x{\isacharparenright}{\kern0pt}\ {\isacharasterisk}{\kern0pt}\ real\ {\isacharparenleft}{\kern0pt}m{\isacharunderscore}{\kern0pt}of\ x{\isacharparenright}{\kern0pt}{\isacharparenright}{\kern0pt}{\isachardoublequoteclose}\isanewline
\ \ \ \ \isacommand{by}\isamarkupfalse%
\ {\isacharparenleft}{\kern0pt}rule\ landau{\isacharunderscore}{\kern0pt}o{\isachardot}{\kern0pt}big{\isacharunderscore}{\kern0pt}mult{\isacharunderscore}{\kern0pt}{\isadigit{1}}{\isacharprime}{\kern0pt}{\isacharbrackleft}{\kern0pt}OF\ unit{\isacharunderscore}{\kern0pt}{\isadigit{3}}\ unit{\isacharunderscore}{\kern0pt}{\isadigit{4}}{\isacharbrackright}{\kern0pt}{\isacharparenright}{\kern0pt}\isanewline
\isanewline
\ \ \isacommand{have}\isamarkupfalse%
\ zero{\isacharunderscore}{\kern0pt}less{\isacharunderscore}{\kern0pt}eps{\isacharcolon}{\kern0pt}\ {\isachardoublequoteopen}eventually\ {\isacharparenleft}{\kern0pt}{\isasymlambda}x{\isachardot}{\kern0pt}\ {\isadigit{0}}\ {\isacharless}{\kern0pt}\ {\isacharparenleft}{\kern0pt}real{\isacharunderscore}{\kern0pt}of{\isacharunderscore}{\kern0pt}rat\ {\isacharparenleft}{\kern0pt}{\isasymepsilon}{\isacharunderscore}{\kern0pt}of\ x{\isacharparenright}{\kern0pt}{\isacharparenright}{\kern0pt}{\isacharparenright}{\kern0pt}\ {\isacharquery}{\kern0pt}F{\isachardoublequoteclose}\isanewline
\ \ \ \ \isacommand{apply}\isamarkupfalse%
\ {\isacharparenleft}{\kern0pt}simp\ add{\isacharcolon}{\kern0pt}{\isasymepsilon}{\isacharunderscore}{\kern0pt}of{\isacharunderscore}{\kern0pt}def\ case{\isacharunderscore}{\kern0pt}prod{\isacharunderscore}{\kern0pt}beta{\isacharprime}{\kern0pt}{\isacharparenright}{\kern0pt}\isanewline
\ \ \ \ \isacommand{apply}\isamarkupfalse%
\ {\isacharparenleft}{\kern0pt}subst\ eventually{\isacharunderscore}{\kern0pt}prod{\isadigit{2}}{\isacharprime}{\kern0pt}{\isacharcomma}{\kern0pt}\ simp{\isacharparenright}{\kern0pt}\isanewline
\ \ \ \ \isacommand{apply}\isamarkupfalse%
\ {\isacharparenleft}{\kern0pt}subst\ eventually{\isacharunderscore}{\kern0pt}prod{\isadigit{2}}{\isacharprime}{\kern0pt}{\isacharcomma}{\kern0pt}\ simp{\isacharparenright}{\kern0pt}\isanewline
\ \ \ \ \isacommand{apply}\isamarkupfalse%
\ {\isacharparenleft}{\kern0pt}subst\ eventually{\isacharunderscore}{\kern0pt}prod{\isadigit{1}}{\isacharprime}{\kern0pt}{\isacharcomma}{\kern0pt}\ simp{\isacharparenright}{\kern0pt}\isanewline
\ \ \ \ \isacommand{by}\isamarkupfalse%
\ {\isacharparenleft}{\kern0pt}rule\ eventually{\isacharunderscore}{\kern0pt}at{\isacharunderscore}{\kern0pt}rightI{\isacharbrackleft}{\kern0pt}\isakeyword{where}\ b{\isacharequal}{\kern0pt}{\isachardoublequoteopen}{\isadigit{1}}{\isachardoublequoteclose}{\isacharbrackright}{\kern0pt}{\isacharcomma}{\kern0pt}\ simp{\isacharcomma}{\kern0pt}\ simp{\isacharparenright}{\kern0pt}\isanewline
\isanewline
\ \ \isacommand{have}\isamarkupfalse%
\ l{\isadigit{1}}{\isacharcolon}{\kern0pt}\ {\isachardoublequoteopen}{\isacharparenleft}{\kern0pt}{\isasymlambda}x{\isachardot}{\kern0pt}\ real\ {\isacharparenleft}{\kern0pt}nat\ {\isasymlceil}{\isadigit{6}}\ {\isacharslash}{\kern0pt}\ {\isacharparenleft}{\kern0pt}{\isasymdelta}{\isacharunderscore}{\kern0pt}of\ x{\isacharparenright}{\kern0pt}\isactrlsup {\isadigit{2}}{\isasymrceil}{\isacharparenright}{\kern0pt}{\isacharparenright}{\kern0pt}\ {\isasymin}\ O{\isacharbrackleft}{\kern0pt}{\isacharquery}{\kern0pt}F{\isacharbrackright}{\kern0pt}{\isacharparenleft}{\kern0pt}{\isasymlambda}x{\isachardot}{\kern0pt}\ {\isadigit{1}}\ {\isacharslash}{\kern0pt}\ {\isacharparenleft}{\kern0pt}real{\isacharunderscore}{\kern0pt}of{\isacharunderscore}{\kern0pt}rat\ {\isacharparenleft}{\kern0pt}{\isasymdelta}{\isacharunderscore}{\kern0pt}of\ x{\isacharparenright}{\kern0pt}{\isacharparenright}{\kern0pt}\isactrlsup {\isadigit{2}}{\isacharparenright}{\kern0pt}{\isachardoublequoteclose}\isanewline
\ \ \ \ \isacommand{apply}\isamarkupfalse%
\ {\isacharparenleft}{\kern0pt}rule\ landau{\isacharunderscore}{\kern0pt}real{\isacharunderscore}{\kern0pt}nat{\isacharparenright}{\kern0pt}\isanewline
\ \ \ \ \isacommand{apply}\isamarkupfalse%
\ {\isacharparenleft}{\kern0pt}subst\ landau{\isacharunderscore}{\kern0pt}o{\isachardot}{\kern0pt}big{\isachardot}{\kern0pt}in{\isacharunderscore}{\kern0pt}cong{\isacharbrackleft}{\kern0pt}\isakeyword{where}\ g{\isacharequal}{\kern0pt}{\isachardoublequoteopen}{\isasymlambda}x{\isachardot}{\kern0pt}\ real{\isacharunderscore}{\kern0pt}of{\isacharunderscore}{\kern0pt}int\ {\isasymlceil}{\isadigit{6}}\ {\isacharslash}{\kern0pt}\ {\isacharparenleft}{\kern0pt}real{\isacharunderscore}{\kern0pt}of{\isacharunderscore}{\kern0pt}rat\ {\isacharparenleft}{\kern0pt}{\isasymdelta}{\isacharunderscore}{\kern0pt}of\ x{\isacharparenright}{\kern0pt}{\isacharparenright}{\kern0pt}\isactrlsup {\isadigit{2}}{\isasymrceil}{\isachardoublequoteclose}{\isacharbrackright}{\kern0pt}{\isacharparenright}{\kern0pt}\isanewline
\ \ \ \ \isacommand{apply}\isamarkupfalse%
\ {\isacharparenleft}{\kern0pt}rule\ always{\isacharunderscore}{\kern0pt}eventually{\isacharcomma}{\kern0pt}\ rule\ allI{\isacharcomma}{\kern0pt}\ rule\ arg{\isacharunderscore}{\kern0pt}cong{\isacharbrackleft}{\kern0pt}\isakeyword{where}\ f{\isacharequal}{\kern0pt}{\isachardoublequoteopen}real{\isacharunderscore}{\kern0pt}of{\isacharunderscore}{\kern0pt}int{\isachardoublequoteclose}{\isacharbrackright}{\kern0pt}{\isacharparenright}{\kern0pt}\ \isanewline
\ \ \ \ \ \isacommand{apply}\isamarkupfalse%
\ {\isacharparenleft}{\kern0pt}metis\ {\isacharparenleft}{\kern0pt}no{\isacharunderscore}{\kern0pt}types{\isacharcomma}{\kern0pt}\ opaque{\isacharunderscore}{\kern0pt}lifting{\isacharparenright}{\kern0pt}\ of{\isacharunderscore}{\kern0pt}rat{\isacharunderscore}{\kern0pt}ceiling\ of{\isacharunderscore}{\kern0pt}rat{\isacharunderscore}{\kern0pt}divide\ of{\isacharunderscore}{\kern0pt}rat{\isacharunderscore}{\kern0pt}numeral{\isacharunderscore}{\kern0pt}eq\ of{\isacharunderscore}{\kern0pt}rat{\isacharunderscore}{\kern0pt}power{\isacharparenright}{\kern0pt}\isanewline
\ \ \ \ \isacommand{apply}\isamarkupfalse%
\ {\isacharparenleft}{\kern0pt}rule\ landau{\isacharunderscore}{\kern0pt}ceil{\isacharbrackleft}{\kern0pt}OF\ unit{\isacharunderscore}{\kern0pt}{\isadigit{1}}{\isacharbrackright}{\kern0pt}{\isacharparenright}{\kern0pt}\isanewline
\ \ \ \ \isacommand{by}\isamarkupfalse%
\ {\isacharparenleft}{\kern0pt}rule\ landau{\isacharunderscore}{\kern0pt}const{\isacharunderscore}{\kern0pt}inv{\isacharcomma}{\kern0pt}\ simp{\isacharcomma}{\kern0pt}\ simp{\isacharparenright}{\kern0pt}\isanewline
\isanewline
\ \ \isacommand{have}\isamarkupfalse%
\ l{\isadigit{2}}{\isacharcolon}{\kern0pt}\ {\isachardoublequoteopen}{\isacharparenleft}{\kern0pt}{\isasymlambda}x{\isachardot}{\kern0pt}\ real\ {\isacharparenleft}{\kern0pt}nat\ {\isasymlceil}{\isacharminus}{\kern0pt}\ {\isacharparenleft}{\kern0pt}{\isadigit{1}}{\isadigit{8}}\ {\isacharasterisk}{\kern0pt}\ ln\ {\isacharparenleft}{\kern0pt}real{\isacharunderscore}{\kern0pt}of{\isacharunderscore}{\kern0pt}rat\ {\isacharparenleft}{\kern0pt}{\isasymepsilon}{\isacharunderscore}{\kern0pt}of\ x{\isacharparenright}{\kern0pt}{\isacharparenright}{\kern0pt}{\isacharparenright}{\kern0pt}{\isasymrceil}{\isacharparenright}{\kern0pt}{\isacharparenright}{\kern0pt}\ {\isasymin}\ O{\isacharbrackleft}{\kern0pt}{\isacharquery}{\kern0pt}F{\isacharbrackright}{\kern0pt}{\isacharparenleft}{\kern0pt}{\isasymlambda}x{\isachardot}{\kern0pt}\ ln\ {\isacharparenleft}{\kern0pt}{\isadigit{1}}\ {\isacharslash}{\kern0pt}\ real{\isacharunderscore}{\kern0pt}of{\isacharunderscore}{\kern0pt}rat\ {\isacharparenleft}{\kern0pt}{\isasymepsilon}{\isacharunderscore}{\kern0pt}of\ x{\isacharparenright}{\kern0pt}{\isacharparenright}{\kern0pt}{\isacharparenright}{\kern0pt}{\isachardoublequoteclose}\isanewline
\ \ \ \ \isacommand{apply}\isamarkupfalse%
\ {\isacharparenleft}{\kern0pt}rule\ landau{\isacharunderscore}{\kern0pt}real{\isacharunderscore}{\kern0pt}nat{\isacharcomma}{\kern0pt}\ rule\ landau{\isacharunderscore}{\kern0pt}ceil{\isacharcomma}{\kern0pt}\ simp\ add{\isacharcolon}{\kern0pt}unit{\isacharunderscore}{\kern0pt}{\isadigit{2}}{\isacharparenright}{\kern0pt}\isanewline
\ \ \ \ \isacommand{apply}\isamarkupfalse%
\ {\isacharparenleft}{\kern0pt}subst\ minus{\isacharunderscore}{\kern0pt}mult{\isacharunderscore}{\kern0pt}right{\isacharparenright}{\kern0pt}\isanewline
\ \ \ \ \isacommand{apply}\isamarkupfalse%
\ {\isacharparenleft}{\kern0pt}subst\ cmult{\isacharunderscore}{\kern0pt}in{\isacharunderscore}{\kern0pt}bigo{\isacharunderscore}{\kern0pt}iff{\isacharcomma}{\kern0pt}\ rule\ disjI{\isadigit{2}}{\isacharparenright}{\kern0pt}\isanewline
\ \ \ \ \isacommand{apply}\isamarkupfalse%
\ {\isacharparenleft}{\kern0pt}rule\ landau{\isacharunderscore}{\kern0pt}o{\isachardot}{\kern0pt}big{\isacharunderscore}{\kern0pt}mono{\isacharparenright}{\kern0pt}\isanewline
\ \ \ \ \isacommand{apply}\isamarkupfalse%
\ {\isacharparenleft}{\kern0pt}rule\ eventually{\isacharunderscore}{\kern0pt}mono{\isacharbrackleft}{\kern0pt}OF\ zero{\isacharunderscore}{\kern0pt}less{\isacharunderscore}{\kern0pt}eps{\isacharbrackright}{\kern0pt}{\isacharparenright}{\kern0pt}\isanewline
\ \ \ \ \isacommand{by}\isamarkupfalse%
\ {\isacharparenleft}{\kern0pt}subst\ ln{\isacharunderscore}{\kern0pt}div{\isacharcomma}{\kern0pt}\ simp{\isacharplus}{\kern0pt}{\isacharparenright}{\kern0pt}\isanewline
\isanewline
\ \ \isacommand{have}\isamarkupfalse%
\ l{\isadigit{3}}{\isacharcolon}{\kern0pt}\ {\isachardoublequoteopen}{\isacharparenleft}{\kern0pt}{\isasymlambda}x{\isachardot}{\kern0pt}\ log\ {\isadigit{2}}\ {\isacharparenleft}{\kern0pt}real\ {\isacharparenleft}{\kern0pt}m{\isacharunderscore}{\kern0pt}of\ x{\isacharparenright}{\kern0pt}\ {\isacharasterisk}{\kern0pt}\ {\isacharparenleft}{\kern0pt}{\isadigit{4}}\ {\isacharplus}{\kern0pt}\ {\isadigit{2}}\ {\isacharasterisk}{\kern0pt}\ real\ {\isacharparenleft}{\kern0pt}n{\isacharunderscore}{\kern0pt}of\ x{\isacharparenright}{\kern0pt}{\isacharparenright}{\kern0pt}\ {\isacharplus}{\kern0pt}\ {\isadigit{1}}{\isacharparenright}{\kern0pt}{\isacharparenright}{\kern0pt}\ {\isasymin}\ O{\isacharbrackleft}{\kern0pt}{\isacharquery}{\kern0pt}F{\isacharbrackright}{\kern0pt}{\isacharparenleft}{\kern0pt}{\isasymlambda}x{\isachardot}{\kern0pt}\ ln\ {\isacharparenleft}{\kern0pt}real\ {\isacharparenleft}{\kern0pt}n{\isacharunderscore}{\kern0pt}of\ x{\isacharparenright}{\kern0pt}{\isacharparenright}{\kern0pt}\ {\isacharplus}{\kern0pt}\ ln\ {\isacharparenleft}{\kern0pt}real\ {\isacharparenleft}{\kern0pt}m{\isacharunderscore}{\kern0pt}of\ x{\isacharparenright}{\kern0pt}{\isacharparenright}{\kern0pt}{\isacharparenright}{\kern0pt}{\isachardoublequoteclose}\isanewline
\ \ \ \ \isacommand{apply}\isamarkupfalse%
\ {\isacharparenleft}{\kern0pt}simp\ add{\isacharcolon}{\kern0pt}log{\isacharunderscore}{\kern0pt}def{\isacharparenright}{\kern0pt}\isanewline
\ \ \ \ \isacommand{apply}\isamarkupfalse%
\ {\isacharparenleft}{\kern0pt}rule\ landau{\isacharunderscore}{\kern0pt}o{\isachardot}{\kern0pt}big{\isacharunderscore}{\kern0pt}trans{\isacharbrackleft}{\kern0pt}\isakeyword{where}\ g{\isacharequal}{\kern0pt}{\isachardoublequoteopen}{\isasymlambda}x{\isachardot}{\kern0pt}\ ln\ {\isacharparenleft}{\kern0pt}real\ {\isacharparenleft}{\kern0pt}n{\isacharunderscore}{\kern0pt}of\ x{\isacharparenright}{\kern0pt}\ {\isacharasterisk}{\kern0pt}\ real\ {\isacharparenleft}{\kern0pt}m{\isacharunderscore}{\kern0pt}of\ x{\isacharparenright}{\kern0pt}{\isacharparenright}{\kern0pt}{\isachardoublequoteclose}{\isacharbrackright}{\kern0pt}{\isacharparenright}{\kern0pt}\isanewline
\ \ \ \ \ \isacommand{apply}\isamarkupfalse%
\ {\isacharparenleft}{\kern0pt}rule\ landau{\isacharunderscore}{\kern0pt}ln{\isacharunderscore}{\kern0pt}{\isadigit{2}}{\isacharbrackleft}{\kern0pt}\isakeyword{where}\ a{\isacharequal}{\kern0pt}{\isachardoublequoteopen}{\isadigit{2}}{\isachardoublequoteclose}{\isacharbrackright}{\kern0pt}{\isacharcomma}{\kern0pt}\ simp{\isacharcomma}{\kern0pt}\ simp{\isacharparenright}{\kern0pt}\isanewline
\ \ \ \ \ \ \isacommand{apply}\isamarkupfalse%
\ {\isacharparenleft}{\kern0pt}rule\ eventually{\isacharunderscore}{\kern0pt}mono{\isacharbrackleft}{\kern0pt}OF\ eventually{\isacharunderscore}{\kern0pt}conj{\isacharbrackleft}{\kern0pt}OF\ m{\isacharunderscore}{\kern0pt}inf{\isacharbrackleft}{\kern0pt}\isakeyword{where}\ c{\isacharequal}{\kern0pt}{\isachardoublequoteopen}{\isadigit{2}}{\isachardoublequoteclose}{\isacharbrackright}{\kern0pt}\ n{\isacharunderscore}{\kern0pt}inf{\isacharbrackleft}{\kern0pt}\isakeyword{where}\ c{\isacharequal}{\kern0pt}{\isachardoublequoteopen}{\isadigit{1}}{\isachardoublequoteclose}{\isacharbrackright}{\kern0pt}{\isacharbrackright}{\kern0pt}{\isacharbrackright}{\kern0pt}{\isacharparenright}{\kern0pt}\isanewline
\ \ \ \ \ \ \isacommand{apply}\isamarkupfalse%
\ {\isacharparenleft}{\kern0pt}metis\ dual{\isacharunderscore}{\kern0pt}order{\isachardot}{\kern0pt}trans\ mult{\isacharunderscore}{\kern0pt}left{\isacharunderscore}{\kern0pt}mono\ mult{\isacharunderscore}{\kern0pt}of{\isacharunderscore}{\kern0pt}nat{\isacharunderscore}{\kern0pt}commute\ of{\isacharunderscore}{\kern0pt}nat{\isacharunderscore}{\kern0pt}{\isadigit{0}}{\isacharunderscore}{\kern0pt}le{\isacharunderscore}{\kern0pt}iff\ verit{\isacharunderscore}{\kern0pt}prod{\isacharunderscore}{\kern0pt}simplify{\isacharparenleft}{\kern0pt}{\isadigit{1}}{\isacharparenright}{\kern0pt}{\isacharparenright}{\kern0pt}\isanewline
\ \ \ \ \ \isacommand{apply}\isamarkupfalse%
\ {\isacharparenleft}{\kern0pt}rule\ sum{\isacharunderscore}{\kern0pt}in{\isacharunderscore}{\kern0pt}bigo{\isacharparenright}{\kern0pt}\isanewline
\ \ \ \ \ \ \isacommand{apply}\isamarkupfalse%
\ {\isacharparenleft}{\kern0pt}subst\ mult{\isachardot}{\kern0pt}commute{\isacharparenright}{\kern0pt}\isanewline
\ \ \ \ \ \ \isacommand{apply}\isamarkupfalse%
\ {\isacharparenleft}{\kern0pt}rule\ landau{\isacharunderscore}{\kern0pt}o{\isachardot}{\kern0pt}mult{\isacharparenright}{\kern0pt}\isanewline
\ \ \ \ \ \ \isacommand{apply}\isamarkupfalse%
\ {\isacharparenleft}{\kern0pt}rule\ sum{\isacharunderscore}{\kern0pt}in{\isacharunderscore}{\kern0pt}bigo{\isacharcomma}{\kern0pt}\ simp\ add{\isacharcolon}{\kern0pt}unit{\isacharunderscore}{\kern0pt}{\isadigit{3}}{\isacharcomma}{\kern0pt}\ simp{\isacharparenright}{\kern0pt}\isanewline
\ \ \ \ \ \ \isacommand{apply}\isamarkupfalse%
\ simp\isanewline
\ \ \ \ \ \ \isacommand{apply}\isamarkupfalse%
\ {\isacharparenleft}{\kern0pt}simp\ add{\isacharcolon}{\kern0pt}unit{\isacharunderscore}{\kern0pt}{\isadigit{9}}{\isacharparenright}{\kern0pt}\isanewline
\ \ \ \ \isacommand{apply}\isamarkupfalse%
\ {\isacharparenleft}{\kern0pt}subst\ landau{\isacharunderscore}{\kern0pt}o{\isachardot}{\kern0pt}big{\isachardot}{\kern0pt}in{\isacharunderscore}{\kern0pt}cong{\isacharbrackleft}{\kern0pt}\isakeyword{where}\ g{\isacharequal}{\kern0pt}{\isachardoublequoteopen}{\isasymlambda}x{\isachardot}{\kern0pt}\ ln\ {\isacharparenleft}{\kern0pt}real\ {\isacharparenleft}{\kern0pt}n{\isacharunderscore}{\kern0pt}of\ x{\isacharparenright}{\kern0pt}{\isacharparenright}{\kern0pt}\ {\isacharplus}{\kern0pt}\ ln\ {\isacharparenleft}{\kern0pt}real\ {\isacharparenleft}{\kern0pt}m{\isacharunderscore}{\kern0pt}of\ x{\isacharparenright}{\kern0pt}{\isacharparenright}{\kern0pt}{\isachardoublequoteclose}{\isacharbrackright}{\kern0pt}{\isacharparenright}{\kern0pt}\isanewline
\ \ \ \ \ \isacommand{apply}\isamarkupfalse%
\ {\isacharparenleft}{\kern0pt}rule\ eventually{\isacharunderscore}{\kern0pt}mono{\isacharbrackleft}{\kern0pt}OF\ eventually{\isacharunderscore}{\kern0pt}conj{\isacharbrackleft}{\kern0pt}OF\ m{\isacharunderscore}{\kern0pt}inf{\isacharbrackleft}{\kern0pt}\isakeyword{where}\ c{\isacharequal}{\kern0pt}{\isachardoublequoteopen}{\isadigit{1}}{\isachardoublequoteclose}{\isacharbrackright}{\kern0pt}\ n{\isacharunderscore}{\kern0pt}inf{\isacharbrackleft}{\kern0pt}\isakeyword{where}\ c{\isacharequal}{\kern0pt}{\isachardoublequoteopen}{\isadigit{1}}{\isachardoublequoteclose}{\isacharbrackright}{\kern0pt}{\isacharbrackright}{\kern0pt}{\isacharbrackright}{\kern0pt}{\isacharparenright}{\kern0pt}\isanewline
\ \ \ \ \isacommand{by}\isamarkupfalse%
\ {\isacharparenleft}{\kern0pt}subst\ ln{\isacharunderscore}{\kern0pt}mult{\isacharcomma}{\kern0pt}\ simp{\isacharplus}{\kern0pt}{\isacharparenright}{\kern0pt}\isanewline
\isanewline
\ \ \isacommand{have}\isamarkupfalse%
\ l{\isadigit{4}}{\isacharcolon}{\kern0pt}\ {\isachardoublequoteopen}{\isacharparenleft}{\kern0pt}{\isasymlambda}x{\isachardot}{\kern0pt}\ log\ {\isadigit{2}}\ {\isacharparenleft}{\kern0pt}{\isadigit{4}}\ {\isacharplus}{\kern0pt}\ {\isadigit{2}}\ {\isacharasterisk}{\kern0pt}\ real\ {\isacharparenleft}{\kern0pt}n{\isacharunderscore}{\kern0pt}of\ x{\isacharparenright}{\kern0pt}{\isacharparenright}{\kern0pt}{\isacharparenright}{\kern0pt}\ {\isasymin}\ O{\isacharbrackleft}{\kern0pt}{\isacharquery}{\kern0pt}F{\isacharbrackright}{\kern0pt}{\isacharparenleft}{\kern0pt}{\isasymlambda}x{\isachardot}{\kern0pt}\ ln\ {\isacharparenleft}{\kern0pt}real\ {\isacharparenleft}{\kern0pt}n{\isacharunderscore}{\kern0pt}of\ x{\isacharparenright}{\kern0pt}{\isacharparenright}{\kern0pt}\ {\isacharplus}{\kern0pt}\ ln\ {\isacharparenleft}{\kern0pt}real\ {\isacharparenleft}{\kern0pt}m{\isacharunderscore}{\kern0pt}of\ x{\isacharparenright}{\kern0pt}{\isacharparenright}{\kern0pt}{\isacharparenright}{\kern0pt}{\isachardoublequoteclose}\isanewline
\ \ \ \ \isacommand{apply}\isamarkupfalse%
\ {\isacharparenleft}{\kern0pt}rule\ landau{\isacharunderscore}{\kern0pt}sum{\isacharunderscore}{\kern0pt}{\isadigit{1}}{\isacharparenright}{\kern0pt}\isanewline
\ \ \ \ \ \ \isacommand{apply}\isamarkupfalse%
\ {\isacharparenleft}{\kern0pt}rule\ eventually{\isacharunderscore}{\kern0pt}ln{\isacharunderscore}{\kern0pt}ge{\isacharunderscore}{\kern0pt}iff{\isacharbrackleft}{\kern0pt}OF\ n{\isacharunderscore}{\kern0pt}inf{\isacharbrackright}{\kern0pt}{\isacharparenright}{\kern0pt}\isanewline
\ \ \ \ \ \isacommand{apply}\isamarkupfalse%
\ {\isacharparenleft}{\kern0pt}rule\ eventually{\isacharunderscore}{\kern0pt}ln{\isacharunderscore}{\kern0pt}ge{\isacharunderscore}{\kern0pt}iff{\isacharbrackleft}{\kern0pt}OF\ m{\isacharunderscore}{\kern0pt}inf{\isacharbrackright}{\kern0pt}{\isacharparenright}{\kern0pt}\isanewline
\ \ \ \ \isacommand{apply}\isamarkupfalse%
\ {\isacharparenleft}{\kern0pt}simp\ add{\isacharcolon}{\kern0pt}log{\isacharunderscore}{\kern0pt}def{\isacharparenright}{\kern0pt}\isanewline
\ \ \ \ \isacommand{apply}\isamarkupfalse%
\ {\isacharparenleft}{\kern0pt}rule\ landau{\isacharunderscore}{\kern0pt}ln{\isacharunderscore}{\kern0pt}{\isadigit{2}}{\isacharbrackleft}{\kern0pt}\isakeyword{where}\ a{\isacharequal}{\kern0pt}{\isachardoublequoteopen}{\isadigit{2}}{\isachardoublequoteclose}{\isacharbrackright}{\kern0pt}{\isacharcomma}{\kern0pt}\ simp{\isacharcomma}{\kern0pt}\ simp{\isacharcomma}{\kern0pt}\ rule\ n{\isacharunderscore}{\kern0pt}inf{\isacharparenright}{\kern0pt}\isanewline
\ \ \ \ \isacommand{apply}\isamarkupfalse%
\ {\isacharparenleft}{\kern0pt}rule\ sum{\isacharunderscore}{\kern0pt}in{\isacharunderscore}{\kern0pt}bigo{\isacharcomma}{\kern0pt}\ simp{\isacharcomma}{\kern0pt}\ simp\ add{\isacharcolon}{\kern0pt}unit{\isacharunderscore}{\kern0pt}{\isadigit{3}}{\isacharparenright}{\kern0pt}\isanewline
\ \ \ \ \isacommand{by}\isamarkupfalse%
\ simp\isanewline
\isanewline
\ \ \isacommand{have}\isamarkupfalse%
\ l{\isadigit{5}}{\isacharcolon}{\kern0pt}\ {\isachardoublequoteopen}{\isacharparenleft}{\kern0pt}{\isasymlambda}x{\isachardot}{\kern0pt}\ ln\ {\isacharparenleft}{\kern0pt}real\ {\isacharparenleft}{\kern0pt}nat\ {\isasymlceil}{\isadigit{6}}\ {\isacharslash}{\kern0pt}\ {\isacharparenleft}{\kern0pt}{\isasymdelta}{\isacharunderscore}{\kern0pt}of\ x{\isacharparenright}{\kern0pt}\isactrlsup {\isadigit{2}}{\isasymrceil}{\isacharparenright}{\kern0pt}\ {\isacharplus}{\kern0pt}\ {\isadigit{1}}{\isacharparenright}{\kern0pt}{\isacharparenright}{\kern0pt}\ {\isasymin}\ O{\isacharbrackleft}{\kern0pt}{\isacharquery}{\kern0pt}F{\isacharbrackright}{\kern0pt}{\isacharparenleft}{\kern0pt}{\isasymlambda}x{\isachardot}{\kern0pt}\ ln\ {\isacharparenleft}{\kern0pt}{\isadigit{1}}\ {\isacharslash}{\kern0pt}\ real{\isacharunderscore}{\kern0pt}of{\isacharunderscore}{\kern0pt}rat\ {\isacharparenleft}{\kern0pt}{\isasymepsilon}{\isacharunderscore}{\kern0pt}of\ x{\isacharparenright}{\kern0pt}{\isacharparenright}{\kern0pt}\ {\isacharasterisk}{\kern0pt}\isanewline
\ \ \ \ {\isacharparenleft}{\kern0pt}ln\ {\isacharparenleft}{\kern0pt}real\ {\isacharparenleft}{\kern0pt}n{\isacharunderscore}{\kern0pt}of\ x{\isacharparenright}{\kern0pt}{\isacharparenright}{\kern0pt}\ {\isacharplus}{\kern0pt}\ ln\ {\isacharparenleft}{\kern0pt}real\ {\isacharparenleft}{\kern0pt}m{\isacharunderscore}{\kern0pt}of\ x{\isacharparenright}{\kern0pt}{\isacharparenright}{\kern0pt}{\isacharparenright}{\kern0pt}\ {\isacharslash}{\kern0pt}\ {\isacharparenleft}{\kern0pt}real{\isacharunderscore}{\kern0pt}of{\isacharunderscore}{\kern0pt}rat\ {\isacharparenleft}{\kern0pt}{\isasymdelta}{\isacharunderscore}{\kern0pt}of\ x{\isacharparenright}{\kern0pt}{\isacharparenright}{\kern0pt}\isactrlsup {\isadigit{2}}{\isacharparenright}{\kern0pt}{\isachardoublequoteclose}\isanewline
\ \ \ \ \isacommand{apply}\isamarkupfalse%
\ {\isacharparenleft}{\kern0pt}subst\ {\isacharparenleft}{\kern0pt}{\isadigit{2}}{\isacharparenright}{\kern0pt}\ div{\isacharunderscore}{\kern0pt}commute{\isacharparenright}{\kern0pt}\isanewline
\ \ \ \ \isacommand{apply}\isamarkupfalse%
\ {\isacharparenleft}{\kern0pt}rule\ landau{\isacharunderscore}{\kern0pt}o{\isachardot}{\kern0pt}big{\isacharunderscore}{\kern0pt}mult{\isacharunderscore}{\kern0pt}{\isadigit{1}}{\isacharparenright}{\kern0pt}\isanewline
\ \ \ \ \ \isacommand{apply}\isamarkupfalse%
\ {\isacharparenleft}{\kern0pt}rule\ landau{\isacharunderscore}{\kern0pt}ln{\isacharunderscore}{\kern0pt}{\isadigit{3}}{\isacharcomma}{\kern0pt}\ simp{\isacharparenright}{\kern0pt}\isanewline
\ \ \ \ \ \isacommand{apply}\isamarkupfalse%
\ {\isacharparenleft}{\kern0pt}rule\ sum{\isacharunderscore}{\kern0pt}in{\isacharunderscore}{\kern0pt}bigo{\isacharcomma}{\kern0pt}\ rule\ l{\isadigit{1}}{\isacharcomma}{\kern0pt}\ rule\ unit{\isacharunderscore}{\kern0pt}{\isadigit{1}}{\isacharparenright}{\kern0pt}\isanewline
\ \ \ \ \isacommand{by}\isamarkupfalse%
\ {\isacharparenleft}{\kern0pt}rule\ landau{\isacharunderscore}{\kern0pt}o{\isachardot}{\kern0pt}big{\isacharunderscore}{\kern0pt}mult{\isacharunderscore}{\kern0pt}{\isadigit{1}}{\isacharbrackleft}{\kern0pt}OF\ unit{\isacharunderscore}{\kern0pt}{\isadigit{2}}\ unit{\isacharunderscore}{\kern0pt}{\isadigit{6}}{\isacharbrackright}{\kern0pt}{\isacharparenright}{\kern0pt}\isanewline
\isanewline
\ \ \isacommand{have}\isamarkupfalse%
\ l{\isadigit{6}}{\isacharcolon}{\kern0pt}\ {\isachardoublequoteopen}{\isacharparenleft}{\kern0pt}{\isasymlambda}x{\isachardot}{\kern0pt}\ ln\ {\isacharparenleft}{\kern0pt}{\isadigit{4}}\ {\isacharplus}{\kern0pt}\ {\isadigit{2}}\ {\isacharasterisk}{\kern0pt}\ real\ {\isacharparenleft}{\kern0pt}n{\isacharunderscore}{\kern0pt}of\ x{\isacharparenright}{\kern0pt}{\isacharparenright}{\kern0pt}{\isacharparenright}{\kern0pt}\ {\isasymin}\ O{\isacharbrackleft}{\kern0pt}{\isacharquery}{\kern0pt}F{\isacharbrackright}{\kern0pt}{\isacharparenleft}{\kern0pt}{\isasymlambda}x{\isachardot}{\kern0pt}\ ln\ {\isacharparenleft}{\kern0pt}{\isadigit{1}}\ {\isacharslash}{\kern0pt}\ real{\isacharunderscore}{\kern0pt}of{\isacharunderscore}{\kern0pt}rat\ {\isacharparenleft}{\kern0pt}{\isasymepsilon}{\isacharunderscore}{\kern0pt}of\ x{\isacharparenright}{\kern0pt}{\isacharparenright}{\kern0pt}\ {\isacharasterisk}{\kern0pt}\ \isanewline
\ \ \ \ {\isacharparenleft}{\kern0pt}ln\ {\isacharparenleft}{\kern0pt}real\ {\isacharparenleft}{\kern0pt}n{\isacharunderscore}{\kern0pt}of\ x{\isacharparenright}{\kern0pt}{\isacharparenright}{\kern0pt}\ {\isacharplus}{\kern0pt}\ ln\ {\isacharparenleft}{\kern0pt}real\ {\isacharparenleft}{\kern0pt}m{\isacharunderscore}{\kern0pt}of\ x{\isacharparenright}{\kern0pt}{\isacharparenright}{\kern0pt}{\isacharparenright}{\kern0pt}\ {\isacharslash}{\kern0pt}\ {\isacharparenleft}{\kern0pt}real{\isacharunderscore}{\kern0pt}of{\isacharunderscore}{\kern0pt}rat\ {\isacharparenleft}{\kern0pt}{\isasymdelta}{\isacharunderscore}{\kern0pt}of\ x{\isacharparenright}{\kern0pt}{\isacharparenright}{\kern0pt}\isactrlsup {\isadigit{2}}{\isacharparenright}{\kern0pt}{\isachardoublequoteclose}\isanewline
\ \ \ \ \isacommand{apply}\isamarkupfalse%
\ {\isacharparenleft}{\kern0pt}subst\ {\isacharparenleft}{\kern0pt}{\isadigit{2}}{\isacharparenright}{\kern0pt}\ div{\isacharunderscore}{\kern0pt}commute{\isacharparenright}{\kern0pt}\isanewline
\ \ \ \ \isacommand{apply}\isamarkupfalse%
\ {\isacharparenleft}{\kern0pt}rule\ landau{\isacharunderscore}{\kern0pt}o{\isachardot}{\kern0pt}big{\isacharunderscore}{\kern0pt}mult{\isacharunderscore}{\kern0pt}{\isadigit{1}}{\isacharprime}{\kern0pt}{\isacharbrackleft}{\kern0pt}OF\ unit{\isacharunderscore}{\kern0pt}{\isadigit{1}}{\isacharbrackright}{\kern0pt}{\isacharparenright}{\kern0pt}\isanewline
\ \ \ \ \isacommand{apply}\isamarkupfalse%
\ {\isacharparenleft}{\kern0pt}rule\ landau{\isacharunderscore}{\kern0pt}o{\isachardot}{\kern0pt}big{\isacharunderscore}{\kern0pt}mult{\isacharunderscore}{\kern0pt}{\isadigit{1}}{\isacharprime}{\kern0pt}{\isacharbrackleft}{\kern0pt}OF\ unit{\isacharunderscore}{\kern0pt}{\isadigit{2}}{\isacharbrackright}{\kern0pt}{\isacharparenright}{\kern0pt}\isanewline
\ \ \ \ \isacommand{using}\isamarkupfalse%
\ l{\isadigit{4}}\ \isacommand{by}\isamarkupfalse%
\ {\isacharparenleft}{\kern0pt}simp\ add{\isacharcolon}{\kern0pt}log{\isacharunderscore}{\kern0pt}def{\isacharparenright}{\kern0pt}\isanewline
\isanewline
\ \ \isacommand{have}\isamarkupfalse%
\ l{\isadigit{7}}{\isacharcolon}{\kern0pt}\ {\isachardoublequoteopen}{\isacharparenleft}{\kern0pt}{\isasymlambda}x{\isachardot}{\kern0pt}\ ln\ {\isacharparenleft}{\kern0pt}real\ {\isacharparenleft}{\kern0pt}nat\ {\isasymlceil}{\isacharminus}{\kern0pt}\ {\isacharparenleft}{\kern0pt}{\isadigit{1}}{\isadigit{8}}\ {\isacharasterisk}{\kern0pt}\ ln\ {\isacharparenleft}{\kern0pt}real{\isacharunderscore}{\kern0pt}of{\isacharunderscore}{\kern0pt}rat\ {\isacharparenleft}{\kern0pt}{\isasymepsilon}{\isacharunderscore}{\kern0pt}of\ x{\isacharparenright}{\kern0pt}{\isacharparenright}{\kern0pt}{\isacharparenright}{\kern0pt}{\isasymrceil}{\isacharparenright}{\kern0pt}\ {\isacharplus}{\kern0pt}\ {\isadigit{1}}{\isacharparenright}{\kern0pt}{\isacharparenright}{\kern0pt}\ {\isasymin}\ O{\isacharbrackleft}{\kern0pt}{\isacharquery}{\kern0pt}F{\isacharbrackright}{\kern0pt}{\isacharparenleft}{\kern0pt}{\isasymlambda}x{\isachardot}{\kern0pt}\ \isanewline
\ \ \ \ ln\ {\isacharparenleft}{\kern0pt}{\isadigit{1}}\ {\isacharslash}{\kern0pt}\ real{\isacharunderscore}{\kern0pt}of{\isacharunderscore}{\kern0pt}rat\ {\isacharparenleft}{\kern0pt}{\isasymepsilon}{\isacharunderscore}{\kern0pt}of\ x{\isacharparenright}{\kern0pt}{\isacharparenright}{\kern0pt}\ {\isacharasterisk}{\kern0pt}\ {\isacharparenleft}{\kern0pt}ln\ {\isacharparenleft}{\kern0pt}real\ {\isacharparenleft}{\kern0pt}n{\isacharunderscore}{\kern0pt}of\ x{\isacharparenright}{\kern0pt}{\isacharparenright}{\kern0pt}\ {\isacharplus}{\kern0pt}\ ln\ {\isacharparenleft}{\kern0pt}real\ {\isacharparenleft}{\kern0pt}m{\isacharunderscore}{\kern0pt}of\ x{\isacharparenright}{\kern0pt}{\isacharparenright}{\kern0pt}{\isacharparenright}{\kern0pt}\ {\isacharslash}{\kern0pt}\ {\isacharparenleft}{\kern0pt}real{\isacharunderscore}{\kern0pt}of{\isacharunderscore}{\kern0pt}rat\ {\isacharparenleft}{\kern0pt}{\isasymdelta}{\isacharunderscore}{\kern0pt}of\ x{\isacharparenright}{\kern0pt}{\isacharparenright}{\kern0pt}\isactrlsup {\isadigit{2}}{\isacharparenright}{\kern0pt}{\isachardoublequoteclose}\isanewline
\ \ \ \ \isacommand{apply}\isamarkupfalse%
\ {\isacharparenleft}{\kern0pt}subst\ {\isacharparenleft}{\kern0pt}{\isadigit{2}}{\isacharparenright}{\kern0pt}\ div{\isacharunderscore}{\kern0pt}commute{\isacharparenright}{\kern0pt}\isanewline
\ \ \ \ \isacommand{apply}\isamarkupfalse%
\ {\isacharparenleft}{\kern0pt}rule\ landau{\isacharunderscore}{\kern0pt}o{\isachardot}{\kern0pt}big{\isacharunderscore}{\kern0pt}mult{\isacharunderscore}{\kern0pt}{\isadigit{1}}{\isacharprime}{\kern0pt}{\isacharbrackleft}{\kern0pt}OF\ unit{\isacharunderscore}{\kern0pt}{\isadigit{1}}{\isacharbrackright}{\kern0pt}{\isacharparenright}{\kern0pt}\isanewline
\ \ \ \ \isacommand{apply}\isamarkupfalse%
\ {\isacharparenleft}{\kern0pt}rule\ landau{\isacharunderscore}{\kern0pt}o{\isachardot}{\kern0pt}big{\isacharunderscore}{\kern0pt}mult{\isacharunderscore}{\kern0pt}{\isadigit{1}}{\isacharparenright}{\kern0pt}\isanewline
\ \ \ \ \ \isacommand{apply}\isamarkupfalse%
\ {\isacharparenleft}{\kern0pt}rule\ landau{\isacharunderscore}{\kern0pt}ln{\isacharunderscore}{\kern0pt}{\isadigit{2}}{\isacharbrackleft}{\kern0pt}\isakeyword{where}\ a{\isacharequal}{\kern0pt}{\isachardoublequoteopen}{\isadigit{2}}{\isachardoublequoteclose}{\isacharbrackright}{\kern0pt}{\isacharcomma}{\kern0pt}\ simp{\isacharcomma}{\kern0pt}\ simp{\isacharcomma}{\kern0pt}\ simp\ add{\isacharcolon}{\kern0pt}eps{\isacharunderscore}{\kern0pt}inf{\isacharparenright}{\kern0pt}\isanewline
\ \ \ \ \ \isacommand{apply}\isamarkupfalse%
\ {\isacharparenleft}{\kern0pt}rule\ sum{\isacharunderscore}{\kern0pt}in{\isacharunderscore}{\kern0pt}bigo{\isacharparenright}{\kern0pt}\isanewline
\ \ \ \ \ \ \isacommand{apply}\isamarkupfalse%
\ {\isacharparenleft}{\kern0pt}rule\ landau{\isacharunderscore}{\kern0pt}nat{\isacharunderscore}{\kern0pt}ceil{\isacharbrackleft}{\kern0pt}OF\ unit{\isacharunderscore}{\kern0pt}{\isadigit{7}}{\isacharbrackright}{\kern0pt}{\isacharparenright}{\kern0pt}\isanewline
\ \ \ \ \isacommand{apply}\isamarkupfalse%
\ {\isacharparenleft}{\kern0pt}subst\ minus{\isacharunderscore}{\kern0pt}mult{\isacharunderscore}{\kern0pt}right{\isacharparenright}{\kern0pt}\isanewline
\ \ \ \ \ \ \isacommand{apply}\isamarkupfalse%
\ {\isacharparenleft}{\kern0pt}subst\ cmult{\isacharunderscore}{\kern0pt}in{\isacharunderscore}{\kern0pt}bigo{\isacharunderscore}{\kern0pt}iff{\isacharcomma}{\kern0pt}\ rule\ disjI{\isadigit{2}}{\isacharparenright}{\kern0pt}\isanewline
\ \ \ \ \ \ \isacommand{apply}\isamarkupfalse%
\ {\isacharparenleft}{\kern0pt}subst\ landau{\isacharunderscore}{\kern0pt}o{\isachardot}{\kern0pt}big{\isachardot}{\kern0pt}in{\isacharunderscore}{\kern0pt}cong{\isacharbrackleft}{\kern0pt}\isakeyword{where}\ g{\isacharequal}{\kern0pt}{\isachardoublequoteopen}{\isasymlambda}x{\isachardot}{\kern0pt}\ ln{\isacharparenleft}{\kern0pt}\ {\isadigit{1}}\ {\isacharslash}{\kern0pt}\ {\isacharparenleft}{\kern0pt}real{\isacharunderscore}{\kern0pt}of{\isacharunderscore}{\kern0pt}rat\ {\isacharparenleft}{\kern0pt}{\isasymepsilon}{\isacharunderscore}{\kern0pt}of\ x{\isacharparenright}{\kern0pt}{\isacharparenright}{\kern0pt}{\isacharparenright}{\kern0pt}{\isachardoublequoteclose}{\isacharbrackright}{\kern0pt}{\isacharparenright}{\kern0pt}\isanewline
\ \ \ \ \ \ \ \isacommand{apply}\isamarkupfalse%
\ {\isacharparenleft}{\kern0pt}rule\ eventually{\isacharunderscore}{\kern0pt}mono{\isacharbrackleft}{\kern0pt}OF\ zero{\isacharunderscore}{\kern0pt}less{\isacharunderscore}{\kern0pt}eps{\isacharbrackright}{\kern0pt}{\isacharparenright}{\kern0pt}\isanewline
\ \ \ \ \ \ \ \isacommand{apply}\isamarkupfalse%
\ {\isacharparenleft}{\kern0pt}subst\ ln{\isacharunderscore}{\kern0pt}div{\isacharcomma}{\kern0pt}\ simp{\isacharcomma}{\kern0pt}\ simp{\isacharcomma}{\kern0pt}\ simp{\isacharparenright}{\kern0pt}\isanewline
\ \ \ \ \ \ \isacommand{apply}\isamarkupfalse%
\ {\isacharparenleft}{\kern0pt}rule\ landau{\isacharunderscore}{\kern0pt}ln{\isacharunderscore}{\kern0pt}{\isadigit{3}}{\isacharbrackleft}{\kern0pt}OF\ eps{\isacharunderscore}{\kern0pt}inf{\isacharbrackright}{\kern0pt}{\isacharcomma}{\kern0pt}\ simp{\isacharparenright}{\kern0pt}\isanewline
\ \ \ \ \isacommand{apply}\isamarkupfalse%
\ {\isacharparenleft}{\kern0pt}rule\ \ unit{\isacharunderscore}{\kern0pt}{\isadigit{7}}{\isacharparenright}{\kern0pt}\isanewline
\ \ \ \ \isacommand{by}\isamarkupfalse%
\ {\isacharparenleft}{\kern0pt}rule\ unit{\isacharunderscore}{\kern0pt}{\isadigit{6}}{\isacharparenright}{\kern0pt}\isanewline
\isanewline
\ \ \isacommand{have}\isamarkupfalse%
\ {\isachardoublequoteopen}f{\isadigit{2}}{\isacharunderscore}{\kern0pt}space{\isacharunderscore}{\kern0pt}usage\ {\isacharequal}{\kern0pt}\ {\isacharparenleft}{\kern0pt}{\isasymlambda}x{\isachardot}{\kern0pt}\ f{\isadigit{2}}{\isacharunderscore}{\kern0pt}space{\isacharunderscore}{\kern0pt}usage\ {\isacharparenleft}{\kern0pt}n{\isacharunderscore}{\kern0pt}of\ x{\isacharcomma}{\kern0pt}\ m{\isacharunderscore}{\kern0pt}of\ x{\isacharcomma}{\kern0pt}\ {\isasymepsilon}{\isacharunderscore}{\kern0pt}of\ x{\isacharcomma}{\kern0pt}\ {\isasymdelta}{\isacharunderscore}{\kern0pt}of\ x{\isacharparenright}{\kern0pt}{\isacharparenright}{\kern0pt}{\isachardoublequoteclose}\isanewline
\ \ \ \ \isacommand{apply}\isamarkupfalse%
\ {\isacharparenleft}{\kern0pt}rule\ ext{\isacharparenright}{\kern0pt}\isanewline
\ \ \ \ \isacommand{by}\isamarkupfalse%
\ {\isacharparenleft}{\kern0pt}simp\ add{\isacharcolon}{\kern0pt}case{\isacharunderscore}{\kern0pt}prod{\isacharunderscore}{\kern0pt}beta{\isacharprime}{\kern0pt}\ n{\isacharunderscore}{\kern0pt}of{\isacharunderscore}{\kern0pt}def\ {\isasymepsilon}{\isacharunderscore}{\kern0pt}of{\isacharunderscore}{\kern0pt}def\ {\isasymdelta}{\isacharunderscore}{\kern0pt}of{\isacharunderscore}{\kern0pt}def\ m{\isacharunderscore}{\kern0pt}of{\isacharunderscore}{\kern0pt}def{\isacharparenright}{\kern0pt}\isanewline
\ \ \isacommand{also}\isamarkupfalse%
\ \isacommand{have}\isamarkupfalse%
\ {\isachardoublequoteopen}{\isachardot}{\kern0pt}{\isachardot}{\kern0pt}{\isachardot}{\kern0pt}\ {\isasymin}\ O{\isacharbrackleft}{\kern0pt}{\isacharquery}{\kern0pt}F{\isacharbrackright}{\kern0pt}{\isacharparenleft}{\kern0pt}g{\isacharparenright}{\kern0pt}{\isachardoublequoteclose}\isanewline
\ \ \ \ \isacommand{apply}\isamarkupfalse%
\ {\isacharparenleft}{\kern0pt}simp\ add{\isacharcolon}{\kern0pt}g{\isacharunderscore}{\kern0pt}def\ Let{\isacharunderscore}{\kern0pt}def{\isacharparenright}{\kern0pt}\isanewline
\ \ \ \ \isacommand{apply}\isamarkupfalse%
\ {\isacharparenleft}{\kern0pt}rule\ sum{\isacharunderscore}{\kern0pt}in{\isacharunderscore}{\kern0pt}bigo{\isacharunderscore}{\kern0pt}r{\isacharparenright}{\kern0pt}\isanewline
\ \ \ \ \ \isacommand{apply}\isamarkupfalse%
\ {\isacharparenleft}{\kern0pt}subst\ {\isacharparenleft}{\kern0pt}{\isadigit{2}}{\isacharparenright}{\kern0pt}\ div{\isacharunderscore}{\kern0pt}commute{\isacharcomma}{\kern0pt}\ subst\ mult{\isachardot}{\kern0pt}assoc{\isacharparenright}{\kern0pt}\isanewline
\ \ \ \ \ \isacommand{apply}\isamarkupfalse%
\ {\isacharparenleft}{\kern0pt}rule\ landau{\isacharunderscore}{\kern0pt}o{\isachardot}{\kern0pt}mult{\isacharcomma}{\kern0pt}\ simp\ add{\isacharcolon}{\kern0pt}l{\isadigit{1}}{\isacharparenright}{\kern0pt}\isanewline
\ \ \ \ \ \isacommand{apply}\isamarkupfalse%
\ {\isacharparenleft}{\kern0pt}rule\ landau{\isacharunderscore}{\kern0pt}o{\isachardot}{\kern0pt}mult{\isacharcomma}{\kern0pt}\ simp\ add{\isacharcolon}{\kern0pt}l{\isadigit{2}}{\isacharparenright}{\kern0pt}\isanewline
\ \ \ \ \ \isacommand{apply}\isamarkupfalse%
\ {\isacharparenleft}{\kern0pt}rule\ sum{\isacharunderscore}{\kern0pt}in{\isacharunderscore}{\kern0pt}bigo{\isacharunderscore}{\kern0pt}r{\isacharcomma}{\kern0pt}\ simp\ add{\isacharcolon}{\kern0pt}l{\isadigit{3}}{\isacharparenright}{\kern0pt}\isanewline
\ \ \ \ \ \isacommand{apply}\isamarkupfalse%
\ {\isacharparenleft}{\kern0pt}rule\ sum{\isacharunderscore}{\kern0pt}in{\isacharunderscore}{\kern0pt}bigo{\isacharunderscore}{\kern0pt}r{\isacharcomma}{\kern0pt}\ simp\ add{\isacharcolon}{\kern0pt}l{\isadigit{4}}{\isacharcomma}{\kern0pt}\ simp\ add{\isacharcolon}{\kern0pt}unit{\isacharunderscore}{\kern0pt}{\isadigit{6}}{\isacharparenright}{\kern0pt}\isanewline
\ \ \ \ \isacommand{apply}\isamarkupfalse%
\ {\isacharparenleft}{\kern0pt}rule\ sum{\isacharunderscore}{\kern0pt}in{\isacharunderscore}{\kern0pt}bigo{\isacharunderscore}{\kern0pt}r{\isacharcomma}{\kern0pt}\ simp\ add{\isacharcolon}{\kern0pt}log{\isacharunderscore}{\kern0pt}def\ l{\isadigit{6}}{\isacharparenright}{\kern0pt}\isanewline
\ \ \ \ \isacommand{apply}\isamarkupfalse%
\ {\isacharparenleft}{\kern0pt}rule\ sum{\isacharunderscore}{\kern0pt}in{\isacharunderscore}{\kern0pt}bigo{\isacharunderscore}{\kern0pt}r{\isacharcomma}{\kern0pt}\ simp\ add{\isacharcolon}{\kern0pt}log{\isacharunderscore}{\kern0pt}def\ l{\isadigit{7}}{\isacharparenright}{\kern0pt}\isanewline
\ \ \ \ \isacommand{apply}\isamarkupfalse%
\ {\isacharparenleft}{\kern0pt}rule\ sum{\isacharunderscore}{\kern0pt}in{\isacharunderscore}{\kern0pt}bigo{\isacharunderscore}{\kern0pt}r{\isacharcomma}{\kern0pt}\ simp\ add{\isacharcolon}{\kern0pt}log{\isacharunderscore}{\kern0pt}def\ l{\isadigit{5}}{\isacharparenright}{\kern0pt}\isanewline
\ \ \ \ \isacommand{by}\isamarkupfalse%
\ {\isacharparenleft}{\kern0pt}simp\ add{\isacharcolon}{\kern0pt}unit{\isacharunderscore}{\kern0pt}{\isadigit{8}}{\isacharparenright}{\kern0pt}\isanewline
\ \ \isacommand{also}\isamarkupfalse%
\ \isacommand{have}\isamarkupfalse%
\ {\isachardoublequoteopen}{\isachardot}{\kern0pt}{\isachardot}{\kern0pt}{\isachardot}{\kern0pt}\ {\isacharequal}{\kern0pt}\ O{\isacharbrackleft}{\kern0pt}{\isacharquery}{\kern0pt}F{\isacharbrackright}{\kern0pt}{\isacharparenleft}{\kern0pt}{\isacharquery}{\kern0pt}rhs{\isacharparenright}{\kern0pt}{\isachardoublequoteclose}\isanewline
\ \ \ \ \isacommand{apply}\isamarkupfalse%
\ {\isacharparenleft}{\kern0pt}rule\ arg{\isacharunderscore}{\kern0pt}cong{\isadigit{2}}{\isacharbrackleft}{\kern0pt}\isakeyword{where}\ f{\isacharequal}{\kern0pt}{\isachardoublequoteopen}bigo{\isachardoublequoteclose}{\isacharbrackright}{\kern0pt}{\isacharcomma}{\kern0pt}\ simp{\isacharparenright}{\kern0pt}\isanewline
\ \ \ \ \isacommand{apply}\isamarkupfalse%
\ {\isacharparenleft}{\kern0pt}rule\ ext{\isacharparenright}{\kern0pt}\isanewline
\ \ \ \ \isacommand{by}\isamarkupfalse%
\ {\isacharparenleft}{\kern0pt}simp\ add{\isacharcolon}{\kern0pt}case{\isacharunderscore}{\kern0pt}prod{\isacharunderscore}{\kern0pt}beta{\isacharprime}{\kern0pt}\ g{\isacharunderscore}{\kern0pt}def\ n{\isacharunderscore}{\kern0pt}of{\isacharunderscore}{\kern0pt}def\ {\isasymepsilon}{\isacharunderscore}{\kern0pt}of{\isacharunderscore}{\kern0pt}def\ {\isasymdelta}{\isacharunderscore}{\kern0pt}of{\isacharunderscore}{\kern0pt}def\ m{\isacharunderscore}{\kern0pt}of{\isacharunderscore}{\kern0pt}def{\isacharparenright}{\kern0pt}\isanewline
\ \ \isacommand{finally}\isamarkupfalse%
\ \isacommand{show}\isamarkupfalse%
\ {\isacharquery}{\kern0pt}thesis\ \isacommand{by}\isamarkupfalse%
\ simp\isanewline
\isacommand{qed}\isamarkupfalse%
%
\endisatagproof
{\isafoldproof}%
%
\isadelimproof
\isanewline
%
\endisadelimproof
%
\isadelimtheory
\isanewline
%
\endisadelimtheory
%
\isatagtheory
\isacommand{end}\isamarkupfalse%
%
\endisatagtheory
{\isafoldtheory}%
%
\isadelimtheory
%
\endisadelimtheory
%
\end{isabellebody}%
\endinput
%:%file=Frequency_Moment_2.tex%:%
%:%11=1%:%
%:%27=3%:%
%:%28=3%:%
%:%29=4%:%
%:%30=5%:%
%:%31=6%:%
%:%40=8%:%
%:%41=9%:%
%:%42=10%:%
%:%43=11%:%
%:%45=13%:%
%:%46=13%:%
%:%47=14%:%
%:%48=15%:%
%:%49=16%:%
%:%50=16%:%
%:%51=17%:%
%:%52=18%:%
%:%53=18%:%
%:%54=19%:%
%:%61=26%:%
%:%62=27%:%
%:%63=28%:%
%:%64=28%:%
%:%65=29%:%
%:%66=30%:%
%:%67=31%:%
%:%68=32%:%
%:%69=32%:%
%:%70=33%:%
%:%73=36%:%
%:%74=37%:%
%:%75=38%:%
%:%76=38%:%
%:%77=39%:%
%:%78=40%:%
%:%79=41%:%
%:%80=42%:%
%:%81=43%:%
%:%83=45%:%
%:%90=46%:%
%:%91=46%:%
%:%92=47%:%
%:%93=47%:%
%:%94=47%:%
%:%95=47%:%
%:%96=48%:%
%:%97=49%:%
%:%98=49%:%
%:%99=49%:%
%:%100=49%:%
%:%101=50%:%
%:%102=50%:%
%:%103=50%:%
%:%104=51%:%
%:%105=51%:%
%:%106=51%:%
%:%107=52%:%
%:%108=53%:%
%:%109=53%:%
%:%110=54%:%
%:%111=55%:%
%:%112=55%:%
%:%113=56%:%
%:%114=56%:%
%:%115=57%:%
%:%116=57%:%
%:%117=58%:%
%:%118=59%:%
%:%119=59%:%
%:%120=59%:%
%:%121=60%:%
%:%122=61%:%
%:%123=61%:%
%:%124=62%:%
%:%125=62%:%
%:%126=63%:%
%:%127=63%:%
%:%128=64%:%
%:%129=64%:%
%:%130=65%:%
%:%131=65%:%
%:%132=66%:%
%:%133=66%:%
%:%134=66%:%
%:%135=67%:%
%:%136=67%:%
%:%137=68%:%
%:%138=68%:%
%:%139=68%:%
%:%140=69%:%
%:%141=69%:%
%:%142=70%:%
%:%143=70%:%
%:%144=70%:%
%:%145=70%:%
%:%146=71%:%
%:%147=72%:%
%:%148=72%:%
%:%149=73%:%
%:%150=73%:%
%:%151=74%:%
%:%152=74%:%
%:%153=75%:%
%:%154=75%:%
%:%155=76%:%
%:%156=76%:%
%:%157=76%:%
%:%158=76%:%
%:%159=77%:%
%:%160=77%:%
%:%161=77%:%
%:%162=78%:%
%:%163=78%:%
%:%164=79%:%
%:%165=79%:%
%:%166=79%:%
%:%167=79%:%
%:%168=80%:%
%:%169=81%:%
%:%170=81%:%
%:%172=83%:%
%:%173=84%:%
%:%174=84%:%
%:%175=85%:%
%:%176=85%:%
%:%177=85%:%
%:%179=87%:%
%:%180=88%:%
%:%181=88%:%
%:%182=89%:%
%:%183=89%:%
%:%184=90%:%
%:%185=90%:%
%:%186=91%:%
%:%187=91%:%
%:%188=92%:%
%:%189=92%:%
%:%190=92%:%
%:%191=93%:%
%:%192=93%:%
%:%193=94%:%
%:%194=94%:%
%:%195=95%:%
%:%196=95%:%
%:%197=95%:%
%:%198=96%:%
%:%199=97%:%
%:%200=97%:%
%:%201=98%:%
%:%202=98%:%
%:%203=98%:%
%:%204=99%:%
%:%205=99%:%
%:%206=100%:%
%:%207=101%:%
%:%208=101%:%
%:%209=102%:%
%:%210=102%:%
%:%211=103%:%
%:%212=103%:%
%:%213=104%:%
%:%219=104%:%
%:%222=105%:%
%:%223=106%:%
%:%224=106%:%
%:%225=107%:%
%:%226=108%:%
%:%227=109%:%
%:%228=110%:%
%:%229=111%:%
%:%230=112%:%
%:%231=113%:%
%:%238=114%:%
%:%239=114%:%
%:%240=115%:%
%:%241=115%:%
%:%242=116%:%
%:%243=116%:%
%:%244=117%:%
%:%245=117%:%
%:%246=118%:%
%:%247=118%:%
%:%248=119%:%
%:%249=120%:%
%:%250=120%:%
%:%251=121%:%
%:%252=122%:%
%:%253=122%:%
%:%254=123%:%
%:%255=123%:%
%:%256=123%:%
%:%257=124%:%
%:%258=125%:%
%:%259=125%:%
%:%260=126%:%
%:%261=126%:%
%:%262=127%:%
%:%263=128%:%
%:%264=128%:%
%:%265=128%:%
%:%266=128%:%
%:%267=129%:%
%:%268=130%:%
%:%269=130%:%
%:%270=131%:%
%:%271=131%:%
%:%272=132%:%
%:%273=132%:%
%:%274=133%:%
%:%275=133%:%
%:%276=134%:%
%:%277=135%:%
%:%278=135%:%
%:%279=135%:%
%:%280=136%:%
%:%281=137%:%
%:%282=137%:%
%:%283=138%:%
%:%284=138%:%
%:%285=139%:%
%:%286=139%:%
%:%287=139%:%
%:%288=139%:%
%:%289=140%:%
%:%290=140%:%
%:%291=141%:%
%:%292=142%:%
%:%293=143%:%
%:%294=143%:%
%:%295=144%:%
%:%296=144%:%
%:%297=145%:%
%:%298=145%:%
%:%299=146%:%
%:%300=146%:%
%:%301=146%:%
%:%302=146%:%
%:%303=147%:%
%:%304=147%:%
%:%305=148%:%
%:%306=148%:%
%:%307=149%:%
%:%308=149%:%
%:%309=150%:%
%:%310=150%:%
%:%311=151%:%
%:%312=151%:%
%:%313=152%:%
%:%314=152%:%
%:%315=153%:%
%:%316=153%:%
%:%317=154%:%
%:%318=154%:%
%:%319=155%:%
%:%320=155%:%
%:%321=156%:%
%:%322=156%:%
%:%323=157%:%
%:%324=157%:%
%:%325=158%:%
%:%326=159%:%
%:%327=159%:%
%:%328=159%:%
%:%329=160%:%
%:%330=161%:%
%:%331=161%:%
%:%332=162%:%
%:%333=162%:%
%:%334=163%:%
%:%335=163%:%
%:%336=164%:%
%:%337=164%:%
%:%338=165%:%
%:%339=165%:%
%:%340=166%:%
%:%341=167%:%
%:%342=167%:%
%:%343=168%:%
%:%344=168%:%
%:%345=169%:%
%:%346=169%:%
%:%347=170%:%
%:%348=170%:%
%:%349=171%:%
%:%350=171%:%
%:%351=171%:%
%:%352=171%:%
%:%353=172%:%
%:%354=172%:%
%:%355=172%:%
%:%356=173%:%
%:%357=173%:%
%:%358=174%:%
%:%359=174%:%
%:%360=174%:%
%:%361=174%:%
%:%362=175%:%
%:%363=176%:%
%:%364=176%:%
%:%365=177%:%
%:%366=177%:%
%:%367=178%:%
%:%368=178%:%
%:%369=179%:%
%:%370=179%:%
%:%371=179%:%
%:%372=180%:%
%:%373=180%:%
%:%374=181%:%
%:%375=181%:%
%:%376=182%:%
%:%377=182%:%
%:%378=183%:%
%:%379=183%:%
%:%380=184%:%
%:%381=184%:%
%:%382=185%:%
%:%383=185%:%
%:%384=186%:%
%:%385=186%:%
%:%386=187%:%
%:%387=187%:%
%:%388=188%:%
%:%389=188%:%
%:%390=188%:%
%:%391=189%:%
%:%392=189%:%
%:%393=190%:%
%:%394=190%:%
%:%395=191%:%
%:%396=191%:%
%:%397=192%:%
%:%398=192%:%
%:%399=192%:%
%:%400=193%:%
%:%401=193%:%
%:%402=194%:%
%:%403=194%:%
%:%404=194%:%
%:%405=194%:%
%:%406=195%:%
%:%407=195%:%
%:%408=196%:%
%:%409=197%:%
%:%410=197%:%
%:%411=198%:%
%:%412=198%:%
%:%413=199%:%
%:%414=199%:%
%:%415=200%:%
%:%416=200%:%
%:%417=201%:%
%:%418=201%:%
%:%419=201%:%
%:%420=202%:%
%:%421=202%:%
%:%422=202%:%
%:%423=203%:%
%:%424=203%:%
%:%425=204%:%
%:%426=204%:%
%:%427=205%:%
%:%428=205%:%
%:%429=205%:%
%:%430=206%:%
%:%431=206%:%
%:%432=207%:%
%:%433=207%:%
%:%434=207%:%
%:%435=208%:%
%:%436=208%:%
%:%437=208%:%
%:%438=209%:%
%:%439=209%:%
%:%440=210%:%
%:%441=210%:%
%:%442=211%:%
%:%443=211%:%
%:%444=212%:%
%:%445=212%:%
%:%446=213%:%
%:%447=214%:%
%:%448=214%:%
%:%449=214%:%
%:%450=214%:%
%:%451=215%:%
%:%452=215%:%
%:%453=216%:%
%:%454=217%:%
%:%455=217%:%
%:%456=218%:%
%:%457=218%:%
%:%458=219%:%
%:%459=220%:%
%:%460=220%:%
%:%461=221%:%
%:%462=221%:%
%:%463=222%:%
%:%464=222%:%
%:%465=222%:%
%:%466=223%:%
%:%467=223%:%
%:%468=224%:%
%:%469=224%:%
%:%470=225%:%
%:%471=225%:%
%:%472=226%:%
%:%473=226%:%
%:%474=227%:%
%:%475=227%:%
%:%476=227%:%
%:%477=228%:%
%:%478=229%:%
%:%479=229%:%
%:%480=230%:%
%:%481=230%:%
%:%482=231%:%
%:%483=231%:%
%:%484=232%:%
%:%485=232%:%
%:%486=232%:%
%:%487=233%:%
%:%488=233%:%
%:%489=234%:%
%:%490=234%:%
%:%491=235%:%
%:%492=235%:%
%:%493=235%:%
%:%494=236%:%
%:%495=236%:%
%:%496=237%:%
%:%497=237%:%
%:%498=238%:%
%:%499=238%:%
%:%500=238%:%
%:%501=238%:%
%:%502=239%:%
%:%503=240%:%
%:%504=240%:%
%:%505=241%:%
%:%506=242%:%
%:%507=242%:%
%:%508=243%:%
%:%509=243%:%
%:%510=244%:%
%:%511=244%:%
%:%512=244%:%
%:%513=245%:%
%:%514=246%:%
%:%515=246%:%
%:%516=247%:%
%:%517=247%:%
%:%518=248%:%
%:%519=248%:%
%:%520=249%:%
%:%521=249%:%
%:%522=250%:%
%:%523=250%:%
%:%524=251%:%
%:%525=251%:%
%:%526=252%:%
%:%527=252%:%
%:%528=252%:%
%:%530=254%:%
%:%531=255%:%
%:%532=255%:%
%:%533=256%:%
%:%534=256%:%
%:%535=257%:%
%:%536=257%:%
%:%537=258%:%
%:%538=258%:%
%:%539=258%:%
%:%540=259%:%
%:%541=260%:%
%:%542=260%:%
%:%543=261%:%
%:%544=261%:%
%:%545=262%:%
%:%546=262%:%
%:%547=263%:%
%:%548=263%:%
%:%549=263%:%
%:%550=264%:%
%:%551=264%:%
%:%552=265%:%
%:%553=265%:%
%:%554=266%:%
%:%555=266%:%
%:%556=267%:%
%:%557=267%:%
%:%558=268%:%
%:%559=268%:%
%:%560=269%:%
%:%561=269%:%
%:%562=269%:%
%:%563=270%:%
%:%564=270%:%
%:%565=271%:%
%:%566=271%:%
%:%567=272%:%
%:%568=272%:%
%:%569=273%:%
%:%570=273%:%
%:%571=273%:%
%:%572=274%:%
%:%573=274%:%
%:%574=274%:%
%:%575=275%:%
%:%576=275%:%
%:%577=276%:%
%:%578=277%:%
%:%579=277%:%
%:%580=277%:%
%:%581=278%:%
%:%582=278%:%
%:%583=279%:%
%:%584=280%:%
%:%585=280%:%
%:%586=281%:%
%:%587=281%:%
%:%588=282%:%
%:%589=282%:%
%:%590=283%:%
%:%591=283%:%
%:%592=283%:%
%:%593=284%:%
%:%594=285%:%
%:%595=285%:%
%:%596=285%:%
%:%597=286%:%
%:%603=286%:%
%:%606=287%:%
%:%607=288%:%
%:%608=288%:%
%:%609=289%:%
%:%610=290%:%
%:%611=291%:%
%:%612=292%:%
%:%613=293%:%
%:%614=294%:%
%:%615=295%:%
%:%616=296%:%
%:%617=297%:%
%:%618=298%:%
%:%619=299%:%
%:%626=300%:%
%:%627=300%:%
%:%628=301%:%
%:%629=301%:%
%:%630=302%:%
%:%631=302%:%
%:%632=303%:%
%:%633=303%:%
%:%634=304%:%
%:%635=304%:%
%:%636=304%:%
%:%637=305%:%
%:%638=306%:%
%:%639=306%:%
%:%640=307%:%
%:%641=307%:%
%:%642=308%:%
%:%643=308%:%
%:%644=308%:%
%:%645=309%:%
%:%646=309%:%
%:%647=310%:%
%:%648=310%:%
%:%649=311%:%
%:%650=311%:%
%:%651=312%:%
%:%652=312%:%
%:%653=313%:%
%:%654=313%:%
%:%655=314%:%
%:%656=314%:%
%:%657=315%:%
%:%658=315%:%
%:%659=315%:%
%:%660=316%:%
%:%661=316%:%
%:%662=317%:%
%:%663=317%:%
%:%664=318%:%
%:%665=318%:%
%:%666=319%:%
%:%667=319%:%
%:%668=320%:%
%:%669=320%:%
%:%670=321%:%
%:%671=321%:%
%:%672=321%:%
%:%673=322%:%
%:%674=323%:%
%:%675=323%:%
%:%676=324%:%
%:%677=324%:%
%:%678=324%:%
%:%679=324%:%
%:%680=325%:%
%:%686=325%:%
%:%689=326%:%
%:%690=327%:%
%:%691=327%:%
%:%692=328%:%
%:%693=329%:%
%:%694=330%:%
%:%695=331%:%
%:%696=332%:%
%:%703=333%:%
%:%704=333%:%
%:%705=334%:%
%:%706=334%:%
%:%707=335%:%
%:%708=335%:%
%:%709=336%:%
%:%710=336%:%
%:%711=337%:%
%:%712=337%:%
%:%713=338%:%
%:%714=339%:%
%:%715=340%:%
%:%716=340%:%
%:%717=341%:%
%:%718=341%:%
%:%719=342%:%
%:%720=342%:%
%:%721=343%:%
%:%722=343%:%
%:%723=344%:%
%:%724=344%:%
%:%725=345%:%
%:%726=346%:%
%:%727=346%:%
%:%728=347%:%
%:%729=347%:%
%:%730=348%:%
%:%731=348%:%
%:%732=348%:%
%:%733=349%:%
%:%734=350%:%
%:%735=350%:%
%:%736=351%:%
%:%737=351%:%
%:%738=352%:%
%:%739=352%:%
%:%740=353%:%
%:%741=354%:%
%:%742=354%:%
%:%743=354%:%
%:%744=355%:%
%:%745=356%:%
%:%746=356%:%
%:%747=357%:%
%:%748=357%:%
%:%749=358%:%
%:%750=359%:%
%:%751=359%:%
%:%752=359%:%
%:%753=359%:%
%:%754=360%:%
%:%755=361%:%
%:%756=361%:%
%:%757=362%:%
%:%758=362%:%
%:%759=363%:%
%:%760=364%:%
%:%761=364%:%
%:%762=365%:%
%:%763=365%:%
%:%764=366%:%
%:%765=366%:%
%:%766=367%:%
%:%767=367%:%
%:%768=368%:%
%:%769=369%:%
%:%770=369%:%
%:%771=370%:%
%:%772=370%:%
%:%773=371%:%
%:%774=371%:%
%:%775=371%:%
%:%776=372%:%
%:%777=372%:%
%:%778=373%:%
%:%779=373%:%
%:%780=374%:%
%:%781=375%:%
%:%782=375%:%
%:%783=376%:%
%:%784=376%:%
%:%785=377%:%
%:%786=377%:%
%:%787=377%:%
%:%788=378%:%
%:%789=379%:%
%:%790=379%:%
%:%791=380%:%
%:%792=380%:%
%:%793=380%:%
%:%794=381%:%
%:%795=382%:%
%:%796=382%:%
%:%797=383%:%
%:%798=383%:%
%:%799=383%:%
%:%800=384%:%
%:%801=385%:%
%:%802=385%:%
%:%803=386%:%
%:%804=386%:%
%:%805=387%:%
%:%806=388%:%
%:%807=388%:%
%:%808=389%:%
%:%809=389%:%
%:%810=390%:%
%:%811=391%:%
%:%812=391%:%
%:%816=395%:%
%:%817=396%:%
%:%818=397%:%
%:%819=397%:%
%:%820=398%:%
%:%821=399%:%
%:%822=400%:%
%:%823=400%:%
%:%824=401%:%
%:%825=402%:%
%:%826=402%:%
%:%827=403%:%
%:%828=403%:%
%:%829=404%:%
%:%830=404%:%
%:%831=405%:%
%:%832=405%:%
%:%833=406%:%
%:%834=406%:%
%:%835=407%:%
%:%836=407%:%
%:%837=408%:%
%:%838=408%:%
%:%839=408%:%
%:%840=409%:%
%:%841=410%:%
%:%842=410%:%
%:%843=411%:%
%:%844=411%:%
%:%845=412%:%
%:%846=412%:%
%:%847=413%:%
%:%848=413%:%
%:%849=414%:%
%:%850=414%:%
%:%851=415%:%
%:%852=415%:%
%:%853=416%:%
%:%854=416%:%
%:%855=417%:%
%:%856=417%:%
%:%857=418%:%
%:%858=418%:%
%:%859=419%:%
%:%860=419%:%
%:%861=420%:%
%:%862=420%:%
%:%863=420%:%
%:%864=421%:%
%:%865=421%:%
%:%866=422%:%
%:%867=422%:%
%:%868=423%:%
%:%869=423%:%
%:%870=424%:%
%:%871=424%:%
%:%872=425%:%
%:%873=425%:%
%:%874=426%:%
%:%875=426%:%
%:%876=427%:%
%:%877=427%:%
%:%878=428%:%
%:%879=428%:%
%:%880=428%:%
%:%881=429%:%
%:%882=429%:%
%:%883=430%:%
%:%884=430%:%
%:%885=431%:%
%:%886=431%:%
%:%887=432%:%
%:%888=432%:%
%:%889=432%:%
%:%890=433%:%
%:%891=433%:%
%:%892=434%:%
%:%893=434%:%
%:%894=435%:%
%:%895=435%:%
%:%896=436%:%
%:%897=436%:%
%:%898=437%:%
%:%899=437%:%
%:%900=438%:%
%:%901=438%:%
%:%902=438%:%
%:%903=439%:%
%:%904=439%:%
%:%905=439%:%
%:%906=440%:%
%:%907=440%:%
%:%908=440%:%
%:%909=441%:%
%:%910=441%:%
%:%911=442%:%
%:%912=442%:%
%:%913=442%:%
%:%914=443%:%
%:%915=443%:%
%:%916=444%:%
%:%917=444%:%
%:%918=445%:%
%:%919=446%:%
%:%920=446%:%
%:%921=447%:%
%:%922=447%:%
%:%923=448%:%
%:%924=448%:%
%:%925=449%:%
%:%926=449%:%
%:%927=450%:%
%:%928=450%:%
%:%929=451%:%
%:%930=451%:%
%:%931=452%:%
%:%932=452%:%
%:%933=453%:%
%:%934=453%:%
%:%935=454%:%
%:%936=454%:%
%:%937=455%:%
%:%938=455%:%
%:%939=455%:%
%:%940=456%:%
%:%941=456%:%
%:%942=457%:%
%:%943=457%:%
%:%944=458%:%
%:%945=458%:%
%:%946=459%:%
%:%947=459%:%
%:%948=460%:%
%:%949=460%:%
%:%950=461%:%
%:%951=461%:%
%:%952=462%:%
%:%953=462%:%
%:%954=463%:%
%:%955=463%:%
%:%956=463%:%
%:%957=464%:%
%:%958=464%:%
%:%959=465%:%
%:%960=465%:%
%:%961=465%:%
%:%962=466%:%
%:%963=466%:%
%:%964=467%:%
%:%965=467%:%
%:%966=468%:%
%:%967=469%:%
%:%968=469%:%
%:%969=470%:%
%:%970=470%:%
%:%971=471%:%
%:%972=471%:%
%:%973=471%:%
%:%974=472%:%
%:%975=472%:%
%:%976=473%:%
%:%977=474%:%
%:%978=474%:%
%:%979=475%:%
%:%980=475%:%
%:%981=476%:%
%:%982=476%:%
%:%983=477%:%
%:%984=477%:%
%:%985=478%:%
%:%986=478%:%
%:%987=479%:%
%:%988=479%:%
%:%989=480%:%
%:%990=480%:%
%:%991=481%:%
%:%992=481%:%
%:%993=482%:%
%:%994=483%:%
%:%995=483%:%
%:%996=484%:%
%:%997=484%:%
%:%998=485%:%
%:%999=485%:%
%:%1000=486%:%
%:%1001=486%:%
%:%1002=487%:%
%:%1003=487%:%
%:%1004=488%:%
%:%1005=488%:%
%:%1006=489%:%
%:%1007=489%:%
%:%1008=490%:%
%:%1009=491%:%
%:%1010=491%:%
%:%1011=492%:%
%:%1012=492%:%
%:%1013=493%:%
%:%1014=493%:%
%:%1015=494%:%
%:%1016=494%:%
%:%1017=495%:%
%:%1018=495%:%
%:%1019=496%:%
%:%1020=496%:%
%:%1021=497%:%
%:%1022=497%:%
%:%1023=498%:%
%:%1024=498%:%
%:%1025=499%:%
%:%1026=499%:%
%:%1027=500%:%
%:%1028=501%:%
%:%1029=501%:%
%:%1030=502%:%
%:%1031=502%:%
%:%1032=503%:%
%:%1033=503%:%
%:%1034=503%:%
%:%1035=504%:%
%:%1036=505%:%
%:%1037=505%:%
%:%1038=506%:%
%:%1039=507%:%
%:%1040=507%:%
%:%1041=508%:%
%:%1042=508%:%
%:%1043=509%:%
%:%1044=509%:%
%:%1045=510%:%
%:%1046=510%:%
%:%1047=511%:%
%:%1048=511%:%
%:%1049=512%:%
%:%1050=512%:%
%:%1051=513%:%
%:%1052=513%:%
%:%1053=513%:%
%:%1054=513%:%
%:%1055=513%:%
%:%1056=514%:%
%:%1057=514%:%
%:%1058=515%:%
%:%1059=515%:%
%:%1060=516%:%
%:%1061=516%:%
%:%1062=516%:%
%:%1063=516%:%
%:%1064=517%:%
%:%1065=518%:%
%:%1066=518%:%
%:%1067=519%:%
%:%1068=519%:%
%:%1069=520%:%
%:%1070=520%:%
%:%1071=520%:%
%:%1072=521%:%
%:%1073=521%:%
%:%1074=522%:%
%:%1075=522%:%
%:%1076=522%:%
%:%1077=523%:%
%:%1078=523%:%
%:%1079=524%:%
%:%1080=524%:%
%:%1081=525%:%
%:%1082=525%:%
%:%1083=526%:%
%:%1084=526%:%
%:%1085=527%:%
%:%1086=527%:%
%:%1087=528%:%
%:%1088=529%:%
%:%1089=529%:%
%:%1090=530%:%
%:%1091=530%:%
%:%1092=531%:%
%:%1093=531%:%
%:%1094=532%:%
%:%1095=532%:%
%:%1096=532%:%
%:%1097=533%:%
%:%1098=533%:%
%:%1099=534%:%
%:%1100=535%:%
%:%1101=535%:%
%:%1102=536%:%
%:%1103=536%:%
%:%1104=536%:%
%:%1105=537%:%
%:%1106=537%:%
%:%1107=538%:%
%:%1108=538%:%
%:%1109=538%:%
%:%1110=539%:%
%:%1111=539%:%
%:%1112=540%:%
%:%1113=540%:%
%:%1114=540%:%
%:%1115=541%:%
%:%1116=541%:%
%:%1117=541%:%
%:%1118=542%:%
%:%1119=542%:%
%:%1120=543%:%
%:%1121=543%:%
%:%1122=544%:%
%:%1123=544%:%
%:%1124=545%:%
%:%1125=546%:%
%:%1126=546%:%
%:%1127=547%:%
%:%1128=547%:%
%:%1129=548%:%
%:%1130=548%:%
%:%1131=549%:%
%:%1132=549%:%
%:%1133=550%:%
%:%1134=550%:%
%:%1135=550%:%
%:%1136=551%:%
%:%1137=551%:%
%:%1138=552%:%
%:%1139=552%:%
%:%1140=553%:%
%:%1141=553%:%
%:%1142=553%:%
%:%1143=554%:%
%:%1149=554%:%
%:%1152=555%:%
%:%1153=556%:%
%:%1154=556%:%
%:%1155=557%:%
%:%1162=564%:%
%:%1163=565%:%
%:%1164=566%:%
%:%1165=566%:%
%:%1166=567%:%
%:%1171=572%:%
%:%1172=573%:%
%:%1173=574%:%
%:%1174=574%:%
%:%1177=575%:%
%:%1181=575%:%
%:%1182=575%:%
%:%1183=576%:%
%:%1184=576%:%
%:%1185=577%:%
%:%1186=577%:%
%:%1187=578%:%
%:%1188=578%:%
%:%1189=579%:%
%:%1190=579%:%
%:%1191=580%:%
%:%1192=580%:%
%:%1193=581%:%
%:%1194=581%:%
%:%1199=581%:%
%:%1202=582%:%
%:%1203=583%:%
%:%1204=583%:%
%:%1205=584%:%
%:%1206=585%:%
%:%1207=586%:%
%:%1208=587%:%
%:%1209=588%:%
%:%1216=589%:%
%:%1217=589%:%
%:%1218=590%:%
%:%1219=590%:%
%:%1220=591%:%
%:%1221=591%:%
%:%1222=592%:%
%:%1223=592%:%
%:%1224=593%:%
%:%1225=594%:%
%:%1226=594%:%
%:%1227=595%:%
%:%1228=595%:%
%:%1229=596%:%
%:%1230=597%:%
%:%1231=597%:%
%:%1232=598%:%
%:%1233=598%:%
%:%1234=599%:%
%:%1235=599%:%
%:%1236=600%:%
%:%1237=600%:%
%:%1238=601%:%
%:%1239=601%:%
%:%1240=602%:%
%:%1241=602%:%
%:%1242=603%:%
%:%1243=603%:%
%:%1244=604%:%
%:%1245=605%:%
%:%1246=605%:%
%:%1249=608%:%
%:%1250=609%:%
%:%1251=609%:%
%:%1252=610%:%
%:%1253=610%:%
%:%1254=611%:%
%:%1255=611%:%
%:%1256=612%:%
%:%1257=613%:%
%:%1258=613%:%
%:%1259=613%:%
%:%1260=613%:%
%:%1261=614%:%
%:%1262=615%:%
%:%1263=615%:%
%:%1264=616%:%
%:%1265=617%:%
%:%1266=617%:%
%:%1267=618%:%
%:%1268=618%:%
%:%1269=619%:%
%:%1270=619%:%
%:%1271=620%:%
%:%1272=620%:%
%:%1273=621%:%
%:%1274=621%:%
%:%1275=622%:%
%:%1276=622%:%
%:%1277=622%:%
%:%1278=623%:%
%:%1279=623%:%
%:%1280=623%:%
%:%1281=624%:%
%:%1282=624%:%
%:%1283=625%:%
%:%1284=625%:%
%:%1285=626%:%
%:%1286=626%:%
%:%1287=626%:%
%:%1288=627%:%
%:%1289=627%:%
%:%1290=627%:%
%:%1291=628%:%
%:%1292=628%:%
%:%1293=629%:%
%:%1294=629%:%
%:%1295=629%:%
%:%1296=630%:%
%:%1297=630%:%
%:%1298=631%:%
%:%1299=631%:%
%:%1300=632%:%
%:%1301=633%:%
%:%1302=633%:%
%:%1304=635%:%
%:%1305=636%:%
%:%1306=636%:%
%:%1307=637%:%
%:%1308=637%:%
%:%1309=638%:%
%:%1310=638%:%
%:%1311=639%:%
%:%1312=639%:%
%:%1313=639%:%
%:%1314=640%:%
%:%1315=640%:%
%:%1316=641%:%
%:%1317=641%:%
%:%1318=642%:%
%:%1319=642%:%
%:%1320=643%:%
%:%1321=643%:%
%:%1322=643%:%
%:%1323=644%:%
%:%1324=644%:%
%:%1325=645%:%
%:%1326=645%:%
%:%1327=646%:%
%:%1328=646%:%
%:%1329=646%:%
%:%1330=647%:%
%:%1331=647%:%
%:%1332=647%:%
%:%1333=648%:%
%:%1334=648%:%
%:%1335=649%:%
%:%1336=649%:%
%:%1337=649%:%
%:%1338=650%:%
%:%1339=650%:%
%:%1340=651%:%
%:%1341=651%:%
%:%1342=651%:%
%:%1343=652%:%
%:%1344=652%:%
%:%1345=653%:%
%:%1346=653%:%
%:%1347=653%:%
%:%1348=654%:%
%:%1349=654%:%
%:%1350=655%:%
%:%1351=655%:%
%:%1352=656%:%
%:%1353=657%:%
%:%1354=657%:%
%:%1360=663%:%
%:%1361=664%:%
%:%1362=664%:%
%:%1363=665%:%
%:%1364=665%:%
%:%1365=666%:%
%:%1366=667%:%
%:%1367=668%:%
%:%1368=668%:%
%:%1369=669%:%
%:%1370=669%:%
%:%1371=670%:%
%:%1372=670%:%
%:%1373=671%:%
%:%1374=671%:%
%:%1375=672%:%
%:%1376=672%:%
%:%1377=673%:%
%:%1378=673%:%
%:%1379=674%:%
%:%1380=674%:%
%:%1381=675%:%
%:%1382=675%:%
%:%1383=676%:%
%:%1384=676%:%
%:%1385=677%:%
%:%1386=677%:%
%:%1387=677%:%
%:%1388=678%:%
%:%1389=678%:%
%:%1390=679%:%
%:%1391=679%:%
%:%1392=679%:%
%:%1394=681%:%
%:%1395=681%:%
%:%1396=682%:%
%:%1397=682%:%
%:%1398=683%:%
%:%1399=684%:%
%:%1400=684%:%
%:%1401=685%:%
%:%1402=685%:%
%:%1403=686%:%
%:%1404=686%:%
%:%1405=687%:%
%:%1406=687%:%
%:%1407=688%:%
%:%1408=688%:%
%:%1409=689%:%
%:%1410=689%:%
%:%1411=690%:%
%:%1412=690%:%
%:%1413=691%:%
%:%1414=691%:%
%:%1415=692%:%
%:%1416=692%:%
%:%1417=693%:%
%:%1423=693%:%
%:%1426=694%:%
%:%1427=695%:%
%:%1428=695%:%
%:%1429=696%:%
%:%1430=697%:%
%:%1431=698%:%
%:%1438=699%:%
%:%1439=699%:%
%:%1440=700%:%
%:%1441=700%:%
%:%1442=701%:%
%:%1443=701%:%
%:%1444=702%:%
%:%1445=702%:%
%:%1446=703%:%
%:%1447=703%:%
%:%1448=704%:%
%:%1449=705%:%
%:%1450=705%:%
%:%1451=706%:%
%:%1452=707%:%
%:%1453=707%:%
%:%1454=708%:%
%:%1455=708%:%
%:%1456=709%:%
%:%1457=709%:%
%:%1458=710%:%
%:%1459=710%:%
%:%1460=711%:%
%:%1461=712%:%
%:%1462=712%:%
%:%1463=713%:%
%:%1464=713%:%
%:%1465=714%:%
%:%1466=714%:%
%:%1467=715%:%
%:%1468=715%:%
%:%1469=716%:%
%:%1470=716%:%
%:%1471=717%:%
%:%1472=718%:%
%:%1473=718%:%
%:%1474=719%:%
%:%1475=719%:%
%:%1476=720%:%
%:%1477=720%:%
%:%1478=721%:%
%:%1479=721%:%
%:%1480=722%:%
%:%1481=722%:%
%:%1482=723%:%
%:%1483=723%:%
%:%1484=724%:%
%:%1485=725%:%
%:%1486=725%:%
%:%1487=726%:%
%:%1488=726%:%
%:%1489=727%:%
%:%1490=727%:%
%:%1491=728%:%
%:%1492=728%:%
%:%1493=729%:%
%:%1494=729%:%
%:%1495=730%:%
%:%1496=730%:%
%:%1497=731%:%
%:%1498=732%:%
%:%1499=732%:%
%:%1500=733%:%
%:%1501=733%:%
%:%1502=734%:%
%:%1503=734%:%
%:%1504=735%:%
%:%1505=735%:%
%:%1506=736%:%
%:%1507=736%:%
%:%1508=737%:%
%:%1509=737%:%
%:%1510=738%:%
%:%1511=739%:%
%:%1512=739%:%
%:%1513=740%:%
%:%1514=740%:%
%:%1515=741%:%
%:%1516=741%:%
%:%1517=742%:%
%:%1518=742%:%
%:%1519=743%:%
%:%1520=743%:%
%:%1521=744%:%
%:%1522=744%:%
%:%1523=745%:%
%:%1524=746%:%
%:%1525=746%:%
%:%1526=747%:%
%:%1527=747%:%
%:%1528=748%:%
%:%1529=748%:%
%:%1530=749%:%
%:%1531=749%:%
%:%1532=750%:%
%:%1533=751%:%
%:%1534=751%:%
%:%1535=752%:%
%:%1536=752%:%
%:%1537=753%:%
%:%1538=753%:%
%:%1539=754%:%
%:%1540=754%:%
%:%1541=755%:%
%:%1542=756%:%
%:%1543=756%:%
%:%1544=757%:%
%:%1545=757%:%
%:%1546=758%:%
%:%1547=759%:%
%:%1548=759%:%
%:%1549=760%:%
%:%1550=760%:%
%:%1551=761%:%
%:%1552=762%:%
%:%1553=762%:%
%:%1554=763%:%
%:%1555=763%:%
%:%1556=764%:%
%:%1557=764%:%
%:%1558=765%:%
%:%1559=765%:%
%:%1560=766%:%
%:%1561=767%:%
%:%1562=767%:%
%:%1563=768%:%
%:%1564=768%:%
%:%1565=769%:%
%:%1566=769%:%
%:%1567=770%:%
%:%1568=770%:%
%:%1569=771%:%
%:%1570=771%:%
%:%1571=772%:%
%:%1572=773%:%
%:%1573=773%:%
%:%1574=774%:%
%:%1575=774%:%
%:%1576=775%:%
%:%1577=775%:%
%:%1578=776%:%
%:%1579=776%:%
%:%1580=777%:%
%:%1581=778%:%
%:%1582=778%:%
%:%1583=779%:%
%:%1584=780%:%
%:%1585=780%:%
%:%1586=781%:%
%:%1587=781%:%
%:%1588=782%:%
%:%1589=782%:%
%:%1590=783%:%
%:%1591=784%:%
%:%1592=784%:%
%:%1593=785%:%
%:%1594=785%:%
%:%1595=786%:%
%:%1596=787%:%
%:%1597=787%:%
%:%1598=788%:%
%:%1599=788%:%
%:%1600=789%:%
%:%1601=789%:%
%:%1602=790%:%
%:%1603=790%:%
%:%1604=791%:%
%:%1605=791%:%
%:%1606=792%:%
%:%1607=792%:%
%:%1608=793%:%
%:%1609=794%:%
%:%1610=794%:%
%:%1611=795%:%
%:%1612=795%:%
%:%1613=796%:%
%:%1614=796%:%
%:%1615=797%:%
%:%1616=797%:%
%:%1617=798%:%
%:%1618=798%:%
%:%1619=799%:%
%:%1620=799%:%
%:%1621=800%:%
%:%1622=800%:%
%:%1623=801%:%
%:%1624=802%:%
%:%1625=802%:%
%:%1626=803%:%
%:%1627=803%:%
%:%1628=804%:%
%:%1629=804%:%
%:%1630=805%:%
%:%1631=805%:%
%:%1632=806%:%
%:%1633=806%:%
%:%1634=807%:%
%:%1635=807%:%
%:%1636=808%:%
%:%1637=808%:%
%:%1638=809%:%
%:%1639=810%:%
%:%1640=810%:%
%:%1641=811%:%
%:%1642=811%:%
%:%1643=812%:%
%:%1644=812%:%
%:%1645=813%:%
%:%1646=813%:%
%:%1647=814%:%
%:%1648=814%:%
%:%1649=815%:%
%:%1650=815%:%
%:%1651=816%:%
%:%1652=816%:%
%:%1653=817%:%
%:%1654=817%:%
%:%1655=818%:%
%:%1656=818%:%
%:%1657=819%:%
%:%1658=819%:%
%:%1659=820%:%
%:%1660=820%:%
%:%1661=821%:%
%:%1662=821%:%
%:%1663=822%:%
%:%1664=822%:%
%:%1665=823%:%
%:%1666=823%:%
%:%1667=824%:%
%:%1668=824%:%
%:%1669=825%:%
%:%1670=826%:%
%:%1671=826%:%
%:%1672=827%:%
%:%1673=827%:%
%:%1674=828%:%
%:%1675=828%:%
%:%1676=829%:%
%:%1677=829%:%
%:%1678=830%:%
%:%1679=830%:%
%:%1680=831%:%
%:%1681=831%:%
%:%1682=832%:%
%:%1683=832%:%
%:%1684=833%:%
%:%1685=833%:%
%:%1686=834%:%
%:%1687=835%:%
%:%1688=835%:%
%:%1689=836%:%
%:%1690=837%:%
%:%1691=837%:%
%:%1692=838%:%
%:%1693=838%:%
%:%1694=839%:%
%:%1695=839%:%
%:%1696=840%:%
%:%1697=840%:%
%:%1698=841%:%
%:%1699=841%:%
%:%1700=842%:%
%:%1701=843%:%
%:%1702=843%:%
%:%1703=844%:%
%:%1704=845%:%
%:%1705=845%:%
%:%1706=846%:%
%:%1707=846%:%
%:%1708=847%:%
%:%1709=847%:%
%:%1710=848%:%
%:%1711=848%:%
%:%1712=848%:%
%:%1713=849%:%
%:%1714=850%:%
%:%1715=850%:%
%:%1716=851%:%
%:%1717=852%:%
%:%1718=852%:%
%:%1719=853%:%
%:%1720=853%:%
%:%1721=854%:%
%:%1722=854%:%
%:%1723=855%:%
%:%1724=855%:%
%:%1725=856%:%
%:%1726=856%:%
%:%1727=857%:%
%:%1728=857%:%
%:%1729=858%:%
%:%1730=858%:%
%:%1731=859%:%
%:%1732=859%:%
%:%1733=860%:%
%:%1734=860%:%
%:%1735=861%:%
%:%1736=861%:%
%:%1737=862%:%
%:%1738=862%:%
%:%1739=863%:%
%:%1740=863%:%
%:%1741=864%:%
%:%1742=864%:%
%:%1743=865%:%
%:%1744=865%:%
%:%1745=866%:%
%:%1746=867%:%
%:%1747=867%:%
%:%1748=868%:%
%:%1749=868%:%
%:%1750=869%:%
%:%1751=869%:%
%:%1752=870%:%
%:%1753=870%:%
%:%1754=870%:%
%:%1755=871%:%
%:%1756=871%:%
%:%1757=872%:%
%:%1758=872%:%
%:%1759=873%:%
%:%1760=873%:%
%:%1761=874%:%
%:%1762=874%:%
%:%1763=875%:%
%:%1764=875%:%
%:%1765=876%:%
%:%1766=876%:%
%:%1767=877%:%
%:%1768=877%:%
%:%1769=878%:%
%:%1770=878%:%
%:%1771=879%:%
%:%1772=879%:%
%:%1773=880%:%
%:%1774=880%:%
%:%1775=881%:%
%:%1776=881%:%
%:%1777=882%:%
%:%1778=882%:%
%:%1779=882%:%
%:%1780=883%:%
%:%1781=883%:%
%:%1782=884%:%
%:%1783=884%:%
%:%1784=885%:%
%:%1785=885%:%
%:%1786=886%:%
%:%1787=886%:%
%:%1788=886%:%
%:%1789=886%:%
%:%1790=887%:%
%:%1796=887%:%
%:%1801=888%:%
%:%1806=889%:%

%
\begin{isabellebody}%
\setisabellecontext{Frequency{\isacharunderscore}{\kern0pt}Moment{\isacharunderscore}{\kern0pt}k}%
%
\isadelimdocument
%
\endisadelimdocument
%
\isatagdocument
%
\isamarkupsection{Frequency Moment $k$%
}
\isamarkuptrue%
%
\endisatagdocument
{\isafolddocument}%
%
\isadelimdocument
%
\endisadelimdocument
%
\isadelimtheory
%
\endisadelimtheory
%
\isatagtheory
\isacommand{theory}\isamarkupfalse%
\ Frequency{\isacharunderscore}{\kern0pt}Moment{\isacharunderscore}{\kern0pt}k\isanewline
\ \ \isakeyword{imports}\ Main\ Median\ Product{\isacharunderscore}{\kern0pt}PMF{\isacharunderscore}{\kern0pt}Ext\ Lp{\isachardot}{\kern0pt}Lp\ List{\isacharunderscore}{\kern0pt}Ext\ Encoding\ Frequency{\isacharunderscore}{\kern0pt}Moments\ Landau{\isacharunderscore}{\kern0pt}Ext\isanewline
\isakeyword{begin}%
\endisatagtheory
{\isafoldtheory}%
%
\isadelimtheory
%
\endisadelimtheory
%
\begin{isamarkuptext}%
This section contains a formalization of the algorithm for the $k$-th frequency moment.
It is based on the algorithm described in \cite[\textsection 2.1]{alon1999}.%
\end{isamarkuptext}\isamarkuptrue%
\isacommand{type{\isacharunderscore}{\kern0pt}synonym}\isamarkupfalse%
\ fk{\isacharunderscore}{\kern0pt}state\ {\isacharequal}{\kern0pt}\ {\isachardoublequoteopen}nat\ {\isasymtimes}\ nat\ {\isasymtimes}\ nat\ {\isasymtimes}\ nat\ {\isasymtimes}\ {\isacharparenleft}{\kern0pt}nat\ {\isasymtimes}\ nat\ {\isasymRightarrow}\ {\isacharparenleft}{\kern0pt}nat\ {\isasymtimes}\ nat{\isacharparenright}{\kern0pt}{\isacharparenright}{\kern0pt}{\isachardoublequoteclose}\isanewline
\isanewline
\isacommand{fun}\isamarkupfalse%
\ fk{\isacharunderscore}{\kern0pt}init\ {\isacharcolon}{\kern0pt}{\isacharcolon}{\kern0pt}\ {\isachardoublequoteopen}nat\ {\isasymRightarrow}\ rat\ {\isasymRightarrow}\ rat\ {\isasymRightarrow}\ nat\ {\isasymRightarrow}\ fk{\isacharunderscore}{\kern0pt}state\ pmf{\isachardoublequoteclose}\ \isakeyword{where}\isanewline
\ \ {\isachardoublequoteopen}fk{\isacharunderscore}{\kern0pt}init\ k\ {\isasymdelta}\ {\isasymepsilon}\ n\ {\isacharequal}{\kern0pt}\isanewline
\ \ \ \ do\ {\isacharbraceleft}{\kern0pt}\isanewline
\ \ \ \ \ \ let\ s\isactrlsub {\isadigit{1}}\ {\isacharequal}{\kern0pt}\ nat\ {\isasymlceil}{\isadigit{3}}{\isacharasterisk}{\kern0pt}real\ k{\isacharasterisk}{\kern0pt}{\isacharparenleft}{\kern0pt}real\ n{\isacharparenright}{\kern0pt}\ powr\ {\isacharparenleft}{\kern0pt}{\isadigit{1}}{\isacharminus}{\kern0pt}{\isadigit{1}}{\isacharslash}{\kern0pt}\ real\ k{\isacharparenright}{\kern0pt}{\isacharslash}{\kern0pt}\ {\isacharparenleft}{\kern0pt}real{\isacharunderscore}{\kern0pt}of{\isacharunderscore}{\kern0pt}rat\ {\isasymdelta}{\isacharparenright}{\kern0pt}\isactrlsup {\isadigit{2}}{\isasymrceil}{\isacharsemicolon}{\kern0pt}\isanewline
\ \ \ \ \ \ let\ s\isactrlsub {\isadigit{2}}\ {\isacharequal}{\kern0pt}\ nat\ {\isasymlceil}{\isacharminus}{\kern0pt}{\isadigit{1}}{\isadigit{8}}\ {\isacharasterisk}{\kern0pt}\ ln\ {\isacharparenleft}{\kern0pt}real{\isacharunderscore}{\kern0pt}of{\isacharunderscore}{\kern0pt}rat\ {\isasymepsilon}{\isacharparenright}{\kern0pt}{\isasymrceil}{\isacharsemicolon}{\kern0pt}\isanewline
\ \ \ \ \ \ return{\isacharunderscore}{\kern0pt}pmf\ {\isacharparenleft}{\kern0pt}s\isactrlsub {\isadigit{1}}{\isacharcomma}{\kern0pt}\ s\isactrlsub {\isadigit{2}}{\isacharcomma}{\kern0pt}\ k{\isacharcomma}{\kern0pt}\ {\isadigit{0}}{\isacharcomma}{\kern0pt}\ {\isacharparenleft}{\kern0pt}{\isasymlambda}{\isacharunderscore}{\kern0pt}\ {\isasymin}\ {\isacharbraceleft}{\kern0pt}{\isadigit{0}}{\isachardot}{\kern0pt}{\isachardot}{\kern0pt}{\isacharless}{\kern0pt}s\isactrlsub {\isadigit{1}}{\isacharbraceright}{\kern0pt}\ {\isasymtimes}\ {\isacharbraceleft}{\kern0pt}{\isadigit{0}}{\isachardot}{\kern0pt}{\isachardot}{\kern0pt}{\isacharless}{\kern0pt}s\isactrlsub {\isadigit{2}}{\isacharbraceright}{\kern0pt}{\isachardot}{\kern0pt}\ {\isacharparenleft}{\kern0pt}{\isadigit{0}}{\isacharcomma}{\kern0pt}{\isadigit{0}}{\isacharparenright}{\kern0pt}{\isacharparenright}{\kern0pt}{\isacharparenright}{\kern0pt}\isanewline
\ \ \ \ {\isacharbraceright}{\kern0pt}{\isachardoublequoteclose}\isanewline
\isanewline
\isacommand{fun}\isamarkupfalse%
\ fk{\isacharunderscore}{\kern0pt}update\ {\isacharcolon}{\kern0pt}{\isacharcolon}{\kern0pt}\ {\isachardoublequoteopen}nat\ {\isasymRightarrow}\ fk{\isacharunderscore}{\kern0pt}state\ {\isasymRightarrow}\ fk{\isacharunderscore}{\kern0pt}state\ pmf{\isachardoublequoteclose}\ \isakeyword{where}\isanewline
\ \ {\isachardoublequoteopen}fk{\isacharunderscore}{\kern0pt}update\ a\ {\isacharparenleft}{\kern0pt}s\isactrlsub {\isadigit{1}}{\isacharcomma}{\kern0pt}\ s\isactrlsub {\isadigit{2}}{\isacharcomma}{\kern0pt}\ k{\isacharcomma}{\kern0pt}\ m{\isacharcomma}{\kern0pt}\ r{\isacharparenright}{\kern0pt}\ {\isacharequal}{\kern0pt}\ \isanewline
\ \ \ \ do\ {\isacharbraceleft}{\kern0pt}\isanewline
\ \ \ \ \ \ coins\ {\isasymleftarrow}\ prod{\isacharunderscore}{\kern0pt}pmf\ {\isacharparenleft}{\kern0pt}{\isacharbraceleft}{\kern0pt}{\isadigit{0}}{\isachardot}{\kern0pt}{\isachardot}{\kern0pt}{\isacharless}{\kern0pt}s\isactrlsub {\isadigit{1}}{\isacharbraceright}{\kern0pt}\ {\isasymtimes}\ {\isacharbraceleft}{\kern0pt}{\isadigit{0}}{\isachardot}{\kern0pt}{\isachardot}{\kern0pt}{\isacharless}{\kern0pt}s\isactrlsub {\isadigit{2}}{\isacharbraceright}{\kern0pt}{\isacharparenright}{\kern0pt}\ {\isacharparenleft}{\kern0pt}{\isasymlambda}{\isacharunderscore}{\kern0pt}{\isachardot}{\kern0pt}\ bernoulli{\isacharunderscore}{\kern0pt}pmf\ {\isacharparenleft}{\kern0pt}{\isadigit{1}}{\isacharslash}{\kern0pt}{\isacharparenleft}{\kern0pt}real\ m{\isacharplus}{\kern0pt}{\isadigit{1}}{\isacharparenright}{\kern0pt}{\isacharparenright}{\kern0pt}{\isacharparenright}{\kern0pt}{\isacharsemicolon}{\kern0pt}\isanewline
\ \ \ \ \ \ return{\isacharunderscore}{\kern0pt}pmf\ {\isacharparenleft}{\kern0pt}s\isactrlsub {\isadigit{1}}{\isacharcomma}{\kern0pt}\ s\isactrlsub {\isadigit{2}}{\isacharcomma}{\kern0pt}\ k{\isacharcomma}{\kern0pt}\ m{\isacharplus}{\kern0pt}{\isadigit{1}}{\isacharcomma}{\kern0pt}\ {\isasymlambda}i\ {\isasymin}\ {\isacharbraceleft}{\kern0pt}{\isadigit{0}}{\isachardot}{\kern0pt}{\isachardot}{\kern0pt}{\isacharless}{\kern0pt}s\isactrlsub {\isadigit{1}}{\isacharbraceright}{\kern0pt}\ {\isasymtimes}\ {\isacharbraceleft}{\kern0pt}{\isadigit{0}}{\isachardot}{\kern0pt}{\isachardot}{\kern0pt}{\isacharless}{\kern0pt}s\isactrlsub {\isadigit{2}}{\isacharbraceright}{\kern0pt}{\isachardot}{\kern0pt}\ \isanewline
\ \ \ \ \ \ \ \ if\ coins\ i\ then\ \isanewline
\ \ \ \ \ \ \ \ \ \ {\isacharparenleft}{\kern0pt}a{\isacharcomma}{\kern0pt}{\isadigit{0}}{\isacharparenright}{\kern0pt}\ \isanewline
\ \ \ \ \ \ \ \ else\ {\isacharparenleft}{\kern0pt}\isanewline
\ \ \ \ \ \ \ \ \ \ let\ {\isacharparenleft}{\kern0pt}x{\isacharcomma}{\kern0pt}l{\isacharparenright}{\kern0pt}\ {\isacharequal}{\kern0pt}\ r\ i\ in\ {\isacharparenleft}{\kern0pt}x{\isacharcomma}{\kern0pt}\ l\ {\isacharplus}{\kern0pt}\ of{\isacharunderscore}{\kern0pt}bool\ {\isacharparenleft}{\kern0pt}x{\isacharequal}{\kern0pt}a{\isacharparenright}{\kern0pt}{\isacharparenright}{\kern0pt}\isanewline
\ \ \ \ \ \ \ \ {\isacharparenright}{\kern0pt}\isanewline
\ \ \ \ \ \ {\isacharparenright}{\kern0pt}\isanewline
\ \ \ \ {\isacharbraceright}{\kern0pt}{\isachardoublequoteclose}\isanewline
\isanewline
\isacommand{fun}\isamarkupfalse%
\ fk{\isacharunderscore}{\kern0pt}result\ {\isacharcolon}{\kern0pt}{\isacharcolon}{\kern0pt}\ {\isachardoublequoteopen}fk{\isacharunderscore}{\kern0pt}state\ {\isasymRightarrow}\ rat\ pmf{\isachardoublequoteclose}\ \isakeyword{where}\isanewline
\ \ {\isachardoublequoteopen}fk{\isacharunderscore}{\kern0pt}result\ {\isacharparenleft}{\kern0pt}s\isactrlsub {\isadigit{1}}{\isacharcomma}{\kern0pt}\ s\isactrlsub {\isadigit{2}}{\isacharcomma}{\kern0pt}\ k{\isacharcomma}{\kern0pt}\ m{\isacharcomma}{\kern0pt}\ r{\isacharparenright}{\kern0pt}\ {\isacharequal}{\kern0pt}\ \isanewline
\ \ \ \ return{\isacharunderscore}{\kern0pt}pmf\ {\isacharparenleft}{\kern0pt}median\ s\isactrlsub {\isadigit{2}}\ {\isacharparenleft}{\kern0pt}{\isasymlambda}i\isactrlsub {\isadigit{2}}\ {\isasymin}\ {\isacharbraceleft}{\kern0pt}{\isadigit{0}}{\isachardot}{\kern0pt}{\isachardot}{\kern0pt}{\isacharless}{\kern0pt}s\isactrlsub {\isadigit{2}}{\isacharbraceright}{\kern0pt}{\isachardot}{\kern0pt}\isanewline
\ \ \ \ \ \ {\isacharparenleft}{\kern0pt}{\isasymSum}i\isactrlsub {\isadigit{1}}{\isasymin}{\isacharbraceleft}{\kern0pt}{\isadigit{0}}{\isachardot}{\kern0pt}{\isachardot}{\kern0pt}{\isacharless}{\kern0pt}s\isactrlsub {\isadigit{1}}{\isacharbraceright}{\kern0pt}\ {\isachardot}{\kern0pt}\ rat{\isacharunderscore}{\kern0pt}of{\isacharunderscore}{\kern0pt}nat\ {\isacharparenleft}{\kern0pt}let\ t\ {\isacharequal}{\kern0pt}\ snd\ {\isacharparenleft}{\kern0pt}r\ {\isacharparenleft}{\kern0pt}i\isactrlsub {\isadigit{1}}{\isacharcomma}{\kern0pt}\ i\isactrlsub {\isadigit{2}}{\isacharparenright}{\kern0pt}{\isacharparenright}{\kern0pt}\ {\isacharplus}{\kern0pt}\ {\isadigit{1}}\ in\ m\ {\isacharasterisk}{\kern0pt}\ {\isacharparenleft}{\kern0pt}t{\isacharcircum}{\kern0pt}k\ {\isacharminus}{\kern0pt}\ {\isacharparenleft}{\kern0pt}t\ {\isacharminus}{\kern0pt}\ {\isadigit{1}}{\isacharparenright}{\kern0pt}{\isacharcircum}{\kern0pt}k{\isacharparenright}{\kern0pt}{\isacharparenright}{\kern0pt}{\isacharparenright}{\kern0pt}\ {\isacharslash}{\kern0pt}\ {\isacharparenleft}{\kern0pt}rat{\isacharunderscore}{\kern0pt}of{\isacharunderscore}{\kern0pt}nat\ s\isactrlsub {\isadigit{1}}{\isacharparenright}{\kern0pt}{\isacharparenright}{\kern0pt}\isanewline
\ \ \ \ {\isacharparenright}{\kern0pt}{\isachardoublequoteclose}\isanewline
\isanewline
\isacommand{fun}\isamarkupfalse%
\ fk{\isacharunderscore}{\kern0pt}update{\isacharprime}{\kern0pt}\ {\isacharcolon}{\kern0pt}{\isacharcolon}{\kern0pt}\ {\isachardoublequoteopen}{\isacharprime}{\kern0pt}a\ {\isasymRightarrow}\ nat\ {\isasymRightarrow}\ nat\ {\isasymRightarrow}\ nat\ {\isasymRightarrow}\ {\isacharparenleft}{\kern0pt}nat\ {\isasymtimes}\ nat\ {\isasymRightarrow}\ {\isacharparenleft}{\kern0pt}{\isacharprime}{\kern0pt}a\ {\isasymtimes}\ nat{\isacharparenright}{\kern0pt}{\isacharparenright}{\kern0pt}\ {\isasymRightarrow}\ {\isacharparenleft}{\kern0pt}nat\ {\isasymtimes}\ nat\ {\isasymRightarrow}\ {\isacharparenleft}{\kern0pt}{\isacharprime}{\kern0pt}a\ {\isasymtimes}\ nat{\isacharparenright}{\kern0pt}{\isacharparenright}{\kern0pt}\ pmf{\isachardoublequoteclose}\ \isakeyword{where}\isanewline
\ \ {\isachardoublequoteopen}fk{\isacharunderscore}{\kern0pt}update{\isacharprime}{\kern0pt}\ a\ s\isactrlsub {\isadigit{1}}\ s\isactrlsub {\isadigit{2}}\ m\ r\ {\isacharequal}{\kern0pt}\ \isanewline
\ \ \ \ do\ {\isacharbraceleft}{\kern0pt}\isanewline
\ \ \ \ \ \ coins\ {\isasymleftarrow}\ prod{\isacharunderscore}{\kern0pt}pmf\ {\isacharparenleft}{\kern0pt}{\isacharbraceleft}{\kern0pt}{\isadigit{0}}{\isachardot}{\kern0pt}{\isachardot}{\kern0pt}{\isacharless}{\kern0pt}s\isactrlsub {\isadigit{1}}{\isacharbraceright}{\kern0pt}\ {\isasymtimes}\ {\isacharbraceleft}{\kern0pt}{\isadigit{0}}{\isachardot}{\kern0pt}{\isachardot}{\kern0pt}{\isacharless}{\kern0pt}s\isactrlsub {\isadigit{2}}{\isacharbraceright}{\kern0pt}{\isacharparenright}{\kern0pt}\ {\isacharparenleft}{\kern0pt}{\isasymlambda}{\isacharunderscore}{\kern0pt}{\isachardot}{\kern0pt}\ bernoulli{\isacharunderscore}{\kern0pt}pmf\ {\isacharparenleft}{\kern0pt}{\isadigit{1}}{\isacharslash}{\kern0pt}{\isacharparenleft}{\kern0pt}real\ m{\isacharplus}{\kern0pt}{\isadigit{1}}{\isacharparenright}{\kern0pt}{\isacharparenright}{\kern0pt}{\isacharparenright}{\kern0pt}{\isacharsemicolon}{\kern0pt}\isanewline
\ \ \ \ \ \ return{\isacharunderscore}{\kern0pt}pmf\ {\isacharparenleft}{\kern0pt}{\isasymlambda}i\ {\isasymin}\ {\isacharbraceleft}{\kern0pt}{\isadigit{0}}{\isachardot}{\kern0pt}{\isachardot}{\kern0pt}{\isacharless}{\kern0pt}s\isactrlsub {\isadigit{1}}{\isacharbraceright}{\kern0pt}\ {\isasymtimes}\ {\isacharbraceleft}{\kern0pt}{\isadigit{0}}{\isachardot}{\kern0pt}{\isachardot}{\kern0pt}{\isacharless}{\kern0pt}s\isactrlsub {\isadigit{2}}{\isacharbraceright}{\kern0pt}{\isachardot}{\kern0pt}\ \isanewline
\ \ \ \ \ \ \ \ if\ coins\ i\ then\ \isanewline
\ \ \ \ \ \ \ \ \ \ {\isacharparenleft}{\kern0pt}a{\isacharcomma}{\kern0pt}{\isadigit{0}}{\isacharparenright}{\kern0pt}\ \isanewline
\ \ \ \ \ \ \ \ else\ {\isacharparenleft}{\kern0pt}\isanewline
\ \ \ \ \ \ \ \ \ \ let\ {\isacharparenleft}{\kern0pt}x{\isacharcomma}{\kern0pt}l{\isacharparenright}{\kern0pt}\ {\isacharequal}{\kern0pt}\ r\ i\ in\ {\isacharparenleft}{\kern0pt}x{\isacharcomma}{\kern0pt}\ l\ {\isacharplus}{\kern0pt}\ of{\isacharunderscore}{\kern0pt}bool\ {\isacharparenleft}{\kern0pt}x{\isacharequal}{\kern0pt}a{\isacharparenright}{\kern0pt}{\isacharparenright}{\kern0pt}\isanewline
\ \ \ \ \ \ \ \ {\isacharparenright}{\kern0pt}\isanewline
\ \ \ \ \ \ {\isacharparenright}{\kern0pt}\isanewline
\ \ \ \ {\isacharbraceright}{\kern0pt}{\isachardoublequoteclose}\isanewline
\isanewline
\isacommand{fun}\isamarkupfalse%
\ fk{\isacharunderscore}{\kern0pt}update{\isacharprime}{\kern0pt}{\isacharprime}{\kern0pt}\ {\isacharcolon}{\kern0pt}{\isacharcolon}{\kern0pt}\ {\isachardoublequoteopen}{\isacharprime}{\kern0pt}a\ {\isasymRightarrow}\ nat\ {\isasymRightarrow}\ {\isacharparenleft}{\kern0pt}{\isacharprime}{\kern0pt}a\ {\isasymtimes}\ nat{\isacharparenright}{\kern0pt}\ {\isasymRightarrow}\ {\isacharparenleft}{\kern0pt}{\isacharparenleft}{\kern0pt}{\isacharprime}{\kern0pt}a\ {\isasymtimes}\ nat{\isacharparenright}{\kern0pt}{\isacharparenright}{\kern0pt}\ pmf{\isachardoublequoteclose}\ \isakeyword{where}\isanewline
\ \ {\isachardoublequoteopen}fk{\isacharunderscore}{\kern0pt}update{\isacharprime}{\kern0pt}{\isacharprime}{\kern0pt}\ a\ m\ {\isacharparenleft}{\kern0pt}x{\isacharcomma}{\kern0pt}l{\isacharparenright}{\kern0pt}\ {\isacharequal}{\kern0pt}\ \isanewline
\ \ \ \ do\ {\isacharbraceleft}{\kern0pt}\isanewline
\ \ \ \ \ \ coin\ {\isasymleftarrow}\ bernoulli{\isacharunderscore}{\kern0pt}pmf\ {\isacharparenleft}{\kern0pt}{\isadigit{1}}{\isacharslash}{\kern0pt}{\isacharparenleft}{\kern0pt}real\ m{\isacharplus}{\kern0pt}{\isadigit{1}}{\isacharparenright}{\kern0pt}{\isacharparenright}{\kern0pt}{\isacharsemicolon}{\kern0pt}\isanewline
\ \ \ \ \ \ return{\isacharunderscore}{\kern0pt}pmf\ {\isacharparenleft}{\kern0pt}\ \isanewline
\ \ \ \ \ \ \ \ if\ coin\ then\ \isanewline
\ \ \ \ \ \ \ \ \ \ {\isacharparenleft}{\kern0pt}a{\isacharcomma}{\kern0pt}{\isadigit{0}}{\isacharparenright}{\kern0pt}\ \isanewline
\ \ \ \ \ \ \ \ else\ {\isacharparenleft}{\kern0pt}\isanewline
\ \ \ \ \ \ \ \ \ \ {\isacharparenleft}{\kern0pt}x{\isacharcomma}{\kern0pt}\ l\ {\isacharplus}{\kern0pt}\ of{\isacharunderscore}{\kern0pt}bool\ {\isacharparenleft}{\kern0pt}x{\isacharequal}{\kern0pt}a{\isacharparenright}{\kern0pt}{\isacharparenright}{\kern0pt}\isanewline
\ \ \ \ \ \ \ \ {\isacharparenright}{\kern0pt}\isanewline
\ \ \ \ \ \ {\isacharparenright}{\kern0pt}\isanewline
\ \ \ \ {\isacharbraceright}{\kern0pt}{\isachardoublequoteclose}\isanewline
\isanewline
\isacommand{lemma}\isamarkupfalse%
\ bernoulli{\isacharunderscore}{\kern0pt}pmf{\isacharunderscore}{\kern0pt}{\isadigit{1}}{\isacharcolon}{\kern0pt}\ {\isachardoublequoteopen}bernoulli{\isacharunderscore}{\kern0pt}pmf\ {\isadigit{1}}\ {\isacharequal}{\kern0pt}\ return{\isacharunderscore}{\kern0pt}pmf\ True{\isachardoublequoteclose}\isanewline
%
\isadelimproof
\ \ \ \ %
\endisadelimproof
%
\isatagproof
\isacommand{by}\isamarkupfalse%
\ {\isacharparenleft}{\kern0pt}rule\ pmf{\isacharunderscore}{\kern0pt}eqI{\isacharcomma}{\kern0pt}\ simp\ add{\isacharcolon}{\kern0pt}indicator{\isacharunderscore}{\kern0pt}def{\isacharparenright}{\kern0pt}%
\endisatagproof
{\isafoldproof}%
%
\isadelimproof
\isanewline
%
\endisadelimproof
\isanewline
\isacommand{lemma}\isamarkupfalse%
\ split{\isacharunderscore}{\kern0pt}space{\isacharcolon}{\kern0pt}\isanewline
\ \ {\isachardoublequoteopen}{\isacharparenleft}{\kern0pt}{\isasymSum}a{\isasymin}{\isacharbraceleft}{\kern0pt}{\isacharparenleft}{\kern0pt}u{\isacharcomma}{\kern0pt}\ v{\isacharparenright}{\kern0pt}{\isachardot}{\kern0pt}\ v\ {\isacharless}{\kern0pt}\ count{\isacharunderscore}{\kern0pt}list\ as\ u{\isacharbraceright}{\kern0pt}{\isachardot}{\kern0pt}\ {\isacharparenleft}{\kern0pt}f\ {\isacharparenleft}{\kern0pt}snd\ a{\isacharparenright}{\kern0pt}{\isacharparenright}{\kern0pt}{\isacharparenright}{\kern0pt}\ {\isacharequal}{\kern0pt}\ \isanewline
\ \ {\isacharparenleft}{\kern0pt}{\isasymSum}u\ {\isasymin}\ set\ as{\isachardot}{\kern0pt}\ {\isacharparenleft}{\kern0pt}{\isasymSum}v\ {\isasymin}{\isacharbraceleft}{\kern0pt}{\isadigit{0}}{\isachardot}{\kern0pt}{\isachardot}{\kern0pt}{\isacharless}{\kern0pt}count{\isacharunderscore}{\kern0pt}list\ as\ u{\isacharbraceright}{\kern0pt}{\isachardot}{\kern0pt}\ {\isacharparenleft}{\kern0pt}f\ v{\isacharparenright}{\kern0pt}{\isacharparenright}{\kern0pt}{\isacharparenright}{\kern0pt}{\isachardoublequoteclose}\ {\isacharparenleft}{\kern0pt}\isakeyword{is}\ {\isachardoublequoteopen}{\isacharquery}{\kern0pt}lhs\ {\isacharequal}{\kern0pt}\ {\isacharquery}{\kern0pt}rhs{\isachardoublequoteclose}{\isacharparenright}{\kern0pt}\isanewline
%
\isadelimproof
%
\endisadelimproof
%
\isatagproof
\isacommand{proof}\isamarkupfalse%
\ {\isacharminus}{\kern0pt}\isanewline
\ \ \isacommand{define}\isamarkupfalse%
\ A\ \isakeyword{where}\ {\isachardoublequoteopen}A\ {\isacharequal}{\kern0pt}\ {\isacharparenleft}{\kern0pt}{\isasymlambda}u{\isachardot}{\kern0pt}\ {\isacharbraceleft}{\kern0pt}u{\isacharbraceright}{\kern0pt}\ {\isasymtimes}\ {\isacharbraceleft}{\kern0pt}v{\isachardot}{\kern0pt}\ v\ {\isacharless}{\kern0pt}\ count{\isacharunderscore}{\kern0pt}list\ as\ u{\isacharbraceright}{\kern0pt}{\isacharparenright}{\kern0pt}{\isachardoublequoteclose}\isanewline
\isanewline
\ \ \isacommand{have}\isamarkupfalse%
\ a\ {\isacharcolon}{\kern0pt}{\isachardoublequoteopen}{\isasymAnd}u\ v{\isachardot}{\kern0pt}\ u\ {\isacharless}{\kern0pt}\ count{\isacharunderscore}{\kern0pt}list\ as\ v\ {\isasymLongrightarrow}\ v\ {\isasymin}\ set\ as{\isachardoublequoteclose}\ \isanewline
\ \ \ \ \isacommand{by}\isamarkupfalse%
\ {\isacharparenleft}{\kern0pt}subst\ count{\isacharunderscore}{\kern0pt}list{\isacharunderscore}{\kern0pt}gr{\isacharunderscore}{\kern0pt}{\isadigit{1}}{\isacharcomma}{\kern0pt}\ force{\isacharparenright}{\kern0pt}\isanewline
\isanewline
\ \ \isacommand{have}\isamarkupfalse%
\ {\isachardoublequoteopen}{\isacharquery}{\kern0pt}lhs\ {\isacharequal}{\kern0pt}\ sum\ {\isacharparenleft}{\kern0pt}f\ {\isasymcirc}\ snd{\isacharparenright}{\kern0pt}\ \ {\isacharparenleft}{\kern0pt}{\isasymUnion}\ {\isacharparenleft}{\kern0pt}A\ {\isacharbackquote}{\kern0pt}\ set\ as{\isacharparenright}{\kern0pt}{\isacharparenright}{\kern0pt}{\isachardoublequoteclose}\isanewline
\ \ \ \ \isacommand{apply}\isamarkupfalse%
\ {\isacharparenleft}{\kern0pt}rule\ sum{\isachardot}{\kern0pt}cong{\isacharcomma}{\kern0pt}\ rule\ order{\isacharunderscore}{\kern0pt}antisym{\isacharparenright}{\kern0pt}\isanewline
\ \ \ \ \isacommand{apply}\isamarkupfalse%
\ {\isacharparenleft}{\kern0pt}rule\ subsetI{\isacharcomma}{\kern0pt}\ simp\ add{\isacharcolon}{\kern0pt}A{\isacharunderscore}{\kern0pt}def\ case{\isacharunderscore}{\kern0pt}prod{\isacharunderscore}{\kern0pt}beta{\isacharprime}{\kern0pt}\ mem{\isacharunderscore}{\kern0pt}Times{\isacharunderscore}{\kern0pt}iff\ a{\isacharparenright}{\kern0pt}\isanewline
\ \ \ \ \isacommand{apply}\isamarkupfalse%
\ {\isacharparenleft}{\kern0pt}rule\ subsetI{\isacharcomma}{\kern0pt}\ simp\ add{\isacharcolon}{\kern0pt}A{\isacharunderscore}{\kern0pt}def\ case{\isacharunderscore}{\kern0pt}prod{\isacharunderscore}{\kern0pt}beta{\isacharprime}{\kern0pt}\ mem{\isacharunderscore}{\kern0pt}Times{\isacharunderscore}{\kern0pt}iff\ a{\isacharparenright}{\kern0pt}\isanewline
\ \ \ \ \isacommand{by}\isamarkupfalse%
\ simp\isanewline
\ \ \isacommand{also}\isamarkupfalse%
\ \isacommand{have}\isamarkupfalse%
\ {\isachardoublequoteopen}{\isachardot}{\kern0pt}{\isachardot}{\kern0pt}{\isachardot}{\kern0pt}\ {\isacharequal}{\kern0pt}\ sum\ {\isacharparenleft}{\kern0pt}{\isasymlambda}x{\isachardot}{\kern0pt}\ sum\ {\isacharparenleft}{\kern0pt}f\ {\isasymcirc}\ snd{\isacharparenright}{\kern0pt}\ {\isacharparenleft}{\kern0pt}A\ x{\isacharparenright}{\kern0pt}{\isacharparenright}{\kern0pt}\ {\isacharparenleft}{\kern0pt}set\ as{\isacharparenright}{\kern0pt}{\isachardoublequoteclose}\isanewline
\ \ \ \ \isacommand{by}\isamarkupfalse%
\ {\isacharparenleft}{\kern0pt}rule\ sum{\isachardot}{\kern0pt}UNION{\isacharunderscore}{\kern0pt}disjoint{\isacharcomma}{\kern0pt}\ simp{\isacharcomma}{\kern0pt}\ simp\ add{\isacharcolon}{\kern0pt}A{\isacharunderscore}{\kern0pt}def{\isacharcomma}{\kern0pt}\ simp\ add{\isacharcolon}{\kern0pt}A{\isacharunderscore}{\kern0pt}def{\isacharcomma}{\kern0pt}\ blast{\isacharparenright}{\kern0pt}\ \isanewline
\ \ \isacommand{also}\isamarkupfalse%
\ \isacommand{have}\isamarkupfalse%
\ {\isachardoublequoteopen}{\isachardot}{\kern0pt}{\isachardot}{\kern0pt}{\isachardot}{\kern0pt}\ {\isacharequal}{\kern0pt}\ {\isacharquery}{\kern0pt}rhs{\isachardoublequoteclose}\isanewline
\ \ \ \ \isacommand{apply}\isamarkupfalse%
\ {\isacharparenleft}{\kern0pt}rule\ sum{\isachardot}{\kern0pt}cong{\isacharcomma}{\kern0pt}\ simp{\isacharparenright}{\kern0pt}\isanewline
\ \ \ \ \isacommand{apply}\isamarkupfalse%
\ {\isacharparenleft}{\kern0pt}subst\ sum{\isachardot}{\kern0pt}reindex{\isacharbrackleft}{\kern0pt}symmetric{\isacharbrackright}{\kern0pt}{\isacharparenright}{\kern0pt}\isanewline
\ \ \ \ \ \isacommand{apply}\isamarkupfalse%
\ {\isacharparenleft}{\kern0pt}simp\ add{\isacharcolon}{\kern0pt}A{\isacharunderscore}{\kern0pt}def\ inj{\isacharunderscore}{\kern0pt}on{\isacharunderscore}{\kern0pt}def{\isacharparenright}{\kern0pt}\ \isanewline
\ \ \ \ \isacommand{apply}\isamarkupfalse%
\ {\isacharparenleft}{\kern0pt}simp\ add{\isacharcolon}{\kern0pt}A{\isacharunderscore}{\kern0pt}def{\isacharparenright}{\kern0pt}\isanewline
\ \ \ \ \isacommand{apply}\isamarkupfalse%
\ {\isacharparenleft}{\kern0pt}rule\ sum{\isachardot}{\kern0pt}cong{\isacharparenright}{\kern0pt}\isanewline
\ \ \ \ \isacommand{using}\isamarkupfalse%
\ lessThan{\isacharunderscore}{\kern0pt}atLeast{\isadigit{0}}\ \isacommand{apply}\isamarkupfalse%
\ blast\isanewline
\ \ \ \ \isacommand{by}\isamarkupfalse%
\ simp\isanewline
\ \ \isacommand{finally}\isamarkupfalse%
\ \isacommand{show}\isamarkupfalse%
\ {\isacharquery}{\kern0pt}thesis\ \isacommand{by}\isamarkupfalse%
\ blast\isanewline
\isacommand{qed}\isamarkupfalse%
%
\endisatagproof
{\isafoldproof}%
%
\isadelimproof
\isanewline
%
\endisadelimproof
\isanewline
\isacommand{lemma}\isamarkupfalse%
\isanewline
\ \ \isakeyword{assumes}\ {\isachardoublequoteopen}as\ {\isasymnoteq}\ {\isacharbrackleft}{\kern0pt}{\isacharbrackright}{\kern0pt}{\isachardoublequoteclose}\isanewline
\ \ \isakeyword{shows}\ fin{\isacharunderscore}{\kern0pt}space{\isacharcolon}{\kern0pt}\ {\isachardoublequoteopen}finite\ {\isacharbraceleft}{\kern0pt}{\isacharparenleft}{\kern0pt}u{\isacharcomma}{\kern0pt}\ v{\isacharparenright}{\kern0pt}{\isachardot}{\kern0pt}\ v\ {\isacharless}{\kern0pt}\ count{\isacharunderscore}{\kern0pt}list\ as\ u{\isacharbraceright}{\kern0pt}{\isachardoublequoteclose}\ \isakeyword{and}\isanewline
\ \ non{\isacharunderscore}{\kern0pt}empty{\isacharunderscore}{\kern0pt}space{\isacharcolon}{\kern0pt}\ {\isachardoublequoteopen}{\isacharbraceleft}{\kern0pt}{\isacharparenleft}{\kern0pt}u{\isacharcomma}{\kern0pt}\ v{\isacharparenright}{\kern0pt}{\isachardot}{\kern0pt}\ v\ {\isacharless}{\kern0pt}\ count{\isacharunderscore}{\kern0pt}list\ as\ u{\isacharbraceright}{\kern0pt}\ {\isasymnoteq}\ {\isacharbraceleft}{\kern0pt}{\isacharbraceright}{\kern0pt}{\isachardoublequoteclose}\ \isakeyword{and}\isanewline
\ \ card{\isacharunderscore}{\kern0pt}space{\isacharcolon}{\kern0pt}\ {\isachardoublequoteopen}card\ {\isacharbraceleft}{\kern0pt}{\isacharparenleft}{\kern0pt}u{\isacharcomma}{\kern0pt}\ v{\isacharparenright}{\kern0pt}{\isachardot}{\kern0pt}\ v\ {\isacharless}{\kern0pt}\ count{\isacharunderscore}{\kern0pt}list\ as\ u{\isacharbraceright}{\kern0pt}\ {\isacharequal}{\kern0pt}\ length\ as{\isachardoublequoteclose}\isanewline
%
\isadelimproof
%
\endisadelimproof
%
\isatagproof
\isacommand{proof}\isamarkupfalse%
\ {\isacharminus}{\kern0pt}\isanewline
\ \ \isacommand{have}\isamarkupfalse%
\ {\isachardoublequoteopen}{\isacharbraceleft}{\kern0pt}{\isacharparenleft}{\kern0pt}u{\isacharcomma}{\kern0pt}\ v{\isacharparenright}{\kern0pt}{\isachardot}{\kern0pt}\ v\ {\isacharless}{\kern0pt}\ count{\isacharunderscore}{\kern0pt}list\ as\ u{\isacharbraceright}{\kern0pt}\ {\isasymsubseteq}\ set\ as\ {\isasymtimes}\ {\isacharbraceleft}{\kern0pt}k{\isachardot}{\kern0pt}\ k\ {\isacharless}{\kern0pt}\ length\ as{\isacharbraceright}{\kern0pt}{\isachardoublequoteclose}\isanewline
\ \ \ \ \isacommand{apply}\isamarkupfalse%
\ {\isacharparenleft}{\kern0pt}rule\ subsetI{\isacharcomma}{\kern0pt}\ simp\ add{\isacharcolon}{\kern0pt}case{\isacharunderscore}{\kern0pt}prod{\isacharunderscore}{\kern0pt}beta\ mem{\isacharunderscore}{\kern0pt}Times{\isacharunderscore}{\kern0pt}iff\ count{\isacharunderscore}{\kern0pt}list{\isacharunderscore}{\kern0pt}gr{\isacharunderscore}{\kern0pt}{\isadigit{1}}{\isacharparenright}{\kern0pt}\isanewline
\ \ \ \ \isacommand{by}\isamarkupfalse%
\ {\isacharparenleft}{\kern0pt}metis\ count{\isacharunderscore}{\kern0pt}le{\isacharunderscore}{\kern0pt}length\ order{\isacharunderscore}{\kern0pt}less{\isacharunderscore}{\kern0pt}le{\isacharunderscore}{\kern0pt}trans{\isacharparenright}{\kern0pt}\isanewline
\isanewline
\ \ \isacommand{thus}\isamarkupfalse%
\ fin{\isacharunderscore}{\kern0pt}space{\isacharcolon}{\kern0pt}\ {\isachardoublequoteopen}finite\ \ {\isacharbraceleft}{\kern0pt}{\isacharparenleft}{\kern0pt}u{\isacharcomma}{\kern0pt}\ v{\isacharparenright}{\kern0pt}{\isachardot}{\kern0pt}\ v\ {\isacharless}{\kern0pt}\ count{\isacharunderscore}{\kern0pt}list\ as\ u{\isacharbraceright}{\kern0pt}{\isachardoublequoteclose}\isanewline
\ \ \ \ \isacommand{using}\isamarkupfalse%
\ finite{\isacharunderscore}{\kern0pt}subset\ \isacommand{by}\isamarkupfalse%
\ blast\isanewline
\isanewline
\ \ \isacommand{have}\isamarkupfalse%
\ {\isachardoublequoteopen}{\isacharparenleft}{\kern0pt}as\ {\isacharbang}{\kern0pt}\ {\isadigit{0}}{\isacharcomma}{\kern0pt}\ {\isadigit{0}}{\isacharparenright}{\kern0pt}\ {\isasymin}\ {\isacharbraceleft}{\kern0pt}{\isacharparenleft}{\kern0pt}u{\isacharcomma}{\kern0pt}\ v{\isacharparenright}{\kern0pt}{\isachardot}{\kern0pt}\ v\ {\isacharless}{\kern0pt}\ count{\isacharunderscore}{\kern0pt}list\ as\ u{\isacharbraceright}{\kern0pt}{\isachardoublequoteclose}\ \isanewline
\ \ \ \ \isacommand{apply}\isamarkupfalse%
\ {\isacharparenleft}{\kern0pt}simp{\isacharparenright}{\kern0pt}\isanewline
\ \ \ \ \isacommand{using}\isamarkupfalse%
\ assms{\isacharparenleft}{\kern0pt}{\isadigit{1}}{\isacharparenright}{\kern0pt}\ \isanewline
\ \ \ \ \isacommand{by}\isamarkupfalse%
\ {\isacharparenleft}{\kern0pt}metis\ count{\isacharunderscore}{\kern0pt}list{\isacharunderscore}{\kern0pt}gr{\isacharunderscore}{\kern0pt}{\isadigit{1}}\ gr{\isadigit{0}}I\ length{\isacharunderscore}{\kern0pt}greater{\isacharunderscore}{\kern0pt}{\isadigit{0}}{\isacharunderscore}{\kern0pt}conv\ not{\isacharunderscore}{\kern0pt}one{\isacharunderscore}{\kern0pt}le{\isacharunderscore}{\kern0pt}zero\ nth{\isacharunderscore}{\kern0pt}mem{\isacharparenright}{\kern0pt}\isanewline
\ \ \isacommand{thus}\isamarkupfalse%
\ {\isachardoublequoteopen}{\isacharbraceleft}{\kern0pt}{\isacharparenleft}{\kern0pt}u{\isacharcomma}{\kern0pt}\ v{\isacharparenright}{\kern0pt}{\isachardot}{\kern0pt}\ v\ {\isacharless}{\kern0pt}\ count{\isacharunderscore}{\kern0pt}list\ as\ u{\isacharbraceright}{\kern0pt}\ {\isasymnoteq}\ {\isacharbraceleft}{\kern0pt}{\isacharbraceright}{\kern0pt}{\isachardoublequoteclose}\ \isacommand{by}\isamarkupfalse%
\ blast\isanewline
\isanewline
\ \ \isacommand{show}\isamarkupfalse%
\ {\isachardoublequoteopen}card\ {\isacharbraceleft}{\kern0pt}{\isacharparenleft}{\kern0pt}u{\isacharcomma}{\kern0pt}\ v{\isacharparenright}{\kern0pt}{\isachardot}{\kern0pt}\ v\ {\isacharless}{\kern0pt}\ count{\isacharunderscore}{\kern0pt}list\ as\ u{\isacharbraceright}{\kern0pt}\ {\isacharequal}{\kern0pt}\ length\ as{\isachardoublequoteclose}\isanewline
\ \ \ \ \isacommand{using}\isamarkupfalse%
\ fin{\isacharunderscore}{\kern0pt}space\ split{\isacharunderscore}{\kern0pt}space{\isacharbrackleft}{\kern0pt}\isakeyword{where}\ f{\isacharequal}{\kern0pt}{\isachardoublequoteopen}{\isasymlambda}{\isacharunderscore}{\kern0pt}{\isachardot}{\kern0pt}\ {\isacharparenleft}{\kern0pt}{\isadigit{1}}{\isacharcolon}{\kern0pt}{\isacharcolon}{\kern0pt}nat{\isacharparenright}{\kern0pt}{\isachardoublequoteclose}{\isacharcomma}{\kern0pt}\ \isakeyword{where}\ as{\isacharequal}{\kern0pt}{\isachardoublequoteopen}as{\isachardoublequoteclose}{\isacharbrackright}{\kern0pt}\isanewline
\ \ \ \ \isacommand{by}\isamarkupfalse%
\ {\isacharparenleft}{\kern0pt}simp\ add{\isacharcolon}{\kern0pt}sum{\isacharunderscore}{\kern0pt}count{\isacharunderscore}{\kern0pt}set{\isacharbrackleft}{\kern0pt}\isakeyword{where}\ X{\isacharequal}{\kern0pt}{\isachardoublequoteopen}set\ as{\isachardoublequoteclose}\ \isakeyword{and}\ xs{\isacharequal}{\kern0pt}{\isachardoublequoteopen}as{\isachardoublequoteclose}{\isacharcomma}{\kern0pt}\ simplified{\isacharbrackright}{\kern0pt}{\isacharparenright}{\kern0pt}\isanewline
\isacommand{qed}\isamarkupfalse%
%
\endisatagproof
{\isafoldproof}%
%
\isadelimproof
\isanewline
%
\endisadelimproof
\isanewline
\isacommand{lemma}\isamarkupfalse%
\ fk{\isacharunderscore}{\kern0pt}alg{\isacharunderscore}{\kern0pt}aux{\isacharunderscore}{\kern0pt}{\isadigit{5}}{\isacharcolon}{\kern0pt}\isanewline
\ \ \isakeyword{assumes}\ {\isachardoublequoteopen}as\ {\isasymnoteq}\ {\isacharbrackleft}{\kern0pt}{\isacharbrackright}{\kern0pt}{\isachardoublequoteclose}\isanewline
\ \ \isakeyword{shows}\ {\isachardoublequoteopen}pmf{\isacharunderscore}{\kern0pt}of{\isacharunderscore}{\kern0pt}set\ {\isacharbraceleft}{\kern0pt}k{\isachardot}{\kern0pt}\ k\ {\isacharless}{\kern0pt}\ length\ as{\isacharbraceright}{\kern0pt}\ {\isasymbind}\ {\isacharparenleft}{\kern0pt}{\isasymlambda}k{\isachardot}{\kern0pt}\ return{\isacharunderscore}{\kern0pt}pmf\ {\isacharparenleft}{\kern0pt}as\ {\isacharbang}{\kern0pt}\ k{\isacharcomma}{\kern0pt}\ count{\isacharunderscore}{\kern0pt}list\ {\isacharparenleft}{\kern0pt}drop\ {\isacharparenleft}{\kern0pt}k{\isacharplus}{\kern0pt}{\isadigit{1}}{\isacharparenright}{\kern0pt}\ as{\isacharparenright}{\kern0pt}\ {\isacharparenleft}{\kern0pt}as\ {\isacharbang}{\kern0pt}\ k{\isacharparenright}{\kern0pt}{\isacharparenright}{\kern0pt}{\isacharparenright}{\kern0pt}\isanewline
\ \ {\isacharequal}{\kern0pt}\ pmf{\isacharunderscore}{\kern0pt}of{\isacharunderscore}{\kern0pt}set\ {\isacharbraceleft}{\kern0pt}{\isacharparenleft}{\kern0pt}u{\isacharcomma}{\kern0pt}v{\isacharparenright}{\kern0pt}{\isachardot}{\kern0pt}\ v\ {\isacharless}{\kern0pt}\ count{\isacharunderscore}{\kern0pt}list\ as\ u{\isacharbraceright}{\kern0pt}{\isachardoublequoteclose}\isanewline
%
\isadelimproof
%
\endisadelimproof
%
\isatagproof
\isacommand{proof}\isamarkupfalse%
\ {\isacharminus}{\kern0pt}\isanewline
\ \ \isacommand{define}\isamarkupfalse%
\ f\ \isakeyword{where}\ {\isachardoublequoteopen}f\ {\isacharequal}{\kern0pt}\ {\isacharparenleft}{\kern0pt}{\isasymlambda}k{\isachardot}{\kern0pt}\ {\isacharparenleft}{\kern0pt}as\ {\isacharbang}{\kern0pt}\ k{\isacharcomma}{\kern0pt}\ count{\isacharunderscore}{\kern0pt}list\ {\isacharparenleft}{\kern0pt}drop\ {\isacharparenleft}{\kern0pt}k{\isacharplus}{\kern0pt}{\isadigit{1}}{\isacharparenright}{\kern0pt}\ as{\isacharparenright}{\kern0pt}\ {\isacharparenleft}{\kern0pt}as\ {\isacharbang}{\kern0pt}\ k{\isacharparenright}{\kern0pt}{\isacharparenright}{\kern0pt}{\isacharparenright}{\kern0pt}{\isachardoublequoteclose}\isanewline
\isanewline
\ \ \isacommand{have}\isamarkupfalse%
\ a{\isadigit{3}}{\isacharcolon}{\kern0pt}\ {\isachardoublequoteopen}{\isasymAnd}x\ y{\isachardot}{\kern0pt}\ y\ {\isacharless}{\kern0pt}\ length\ as\ {\isasymLongrightarrow}\ x\ {\isacharless}{\kern0pt}\ y\ {\isasymLongrightarrow}\ as\ {\isacharbang}{\kern0pt}\ x\ {\isacharequal}{\kern0pt}\ as\ {\isacharbang}{\kern0pt}\ y\ {\isasymLongrightarrow}\isanewline
\ \ \ \ \ \ \ \ \ \ \ count{\isacharunderscore}{\kern0pt}list\ {\isacharparenleft}{\kern0pt}drop\ {\isacharparenleft}{\kern0pt}Suc\ x{\isacharparenright}{\kern0pt}\ as{\isacharparenright}{\kern0pt}\ {\isacharparenleft}{\kern0pt}as\ {\isacharbang}{\kern0pt}\ x{\isacharparenright}{\kern0pt}\ {\isasymnoteq}\ count{\isacharunderscore}{\kern0pt}list\ {\isacharparenleft}{\kern0pt}drop\ {\isacharparenleft}{\kern0pt}Suc\ y{\isacharparenright}{\kern0pt}\ as{\isacharparenright}{\kern0pt}\ {\isacharparenleft}{\kern0pt}as\ {\isacharbang}{\kern0pt}\ y{\isacharparenright}{\kern0pt}{\isachardoublequoteclose}\ \isanewline
\ \ \ \ {\isacharparenleft}{\kern0pt}\isakeyword{is}\ {\isachardoublequoteopen}{\isasymAnd}x\ y{\isachardot}{\kern0pt}\ {\isacharunderscore}{\kern0pt}\ {\isasymLongrightarrow}\ {\isacharunderscore}{\kern0pt}\ {\isasymLongrightarrow}\ {\isacharunderscore}{\kern0pt}\ {\isasymLongrightarrow}\ {\isacharquery}{\kern0pt}ths\ x\ y{\isachardoublequoteclose}{\isacharparenright}{\kern0pt}\isanewline
\ \ \isacommand{proof}\isamarkupfalse%
\ {\isacharminus}{\kern0pt}\isanewline
\ \ \ \ \isacommand{fix}\isamarkupfalse%
\ x\ y\isanewline
\ \ \ \ \isacommand{assume}\isamarkupfalse%
\ a{\isadigit{3}}{\isacharunderscore}{\kern0pt}{\isadigit{1}}{\isacharcolon}{\kern0pt}\ {\isachardoublequoteopen}y\ {\isacharless}{\kern0pt}\ length\ as{\isachardoublequoteclose}\isanewline
\ \ \ \ \isacommand{assume}\isamarkupfalse%
\ a{\isadigit{3}}{\isacharunderscore}{\kern0pt}{\isadigit{2}}{\isacharcolon}{\kern0pt}\ {\isachardoublequoteopen}x\ {\isacharless}{\kern0pt}\ y{\isachardoublequoteclose}\isanewline
\ \ \ \ \isacommand{assume}\isamarkupfalse%
\ a{\isadigit{3}}{\isacharunderscore}{\kern0pt}{\isadigit{3}}{\isacharcolon}{\kern0pt}\ {\isachardoublequoteopen}as\ {\isacharbang}{\kern0pt}\ x\ {\isacharequal}{\kern0pt}\ as\ {\isacharbang}{\kern0pt}\ y{\isachardoublequoteclose}\isanewline
\ \ \ \ \isacommand{have}\isamarkupfalse%
\ a{\isadigit{3}}{\isacharunderscore}{\kern0pt}{\isadigit{4}}{\isacharcolon}{\kern0pt}\ {\isachardoublequoteopen}drop\ {\isacharparenleft}{\kern0pt}Suc\ x{\isacharparenright}{\kern0pt}\ as\ {\isacharequal}{\kern0pt}\ take\ {\isacharparenleft}{\kern0pt}y{\isacharminus}{\kern0pt}x{\isacharparenright}{\kern0pt}\ {\isacharparenleft}{\kern0pt}drop\ {\isacharparenleft}{\kern0pt}Suc\ x{\isacharparenright}{\kern0pt}\ as{\isacharparenright}{\kern0pt}{\isacharat}{\kern0pt}\ drop\ {\isacharparenleft}{\kern0pt}Suc\ y{\isacharparenright}{\kern0pt}\ as{\isachardoublequoteclose}\isanewline
\ \ \ \ \ \ \isacommand{apply}\isamarkupfalse%
\ {\isacharparenleft}{\kern0pt}subst\ append{\isacharunderscore}{\kern0pt}take{\isacharunderscore}{\kern0pt}drop{\isacharunderscore}{\kern0pt}id{\isacharbrackleft}{\kern0pt}\isakeyword{where}\ xs{\isacharequal}{\kern0pt}{\isachardoublequoteopen}drop\ {\isacharparenleft}{\kern0pt}Suc\ x{\isacharparenright}{\kern0pt}\ as{\isachardoublequoteclose}\ \isakeyword{and}\ n{\isacharequal}{\kern0pt}{\isachardoublequoteopen}y\ {\isacharminus}{\kern0pt}\ x{\isachardoublequoteclose}{\isacharcomma}{\kern0pt}\ symmetric{\isacharbrackright}{\kern0pt}{\isacharparenright}{\kern0pt}\isanewline
\ \ \ \ \ \ \isacommand{using}\isamarkupfalse%
\ a{\isadigit{3}}{\isacharunderscore}{\kern0pt}{\isadigit{2}}\ \isacommand{by}\isamarkupfalse%
\ simp\isanewline
\ \ \ \ \isacommand{have}\isamarkupfalse%
\ {\isachardoublequoteopen}count{\isacharunderscore}{\kern0pt}list\ {\isacharparenleft}{\kern0pt}drop\ {\isacharparenleft}{\kern0pt}Suc\ x{\isacharparenright}{\kern0pt}\ as{\isacharparenright}{\kern0pt}\ {\isacharparenleft}{\kern0pt}as\ {\isacharbang}{\kern0pt}\ x{\isacharparenright}{\kern0pt}\ {\isacharequal}{\kern0pt}\ count{\isacharunderscore}{\kern0pt}list\ {\isacharparenleft}{\kern0pt}take\ {\isacharparenleft}{\kern0pt}y{\isacharminus}{\kern0pt}x{\isacharparenright}{\kern0pt}\ {\isacharparenleft}{\kern0pt}drop\ {\isacharparenleft}{\kern0pt}Suc\ x{\isacharparenright}{\kern0pt}\ as{\isacharparenright}{\kern0pt}{\isacharparenright}{\kern0pt}\ {\isacharparenleft}{\kern0pt}as\ {\isacharbang}{\kern0pt}\ y{\isacharparenright}{\kern0pt}\ {\isacharplus}{\kern0pt}\isanewline
\ \ \ \ \ \ \ \ count{\isacharunderscore}{\kern0pt}list\ {\isacharparenleft}{\kern0pt}drop\ {\isacharparenleft}{\kern0pt}Suc\ y{\isacharparenright}{\kern0pt}\ as{\isacharparenright}{\kern0pt}\ {\isacharparenleft}{\kern0pt}as\ {\isacharbang}{\kern0pt}\ y{\isacharparenright}{\kern0pt}{\isachardoublequoteclose}\isanewline
\ \ \ \ \ \ \isacommand{using}\isamarkupfalse%
\ a{\isadigit{3}}{\isacharunderscore}{\kern0pt}{\isadigit{3}}\ \isacommand{by}\isamarkupfalse%
\ {\isacharparenleft}{\kern0pt}subst\ a{\isadigit{3}}{\isacharunderscore}{\kern0pt}{\isadigit{4}}{\isacharcomma}{\kern0pt}\ simp\ add{\isacharcolon}{\kern0pt}count{\isacharunderscore}{\kern0pt}list{\isacharunderscore}{\kern0pt}append{\isacharparenright}{\kern0pt}\isanewline
\ \ \ \ \isacommand{moreover}\isamarkupfalse%
\ \isacommand{have}\isamarkupfalse%
\ {\isachardoublequoteopen}count{\isacharunderscore}{\kern0pt}list\ {\isacharparenleft}{\kern0pt}take\ {\isacharparenleft}{\kern0pt}y{\isacharminus}{\kern0pt}x{\isacharparenright}{\kern0pt}\ {\isacharparenleft}{\kern0pt}drop\ {\isacharparenleft}{\kern0pt}Suc\ x{\isacharparenright}{\kern0pt}\ as{\isacharparenright}{\kern0pt}{\isacharparenright}{\kern0pt}\ {\isacharparenleft}{\kern0pt}as\ {\isacharbang}{\kern0pt}\ y{\isacharparenright}{\kern0pt}\ {\isasymge}\ {\isadigit{1}}{\isachardoublequoteclose}\isanewline
\ \ \ \ \ \ \isacommand{apply}\isamarkupfalse%
\ {\isacharparenleft}{\kern0pt}subst\ count{\isacharunderscore}{\kern0pt}list{\isacharunderscore}{\kern0pt}gr{\isacharunderscore}{\kern0pt}{\isadigit{1}}{\isacharbrackleft}{\kern0pt}symmetric{\isacharbrackright}{\kern0pt}{\isacharparenright}{\kern0pt}\ \ \isanewline
\ \ \ \ \ \ \isacommand{apply}\isamarkupfalse%
\ {\isacharparenleft}{\kern0pt}simp\ add{\isacharcolon}{\kern0pt}set{\isacharunderscore}{\kern0pt}conv{\isacharunderscore}{\kern0pt}nth{\isacharparenright}{\kern0pt}\isanewline
\ \ \ \ \ \ \isacommand{apply}\isamarkupfalse%
\ {\isacharparenleft}{\kern0pt}rule\ exI{\isacharbrackleft}{\kern0pt}\isakeyword{where}\ x{\isacharequal}{\kern0pt}{\isachardoublequoteopen}y{\isacharminus}{\kern0pt}x{\isacharminus}{\kern0pt}{\isadigit{1}}{\isachardoublequoteclose}{\isacharbrackright}{\kern0pt}{\isacharparenright}{\kern0pt}\isanewline
\ \ \ \ \ \ \isacommand{apply}\isamarkupfalse%
\ {\isacharparenleft}{\kern0pt}subst\ nth{\isacharunderscore}{\kern0pt}take{\isacharcomma}{\kern0pt}\ meson\ diff{\isacharunderscore}{\kern0pt}less\ a{\isadigit{3}}{\isacharunderscore}{\kern0pt}{\isadigit{2}}\ \ zero{\isacharunderscore}{\kern0pt}less{\isacharunderscore}{\kern0pt}diff\ zero{\isacharunderscore}{\kern0pt}less{\isacharunderscore}{\kern0pt}one{\isacharparenright}{\kern0pt}\isanewline
\ \ \ \ \ \ \isacommand{apply}\isamarkupfalse%
\ {\isacharparenleft}{\kern0pt}subst\ nth{\isacharunderscore}{\kern0pt}drop{\isacharparenright}{\kern0pt}\ \isacommand{using}\isamarkupfalse%
\ a{\isadigit{3}}{\isacharunderscore}{\kern0pt}{\isadigit{1}}\ a{\isadigit{3}}{\isacharunderscore}{\kern0pt}{\isadigit{2}}\ \isacommand{apply}\isamarkupfalse%
\ simp\isanewline
\ \ \ \ \ \ \isacommand{apply}\isamarkupfalse%
\ {\isacharparenleft}{\kern0pt}rule\ conjI{\isacharcomma}{\kern0pt}\ rule\ arg{\isacharunderscore}{\kern0pt}cong{\isadigit{2}}{\isacharbrackleft}{\kern0pt}\isakeyword{where}\ f{\isacharequal}{\kern0pt}{\isachardoublequoteopen}{\isacharparenleft}{\kern0pt}{\isacharbang}{\kern0pt}{\isacharparenright}{\kern0pt}{\isachardoublequoteclose}{\isacharbrackright}{\kern0pt}{\isacharcomma}{\kern0pt}\ simp{\isacharparenright}{\kern0pt}\isanewline
\ \ \ \ \ \ \isacommand{using}\isamarkupfalse%
\ a{\isadigit{3}}{\isacharunderscore}{\kern0pt}{\isadigit{2}}\ \isacommand{apply}\isamarkupfalse%
\ simp\isanewline
\ \ \ \ \ \ \isacommand{apply}\isamarkupfalse%
\ {\isacharparenleft}{\kern0pt}rule\ conjI{\isacharparenright}{\kern0pt}\isanewline
\ \ \ \ \ \ \isacommand{using}\isamarkupfalse%
\ a{\isadigit{3}}{\isacharunderscore}{\kern0pt}{\isadigit{1}}\ a{\isadigit{3}}{\isacharunderscore}{\kern0pt}{\isadigit{2}}\ \isacommand{apply}\isamarkupfalse%
\ simp\isanewline
\ \ \ \ \ \ \isacommand{by}\isamarkupfalse%
\ {\isacharparenleft}{\kern0pt}meson\ diff{\isacharunderscore}{\kern0pt}less\ a{\isadigit{3}}{\isacharunderscore}{\kern0pt}{\isadigit{2}}\ \ zero{\isacharunderscore}{\kern0pt}less{\isacharunderscore}{\kern0pt}diff\ zero{\isacharunderscore}{\kern0pt}less{\isacharunderscore}{\kern0pt}one{\isacharparenright}{\kern0pt}\isanewline
\ \ \ \ \isacommand{ultimately}\isamarkupfalse%
\ \isacommand{show}\isamarkupfalse%
\ {\isachardoublequoteopen}{\isacharquery}{\kern0pt}ths\ x\ y{\isachardoublequoteclose}\ \isacommand{by}\isamarkupfalse%
\ presburger\isanewline
\ \ \isacommand{qed}\isamarkupfalse%
\isanewline
\isanewline
\ \ \isacommand{have}\isamarkupfalse%
\ a{\isadigit{1}}{\isacharcolon}{\kern0pt}\ {\isachardoublequoteopen}inj{\isacharunderscore}{\kern0pt}on\ f\ {\isacharbraceleft}{\kern0pt}k{\isachardot}{\kern0pt}\ k\ {\isacharless}{\kern0pt}\ length\ as{\isacharbraceright}{\kern0pt}{\isachardoublequoteclose}\isanewline
\ \ \isacommand{proof}\isamarkupfalse%
\ {\isacharparenleft}{\kern0pt}rule\ inj{\isacharunderscore}{\kern0pt}onI{\isacharparenright}{\kern0pt}\isanewline
\ \ \ \ \isacommand{fix}\isamarkupfalse%
\ x\ y\isanewline
\ \ \ \ \isacommand{assume}\isamarkupfalse%
\ {\isachardoublequoteopen}x\ {\isasymin}\ {\isacharbraceleft}{\kern0pt}k{\isachardot}{\kern0pt}\ k\ {\isacharless}{\kern0pt}\ length\ as{\isacharbraceright}{\kern0pt}{\isachardoublequoteclose}\isanewline
\ \ \ \ \isacommand{moreover}\isamarkupfalse%
\ \isacommand{assume}\isamarkupfalse%
\ {\isachardoublequoteopen}y\ {\isasymin}\ {\isacharbraceleft}{\kern0pt}k{\isachardot}{\kern0pt}\ k\ {\isacharless}{\kern0pt}\ length\ as{\isacharbraceright}{\kern0pt}{\isachardoublequoteclose}\isanewline
\ \ \ \ \isacommand{moreover}\isamarkupfalse%
\ \isacommand{assume}\isamarkupfalse%
\ {\isachardoublequoteopen}f\ x\ {\isacharequal}{\kern0pt}\ f\ y{\isachardoublequoteclose}\isanewline
\ \ \ \ \isacommand{ultimately}\isamarkupfalse%
\ \isacommand{show}\isamarkupfalse%
\ {\isachardoublequoteopen}x\ {\isacharequal}{\kern0pt}\ y{\isachardoublequoteclose}\isanewline
\ \ \ \ \ \ \isacommand{apply}\isamarkupfalse%
\ {\isacharparenleft}{\kern0pt}cases\ {\isachardoublequoteopen}x\ {\isacharless}{\kern0pt}\ y{\isachardoublequoteclose}{\isacharcomma}{\kern0pt}\ simp\ add{\isacharcolon}{\kern0pt}f{\isacharunderscore}{\kern0pt}def{\isacharcomma}{\kern0pt}\ metis\ a{\isadigit{3}}{\isacharparenright}{\kern0pt}\ \isanewline
\ \ \ \ \ \ \isacommand{apply}\isamarkupfalse%
\ {\isacharparenleft}{\kern0pt}cases\ {\isachardoublequoteopen}y\ {\isacharless}{\kern0pt}\ x{\isachardoublequoteclose}{\isacharcomma}{\kern0pt}\ simp\ add{\isacharcolon}{\kern0pt}f{\isacharunderscore}{\kern0pt}def{\isacharcomma}{\kern0pt}\ metis\ a{\isadigit{3}}{\isacharparenright}{\kern0pt}\ \isanewline
\ \ \ \ \ \ \isacommand{by}\isamarkupfalse%
\ simp\isanewline
\ \ \isacommand{qed}\isamarkupfalse%
\isanewline
\ \ \isacommand{have}\isamarkupfalse%
\ a{\isadigit{2}}{\isacharunderscore}{\kern0pt}{\isadigit{1}}{\isacharcolon}{\kern0pt}\ {\isachardoublequoteopen}\ {\isasymAnd}x{\isachardot}{\kern0pt}\ x\ {\isacharless}{\kern0pt}\ length\ as\ {\isasymLongrightarrow}\ count{\isacharunderscore}{\kern0pt}list\ {\isacharparenleft}{\kern0pt}drop\ {\isacharparenleft}{\kern0pt}Suc\ x{\isacharparenright}{\kern0pt}\ as{\isacharparenright}{\kern0pt}\ {\isacharparenleft}{\kern0pt}as\ {\isacharbang}{\kern0pt}\ x{\isacharparenright}{\kern0pt}\ {\isacharless}{\kern0pt}\ count{\isacharunderscore}{\kern0pt}list\ as\ {\isacharparenleft}{\kern0pt}as\ {\isacharbang}{\kern0pt}\ x{\isacharparenright}{\kern0pt}{\isachardoublequoteclose}\isanewline
\ \ \isacommand{proof}\isamarkupfalse%
\ {\isacharminus}{\kern0pt}\isanewline
\ \ \ \ \isacommand{fix}\isamarkupfalse%
\ x\isanewline
\ \ \ \ \isacommand{assume}\isamarkupfalse%
\ a{\isacharcolon}{\kern0pt}{\isachardoublequoteopen}x\ {\isacharless}{\kern0pt}\ length\ as{\isachardoublequoteclose}\isanewline
\ \ \ \ \isacommand{have}\isamarkupfalse%
\ {\isachardoublequoteopen}{\isadigit{1}}\ {\isasymle}\ count{\isacharunderscore}{\kern0pt}list\ {\isacharparenleft}{\kern0pt}take\ {\isacharparenleft}{\kern0pt}Suc\ x{\isacharparenright}{\kern0pt}\ as{\isacharparenright}{\kern0pt}\ {\isacharparenleft}{\kern0pt}as\ {\isacharbang}{\kern0pt}\ x{\isacharparenright}{\kern0pt}{\isachardoublequoteclose}\isanewline
\ \ \ \ \ \ \isacommand{apply}\isamarkupfalse%
\ {\isacharparenleft}{\kern0pt}subst\ count{\isacharunderscore}{\kern0pt}list{\isacharunderscore}{\kern0pt}gr{\isacharunderscore}{\kern0pt}{\isadigit{1}}{\isacharbrackleft}{\kern0pt}symmetric{\isacharbrackright}{\kern0pt}{\isacharparenright}{\kern0pt}\isanewline
\ \ \ \ \ \ \isacommand{using}\isamarkupfalse%
\ a\ \isacommand{by}\isamarkupfalse%
\ {\isacharparenleft}{\kern0pt}simp\ add{\isacharcolon}{\kern0pt}\ take{\isacharunderscore}{\kern0pt}Suc{\isacharunderscore}{\kern0pt}conv{\isacharunderscore}{\kern0pt}app{\isacharunderscore}{\kern0pt}nth{\isacharparenright}{\kern0pt}\isanewline
\ \ \ \ \isacommand{hence}\isamarkupfalse%
\ {\isachardoublequoteopen}count{\isacharunderscore}{\kern0pt}list\ {\isacharparenleft}{\kern0pt}drop\ {\isacharparenleft}{\kern0pt}Suc\ x{\isacharparenright}{\kern0pt}\ as{\isacharparenright}{\kern0pt}\ {\isacharparenleft}{\kern0pt}as\ {\isacharbang}{\kern0pt}\ x{\isacharparenright}{\kern0pt}\ {\isacharless}{\kern0pt}\ count{\isacharunderscore}{\kern0pt}list\ {\isacharparenleft}{\kern0pt}take\ {\isacharparenleft}{\kern0pt}Suc\ x{\isacharparenright}{\kern0pt}\ as{\isacharparenright}{\kern0pt}\ {\isacharparenleft}{\kern0pt}as\ {\isacharbang}{\kern0pt}\ x{\isacharparenright}{\kern0pt}\ {\isacharplus}{\kern0pt}\ count{\isacharunderscore}{\kern0pt}list\ {\isacharparenleft}{\kern0pt}drop\ {\isacharparenleft}{\kern0pt}Suc\ x{\isacharparenright}{\kern0pt}\ as{\isacharparenright}{\kern0pt}\ {\isacharparenleft}{\kern0pt}as\ {\isacharbang}{\kern0pt}\ x{\isacharparenright}{\kern0pt}{\isachardoublequoteclose}\isanewline
\ \ \ \ \ \ \isacommand{by}\isamarkupfalse%
\ {\isacharparenleft}{\kern0pt}simp{\isacharparenright}{\kern0pt}\isanewline
\ \ \ \ \isacommand{also}\isamarkupfalse%
\ \isacommand{have}\isamarkupfalse%
\ {\isachardoublequoteopen}{\isachardot}{\kern0pt}{\isachardot}{\kern0pt}{\isachardot}{\kern0pt}\ {\isacharequal}{\kern0pt}\ count{\isacharunderscore}{\kern0pt}list\ as\ {\isacharparenleft}{\kern0pt}as\ {\isacharbang}{\kern0pt}\ x{\isacharparenright}{\kern0pt}{\isachardoublequoteclose}\isanewline
\ \ \ \ \ \ \isacommand{by}\isamarkupfalse%
\ {\isacharparenleft}{\kern0pt}simp\ add{\isacharcolon}{\kern0pt}count{\isacharunderscore}{\kern0pt}list{\isacharunderscore}{\kern0pt}append{\isacharbrackleft}{\kern0pt}symmetric{\isacharbrackright}{\kern0pt}{\isacharparenright}{\kern0pt}\isanewline
\ \ \ \ \isacommand{finally}\isamarkupfalse%
\ \isacommand{show}\isamarkupfalse%
\ {\isachardoublequoteopen}count{\isacharunderscore}{\kern0pt}list\ {\isacharparenleft}{\kern0pt}drop\ {\isacharparenleft}{\kern0pt}Suc\ x{\isacharparenright}{\kern0pt}\ as{\isacharparenright}{\kern0pt}\ {\isacharparenleft}{\kern0pt}as\ {\isacharbang}{\kern0pt}\ x{\isacharparenright}{\kern0pt}\ {\isacharless}{\kern0pt}\ count{\isacharunderscore}{\kern0pt}list\ as\ {\isacharparenleft}{\kern0pt}as\ {\isacharbang}{\kern0pt}\ x{\isacharparenright}{\kern0pt}{\isachardoublequoteclose}\isanewline
\ \ \ \ \ \ \isacommand{by}\isamarkupfalse%
\ blast\isanewline
\ \ \isacommand{qed}\isamarkupfalse%
\isanewline
\ \ \isacommand{have}\isamarkupfalse%
\ a{\isadigit{2}}{\isacharcolon}{\kern0pt}\ {\isachardoublequoteopen}f\ {\isacharbackquote}{\kern0pt}\ {\isacharbraceleft}{\kern0pt}k{\isachardot}{\kern0pt}\ k\ {\isacharless}{\kern0pt}\ length\ as{\isacharbraceright}{\kern0pt}\ {\isacharequal}{\kern0pt}\ {\isacharbraceleft}{\kern0pt}{\isacharparenleft}{\kern0pt}u{\isacharcomma}{\kern0pt}\ v{\isacharparenright}{\kern0pt}{\isachardot}{\kern0pt}\ v\ {\isacharless}{\kern0pt}\ count{\isacharunderscore}{\kern0pt}list\ as\ u{\isacharbraceright}{\kern0pt}{\isachardoublequoteclose}\isanewline
\ \ \ \ \isacommand{apply}\isamarkupfalse%
\ {\isacharparenleft}{\kern0pt}rule\ card{\isacharunderscore}{\kern0pt}seteq{\isacharparenright}{\kern0pt}\ \isanewline
\ \ \ \ \ \ \isacommand{apply}\isamarkupfalse%
\ {\isacharparenleft}{\kern0pt}metis\ fin{\isacharunderscore}{\kern0pt}space{\isacharbrackleft}{\kern0pt}OF\ assms{\isacharparenleft}{\kern0pt}{\isadigit{1}}{\isacharparenright}{\kern0pt}{\isacharbrackright}{\kern0pt}{\isacharparenright}{\kern0pt}\isanewline
\ \ \ \ \ \isacommand{apply}\isamarkupfalse%
\ {\isacharparenleft}{\kern0pt}rule\ image{\isacharunderscore}{\kern0pt}subsetI{\isacharcomma}{\kern0pt}\ simp\ add{\isacharcolon}{\kern0pt}f{\isacharunderscore}{\kern0pt}def{\isacharparenright}{\kern0pt}\isanewline
\ \ \ \ \isacommand{apply}\isamarkupfalse%
\ {\isacharparenleft}{\kern0pt}metis\ a{\isadigit{2}}{\isacharunderscore}{\kern0pt}{\isadigit{1}}{\isacharparenright}{\kern0pt}\isanewline
\ \ \ \ \isacommand{apply}\isamarkupfalse%
\ {\isacharparenleft}{\kern0pt}subst\ card{\isacharunderscore}{\kern0pt}image{\isacharbrackleft}{\kern0pt}OF\ a{\isadigit{1}}{\isacharbrackright}{\kern0pt}{\isacharparenright}{\kern0pt}\isanewline
\ \ \ \ \isacommand{by}\isamarkupfalse%
\ {\isacharparenleft}{\kern0pt}subst\ card{\isacharunderscore}{\kern0pt}space{\isacharbrackleft}{\kern0pt}OF\ assms{\isacharparenleft}{\kern0pt}{\isadigit{1}}{\isacharparenright}{\kern0pt}{\isacharbrackright}{\kern0pt}{\isacharcomma}{\kern0pt}\ simp{\isacharparenright}{\kern0pt}\isanewline
\isanewline
\ \ \isacommand{have}\isamarkupfalse%
\ {\isachardoublequoteopen}bij{\isacharunderscore}{\kern0pt}betw\ f\ {\isacharbraceleft}{\kern0pt}k{\isachardot}{\kern0pt}\ k\ {\isacharless}{\kern0pt}\ length\ as{\isacharbraceright}{\kern0pt}\ {\isacharbraceleft}{\kern0pt}{\isacharparenleft}{\kern0pt}u{\isacharcomma}{\kern0pt}\ v{\isacharparenright}{\kern0pt}{\isachardot}{\kern0pt}\ v\ {\isacharless}{\kern0pt}\ count{\isacharunderscore}{\kern0pt}list\ as\ u{\isacharbraceright}{\kern0pt}{\isachardoublequoteclose}\isanewline
\ \ \ \ \isacommand{using}\isamarkupfalse%
\ a{\isadigit{1}}\ a{\isadigit{2}}\ \isacommand{by}\isamarkupfalse%
\ {\isacharparenleft}{\kern0pt}simp\ add{\isacharcolon}{\kern0pt}bij{\isacharunderscore}{\kern0pt}betw{\isacharunderscore}{\kern0pt}def{\isacharparenright}{\kern0pt}\isanewline
\ \ \isacommand{thus}\isamarkupfalse%
\ {\isacharquery}{\kern0pt}thesis\isanewline
\ \ \ \ \isacommand{using}\isamarkupfalse%
\ assms\ \isacommand{apply}\isamarkupfalse%
\ {\isacharparenleft}{\kern0pt}subst\ map{\isacharunderscore}{\kern0pt}pmf{\isacharunderscore}{\kern0pt}def{\isacharbrackleft}{\kern0pt}symmetric{\isacharbrackright}{\kern0pt}{\isacharparenright}{\kern0pt}\isanewline
\ \ \ \ \isacommand{by}\isamarkupfalse%
\ {\isacharparenleft}{\kern0pt}rule\ map{\isacharunderscore}{\kern0pt}pmf{\isacharunderscore}{\kern0pt}of{\isacharunderscore}{\kern0pt}set{\isacharunderscore}{\kern0pt}bij{\isacharunderscore}{\kern0pt}betw{\isacharcomma}{\kern0pt}\ simp\ add{\isacharcolon}{\kern0pt}f{\isacharunderscore}{\kern0pt}def{\isacharcomma}{\kern0pt}\ blast{\isacharcomma}{\kern0pt}\ simp{\isacharparenright}{\kern0pt}\isanewline
\isacommand{qed}\isamarkupfalse%
%
\endisatagproof
{\isafoldproof}%
%
\isadelimproof
\isanewline
%
\endisadelimproof
\isanewline
\isacommand{lemma}\isamarkupfalse%
\ fk{\isacharunderscore}{\kern0pt}alg{\isacharunderscore}{\kern0pt}aux{\isacharunderscore}{\kern0pt}{\isadigit{4}}{\isacharcolon}{\kern0pt}\isanewline
\ \ \isakeyword{assumes}\ {\isachardoublequoteopen}as\ {\isasymnoteq}\ {\isacharbrackleft}{\kern0pt}{\isacharbrackright}{\kern0pt}{\isachardoublequoteclose}\isanewline
\ \ \isakeyword{shows}\ {\isachardoublequoteopen}fold\ {\isacharparenleft}{\kern0pt}{\isasymlambda}x\ {\isacharparenleft}{\kern0pt}c{\isacharcomma}{\kern0pt}state{\isacharparenright}{\kern0pt}{\isachardot}{\kern0pt}\ {\isacharparenleft}{\kern0pt}c{\isacharplus}{\kern0pt}{\isadigit{1}}{\isacharcomma}{\kern0pt}\ state\ {\isasymbind}\ fk{\isacharunderscore}{\kern0pt}update{\isacharprime}{\kern0pt}{\isacharprime}{\kern0pt}\ x\ c{\isacharparenright}{\kern0pt}{\isacharparenright}{\kern0pt}\ as\ {\isacharparenleft}{\kern0pt}{\isadigit{0}}{\isacharcomma}{\kern0pt}\ return{\isacharunderscore}{\kern0pt}pmf\ {\isacharparenleft}{\kern0pt}{\isadigit{0}}{\isacharcomma}{\kern0pt}{\isadigit{0}}{\isacharparenright}{\kern0pt}{\isacharparenright}{\kern0pt}\ {\isacharequal}{\kern0pt}\isanewline
\ \ {\isacharparenleft}{\kern0pt}length\ as{\isacharcomma}{\kern0pt}\ pmf{\isacharunderscore}{\kern0pt}of{\isacharunderscore}{\kern0pt}set\ {\isacharbraceleft}{\kern0pt}k{\isachardot}{\kern0pt}\ k\ {\isacharless}{\kern0pt}\ length\ as{\isacharbraceright}{\kern0pt}\ {\isasymbind}\ {\isacharparenleft}{\kern0pt}{\isasymlambda}k{\isachardot}{\kern0pt}\ return{\isacharunderscore}{\kern0pt}pmf\ {\isacharparenleft}{\kern0pt}as\ {\isacharbang}{\kern0pt}\ k{\isacharcomma}{\kern0pt}\ count{\isacharunderscore}{\kern0pt}list\ {\isacharparenleft}{\kern0pt}drop\ {\isacharparenleft}{\kern0pt}k{\isacharplus}{\kern0pt}{\isadigit{1}}{\isacharparenright}{\kern0pt}\ as{\isacharparenright}{\kern0pt}\ {\isacharparenleft}{\kern0pt}as\ {\isacharbang}{\kern0pt}\ k{\isacharparenright}{\kern0pt}{\isacharparenright}{\kern0pt}{\isacharparenright}{\kern0pt}{\isacharparenright}{\kern0pt}{\isachardoublequoteclose}\isanewline
%
\isadelimproof
\ \ %
\endisadelimproof
%
\isatagproof
\isacommand{using}\isamarkupfalse%
\ assms\isanewline
\isacommand{proof}\isamarkupfalse%
\ {\isacharparenleft}{\kern0pt}induction\ as\ rule{\isacharcolon}{\kern0pt}rev{\isacharunderscore}{\kern0pt}nonempty{\isacharunderscore}{\kern0pt}induct{\isacharparenright}{\kern0pt}\isanewline
\ \ \isacommand{case}\isamarkupfalse%
\ {\isacharparenleft}{\kern0pt}single\ x{\isacharparenright}{\kern0pt}\isanewline
\ \ \isacommand{have}\isamarkupfalse%
\ a{\isacharcolon}{\kern0pt}{\isachardoublequoteopen}bernoulli{\isacharunderscore}{\kern0pt}pmf\ {\isadigit{1}}\ {\isacharequal}{\kern0pt}\ return{\isacharunderscore}{\kern0pt}pmf\ True{\isachardoublequoteclose}\isanewline
\ \ \ \ \isacommand{by}\isamarkupfalse%
\ {\isacharparenleft}{\kern0pt}rule\ pmf{\isacharunderscore}{\kern0pt}eqI{\isacharcomma}{\kern0pt}\ simp\ add{\isacharcolon}{\kern0pt}indicator{\isacharunderscore}{\kern0pt}def{\isacharparenright}{\kern0pt}\isanewline
\ \ \isacommand{show}\isamarkupfalse%
\ {\isacharquery}{\kern0pt}case\ \isacommand{using}\isamarkupfalse%
\ single\ \isanewline
\ \ \ \ \isacommand{by}\isamarkupfalse%
\ {\isacharparenleft}{\kern0pt}simp\ add{\isacharcolon}{\kern0pt}bind{\isacharunderscore}{\kern0pt}return{\isacharunderscore}{\kern0pt}pmf\ pmf{\isacharunderscore}{\kern0pt}of{\isacharunderscore}{\kern0pt}set{\isacharunderscore}{\kern0pt}singleton\ a{\isacharparenright}{\kern0pt}\ \isanewline
\isacommand{next}\isamarkupfalse%
\isanewline
\ \ \isacommand{case}\isamarkupfalse%
\ {\isacharparenleft}{\kern0pt}snoc\ x\ xs{\isacharparenright}{\kern0pt}\isanewline
\ \ \isacommand{have}\isamarkupfalse%
\ c{\isacharcolon}{\kern0pt}{\isachardoublequoteopen}{\isasymAnd}c{\isachardot}{\kern0pt}\ fk{\isacharunderscore}{\kern0pt}update{\isacharprime}{\kern0pt}{\isacharprime}{\kern0pt}\ x\ c\ {\isacharequal}{\kern0pt}\ {\isacharparenleft}{\kern0pt}{\isasymlambda}a{\isachardot}{\kern0pt}\ fk{\isacharunderscore}{\kern0pt}update{\isacharprime}{\kern0pt}{\isacharprime}{\kern0pt}\ x\ c\ {\isacharparenleft}{\kern0pt}fst\ a{\isacharcomma}{\kern0pt}snd\ a{\isacharparenright}{\kern0pt}{\isacharparenright}{\kern0pt}{\isachardoublequoteclose}\ \isanewline
\ \ \ \ \isacommand{by}\isamarkupfalse%
\ auto\isanewline
\ \ \isacommand{have}\isamarkupfalse%
\ a{\isacharcolon}{\kern0pt}\ {\isachardoublequoteopen}{\isasymAnd}y{\isachardot}{\kern0pt}\ pmf{\isacharunderscore}{\kern0pt}of{\isacharunderscore}{\kern0pt}set\ {\isacharbraceleft}{\kern0pt}k{\isachardot}{\kern0pt}\ k\ {\isacharless}{\kern0pt}\ length\ xs{\isacharbraceright}{\kern0pt}\ {\isasymbind}\ {\isacharparenleft}{\kern0pt}{\isasymlambda}k{\isachardot}{\kern0pt}\ return{\isacharunderscore}{\kern0pt}pmf\ {\isacharparenleft}{\kern0pt}xs\ {\isacharbang}{\kern0pt}\ k{\isacharcomma}{\kern0pt}\ count{\isacharunderscore}{\kern0pt}list\ {\isacharparenleft}{\kern0pt}drop\ {\isacharparenleft}{\kern0pt}Suc\ k{\isacharparenright}{\kern0pt}\ xs{\isacharparenright}{\kern0pt}\ {\isacharparenleft}{\kern0pt}xs\ {\isacharbang}{\kern0pt}\ k{\isacharparenright}{\kern0pt}{\isacharparenright}{\kern0pt}\ {\isasymbind}\isanewline
\ \ \ \ \ \ \ \ \ \ {\isacharparenleft}{\kern0pt}{\isasymlambda}xa{\isachardot}{\kern0pt}\ return{\isacharunderscore}{\kern0pt}pmf\ {\isacharparenleft}{\kern0pt}if\ y\ then\ {\isacharparenleft}{\kern0pt}x{\isacharcomma}{\kern0pt}\ {\isadigit{0}}{\isacharparenright}{\kern0pt}\ else\ {\isacharparenleft}{\kern0pt}fst\ xa{\isacharcomma}{\kern0pt}\ snd\ xa\ {\isacharplus}{\kern0pt}\ {\isacharparenleft}{\kern0pt}of{\isacharunderscore}{\kern0pt}bool\ {\isacharparenleft}{\kern0pt}fst\ xa\ {\isacharequal}{\kern0pt}\ x{\isacharparenright}{\kern0pt}{\isacharparenright}{\kern0pt}{\isacharparenright}{\kern0pt}{\isacharparenright}{\kern0pt}{\isacharparenright}{\kern0pt}{\isacharparenright}{\kern0pt}\isanewline
\ \ \ \ \ \ {\isacharequal}{\kern0pt}\ pmf{\isacharunderscore}{\kern0pt}of{\isacharunderscore}{\kern0pt}set\ {\isacharbraceleft}{\kern0pt}k{\isachardot}{\kern0pt}\ k\ {\isacharless}{\kern0pt}\ length\ xs{\isacharbraceright}{\kern0pt}\ {\isasymbind}\ {\isacharparenleft}{\kern0pt}{\isasymlambda}k{\isachardot}{\kern0pt}\ return{\isacharunderscore}{\kern0pt}pmf\ {\isacharparenleft}{\kern0pt}if\ y\ then\ {\isacharparenleft}{\kern0pt}length\ xs{\isacharparenright}{\kern0pt}\ else\ k{\isacharparenright}{\kern0pt}\ {\isasymbind}\ {\isacharparenleft}{\kern0pt}{\isasymlambda}k{\isachardot}{\kern0pt}\ return{\isacharunderscore}{\kern0pt}pmf\ {\isacharparenleft}{\kern0pt}{\isacharparenleft}{\kern0pt}xs{\isacharat}{\kern0pt}{\isacharbrackleft}{\kern0pt}x{\isacharbrackright}{\kern0pt}{\isacharparenright}{\kern0pt}\ {\isacharbang}{\kern0pt}\ k{\isacharcomma}{\kern0pt}\ count{\isacharunderscore}{\kern0pt}list\ {\isacharparenleft}{\kern0pt}drop\ {\isacharparenleft}{\kern0pt}Suc\ k{\isacharparenright}{\kern0pt}\ {\isacharparenleft}{\kern0pt}xs{\isacharat}{\kern0pt}{\isacharbrackleft}{\kern0pt}x{\isacharbrackright}{\kern0pt}{\isacharparenright}{\kern0pt}{\isacharparenright}{\kern0pt}\ {\isacharparenleft}{\kern0pt}{\isacharparenleft}{\kern0pt}xs{\isacharat}{\kern0pt}{\isacharbrackleft}{\kern0pt}x{\isacharbrackright}{\kern0pt}{\isacharparenright}{\kern0pt}\ {\isacharbang}{\kern0pt}\ k{\isacharparenright}{\kern0pt}{\isacharparenright}{\kern0pt}{\isacharparenright}{\kern0pt}{\isacharparenright}{\kern0pt}{\isachardoublequoteclose}\isanewline
\ \ \ \ \isacommand{apply}\isamarkupfalse%
\ {\isacharparenleft}{\kern0pt}simp\ add{\isacharcolon}{\kern0pt}bind{\isacharunderscore}{\kern0pt}return{\isacharunderscore}{\kern0pt}pmf{\isacharparenright}{\kern0pt}\isanewline
\ \ \ \ \isacommand{apply}\isamarkupfalse%
\ {\isacharparenleft}{\kern0pt}rule\ bind{\isacharunderscore}{\kern0pt}pmf{\isacharunderscore}{\kern0pt}cong{\isacharcomma}{\kern0pt}\ simp{\isacharparenright}{\kern0pt}\isanewline
\ \ \ \ \isacommand{apply}\isamarkupfalse%
\ {\isacharparenleft}{\kern0pt}subst\ {\isacharparenleft}{\kern0pt}asm{\isacharparenright}{\kern0pt}\ set{\isacharunderscore}{\kern0pt}pmf{\isacharunderscore}{\kern0pt}of{\isacharunderscore}{\kern0pt}set{\isacharparenright}{\kern0pt}\isanewline
\ \ \ \ \isacommand{using}\isamarkupfalse%
\ snoc\ \isacommand{apply}\isamarkupfalse%
\ blast\ \isacommand{apply}\isamarkupfalse%
\ simp\isanewline
\ \ \ \ \isacommand{by}\isamarkupfalse%
\ {\isacharparenleft}{\kern0pt}simp\ add{\isacharcolon}{\kern0pt}nth{\isacharunderscore}{\kern0pt}append\ count{\isacharunderscore}{\kern0pt}list{\isacharunderscore}{\kern0pt}append{\isacharparenright}{\kern0pt}\isanewline
\isanewline
\ \ \isacommand{show}\isamarkupfalse%
\ {\isacharquery}{\kern0pt}case\ \isacommand{using}\isamarkupfalse%
\ snoc\ \isanewline
\ \ \ \ \isacommand{apply}\isamarkupfalse%
\ {\isacharparenleft}{\kern0pt}simp\ del{\isacharcolon}{\kern0pt}drop{\isacharunderscore}{\kern0pt}append{\isacharcomma}{\kern0pt}\ subst\ c{\isacharcomma}{\kern0pt}\ subst\ fk{\isacharunderscore}{\kern0pt}update{\isacharprime}{\kern0pt}{\isacharprime}{\kern0pt}{\isachardot}{\kern0pt}simps{\isacharparenright}{\kern0pt}\isanewline
\ \ \ \ \isacommand{apply}\isamarkupfalse%
\ {\isacharparenleft}{\kern0pt}subst\ bind{\isacharunderscore}{\kern0pt}commute{\isacharunderscore}{\kern0pt}pmf{\isacharparenright}{\kern0pt}\isanewline
\ \ \ \ \isacommand{apply}\isamarkupfalse%
\ {\isacharparenleft}{\kern0pt}subst\ bind{\isacharunderscore}{\kern0pt}assoc{\isacharunderscore}{\kern0pt}pmf{\isacharparenright}{\kern0pt}\isanewline
\ \ \ \ \isacommand{apply}\isamarkupfalse%
\ {\isacharparenleft}{\kern0pt}simp\ add{\isacharcolon}{\kern0pt}a\ del{\isacharcolon}{\kern0pt}drop{\isacharunderscore}{\kern0pt}append{\isacharparenright}{\kern0pt}\isanewline
\ \ \ \ \isacommand{apply}\isamarkupfalse%
\ {\isacharparenleft}{\kern0pt}subst\ bind{\isacharunderscore}{\kern0pt}assoc{\isacharunderscore}{\kern0pt}pmf{\isacharbrackleft}{\kern0pt}symmetric{\isacharbrackright}{\kern0pt}{\isacharparenright}{\kern0pt}\isanewline
\ \ \ \ \isacommand{apply}\isamarkupfalse%
\ {\isacharparenleft}{\kern0pt}subst\ bind{\isacharunderscore}{\kern0pt}assoc{\isacharunderscore}{\kern0pt}pmf{\isacharbrackleft}{\kern0pt}symmetric{\isacharbrackright}{\kern0pt}{\isacharparenright}{\kern0pt}\isanewline
\ \ \ \ \isacommand{apply}\isamarkupfalse%
\ {\isacharparenleft}{\kern0pt}rule\ arg{\isacharunderscore}{\kern0pt}cong{\isadigit{2}}{\isacharbrackleft}{\kern0pt}\isakeyword{where}\ f{\isacharequal}{\kern0pt}{\isachardoublequoteopen}bind{\isacharunderscore}{\kern0pt}pmf{\isachardoublequoteclose}{\isacharbrackright}{\kern0pt}{\isacharparenright}{\kern0pt}\isanewline
\ \ \ \ \ \isacommand{apply}\isamarkupfalse%
\ {\isacharparenleft}{\kern0pt}rule\ pmf{\isacharunderscore}{\kern0pt}eqI{\isacharparenright}{\kern0pt}\isanewline
\ \ \ \ \ \isacommand{apply}\isamarkupfalse%
\ {\isacharparenleft}{\kern0pt}subst\ pmf{\isacharunderscore}{\kern0pt}bind{\isacharparenright}{\kern0pt}\isanewline
\ \ \ \ \ \isacommand{apply}\isamarkupfalse%
\ {\isacharparenleft}{\kern0pt}subst\ pmf{\isacharunderscore}{\kern0pt}of{\isacharunderscore}{\kern0pt}set{\isacharcomma}{\kern0pt}\ blast{\isacharcomma}{\kern0pt}\ simp{\isacharparenright}{\kern0pt}\isanewline
\ \ \ \ \ \isacommand{apply}\isamarkupfalse%
\ {\isacharparenleft}{\kern0pt}subst\ pmf{\isacharunderscore}{\kern0pt}bind{\isacharparenright}{\kern0pt}\isanewline
\ \ \ \ \ \isacommand{apply}\isamarkupfalse%
\ {\isacharparenleft}{\kern0pt}simp{\isacharparenright}{\kern0pt}\isanewline
\ \ \ \ \ \isacommand{apply}\isamarkupfalse%
\ {\isacharparenleft}{\kern0pt}subst\ measure{\isacharunderscore}{\kern0pt}pmf{\isacharunderscore}{\kern0pt}of{\isacharunderscore}{\kern0pt}set{\isacharcomma}{\kern0pt}\ blast{\isacharcomma}{\kern0pt}\ simp{\isacharparenright}{\kern0pt}\isanewline
\ \ \ \ \ \isacommand{apply}\isamarkupfalse%
\ {\isacharparenleft}{\kern0pt}simp\ add{\isacharcolon}{\kern0pt}indicator{\isacharunderscore}{\kern0pt}def{\isacharparenright}{\kern0pt}\isanewline
\ \ \ \ \ \isacommand{apply}\isamarkupfalse%
\ {\isacharparenleft}{\kern0pt}subst\ frac{\isacharunderscore}{\kern0pt}eq{\isacharunderscore}{\kern0pt}eq{\isacharcomma}{\kern0pt}\ simp{\isacharcomma}{\kern0pt}\ linarith{\isacharparenright}{\kern0pt}\isanewline
\ \ \ \ \ \isacommand{apply}\isamarkupfalse%
\ {\isacharparenleft}{\kern0pt}simp\ add{\isacharcolon}{\kern0pt}algebra{\isacharunderscore}{\kern0pt}simps{\isacharparenright}{\kern0pt}\isanewline
\ \ \ \ \isacommand{by}\isamarkupfalse%
\ simp\isanewline
\isacommand{qed}\isamarkupfalse%
%
\endisatagproof
{\isafoldproof}%
%
\isadelimproof
\isanewline
%
\endisadelimproof
\isanewline
\isacommand{definition}\isamarkupfalse%
\ if{\isacharunderscore}{\kern0pt}then{\isacharunderscore}{\kern0pt}else\ \isakeyword{where}\ {\isachardoublequoteopen}if{\isacharunderscore}{\kern0pt}then{\isacharunderscore}{\kern0pt}else\ p\ q\ r\ {\isacharequal}{\kern0pt}\ {\isacharparenleft}{\kern0pt}if\ p\ then\ q\ else\ r{\isacharparenright}{\kern0pt}{\isachardoublequoteclose}%
\begin{isamarkuptext}%
This definition is introduced to be able to temporarily substitute \isa{if\ p\ then\ q\ else\ r}
with \isa{if{\isacharunderscore}{\kern0pt}then{\isacharunderscore}{\kern0pt}else\ p\ q\ r}, which unblocks the simplifier to process \isa{q} and \isa{r}.%
\end{isamarkuptext}\isamarkuptrue%
\isacommand{lemma}\isamarkupfalse%
\ fk{\isacharunderscore}{\kern0pt}alg{\isacharunderscore}{\kern0pt}aux{\isacharunderscore}{\kern0pt}{\isadigit{2}}{\isacharcolon}{\kern0pt}\isanewline
\ \ {\isachardoublequoteopen}fold\ {\isacharparenleft}{\kern0pt}{\isasymlambda}x\ {\isacharparenleft}{\kern0pt}c{\isacharcomma}{\kern0pt}\ state{\isacharparenright}{\kern0pt}{\isachardot}{\kern0pt}\ {\isacharparenleft}{\kern0pt}c{\isacharplus}{\kern0pt}{\isadigit{1}}{\isacharcomma}{\kern0pt}\ state\ {\isasymbind}\ fk{\isacharunderscore}{\kern0pt}update{\isacharprime}{\kern0pt}\ x\ s\isactrlsub {\isadigit{1}}\ s\isactrlsub {\isadigit{2}}\ c{\isacharparenright}{\kern0pt}{\isacharparenright}{\kern0pt}\ as\ {\isacharparenleft}{\kern0pt}{\isadigit{0}}{\isacharcomma}{\kern0pt}\ return{\isacharunderscore}{\kern0pt}pmf\ {\isacharparenleft}{\kern0pt}{\isasymlambda}i\ {\isasymin}\ {\isacharbraceleft}{\kern0pt}{\isadigit{0}}{\isachardot}{\kern0pt}{\isachardot}{\kern0pt}{\isacharless}{\kern0pt}s\isactrlsub {\isadigit{1}}{\isacharbraceright}{\kern0pt}\ {\isasymtimes}\ {\isacharbraceleft}{\kern0pt}{\isadigit{0}}{\isachardot}{\kern0pt}{\isachardot}{\kern0pt}{\isacharless}{\kern0pt}s\isactrlsub {\isadigit{2}}{\isacharbraceright}{\kern0pt}{\isachardot}{\kern0pt}\ {\isacharparenleft}{\kern0pt}{\isadigit{0}}{\isacharcomma}{\kern0pt}{\isadigit{0}}{\isacharparenright}{\kern0pt}{\isacharparenright}{\kern0pt}{\isacharparenright}{\kern0pt}\isanewline
\ \ {\isacharequal}{\kern0pt}\ \ {\isacharparenleft}{\kern0pt}length\ as{\isacharcomma}{\kern0pt}\ prod{\isacharunderscore}{\kern0pt}pmf\ {\isacharparenleft}{\kern0pt}{\isacharbraceleft}{\kern0pt}{\isadigit{0}}{\isachardot}{\kern0pt}{\isachardot}{\kern0pt}{\isacharless}{\kern0pt}s\isactrlsub {\isadigit{1}}{\isacharbraceright}{\kern0pt}\ {\isasymtimes}\ {\isacharbraceleft}{\kern0pt}{\isadigit{0}}{\isachardot}{\kern0pt}{\isachardot}{\kern0pt}{\isacharless}{\kern0pt}s\isactrlsub {\isadigit{2}}{\isacharbraceright}{\kern0pt}{\isacharparenright}{\kern0pt}\ {\isacharparenleft}{\kern0pt}{\isasymlambda}{\isacharunderscore}{\kern0pt}{\isachardot}{\kern0pt}\ {\isacharparenleft}{\kern0pt}snd\ {\isacharparenleft}{\kern0pt}fold\ {\isacharparenleft}{\kern0pt}{\isasymlambda}x\ {\isacharparenleft}{\kern0pt}c{\isacharcomma}{\kern0pt}state{\isacharparenright}{\kern0pt}{\isachardot}{\kern0pt}\ {\isacharparenleft}{\kern0pt}c{\isacharplus}{\kern0pt}{\isadigit{1}}{\isacharcomma}{\kern0pt}\ state\ {\isasymbind}\ fk{\isacharunderscore}{\kern0pt}update{\isacharprime}{\kern0pt}{\isacharprime}{\kern0pt}\ x\ c{\isacharparenright}{\kern0pt}{\isacharparenright}{\kern0pt}\ as\ {\isacharparenleft}{\kern0pt}{\isadigit{0}}{\isacharcomma}{\kern0pt}\ return{\isacharunderscore}{\kern0pt}pmf\ {\isacharparenleft}{\kern0pt}{\isadigit{0}}{\isacharcomma}{\kern0pt}{\isadigit{0}}{\isacharparenright}{\kern0pt}{\isacharparenright}{\kern0pt}{\isacharparenright}{\kern0pt}{\isacharparenright}{\kern0pt}{\isacharparenright}{\kern0pt}{\isacharparenright}{\kern0pt}{\isachardoublequoteclose}\isanewline
\ \ {\isacharparenleft}{\kern0pt}\isakeyword{is}\ {\isachardoublequoteopen}{\isacharquery}{\kern0pt}lhs\ {\isacharequal}{\kern0pt}\ {\isacharquery}{\kern0pt}rhs{\isachardoublequoteclose}{\isacharparenright}{\kern0pt}\isanewline
%
\isadelimproof
%
\endisadelimproof
%
\isatagproof
\isacommand{proof}\isamarkupfalse%
\ {\isacharparenleft}{\kern0pt}induction\ as\ rule{\isacharcolon}{\kern0pt}rev{\isacharunderscore}{\kern0pt}induct{\isacharparenright}{\kern0pt}\isanewline
\ \ \isacommand{case}\isamarkupfalse%
\ Nil\isanewline
\ \ \isacommand{thus}\isamarkupfalse%
\ {\isacharquery}{\kern0pt}case\isanewline
\ \ \ \ \isacommand{apply}\isamarkupfalse%
\ {\isacharparenleft}{\kern0pt}simp{\isacharcomma}{\kern0pt}\ rule\ pmf{\isacharunderscore}{\kern0pt}eqI{\isacharparenright}{\kern0pt}\isanewline
\ \ \ \ \isacommand{apply}\isamarkupfalse%
\ {\isacharparenleft}{\kern0pt}simp\ add{\isacharcolon}{\kern0pt}pmf{\isacharunderscore}{\kern0pt}prod{\isacharunderscore}{\kern0pt}pmf{\isacharparenright}{\kern0pt}\isanewline
\ \ \ \ \isacommand{apply}\isamarkupfalse%
\ {\isacharparenleft}{\kern0pt}rule\ conjI{\isacharcomma}{\kern0pt}\ rule\ impI{\isacharparenright}{\kern0pt}\isanewline
\ \ \ \ \ \isacommand{apply}\isamarkupfalse%
\ {\isacharparenleft}{\kern0pt}simp\ add{\isacharcolon}{\kern0pt}indicator{\isacharunderscore}{\kern0pt}def{\isacharcomma}{\kern0pt}\ rule\ conjI{\isacharcomma}{\kern0pt}\ rule\ impI{\isacharparenright}{\kern0pt}\isanewline
\ \ \ \ \ \ \isacommand{apply}\isamarkupfalse%
\ force\isanewline
\ \ \ \ \ \isacommand{using}\isamarkupfalse%
\ extensional{\isacharunderscore}{\kern0pt}arb\ \isacommand{apply}\isamarkupfalse%
\ fastforce\isanewline
\ \ \ \ \isacommand{apply}\isamarkupfalse%
\ {\isacharparenleft}{\kern0pt}simp\ add{\isacharcolon}{\kern0pt}extensional{\isacharunderscore}{\kern0pt}def\ indicator{\isacharunderscore}{\kern0pt}def{\isacharparenright}{\kern0pt}\isanewline
\ \ \ \ \isacommand{by}\isamarkupfalse%
\ {\isacharparenleft}{\kern0pt}meson\ SigmaD{\isadigit{1}}\ SigmaD{\isadigit{2}}\ atLeastLessThan{\isacharunderscore}{\kern0pt}iff{\isacharparenright}{\kern0pt}\isanewline
\isacommand{next}\isamarkupfalse%
\isanewline
\ \ \isacommand{case}\isamarkupfalse%
\ {\isacharparenleft}{\kern0pt}snoc\ x\ xs{\isacharparenright}{\kern0pt}\isanewline
\ \ \isacommand{obtain}\isamarkupfalse%
\ t{\isadigit{1}}\ t{\isadigit{2}}\ \isakeyword{where}\ t{\isacharunderscore}{\kern0pt}def{\isacharcolon}{\kern0pt}\ \isanewline
\ \ \ \ {\isachardoublequoteopen}{\isacharparenleft}{\kern0pt}t{\isadigit{1}}{\isacharcomma}{\kern0pt}t{\isadigit{2}}{\isacharparenright}{\kern0pt}\ {\isacharequal}{\kern0pt}\ fold\ {\isacharparenleft}{\kern0pt}{\isasymlambda}x\ {\isacharparenleft}{\kern0pt}c{\isacharcomma}{\kern0pt}\ state{\isacharparenright}{\kern0pt}{\isachardot}{\kern0pt}\ {\isacharparenleft}{\kern0pt}Suc\ c{\isacharcomma}{\kern0pt}\ state\ {\isasymbind}\ fk{\isacharunderscore}{\kern0pt}update{\isacharprime}{\kern0pt}{\isacharprime}{\kern0pt}\ x\ c{\isacharparenright}{\kern0pt}{\isacharparenright}{\kern0pt}\ xs\ {\isacharparenleft}{\kern0pt}{\isadigit{0}}{\isacharcomma}{\kern0pt}\ return{\isacharunderscore}{\kern0pt}pmf\ {\isacharparenleft}{\kern0pt}{\isadigit{0}}{\isacharcomma}{\kern0pt}{\isadigit{0}}{\isacharparenright}{\kern0pt}{\isacharparenright}{\kern0pt}{\isachardoublequoteclose}\isanewline
\ \ \ \ \isacommand{by}\isamarkupfalse%
\ {\isacharparenleft}{\kern0pt}metis\ {\isacharparenleft}{\kern0pt}no{\isacharunderscore}{\kern0pt}types{\isacharcomma}{\kern0pt}\ lifting{\isacharparenright}{\kern0pt}\ surj{\isacharunderscore}{\kern0pt}pair{\isacharparenright}{\kern0pt}\isanewline
\ \ \isacommand{have}\isamarkupfalse%
\ a{\isacharcolon}{\kern0pt}{\isachardoublequoteopen}fk{\isacharunderscore}{\kern0pt}update{\isacharprime}{\kern0pt}\ x\ s\isactrlsub {\isadigit{1}}\ s\isactrlsub {\isadigit{2}}\ {\isacharparenleft}{\kern0pt}length\ xs{\isacharparenright}{\kern0pt}\ {\isacharequal}{\kern0pt}\ {\isacharparenleft}{\kern0pt}{\isasymlambda}a{\isachardot}{\kern0pt}\ fk{\isacharunderscore}{\kern0pt}update{\isacharprime}{\kern0pt}\ x\ s\isactrlsub {\isadigit{1}}\ s\isactrlsub {\isadigit{2}}\ {\isacharparenleft}{\kern0pt}length\ xs{\isacharparenright}{\kern0pt}\ a{\isacharparenright}{\kern0pt}{\isachardoublequoteclose}\ \isanewline
\ \ \ \ \isacommand{by}\isamarkupfalse%
\ auto\isanewline
\ \ \isacommand{have}\isamarkupfalse%
\ c{\isacharcolon}{\kern0pt}{\isachardoublequoteopen}{\isasymAnd}c{\isachardot}{\kern0pt}\ fk{\isacharunderscore}{\kern0pt}update{\isacharprime}{\kern0pt}{\isacharprime}{\kern0pt}\ x\ c\ {\isacharequal}{\kern0pt}\ {\isacharparenleft}{\kern0pt}{\isasymlambda}a{\isachardot}{\kern0pt}\ fk{\isacharunderscore}{\kern0pt}update{\isacharprime}{\kern0pt}{\isacharprime}{\kern0pt}\ x\ c\ {\isacharparenleft}{\kern0pt}fst\ a{\isacharcomma}{\kern0pt}snd\ a{\isacharparenright}{\kern0pt}{\isacharparenright}{\kern0pt}{\isachardoublequoteclose}\ \isanewline
\ \ \ \ \isacommand{by}\isamarkupfalse%
\ auto\isanewline
\ \ \isacommand{have}\isamarkupfalse%
\ {\isachardoublequoteopen}fst\ {\isacharparenleft}{\kern0pt}fold\ {\isacharparenleft}{\kern0pt}{\isasymlambda}x\ {\isacharparenleft}{\kern0pt}c{\isacharcomma}{\kern0pt}\ state{\isacharparenright}{\kern0pt}{\isachardot}{\kern0pt}\ {\isacharparenleft}{\kern0pt}Suc\ c{\isacharcomma}{\kern0pt}\ state\ {\isasymbind}\ fk{\isacharunderscore}{\kern0pt}update{\isacharprime}{\kern0pt}{\isacharprime}{\kern0pt}\ x\ c{\isacharparenright}{\kern0pt}{\isacharparenright}{\kern0pt}\ xs\ {\isacharparenleft}{\kern0pt}{\isadigit{0}}{\isacharcomma}{\kern0pt}\ return{\isacharunderscore}{\kern0pt}pmf\ {\isacharparenleft}{\kern0pt}{\isadigit{0}}{\isacharcomma}{\kern0pt}{\isadigit{0}}{\isacharparenright}{\kern0pt}{\isacharparenright}{\kern0pt}{\isacharparenright}{\kern0pt}\ {\isacharequal}{\kern0pt}\ length\ xs{\isachardoublequoteclose}\isanewline
\ \ \ \ \isacommand{by}\isamarkupfalse%
\ {\isacharparenleft}{\kern0pt}induction\ xs\ rule{\isacharcolon}{\kern0pt}rev{\isacharunderscore}{\kern0pt}induct{\isacharcomma}{\kern0pt}\ simp{\isacharcomma}{\kern0pt}\ simp\ add{\isacharcolon}{\kern0pt}case{\isacharunderscore}{\kern0pt}prod{\isacharunderscore}{\kern0pt}beta{\isacharparenright}{\kern0pt}\isanewline
\ \ \isacommand{hence}\isamarkupfalse%
\ d{\isacharcolon}{\kern0pt}{\isachardoublequoteopen}t{\isadigit{1}}\ {\isacharequal}{\kern0pt}\ length\ xs{\isachardoublequoteclose}\ \isanewline
\ \ \ \ \isacommand{by}\isamarkupfalse%
\ {\isacharparenleft}{\kern0pt}metis\ t{\isacharunderscore}{\kern0pt}def\ fst{\isacharunderscore}{\kern0pt}conv{\isacharparenright}{\kern0pt}\isanewline
\isanewline
\ \ \isacommand{show}\isamarkupfalse%
\ {\isacharquery}{\kern0pt}case\ \isacommand{using}\isamarkupfalse%
\ snoc\isanewline
\ \ \ \ \isacommand{apply}\isamarkupfalse%
\ {\isacharparenleft}{\kern0pt}simp\ del{\isacharcolon}{\kern0pt}fk{\isacharunderscore}{\kern0pt}update{\isacharprime}{\kern0pt}{\isacharprime}{\kern0pt}{\isachardot}{\kern0pt}simps\ fk{\isacharunderscore}{\kern0pt}update{\isacharprime}{\kern0pt}{\isachardot}{\kern0pt}simps{\isacharparenright}{\kern0pt}\isanewline
\ \ \ \ \isacommand{apply}\isamarkupfalse%
\ {\isacharparenleft}{\kern0pt}simp\ add{\isacharcolon}{\kern0pt}t{\isacharunderscore}{\kern0pt}def{\isacharbrackleft}{\kern0pt}symmetric{\isacharbrackright}{\kern0pt}{\isacharparenright}{\kern0pt}\isanewline
\ \ \ \ \isacommand{apply}\isamarkupfalse%
\ {\isacharparenleft}{\kern0pt}subst\ a{\isacharbrackleft}{\kern0pt}simplified{\isacharbrackright}{\kern0pt}{\isacharparenright}{\kern0pt}\isanewline
\ \ \ \ \isacommand{apply}\isamarkupfalse%
\ {\isacharparenleft}{\kern0pt}subst\ pair{\isacharunderscore}{\kern0pt}pmfI{\isacharparenright}{\kern0pt}\isanewline
\ \ \ \ \isacommand{apply}\isamarkupfalse%
\ {\isacharparenleft}{\kern0pt}subst\ pair{\isacharunderscore}{\kern0pt}pmf{\isacharunderscore}{\kern0pt}ptw{\isacharcomma}{\kern0pt}\ simp{\isacharparenright}{\kern0pt}\isanewline
\ \ \ \ \isacommand{apply}\isamarkupfalse%
\ {\isacharparenleft}{\kern0pt}subst\ bind{\isacharunderscore}{\kern0pt}assoc{\isacharunderscore}{\kern0pt}pmf{\isacharparenright}{\kern0pt}\isanewline
\ \ \ \ \isacommand{apply}\isamarkupfalse%
\ {\isacharparenleft}{\kern0pt}subst\ bind{\isacharunderscore}{\kern0pt}return{\isacharunderscore}{\kern0pt}pmf{\isacharparenright}{\kern0pt}\isanewline
\ \ \ \ \isacommand{apply}\isamarkupfalse%
\ {\isacharparenleft}{\kern0pt}subst\ if{\isacharunderscore}{\kern0pt}then{\isacharunderscore}{\kern0pt}else{\isacharunderscore}{\kern0pt}def{\isacharbrackleft}{\kern0pt}symmetric{\isacharbrackright}{\kern0pt}{\isacharparenright}{\kern0pt}\isanewline
\ \ \ \ \isacommand{apply}\isamarkupfalse%
\ {\isacharparenleft}{\kern0pt}simp\ add{\isacharcolon}{\kern0pt}comp{\isacharunderscore}{\kern0pt}def\ cong{\isacharcolon}{\kern0pt}restrict{\isacharunderscore}{\kern0pt}cong{\isacharparenright}{\kern0pt}\isanewline
\ \ \ \ \isacommand{apply}\isamarkupfalse%
\ {\isacharparenleft}{\kern0pt}subst\ map{\isacharunderscore}{\kern0pt}ptw{\isacharcomma}{\kern0pt}\ simp{\isacharparenright}{\kern0pt}\isanewline
\ \ \ \ \isacommand{apply}\isamarkupfalse%
\ {\isacharparenleft}{\kern0pt}subst\ if{\isacharunderscore}{\kern0pt}then{\isacharunderscore}{\kern0pt}else{\isacharunderscore}{\kern0pt}def{\isacharparenright}{\kern0pt}\isanewline
\ \ \ \ \isacommand{apply}\isamarkupfalse%
\ {\isacharparenleft}{\kern0pt}rule\ arg{\isacharunderscore}{\kern0pt}cong{\isadigit{2}}{\isacharbrackleft}{\kern0pt}\isakeyword{where}\ f{\isacharequal}{\kern0pt}{\isachardoublequoteopen}prod{\isacharunderscore}{\kern0pt}pmf{\isachardoublequoteclose}{\isacharbrackright}{\kern0pt}{\isacharcomma}{\kern0pt}\ simp{\isacharparenright}{\kern0pt}\isanewline
\ \ \ \ \isacommand{apply}\isamarkupfalse%
\ {\isacharparenleft}{\kern0pt}rule\ ext{\isacharparenright}{\kern0pt}\isanewline
\ \ \ \ \isacommand{apply}\isamarkupfalse%
\ {\isacharparenleft}{\kern0pt}subst\ c{\isacharcomma}{\kern0pt}\ subst\ fk{\isacharunderscore}{\kern0pt}update{\isacharprime}{\kern0pt}{\isacharprime}{\kern0pt}{\isachardot}{\kern0pt}simps{\isacharcomma}{\kern0pt}\ simp{\isacharparenright}{\kern0pt}\isanewline
\ \ \ \ \isacommand{apply}\isamarkupfalse%
\ {\isacharparenleft}{\kern0pt}simp\ add{\isacharcolon}{\kern0pt}d{\isacharparenright}{\kern0pt}\isanewline
\ \ \ \ \isacommand{apply}\isamarkupfalse%
\ {\isacharparenleft}{\kern0pt}subst\ pair{\isacharunderscore}{\kern0pt}pmfI{\isacharparenright}{\kern0pt}\isanewline
\ \ \ \ \isacommand{apply}\isamarkupfalse%
\ {\isacharparenleft}{\kern0pt}rule\ arg{\isacharunderscore}{\kern0pt}cong{\isadigit{2}}{\isacharbrackleft}{\kern0pt}\isakeyword{where}\ f{\isacharequal}{\kern0pt}{\isachardoublequoteopen}bind{\isacharunderscore}{\kern0pt}pmf{\isachardoublequoteclose}{\isacharbrackright}{\kern0pt}{\isacharcomma}{\kern0pt}\ simp{\isacharparenright}{\kern0pt}\isanewline
\ \ \ \ \isacommand{by}\isamarkupfalse%
\ force\isanewline
\isacommand{qed}\isamarkupfalse%
%
\endisatagproof
{\isafoldproof}%
%
\isadelimproof
\isanewline
%
\endisadelimproof
\isanewline
\isacommand{lemma}\isamarkupfalse%
\ fk{\isacharunderscore}{\kern0pt}alg{\isacharunderscore}{\kern0pt}aux{\isacharunderscore}{\kern0pt}{\isadigit{1}}{\isacharcolon}{\kern0pt}\isanewline
\ \ \isakeyword{fixes}\ k\ {\isacharcolon}{\kern0pt}{\isacharcolon}{\kern0pt}\ nat\isanewline
\ \ \isakeyword{fixes}\ {\isasymepsilon}\ {\isacharcolon}{\kern0pt}{\isacharcolon}{\kern0pt}\ rat\isanewline
\ \ \isakeyword{assumes}\ {\isachardoublequoteopen}{\isasymdelta}\ {\isachargreater}{\kern0pt}\ {\isadigit{0}}{\isachardoublequoteclose}\isanewline
\ \ \isakeyword{assumes}\ {\isachardoublequoteopen}set\ as\ {\isasymsubseteq}\ {\isacharbraceleft}{\kern0pt}{\isadigit{0}}{\isachardot}{\kern0pt}{\isachardot}{\kern0pt}{\isacharless}{\kern0pt}n{\isacharbraceright}{\kern0pt}{\isachardoublequoteclose}\isanewline
\ \ \isakeyword{assumes}\ {\isachardoublequoteopen}as\ {\isasymnoteq}\ {\isacharbrackleft}{\kern0pt}{\isacharbrackright}{\kern0pt}{\isachardoublequoteclose}\isanewline
\ \ \isakeyword{defines}\ {\isachardoublequoteopen}sketch\ {\isasymequiv}\ fold\ {\isacharparenleft}{\kern0pt}{\isasymlambda}a\ state{\isachardot}{\kern0pt}\ state\ {\isasymbind}\ fk{\isacharunderscore}{\kern0pt}update\ a{\isacharparenright}{\kern0pt}\ as\ {\isacharparenleft}{\kern0pt}fk{\isacharunderscore}{\kern0pt}init\ k\ {\isasymdelta}\ {\isasymepsilon}\ n{\isacharparenright}{\kern0pt}{\isachardoublequoteclose}\isanewline
\ \ \isakeyword{defines}\ {\isachardoublequoteopen}s\isactrlsub {\isadigit{1}}\ {\isasymequiv}\ nat\ {\isasymlceil}{\isadigit{3}}{\isacharasterisk}{\kern0pt}real\ k{\isacharasterisk}{\kern0pt}{\isacharparenleft}{\kern0pt}real\ n{\isacharparenright}{\kern0pt}\ powr\ {\isacharparenleft}{\kern0pt}{\isadigit{1}}{\isacharminus}{\kern0pt}{\isadigit{1}}{\isacharslash}{\kern0pt}\ real\ k{\isacharparenright}{\kern0pt}{\isacharslash}{\kern0pt}\ {\isacharparenleft}{\kern0pt}real{\isacharunderscore}{\kern0pt}of{\isacharunderscore}{\kern0pt}rat\ {\isasymdelta}{\isacharparenright}{\kern0pt}\isactrlsup {\isadigit{2}}{\isasymrceil}{\isachardoublequoteclose}\isanewline
\ \ \isakeyword{defines}\ {\isachardoublequoteopen}s\isactrlsub {\isadigit{2}}\ {\isasymequiv}\ nat\ {\isasymlceil}{\isacharminus}{\kern0pt}{\isacharparenleft}{\kern0pt}{\isadigit{1}}{\isadigit{8}}\ {\isacharasterisk}{\kern0pt}\ ln\ {\isacharparenleft}{\kern0pt}real{\isacharunderscore}{\kern0pt}of{\isacharunderscore}{\kern0pt}rat\ {\isasymepsilon}{\isacharparenright}{\kern0pt}{\isacharparenright}{\kern0pt}{\isasymrceil}{\isachardoublequoteclose}\isanewline
\ \ \isakeyword{shows}\ {\isachardoublequoteopen}sketch\ {\isacharequal}{\kern0pt}\ \isanewline
\ \ \ \ map{\isacharunderscore}{\kern0pt}pmf\ {\isacharparenleft}{\kern0pt}{\isasymlambda}x{\isachardot}{\kern0pt}\ {\isacharparenleft}{\kern0pt}s\isactrlsub {\isadigit{1}}{\isacharcomma}{\kern0pt}s\isactrlsub {\isadigit{2}}{\isacharcomma}{\kern0pt}k{\isacharcomma}{\kern0pt}length\ as{\isacharcomma}{\kern0pt}\ x{\isacharparenright}{\kern0pt}{\isacharparenright}{\kern0pt}\ \isanewline
\ \ \ \ {\isacharparenleft}{\kern0pt}snd\ {\isacharparenleft}{\kern0pt}fold\ {\isacharparenleft}{\kern0pt}{\isasymlambda}x\ {\isacharparenleft}{\kern0pt}c{\isacharcomma}{\kern0pt}\ state{\isacharparenright}{\kern0pt}{\isachardot}{\kern0pt}\ {\isacharparenleft}{\kern0pt}c{\isacharplus}{\kern0pt}{\isadigit{1}}{\isacharcomma}{\kern0pt}\ state\ {\isasymbind}\ fk{\isacharunderscore}{\kern0pt}update{\isacharprime}{\kern0pt}\ x\ s\isactrlsub {\isadigit{1}}\ s\isactrlsub {\isadigit{2}}\ c{\isacharparenright}{\kern0pt}{\isacharparenright}{\kern0pt}\ as\ {\isacharparenleft}{\kern0pt}{\isadigit{0}}{\isacharcomma}{\kern0pt}\ return{\isacharunderscore}{\kern0pt}pmf\ {\isacharparenleft}{\kern0pt}{\isasymlambda}i\ {\isasymin}\ {\isacharbraceleft}{\kern0pt}{\isadigit{0}}{\isachardot}{\kern0pt}{\isachardot}{\kern0pt}{\isacharless}{\kern0pt}s\isactrlsub {\isadigit{1}}{\isacharbraceright}{\kern0pt}\ {\isasymtimes}\ {\isacharbraceleft}{\kern0pt}{\isadigit{0}}{\isachardot}{\kern0pt}{\isachardot}{\kern0pt}{\isacharless}{\kern0pt}s\isactrlsub {\isadigit{2}}{\isacharbraceright}{\kern0pt}{\isachardot}{\kern0pt}\ {\isacharparenleft}{\kern0pt}{\isadigit{0}}{\isacharcomma}{\kern0pt}{\isadigit{0}}{\isacharparenright}{\kern0pt}{\isacharparenright}{\kern0pt}{\isacharparenright}{\kern0pt}{\isacharparenright}{\kern0pt}{\isacharparenright}{\kern0pt}{\isachardoublequoteclose}\isanewline
%
\isadelimproof
\ \ %
\endisadelimproof
%
\isatagproof
\isacommand{using}\isamarkupfalse%
\ assms{\isacharparenleft}{\kern0pt}{\isadigit{3}}{\isacharparenright}{\kern0pt}\isanewline
\isacommand{proof}\isamarkupfalse%
\ {\isacharparenleft}{\kern0pt}subst\ sketch{\isacharunderscore}{\kern0pt}def{\isacharcomma}{\kern0pt}\ induction\ as\ rule{\isacharcolon}{\kern0pt}rev{\isacharunderscore}{\kern0pt}nonempty{\isacharunderscore}{\kern0pt}induct{\isacharparenright}{\kern0pt}\isanewline
\ \ \isacommand{case}\isamarkupfalse%
\ {\isacharparenleft}{\kern0pt}single\ x{\isacharparenright}{\kern0pt}\isanewline
\ \ \isacommand{then}\isamarkupfalse%
\ \isacommand{show}\isamarkupfalse%
\ {\isacharquery}{\kern0pt}case\ \isanewline
\ \ \ \ \isacommand{apply}\isamarkupfalse%
\ {\isacharparenleft}{\kern0pt}simp\ add{\isacharcolon}{\kern0pt}\ map{\isacharunderscore}{\kern0pt}bind{\isacharunderscore}{\kern0pt}pmf\ bind{\isacharunderscore}{\kern0pt}return{\isacharunderscore}{\kern0pt}pmf\ s\isactrlsub {\isadigit{1}}{\isacharunderscore}{\kern0pt}def{\isacharbrackleft}{\kern0pt}symmetric{\isacharbrackright}{\kern0pt}\ s\isactrlsub {\isadigit{2}}{\isacharunderscore}{\kern0pt}def{\isacharbrackleft}{\kern0pt}symmetric{\isacharbrackright}{\kern0pt}{\isacharparenright}{\kern0pt}\isanewline
\ \ \ \ \isacommand{apply}\isamarkupfalse%
\ {\isacharparenleft}{\kern0pt}rule\ arg{\isacharunderscore}{\kern0pt}cong{\isadigit{2}}{\isacharbrackleft}{\kern0pt}\isakeyword{where}\ f{\isacharequal}{\kern0pt}{\isachardoublequoteopen}bind{\isacharunderscore}{\kern0pt}pmf{\isachardoublequoteclose}{\isacharbrackright}{\kern0pt}{\isacharcomma}{\kern0pt}\ simp{\isacharparenright}{\kern0pt}\isanewline
\ \ \ \ \isacommand{by}\isamarkupfalse%
\ {\isacharparenleft}{\kern0pt}rule\ ext{\isacharcomma}{\kern0pt}\ subst\ restrict{\isacharunderscore}{\kern0pt}def{\isacharcomma}{\kern0pt}\ simp{\isacharparenright}{\kern0pt}\isanewline
\isacommand{next}\isamarkupfalse%
\isanewline
\ \ \isacommand{case}\isamarkupfalse%
\ {\isacharparenleft}{\kern0pt}snoc\ x\ xs{\isacharparenright}{\kern0pt}\isanewline
\ \ \isacommand{obtain}\isamarkupfalse%
\ t{\isadigit{1}}\ t{\isadigit{2}}\ \isakeyword{where}\ t{\isacharcolon}{\kern0pt}\isanewline
\ \ \ \ {\isachardoublequoteopen}fold\ {\isacharparenleft}{\kern0pt}{\isasymlambda}x\ {\isacharparenleft}{\kern0pt}c{\isacharcomma}{\kern0pt}\ state{\isacharparenright}{\kern0pt}{\isachardot}{\kern0pt}\ {\isacharparenleft}{\kern0pt}Suc\ c{\isacharcomma}{\kern0pt}\ state\ {\isasymbind}\ fk{\isacharunderscore}{\kern0pt}update{\isacharprime}{\kern0pt}\ x\ s\isactrlsub {\isadigit{1}}\ s\isactrlsub {\isadigit{2}}\ c{\isacharparenright}{\kern0pt}{\isacharparenright}{\kern0pt}\ xs\ {\isacharparenleft}{\kern0pt}{\isadigit{0}}{\isacharcomma}{\kern0pt}\ return{\isacharunderscore}{\kern0pt}pmf\ {\isacharparenleft}{\kern0pt}{\isasymlambda}i{\isachardot}{\kern0pt}\ if\ i\ {\isasymin}\ {\isacharbraceleft}{\kern0pt}{\isadigit{0}}{\isachardot}{\kern0pt}{\isachardot}{\kern0pt}{\isacharless}{\kern0pt}s\isactrlsub {\isadigit{1}}{\isacharbraceright}{\kern0pt}\ {\isasymtimes}\ {\isacharbraceleft}{\kern0pt}{\isadigit{0}}{\isachardot}{\kern0pt}{\isachardot}{\kern0pt}{\isacharless}{\kern0pt}s\isactrlsub {\isadigit{2}}{\isacharbraceright}{\kern0pt}\ then\ {\isacharparenleft}{\kern0pt}{\isadigit{0}}{\isacharcomma}{\kern0pt}{\isadigit{0}}{\isacharparenright}{\kern0pt}\ else\ undefined{\isacharparenright}{\kern0pt}{\isacharparenright}{\kern0pt}\ \isanewline
\ \ \ \ {\isacharequal}{\kern0pt}\ {\isacharparenleft}{\kern0pt}t{\isadigit{1}}{\isacharcomma}{\kern0pt}t{\isadigit{2}}{\isacharparenright}{\kern0pt}{\isachardoublequoteclose}\isanewline
\ \ \ \ \isacommand{by}\isamarkupfalse%
\ fastforce\isanewline
\isanewline
\ \ \isacommand{have}\isamarkupfalse%
\ {\isachardoublequoteopen}fst\ {\isacharparenleft}{\kern0pt}fold\ {\isacharparenleft}{\kern0pt}{\isasymlambda}x\ {\isacharparenleft}{\kern0pt}c{\isacharcomma}{\kern0pt}\ state{\isacharparenright}{\kern0pt}{\isachardot}{\kern0pt}\ {\isacharparenleft}{\kern0pt}Suc\ c{\isacharcomma}{\kern0pt}\ state\ {\isasymbind}\ fk{\isacharunderscore}{\kern0pt}update{\isacharprime}{\kern0pt}\ x\ s\isactrlsub {\isadigit{1}}\ s\isactrlsub {\isadigit{2}}\ c{\isacharparenright}{\kern0pt}{\isacharparenright}{\kern0pt}\ xs\ {\isacharparenleft}{\kern0pt}{\isadigit{0}}{\isacharcomma}{\kern0pt}\ return{\isacharunderscore}{\kern0pt}pmf\ {\isacharparenleft}{\kern0pt}{\isasymlambda}i{\isachardot}{\kern0pt}\ if\ i\ {\isasymin}\ {\isacharbraceleft}{\kern0pt}{\isadigit{0}}{\isachardot}{\kern0pt}{\isachardot}{\kern0pt}{\isacharless}{\kern0pt}s\isactrlsub {\isadigit{1}}{\isacharbraceright}{\kern0pt}\ {\isasymtimes}\ {\isacharbraceleft}{\kern0pt}{\isadigit{0}}{\isachardot}{\kern0pt}{\isachardot}{\kern0pt}{\isacharless}{\kern0pt}s\isactrlsub {\isadigit{2}}{\isacharbraceright}{\kern0pt}\ then\ {\isacharparenleft}{\kern0pt}{\isadigit{0}}{\isacharcomma}{\kern0pt}{\isadigit{0}}{\isacharparenright}{\kern0pt}\ else\ undefined{\isacharparenright}{\kern0pt}{\isacharparenright}{\kern0pt}{\isacharparenright}{\kern0pt}\isanewline
\ \ \ \ {\isacharequal}{\kern0pt}\ length\ xs{\isachardoublequoteclose}\isanewline
\ \ \ \ \isacommand{by}\isamarkupfalse%
\ {\isacharparenleft}{\kern0pt}induction\ xs\ rule{\isacharcolon}{\kern0pt}rev{\isacharunderscore}{\kern0pt}induct{\isacharcomma}{\kern0pt}\ simp{\isacharcomma}{\kern0pt}\ simp\ add{\isacharcolon}{\kern0pt}split{\isacharunderscore}{\kern0pt}beta{\isacharparenright}{\kern0pt}\ \isanewline
\ \ \isacommand{hence}\isamarkupfalse%
\ t{\isadigit{1}}{\isacharcolon}{\kern0pt}\ {\isachardoublequoteopen}t{\isadigit{1}}\ {\isacharequal}{\kern0pt}\ length\ xs{\isachardoublequoteclose}\ \isacommand{using}\isamarkupfalse%
\ t\ fst{\isacharunderscore}{\kern0pt}conv\ \isacommand{by}\isamarkupfalse%
\ auto\isanewline
\isanewline
\ \ \isacommand{show}\isamarkupfalse%
\ {\isacharquery}{\kern0pt}case\ \isacommand{using}\isamarkupfalse%
\ snoc\isanewline
\ \ \ \ \isacommand{apply}\isamarkupfalse%
\ {\isacharparenleft}{\kern0pt}simp\ add{\isacharcolon}{\kern0pt}\ \ s\isactrlsub {\isadigit{1}}{\isacharunderscore}{\kern0pt}def{\isacharbrackleft}{\kern0pt}symmetric{\isacharbrackright}{\kern0pt}\ s\isactrlsub {\isadigit{2}}{\isacharunderscore}{\kern0pt}def{\isacharbrackleft}{\kern0pt}symmetric{\isacharbrackright}{\kern0pt}\ t\ del{\isacharcolon}{\kern0pt}fk{\isacharunderscore}{\kern0pt}update{\isacharprime}{\kern0pt}{\isachardot}{\kern0pt}simps\ fk{\isacharunderscore}{\kern0pt}update{\isachardot}{\kern0pt}simps{\isacharparenright}{\kern0pt}\isanewline
\ \ \ \ \isacommand{apply}\isamarkupfalse%
\ {\isacharparenleft}{\kern0pt}subst\ bind{\isacharunderscore}{\kern0pt}map{\isacharunderscore}{\kern0pt}pmf{\isacharparenright}{\kern0pt}\isanewline
\ \ \ \ \isacommand{apply}\isamarkupfalse%
\ {\isacharparenleft}{\kern0pt}subst\ map{\isacharunderscore}{\kern0pt}bind{\isacharunderscore}{\kern0pt}pmf{\isacharparenright}{\kern0pt}\isanewline
\ \ \ \ \isacommand{apply}\isamarkupfalse%
\ simp\isanewline
\ \ \ \ \isacommand{by}\isamarkupfalse%
\ {\isacharparenleft}{\kern0pt}subst\ map{\isacharunderscore}{\kern0pt}bind{\isacharunderscore}{\kern0pt}pmf{\isacharcomma}{\kern0pt}\ simp\ add{\isacharcolon}{\kern0pt}t{\isadigit{1}}{\isacharparenright}{\kern0pt}\isanewline
\isacommand{qed}\isamarkupfalse%
%
\endisatagproof
{\isafoldproof}%
%
\isadelimproof
\isanewline
%
\endisadelimproof
\isanewline
\isacommand{lemma}\isamarkupfalse%
\ power{\isacharunderscore}{\kern0pt}diff{\isacharunderscore}{\kern0pt}sum{\isacharcolon}{\kern0pt}\isanewline
\ \ \isakeyword{assumes}\ {\isachardoublequoteopen}k\ {\isachargreater}{\kern0pt}\ {\isadigit{0}}{\isachardoublequoteclose}\isanewline
\ \ \isakeyword{shows}\ {\isachardoublequoteopen}{\isacharparenleft}{\kern0pt}a\ {\isacharcolon}{\kern0pt}{\isacharcolon}{\kern0pt}\ {\isacharprime}{\kern0pt}a\ {\isacharcolon}{\kern0pt}{\isacharcolon}{\kern0pt}\ {\isacharbraceleft}{\kern0pt}comm{\isacharunderscore}{\kern0pt}ring{\isacharunderscore}{\kern0pt}{\isadigit{1}}{\isacharcomma}{\kern0pt}power{\isacharbraceright}{\kern0pt}{\isacharparenright}{\kern0pt}{\isacharcircum}{\kern0pt}k\ {\isacharminus}{\kern0pt}b{\isacharcircum}{\kern0pt}k\ {\isacharequal}{\kern0pt}\ {\isacharparenleft}{\kern0pt}a{\isacharminus}{\kern0pt}b{\isacharparenright}{\kern0pt}\ {\isacharasterisk}{\kern0pt}\ sum\ {\isacharparenleft}{\kern0pt}{\isasymlambda}i{\isachardot}{\kern0pt}\ a{\isacharcircum}{\kern0pt}i\ {\isacharasterisk}{\kern0pt}\ b{\isacharcircum}{\kern0pt}{\isacharparenleft}{\kern0pt}k{\isacharminus}{\kern0pt}{\isadigit{1}}{\isacharminus}{\kern0pt}i{\isacharparenright}{\kern0pt}{\isacharparenright}{\kern0pt}\ {\isacharbraceleft}{\kern0pt}{\isadigit{0}}{\isachardot}{\kern0pt}{\isachardot}{\kern0pt}{\isacharless}{\kern0pt}k{\isacharbraceright}{\kern0pt}{\isachardoublequoteclose}\ {\isacharparenleft}{\kern0pt}\isakeyword{is}\ {\isachardoublequoteopen}{\isacharquery}{\kern0pt}lhs\ {\isacharequal}{\kern0pt}\ {\isacharquery}{\kern0pt}rhs{\isachardoublequoteclose}{\isacharparenright}{\kern0pt}\isanewline
%
\isadelimproof
%
\endisadelimproof
%
\isatagproof
\isacommand{proof}\isamarkupfalse%
\ {\isacharminus}{\kern0pt}\isanewline
\ \ \isacommand{have}\isamarkupfalse%
\ {\isachardoublequoteopen}{\isacharquery}{\kern0pt}rhs\ {\isacharequal}{\kern0pt}\ sum\ {\isacharparenleft}{\kern0pt}{\isasymlambda}i{\isachardot}{\kern0pt}\ a\ {\isacharasterisk}{\kern0pt}\ {\isacharparenleft}{\kern0pt}a{\isacharcircum}{\kern0pt}i\ {\isacharasterisk}{\kern0pt}\ b{\isacharcircum}{\kern0pt}{\isacharparenleft}{\kern0pt}k{\isacharminus}{\kern0pt}{\isadigit{1}}{\isacharminus}{\kern0pt}i{\isacharparenright}{\kern0pt}{\isacharparenright}{\kern0pt}{\isacharparenright}{\kern0pt}\ {\isacharbraceleft}{\kern0pt}{\isadigit{0}}{\isachardot}{\kern0pt}{\isachardot}{\kern0pt}{\isacharless}{\kern0pt}k{\isacharbraceright}{\kern0pt}\ {\isacharminus}{\kern0pt}\ sum\ {\isacharparenleft}{\kern0pt}{\isasymlambda}i{\isachardot}{\kern0pt}\ b\ {\isacharasterisk}{\kern0pt}\ {\isacharparenleft}{\kern0pt}a{\isacharcircum}{\kern0pt}i\ {\isacharasterisk}{\kern0pt}\ b{\isacharcircum}{\kern0pt}{\isacharparenleft}{\kern0pt}k{\isacharminus}{\kern0pt}{\isadigit{1}}{\isacharminus}{\kern0pt}i{\isacharparenright}{\kern0pt}{\isacharparenright}{\kern0pt}{\isacharparenright}{\kern0pt}\ {\isacharbraceleft}{\kern0pt}{\isadigit{0}}{\isachardot}{\kern0pt}{\isachardot}{\kern0pt}{\isacharless}{\kern0pt}k{\isacharbraceright}{\kern0pt}{\isachardoublequoteclose}\isanewline
\ \ \ \ \isacommand{by}\isamarkupfalse%
\ {\isacharparenleft}{\kern0pt}simp\ add{\isacharcolon}{\kern0pt}\ sum{\isacharunderscore}{\kern0pt}distrib{\isacharunderscore}{\kern0pt}left{\isacharbrackleft}{\kern0pt}symmetric{\isacharbrackright}{\kern0pt}\ algebra{\isacharunderscore}{\kern0pt}simps{\isacharparenright}{\kern0pt}\isanewline
\ \ \isacommand{also}\isamarkupfalse%
\ \isacommand{have}\isamarkupfalse%
\ {\isachardoublequoteopen}{\isachardot}{\kern0pt}{\isachardot}{\kern0pt}{\isachardot}{\kern0pt}\ {\isacharequal}{\kern0pt}\ sum\ {\isacharparenleft}{\kern0pt}{\isacharparenleft}{\kern0pt}{\isasymlambda}i{\isachardot}{\kern0pt}\ {\isacharparenleft}{\kern0pt}a{\isacharcircum}{\kern0pt}i\ {\isacharasterisk}{\kern0pt}\ b{\isacharcircum}{\kern0pt}{\isacharparenleft}{\kern0pt}k{\isacharminus}{\kern0pt}i{\isacharparenright}{\kern0pt}{\isacharparenright}{\kern0pt}{\isacharparenright}{\kern0pt}\ {\isasymcirc}\ {\isacharparenleft}{\kern0pt}{\isasymlambda}i{\isachardot}{\kern0pt}\ i{\isacharplus}{\kern0pt}{\isadigit{1}}{\isacharparenright}{\kern0pt}{\isacharparenright}{\kern0pt}\ {\isacharbraceleft}{\kern0pt}{\isadigit{0}}{\isachardot}{\kern0pt}{\isachardot}{\kern0pt}{\isacharless}{\kern0pt}k{\isacharbraceright}{\kern0pt}\ {\isacharminus}{\kern0pt}\ sum\ {\isacharparenleft}{\kern0pt}{\isasymlambda}i{\isachardot}{\kern0pt}\ {\isacharparenleft}{\kern0pt}a{\isacharcircum}{\kern0pt}i\ {\isacharasterisk}{\kern0pt}\ b{\isacharcircum}{\kern0pt}{\isacharparenleft}{\kern0pt}k{\isacharminus}{\kern0pt}i{\isacharparenright}{\kern0pt}{\isacharparenright}{\kern0pt}{\isacharparenright}{\kern0pt}\ {\isacharbraceleft}{\kern0pt}{\isadigit{0}}{\isachardot}{\kern0pt}{\isachardot}{\kern0pt}{\isacharless}{\kern0pt}k{\isacharbraceright}{\kern0pt}{\isachardoublequoteclose}\isanewline
\ \ \ \ \isacommand{apply}\isamarkupfalse%
\ {\isacharparenleft}{\kern0pt}rule\ arg{\isacharunderscore}{\kern0pt}cong{\isadigit{2}}{\isacharbrackleft}{\kern0pt}\isakeyword{where}\ f{\isacharequal}{\kern0pt}{\isachardoublequoteopen}{\isacharparenleft}{\kern0pt}{\isacharminus}{\kern0pt}{\isacharparenright}{\kern0pt}{\isachardoublequoteclose}{\isacharbrackright}{\kern0pt}{\isacharparenright}{\kern0pt}\isanewline
\ \ \ \ \isacommand{apply}\isamarkupfalse%
\ {\isacharparenleft}{\kern0pt}rule\ sum{\isachardot}{\kern0pt}cong{\isacharcomma}{\kern0pt}\ simp{\isacharcomma}{\kern0pt}\ simp\ add{\isacharcolon}{\kern0pt}algebra{\isacharunderscore}{\kern0pt}simps{\isacharparenright}{\kern0pt}\isanewline
\ \ \ \ \isacommand{apply}\isamarkupfalse%
\ {\isacharparenleft}{\kern0pt}rule\ sum{\isachardot}{\kern0pt}cong{\isacharcomma}{\kern0pt}\ simp{\isacharparenright}{\kern0pt}\isanewline
\ \ \ \ \isacommand{apply}\isamarkupfalse%
\ {\isacharparenleft}{\kern0pt}subst\ mult{\isachardot}{\kern0pt}assoc{\isacharbrackleft}{\kern0pt}symmetric{\isacharbrackright}{\kern0pt}{\isacharcomma}{\kern0pt}\ subst\ mult{\isachardot}{\kern0pt}commute{\isacharcomma}{\kern0pt}\ subst\ mult{\isachardot}{\kern0pt}assoc{\isacharparenright}{\kern0pt}\isanewline
\ \ \ \ \isacommand{by}\isamarkupfalse%
\ {\isacharparenleft}{\kern0pt}rule\ arg{\isacharunderscore}{\kern0pt}cong{\isadigit{2}}{\isacharbrackleft}{\kern0pt}\isakeyword{where}\ f{\isacharequal}{\kern0pt}{\isachardoublequoteopen}{\isacharparenleft}{\kern0pt}{\isacharasterisk}{\kern0pt}{\isacharparenright}{\kern0pt}{\isachardoublequoteclose}{\isacharbrackright}{\kern0pt}{\isacharcomma}{\kern0pt}\ simp{\isacharcomma}{\kern0pt}\ simp\ add{\isacharcolon}{\kern0pt}\ power{\isacharunderscore}{\kern0pt}eq{\isacharunderscore}{\kern0pt}if{\isacharparenright}{\kern0pt}\isanewline
\ \ \isacommand{also}\isamarkupfalse%
\ \isacommand{have}\isamarkupfalse%
\ {\isachardoublequoteopen}{\isachardot}{\kern0pt}{\isachardot}{\kern0pt}{\isachardot}{\kern0pt}\ {\isacharequal}{\kern0pt}\ sum\ {\isacharparenleft}{\kern0pt}{\isasymlambda}i{\isachardot}{\kern0pt}\ {\isacharparenleft}{\kern0pt}a{\isacharcircum}{\kern0pt}i\ {\isacharasterisk}{\kern0pt}\ b{\isacharcircum}{\kern0pt}{\isacharparenleft}{\kern0pt}k{\isacharminus}{\kern0pt}i{\isacharparenright}{\kern0pt}{\isacharparenright}{\kern0pt}{\isacharparenright}{\kern0pt}\ {\isacharparenleft}{\kern0pt}insert\ k\ {\isacharbraceleft}{\kern0pt}{\isadigit{1}}{\isachardot}{\kern0pt}{\isachardot}{\kern0pt}{\isacharless}{\kern0pt}k{\isacharbraceright}{\kern0pt}{\isacharparenright}{\kern0pt}\ {\isacharminus}{\kern0pt}\ sum\ {\isacharparenleft}{\kern0pt}{\isasymlambda}i{\isachardot}{\kern0pt}\ {\isacharparenleft}{\kern0pt}a{\isacharcircum}{\kern0pt}i\ {\isacharasterisk}{\kern0pt}\ b{\isacharcircum}{\kern0pt}{\isacharparenleft}{\kern0pt}k{\isacharminus}{\kern0pt}i{\isacharparenright}{\kern0pt}{\isacharparenright}{\kern0pt}{\isacharparenright}{\kern0pt}\ {\isacharparenleft}{\kern0pt}insert\ {\isadigit{0}}\ {\isacharbraceleft}{\kern0pt}{\isadigit{1}}{\isachardot}{\kern0pt}{\isachardot}{\kern0pt}{\isacharless}{\kern0pt}k{\isacharbraceright}{\kern0pt}{\isacharparenright}{\kern0pt}{\isachardoublequoteclose}\isanewline
\ \ \ \ \isacommand{apply}\isamarkupfalse%
\ {\isacharparenleft}{\kern0pt}rule\ arg{\isacharunderscore}{\kern0pt}cong{\isadigit{2}}{\isacharbrackleft}{\kern0pt}\isakeyword{where}\ f{\isacharequal}{\kern0pt}{\isachardoublequoteopen}{\isacharparenleft}{\kern0pt}{\isacharminus}{\kern0pt}{\isacharparenright}{\kern0pt}{\isachardoublequoteclose}{\isacharbrackright}{\kern0pt}{\isacharparenright}{\kern0pt}\isanewline
\ \ \ \ \isacommand{apply}\isamarkupfalse%
\ {\isacharparenleft}{\kern0pt}subst\ sum{\isachardot}{\kern0pt}reindex{\isacharbrackleft}{\kern0pt}symmetric{\isacharbrackright}{\kern0pt}{\isacharcomma}{\kern0pt}\ simp{\isacharparenright}{\kern0pt}\isanewline
\ \ \ \ \ \isacommand{apply}\isamarkupfalse%
\ {\isacharparenleft}{\kern0pt}rule\ sum{\isachardot}{\kern0pt}cong{\isacharparenright}{\kern0pt}\ \isacommand{using}\isamarkupfalse%
\ assms\ \isacommand{apply}\isamarkupfalse%
\ {\isacharparenleft}{\kern0pt}simp\ add{\isacharcolon}{\kern0pt}atLeastLessThanSuc{\isacharcomma}{\kern0pt}\ simp{\isacharparenright}{\kern0pt}\isanewline
\ \ \ \ \isacommand{apply}\isamarkupfalse%
\ {\isacharparenleft}{\kern0pt}rule\ sum{\isachardot}{\kern0pt}cong{\isacharparenright}{\kern0pt}\ \isacommand{using}\isamarkupfalse%
\ assms\ Icc{\isacharunderscore}{\kern0pt}eq{\isacharunderscore}{\kern0pt}insert{\isacharunderscore}{\kern0pt}lb{\isacharunderscore}{\kern0pt}nat\ \isanewline
\ \ \ \ \ \isacommand{apply}\isamarkupfalse%
\ {\isacharparenleft}{\kern0pt}metis\ One{\isacharunderscore}{\kern0pt}nat{\isacharunderscore}{\kern0pt}def\ Suc{\isacharunderscore}{\kern0pt}pred\ atLeastLessThanSuc{\isacharunderscore}{\kern0pt}atLeastAtMost\ le{\isacharunderscore}{\kern0pt}add{\isadigit{1}}\ le{\isacharunderscore}{\kern0pt}add{\isacharunderscore}{\kern0pt}same{\isacharunderscore}{\kern0pt}cancel{\isadigit{1}}{\isacharparenright}{\kern0pt}\isanewline
\ \ \ \ \isacommand{by}\isamarkupfalse%
\ simp\isanewline
\ \ \isacommand{also}\isamarkupfalse%
\ \isacommand{have}\isamarkupfalse%
\ {\isachardoublequoteopen}{\isachardot}{\kern0pt}{\isachardot}{\kern0pt}{\isachardot}{\kern0pt}\ {\isacharequal}{\kern0pt}\ {\isacharquery}{\kern0pt}lhs{\isachardoublequoteclose}\isanewline
\ \ \ \ \isacommand{by}\isamarkupfalse%
\ simp\isanewline
\ \ \isacommand{finally}\isamarkupfalse%
\ \isacommand{show}\isamarkupfalse%
\ {\isacharquery}{\kern0pt}thesis\ \isacommand{by}\isamarkupfalse%
\ presburger\isanewline
\isacommand{qed}\isamarkupfalse%
%
\endisatagproof
{\isafoldproof}%
%
\isadelimproof
\isanewline
%
\endisadelimproof
\isanewline
\isacommand{lemma}\isamarkupfalse%
\ power{\isacharunderscore}{\kern0pt}diff{\isacharunderscore}{\kern0pt}est{\isacharcolon}{\kern0pt}\isanewline
\ \ \isakeyword{assumes}\ {\isachardoublequoteopen}k\ {\isachargreater}{\kern0pt}\ {\isadigit{0}}{\isachardoublequoteclose}\isanewline
\ \ \isakeyword{assumes}\ {\isachardoublequoteopen}{\isacharparenleft}{\kern0pt}a\ {\isacharcolon}{\kern0pt}{\isacharcolon}{\kern0pt}\ real{\isacharparenright}{\kern0pt}\ {\isasymge}\ b{\isachardoublequoteclose}\isanewline
\ \ \isakeyword{assumes}\ {\isachardoublequoteopen}b\ {\isasymge}\ {\isadigit{0}}{\isachardoublequoteclose}\isanewline
\ \ \isakeyword{shows}\ {\isachardoublequoteopen}a{\isacharcircum}{\kern0pt}k\ {\isacharminus}{\kern0pt}b{\isacharcircum}{\kern0pt}k\ {\isasymle}\ {\isacharparenleft}{\kern0pt}a{\isacharminus}{\kern0pt}b{\isacharparenright}{\kern0pt}\ {\isacharasterisk}{\kern0pt}\ k\ {\isacharasterisk}{\kern0pt}\ a{\isacharcircum}{\kern0pt}{\isacharparenleft}{\kern0pt}k{\isacharminus}{\kern0pt}{\isadigit{1}}{\isacharparenright}{\kern0pt}{\isachardoublequoteclose}\isanewline
%
\isadelimproof
%
\endisadelimproof
%
\isatagproof
\isacommand{proof}\isamarkupfalse%
\ {\isacharminus}{\kern0pt}\isanewline
\ \ \isacommand{have}\isamarkupfalse%
\ {\isachardoublequoteopen}\ {\isasymAnd}i{\isachardot}{\kern0pt}\ i\ {\isacharless}{\kern0pt}\ k\ {\isasymLongrightarrow}\ a\ {\isacharcircum}{\kern0pt}\ i\ {\isacharasterisk}{\kern0pt}\ b\ {\isacharcircum}{\kern0pt}\ {\isacharparenleft}{\kern0pt}k\ {\isacharminus}{\kern0pt}\ {\isadigit{1}}\ {\isacharminus}{\kern0pt}\ i{\isacharparenright}{\kern0pt}\ {\isasymle}\ a\ {\isacharcircum}{\kern0pt}\ i\ {\isacharasterisk}{\kern0pt}\ a\ {\isacharcircum}{\kern0pt}\ {\isacharparenleft}{\kern0pt}k{\isacharminus}{\kern0pt}{\isadigit{1}}{\isacharminus}{\kern0pt}i{\isacharparenright}{\kern0pt}{\isachardoublequoteclose}\isanewline
\ \ \ \ \isacommand{apply}\isamarkupfalse%
\ {\isacharparenleft}{\kern0pt}rule\ mult{\isacharunderscore}{\kern0pt}left{\isacharunderscore}{\kern0pt}mono{\isacharcomma}{\kern0pt}\ rule\ power{\isacharunderscore}{\kern0pt}mono{\isacharcomma}{\kern0pt}\ metis\ assms{\isacharparenleft}{\kern0pt}{\isadigit{2}}{\isacharparenright}{\kern0pt}{\isacharcomma}{\kern0pt}\ metis\ assms{\isacharparenleft}{\kern0pt}{\isadigit{3}}{\isacharparenright}{\kern0pt}{\isacharparenright}{\kern0pt}\isanewline
\ \ \ \ \isacommand{using}\isamarkupfalse%
\ assms\ \isacommand{by}\isamarkupfalse%
\ simp\isanewline
\ \ \isacommand{also}\isamarkupfalse%
\ \isacommand{have}\isamarkupfalse%
\ {\isachardoublequoteopen}{\isasymAnd}i{\isachardot}{\kern0pt}\ i\ {\isacharless}{\kern0pt}\ k\ {\isasymLongrightarrow}\ a\ {\isacharcircum}{\kern0pt}\ i\ {\isacharasterisk}{\kern0pt}\ a\ {\isacharcircum}{\kern0pt}\ {\isacharparenleft}{\kern0pt}k\ {\isacharminus}{\kern0pt}\ {\isadigit{1}}\ {\isacharminus}{\kern0pt}\ i{\isacharparenright}{\kern0pt}\ {\isacharequal}{\kern0pt}\ a\ {\isacharcircum}{\kern0pt}\ {\isacharparenleft}{\kern0pt}k\ {\isacharminus}{\kern0pt}\ Suc\ {\isadigit{0}}{\isacharparenright}{\kern0pt}{\isachardoublequoteclose}\isanewline
\ \ \ \ \isacommand{apply}\isamarkupfalse%
\ {\isacharparenleft}{\kern0pt}subst\ power{\isacharunderscore}{\kern0pt}add{\isacharbrackleft}{\kern0pt}symmetric{\isacharbrackright}{\kern0pt}{\isacharparenright}{\kern0pt}\isanewline
\ \ \ \ \isacommand{apply}\isamarkupfalse%
\ {\isacharparenleft}{\kern0pt}rule\ arg{\isacharunderscore}{\kern0pt}cong{\isadigit{2}}{\isacharbrackleft}{\kern0pt}\isakeyword{where}\ f{\isacharequal}{\kern0pt}{\isachardoublequoteopen}power{\isachardoublequoteclose}{\isacharbrackright}{\kern0pt}{\isacharcomma}{\kern0pt}\ simp{\isacharparenright}{\kern0pt}\isanewline
\ \ \ \ \isacommand{using}\isamarkupfalse%
\ assms{\isacharparenleft}{\kern0pt}{\isadigit{1}}{\isacharparenright}{\kern0pt}\ \isacommand{by}\isamarkupfalse%
\ simp\isanewline
\ \ \isacommand{finally}\isamarkupfalse%
\ \isacommand{have}\isamarkupfalse%
\ t{\isacharcolon}{\kern0pt}\ {\isachardoublequoteopen}{\isasymAnd}i{\isachardot}{\kern0pt}\ i\ {\isacharless}{\kern0pt}\ k\ {\isasymLongrightarrow}\ a\ {\isacharcircum}{\kern0pt}\ i\ {\isacharasterisk}{\kern0pt}\ b\ {\isacharcircum}{\kern0pt}\ {\isacharparenleft}{\kern0pt}k\ {\isacharminus}{\kern0pt}\ {\isadigit{1}}\ {\isacharminus}{\kern0pt}\ i{\isacharparenright}{\kern0pt}\ {\isasymle}\ a\ {\isacharcircum}{\kern0pt}\ {\isacharparenleft}{\kern0pt}k\ {\isacharminus}{\kern0pt}\ Suc\ {\isadigit{0}}{\isacharparenright}{\kern0pt}{\isachardoublequoteclose}\isanewline
\ \ \ \ \isacommand{by}\isamarkupfalse%
\ blast\isanewline
\ \ \isacommand{have}\isamarkupfalse%
\ {\isachardoublequoteopen}a{\isacharcircum}{\kern0pt}k\ {\isacharminus}{\kern0pt}\ b{\isacharcircum}{\kern0pt}k\ {\isacharequal}{\kern0pt}\ {\isacharparenleft}{\kern0pt}a{\isacharminus}{\kern0pt}b{\isacharparenright}{\kern0pt}\ {\isacharasterisk}{\kern0pt}\ sum\ \ {\isacharparenleft}{\kern0pt}{\isasymlambda}i{\isachardot}{\kern0pt}\ a{\isacharcircum}{\kern0pt}i\ {\isacharasterisk}{\kern0pt}\ b{\isacharcircum}{\kern0pt}{\isacharparenleft}{\kern0pt}k{\isacharminus}{\kern0pt}{\isadigit{1}}{\isacharminus}{\kern0pt}i{\isacharparenright}{\kern0pt}{\isacharparenright}{\kern0pt}\ {\isacharbraceleft}{\kern0pt}{\isadigit{0}}{\isachardot}{\kern0pt}{\isachardot}{\kern0pt}{\isacharless}{\kern0pt}k{\isacharbraceright}{\kern0pt}{\isachardoublequoteclose}\isanewline
\ \ \ \ \isacommand{by}\isamarkupfalse%
\ {\isacharparenleft}{\kern0pt}rule\ power{\isacharunderscore}{\kern0pt}diff{\isacharunderscore}{\kern0pt}sum{\isacharbrackleft}{\kern0pt}OF\ assms{\isacharparenleft}{\kern0pt}{\isadigit{1}}{\isacharparenright}{\kern0pt}{\isacharbrackright}{\kern0pt}{\isacharparenright}{\kern0pt}\isanewline
\ \ \isacommand{also}\isamarkupfalse%
\ \isacommand{have}\isamarkupfalse%
\ {\isachardoublequoteopen}{\isachardot}{\kern0pt}{\isachardot}{\kern0pt}{\isachardot}{\kern0pt}\ {\isasymle}\ {\isacharparenleft}{\kern0pt}a{\isacharminus}{\kern0pt}b{\isacharparenright}{\kern0pt}\ {\isacharasterisk}{\kern0pt}\ k\ {\isacharasterisk}{\kern0pt}\ a{\isacharcircum}{\kern0pt}{\isacharparenleft}{\kern0pt}k{\isacharminus}{\kern0pt}Suc\ {\isadigit{0}}{\isacharparenright}{\kern0pt}{\isachardoublequoteclose}\isanewline
\ \ \ \ \isacommand{apply}\isamarkupfalse%
\ {\isacharparenleft}{\kern0pt}subst\ mult{\isachardot}{\kern0pt}assoc{\isacharparenright}{\kern0pt}\isanewline
\ \ \ \ \isacommand{apply}\isamarkupfalse%
\ {\isacharparenleft}{\kern0pt}rule\ mult{\isacharunderscore}{\kern0pt}left{\isacharunderscore}{\kern0pt}mono{\isacharparenright}{\kern0pt}\isanewline
\ \ \ \ \ \isacommand{apply}\isamarkupfalse%
\ {\isacharparenleft}{\kern0pt}rule\ sum{\isacharunderscore}{\kern0pt}mono{\isacharbrackleft}{\kern0pt}\isakeyword{where}\ g{\isacharequal}{\kern0pt}{\isachardoublequoteopen}{\isasymlambda}{\isacharunderscore}{\kern0pt}{\isachardot}{\kern0pt}\ a{\isacharcircum}{\kern0pt}{\isacharparenleft}{\kern0pt}k{\isacharminus}{\kern0pt}{\isadigit{1}}{\isacharparenright}{\kern0pt}{\isachardoublequoteclose}\ \isakeyword{and}\ K{\isacharequal}{\kern0pt}{\isachardoublequoteopen}{\isacharbraceleft}{\kern0pt}{\isadigit{0}}{\isachardot}{\kern0pt}{\isachardot}{\kern0pt}{\isacharless}{\kern0pt}k{\isacharbraceright}{\kern0pt}{\isachardoublequoteclose}{\isacharcomma}{\kern0pt}\ simplified{\isacharbrackright}{\kern0pt}{\isacharparenright}{\kern0pt}\isanewline
\ \ \ \ \ \isacommand{apply}\isamarkupfalse%
\ {\isacharparenleft}{\kern0pt}metis\ t{\isacharparenright}{\kern0pt}\isanewline
\ \ \ \ \isacommand{using}\isamarkupfalse%
\ assms{\isacharparenleft}{\kern0pt}{\isadigit{2}}{\isacharparenright}{\kern0pt}\ \isacommand{by}\isamarkupfalse%
\ auto\isanewline
\ \ \isacommand{finally}\isamarkupfalse%
\ \isacommand{show}\isamarkupfalse%
\ {\isacharquery}{\kern0pt}thesis\ \isacommand{by}\isamarkupfalse%
\ simp\isanewline
\isacommand{qed}\isamarkupfalse%
%
\endisatagproof
{\isafoldproof}%
%
\isadelimproof
%
\endisadelimproof
%
\begin{isamarkuptext}%
Specialization of the Hoelder inquality for sums.%
\end{isamarkuptext}\isamarkuptrue%
\isacommand{lemma}\isamarkupfalse%
\ Holder{\isacharunderscore}{\kern0pt}inequality{\isacharunderscore}{\kern0pt}sum{\isacharcolon}{\kern0pt}\isanewline
\ \ \isakeyword{assumes}\ {\isachardoublequoteopen}p\ {\isachargreater}{\kern0pt}\ {\isacharparenleft}{\kern0pt}{\isadigit{0}}{\isacharcolon}{\kern0pt}{\isacharcolon}{\kern0pt}real{\isacharparenright}{\kern0pt}{\isachardoublequoteclose}\ {\isachardoublequoteopen}q\ {\isachargreater}{\kern0pt}\ {\isadigit{0}}{\isachardoublequoteclose}\ {\isachardoublequoteopen}{\isadigit{1}}{\isacharslash}{\kern0pt}p\ {\isacharplus}{\kern0pt}\ {\isadigit{1}}{\isacharslash}{\kern0pt}q\ {\isacharequal}{\kern0pt}\ {\isadigit{1}}{\isachardoublequoteclose}\isanewline
\ \ \isakeyword{assumes}\ {\isachardoublequoteopen}finite\ A{\isachardoublequoteclose}\isanewline
\ \ \isakeyword{shows}\ {\isachardoublequoteopen}{\isasymbar}sum\ {\isacharparenleft}{\kern0pt}{\isasymlambda}x{\isachardot}{\kern0pt}\ f\ x\ {\isacharasterisk}{\kern0pt}\ g\ x{\isacharparenright}{\kern0pt}\ A{\isasymbar}\ {\isasymle}\ {\isacharparenleft}{\kern0pt}sum\ {\isacharparenleft}{\kern0pt}{\isasymlambda}x{\isachardot}{\kern0pt}\ {\isasymbar}f\ x{\isasymbar}\ powr\ p{\isacharparenright}{\kern0pt}\ A{\isacharparenright}{\kern0pt}\ powr\ {\isacharparenleft}{\kern0pt}{\isadigit{1}}{\isacharslash}{\kern0pt}p{\isacharparenright}{\kern0pt}\ {\isacharasterisk}{\kern0pt}\ {\isacharparenleft}{\kern0pt}sum\ {\isacharparenleft}{\kern0pt}{\isasymlambda}x{\isachardot}{\kern0pt}\ {\isasymbar}g\ x{\isasymbar}\ powr\ q{\isacharparenright}{\kern0pt}\ A{\isacharparenright}{\kern0pt}\ powr\ {\isacharparenleft}{\kern0pt}{\isadigit{1}}{\isacharslash}{\kern0pt}q{\isacharparenright}{\kern0pt}{\isachardoublequoteclose}\isanewline
%
\isadelimproof
\ \ %
\endisadelimproof
%
\isatagproof
\isacommand{using}\isamarkupfalse%
\ assms\ \isacommand{apply}\isamarkupfalse%
\ {\isacharparenleft}{\kern0pt}simp\ add{\isacharcolon}{\kern0pt}\ lebesgue{\isacharunderscore}{\kern0pt}integral{\isacharunderscore}{\kern0pt}count{\isacharunderscore}{\kern0pt}space{\isacharunderscore}{\kern0pt}finite{\isacharbrackleft}{\kern0pt}symmetric{\isacharbrackright}{\kern0pt}{\isacharparenright}{\kern0pt}\isanewline
\ \ \isacommand{apply}\isamarkupfalse%
\ {\isacharparenleft}{\kern0pt}rule\ Lp{\isachardot}{\kern0pt}Holder{\isacharunderscore}{\kern0pt}inequality{\isacharparenright}{\kern0pt}\isanewline
\ \ \isacommand{by}\isamarkupfalse%
\ {\isacharparenleft}{\kern0pt}simp\ add{\isacharcolon}{\kern0pt}integrable{\isacharunderscore}{\kern0pt}count{\isacharunderscore}{\kern0pt}space{\isacharparenright}{\kern0pt}{\isacharplus}{\kern0pt}%
\endisatagproof
{\isafoldproof}%
%
\isadelimproof
\isanewline
%
\endisadelimproof
\isanewline
\isacommand{lemma}\isamarkupfalse%
\ fk{\isacharunderscore}{\kern0pt}estimate{\isacharcolon}{\kern0pt}\isanewline
\ \ \isakeyword{assumes}\ {\isachardoublequoteopen}as\ {\isasymnoteq}\ {\isacharbrackleft}{\kern0pt}{\isacharbrackright}{\kern0pt}{\isachardoublequoteclose}\isanewline
\ \ \isakeyword{assumes}\ {\isachardoublequoteopen}set\ as\ {\isasymsubseteq}\ {\isacharbraceleft}{\kern0pt}{\isadigit{0}}{\isachardot}{\kern0pt}{\isachardot}{\kern0pt}{\isacharless}{\kern0pt}n{\isacharbraceright}{\kern0pt}{\isachardoublequoteclose}\isanewline
\ \ \isakeyword{assumes}\ {\isachardoublequoteopen}k\ {\isasymge}\ {\isadigit{1}}{\isachardoublequoteclose}\isanewline
\ \ \isakeyword{shows}\ {\isachardoublequoteopen}real\ {\isacharparenleft}{\kern0pt}length\ as{\isacharparenright}{\kern0pt}\ {\isacharasterisk}{\kern0pt}\ real{\isacharunderscore}{\kern0pt}of{\isacharunderscore}{\kern0pt}rat\ {\isacharparenleft}{\kern0pt}F\ {\isacharparenleft}{\kern0pt}{\isadigit{2}}{\isacharasterisk}{\kern0pt}k{\isacharminus}{\kern0pt}{\isadigit{1}}{\isacharparenright}{\kern0pt}\ as{\isacharparenright}{\kern0pt}\ {\isasymle}\ real\ n\ powr\ {\isacharparenleft}{\kern0pt}{\isadigit{1}}\ {\isacharminus}{\kern0pt}\ {\isadigit{1}}\ {\isacharslash}{\kern0pt}\ real\ k{\isacharparenright}{\kern0pt}\ {\isacharasterisk}{\kern0pt}\ {\isacharparenleft}{\kern0pt}real{\isacharunderscore}{\kern0pt}of{\isacharunderscore}{\kern0pt}rat\ {\isacharparenleft}{\kern0pt}F\ k\ as{\isacharparenright}{\kern0pt}{\isacharparenright}{\kern0pt}{\isacharcircum}{\kern0pt}{\isadigit{2}}{\isachardoublequoteclose}\isanewline
\ \ {\isacharparenleft}{\kern0pt}\isakeyword{is}\ {\isachardoublequoteopen}{\isacharquery}{\kern0pt}lhs\ {\isasymle}\ {\isacharquery}{\kern0pt}rhs{\isachardoublequoteclose}{\isacharparenright}{\kern0pt}\isanewline
%
\isadelimproof
%
\endisadelimproof
%
\isatagproof
\isacommand{proof}\isamarkupfalse%
\ {\isacharparenleft}{\kern0pt}cases\ {\isachardoublequoteopen}k\ {\isasymge}\ {\isadigit{2}}{\isachardoublequoteclose}{\isacharparenright}{\kern0pt}\isanewline
\ \ \isacommand{case}\isamarkupfalse%
\ True\isanewline
\ \ \isacommand{define}\isamarkupfalse%
\ M\ \isakeyword{where}\ {\isachardoublequoteopen}M\ {\isacharequal}{\kern0pt}\ Max\ {\isacharparenleft}{\kern0pt}count{\isacharunderscore}{\kern0pt}list\ as\ {\isacharbackquote}{\kern0pt}\ set\ as{\isacharparenright}{\kern0pt}{\isachardoublequoteclose}\ \isanewline
\ \ \isacommand{then}\isamarkupfalse%
\ \isacommand{obtain}\isamarkupfalse%
\ m\ \isakeyword{where}\ m{\isacharunderscore}{\kern0pt}in{\isacharcolon}{\kern0pt}\ {\isachardoublequoteopen}m\ {\isasymin}\ set\ as{\isachardoublequoteclose}\ \isakeyword{and}\ m{\isacharunderscore}{\kern0pt}def{\isacharcolon}{\kern0pt}\ {\isachardoublequoteopen}M\ {\isacharequal}{\kern0pt}\ count{\isacharunderscore}{\kern0pt}list\ as\ m{\isachardoublequoteclose}\isanewline
\ \ \ \ \isacommand{by}\isamarkupfalse%
\ {\isacharparenleft}{\kern0pt}metis\ {\isacharparenleft}{\kern0pt}mono{\isacharunderscore}{\kern0pt}tags{\isacharcomma}{\kern0pt}\ lifting{\isacharparenright}{\kern0pt}\ List{\isachardot}{\kern0pt}finite{\isacharunderscore}{\kern0pt}set\ Max{\isacharunderscore}{\kern0pt}in\ finite{\isacharunderscore}{\kern0pt}imageI\ image{\isacharunderscore}{\kern0pt}iff\ image{\isacharunderscore}{\kern0pt}is{\isacharunderscore}{\kern0pt}empty\ set{\isacharunderscore}{\kern0pt}empty\ assms{\isacharparenleft}{\kern0pt}{\isadigit{1}}{\isacharparenright}{\kern0pt}{\isacharparenright}{\kern0pt}\isanewline
\isanewline
\ \ \isacommand{have}\isamarkupfalse%
\ a{\isadigit{2}}{\isacharcolon}{\kern0pt}\ {\isachardoublequoteopen}real\ M\ {\isachargreater}{\kern0pt}\ {\isadigit{0}}{\isachardoublequoteclose}\ \isacommand{apply}\isamarkupfalse%
\ {\isacharparenleft}{\kern0pt}simp\ add{\isacharcolon}{\kern0pt}M{\isacharunderscore}{\kern0pt}def{\isacharparenright}{\kern0pt}\isanewline
\ \ \ \ \isacommand{by}\isamarkupfalse%
\ {\isacharparenleft}{\kern0pt}metis\ {\isacharparenleft}{\kern0pt}mono{\isacharunderscore}{\kern0pt}tags{\isacharcomma}{\kern0pt}\ opaque{\isacharunderscore}{\kern0pt}lifting{\isacharparenright}{\kern0pt}\ List{\isachardot}{\kern0pt}finite{\isacharunderscore}{\kern0pt}set\ assms{\isacharparenleft}{\kern0pt}{\isadigit{1}}{\isacharparenright}{\kern0pt}\ Max{\isacharunderscore}{\kern0pt}in\ bot{\isacharunderscore}{\kern0pt}nat{\isacharunderscore}{\kern0pt}{\isadigit{0}}{\isachardot}{\kern0pt}not{\isacharunderscore}{\kern0pt}eq{\isacharunderscore}{\kern0pt}extremum\ count{\isacharunderscore}{\kern0pt}list{\isacharunderscore}{\kern0pt}gr{\isacharunderscore}{\kern0pt}{\isadigit{1}}\ finite{\isacharunderscore}{\kern0pt}imageI\ imageE\ image{\isacharunderscore}{\kern0pt}is{\isacharunderscore}{\kern0pt}empty\ linorder{\isacharunderscore}{\kern0pt}not{\isacharunderscore}{\kern0pt}less\ set{\isacharunderscore}{\kern0pt}empty\ zero{\isacharunderscore}{\kern0pt}less{\isacharunderscore}{\kern0pt}one{\isacharparenright}{\kern0pt}\isanewline
\ \ \isacommand{have}\isamarkupfalse%
\ a{\isadigit{1}}{\isacharcolon}{\kern0pt}\ {\isachardoublequoteopen}{\isadigit{2}}{\isacharasterisk}{\kern0pt}k{\isacharminus}{\kern0pt}{\isadigit{1}}\ {\isacharequal}{\kern0pt}\ {\isacharparenleft}{\kern0pt}k{\isacharminus}{\kern0pt}{\isadigit{1}}{\isacharparenright}{\kern0pt}\ {\isacharplus}{\kern0pt}\ k{\isachardoublequoteclose}\ \isacommand{by}\isamarkupfalse%
\ simp\isanewline
\ \ \isacommand{have}\isamarkupfalse%
\ a{\isadigit{4}}{\isacharcolon}{\kern0pt}\ {\isachardoublequoteopen}{\isacharparenleft}{\kern0pt}k{\isacharminus}{\kern0pt}{\isadigit{1}}{\isacharparenright}{\kern0pt}\ {\isacharequal}{\kern0pt}\ k\ {\isacharasterisk}{\kern0pt}\ {\isacharparenleft}{\kern0pt}{\isacharparenleft}{\kern0pt}k{\isacharminus}{\kern0pt}{\isadigit{1}}{\isacharparenright}{\kern0pt}{\isacharslash}{\kern0pt}k{\isacharparenright}{\kern0pt}{\isachardoublequoteclose}\ \isacommand{by}\isamarkupfalse%
\ simp\isanewline
\isanewline
\ \ \isacommand{have}\isamarkupfalse%
\ a{\isadigit{3}}{\isacharcolon}{\kern0pt}\ {\isachardoublequoteopen}M\ powr\ k\ {\isasymle}\ real{\isacharunderscore}{\kern0pt}of{\isacharunderscore}{\kern0pt}rat\ {\isacharparenleft}{\kern0pt}F\ k\ as{\isacharparenright}{\kern0pt}{\isachardoublequoteclose}\isanewline
\ \ \ \ \isacommand{apply}\isamarkupfalse%
\ {\isacharparenleft}{\kern0pt}simp\ add{\isacharcolon}{\kern0pt}m{\isacharunderscore}{\kern0pt}def\ F{\isacharunderscore}{\kern0pt}def\ of{\isacharunderscore}{\kern0pt}rat{\isacharunderscore}{\kern0pt}sum\ of{\isacharunderscore}{\kern0pt}rat{\isacharunderscore}{\kern0pt}power{\isacharparenright}{\kern0pt}\isanewline
\ \ \ \ \isacommand{apply}\isamarkupfalse%
\ {\isacharparenleft}{\kern0pt}subst\ powr{\isacharunderscore}{\kern0pt}realpow{\isacharcomma}{\kern0pt}\ simp{\isacharparenright}{\kern0pt}\ \isanewline
\ \ \ \ \isacommand{using}\isamarkupfalse%
\ m{\isacharunderscore}{\kern0pt}in\ \ count{\isacharunderscore}{\kern0pt}list{\isacharunderscore}{\kern0pt}gr{\isacharunderscore}{\kern0pt}{\isadigit{1}}\ \isacommand{apply}\isamarkupfalse%
\ force\isanewline
\ \ \ \ \isacommand{by}\isamarkupfalse%
\ {\isacharparenleft}{\kern0pt}rule\ member{\isacharunderscore}{\kern0pt}le{\isacharunderscore}{\kern0pt}sum{\isacharcomma}{\kern0pt}\ metis\ m{\isacharunderscore}{\kern0pt}in{\isacharcomma}{\kern0pt}\ simp{\isacharcomma}{\kern0pt}\ simp{\isacharparenright}{\kern0pt}\isanewline
\isanewline
\ \ \isacommand{have}\isamarkupfalse%
\ a{\isadigit{5}}{\isacharcolon}{\kern0pt}\ {\isachardoublequoteopen}{\isadigit{0}}\ {\isasymle}\ real{\isacharunderscore}{\kern0pt}of{\isacharunderscore}{\kern0pt}rat\ {\isacharparenleft}{\kern0pt}F\ k\ as{\isacharparenright}{\kern0pt}{\isachardoublequoteclose}\ \isanewline
\ \ \ \ \isacommand{using}\isamarkupfalse%
\ F{\isacharunderscore}{\kern0pt}gr{\isacharunderscore}{\kern0pt}{\isadigit{0}}{\isacharbrackleft}{\kern0pt}OF\ assms{\isacharparenleft}{\kern0pt}{\isadigit{1}}{\isacharparenright}{\kern0pt}{\isacharbrackright}{\kern0pt}\ \isanewline
\ \ \ \ \isacommand{by}\isamarkupfalse%
\ {\isacharparenleft}{\kern0pt}simp\ add{\isacharcolon}{\kern0pt}\ order{\isacharunderscore}{\kern0pt}le{\isacharunderscore}{\kern0pt}less{\isacharparenright}{\kern0pt}\isanewline
\ \ \isacommand{hence}\isamarkupfalse%
\ a{\isadigit{6}}{\isacharcolon}{\kern0pt}\ {\isachardoublequoteopen}real{\isacharunderscore}{\kern0pt}of{\isacharunderscore}{\kern0pt}rat\ {\isacharparenleft}{\kern0pt}F\ k\ as{\isacharparenright}{\kern0pt}\ {\isacharequal}{\kern0pt}\ real{\isacharunderscore}{\kern0pt}of{\isacharunderscore}{\kern0pt}rat\ {\isacharparenleft}{\kern0pt}F\ k\ as{\isacharparenright}{\kern0pt}\ powr\ {\isadigit{1}}{\isachardoublequoteclose}\ \isacommand{by}\isamarkupfalse%
\ simp\isanewline
\isanewline
\ \ \isacommand{have}\isamarkupfalse%
\ {\isachardoublequoteopen}real\ {\isacharparenleft}{\kern0pt}k\ {\isacharminus}{\kern0pt}\ {\isadigit{1}}{\isacharparenright}{\kern0pt}\ {\isacharslash}{\kern0pt}\ real\ k\ {\isacharplus}{\kern0pt}\ {\isadigit{1}}\ {\isacharequal}{\kern0pt}\ real\ {\isacharparenleft}{\kern0pt}k\ {\isacharminus}{\kern0pt}\ {\isadigit{1}}{\isacharparenright}{\kern0pt}\ {\isacharslash}{\kern0pt}\ real\ k\ {\isacharplus}{\kern0pt}\ real\ k\ {\isacharslash}{\kern0pt}\ real\ k{\isachardoublequoteclose}\isanewline
\ \ \ \ \isacommand{using}\isamarkupfalse%
\ assms\ True\ \isacommand{by}\isamarkupfalse%
\ simp\isanewline
\ \ \isacommand{also}\isamarkupfalse%
\ \isacommand{have}\isamarkupfalse%
\ {\isachardoublequoteopen}{\isachardot}{\kern0pt}{\isachardot}{\kern0pt}{\isachardot}{\kern0pt}\ {\isacharequal}{\kern0pt}\ \ real\ {\isacharparenleft}{\kern0pt}{\isadigit{2}}\ {\isacharasterisk}{\kern0pt}\ k\ {\isacharminus}{\kern0pt}\ {\isadigit{1}}{\isacharparenright}{\kern0pt}\ {\isacharslash}{\kern0pt}\ real\ k{\isachardoublequoteclose}\isanewline
\ \ \ \ \isacommand{apply}\isamarkupfalse%
\ {\isacharparenleft}{\kern0pt}subst\ add{\isacharunderscore}{\kern0pt}divide{\isacharunderscore}{\kern0pt}distrib{\isacharbrackleft}{\kern0pt}symmetric{\isacharbrackright}{\kern0pt}{\isacharparenright}{\kern0pt}\isanewline
\ \ \ \ \isacommand{apply}\isamarkupfalse%
\ {\isacharparenleft}{\kern0pt}rule\ arg{\isacharunderscore}{\kern0pt}cong{\isadigit{2}}{\isacharbrackleft}{\kern0pt}\isakeyword{where}\ f{\isacharequal}{\kern0pt}{\isachardoublequoteopen}{\isacharparenleft}{\kern0pt}{\isacharslash}{\kern0pt}{\isacharparenright}{\kern0pt}{\isachardoublequoteclose}{\isacharbrackright}{\kern0pt}{\isacharparenright}{\kern0pt}\isanewline
\ \ \ \ \isacommand{apply}\isamarkupfalse%
\ {\isacharparenleft}{\kern0pt}subst\ of{\isacharunderscore}{\kern0pt}nat{\isacharunderscore}{\kern0pt}diff{\isacharparenright}{\kern0pt}\ \isacommand{using}\isamarkupfalse%
\ True\ \isacommand{apply}\isamarkupfalse%
\ linarith\isanewline
\ \ \ \ \isacommand{apply}\isamarkupfalse%
\ {\isacharparenleft}{\kern0pt}subst\ of{\isacharunderscore}{\kern0pt}nat{\isacharunderscore}{\kern0pt}diff{\isacharparenright}{\kern0pt}\ \isacommand{using}\isamarkupfalse%
\ True\ \isacommand{apply}\isamarkupfalse%
\ linarith\isanewline
\ \ \ \ \isacommand{by}\isamarkupfalse%
\ simp{\isacharplus}{\kern0pt}\isanewline
\ \ \isacommand{finally}\isamarkupfalse%
\ \isacommand{have}\isamarkupfalse%
\ a{\isadigit{7}}{\isacharcolon}{\kern0pt}\ {\isachardoublequoteopen}real\ {\isacharparenleft}{\kern0pt}k\ {\isacharminus}{\kern0pt}\ {\isadigit{1}}{\isacharparenright}{\kern0pt}\ {\isacharslash}{\kern0pt}\ real\ k\ {\isacharplus}{\kern0pt}\ {\isadigit{1}}\ {\isacharequal}{\kern0pt}\ real\ {\isacharparenleft}{\kern0pt}{\isadigit{2}}\ {\isacharasterisk}{\kern0pt}\ k\ {\isacharminus}{\kern0pt}\ {\isadigit{1}}{\isacharparenright}{\kern0pt}\ {\isacharslash}{\kern0pt}\ real\ k{\isachardoublequoteclose}\isanewline
\ \ \ \ \isacommand{by}\isamarkupfalse%
\ blast\isanewline
\isanewline
\ \ \isacommand{have}\isamarkupfalse%
\ a{\isacharcolon}{\kern0pt}\ {\isachardoublequoteopen}real{\isacharunderscore}{\kern0pt}of{\isacharunderscore}{\kern0pt}rat\ {\isacharparenleft}{\kern0pt}F\ {\isacharparenleft}{\kern0pt}{\isadigit{2}}{\isacharasterisk}{\kern0pt}k{\isacharminus}{\kern0pt}{\isadigit{1}}{\isacharparenright}{\kern0pt}\ as{\isacharparenright}{\kern0pt}\ {\isasymle}\ M\ powr\ {\isacharparenleft}{\kern0pt}k{\isacharminus}{\kern0pt}{\isadigit{1}}{\isacharparenright}{\kern0pt}\ {\isacharasterisk}{\kern0pt}\ {\isacharparenleft}{\kern0pt}real{\isacharunderscore}{\kern0pt}of{\isacharunderscore}{\kern0pt}rat\ {\isacharparenleft}{\kern0pt}F\ k\ as{\isacharparenright}{\kern0pt}{\isacharparenright}{\kern0pt}\ {\isachardoublequoteclose}\isanewline
\ \ \ \ \isacommand{using}\isamarkupfalse%
\ a{\isadigit{1}}\ \isacommand{apply}\isamarkupfalse%
\ {\isacharparenleft}{\kern0pt}simp\ add{\isacharcolon}{\kern0pt}F{\isacharunderscore}{\kern0pt}def\ of{\isacharunderscore}{\kern0pt}rat{\isacharunderscore}{\kern0pt}sum\ sum{\isacharunderscore}{\kern0pt}distrib{\isacharunderscore}{\kern0pt}left\ of{\isacharunderscore}{\kern0pt}rat{\isacharunderscore}{\kern0pt}mult\ power{\isacharunderscore}{\kern0pt}add\ of{\isacharunderscore}{\kern0pt}rat{\isacharunderscore}{\kern0pt}power{\isacharparenright}{\kern0pt}\isanewline
\ \ \ \ \isacommand{apply}\isamarkupfalse%
\ {\isacharparenleft}{\kern0pt}rule\ sum{\isacharunderscore}{\kern0pt}mono{\isacharparenright}{\kern0pt}\isanewline
\ \ \ \ \isacommand{apply}\isamarkupfalse%
\ {\isacharparenleft}{\kern0pt}rule\ mult{\isacharunderscore}{\kern0pt}right{\isacharunderscore}{\kern0pt}mono{\isacharparenright}{\kern0pt}\isanewline
\ \ \ \ \ \isacommand{apply}\isamarkupfalse%
\ {\isacharparenleft}{\kern0pt}subst\ powr{\isacharunderscore}{\kern0pt}realpow{\isacharparenright}{\kern0pt}\isanewline
\ \ \ \ \ \ \isacommand{apply}\isamarkupfalse%
\ {\isacharparenleft}{\kern0pt}metis\ a{\isadigit{2}}{\isacharparenright}{\kern0pt}\isanewline
\ \ \ \ \ \isacommand{apply}\isamarkupfalse%
\ {\isacharparenleft}{\kern0pt}subst\ power{\isacharunderscore}{\kern0pt}mono{\isacharparenright}{\kern0pt}\isanewline
\ \ \ \ \isacommand{by}\isamarkupfalse%
\ {\isacharparenleft}{\kern0pt}simp\ add{\isacharcolon}{\kern0pt}M{\isacharunderscore}{\kern0pt}def{\isacharparenright}{\kern0pt}{\isacharplus}{\kern0pt}\isanewline
\ \ \isacommand{also}\isamarkupfalse%
\ \isacommand{have}\isamarkupfalse%
\ {\isachardoublequoteopen}{\isachardot}{\kern0pt}{\isachardot}{\kern0pt}{\isachardot}{\kern0pt}\ {\isasymle}\ \ {\isacharparenleft}{\kern0pt}real{\isacharunderscore}{\kern0pt}of{\isacharunderscore}{\kern0pt}rat\ {\isacharparenleft}{\kern0pt}F\ k\ as{\isacharparenright}{\kern0pt}{\isacharparenright}{\kern0pt}\ powr\ {\isacharparenleft}{\kern0pt}{\isacharparenleft}{\kern0pt}k{\isacharminus}{\kern0pt}{\isadigit{1}}{\isacharparenright}{\kern0pt}{\isacharslash}{\kern0pt}k{\isacharparenright}{\kern0pt}\ {\isacharasterisk}{\kern0pt}\ {\isacharparenleft}{\kern0pt}real{\isacharunderscore}{\kern0pt}of{\isacharunderscore}{\kern0pt}rat\ {\isacharparenleft}{\kern0pt}F\ k\ as{\isacharparenright}{\kern0pt}{\isacharparenright}{\kern0pt}{\isachardoublequoteclose}\isanewline
\ \ \ \ \isacommand{apply}\isamarkupfalse%
\ {\isacharparenleft}{\kern0pt}rule\ mult{\isacharunderscore}{\kern0pt}right{\isacharunderscore}{\kern0pt}mono{\isacharparenright}{\kern0pt}\isanewline
\ \ \ \ \ \isacommand{apply}\isamarkupfalse%
\ {\isacharparenleft}{\kern0pt}subst\ a{\isadigit{4}}{\isacharparenright}{\kern0pt}\isanewline
\ \ \ \ \ \isacommand{apply}\isamarkupfalse%
\ {\isacharparenleft}{\kern0pt}subst\ powr{\isacharunderscore}{\kern0pt}powr{\isacharbrackleft}{\kern0pt}symmetric{\isacharbrackright}{\kern0pt}{\isacharparenright}{\kern0pt}\isanewline
\ \ \ \ \isacommand{by}\isamarkupfalse%
\ {\isacharparenleft}{\kern0pt}subst\ powr{\isacharunderscore}{\kern0pt}mono{\isadigit{2}}{\isacharcomma}{\kern0pt}\ simp{\isacharcomma}{\kern0pt}\ simp{\isacharcomma}{\kern0pt}\ metis\ a{\isadigit{3}}{\isacharcomma}{\kern0pt}\ simp{\isacharcomma}{\kern0pt}\ metis\ a{\isadigit{5}}{\isacharparenright}{\kern0pt}\isanewline
\ \ \isacommand{also}\isamarkupfalse%
\ \isacommand{have}\isamarkupfalse%
\ {\isachardoublequoteopen}{\isachardot}{\kern0pt}{\isachardot}{\kern0pt}{\isachardot}{\kern0pt}\ {\isacharequal}{\kern0pt}\ {\isacharparenleft}{\kern0pt}real{\isacharunderscore}{\kern0pt}of{\isacharunderscore}{\kern0pt}rat\ {\isacharparenleft}{\kern0pt}F\ k\ as{\isacharparenright}{\kern0pt}{\isacharparenright}{\kern0pt}\ powr\ {\isacharparenleft}{\kern0pt}{\isacharparenleft}{\kern0pt}{\isadigit{2}}{\isacharasterisk}{\kern0pt}k{\isacharminus}{\kern0pt}{\isadigit{1}}{\isacharparenright}{\kern0pt}\ {\isacharslash}{\kern0pt}\ k{\isacharparenright}{\kern0pt}{\isachardoublequoteclose}\isanewline
\ \ \ \ \isacommand{apply}\isamarkupfalse%
\ {\isacharparenleft}{\kern0pt}subst\ {\isacharparenleft}{\kern0pt}{\isadigit{2}}{\isacharparenright}{\kern0pt}\ a{\isadigit{6}}{\isacharparenright}{\kern0pt}\isanewline
\ \ \ \ \isacommand{apply}\isamarkupfalse%
\ {\isacharparenleft}{\kern0pt}subst\ powr{\isacharunderscore}{\kern0pt}add{\isacharbrackleft}{\kern0pt}symmetric{\isacharbrackright}{\kern0pt}{\isacharparenright}{\kern0pt}\isanewline
\ \ \ \ \isacommand{by}\isamarkupfalse%
\ {\isacharparenleft}{\kern0pt}rule\ arg{\isacharunderscore}{\kern0pt}cong{\isadigit{2}}{\isacharbrackleft}{\kern0pt}\isakeyword{where}\ f{\isacharequal}{\kern0pt}{\isachardoublequoteopen}{\isacharparenleft}{\kern0pt}powr{\isacharparenright}{\kern0pt}{\isachardoublequoteclose}{\isacharbrackright}{\kern0pt}{\isacharcomma}{\kern0pt}\ simp{\isacharcomma}{\kern0pt}\ metis\ a{\isadigit{7}}{\isacharparenright}{\kern0pt}\isanewline
\ \ \isacommand{finally}\isamarkupfalse%
\ \isacommand{have}\isamarkupfalse%
\ a{\isacharcolon}{\kern0pt}\ {\isachardoublequoteopen}real{\isacharunderscore}{\kern0pt}of{\isacharunderscore}{\kern0pt}rat\ {\isacharparenleft}{\kern0pt}F\ {\isacharparenleft}{\kern0pt}{\isadigit{2}}{\isacharasterisk}{\kern0pt}k{\isacharminus}{\kern0pt}{\isadigit{1}}{\isacharparenright}{\kern0pt}\ as{\isacharparenright}{\kern0pt}\ {\isasymle}\ {\isacharparenleft}{\kern0pt}real{\isacharunderscore}{\kern0pt}of{\isacharunderscore}{\kern0pt}rat\ {\isacharparenleft}{\kern0pt}F\ k\ as{\isacharparenright}{\kern0pt}{\isacharparenright}{\kern0pt}\ powr\ {\isacharparenleft}{\kern0pt}{\isacharparenleft}{\kern0pt}{\isadigit{2}}{\isacharasterisk}{\kern0pt}k{\isacharminus}{\kern0pt}{\isadigit{1}}{\isacharparenright}{\kern0pt}\ {\isacharslash}{\kern0pt}\ k{\isacharparenright}{\kern0pt}{\isachardoublequoteclose}\isanewline
\ \ \ \ \isacommand{by}\isamarkupfalse%
\ blast\isanewline
\isanewline
\ \ \isacommand{have}\isamarkupfalse%
\ b{\isadigit{1}}{\isacharcolon}{\kern0pt}\ {\isachardoublequoteopen}card\ {\isacharparenleft}{\kern0pt}set\ as{\isacharparenright}{\kern0pt}\ {\isasymle}\ n{\isachardoublequoteclose}\isanewline
\ \ \ \ \isacommand{by}\isamarkupfalse%
\ {\isacharparenleft}{\kern0pt}rule\ card{\isacharunderscore}{\kern0pt}mono{\isacharbrackleft}{\kern0pt}\isakeyword{where}\ B{\isacharequal}{\kern0pt}{\isachardoublequoteopen}{\isacharbraceleft}{\kern0pt}{\isadigit{0}}{\isachardot}{\kern0pt}{\isachardot}{\kern0pt}{\isacharless}{\kern0pt}n{\isacharbraceright}{\kern0pt}{\isachardoublequoteclose}{\isacharcomma}{\kern0pt}\ simplified{\isacharbrackright}{\kern0pt}{\isacharcomma}{\kern0pt}\ rule\ assms{\isacharparenleft}{\kern0pt}{\isadigit{2}}{\isacharparenright}{\kern0pt}{\isacharparenright}{\kern0pt}\isanewline
\isanewline
\ \ \isacommand{have}\isamarkupfalse%
\ {\isachardoublequoteopen}real\ {\isacharparenleft}{\kern0pt}length\ as{\isacharparenright}{\kern0pt}\ {\isacharequal}{\kern0pt}\ abs\ {\isacharparenleft}{\kern0pt}sum\ {\isacharparenleft}{\kern0pt}{\isasymlambda}x{\isachardot}{\kern0pt}\ real\ {\isacharparenleft}{\kern0pt}count{\isacharunderscore}{\kern0pt}list\ as\ x{\isacharparenright}{\kern0pt}{\isacharparenright}{\kern0pt}\ {\isacharparenleft}{\kern0pt}set\ as{\isacharparenright}{\kern0pt}{\isacharparenright}{\kern0pt}{\isachardoublequoteclose}\isanewline
\ \ \ \ \isacommand{apply}\isamarkupfalse%
\ {\isacharparenleft}{\kern0pt}subst\ of{\isacharunderscore}{\kern0pt}nat{\isacharunderscore}{\kern0pt}sum{\isacharbrackleft}{\kern0pt}symmetric{\isacharbrackright}{\kern0pt}{\isacharparenright}{\kern0pt}\isanewline
\ \ \ \ \isacommand{by}\isamarkupfalse%
\ {\isacharparenleft}{\kern0pt}simp\ add{\isacharcolon}{\kern0pt}\ sum{\isacharunderscore}{\kern0pt}count{\isacharunderscore}{\kern0pt}set{\isacharparenright}{\kern0pt}\isanewline
\ \ \isacommand{also}\isamarkupfalse%
\ \isacommand{have}\isamarkupfalse%
\ {\isachardoublequoteopen}{\isachardot}{\kern0pt}{\isachardot}{\kern0pt}{\isachardot}{\kern0pt}\ {\isasymle}\ {\isacharparenleft}{\kern0pt}real\ {\isacharparenleft}{\kern0pt}card\ {\isacharparenleft}{\kern0pt}set\ as{\isacharparenright}{\kern0pt}{\isacharparenright}{\kern0pt}{\isacharparenright}{\kern0pt}\ powr\ {\isacharparenleft}{\kern0pt}{\isacharparenleft}{\kern0pt}k{\isacharminus}{\kern0pt}Suc\ {\isadigit{0}}{\isacharparenright}{\kern0pt}{\isacharslash}{\kern0pt}k{\isacharparenright}{\kern0pt}\ {\isacharasterisk}{\kern0pt}\ {\isacharparenleft}{\kern0pt}sum\ {\isacharparenleft}{\kern0pt}{\isasymlambda}x{\isachardot}{\kern0pt}\ abs\ {\isacharparenleft}{\kern0pt}real\ {\isacharparenleft}{\kern0pt}count{\isacharunderscore}{\kern0pt}list\ as\ x{\isacharparenright}{\kern0pt}{\isacharparenright}{\kern0pt}\ powr\ k{\isacharparenright}{\kern0pt}\ {\isacharparenleft}{\kern0pt}set\ as{\isacharparenright}{\kern0pt}{\isacharparenright}{\kern0pt}\ powr\ {\isacharparenleft}{\kern0pt}{\isadigit{1}}{\isacharslash}{\kern0pt}k{\isacharparenright}{\kern0pt}{\isachardoublequoteclose}\isanewline
\ \ \ \ \isacommand{apply}\isamarkupfalse%
\ {\isacharparenleft}{\kern0pt}rule\ Holder{\isacharunderscore}{\kern0pt}inequality{\isacharunderscore}{\kern0pt}sum{\isacharbrackleft}{\kern0pt}\isakeyword{where}\ p{\isacharequal}{\kern0pt}{\isachardoublequoteopen}k{\isacharslash}{\kern0pt}{\isacharparenleft}{\kern0pt}k{\isacharminus}{\kern0pt}{\isadigit{1}}{\isacharparenright}{\kern0pt}{\isachardoublequoteclose}\ \isakeyword{and}\ q{\isacharequal}{\kern0pt}{\isachardoublequoteopen}k{\isachardoublequoteclose}\ \isakeyword{and}\ A{\isacharequal}{\kern0pt}{\isachardoublequoteopen}set\ as{\isachardoublequoteclose}\ \isakeyword{and}\ f{\isacharequal}{\kern0pt}{\isachardoublequoteopen}{\isasymlambda}{\isacharunderscore}{\kern0pt}{\isachardot}{\kern0pt}{\isadigit{1}}{\isachardoublequoteclose}{\isacharcomma}{\kern0pt}\ simplified{\isacharbrackright}{\kern0pt}{\isacharparenright}{\kern0pt}\isanewline
\ \ \ \ \isacommand{using}\isamarkupfalse%
\ assms\ True\ \isacommand{apply}\isamarkupfalse%
\ {\isacharparenleft}{\kern0pt}simp{\isacharparenright}{\kern0pt}\isanewline
\ \ \ \ \isacommand{using}\isamarkupfalse%
\ assms\ True\ \isacommand{apply}\isamarkupfalse%
\ {\isacharparenleft}{\kern0pt}simp{\isacharparenright}{\kern0pt}\isanewline
\ \ \ \ \isacommand{apply}\isamarkupfalse%
\ {\isacharparenleft}{\kern0pt}subst\ add{\isacharunderscore}{\kern0pt}divide{\isacharunderscore}{\kern0pt}distrib{\isacharbrackleft}{\kern0pt}symmetric{\isacharbrackright}{\kern0pt}{\isacharparenright}{\kern0pt}\isanewline
\ \ \ \ \isacommand{using}\isamarkupfalse%
\ assms\ True\ \isacommand{by}\isamarkupfalse%
\ simp\isanewline
\ \ \isacommand{also}\isamarkupfalse%
\ \isacommand{have}\isamarkupfalse%
\ {\isachardoublequoteopen}{\isachardot}{\kern0pt}{\isachardot}{\kern0pt}{\isachardot}{\kern0pt}\ {\isasymle}\ real\ n\ powr\ {\isacharparenleft}{\kern0pt}{\isadigit{1}}\ {\isacharminus}{\kern0pt}\ {\isadigit{1}}\ {\isacharslash}{\kern0pt}\ real\ k{\isacharparenright}{\kern0pt}\ {\isacharasterisk}{\kern0pt}\ real{\isacharunderscore}{\kern0pt}of{\isacharunderscore}{\kern0pt}rat\ {\isacharparenleft}{\kern0pt}F\ k\ as{\isacharparenright}{\kern0pt}\ powr\ {\isacharparenleft}{\kern0pt}{\isadigit{1}}{\isacharslash}{\kern0pt}real\ k{\isacharparenright}{\kern0pt}{\isachardoublequoteclose}\isanewline
\ \ \ \ \isacommand{apply}\isamarkupfalse%
\ {\isacharparenleft}{\kern0pt}rule\ mult{\isacharunderscore}{\kern0pt}mono{\isacharparenright}{\kern0pt}\isanewline
\ \ \ \ \ \ \ \isacommand{apply}\isamarkupfalse%
\ {\isacharparenleft}{\kern0pt}subst\ of{\isacharunderscore}{\kern0pt}nat{\isacharunderscore}{\kern0pt}diff{\isacharparenright}{\kern0pt}\ \isacommand{using}\isamarkupfalse%
\ assms\ True\ \isacommand{apply}\isamarkupfalse%
\ linarith\isanewline
\ \ \ \ \ \ \ \isacommand{apply}\isamarkupfalse%
\ {\isacharparenleft}{\kern0pt}subst\ diff{\isacharunderscore}{\kern0pt}divide{\isacharunderscore}{\kern0pt}distrib{\isacharparenright}{\kern0pt}\ \isacommand{using}\isamarkupfalse%
\ assms\ True\ \isacommand{apply}\isamarkupfalse%
\ simp\isanewline
\ \ \ \ \ \ \ \isacommand{apply}\isamarkupfalse%
\ {\isacharparenleft}{\kern0pt}rule\ powr{\isacharunderscore}{\kern0pt}mono{\isadigit{2}}{\isacharcomma}{\kern0pt}\ force{\isacharcomma}{\kern0pt}\ simp{\isacharparenright}{\kern0pt}\isanewline
\ \ \ \ \isacommand{using}\isamarkupfalse%
\ b{\isadigit{1}}\ \ of{\isacharunderscore}{\kern0pt}nat{\isacharunderscore}{\kern0pt}le{\isacharunderscore}{\kern0pt}iff\ \isacommand{apply}\isamarkupfalse%
\ blast\isanewline
\ \ \ \ \ \ \isacommand{apply}\isamarkupfalse%
\ {\isacharparenleft}{\kern0pt}rule\ powr{\isacharunderscore}{\kern0pt}mono{\isadigit{2}}{\isacharcomma}{\kern0pt}\ force{\isacharparenright}{\kern0pt}\isanewline
\ \ \ \ \ \ \ \isacommand{apply}\isamarkupfalse%
\ {\isacharparenleft}{\kern0pt}rule\ sum{\isacharunderscore}{\kern0pt}mono{\isacharbrackleft}{\kern0pt}\isakeyword{where}\ f{\isacharequal}{\kern0pt}{\isachardoublequoteopen}{\isasymlambda}{\isacharunderscore}{\kern0pt}{\isachardot}{\kern0pt}\ {\isadigit{0}}{\isachardoublequoteclose}{\isacharcomma}{\kern0pt}\ simplified{\isacharbrackright}{\kern0pt}{\isacharparenright}{\kern0pt}\isanewline
\ \ \ \ \ \ \ \isacommand{apply}\isamarkupfalse%
\ simp\isanewline
\ \ \ \ \ \ \isacommand{apply}\isamarkupfalse%
\ {\isacharparenleft}{\kern0pt}simp\ add{\isacharcolon}{\kern0pt}F{\isacharunderscore}{\kern0pt}def\ of{\isacharunderscore}{\kern0pt}rat{\isacharunderscore}{\kern0pt}sum\ of{\isacharunderscore}{\kern0pt}rat{\isacharunderscore}{\kern0pt}power{\isacharparenright}{\kern0pt}\isanewline
\ \ \ \ \isacommand{apply}\isamarkupfalse%
\ {\isacharparenleft}{\kern0pt}rule\ sum{\isacharunderscore}{\kern0pt}mono{\isacharparenright}{\kern0pt}\isanewline
\ \ \ \ \ \ \isacommand{apply}\isamarkupfalse%
\ {\isacharparenleft}{\kern0pt}subst\ powr{\isacharunderscore}{\kern0pt}realpow{\isacharcomma}{\kern0pt}\ simp{\isacharparenright}{\kern0pt}\isanewline
\ \ \ \ \isacommand{using}\isamarkupfalse%
\ count{\isacharunderscore}{\kern0pt}list{\isacharunderscore}{\kern0pt}gr{\isacharunderscore}{\kern0pt}{\isadigit{1}}\ \isanewline
\ \ \ \ \isacommand{by}\isamarkupfalse%
\ {\isacharparenleft}{\kern0pt}metis\ gr{\isadigit{0}}I\ not{\isacharunderscore}{\kern0pt}one{\isacharunderscore}{\kern0pt}le{\isacharunderscore}{\kern0pt}zero{\isacharcomma}{\kern0pt}\ simp{\isacharcomma}{\kern0pt}\ simp{\isacharcomma}{\kern0pt}\ simp{\isacharparenright}{\kern0pt}\isanewline
\ \ \isacommand{finally}\isamarkupfalse%
\ \isacommand{have}\isamarkupfalse%
\ b{\isacharcolon}{\kern0pt}\ {\isachardoublequoteopen}real\ {\isacharparenleft}{\kern0pt}length\ as{\isacharparenright}{\kern0pt}\ {\isasymle}\ real\ n\ powr\ {\isacharparenleft}{\kern0pt}{\isadigit{1}}\ {\isacharminus}{\kern0pt}\ {\isadigit{1}}\ {\isacharslash}{\kern0pt}\ real\ k{\isacharparenright}{\kern0pt}\ {\isacharasterisk}{\kern0pt}\ real{\isacharunderscore}{\kern0pt}of{\isacharunderscore}{\kern0pt}rat\ {\isacharparenleft}{\kern0pt}F\ k\ as{\isacharparenright}{\kern0pt}\ powr\ {\isacharparenleft}{\kern0pt}{\isadigit{1}}{\isacharslash}{\kern0pt}real\ k{\isacharparenright}{\kern0pt}{\isachardoublequoteclose}\isanewline
\ \ \ \ \isacommand{by}\isamarkupfalse%
\ blast\isanewline
\isanewline
\ \ \isacommand{have}\isamarkupfalse%
\ c{\isacharcolon}{\kern0pt}{\isachardoublequoteopen}{\isadigit{1}}\ {\isacharslash}{\kern0pt}\ real\ k\ {\isacharplus}{\kern0pt}\ real\ {\isacharparenleft}{\kern0pt}{\isadigit{2}}\ {\isacharasterisk}{\kern0pt}\ k\ {\isacharminus}{\kern0pt}\ {\isadigit{1}}{\isacharparenright}{\kern0pt}\ {\isacharslash}{\kern0pt}\ real\ k\ {\isacharequal}{\kern0pt}\ real\ {\isadigit{2}}{\isachardoublequoteclose}\isanewline
\ \ \ \ \isacommand{apply}\isamarkupfalse%
\ {\isacharparenleft}{\kern0pt}subst\ add{\isacharunderscore}{\kern0pt}divide{\isacharunderscore}{\kern0pt}distrib{\isacharbrackleft}{\kern0pt}symmetric{\isacharbrackright}{\kern0pt}{\isacharparenright}{\kern0pt}\isanewline
\ \ \ \ \isacommand{apply}\isamarkupfalse%
\ {\isacharparenleft}{\kern0pt}subst\ of{\isacharunderscore}{\kern0pt}nat{\isacharunderscore}{\kern0pt}diff{\isacharparenright}{\kern0pt}\ \isacommand{using}\isamarkupfalse%
\ True\ \isacommand{apply}\isamarkupfalse%
\ linarith\isanewline
\ \ \ \ \isacommand{using}\isamarkupfalse%
\ assms{\isacharparenleft}{\kern0pt}{\isadigit{2}}{\isacharparenright}{\kern0pt}\ True\ \isacommand{by}\isamarkupfalse%
\ simp\isanewline
\isanewline
\ \ \isacommand{have}\isamarkupfalse%
\ {\isachardoublequoteopen}{\isacharquery}{\kern0pt}lhs\ {\isasymle}\ real\ n\ powr\ {\isacharparenleft}{\kern0pt}{\isadigit{1}}\ {\isacharminus}{\kern0pt}\ {\isadigit{1}}\ {\isacharslash}{\kern0pt}\ real\ k{\isacharparenright}{\kern0pt}\ {\isacharasterisk}{\kern0pt}\ real{\isacharunderscore}{\kern0pt}of{\isacharunderscore}{\kern0pt}rat\ {\isacharparenleft}{\kern0pt}F\ k\ as{\isacharparenright}{\kern0pt}\ powr\ {\isacharparenleft}{\kern0pt}{\isadigit{1}}{\isacharslash}{\kern0pt}real\ k{\isacharparenright}{\kern0pt}\ {\isacharasterisk}{\kern0pt}\ \ {\isacharparenleft}{\kern0pt}real{\isacharunderscore}{\kern0pt}of{\isacharunderscore}{\kern0pt}rat\ {\isacharparenleft}{\kern0pt}F\ k\ as{\isacharparenright}{\kern0pt}{\isacharparenright}{\kern0pt}\ powr\ {\isacharparenleft}{\kern0pt}{\isacharparenleft}{\kern0pt}{\isadigit{2}}{\isacharasterisk}{\kern0pt}k{\isacharminus}{\kern0pt}{\isadigit{1}}{\isacharparenright}{\kern0pt}\ {\isacharslash}{\kern0pt}\ k{\isacharparenright}{\kern0pt}{\isachardoublequoteclose}\isanewline
\ \ \ \ \isacommand{apply}\isamarkupfalse%
\ {\isacharparenleft}{\kern0pt}rule\ mult{\isacharunderscore}{\kern0pt}mono{\isacharcomma}{\kern0pt}\ metis\ b{\isacharcomma}{\kern0pt}\ metis\ a{\isacharcomma}{\kern0pt}\ simp{\isacharcomma}{\kern0pt}\ simp\ add{\isacharcolon}{\kern0pt}F{\isacharunderscore}{\kern0pt}def{\isacharparenright}{\kern0pt}\isanewline
\ \ \ \ \isacommand{apply}\isamarkupfalse%
\ {\isacharparenleft}{\kern0pt}rule\ sum{\isacharunderscore}{\kern0pt}mono{\isacharbrackleft}{\kern0pt}\isakeyword{where}\ f{\isacharequal}{\kern0pt}{\isachardoublequoteopen}{\isasymlambda}{\isacharunderscore}{\kern0pt}{\isachardot}{\kern0pt}\ {\isacharparenleft}{\kern0pt}{\isadigit{0}}{\isacharcolon}{\kern0pt}{\isacharcolon}{\kern0pt}rat{\isacharparenright}{\kern0pt}{\isachardoublequoteclose}{\isacharcomma}{\kern0pt}\ simplified{\isacharbrackright}{\kern0pt}{\isacharparenright}{\kern0pt}\isanewline
\ \ \ \ \isacommand{by}\isamarkupfalse%
\ auto\isanewline
\ \ \isacommand{also}\isamarkupfalse%
\ \isacommand{have}\isamarkupfalse%
\ {\isachardoublequoteopen}{\isachardot}{\kern0pt}{\isachardot}{\kern0pt}{\isachardot}{\kern0pt}\ {\isasymle}\ {\isacharquery}{\kern0pt}rhs{\isachardoublequoteclose}\isanewline
\ \ \ \ \isacommand{apply}\isamarkupfalse%
\ {\isacharparenleft}{\kern0pt}subst\ mult{\isachardot}{\kern0pt}assoc{\isacharcomma}{\kern0pt}\ subst\ powr{\isacharunderscore}{\kern0pt}add{\isacharbrackleft}{\kern0pt}symmetric{\isacharbrackright}{\kern0pt}{\isacharcomma}{\kern0pt}\ subst\ mult{\isacharunderscore}{\kern0pt}left{\isacharunderscore}{\kern0pt}mono{\isacharparenright}{\kern0pt}\isanewline
\ \ \ \ \isacommand{apply}\isamarkupfalse%
\ {\isacharparenleft}{\kern0pt}subst\ c{\isacharcomma}{\kern0pt}\ subst\ powr{\isacharunderscore}{\kern0pt}realpow{\isacharparenright}{\kern0pt}\isanewline
\ \ \ \ \isacommand{using}\isamarkupfalse%
\ \ F{\isacharunderscore}{\kern0pt}gr{\isacharunderscore}{\kern0pt}{\isadigit{0}}{\isacharbrackleft}{\kern0pt}OF\ assms{\isacharparenleft}{\kern0pt}{\isadigit{1}}{\isacharparenright}{\kern0pt}{\isacharbrackright}{\kern0pt}\ \isacommand{by}\isamarkupfalse%
\ simp{\isacharplus}{\kern0pt}\isanewline
\ \ \isacommand{finally}\isamarkupfalse%
\ \isacommand{show}\isamarkupfalse%
\ {\isacharquery}{\kern0pt}thesis\isanewline
\ \ \ \ \isacommand{by}\isamarkupfalse%
\ blast\isanewline
\isacommand{next}\isamarkupfalse%
\isanewline
\ \ \isacommand{case}\isamarkupfalse%
\ False\isanewline
\ \ \isacommand{have}\isamarkupfalse%
\ {\isachardoublequoteopen}n\ {\isachargreater}{\kern0pt}\ {\isadigit{0}}{\isachardoublequoteclose}\ \isanewline
\ \ \ \ \isacommand{apply}\isamarkupfalse%
\ {\isacharparenleft}{\kern0pt}cases\ {\isachardoublequoteopen}n{\isacharequal}{\kern0pt}{\isadigit{0}}{\isachardoublequoteclose}{\isacharparenright}{\kern0pt}\ \isanewline
\ \ \ \ \isacommand{using}\isamarkupfalse%
\ assms{\isacharparenleft}{\kern0pt}{\isadigit{1}}{\isacharparenright}{\kern0pt}\ assms{\isacharparenleft}{\kern0pt}{\isadigit{2}}{\isacharparenright}{\kern0pt}\ equals{\isadigit{0}}I\ \isacommand{by}\isamarkupfalse%
\ {\isacharparenleft}{\kern0pt}simp{\isacharcomma}{\kern0pt}\ blast{\isacharparenright}{\kern0pt}\isanewline
\ \ \isacommand{moreover}\isamarkupfalse%
\ \isacommand{have}\isamarkupfalse%
\ {\isachardoublequoteopen}k\ {\isacharequal}{\kern0pt}\ {\isadigit{1}}{\isachardoublequoteclose}\ \isacommand{using}\isamarkupfalse%
\ assms\ False\ \isacommand{by}\isamarkupfalse%
\ linarith\isanewline
\ \ \isacommand{ultimately}\isamarkupfalse%
\ \isacommand{show}\isamarkupfalse%
\ {\isacharquery}{\kern0pt}thesis\isanewline
\ \ \ \ \isacommand{apply}\isamarkupfalse%
\ {\isacharparenleft}{\kern0pt}simp\ add{\isacharcolon}{\kern0pt}power{\isadigit{2}}{\isacharunderscore}{\kern0pt}eq{\isacharunderscore}{\kern0pt}square{\isacharparenright}{\kern0pt}\isanewline
\ \ \ \ \isacommand{apply}\isamarkupfalse%
\ {\isacharparenleft}{\kern0pt}rule\ mult{\isacharunderscore}{\kern0pt}right{\isacharunderscore}{\kern0pt}mono{\isacharparenright}{\kern0pt}\isanewline
\ \ \ \ \isacommand{apply}\isamarkupfalse%
\ {\isacharparenleft}{\kern0pt}simp\ add{\isacharcolon}{\kern0pt}F{\isacharunderscore}{\kern0pt}def\ sum{\isacharunderscore}{\kern0pt}count{\isacharunderscore}{\kern0pt}set\ of{\isacharunderscore}{\kern0pt}nat{\isacharunderscore}{\kern0pt}sum{\isacharbrackleft}{\kern0pt}symmetric{\isacharbrackright}{\kern0pt}\ del{\isacharcolon}{\kern0pt}of{\isacharunderscore}{\kern0pt}nat{\isacharunderscore}{\kern0pt}sum{\isacharparenright}{\kern0pt}\isanewline
\ \ \ \ \isacommand{using}\isamarkupfalse%
\ F{\isacharunderscore}{\kern0pt}gr{\isacharunderscore}{\kern0pt}{\isadigit{0}}{\isacharbrackleft}{\kern0pt}OF\ assms{\isacharparenleft}{\kern0pt}{\isadigit{1}}{\isacharparenright}{\kern0pt}{\isacharbrackright}{\kern0pt}\ order{\isacharunderscore}{\kern0pt}le{\isacharunderscore}{\kern0pt}less\ \isacommand{by}\isamarkupfalse%
\ auto\isanewline
\isacommand{qed}\isamarkupfalse%
%
\endisatagproof
{\isafoldproof}%
%
\isadelimproof
\isanewline
%
\endisadelimproof
\isanewline
\isacommand{lemma}\isamarkupfalse%
\ fk{\isacharunderscore}{\kern0pt}alg{\isacharunderscore}{\kern0pt}core{\isacharunderscore}{\kern0pt}exp{\isacharcolon}{\kern0pt}\isanewline
\ \ \isakeyword{assumes}\ {\isachardoublequoteopen}as\ {\isasymnoteq}\ {\isacharbrackleft}{\kern0pt}{\isacharbrackright}{\kern0pt}{\isachardoublequoteclose}\isanewline
\ \ \isakeyword{assumes}\ {\isachardoublequoteopen}k\ {\isasymge}\ {\isadigit{1}}{\isachardoublequoteclose}\isanewline
\ \ \isakeyword{shows}\ {\isachardoublequoteopen}has{\isacharunderscore}{\kern0pt}bochner{\isacharunderscore}{\kern0pt}integral\ {\isacharparenleft}{\kern0pt}measure{\isacharunderscore}{\kern0pt}pmf\ {\isacharparenleft}{\kern0pt}pmf{\isacharunderscore}{\kern0pt}of{\isacharunderscore}{\kern0pt}set\ {\isacharbraceleft}{\kern0pt}{\isacharparenleft}{\kern0pt}u{\isacharcomma}{\kern0pt}\ v{\isacharparenright}{\kern0pt}{\isachardot}{\kern0pt}\ v\ {\isacharless}{\kern0pt}\ count{\isacharunderscore}{\kern0pt}list\ as\ u{\isacharbraceright}{\kern0pt}{\isacharparenright}{\kern0pt}{\isacharparenright}{\kern0pt}\isanewline
\ \ \ \ \ \ \ \ {\isacharparenleft}{\kern0pt}{\isasymlambda}a{\isachardot}{\kern0pt}\ real\ {\isacharparenleft}{\kern0pt}length\ as{\isacharparenright}{\kern0pt}\ {\isacharasterisk}{\kern0pt}\ real\ {\isacharparenleft}{\kern0pt}Suc\ {\isacharparenleft}{\kern0pt}snd\ a{\isacharparenright}{\kern0pt}\ {\isacharcircum}{\kern0pt}\ k\ {\isacharminus}{\kern0pt}\ snd\ a\ {\isacharcircum}{\kern0pt}\ k{\isacharparenright}{\kern0pt}{\isacharparenright}{\kern0pt}\ {\isacharparenleft}{\kern0pt}real{\isacharunderscore}{\kern0pt}of{\isacharunderscore}{\kern0pt}rat\ {\isacharparenleft}{\kern0pt}F\ k\ as{\isacharparenright}{\kern0pt}{\isacharparenright}{\kern0pt}{\isachardoublequoteclose}\isanewline
%
\isadelimproof
%
\endisadelimproof
%
\isatagproof
\isacommand{proof}\isamarkupfalse%
\ {\isacharminus}{\kern0pt}\isanewline
\ \ \isacommand{show}\isamarkupfalse%
\ {\isacharquery}{\kern0pt}thesis\isanewline
\ \ \ \ \isacommand{apply}\isamarkupfalse%
\ {\isacharparenleft}{\kern0pt}subst\ has{\isacharunderscore}{\kern0pt}bochner{\isacharunderscore}{\kern0pt}integral{\isacharunderscore}{\kern0pt}iff{\isacharparenright}{\kern0pt}\isanewline
\ \ \ \ \isacommand{apply}\isamarkupfalse%
\ {\isacharparenleft}{\kern0pt}rule\ conjI{\isacharparenright}{\kern0pt}\isanewline
\ \ \ \ \ \isacommand{apply}\isamarkupfalse%
\ {\isacharparenleft}{\kern0pt}rule\ integrable{\isacharunderscore}{\kern0pt}measure{\isacharunderscore}{\kern0pt}pmf{\isacharunderscore}{\kern0pt}finite{\isacharparenright}{\kern0pt}\isanewline
\ \ \ \ \ \isacommand{apply}\isamarkupfalse%
\ {\isacharparenleft}{\kern0pt}subst\ set{\isacharunderscore}{\kern0pt}pmf{\isacharunderscore}{\kern0pt}of{\isacharunderscore}{\kern0pt}set{\isacharcomma}{\kern0pt}\ metis\ non{\isacharunderscore}{\kern0pt}empty{\isacharunderscore}{\kern0pt}space\ assms{\isacharparenleft}{\kern0pt}{\isadigit{1}}{\isacharparenright}{\kern0pt}{\isacharcomma}{\kern0pt}\ metis\ fin{\isacharunderscore}{\kern0pt}space\ assms{\isacharparenleft}{\kern0pt}{\isadigit{1}}{\isacharparenright}{\kern0pt}{\isacharparenright}{\kern0pt}\isanewline
\ \ \ \ \isacommand{apply}\isamarkupfalse%
\ {\isacharparenleft}{\kern0pt}subst\ integral{\isacharunderscore}{\kern0pt}measure{\isacharunderscore}{\kern0pt}pmf{\isacharunderscore}{\kern0pt}real{\isacharbrackleft}{\kern0pt}OF\ fin{\isacharunderscore}{\kern0pt}space{\isacharbrackleft}{\kern0pt}OF\ assms{\isacharparenleft}{\kern0pt}{\isadigit{1}}{\isacharparenright}{\kern0pt}{\isacharbrackright}{\kern0pt}{\isacharbrackright}{\kern0pt}{\isacharparenright}{\kern0pt}\isanewline
\ \ \ \ \ \isacommand{apply}\isamarkupfalse%
\ {\isacharparenleft}{\kern0pt}subst\ {\isacharparenleft}{\kern0pt}asm{\isacharparenright}{\kern0pt}\ set{\isacharunderscore}{\kern0pt}pmf{\isacharunderscore}{\kern0pt}of{\isacharunderscore}{\kern0pt}set{\isacharbrackleft}{\kern0pt}OF\ non{\isacharunderscore}{\kern0pt}empty{\isacharunderscore}{\kern0pt}space{\isacharbrackleft}{\kern0pt}OF\ assms{\isacharparenleft}{\kern0pt}{\isadigit{1}}{\isacharparenright}{\kern0pt}{\isacharbrackright}{\kern0pt}\ fin{\isacharunderscore}{\kern0pt}space{\isacharbrackleft}{\kern0pt}OF\ assms{\isacharparenleft}{\kern0pt}{\isadigit{1}}{\isacharparenright}{\kern0pt}{\isacharbrackright}{\kern0pt}{\isacharbrackright}{\kern0pt}{\isacharcomma}{\kern0pt}\ simp{\isacharparenright}{\kern0pt}\isanewline
\ \ \ \ \isacommand{apply}\isamarkupfalse%
\ {\isacharparenleft}{\kern0pt}subst\ pmf{\isacharunderscore}{\kern0pt}of{\isacharunderscore}{\kern0pt}set{\isacharbrackleft}{\kern0pt}OF\ non{\isacharunderscore}{\kern0pt}empty{\isacharunderscore}{\kern0pt}space{\isacharbrackleft}{\kern0pt}OF\ assms{\isacharparenleft}{\kern0pt}{\isadigit{1}}{\isacharparenright}{\kern0pt}{\isacharbrackright}{\kern0pt}\ fin{\isacharunderscore}{\kern0pt}space{\isacharbrackleft}{\kern0pt}OF\ assms{\isacharparenleft}{\kern0pt}{\isadigit{1}}{\isacharparenright}{\kern0pt}{\isacharbrackright}{\kern0pt}{\isacharbrackright}{\kern0pt}{\isacharparenright}{\kern0pt}\isanewline
\ \ \ \ \isacommand{using}\isamarkupfalse%
\ assms{\isacharparenleft}{\kern0pt}{\isadigit{1}}{\isacharparenright}{\kern0pt}\ \isacommand{apply}\isamarkupfalse%
\ {\isacharparenleft}{\kern0pt}simp\ add{\isacharcolon}{\kern0pt}card{\isacharunderscore}{\kern0pt}space\ F{\isacharunderscore}{\kern0pt}def\ of{\isacharunderscore}{\kern0pt}rat{\isacharunderscore}{\kern0pt}sum\ of{\isacharunderscore}{\kern0pt}rat{\isacharunderscore}{\kern0pt}power{\isacharparenright}{\kern0pt}\isanewline
\ \ \ \ \isacommand{apply}\isamarkupfalse%
\ {\isacharparenleft}{\kern0pt}subst\ split{\isacharunderscore}{\kern0pt}space{\isacharparenright}{\kern0pt}\isanewline
\ \ \ \ \isacommand{apply}\isamarkupfalse%
\ {\isacharparenleft}{\kern0pt}rule\ sum{\isachardot}{\kern0pt}cong{\isacharcomma}{\kern0pt}\ simp{\isacharparenright}{\kern0pt}\isanewline
\ \ \ \ \isacommand{apply}\isamarkupfalse%
\ {\isacharparenleft}{\kern0pt}subst\ of{\isacharunderscore}{\kern0pt}nat{\isacharunderscore}{\kern0pt}diff{\isacharparenright}{\kern0pt}\ \isanewline
\ \ \ \ \isacommand{apply}\isamarkupfalse%
\ {\isacharparenleft}{\kern0pt}simp\ add{\isacharcolon}{\kern0pt}\ power{\isacharunderscore}{\kern0pt}mono{\isacharparenright}{\kern0pt}\isanewline
\ \ \ \ \isacommand{apply}\isamarkupfalse%
\ {\isacharparenleft}{\kern0pt}subst\ sum{\isacharunderscore}{\kern0pt}Suc{\isacharunderscore}{\kern0pt}diff{\isacharprime}{\kern0pt}{\isacharcomma}{\kern0pt}\ simp{\isacharcomma}{\kern0pt}\ simp{\isacharparenright}{\kern0pt}\isanewline
\ \ \ \ \isacommand{using}\isamarkupfalse%
\ assms\ \isacommand{by}\isamarkupfalse%
\ linarith\isanewline
\isacommand{qed}\isamarkupfalse%
%
\endisatagproof
{\isafoldproof}%
%
\isadelimproof
\isanewline
%
\endisadelimproof
\isanewline
\isacommand{lemma}\isamarkupfalse%
\ fk{\isacharunderscore}{\kern0pt}alg{\isacharunderscore}{\kern0pt}core{\isacharunderscore}{\kern0pt}var{\isacharcolon}{\kern0pt}\isanewline
\ \ \isakeyword{assumes}\ {\isachardoublequoteopen}as\ {\isasymnoteq}\ {\isacharbrackleft}{\kern0pt}{\isacharbrackright}{\kern0pt}{\isachardoublequoteclose}\isanewline
\ \ \isakeyword{assumes}\ {\isachardoublequoteopen}k\ {\isasymge}\ {\isadigit{1}}{\isachardoublequoteclose}\isanewline
\ \ \isakeyword{assumes}\ {\isachardoublequoteopen}set\ as\ {\isasymsubseteq}\ {\isacharbraceleft}{\kern0pt}{\isadigit{0}}{\isachardot}{\kern0pt}{\isachardot}{\kern0pt}{\isacharless}{\kern0pt}n{\isacharbraceright}{\kern0pt}{\isachardoublequoteclose}\isanewline
\ \ \isakeyword{shows}\ {\isachardoublequoteopen}prob{\isacharunderscore}{\kern0pt}space{\isachardot}{\kern0pt}variance\ {\isacharparenleft}{\kern0pt}measure{\isacharunderscore}{\kern0pt}pmf\ {\isacharparenleft}{\kern0pt}pmf{\isacharunderscore}{\kern0pt}of{\isacharunderscore}{\kern0pt}set\ {\isacharbraceleft}{\kern0pt}{\isacharparenleft}{\kern0pt}u{\isacharcomma}{\kern0pt}\ v{\isacharparenright}{\kern0pt}{\isachardot}{\kern0pt}\ v\ {\isacharless}{\kern0pt}\ count{\isacharunderscore}{\kern0pt}list\ as\ u{\isacharbraceright}{\kern0pt}{\isacharparenright}{\kern0pt}{\isacharparenright}{\kern0pt}\isanewline
\ \ \ \ \ \ \ \ {\isacharparenleft}{\kern0pt}{\isasymlambda}a{\isachardot}{\kern0pt}\ real\ {\isacharparenleft}{\kern0pt}length\ as{\isacharparenright}{\kern0pt}\ {\isacharasterisk}{\kern0pt}\ real\ {\isacharparenleft}{\kern0pt}Suc\ {\isacharparenleft}{\kern0pt}snd\ a{\isacharparenright}{\kern0pt}\ {\isacharcircum}{\kern0pt}\ k\ {\isacharminus}{\kern0pt}\ snd\ a\ {\isacharcircum}{\kern0pt}\ k{\isacharparenright}{\kern0pt}{\isacharparenright}{\kern0pt}\isanewline
\ \ \ \ \ \ \ \ \ {\isasymle}\ {\isacharparenleft}{\kern0pt}real{\isacharunderscore}{\kern0pt}of{\isacharunderscore}{\kern0pt}rat\ {\isacharparenleft}{\kern0pt}F\ k\ as{\isacharparenright}{\kern0pt}{\isacharparenright}{\kern0pt}\isactrlsup {\isadigit{2}}\ {\isacharasterisk}{\kern0pt}\ real\ k\ {\isacharasterisk}{\kern0pt}\ real\ n\ powr\ {\isacharparenleft}{\kern0pt}{\isadigit{1}}\ {\isacharminus}{\kern0pt}\ {\isadigit{1}}\ {\isacharslash}{\kern0pt}\ real\ k{\isacharparenright}{\kern0pt}{\isachardoublequoteclose}\isanewline
%
\isadelimproof
%
\endisadelimproof
%
\isatagproof
\isacommand{proof}\isamarkupfalse%
\ {\isacharminus}{\kern0pt}\isanewline
\ \ \isacommand{define}\isamarkupfalse%
\ f\ {\isacharcolon}{\kern0pt}{\isacharcolon}{\kern0pt}\ {\isachardoublequoteopen}nat\ {\isasymtimes}\ nat\ {\isasymRightarrow}\ real{\isachardoublequoteclose}\ \isanewline
\ \ \ \ \isakeyword{where}\ {\isachardoublequoteopen}f\ {\isacharequal}{\kern0pt}\ {\isacharparenleft}{\kern0pt}{\isasymlambda}x{\isachardot}{\kern0pt}\ {\isacharparenleft}{\kern0pt}real\ {\isacharparenleft}{\kern0pt}length\ as{\isacharparenright}{\kern0pt}\ {\isacharasterisk}{\kern0pt}\ real\ {\isacharparenleft}{\kern0pt}Suc\ {\isacharparenleft}{\kern0pt}snd\ x{\isacharparenright}{\kern0pt}\ {\isacharcircum}{\kern0pt}\ k\ {\isacharminus}{\kern0pt}\ snd\ x\ {\isacharcircum}{\kern0pt}\ k{\isacharparenright}{\kern0pt}{\isacharparenright}{\kern0pt}{\isacharparenright}{\kern0pt}{\isachardoublequoteclose}\isanewline
\ \ \isacommand{define}\isamarkupfalse%
\ {\isasymOmega}\ \isakeyword{where}\ {\isachardoublequoteopen}{\isasymOmega}\ {\isacharequal}{\kern0pt}\ pmf{\isacharunderscore}{\kern0pt}of{\isacharunderscore}{\kern0pt}set\ {\isacharbraceleft}{\kern0pt}{\isacharparenleft}{\kern0pt}u{\isacharcomma}{\kern0pt}\ v{\isacharparenright}{\kern0pt}{\isachardot}{\kern0pt}\ v\ {\isacharless}{\kern0pt}\ count{\isacharunderscore}{\kern0pt}list\ as\ u{\isacharbraceright}{\kern0pt}{\isachardoublequoteclose}\isanewline
\ \ \isanewline
\ \ \isacommand{have}\isamarkupfalse%
\ integrable{\isacharcolon}{\kern0pt}\ {\isachardoublequoteopen}{\isasymAnd}k\ f{\isachardot}{\kern0pt}\ integrable\ {\isacharparenleft}{\kern0pt}measure{\isacharunderscore}{\kern0pt}pmf\ {\isasymOmega}{\isacharparenright}{\kern0pt}\ {\isacharparenleft}{\kern0pt}{\isasymlambda}{\isasymomega}{\isachardot}{\kern0pt}\ {\isacharparenleft}{\kern0pt}f\ {\isasymomega}{\isacharparenright}{\kern0pt}{\isacharcolon}{\kern0pt}{\isacharcolon}{\kern0pt}real{\isacharparenright}{\kern0pt}{\isachardoublequoteclose}\isanewline
\ \ \ \ \isacommand{apply}\isamarkupfalse%
\ {\isacharparenleft}{\kern0pt}simp\ add{\isacharcolon}{\kern0pt}{\isasymOmega}{\isacharunderscore}{\kern0pt}def{\isacharparenright}{\kern0pt}\isanewline
\ \ \ \ \isacommand{apply}\isamarkupfalse%
\ {\isacharparenleft}{\kern0pt}rule\ integrable{\isacharunderscore}{\kern0pt}measure{\isacharunderscore}{\kern0pt}pmf{\isacharunderscore}{\kern0pt}finite{\isacharparenright}{\kern0pt}\isanewline
\ \ \ \ \isacommand{apply}\isamarkupfalse%
\ {\isacharparenleft}{\kern0pt}subst\ set{\isacharunderscore}{\kern0pt}pmf{\isacharunderscore}{\kern0pt}of{\isacharunderscore}{\kern0pt}set{\isacharparenright}{\kern0pt}\isanewline
\ \ \ \ \isacommand{using}\isamarkupfalse%
\ assms{\isacharparenleft}{\kern0pt}{\isadigit{1}}{\isacharparenright}{\kern0pt}\ fin{\isacharunderscore}{\kern0pt}space\ non{\isacharunderscore}{\kern0pt}empty{\isacharunderscore}{\kern0pt}space\ \isacommand{by}\isamarkupfalse%
\ auto\isanewline
\isanewline
\ \ \isacommand{have}\isamarkupfalse%
\ k{\isacharunderscore}{\kern0pt}g{\isacharunderscore}{\kern0pt}{\isadigit{0}}{\isacharcolon}{\kern0pt}\ {\isachardoublequoteopen}k\ {\isachargreater}{\kern0pt}\ {\isadigit{0}}{\isachardoublequoteclose}\ \isacommand{using}\isamarkupfalse%
\ assms\ \isacommand{by}\isamarkupfalse%
\ linarith\isanewline
\isanewline
\ \ \isacommand{have}\isamarkupfalse%
\ c{\isacharcolon}{\kern0pt}{\isachardoublequoteopen}{\isasymAnd}a\ v{\isachardot}{\kern0pt}\ v\ {\isacharless}{\kern0pt}\ count{\isacharunderscore}{\kern0pt}list\ as\ a\ {\isasymLongrightarrow}\ real\ {\isacharparenleft}{\kern0pt}Suc\ v\ {\isacharcircum}{\kern0pt}\ k{\isacharparenright}{\kern0pt}\ {\isacharminus}{\kern0pt}\ real\ {\isacharparenleft}{\kern0pt}v\ {\isacharcircum}{\kern0pt}\ k{\isacharparenright}{\kern0pt}\ {\isasymle}\ real\ k\ {\isacharasterisk}{\kern0pt}\ real\ {\isacharparenleft}{\kern0pt}count{\isacharunderscore}{\kern0pt}list\ as\ a{\isacharparenright}{\kern0pt}\ {\isacharcircum}{\kern0pt}\ {\isacharparenleft}{\kern0pt}k\ {\isacharminus}{\kern0pt}\ Suc\ {\isadigit{0}}{\isacharparenright}{\kern0pt}{\isachardoublequoteclose}\isanewline
\ \ \isacommand{proof}\isamarkupfalse%
\ {\isacharminus}{\kern0pt}\isanewline
\ \ \ \ \isacommand{fix}\isamarkupfalse%
\ a\ v\isanewline
\ \ \ \ \isacommand{assume}\isamarkupfalse%
\ c{\isacharunderscore}{\kern0pt}{\isadigit{1}}{\isacharcolon}{\kern0pt}\ {\isachardoublequoteopen}v\ {\isacharless}{\kern0pt}\ count{\isacharunderscore}{\kern0pt}list\ as\ a{\isachardoublequoteclose}\isanewline
\ \ \ \ \isacommand{have}\isamarkupfalse%
\ {\isachardoublequoteopen}real\ {\isacharparenleft}{\kern0pt}Suc\ v\ {\isacharcircum}{\kern0pt}\ k{\isacharparenright}{\kern0pt}\ {\isacharminus}{\kern0pt}\ real\ {\isacharparenleft}{\kern0pt}v\ {\isacharcircum}{\kern0pt}\ k{\isacharparenright}{\kern0pt}\ {\isasymle}\ {\isacharparenleft}{\kern0pt}real\ {\isacharparenleft}{\kern0pt}v{\isacharplus}{\kern0pt}{\isadigit{1}}{\isacharparenright}{\kern0pt}\ {\isacharminus}{\kern0pt}\ real\ v{\isacharparenright}{\kern0pt}\ {\isacharasterisk}{\kern0pt}\ real\ k\ {\isacharasterisk}{\kern0pt}\ {\isacharparenleft}{\kern0pt}{\isadigit{1}}\ {\isacharplus}{\kern0pt}\ real\ v{\isacharparenright}{\kern0pt}\ {\isacharcircum}{\kern0pt}\ {\isacharparenleft}{\kern0pt}k\ {\isacharminus}{\kern0pt}\ Suc\ {\isadigit{0}}{\isacharparenright}{\kern0pt}{\isachardoublequoteclose}\isanewline
\ \ \ \ \ \ \isacommand{using}\isamarkupfalse%
\ k{\isacharunderscore}{\kern0pt}g{\isacharunderscore}{\kern0pt}{\isadigit{0}}\ power{\isacharunderscore}{\kern0pt}diff{\isacharunderscore}{\kern0pt}est{\isacharbrackleft}{\kern0pt}\isakeyword{where}\ a{\isacharequal}{\kern0pt}{\isachardoublequoteopen}Suc\ v{\isachardoublequoteclose}\ \isakeyword{and}\ b{\isacharequal}{\kern0pt}{\isachardoublequoteopen}v{\isachardoublequoteclose}\ \isakeyword{and}\ k\ {\isacharequal}{\kern0pt}{\isachardoublequoteopen}k{\isachardoublequoteclose}{\isacharbrackright}{\kern0pt}\isanewline
\ \ \ \ \ \ \isacommand{by}\isamarkupfalse%
\ simp\isanewline
\ \ \ \ \isacommand{moreover}\isamarkupfalse%
\ \isacommand{have}\isamarkupfalse%
\ {\isachardoublequoteopen}{\isacharparenleft}{\kern0pt}real\ {\isacharparenleft}{\kern0pt}v{\isacharplus}{\kern0pt}{\isadigit{1}}{\isacharparenright}{\kern0pt}\ {\isacharminus}{\kern0pt}\ real\ v{\isacharparenright}{\kern0pt}\ {\isacharequal}{\kern0pt}\ {\isadigit{1}}{\isachardoublequoteclose}\ \isacommand{by}\isamarkupfalse%
\ auto\isanewline
\ \ \ \ \isacommand{ultimately}\isamarkupfalse%
\ \isacommand{have}\isamarkupfalse%
\ {\isachardoublequoteopen}real\ {\isacharparenleft}{\kern0pt}Suc\ v\ {\isacharcircum}{\kern0pt}\ k{\isacharparenright}{\kern0pt}\ {\isacharminus}{\kern0pt}\ real\ {\isacharparenleft}{\kern0pt}v\ {\isacharcircum}{\kern0pt}\ k{\isacharparenright}{\kern0pt}\ {\isasymle}\ real\ k\ {\isacharasterisk}{\kern0pt}\ {\isacharparenleft}{\kern0pt}{\isadigit{1}}\ {\isacharplus}{\kern0pt}\ real\ v{\isacharparenright}{\kern0pt}\ {\isacharcircum}{\kern0pt}\ {\isacharparenleft}{\kern0pt}k\ {\isacharminus}{\kern0pt}\ Suc\ {\isadigit{0}}{\isacharparenright}{\kern0pt}{\isachardoublequoteclose}\isanewline
\ \ \ \ \ \ \isacommand{by}\isamarkupfalse%
\ auto\isanewline
\ \ \ \ \isacommand{also}\isamarkupfalse%
\ \isacommand{have}\isamarkupfalse%
\ {\isachardoublequoteopen}{\isachardot}{\kern0pt}{\isachardot}{\kern0pt}{\isachardot}{\kern0pt}\ {\isasymle}\ real\ k\ {\isacharasterisk}{\kern0pt}\ real\ {\isacharparenleft}{\kern0pt}count{\isacharunderscore}{\kern0pt}list\ as\ a{\isacharparenright}{\kern0pt}\ {\isacharcircum}{\kern0pt}\ {\isacharparenleft}{\kern0pt}k{\isacharminus}{\kern0pt}\ Suc\ {\isadigit{0}}{\isacharparenright}{\kern0pt}{\isachardoublequoteclose}\isanewline
\ \ \ \ \ \ \isacommand{apply}\isamarkupfalse%
\ {\isacharparenleft}{\kern0pt}rule\ mult{\isacharunderscore}{\kern0pt}left{\isacharunderscore}{\kern0pt}mono{\isacharcomma}{\kern0pt}\ rule\ power{\isacharunderscore}{\kern0pt}mono{\isacharparenright}{\kern0pt}\isanewline
\ \ \ \ \ \ \isacommand{using}\isamarkupfalse%
\ c{\isacharunderscore}{\kern0pt}{\isadigit{1}}\ \isacommand{apply}\isamarkupfalse%
\ linarith\isanewline
\ \ \ \ \ \ \isacommand{by}\isamarkupfalse%
\ simp{\isacharplus}{\kern0pt}\isanewline
\ \ \ \ \isacommand{finally}\isamarkupfalse%
\ \isacommand{show}\isamarkupfalse%
\ {\isachardoublequoteopen}real\ {\isacharparenleft}{\kern0pt}Suc\ v\ {\isacharcircum}{\kern0pt}\ k{\isacharparenright}{\kern0pt}\ {\isacharminus}{\kern0pt}\ real\ {\isacharparenleft}{\kern0pt}v\ {\isacharcircum}{\kern0pt}\ k{\isacharparenright}{\kern0pt}\ {\isasymle}\ real\ k\ {\isacharasterisk}{\kern0pt}\ real\ {\isacharparenleft}{\kern0pt}count{\isacharunderscore}{\kern0pt}list\ as\ a{\isacharparenright}{\kern0pt}\ {\isacharcircum}{\kern0pt}\ {\isacharparenleft}{\kern0pt}k{\isacharminus}{\kern0pt}\ Suc\ {\isadigit{0}}{\isacharparenright}{\kern0pt}{\isachardoublequoteclose}\isanewline
\ \ \ \ \ \ \isacommand{by}\isamarkupfalse%
\ blast\isanewline
\ \ \isacommand{qed}\isamarkupfalse%
\isanewline
\ \ \ \ \ \ \isanewline
\ \ \isacommand{have}\isamarkupfalse%
\ {\isachardoublequoteopen}real\ {\isacharparenleft}{\kern0pt}length\ as{\isacharparenright}{\kern0pt}\ {\isacharasterisk}{\kern0pt}\ {\isacharparenleft}{\kern0pt}{\isasymSum}a{\isasymin}\ set\ as{\isachardot}{\kern0pt}\ {\isacharparenleft}{\kern0pt}{\isasymSum}v\ {\isasymin}\ {\isacharbraceleft}{\kern0pt}{\isadigit{0}}{\isachardot}{\kern0pt}{\isachardot}{\kern0pt}{\isacharless}{\kern0pt}\ count{\isacharunderscore}{\kern0pt}list\ as\ a{\isacharbraceright}{\kern0pt}{\isachardot}{\kern0pt}\ {\isacharparenleft}{\kern0pt}real\ {\isacharparenleft}{\kern0pt}Suc\ v\ {\isacharcircum}{\kern0pt}\ k\ {\isacharminus}{\kern0pt}\ v\ {\isacharcircum}{\kern0pt}\ k{\isacharparenright}{\kern0pt}{\isacharparenright}{\kern0pt}\isactrlsup {\isadigit{2}}{\isacharparenright}{\kern0pt}{\isacharparenright}{\kern0pt}\isanewline
\ \ \ \ {\isasymle}\ real\ {\isacharparenleft}{\kern0pt}length\ as{\isacharparenright}{\kern0pt}\ {\isacharasterisk}{\kern0pt}\ {\isacharparenleft}{\kern0pt}{\isasymSum}a{\isasymin}\ set\ as{\isachardot}{\kern0pt}\ {\isacharparenleft}{\kern0pt}{\isasymSum}v\ {\isasymin}\ {\isacharbraceleft}{\kern0pt}{\isadigit{0}}{\isachardot}{\kern0pt}{\isachardot}{\kern0pt}{\isacharless}{\kern0pt}\ count{\isacharunderscore}{\kern0pt}list\ as\ a{\isacharbraceright}{\kern0pt}{\isachardot}{\kern0pt}\ {\isacharparenleft}{\kern0pt}real\ {\isacharparenleft}{\kern0pt}k\ {\isacharasterisk}{\kern0pt}\ count{\isacharunderscore}{\kern0pt}list\ as\ a\ {\isacharcircum}{\kern0pt}\ {\isacharparenleft}{\kern0pt}k{\isacharminus}{\kern0pt}{\isadigit{1}}{\isacharparenright}{\kern0pt}\ {\isacharasterisk}{\kern0pt}\ {\isacharparenleft}{\kern0pt}Suc\ v\ {\isacharcircum}{\kern0pt}\ k\ {\isacharminus}{\kern0pt}\ v\ {\isacharcircum}{\kern0pt}\ k{\isacharparenright}{\kern0pt}{\isacharparenright}{\kern0pt}{\isacharparenright}{\kern0pt}{\isacharparenright}{\kern0pt}{\isacharparenright}{\kern0pt}{\isachardoublequoteclose}\isanewline
\ \ \ \ \isacommand{apply}\isamarkupfalse%
\ {\isacharparenleft}{\kern0pt}rule\ mult{\isacharunderscore}{\kern0pt}left{\isacharunderscore}{\kern0pt}mono{\isacharparenright}{\kern0pt}\isanewline
\ \ \ \ \ \isacommand{apply}\isamarkupfalse%
\ {\isacharparenleft}{\kern0pt}rule\ sum{\isacharunderscore}{\kern0pt}mono{\isacharcomma}{\kern0pt}\ rule\ sum{\isacharunderscore}{\kern0pt}mono{\isacharparenright}{\kern0pt}\isanewline
\ \ \ \ \ \isacommand{apply}\isamarkupfalse%
\ {\isacharparenleft}{\kern0pt}simp\ add{\isacharcolon}{\kern0pt}power{\isadigit{2}}{\isacharunderscore}{\kern0pt}eq{\isacharunderscore}{\kern0pt}square{\isacharparenright}{\kern0pt}\isanewline
\ \ \ \ \ \isacommand{apply}\isamarkupfalse%
\ {\isacharparenleft}{\kern0pt}rule\ mult{\isacharunderscore}{\kern0pt}right{\isacharunderscore}{\kern0pt}mono{\isacharparenright}{\kern0pt}\isanewline
\ \ \ \ \ \ \isacommand{apply}\isamarkupfalse%
\ {\isacharparenleft}{\kern0pt}subst\ of{\isacharunderscore}{\kern0pt}nat{\isacharunderscore}{\kern0pt}diff{\isacharcomma}{\kern0pt}\ simp\ add{\isacharcolon}{\kern0pt}power{\isacharunderscore}{\kern0pt}mono{\isacharparenright}{\kern0pt}\isanewline
\ \ \ \ \isacommand{by}\isamarkupfalse%
\ {\isacharparenleft}{\kern0pt}metis\ c{\isacharcomma}{\kern0pt}\ simp{\isacharcomma}{\kern0pt}\ simp{\isacharparenright}{\kern0pt}\isanewline
\ \ \isacommand{also}\isamarkupfalse%
\ \isacommand{have}\isamarkupfalse%
\ {\isachardoublequoteopen}{\isachardot}{\kern0pt}{\isachardot}{\kern0pt}{\isachardot}{\kern0pt}\ {\isacharequal}{\kern0pt}\ real\ {\isacharparenleft}{\kern0pt}length\ as{\isacharparenright}{\kern0pt}\ {\isacharasterisk}{\kern0pt}\ {\isacharparenleft}{\kern0pt}{\isasymSum}a{\isasymin}\ set\ as{\isachardot}{\kern0pt}\ real\ {\isacharparenleft}{\kern0pt}k\ {\isacharasterisk}{\kern0pt}\ count{\isacharunderscore}{\kern0pt}list\ as\ a\ {\isacharcircum}{\kern0pt}\ {\isacharparenleft}{\kern0pt}{\isadigit{2}}{\isacharasterisk}{\kern0pt}k{\isacharminus}{\kern0pt}{\isadigit{1}}{\isacharparenright}{\kern0pt}{\isacharparenright}{\kern0pt}{\isacharparenright}{\kern0pt}{\isachardoublequoteclose}\isanewline
\ \ \ \ \isacommand{apply}\isamarkupfalse%
\ {\isacharparenleft}{\kern0pt}rule\ arg{\isacharunderscore}{\kern0pt}cong{\isadigit{2}}{\isacharbrackleft}{\kern0pt}\isakeyword{where}\ f{\isacharequal}{\kern0pt}{\isachardoublequoteopen}{\isacharparenleft}{\kern0pt}{\isacharasterisk}{\kern0pt}{\isacharparenright}{\kern0pt}{\isachardoublequoteclose}{\isacharbrackright}{\kern0pt}{\isacharcomma}{\kern0pt}\ simp{\isacharparenright}{\kern0pt}\isanewline
\ \ \ \ \isacommand{apply}\isamarkupfalse%
\ {\isacharparenleft}{\kern0pt}rule\ sum{\isachardot}{\kern0pt}cong{\isacharcomma}{\kern0pt}\ simp{\isacharparenright}{\kern0pt}\isanewline
\ \ \ \ \isacommand{apply}\isamarkupfalse%
\ {\isacharparenleft}{\kern0pt}simp\ add{\isacharcolon}{\kern0pt}sum{\isacharunderscore}{\kern0pt}distrib{\isacharunderscore}{\kern0pt}left{\isacharbrackleft}{\kern0pt}symmetric{\isacharbrackright}{\kern0pt}{\isacharparenright}{\kern0pt}\isanewline
\ \ \ \ \isacommand{apply}\isamarkupfalse%
\ {\isacharparenleft}{\kern0pt}subst\ of{\isacharunderscore}{\kern0pt}nat{\isacharunderscore}{\kern0pt}diff{\isacharcomma}{\kern0pt}\ rule\ power{\isacharunderscore}{\kern0pt}mono{\isacharcomma}{\kern0pt}\ simp{\isacharcomma}{\kern0pt}\ simp{\isacharparenright}{\kern0pt}\isanewline
\ \ \ \ \isacommand{apply}\isamarkupfalse%
\ {\isacharparenleft}{\kern0pt}subst\ sum{\isacharunderscore}{\kern0pt}Suc{\isacharunderscore}{\kern0pt}diff{\isacharprime}{\kern0pt}{\isacharcomma}{\kern0pt}\ simp{\isacharcomma}{\kern0pt}\ simp\ add{\isacharcolon}{\kern0pt}\ zero{\isacharunderscore}{\kern0pt}power{\isacharbrackleft}{\kern0pt}OF\ k{\isacharunderscore}{\kern0pt}g{\isacharunderscore}{\kern0pt}{\isadigit{0}}{\isacharbrackright}{\kern0pt}\ sum{\isacharunderscore}{\kern0pt}distrib{\isacharunderscore}{\kern0pt}left{\isacharparenright}{\kern0pt}\isanewline
\ \ \ \ \isacommand{apply}\isamarkupfalse%
\ {\isacharparenleft}{\kern0pt}subst\ power{\isacharunderscore}{\kern0pt}add{\isacharbrackleft}{\kern0pt}symmetric{\isacharbrackright}{\kern0pt}{\isacharparenright}{\kern0pt}\ \isanewline
\ \ \ \ \isacommand{using}\isamarkupfalse%
\ assms\ \isacommand{by}\isamarkupfalse%
\ {\isacharparenleft}{\kern0pt}simp\ add{\isacharcolon}{\kern0pt}\ mult{\isacharunderscore}{\kern0pt}{\isadigit{2}}{\isacharparenright}{\kern0pt}\isanewline
\ \ \isacommand{also}\isamarkupfalse%
\ \isacommand{have}\isamarkupfalse%
\ {\isachardoublequoteopen}{\isachardot}{\kern0pt}{\isachardot}{\kern0pt}{\isachardot}{\kern0pt}\ {\isacharequal}{\kern0pt}\ real\ {\isacharparenleft}{\kern0pt}length\ as{\isacharparenright}{\kern0pt}\ {\isacharasterisk}{\kern0pt}\ real\ k\ {\isacharasterisk}{\kern0pt}\ real{\isacharunderscore}{\kern0pt}of{\isacharunderscore}{\kern0pt}rat\ {\isacharparenleft}{\kern0pt}F\ {\isacharparenleft}{\kern0pt}{\isadigit{2}}{\isacharasterisk}{\kern0pt}k{\isacharminus}{\kern0pt}{\isadigit{1}}{\isacharparenright}{\kern0pt}\ as{\isacharparenright}{\kern0pt}{\isachardoublequoteclose}\isanewline
\ \ \ \ \isacommand{apply}\isamarkupfalse%
\ {\isacharparenleft}{\kern0pt}subst\ mult{\isachardot}{\kern0pt}assoc{\isacharparenright}{\kern0pt}\isanewline
\ \ \ \ \isacommand{apply}\isamarkupfalse%
\ {\isacharparenleft}{\kern0pt}rule\ arg{\isacharunderscore}{\kern0pt}cong{\isadigit{2}}{\isacharbrackleft}{\kern0pt}\isakeyword{where}\ f{\isacharequal}{\kern0pt}{\isachardoublequoteopen}{\isacharparenleft}{\kern0pt}{\isacharasterisk}{\kern0pt}{\isacharparenright}{\kern0pt}{\isachardoublequoteclose}{\isacharbrackright}{\kern0pt}{\isacharcomma}{\kern0pt}\ simp{\isacharparenright}{\kern0pt}\isanewline
\ \ \ \ \isacommand{by}\isamarkupfalse%
\ {\isacharparenleft}{\kern0pt}simp\ add{\isacharcolon}{\kern0pt}sum{\isacharunderscore}{\kern0pt}distrib{\isacharunderscore}{\kern0pt}left{\isacharbrackleft}{\kern0pt}symmetric{\isacharbrackright}{\kern0pt}\ F{\isacharunderscore}{\kern0pt}def\ of{\isacharunderscore}{\kern0pt}rat{\isacharunderscore}{\kern0pt}sum\ of{\isacharunderscore}{\kern0pt}rat{\isacharunderscore}{\kern0pt}power{\isacharparenright}{\kern0pt}\isanewline
\ \ \isacommand{also}\isamarkupfalse%
\ \isacommand{have}\isamarkupfalse%
\ {\isachardoublequoteopen}{\isachardot}{\kern0pt}{\isachardot}{\kern0pt}{\isachardot}{\kern0pt}\ {\isasymle}\ real\ k\ {\isacharasterisk}{\kern0pt}\ {\isacharparenleft}{\kern0pt}{\isacharparenleft}{\kern0pt}real{\isacharunderscore}{\kern0pt}of{\isacharunderscore}{\kern0pt}rat\ {\isacharparenleft}{\kern0pt}F\ k\ as{\isacharparenright}{\kern0pt}{\isacharparenright}{\kern0pt}\isactrlsup {\isadigit{2}}\ {\isacharasterisk}{\kern0pt}\ real\ n\ powr\ {\isacharparenleft}{\kern0pt}{\isadigit{1}}\ {\isacharminus}{\kern0pt}\ {\isadigit{1}}\ {\isacharslash}{\kern0pt}\ real\ k{\isacharparenright}{\kern0pt}{\isacharparenright}{\kern0pt}{\isachardoublequoteclose}\isanewline
\ \ \ \ \isacommand{apply}\isamarkupfalse%
\ {\isacharparenleft}{\kern0pt}subst\ mult{\isachardot}{\kern0pt}commute{\isacharparenright}{\kern0pt}\isanewline
\ \ \ \ \isacommand{apply}\isamarkupfalse%
\ {\isacharparenleft}{\kern0pt}subst\ mult{\isachardot}{\kern0pt}assoc{\isacharparenright}{\kern0pt}\isanewline
\ \ \ \ \isacommand{apply}\isamarkupfalse%
\ {\isacharparenleft}{\kern0pt}rule\ mult{\isacharunderscore}{\kern0pt}left{\isacharunderscore}{\kern0pt}mono{\isacharparenright}{\kern0pt}\isanewline
\ \ \ \ \isacommand{using}\isamarkupfalse%
\ fk{\isacharunderscore}{\kern0pt}estimate{\isacharbrackleft}{\kern0pt}OF\ assms{\isacharparenleft}{\kern0pt}{\isadigit{1}}{\isacharparenright}{\kern0pt}\ assms{\isacharparenleft}{\kern0pt}{\isadigit{3}}{\isacharparenright}{\kern0pt}\ assms{\isacharparenleft}{\kern0pt}{\isadigit{2}}{\isacharparenright}{\kern0pt}{\isacharbrackright}{\kern0pt}\ \isanewline
\ \ \ \ \isacommand{by}\isamarkupfalse%
\ {\isacharparenleft}{\kern0pt}simp\ add{\isacharcolon}{\kern0pt}\ mult{\isachardot}{\kern0pt}commute{\isacharcomma}{\kern0pt}\ simp{\isacharparenright}{\kern0pt}\isanewline
\ \ \isacommand{finally}\isamarkupfalse%
\ \isacommand{have}\isamarkupfalse%
\ b{\isacharcolon}{\kern0pt}\ {\isachardoublequoteopen}real\ {\isacharparenleft}{\kern0pt}length\ as{\isacharparenright}{\kern0pt}\ {\isacharasterisk}{\kern0pt}\ {\isacharparenleft}{\kern0pt}{\isasymSum}a{\isasymin}\ set\ as{\isachardot}{\kern0pt}\ {\isacharparenleft}{\kern0pt}{\isasymSum}v\ {\isasymin}\ {\isacharbraceleft}{\kern0pt}{\isadigit{0}}{\isachardot}{\kern0pt}{\isachardot}{\kern0pt}{\isacharless}{\kern0pt}\ count{\isacharunderscore}{\kern0pt}list\ as\ a{\isacharbraceright}{\kern0pt}{\isachardot}{\kern0pt}\ {\isacharparenleft}{\kern0pt}real\ {\isacharparenleft}{\kern0pt}Suc\ v\ {\isacharcircum}{\kern0pt}\ k\ {\isacharminus}{\kern0pt}\ v\ {\isacharcircum}{\kern0pt}\ k{\isacharparenright}{\kern0pt}{\isacharparenright}{\kern0pt}\isactrlsup {\isadigit{2}}{\isacharparenright}{\kern0pt}{\isacharparenright}{\kern0pt}\isanewline
\ \ \ \ {\isasymle}\ real\ k\ {\isacharasterisk}{\kern0pt}\ {\isacharparenleft}{\kern0pt}{\isacharparenleft}{\kern0pt}real{\isacharunderscore}{\kern0pt}of{\isacharunderscore}{\kern0pt}rat\ {\isacharparenleft}{\kern0pt}F\ k\ as{\isacharparenright}{\kern0pt}{\isacharparenright}{\kern0pt}\isactrlsup {\isadigit{2}}\ {\isacharasterisk}{\kern0pt}\ real\ n\ powr\ {\isacharparenleft}{\kern0pt}{\isadigit{1}}\ {\isacharminus}{\kern0pt}\ {\isadigit{1}}\ {\isacharslash}{\kern0pt}\ real\ k{\isacharparenright}{\kern0pt}{\isacharparenright}{\kern0pt}{\isachardoublequoteclose}\isanewline
\ \ \ \ \isacommand{by}\isamarkupfalse%
\ blast\isanewline
\isanewline
\ \ \isacommand{have}\isamarkupfalse%
\ {\isachardoublequoteopen}measure{\isacharunderscore}{\kern0pt}pmf{\isachardot}{\kern0pt}expectation\ {\isasymOmega}\ {\isacharparenleft}{\kern0pt}{\isasymlambda}{\isasymomega}{\isachardot}{\kern0pt}\ f\ {\isasymomega}{\isacharcircum}{\kern0pt}{\isadigit{2}}{\isacharparenright}{\kern0pt}\ {\isacharminus}{\kern0pt}\ {\isacharparenleft}{\kern0pt}measure{\isacharunderscore}{\kern0pt}pmf{\isachardot}{\kern0pt}expectation\ {\isasymOmega}\ f{\isacharparenright}{\kern0pt}{\isacharcircum}{\kern0pt}{\isadigit{2}}\ {\isasymle}\ \isanewline
\ \ \ \ measure{\isacharunderscore}{\kern0pt}pmf{\isachardot}{\kern0pt}expectation\ {\isasymOmega}\ {\isacharparenleft}{\kern0pt}{\isasymlambda}{\isasymomega}{\isachardot}{\kern0pt}\ f\ {\isasymomega}{\isacharcircum}{\kern0pt}{\isadigit{2}}{\isacharparenright}{\kern0pt}{\isachardoublequoteclose}\ \isanewline
\ \ \ \ \isacommand{by}\isamarkupfalse%
\ simp\isanewline
\ \ \isacommand{also}\isamarkupfalse%
\ \isacommand{have}\isamarkupfalse%
\ {\isachardoublequoteopen}\ measure{\isacharunderscore}{\kern0pt}pmf{\isachardot}{\kern0pt}expectation\ {\isasymOmega}\ {\isacharparenleft}{\kern0pt}{\isasymlambda}{\isasymomega}{\isachardot}{\kern0pt}\ f\ {\isasymomega}{\isacharcircum}{\kern0pt}{\isadigit{2}}{\isacharparenright}{\kern0pt}\ {\isasymle}\ {\isacharparenleft}{\kern0pt}\isanewline
\ \ \ \ real{\isacharunderscore}{\kern0pt}of{\isacharunderscore}{\kern0pt}rat\ {\isacharparenleft}{\kern0pt}F\ k\ as{\isacharparenright}{\kern0pt}{\isacharparenright}{\kern0pt}\isactrlsup {\isadigit{2}}\ {\isacharasterisk}{\kern0pt}\ real\ k\ {\isacharasterisk}{\kern0pt}\ real\ n\ powr\ {\isacharparenleft}{\kern0pt}{\isadigit{1}}\ {\isacharminus}{\kern0pt}\ {\isadigit{1}}\ {\isacharslash}{\kern0pt}\ real\ k{\isacharparenright}{\kern0pt}{\isachardoublequoteclose}\isanewline
\ \ \ \ \isacommand{apply}\isamarkupfalse%
\ {\isacharparenleft}{\kern0pt}simp\ add{\isacharcolon}{\kern0pt}{\isasymOmega}{\isacharunderscore}{\kern0pt}def\ f{\isacharunderscore}{\kern0pt}def{\isacharparenright}{\kern0pt}\isanewline
\ \ \ \ \isacommand{apply}\isamarkupfalse%
\ {\isacharparenleft}{\kern0pt}subst\ integral{\isacharunderscore}{\kern0pt}measure{\isacharunderscore}{\kern0pt}pmf{\isacharunderscore}{\kern0pt}real{\isacharbrackleft}{\kern0pt}OF\ fin{\isacharunderscore}{\kern0pt}space{\isacharbrackleft}{\kern0pt}OF\ assms{\isacharparenleft}{\kern0pt}{\isadigit{1}}{\isacharparenright}{\kern0pt}{\isacharbrackright}{\kern0pt}{\isacharbrackright}{\kern0pt}{\isacharparenright}{\kern0pt}\isanewline
\ \ \ \ \ \isacommand{apply}\isamarkupfalse%
\ {\isacharparenleft}{\kern0pt}subst\ {\isacharparenleft}{\kern0pt}asm{\isacharparenright}{\kern0pt}\ set{\isacharunderscore}{\kern0pt}pmf{\isacharunderscore}{\kern0pt}of{\isacharunderscore}{\kern0pt}set{\isacharbrackleft}{\kern0pt}OF\ non{\isacharunderscore}{\kern0pt}empty{\isacharunderscore}{\kern0pt}space\ fin{\isacharunderscore}{\kern0pt}space{\isacharbrackright}{\kern0pt}{\isacharcomma}{\kern0pt}\ metis\ assms{\isacharparenleft}{\kern0pt}{\isadigit{1}}{\isacharparenright}{\kern0pt}{\isacharcomma}{\kern0pt}\ simp{\isacharparenright}{\kern0pt}\isanewline
\ \ \ \ \isacommand{apply}\isamarkupfalse%
\ {\isacharparenleft}{\kern0pt}subst\ pmf{\isacharunderscore}{\kern0pt}of{\isacharunderscore}{\kern0pt}set{\isacharbrackleft}{\kern0pt}OF\ non{\isacharunderscore}{\kern0pt}empty{\isacharunderscore}{\kern0pt}space\ fin{\isacharunderscore}{\kern0pt}space{\isacharbrackright}{\kern0pt}{\isacharcomma}{\kern0pt}\ metis\ assms{\isacharparenleft}{\kern0pt}{\isadigit{1}}{\isacharparenright}{\kern0pt}{\isacharparenright}{\kern0pt}\isanewline
\ \ \ \ \isacommand{apply}\isamarkupfalse%
\ {\isacharparenleft}{\kern0pt}simp\ add{\isacharcolon}{\kern0pt}card{\isacharunderscore}{\kern0pt}space{\isacharbrackleft}{\kern0pt}OF\ assms{\isacharparenleft}{\kern0pt}{\isadigit{1}}{\isacharparenright}{\kern0pt}{\isacharbrackright}{\kern0pt}\ power{\isacharunderscore}{\kern0pt}mult{\isacharunderscore}{\kern0pt}distrib{\isacharparenright}{\kern0pt}\isanewline
\ \ \ \ \isacommand{apply}\isamarkupfalse%
\ {\isacharparenleft}{\kern0pt}subst\ mult{\isachardot}{\kern0pt}commute{\isacharcomma}{\kern0pt}\ subst\ {\isacharparenleft}{\kern0pt}{\isadigit{2}}{\isacharparenright}{\kern0pt}\ power{\isadigit{2}}{\isacharunderscore}{\kern0pt}eq{\isacharunderscore}{\kern0pt}square{\isacharcomma}{\kern0pt}\ subst\ split{\isacharunderscore}{\kern0pt}space{\isacharparenright}{\kern0pt}\isanewline
\ \ \ \ \isacommand{using}\isamarkupfalse%
\ assms{\isacharparenleft}{\kern0pt}{\isadigit{1}}{\isacharparenright}{\kern0pt}\ \isacommand{by}\isamarkupfalse%
\ {\isacharparenleft}{\kern0pt}simp\ add{\isacharcolon}{\kern0pt}algebra{\isacharunderscore}{\kern0pt}simps\ sum{\isacharunderscore}{\kern0pt}distrib{\isacharunderscore}{\kern0pt}left{\isacharbrackleft}{\kern0pt}symmetric{\isacharbrackright}{\kern0pt}\ b{\isacharparenright}{\kern0pt}\isanewline
\ \ \isacommand{finally}\isamarkupfalse%
\ \isacommand{have}\isamarkupfalse%
\ a{\isacharcolon}{\kern0pt}{\isachardoublequoteopen}measure{\isacharunderscore}{\kern0pt}pmf{\isachardot}{\kern0pt}expectation\ {\isasymOmega}\ {\isacharparenleft}{\kern0pt}{\isasymlambda}{\isasymomega}{\isachardot}{\kern0pt}\ f\ {\isasymomega}{\isacharcircum}{\kern0pt}{\isadigit{2}}{\isacharparenright}{\kern0pt}\ {\isacharminus}{\kern0pt}\ {\isacharparenleft}{\kern0pt}measure{\isacharunderscore}{\kern0pt}pmf{\isachardot}{\kern0pt}expectation\ {\isasymOmega}\ f{\isacharparenright}{\kern0pt}{\isacharcircum}{\kern0pt}{\isadigit{2}}\ {\isasymle}\ \isanewline
\ \ \ \ {\isacharparenleft}{\kern0pt}real{\isacharunderscore}{\kern0pt}of{\isacharunderscore}{\kern0pt}rat\ {\isacharparenleft}{\kern0pt}F\ k\ as{\isacharparenright}{\kern0pt}{\isacharparenright}{\kern0pt}\isactrlsup {\isadigit{2}}\ {\isacharasterisk}{\kern0pt}\ real\ k\ {\isacharasterisk}{\kern0pt}\ real\ n\ powr\ {\isacharparenleft}{\kern0pt}{\isadigit{1}}\ {\isacharminus}{\kern0pt}\ {\isadigit{1}}\ {\isacharslash}{\kern0pt}\ real\ k{\isacharparenright}{\kern0pt}{\isachardoublequoteclose}\isanewline
\ \ \ \ \isacommand{by}\isamarkupfalse%
\ blast\isanewline
\isanewline
\ \ \isacommand{show}\isamarkupfalse%
\ {\isacharquery}{\kern0pt}thesis\isanewline
\ \ \ \ \isacommand{apply}\isamarkupfalse%
\ {\isacharparenleft}{\kern0pt}subst\ measure{\isacharunderscore}{\kern0pt}pmf{\isachardot}{\kern0pt}variance{\isacharunderscore}{\kern0pt}eq{\isacharparenright}{\kern0pt}\isanewline
\ \ \ \ \isacommand{apply}\isamarkupfalse%
\ {\isacharparenleft}{\kern0pt}subst\ {\isasymOmega}{\isacharunderscore}{\kern0pt}def{\isacharbrackleft}{\kern0pt}symmetric{\isacharbrackright}{\kern0pt}{\isacharcomma}{\kern0pt}\ metis\ integrable{\isacharparenright}{\kern0pt}\isanewline
\ \ \ \ \isacommand{apply}\isamarkupfalse%
\ {\isacharparenleft}{\kern0pt}subst\ {\isasymOmega}{\isacharunderscore}{\kern0pt}def{\isacharbrackleft}{\kern0pt}symmetric{\isacharbrackright}{\kern0pt}{\isacharcomma}{\kern0pt}\ metis\ integrable{\isacharparenright}{\kern0pt}\isanewline
\ \ \ \ \isacommand{apply}\isamarkupfalse%
\ {\isacharparenleft}{\kern0pt}simp\ add{\isacharcolon}{\kern0pt}\ {\isasymOmega}{\isacharunderscore}{\kern0pt}def{\isacharbrackleft}{\kern0pt}symmetric{\isacharbrackright}{\kern0pt}{\isacharparenright}{\kern0pt}\isanewline
\ \ \ \ \isacommand{using}\isamarkupfalse%
\ a\ f{\isacharunderscore}{\kern0pt}def\ \isacommand{by}\isamarkupfalse%
\ simp\isanewline
\isacommand{qed}\isamarkupfalse%
%
\endisatagproof
{\isafoldproof}%
%
\isadelimproof
\isanewline
%
\endisadelimproof
\isanewline
\isacommand{theorem}\isamarkupfalse%
\ fk{\isacharunderscore}{\kern0pt}alg{\isacharunderscore}{\kern0pt}sketch{\isacharcolon}{\kern0pt}\isanewline
\ \ \isakeyword{fixes}\ {\isasymepsilon}\ {\isacharcolon}{\kern0pt}{\isacharcolon}{\kern0pt}\ rat\isanewline
\ \ \isakeyword{assumes}\ {\isachardoublequoteopen}k\ {\isasymge}\ {\isadigit{1}}{\isachardoublequoteclose}\isanewline
\ \ \isakeyword{assumes}\ {\isachardoublequoteopen}{\isasymdelta}\ {\isachargreater}{\kern0pt}\ {\isadigit{0}}{\isachardoublequoteclose}\isanewline
\ \ \isakeyword{assumes}\ {\isachardoublequoteopen}set\ as\ {\isasymsubseteq}\ {\isacharbraceleft}{\kern0pt}{\isadigit{0}}{\isachardot}{\kern0pt}{\isachardot}{\kern0pt}{\isacharless}{\kern0pt}n{\isacharbraceright}{\kern0pt}{\isachardoublequoteclose}\isanewline
\ \ \isakeyword{assumes}\ {\isachardoublequoteopen}as\ {\isasymnoteq}\ {\isacharbrackleft}{\kern0pt}{\isacharbrackright}{\kern0pt}{\isachardoublequoteclose}\isanewline
\ \ \isakeyword{defines}\ {\isachardoublequoteopen}sketch\ {\isasymequiv}\ fold\ {\isacharparenleft}{\kern0pt}{\isasymlambda}a\ state{\isachardot}{\kern0pt}\ state\ {\isasymbind}\ fk{\isacharunderscore}{\kern0pt}update\ a{\isacharparenright}{\kern0pt}\ as\ {\isacharparenleft}{\kern0pt}fk{\isacharunderscore}{\kern0pt}init\ k\ {\isasymdelta}\ {\isasymepsilon}\ n{\isacharparenright}{\kern0pt}{\isachardoublequoteclose}\isanewline
\ \ \isakeyword{defines}\ {\isachardoublequoteopen}s\isactrlsub {\isadigit{1}}\ {\isasymequiv}\ nat\ {\isasymlceil}{\isadigit{3}}{\isacharasterisk}{\kern0pt}real\ k{\isacharasterisk}{\kern0pt}{\isacharparenleft}{\kern0pt}real\ n{\isacharparenright}{\kern0pt}\ powr\ {\isacharparenleft}{\kern0pt}{\isadigit{1}}{\isacharminus}{\kern0pt}{\isadigit{1}}{\isacharslash}{\kern0pt}\ real\ k{\isacharparenright}{\kern0pt}{\isacharslash}{\kern0pt}\ {\isacharparenleft}{\kern0pt}real{\isacharunderscore}{\kern0pt}of{\isacharunderscore}{\kern0pt}rat\ {\isasymdelta}{\isacharparenright}{\kern0pt}\isactrlsup {\isadigit{2}}{\isasymrceil}{\isachardoublequoteclose}\isanewline
\ \ \isakeyword{defines}\ {\isachardoublequoteopen}s\isactrlsub {\isadigit{2}}\ {\isasymequiv}\ nat\ {\isasymlceil}{\isacharminus}{\kern0pt}{\isacharparenleft}{\kern0pt}{\isadigit{1}}{\isadigit{8}}\ {\isacharasterisk}{\kern0pt}\ ln\ {\isacharparenleft}{\kern0pt}real{\isacharunderscore}{\kern0pt}of{\isacharunderscore}{\kern0pt}rat\ {\isasymepsilon}{\isacharparenright}{\kern0pt}{\isacharparenright}{\kern0pt}{\isasymrceil}{\isachardoublequoteclose}\isanewline
\ \ \isakeyword{shows}\ {\isachardoublequoteopen}sketch\ {\isacharequal}{\kern0pt}\ map{\isacharunderscore}{\kern0pt}pmf\ {\isacharparenleft}{\kern0pt}{\isasymlambda}x{\isachardot}{\kern0pt}\ {\isacharparenleft}{\kern0pt}s\isactrlsub {\isadigit{1}}{\isacharcomma}{\kern0pt}s\isactrlsub {\isadigit{2}}{\isacharcomma}{\kern0pt}k{\isacharcomma}{\kern0pt}length\ as{\isacharcomma}{\kern0pt}\ x{\isacharparenright}{\kern0pt}{\isacharparenright}{\kern0pt}\ \isanewline
\ \ \ \ {\isacharparenleft}{\kern0pt}prod{\isacharunderscore}{\kern0pt}pmf\ {\isacharparenleft}{\kern0pt}{\isacharbraceleft}{\kern0pt}{\isadigit{0}}{\isachardot}{\kern0pt}{\isachardot}{\kern0pt}{\isacharless}{\kern0pt}s\isactrlsub {\isadigit{1}}{\isacharbraceright}{\kern0pt}\ {\isasymtimes}\ {\isacharbraceleft}{\kern0pt}{\isadigit{0}}{\isachardot}{\kern0pt}{\isachardot}{\kern0pt}{\isacharless}{\kern0pt}s\isactrlsub {\isadigit{2}}{\isacharbraceright}{\kern0pt}{\isacharparenright}{\kern0pt}\ {\isacharparenleft}{\kern0pt}{\isasymlambda}{\isacharunderscore}{\kern0pt}{\isachardot}{\kern0pt}\ pmf{\isacharunderscore}{\kern0pt}of{\isacharunderscore}{\kern0pt}set\ {\isacharbraceleft}{\kern0pt}{\isacharparenleft}{\kern0pt}u{\isacharcomma}{\kern0pt}v{\isacharparenright}{\kern0pt}{\isachardot}{\kern0pt}\ v\ {\isacharless}{\kern0pt}\ count{\isacharunderscore}{\kern0pt}list\ as\ u{\isacharbraceright}{\kern0pt}{\isacharparenright}{\kern0pt}{\isacharparenright}{\kern0pt}{\isachardoublequoteclose}\ \isanewline
%
\isadelimproof
\ \ %
\endisadelimproof
%
\isatagproof
\isacommand{apply}\isamarkupfalse%
\ {\isacharparenleft}{\kern0pt}simp\ add{\isacharcolon}{\kern0pt}sketch{\isacharunderscore}{\kern0pt}def{\isacharparenright}{\kern0pt}\isanewline
\ \ \isacommand{using}\isamarkupfalse%
\ fk{\isacharunderscore}{\kern0pt}alg{\isacharunderscore}{\kern0pt}aux{\isacharunderscore}{\kern0pt}{\isadigit{1}}{\isacharbrackleft}{\kern0pt}OF\ assms{\isacharparenleft}{\kern0pt}{\isadigit{2}}{\isacharparenright}{\kern0pt}\ assms{\isacharparenleft}{\kern0pt}{\isadigit{3}}{\isacharparenright}{\kern0pt}\ assms{\isacharparenleft}{\kern0pt}{\isadigit{4}}{\isacharparenright}{\kern0pt}{\isacharcomma}{\kern0pt}\ \isakeyword{where}\ k{\isacharequal}{\kern0pt}{\isachardoublequoteopen}k{\isachardoublequoteclose}\ \isakeyword{and}\ {\isasymepsilon}{\isacharequal}{\kern0pt}{\isachardoublequoteopen}{\isasymepsilon}{\isachardoublequoteclose}{\isacharbrackright}{\kern0pt}\isanewline
\ \ \isacommand{apply}\isamarkupfalse%
\ {\isacharparenleft}{\kern0pt}simp\ add{\isacharcolon}{\kern0pt}s\isactrlsub {\isadigit{1}}{\isacharunderscore}{\kern0pt}def{\isacharbrackleft}{\kern0pt}symmetric{\isacharbrackright}{\kern0pt}\ s\isactrlsub {\isadigit{2}}{\isacharunderscore}{\kern0pt}def{\isacharbrackleft}{\kern0pt}symmetric{\isacharbrackright}{\kern0pt}{\isacharparenright}{\kern0pt}\isanewline
\ \ \isacommand{apply}\isamarkupfalse%
\ {\isacharparenleft}{\kern0pt}rule\ arg{\isacharunderscore}{\kern0pt}cong{\isadigit{2}}{\isacharbrackleft}{\kern0pt}\isakeyword{where}\ f{\isacharequal}{\kern0pt}{\isachardoublequoteopen}map{\isacharunderscore}{\kern0pt}pmf{\isachardoublequoteclose}{\isacharbrackright}{\kern0pt}{\isacharcomma}{\kern0pt}\ simp{\isacharparenright}{\kern0pt}\isanewline
\ \ \isacommand{using}\isamarkupfalse%
\ fk{\isacharunderscore}{\kern0pt}alg{\isacharunderscore}{\kern0pt}aux{\isacharunderscore}{\kern0pt}{\isadigit{2}}\isanewline
\ \ \isacommand{apply}\isamarkupfalse%
\ {\isacharparenleft}{\kern0pt}subst\ fk{\isacharunderscore}{\kern0pt}alg{\isacharunderscore}{\kern0pt}aux{\isacharunderscore}{\kern0pt}{\isadigit{2}}{\isacharbrackleft}{\kern0pt}simplified{\isacharbrackright}{\kern0pt}{\isacharcomma}{\kern0pt}\ simp{\isacharparenright}{\kern0pt}\isanewline
\ \ \isacommand{apply}\isamarkupfalse%
\ {\isacharparenleft}{\kern0pt}subst\ fk{\isacharunderscore}{\kern0pt}alg{\isacharunderscore}{\kern0pt}aux{\isacharunderscore}{\kern0pt}{\isadigit{4}}{\isacharbrackleft}{\kern0pt}OF\ assms{\isacharparenleft}{\kern0pt}{\isadigit{4}}{\isacharparenright}{\kern0pt}{\isacharcomma}{\kern0pt}\ simplified{\isacharbrackright}{\kern0pt}{\isacharcomma}{\kern0pt}\ simp{\isacharparenright}{\kern0pt}\isanewline
\ \ \isacommand{by}\isamarkupfalse%
\ {\isacharparenleft}{\kern0pt}subst\ fk{\isacharunderscore}{\kern0pt}alg{\isacharunderscore}{\kern0pt}aux{\isacharunderscore}{\kern0pt}{\isadigit{5}}{\isacharbrackleft}{\kern0pt}OF\ assms{\isacharparenleft}{\kern0pt}{\isadigit{4}}{\isacharparenright}{\kern0pt}{\isacharcomma}{\kern0pt}\ simplified{\isacharbrackright}{\kern0pt}{\isacharcomma}{\kern0pt}\ simp{\isacharparenright}{\kern0pt}%
\endisatagproof
{\isafoldproof}%
%
\isadelimproof
\isanewline
%
\endisadelimproof
\isanewline
\isacommand{lemma}\isamarkupfalse%
\ fk{\isacharunderscore}{\kern0pt}alg{\isacharunderscore}{\kern0pt}correct{\isacharcolon}{\kern0pt}\isanewline
\ \ \isakeyword{assumes}\ {\isachardoublequoteopen}k\ {\isasymge}\ {\isadigit{1}}{\isachardoublequoteclose}\isanewline
\ \ \isakeyword{assumes}\ {\isachardoublequoteopen}{\isasymepsilon}\ {\isasymin}\ {\isacharbraceleft}{\kern0pt}{\isadigit{0}}{\isacharless}{\kern0pt}{\isachardot}{\kern0pt}{\isachardot}{\kern0pt}{\isacharless}{\kern0pt}{\isadigit{1}}{\isacharbraceright}{\kern0pt}{\isachardoublequoteclose}\isanewline
\ \ \isakeyword{assumes}\ {\isachardoublequoteopen}{\isasymdelta}\ {\isachargreater}{\kern0pt}\ {\isadigit{0}}{\isachardoublequoteclose}\isanewline
\ \ \isakeyword{assumes}\ {\isachardoublequoteopen}set\ as\ {\isasymsubseteq}\ {\isacharbraceleft}{\kern0pt}{\isadigit{0}}{\isachardot}{\kern0pt}{\isachardot}{\kern0pt}{\isacharless}{\kern0pt}n{\isacharbraceright}{\kern0pt}{\isachardoublequoteclose}\isanewline
\ \ \isakeyword{defines}\ {\isachardoublequoteopen}M\ {\isasymequiv}\ fold\ {\isacharparenleft}{\kern0pt}{\isasymlambda}a\ state{\isachardot}{\kern0pt}\ state\ {\isasymbind}\ fk{\isacharunderscore}{\kern0pt}update\ a{\isacharparenright}{\kern0pt}\ as\ {\isacharparenleft}{\kern0pt}fk{\isacharunderscore}{\kern0pt}init\ k\ {\isasymdelta}\ {\isasymepsilon}\ n{\isacharparenright}{\kern0pt}\ {\isasymbind}\ fk{\isacharunderscore}{\kern0pt}result{\isachardoublequoteclose}\isanewline
\ \ \isakeyword{shows}\ {\isachardoublequoteopen}{\isasymP}{\isacharparenleft}{\kern0pt}{\isasymomega}\ in\ measure{\isacharunderscore}{\kern0pt}pmf\ M{\isachardot}{\kern0pt}\ {\isasymbar}{\isasymomega}\ {\isacharminus}{\kern0pt}\ F\ k\ as{\isasymbar}\ {\isasymle}\ {\isasymdelta}\ {\isacharasterisk}{\kern0pt}\ F\ k\ as{\isacharparenright}{\kern0pt}\ {\isasymge}\ {\isadigit{1}}\ {\isacharminus}{\kern0pt}\ of{\isacharunderscore}{\kern0pt}rat\ {\isasymepsilon}{\isachardoublequoteclose}\isanewline
%
\isadelimproof
%
\endisadelimproof
%
\isatagproof
\isacommand{proof}\isamarkupfalse%
\ {\isacharparenleft}{\kern0pt}cases\ {\isachardoublequoteopen}as\ {\isacharequal}{\kern0pt}\ {\isacharbrackleft}{\kern0pt}{\isacharbrackright}{\kern0pt}{\isachardoublequoteclose}{\isacharparenright}{\kern0pt}\isanewline
\ \ \isacommand{case}\isamarkupfalse%
\ True\isanewline
\ \ \isacommand{have}\isamarkupfalse%
\ a{\isacharcolon}{\kern0pt}\ {\isachardoublequoteopen}nat\ {\isasymlceil}{\isacharminus}{\kern0pt}\ {\isacharparenleft}{\kern0pt}{\isadigit{1}}{\isadigit{8}}\ {\isacharasterisk}{\kern0pt}\ ln\ {\isacharparenleft}{\kern0pt}real{\isacharunderscore}{\kern0pt}of{\isacharunderscore}{\kern0pt}rat\ {\isasymepsilon}{\isacharparenright}{\kern0pt}{\isacharparenright}{\kern0pt}{\isasymrceil}\ {\isachargreater}{\kern0pt}\ {\isadigit{0}}{\isachardoublequoteclose}\ \ \isacommand{using}\isamarkupfalse%
\ assms\ \isacommand{by}\isamarkupfalse%
\ simp\ \isanewline
\ \ \isacommand{show}\isamarkupfalse%
\ {\isacharquery}{\kern0pt}thesis\ \isacommand{using}\isamarkupfalse%
\ True\ \isacommand{apply}\isamarkupfalse%
\ {\isacharparenleft}{\kern0pt}simp\ add{\isacharcolon}{\kern0pt}F{\isacharunderscore}{\kern0pt}def\ M{\isacharunderscore}{\kern0pt}def\ bind{\isacharunderscore}{\kern0pt}return{\isacharunderscore}{\kern0pt}pmf\ median{\isacharunderscore}{\kern0pt}const{\isacharbrackleft}{\kern0pt}OF\ a{\isacharbrackright}{\kern0pt}\ Let{\isacharunderscore}{\kern0pt}def{\isacharparenright}{\kern0pt}\isanewline
\ \ \ \ \isacommand{using}\isamarkupfalse%
\ assms{\isacharparenleft}{\kern0pt}{\isadigit{2}}{\isacharparenright}{\kern0pt}\ \isacommand{by}\isamarkupfalse%
\ simp\isanewline
\isacommand{next}\isamarkupfalse%
\isanewline
\ \ \isacommand{case}\isamarkupfalse%
\ False\isanewline
\ \ \isacommand{define}\isamarkupfalse%
\ s\isactrlsub {\isadigit{1}}\ \isakeyword{where}\ {\isachardoublequoteopen}s\isactrlsub {\isadigit{1}}\ {\isacharequal}{\kern0pt}\ nat\ {\isasymlceil}{\isadigit{3}}{\isacharasterisk}{\kern0pt}real\ k{\isacharasterisk}{\kern0pt}{\isacharparenleft}{\kern0pt}real\ n{\isacharparenright}{\kern0pt}\ powr\ {\isacharparenleft}{\kern0pt}{\isadigit{1}}{\isacharminus}{\kern0pt}{\isadigit{1}}{\isacharslash}{\kern0pt}\ real\ k{\isacharparenright}{\kern0pt}{\isacharslash}{\kern0pt}\ {\isacharparenleft}{\kern0pt}real{\isacharunderscore}{\kern0pt}of{\isacharunderscore}{\kern0pt}rat\ {\isasymdelta}{\isacharparenright}{\kern0pt}\isactrlsup {\isadigit{2}}{\isasymrceil}{\isachardoublequoteclose}\isanewline
\ \ \isacommand{define}\isamarkupfalse%
\ s\isactrlsub {\isadigit{2}}\ \isakeyword{where}\ {\isachardoublequoteopen}s\isactrlsub {\isadigit{2}}\ {\isacharequal}{\kern0pt}\ nat\ {\isasymlceil}{\isacharminus}{\kern0pt}{\isacharparenleft}{\kern0pt}{\isadigit{1}}{\isadigit{8}}\ {\isacharasterisk}{\kern0pt}\ ln\ {\isacharparenleft}{\kern0pt}real{\isacharunderscore}{\kern0pt}of{\isacharunderscore}{\kern0pt}rat\ {\isasymepsilon}{\isacharparenright}{\kern0pt}{\isacharparenright}{\kern0pt}{\isasymrceil}{\isachardoublequoteclose}\isanewline
\isanewline
\ \ \isacommand{define}\isamarkupfalse%
\ f\ {\isacharcolon}{\kern0pt}{\isacharcolon}{\kern0pt}\ {\isachardoublequoteopen}{\isacharparenleft}{\kern0pt}nat\ {\isasymtimes}\ nat\ {\isasymRightarrow}\ {\isacharparenleft}{\kern0pt}nat\ {\isasymtimes}\ nat{\isacharparenright}{\kern0pt}{\isacharparenright}{\kern0pt}\ {\isasymRightarrow}\ rat{\isachardoublequoteclose}\isanewline
\ \ \ \ \isakeyword{where}\ {\isachardoublequoteopen}f\ {\isacharequal}{\kern0pt}\ {\isacharparenleft}{\kern0pt}{\isasymlambda}x{\isachardot}{\kern0pt}\ median\ s\isactrlsub {\isadigit{2}}\ {\isacharparenleft}{\kern0pt}{\isasymlambda}i\isactrlsub {\isadigit{2}}{\isasymin}{\isacharbraceleft}{\kern0pt}{\isadigit{0}}{\isachardot}{\kern0pt}{\isachardot}{\kern0pt}{\isacharless}{\kern0pt}s\isactrlsub {\isadigit{2}}{\isacharbraceright}{\kern0pt}{\isachardot}{\kern0pt}\isanewline
\ \ \ \ \ \ \ {\isacharparenleft}{\kern0pt}{\isasymSum}i\isactrlsub {\isadigit{1}}\ {\isacharequal}{\kern0pt}\ {\isadigit{0}}{\isachardot}{\kern0pt}{\isachardot}{\kern0pt}{\isacharless}{\kern0pt}s\isactrlsub {\isadigit{1}}{\isachardot}{\kern0pt}\ rat{\isacharunderscore}{\kern0pt}of{\isacharunderscore}{\kern0pt}nat\ {\isacharparenleft}{\kern0pt}length\ as\ {\isacharasterisk}{\kern0pt}\ {\isacharparenleft}{\kern0pt}Suc\ {\isacharparenleft}{\kern0pt}snd\ {\isacharparenleft}{\kern0pt}x\ {\isacharparenleft}{\kern0pt}i\isactrlsub {\isadigit{1}}{\isacharcomma}{\kern0pt}\ i\isactrlsub {\isadigit{2}}{\isacharparenright}{\kern0pt}{\isacharparenright}{\kern0pt}{\isacharparenright}{\kern0pt}\ {\isacharcircum}{\kern0pt}\ k\ {\isacharminus}{\kern0pt}\ snd\ {\isacharparenleft}{\kern0pt}x\ {\isacharparenleft}{\kern0pt}i\isactrlsub {\isadigit{1}}{\isacharcomma}{\kern0pt}\ i\isactrlsub {\isadigit{2}}{\isacharparenright}{\kern0pt}{\isacharparenright}{\kern0pt}\ {\isacharcircum}{\kern0pt}\ k{\isacharparenright}{\kern0pt}{\isacharparenright}{\kern0pt}{\isacharparenright}{\kern0pt}\ {\isacharslash}{\kern0pt}\isanewline
\ \ \ \ \ \ \ rat{\isacharunderscore}{\kern0pt}of{\isacharunderscore}{\kern0pt}nat\ s\isactrlsub {\isadigit{1}}{\isacharparenright}{\kern0pt}{\isacharparenright}{\kern0pt}{\isachardoublequoteclose}\isanewline
\isanewline
\ \ \isacommand{define}\isamarkupfalse%
\ f{\isadigit{2}}\ {\isacharcolon}{\kern0pt}{\isacharcolon}{\kern0pt}\ {\isachardoublequoteopen}{\isacharparenleft}{\kern0pt}nat\ {\isasymtimes}\ nat\ {\isasymRightarrow}\ {\isacharparenleft}{\kern0pt}nat\ {\isasymtimes}\ nat{\isacharparenright}{\kern0pt}{\isacharparenright}{\kern0pt}\ {\isasymRightarrow}\ {\isacharparenleft}{\kern0pt}nat\ {\isasymRightarrow}\ nat\ {\isasymRightarrow}\ real{\isacharparenright}{\kern0pt}{\isachardoublequoteclose}\isanewline
\ \ \ \ \isakeyword{where}\ {\isachardoublequoteopen}f{\isadigit{2}}\ {\isacharequal}{\kern0pt}\ {\isacharparenleft}{\kern0pt}{\isasymlambda}x\ i\isactrlsub {\isadigit{1}}\ i\isactrlsub {\isadigit{2}}{\isachardot}{\kern0pt}\ real\ {\isacharparenleft}{\kern0pt}length\ as\ {\isacharasterisk}{\kern0pt}\ {\isacharparenleft}{\kern0pt}Suc\ {\isacharparenleft}{\kern0pt}snd\ {\isacharparenleft}{\kern0pt}x\ {\isacharparenleft}{\kern0pt}i\isactrlsub {\isadigit{1}}{\isacharcomma}{\kern0pt}\ i\isactrlsub {\isadigit{2}}{\isacharparenright}{\kern0pt}{\isacharparenright}{\kern0pt}{\isacharparenright}{\kern0pt}\ {\isacharcircum}{\kern0pt}\ k\ {\isacharminus}{\kern0pt}\ snd\ {\isacharparenleft}{\kern0pt}x\ {\isacharparenleft}{\kern0pt}i\isactrlsub {\isadigit{1}}{\isacharcomma}{\kern0pt}\ i\isactrlsub {\isadigit{2}}{\isacharparenright}{\kern0pt}{\isacharparenright}{\kern0pt}\ {\isacharcircum}{\kern0pt}\ k{\isacharparenright}{\kern0pt}{\isacharparenright}{\kern0pt}{\isacharparenright}{\kern0pt}{\isachardoublequoteclose}\isanewline
\ \ \isacommand{define}\isamarkupfalse%
\ f{\isadigit{1}}\ {\isacharcolon}{\kern0pt}{\isacharcolon}{\kern0pt}\ {\isachardoublequoteopen}{\isacharparenleft}{\kern0pt}nat\ {\isasymtimes}\ nat\ {\isasymRightarrow}\ {\isacharparenleft}{\kern0pt}nat\ {\isasymtimes}\ nat{\isacharparenright}{\kern0pt}{\isacharparenright}{\kern0pt}\ {\isasymRightarrow}\ {\isacharparenleft}{\kern0pt}nat\ {\isasymRightarrow}\ real{\isacharparenright}{\kern0pt}{\isachardoublequoteclose}\isanewline
\ \ \ \ \isakeyword{where}\ {\isachardoublequoteopen}f{\isadigit{1}}\ {\isacharequal}{\kern0pt}\ {\isacharparenleft}{\kern0pt}{\isasymlambda}x\ i\isactrlsub {\isadigit{2}}{\isachardot}{\kern0pt}\ {\isacharparenleft}{\kern0pt}{\isasymSum}i\isactrlsub {\isadigit{1}}\ {\isacharequal}{\kern0pt}\ {\isadigit{0}}{\isachardot}{\kern0pt}{\isachardot}{\kern0pt}{\isacharless}{\kern0pt}s\isactrlsub {\isadigit{1}}{\isachardot}{\kern0pt}\ f{\isadigit{2}}\ x\ i\isactrlsub {\isadigit{1}}\ i\isactrlsub {\isadigit{2}}{\isacharparenright}{\kern0pt}\ {\isacharslash}{\kern0pt}\ real\ s\isactrlsub {\isadigit{1}}{\isacharparenright}{\kern0pt}{\isachardoublequoteclose}\isanewline
\ \ \isacommand{define}\isamarkupfalse%
\ f{\isacharprime}{\kern0pt}\ {\isacharcolon}{\kern0pt}{\isacharcolon}{\kern0pt}\ {\isachardoublequoteopen}{\isacharparenleft}{\kern0pt}nat\ {\isasymtimes}\ nat\ {\isasymRightarrow}\ {\isacharparenleft}{\kern0pt}nat\ {\isasymtimes}\ nat{\isacharparenright}{\kern0pt}{\isacharparenright}{\kern0pt}\ {\isasymRightarrow}\ real{\isachardoublequoteclose}\isanewline
\ \ \ \ \isakeyword{where}\ {\isachardoublequoteopen}f{\isacharprime}{\kern0pt}\ {\isacharequal}{\kern0pt}\ {\isacharparenleft}{\kern0pt}{\isasymlambda}x{\isachardot}{\kern0pt}\ median\ s\isactrlsub {\isadigit{2}}\ {\isacharparenleft}{\kern0pt}f{\isadigit{1}}\ x{\isacharparenright}{\kern0pt}{\isacharparenright}{\kern0pt}{\isachardoublequoteclose}\isanewline
\isanewline
\ \ \isacommand{have}\isamarkupfalse%
\ {\isachardoublequoteopen}set\ as\ {\isasymnoteq}\ {\isacharbraceleft}{\kern0pt}{\isacharbraceright}{\kern0pt}{\isachardoublequoteclose}\ \isacommand{using}\isamarkupfalse%
\ assms\ False\ \isacommand{by}\isamarkupfalse%
\ blast\isanewline
\ \ \isacommand{hence}\isamarkupfalse%
\ n{\isacharunderscore}{\kern0pt}nonzero{\isacharcolon}{\kern0pt}\ {\isachardoublequoteopen}n\ {\isachargreater}{\kern0pt}\ {\isadigit{0}}{\isachardoublequoteclose}\ \isacommand{using}\isamarkupfalse%
\ assms{\isacharparenleft}{\kern0pt}{\isadigit{4}}{\isacharparenright}{\kern0pt}\ \isacommand{by}\isamarkupfalse%
\ fastforce\isanewline
\isanewline
\ \ \isacommand{have}\isamarkupfalse%
\ fk{\isacharunderscore}{\kern0pt}nonzero{\isacharcolon}{\kern0pt}\ {\isachardoublequoteopen}F\ k\ as\ {\isachargreater}{\kern0pt}\ {\isadigit{0}}{\isachardoublequoteclose}\ \isacommand{using}\isamarkupfalse%
\ F{\isacharunderscore}{\kern0pt}gr{\isacharunderscore}{\kern0pt}{\isadigit{0}}\ assms\ False\ \isacommand{by}\isamarkupfalse%
\ simp\isanewline
\isanewline
\ \ \isacommand{have}\isamarkupfalse%
\ s{\isadigit{1}}{\isacharunderscore}{\kern0pt}nonzero{\isacharcolon}{\kern0pt}\ {\isachardoublequoteopen}s\isactrlsub {\isadigit{1}}\ {\isachargreater}{\kern0pt}\ {\isadigit{0}}{\isachardoublequoteclose}\isanewline
\ \ \ \ \isacommand{apply}\isamarkupfalse%
\ {\isacharparenleft}{\kern0pt}simp\ add{\isacharcolon}{\kern0pt}s\isactrlsub {\isadigit{1}}{\isacharunderscore}{\kern0pt}def{\isacharparenright}{\kern0pt}\isanewline
\ \ \ \ \isacommand{apply}\isamarkupfalse%
\ {\isacharparenleft}{\kern0pt}rule\ divide{\isacharunderscore}{\kern0pt}pos{\isacharunderscore}{\kern0pt}pos{\isacharparenright}{\kern0pt}\isanewline
\ \ \ \ \isacommand{apply}\isamarkupfalse%
\ {\isacharparenleft}{\kern0pt}rule\ mult{\isacharunderscore}{\kern0pt}pos{\isacharunderscore}{\kern0pt}pos{\isacharparenright}{\kern0pt}\isanewline
\ \ \ \ \isacommand{using}\isamarkupfalse%
\ assms\ \isacommand{apply}\isamarkupfalse%
\ linarith\isanewline
\ \ \ \ \isacommand{apply}\isamarkupfalse%
\ {\isacharparenleft}{\kern0pt}simp\ add{\isacharcolon}{\kern0pt}n{\isacharunderscore}{\kern0pt}nonzero{\isacharparenright}{\kern0pt}\isanewline
\ \ \ \ \isacommand{by}\isamarkupfalse%
\ {\isacharparenleft}{\kern0pt}meson\ assms\ zero{\isacharunderscore}{\kern0pt}less{\isacharunderscore}{\kern0pt}of{\isacharunderscore}{\kern0pt}rat{\isacharunderscore}{\kern0pt}iff\ zero{\isacharunderscore}{\kern0pt}less{\isacharunderscore}{\kern0pt}power{\isacharparenright}{\kern0pt}\isanewline
\ \ \isacommand{have}\isamarkupfalse%
\ s{\isadigit{2}}{\isacharunderscore}{\kern0pt}nonzero{\isacharcolon}{\kern0pt}\ {\isachardoublequoteopen}s\isactrlsub {\isadigit{2}}\ {\isachargreater}{\kern0pt}\ {\isadigit{0}}{\isachardoublequoteclose}\ \isacommand{using}\isamarkupfalse%
\ assms\ \isacommand{by}\isamarkupfalse%
\ {\isacharparenleft}{\kern0pt}simp\ add{\isacharcolon}{\kern0pt}s\isactrlsub {\isadigit{2}}{\isacharunderscore}{\kern0pt}def{\isacharparenright}{\kern0pt}\ \isanewline
\ \ \isacommand{have}\isamarkupfalse%
\ real{\isacharunderscore}{\kern0pt}of{\isacharunderscore}{\kern0pt}rat{\isacharunderscore}{\kern0pt}f{\isacharcolon}{\kern0pt}\ {\isachardoublequoteopen}{\isasymAnd}x{\isachardot}{\kern0pt}\ f{\isacharprime}{\kern0pt}\ x\ {\isacharequal}{\kern0pt}\ real{\isacharunderscore}{\kern0pt}of{\isacharunderscore}{\kern0pt}rat\ {\isacharparenleft}{\kern0pt}f\ x{\isacharparenright}{\kern0pt}{\isachardoublequoteclose}\isanewline
\ \ \ \ \isacommand{using}\isamarkupfalse%
\ s{\isadigit{2}}{\isacharunderscore}{\kern0pt}nonzero\ \isacommand{apply}\isamarkupfalse%
\ {\isacharparenleft}{\kern0pt}simp\ add{\isacharcolon}{\kern0pt}f{\isacharunderscore}{\kern0pt}def\ f{\isacharprime}{\kern0pt}{\isacharunderscore}{\kern0pt}def\ f{\isadigit{1}}{\isacharunderscore}{\kern0pt}def\ f{\isadigit{2}}{\isacharunderscore}{\kern0pt}def\ median{\isacharunderscore}{\kern0pt}rat\ median{\isacharunderscore}{\kern0pt}restrict{\isacharparenright}{\kern0pt}\isanewline
\ \ \ \ \isacommand{apply}\isamarkupfalse%
\ {\isacharparenleft}{\kern0pt}rule\ arg{\isacharunderscore}{\kern0pt}cong{\isadigit{2}}{\isacharbrackleft}{\kern0pt}\isakeyword{where}\ f{\isacharequal}{\kern0pt}{\isachardoublequoteopen}median{\isachardoublequoteclose}{\isacharbrackright}{\kern0pt}{\isacharcomma}{\kern0pt}\ simp{\isacharparenright}{\kern0pt}\isanewline
\ \ \ \ \isacommand{by}\isamarkupfalse%
\ {\isacharparenleft}{\kern0pt}simp\ add{\isacharcolon}{\kern0pt}of{\isacharunderscore}{\kern0pt}rat{\isacharunderscore}{\kern0pt}divide\ of{\isacharunderscore}{\kern0pt}rat{\isacharunderscore}{\kern0pt}sum\ of{\isacharunderscore}{\kern0pt}rat{\isacharunderscore}{\kern0pt}mult{\isacharparenright}{\kern0pt}\isanewline
\isanewline
\ \ \isacommand{define}\isamarkupfalse%
\ {\isasymOmega}\ \isakeyword{where}\ {\isachardoublequoteopen}{\isasymOmega}\ {\isacharequal}{\kern0pt}\ pmf{\isacharunderscore}{\kern0pt}of{\isacharunderscore}{\kern0pt}set\ {\isacharbraceleft}{\kern0pt}{\isacharparenleft}{\kern0pt}u{\isacharcomma}{\kern0pt}\ v{\isacharparenright}{\kern0pt}{\isachardot}{\kern0pt}\ v\ {\isacharless}{\kern0pt}\ count{\isacharunderscore}{\kern0pt}list\ as\ u{\isacharbraceright}{\kern0pt}{\isachardoublequoteclose}\isanewline
\ \ \isacommand{have}\isamarkupfalse%
\ fin{\isacharunderscore}{\kern0pt}omega{\isacharcolon}{\kern0pt}\ {\isachardoublequoteopen}finite\ {\isacharparenleft}{\kern0pt}set{\isacharunderscore}{\kern0pt}pmf\ {\isasymOmega}{\isacharparenright}{\kern0pt}{\isachardoublequoteclose}\isanewline
\ \ \ \ \isacommand{apply}\isamarkupfalse%
\ {\isacharparenleft}{\kern0pt}subst\ {\isasymOmega}{\isacharunderscore}{\kern0pt}def{\isacharcomma}{\kern0pt}\ subst\ set{\isacharunderscore}{\kern0pt}pmf{\isacharunderscore}{\kern0pt}of{\isacharunderscore}{\kern0pt}set{\isacharparenright}{\kern0pt}\isanewline
\ \ \ \ \isacommand{using}\isamarkupfalse%
\ assms{\isacharparenleft}{\kern0pt}{\isadigit{5}}{\isacharparenright}{\kern0pt}\ fin{\isacharunderscore}{\kern0pt}space\ non{\isacharunderscore}{\kern0pt}empty{\isacharunderscore}{\kern0pt}space\ False\ \isacommand{by}\isamarkupfalse%
\ auto\isanewline
\ \ \isacommand{have}\isamarkupfalse%
\ fin{\isacharunderscore}{\kern0pt}omega{\isacharunderscore}{\kern0pt}{\isadigit{2}}{\isacharcolon}{\kern0pt}\ {\isachardoublequoteopen}finite\ {\isacharparenleft}{\kern0pt}set{\isacharunderscore}{\kern0pt}pmf\ {\isacharparenleft}{\kern0pt}{\isacharparenleft}{\kern0pt}prod{\isacharunderscore}{\kern0pt}pmf\ {\isacharparenleft}{\kern0pt}{\isacharbraceleft}{\kern0pt}{\isadigit{0}}{\isachardot}{\kern0pt}{\isachardot}{\kern0pt}{\isacharless}{\kern0pt}s\isactrlsub {\isadigit{1}}{\isacharbraceright}{\kern0pt}\ {\isasymtimes}\ {\isacharbraceleft}{\kern0pt}{\isadigit{0}}{\isachardot}{\kern0pt}{\isachardot}{\kern0pt}{\isacharless}{\kern0pt}s\isactrlsub {\isadigit{2}}{\isacharbraceright}{\kern0pt}{\isacharparenright}{\kern0pt}\ {\isacharparenleft}{\kern0pt}{\isasymlambda}{\isacharunderscore}{\kern0pt}{\isachardot}{\kern0pt}\ {\isasymOmega}{\isacharparenright}{\kern0pt}{\isacharparenright}{\kern0pt}{\isacharparenright}{\kern0pt}{\isacharparenright}{\kern0pt}{\isachardoublequoteclose}\isanewline
\ \ \ \ \isacommand{apply}\isamarkupfalse%
\ {\isacharparenleft}{\kern0pt}subst\ set{\isacharunderscore}{\kern0pt}prod{\isacharunderscore}{\kern0pt}pmf{\isacharcomma}{\kern0pt}\ simp{\isacharparenright}{\kern0pt}\isanewline
\ \ \ \ \isacommand{apply}\isamarkupfalse%
\ {\isacharparenleft}{\kern0pt}rule\ finite{\isacharunderscore}{\kern0pt}PiE{\isacharcomma}{\kern0pt}\ simp{\isacharparenright}{\kern0pt}\isanewline
\ \ \ \ \isacommand{by}\isamarkupfalse%
\ {\isacharparenleft}{\kern0pt}simp\ add{\isacharcolon}{\kern0pt}fin{\isacharunderscore}{\kern0pt}omega{\isacharparenright}{\kern0pt}\isanewline
\isanewline
\ \ \isacommand{have}\isamarkupfalse%
\ a{\isacharcolon}{\kern0pt}{\isachardoublequoteopen}fold\ {\isacharparenleft}{\kern0pt}{\isasymlambda}x\ state{\isachardot}{\kern0pt}\ state\ {\isasymbind}\ fk{\isacharunderscore}{\kern0pt}update\ x{\isacharparenright}{\kern0pt}\ as\ {\isacharparenleft}{\kern0pt}fk{\isacharunderscore}{\kern0pt}init\ k\ {\isasymdelta}\ {\isasymepsilon}\ n{\isacharparenright}{\kern0pt}\ {\isacharequal}{\kern0pt}\ map{\isacharunderscore}{\kern0pt}pmf\ {\isacharparenleft}{\kern0pt}{\isasymlambda}x{\isachardot}{\kern0pt}\ {\isacharparenleft}{\kern0pt}s\isactrlsub {\isadigit{1}}{\isacharcomma}{\kern0pt}s\isactrlsub {\isadigit{2}}{\isacharcomma}{\kern0pt}k{\isacharcomma}{\kern0pt}length\ as{\isacharcomma}{\kern0pt}\ x{\isacharparenright}{\kern0pt}{\isacharparenright}{\kern0pt}\ \isanewline
\ \ \ \ {\isacharparenleft}{\kern0pt}prod{\isacharunderscore}{\kern0pt}pmf\ {\isacharparenleft}{\kern0pt}{\isacharbraceleft}{\kern0pt}{\isadigit{0}}{\isachardot}{\kern0pt}{\isachardot}{\kern0pt}{\isacharless}{\kern0pt}s\isactrlsub {\isadigit{1}}{\isacharbraceright}{\kern0pt}\ {\isasymtimes}\ {\isacharbraceleft}{\kern0pt}{\isadigit{0}}{\isachardot}{\kern0pt}{\isachardot}{\kern0pt}{\isacharless}{\kern0pt}s\isactrlsub {\isadigit{2}}{\isacharbraceright}{\kern0pt}{\isacharparenright}{\kern0pt}\ {\isacharparenleft}{\kern0pt}{\isasymlambda}{\isacharunderscore}{\kern0pt}{\isachardot}{\kern0pt}\ pmf{\isacharunderscore}{\kern0pt}of{\isacharunderscore}{\kern0pt}set\ {\isacharbraceleft}{\kern0pt}{\isacharparenleft}{\kern0pt}u{\isacharcomma}{\kern0pt}v{\isacharparenright}{\kern0pt}{\isachardot}{\kern0pt}\ v\ {\isacharless}{\kern0pt}\ count{\isacharunderscore}{\kern0pt}list\ as\ u{\isacharbraceright}{\kern0pt}{\isacharparenright}{\kern0pt}{\isacharparenright}{\kern0pt}{\isachardoublequoteclose}\isanewline
\ \ \ \ \isacommand{apply}\isamarkupfalse%
\ {\isacharparenleft}{\kern0pt}subst\ fk{\isacharunderscore}{\kern0pt}alg{\isacharunderscore}{\kern0pt}sketch{\isacharbrackleft}{\kern0pt}OF\ assms{\isacharparenleft}{\kern0pt}{\isadigit{1}}{\isacharparenright}{\kern0pt}\ assms{\isacharparenleft}{\kern0pt}{\isadigit{3}}{\isacharparenright}{\kern0pt}\ assms{\isacharparenleft}{\kern0pt}{\isadigit{4}}{\isacharparenright}{\kern0pt}\ False{\isacharbrackright}{\kern0pt}{\isacharparenright}{\kern0pt}\isanewline
\ \ \ \ \isacommand{by}\isamarkupfalse%
\ {\isacharparenleft}{\kern0pt}simp\ add{\isacharcolon}{\kern0pt}s\isactrlsub {\isadigit{1}}{\isacharunderscore}{\kern0pt}def{\isacharbrackleft}{\kern0pt}symmetric{\isacharbrackright}{\kern0pt}\ s\isactrlsub {\isadigit{2}}{\isacharunderscore}{\kern0pt}def{\isacharbrackleft}{\kern0pt}symmetric{\isacharbrackright}{\kern0pt}{\isacharparenright}{\kern0pt}\isanewline
\isanewline
\ \ \isacommand{have}\isamarkupfalse%
\ fk{\isacharunderscore}{\kern0pt}result{\isacharunderscore}{\kern0pt}exp{\isacharcolon}{\kern0pt}\ {\isachardoublequoteopen}fk{\isacharunderscore}{\kern0pt}result\ {\isacharequal}{\kern0pt}\ {\isacharparenleft}{\kern0pt}{\isasymlambda}{\isacharparenleft}{\kern0pt}x{\isacharcomma}{\kern0pt}y{\isacharcomma}{\kern0pt}z{\isacharcomma}{\kern0pt}u{\isacharcomma}{\kern0pt}v{\isacharparenright}{\kern0pt}{\isachardot}{\kern0pt}\ fk{\isacharunderscore}{\kern0pt}result\ {\isacharparenleft}{\kern0pt}x{\isacharcomma}{\kern0pt}y{\isacharcomma}{\kern0pt}z{\isacharcomma}{\kern0pt}u{\isacharcomma}{\kern0pt}v{\isacharparenright}{\kern0pt}{\isacharparenright}{\kern0pt}{\isachardoublequoteclose}\ \isanewline
\ \ \ \ \isacommand{by}\isamarkupfalse%
\ {\isacharparenleft}{\kern0pt}rule\ ext{\isacharcomma}{\kern0pt}\ fastforce{\isacharparenright}{\kern0pt}\ \isanewline
\isanewline
\ \ \isacommand{have}\isamarkupfalse%
\ b{\isacharcolon}{\kern0pt}{\isachardoublequoteopen}M\ {\isacharequal}{\kern0pt}\ prod{\isacharunderscore}{\kern0pt}pmf\ {\isacharparenleft}{\kern0pt}{\isacharbraceleft}{\kern0pt}{\isadigit{0}}{\isachardot}{\kern0pt}{\isachardot}{\kern0pt}{\isacharless}{\kern0pt}s\isactrlsub {\isadigit{1}}{\isacharbraceright}{\kern0pt}\ {\isasymtimes}\ {\isacharbraceleft}{\kern0pt}{\isadigit{0}}{\isachardot}{\kern0pt}{\isachardot}{\kern0pt}{\isacharless}{\kern0pt}s\isactrlsub {\isadigit{2}}{\isacharbraceright}{\kern0pt}{\isacharparenright}{\kern0pt}\ {\isacharparenleft}{\kern0pt}{\isasymlambda}{\isacharunderscore}{\kern0pt}{\isachardot}{\kern0pt}\ {\isasymOmega}{\isacharparenright}{\kern0pt}\ {\isasymbind}\ return{\isacharunderscore}{\kern0pt}pmf\ {\isasymcirc}\ f{\isachardoublequoteclose}\isanewline
\ \ \ \ \isacommand{apply}\isamarkupfalse%
\ {\isacharparenleft}{\kern0pt}subst\ M{\isacharunderscore}{\kern0pt}def{\isacharparenright}{\kern0pt}\isanewline
\ \ \ \ \isacommand{apply}\isamarkupfalse%
\ {\isacharparenleft}{\kern0pt}subst\ a{\isacharparenright}{\kern0pt}\isanewline
\ \ \ \ \isacommand{apply}\isamarkupfalse%
\ {\isacharparenleft}{\kern0pt}subst\ fk{\isacharunderscore}{\kern0pt}result{\isacharunderscore}{\kern0pt}exp{\isacharcomma}{\kern0pt}\ simp{\isacharparenright}{\kern0pt}\isanewline
\ \ \ \ \isacommand{apply}\isamarkupfalse%
\ {\isacharparenleft}{\kern0pt}simp\ add{\isacharcolon}{\kern0pt}map{\isacharunderscore}{\kern0pt}pmf{\isacharunderscore}{\kern0pt}def{\isacharparenright}{\kern0pt}\isanewline
\ \ \ \ \isacommand{apply}\isamarkupfalse%
\ {\isacharparenleft}{\kern0pt}subst\ bind{\isacharunderscore}{\kern0pt}assoc{\isacharunderscore}{\kern0pt}pmf{\isacharparenright}{\kern0pt}\isanewline
\ \ \ \ \isacommand{apply}\isamarkupfalse%
\ {\isacharparenleft}{\kern0pt}subst\ bind{\isacharunderscore}{\kern0pt}return{\isacharunderscore}{\kern0pt}pmf{\isacharparenright}{\kern0pt}\isanewline
\ \ \ \ \isacommand{by}\isamarkupfalse%
\ {\isacharparenleft}{\kern0pt}simp\ add{\isacharcolon}{\kern0pt}f{\isacharunderscore}{\kern0pt}def\ comp{\isacharunderscore}{\kern0pt}def\ {\isasymOmega}{\isacharunderscore}{\kern0pt}def{\isacharparenright}{\kern0pt}\isanewline
\isanewline
\ \ \isacommand{have}\isamarkupfalse%
\ c{\isacharcolon}{\kern0pt}\ {\isachardoublequoteopen}{\isacharbraceleft}{\kern0pt}y{\isachardot}{\kern0pt}\ real{\isacharunderscore}{\kern0pt}of{\isacharunderscore}{\kern0pt}rat\ {\isacharparenleft}{\kern0pt}{\isasymdelta}\ {\isacharasterisk}{\kern0pt}\ F\ k\ as{\isacharparenright}{\kern0pt}\ {\isasymge}\ {\isasymbar}f{\isacharprime}{\kern0pt}\ y\ {\isacharminus}{\kern0pt}\ real{\isacharunderscore}{\kern0pt}of{\isacharunderscore}{\kern0pt}rat\ {\isacharparenleft}{\kern0pt}F\ k\ as{\isacharparenright}{\kern0pt}{\isasymbar}{\isacharbraceright}{\kern0pt}\ {\isacharequal}{\kern0pt}\ \isanewline
\ \ \ \ {\isacharbraceleft}{\kern0pt}y{\isachardot}{\kern0pt}\ {\isacharparenleft}{\kern0pt}{\isasymdelta}\ {\isacharasterisk}{\kern0pt}\ F\ k\ as{\isacharparenright}{\kern0pt}\ {\isasymge}\ {\isasymbar}f\ y\ {\isacharminus}{\kern0pt}\ {\isacharparenleft}{\kern0pt}F\ k\ as{\isacharparenright}{\kern0pt}{\isasymbar}{\isacharbraceright}{\kern0pt}{\isachardoublequoteclose}\isanewline
\ \ \ \ \isacommand{apply}\isamarkupfalse%
\ {\isacharparenleft}{\kern0pt}simp\ add{\isacharcolon}{\kern0pt}real{\isacharunderscore}{\kern0pt}of{\isacharunderscore}{\kern0pt}rat{\isacharunderscore}{\kern0pt}f{\isacharparenright}{\kern0pt}\isanewline
\ \ \ \ \isacommand{by}\isamarkupfalse%
\ {\isacharparenleft}{\kern0pt}metis\ abs{\isacharunderscore}{\kern0pt}of{\isacharunderscore}{\kern0pt}rat\ of{\isacharunderscore}{\kern0pt}rat{\isacharunderscore}{\kern0pt}diff\ of{\isacharunderscore}{\kern0pt}rat{\isacharunderscore}{\kern0pt}less{\isacharunderscore}{\kern0pt}eq{\isacharparenright}{\kern0pt}\isanewline
\isanewline
\ \ \isacommand{have}\isamarkupfalse%
\ f{\isadigit{2}}{\isacharunderscore}{\kern0pt}exp{\isacharcolon}{\kern0pt}\ {\isachardoublequoteopen}{\isasymAnd}i\isactrlsub {\isadigit{1}}\ i\isactrlsub {\isadigit{2}}{\isachardot}{\kern0pt}\ i\isactrlsub {\isadigit{1}}\ {\isacharless}{\kern0pt}\ s\isactrlsub {\isadigit{1}}\ {\isasymLongrightarrow}\ i\isactrlsub {\isadigit{2}}\ {\isacharless}{\kern0pt}\ s\isactrlsub {\isadigit{2}}\ {\isasymLongrightarrow}\ \isanewline
\ \ \ \ has{\isacharunderscore}{\kern0pt}bochner{\isacharunderscore}{\kern0pt}integral\ {\isacharparenleft}{\kern0pt}measure{\isacharunderscore}{\kern0pt}pmf\ {\isacharparenleft}{\kern0pt}prod{\isacharunderscore}{\kern0pt}pmf\ {\isacharparenleft}{\kern0pt}{\isacharbraceleft}{\kern0pt}{\isadigit{0}}{\isachardot}{\kern0pt}{\isachardot}{\kern0pt}{\isacharless}{\kern0pt}s\isactrlsub {\isadigit{1}}{\isacharbraceright}{\kern0pt}\ {\isasymtimes}\ {\isacharbraceleft}{\kern0pt}{\isadigit{0}}{\isachardot}{\kern0pt}{\isachardot}{\kern0pt}{\isacharless}{\kern0pt}s\isactrlsub {\isadigit{2}}{\isacharbraceright}{\kern0pt}{\isacharparenright}{\kern0pt}\ {\isacharparenleft}{\kern0pt}{\isasymlambda}{\isacharunderscore}{\kern0pt}{\isachardot}{\kern0pt}\ {\isasymOmega}{\isacharparenright}{\kern0pt}{\isacharparenright}{\kern0pt}{\isacharparenright}{\kern0pt}\ {\isacharparenleft}{\kern0pt}{\isasymlambda}x{\isachardot}{\kern0pt}\ f{\isadigit{2}}\ x\ i\isactrlsub {\isadigit{1}}\ i\isactrlsub {\isadigit{2}}{\isacharparenright}{\kern0pt}\isanewline
\ \ \ \ \ \ \ \ \ \ \ \ {\isacharparenleft}{\kern0pt}real{\isacharunderscore}{\kern0pt}of{\isacharunderscore}{\kern0pt}rat\ {\isacharparenleft}{\kern0pt}F\ k\ as{\isacharparenright}{\kern0pt}{\isacharparenright}{\kern0pt}{\isachardoublequoteclose}\isanewline
\ \ \ \ \isacommand{apply}\isamarkupfalse%
\ {\isacharparenleft}{\kern0pt}simp\ add{\isacharcolon}{\kern0pt}f{\isadigit{2}}{\isacharunderscore}{\kern0pt}def\ {\isasymOmega}{\isacharunderscore}{\kern0pt}def\ of{\isacharunderscore}{\kern0pt}rat{\isacharunderscore}{\kern0pt}mult\ \ of{\isacharunderscore}{\kern0pt}rat{\isacharunderscore}{\kern0pt}sum\ of{\isacharunderscore}{\kern0pt}rat{\isacharunderscore}{\kern0pt}power{\isacharparenright}{\kern0pt}\isanewline
\ \ \ \ \isacommand{apply}\isamarkupfalse%
\ {\isacharparenleft}{\kern0pt}rule\ has{\isacharunderscore}{\kern0pt}bochner{\isacharunderscore}{\kern0pt}integral{\isacharunderscore}{\kern0pt}prod{\isacharunderscore}{\kern0pt}pmf{\isacharunderscore}{\kern0pt}sliceI{\isacharcomma}{\kern0pt}\ simp{\isacharcomma}{\kern0pt}\ simp{\isacharparenright}{\kern0pt}\isanewline
\ \ \ \ \isacommand{by}\isamarkupfalse%
\ {\isacharparenleft}{\kern0pt}rule\ fk{\isacharunderscore}{\kern0pt}alg{\isacharunderscore}{\kern0pt}core{\isacharunderscore}{\kern0pt}exp{\isacharcomma}{\kern0pt}\ metis\ False{\isacharcomma}{\kern0pt}\ metis\ assms{\isacharparenleft}{\kern0pt}{\isadigit{1}}{\isacharparenright}{\kern0pt}{\isacharparenright}{\kern0pt}\isanewline
\isanewline
\ \ \isacommand{have}\isamarkupfalse%
\ {\isachardoublequoteopen}{\isadigit{3}}\ {\isacharasterisk}{\kern0pt}\ real\ k\ {\isacharasterisk}{\kern0pt}\ real\ n\ powr\ {\isacharparenleft}{\kern0pt}{\isadigit{1}}\ {\isacharminus}{\kern0pt}\ {\isadigit{1}}\ {\isacharslash}{\kern0pt}\ real\ k{\isacharparenright}{\kern0pt}\ {\isacharequal}{\kern0pt}\ {\isacharparenleft}{\kern0pt}real{\isacharunderscore}{\kern0pt}of{\isacharunderscore}{\kern0pt}rat\ {\isasymdelta}{\isacharparenright}{\kern0pt}\isactrlsup {\isadigit{2}}\ {\isacharasterisk}{\kern0pt}\ {\isacharparenleft}{\kern0pt}{\isadigit{3}}\ {\isacharasterisk}{\kern0pt}\ real\ k\ {\isacharasterisk}{\kern0pt}\ real\ n\ powr\ {\isacharparenleft}{\kern0pt}{\isadigit{1}}\ {\isacharminus}{\kern0pt}\ {\isadigit{1}}\ {\isacharslash}{\kern0pt}\ real\ k{\isacharparenright}{\kern0pt}\ {\isacharslash}{\kern0pt}\ {\isacharparenleft}{\kern0pt}real{\isacharunderscore}{\kern0pt}of{\isacharunderscore}{\kern0pt}rat\ {\isasymdelta}{\isacharparenright}{\kern0pt}\isactrlsup {\isadigit{2}}{\isacharparenright}{\kern0pt}{\isachardoublequoteclose}\isanewline
\ \ \ \ \isacommand{using}\isamarkupfalse%
\ assms\ \isacommand{by}\isamarkupfalse%
\ simp\isanewline
\ \ \isacommand{also}\isamarkupfalse%
\ \isacommand{have}\isamarkupfalse%
\ {\isachardoublequoteopen}{\isachardot}{\kern0pt}{\isachardot}{\kern0pt}{\isachardot}{\kern0pt}\ {\isasymle}\ \ {\isacharparenleft}{\kern0pt}real{\isacharunderscore}{\kern0pt}of{\isacharunderscore}{\kern0pt}rat\ {\isasymdelta}{\isacharparenright}{\kern0pt}\isactrlsup {\isadigit{2}}\ {\isacharasterisk}{\kern0pt}\ {\isacharparenleft}{\kern0pt}real\ s\isactrlsub {\isadigit{1}}{\isacharparenright}{\kern0pt}{\isachardoublequoteclose}\isanewline
\ \ \ \ \isacommand{apply}\isamarkupfalse%
\ {\isacharparenleft}{\kern0pt}rule\ mult{\isacharunderscore}{\kern0pt}mono{\isacharcomma}{\kern0pt}\ simp{\isacharparenright}{\kern0pt}\isanewline
\ \ \ \ \isacommand{apply}\isamarkupfalse%
\ {\isacharparenleft}{\kern0pt}simp\ add{\isacharcolon}{\kern0pt}s\isactrlsub {\isadigit{1}}{\isacharunderscore}{\kern0pt}def{\isacharparenright}{\kern0pt}\ \isanewline
\ \ \ \ \ \ \isacommand{apply}\isamarkupfalse%
\ {\isacharparenleft}{\kern0pt}meson\ of{\isacharunderscore}{\kern0pt}nat{\isacharunderscore}{\kern0pt}ceiling{\isacharparenright}{\kern0pt}\isanewline
\ \ \ \ \isacommand{using}\isamarkupfalse%
\ assms\ \isacommand{apply}\isamarkupfalse%
\ simp\isanewline
\ \ \ \ \isacommand{by}\isamarkupfalse%
\ simp\isanewline
\ \ \isacommand{finally}\isamarkupfalse%
\ \isacommand{have}\isamarkupfalse%
\ f{\isadigit{2}}{\isacharunderscore}{\kern0pt}var{\isacharunderscore}{\kern0pt}{\isadigit{2}}{\isacharcolon}{\kern0pt}\ {\isachardoublequoteopen}{\isadigit{3}}\ {\isacharasterisk}{\kern0pt}\ real\ k\ {\isacharasterisk}{\kern0pt}\ real\ n\ powr\ {\isacharparenleft}{\kern0pt}{\isadigit{1}}\ {\isacharminus}{\kern0pt}\ {\isadigit{1}}\ {\isacharslash}{\kern0pt}\ real\ k{\isacharparenright}{\kern0pt}\ {\isasymle}\ {\isacharparenleft}{\kern0pt}real{\isacharunderscore}{\kern0pt}of{\isacharunderscore}{\kern0pt}rat\ {\isasymdelta}{\isacharparenright}{\kern0pt}\isactrlsup {\isadigit{2}}\ {\isacharasterisk}{\kern0pt}\ {\isacharparenleft}{\kern0pt}real\ s\isactrlsub {\isadigit{1}}{\isacharparenright}{\kern0pt}{\isachardoublequoteclose}\isanewline
\ \ \ \ \isacommand{by}\isamarkupfalse%
\ blast\isanewline
\ \ \isacommand{have}\isamarkupfalse%
\ {\isachardoublequoteopen}{\isacharparenleft}{\kern0pt}real{\isacharunderscore}{\kern0pt}of{\isacharunderscore}{\kern0pt}rat\ {\isacharparenleft}{\kern0pt}F\ k\ as{\isacharparenright}{\kern0pt}{\isacharparenright}{\kern0pt}\isactrlsup {\isadigit{2}}\ {\isacharasterisk}{\kern0pt}\ real\ k\ {\isacharasterisk}{\kern0pt}\ real\ n\ powr\ {\isacharparenleft}{\kern0pt}{\isadigit{1}}\ {\isacharminus}{\kern0pt}\ {\isadigit{1}}\ {\isacharslash}{\kern0pt}\ real\ k{\isacharparenright}{\kern0pt}\ {\isacharequal}{\kern0pt}\isanewline
\ \ \ \ {\isacharparenleft}{\kern0pt}real{\isacharunderscore}{\kern0pt}of{\isacharunderscore}{\kern0pt}rat\ {\isacharparenleft}{\kern0pt}F\ k\ as{\isacharparenright}{\kern0pt}{\isacharparenright}{\kern0pt}\isactrlsup {\isadigit{2}}\ {\isacharasterisk}{\kern0pt}\ {\isacharparenleft}{\kern0pt}real\ k\ {\isacharasterisk}{\kern0pt}\ real\ n\ powr\ {\isacharparenleft}{\kern0pt}{\isadigit{1}}\ {\isacharminus}{\kern0pt}\ {\isadigit{1}}\ {\isacharslash}{\kern0pt}\ real\ k{\isacharparenright}{\kern0pt}{\isacharparenright}{\kern0pt}{\isachardoublequoteclose}\isanewline
\ \ \ \ \isacommand{by}\isamarkupfalse%
\ {\isacharparenleft}{\kern0pt}simp\ add{\isacharcolon}{\kern0pt}ac{\isacharunderscore}{\kern0pt}simps{\isacharparenright}{\kern0pt}\isanewline
\ \ \isacommand{also}\isamarkupfalse%
\ \isacommand{have}\isamarkupfalse%
\ {\isachardoublequoteopen}{\isachardot}{\kern0pt}{\isachardot}{\kern0pt}{\isachardot}{\kern0pt}\ {\isasymle}\ {\isacharparenleft}{\kern0pt}real{\isacharunderscore}{\kern0pt}of{\isacharunderscore}{\kern0pt}rat\ {\isacharparenleft}{\kern0pt}F\ k\ as\ {\isacharasterisk}{\kern0pt}\ {\isasymdelta}{\isacharparenright}{\kern0pt}{\isacharparenright}{\kern0pt}\isactrlsup {\isadigit{2}}\ {\isacharasterisk}{\kern0pt}\ {\isacharparenleft}{\kern0pt}real\ s\isactrlsub {\isadigit{1}}\ {\isacharslash}{\kern0pt}\ {\isadigit{3}}{\isacharparenright}{\kern0pt}{\isachardoublequoteclose}\isanewline
\ \ \ \ \isacommand{apply}\isamarkupfalse%
\ {\isacharparenleft}{\kern0pt}subst\ of{\isacharunderscore}{\kern0pt}rat{\isacharunderscore}{\kern0pt}mult{\isacharcomma}{\kern0pt}\ subst\ power{\isacharunderscore}{\kern0pt}mult{\isacharunderscore}{\kern0pt}distrib{\isacharparenright}{\kern0pt}\ \isanewline
\ \ \ \ \isacommand{apply}\isamarkupfalse%
\ {\isacharparenleft}{\kern0pt}subst\ mult{\isachardot}{\kern0pt}assoc{\isacharbrackleft}{\kern0pt}\isakeyword{where}\ c{\isacharequal}{\kern0pt}{\isachardoublequoteopen}real\ s\isactrlsub {\isadigit{1}}\ {\isacharslash}{\kern0pt}\ {\isadigit{3}}{\isachardoublequoteclose}{\isacharbrackright}{\kern0pt}{\isacharparenright}{\kern0pt}\isanewline
\ \ \ \ \isacommand{apply}\isamarkupfalse%
\ {\isacharparenleft}{\kern0pt}rule\ mult{\isacharunderscore}{\kern0pt}mono{\isacharcomma}{\kern0pt}\ simp{\isacharparenright}{\kern0pt}\ \isacommand{using}\isamarkupfalse%
\ f{\isadigit{2}}{\isacharunderscore}{\kern0pt}var{\isacharunderscore}{\kern0pt}{\isadigit{2}}\ \isanewline
\ \ \ \ \isacommand{by}\isamarkupfalse%
\ {\isacharparenleft}{\kern0pt}simp{\isacharplus}{\kern0pt}{\isacharparenright}{\kern0pt}\isanewline
\ \ \isacommand{finally}\isamarkupfalse%
\ \isacommand{have}\isamarkupfalse%
\ f{\isadigit{2}}{\isacharunderscore}{\kern0pt}var{\isacharunderscore}{\kern0pt}{\isadigit{1}}{\isacharcolon}{\kern0pt}\ {\isachardoublequoteopen}{\isacharparenleft}{\kern0pt}real{\isacharunderscore}{\kern0pt}of{\isacharunderscore}{\kern0pt}rat\ {\isacharparenleft}{\kern0pt}F\ k\ as{\isacharparenright}{\kern0pt}{\isacharparenright}{\kern0pt}\isactrlsup {\isadigit{2}}\ {\isacharasterisk}{\kern0pt}\ real\ k\ {\isacharasterisk}{\kern0pt}\ real\ n\ powr\ {\isacharparenleft}{\kern0pt}{\isadigit{1}}\ {\isacharminus}{\kern0pt}\ {\isadigit{1}}\ {\isacharslash}{\kern0pt}\ real\ k{\isacharparenright}{\kern0pt}\ {\isasymle}\ {\isacharparenleft}{\kern0pt}real{\isacharunderscore}{\kern0pt}of{\isacharunderscore}{\kern0pt}rat\ {\isacharparenleft}{\kern0pt}{\isasymdelta}\ {\isacharasterisk}{\kern0pt}\ F\ k\ as{\isacharparenright}{\kern0pt}{\isacharparenright}{\kern0pt}\isactrlsup {\isadigit{2}}\ {\isacharasterisk}{\kern0pt}\ real\ s\isactrlsub {\isadigit{1}}\ {\isacharslash}{\kern0pt}\ {\isadigit{3}}{\isachardoublequoteclose}\isanewline
\ \ \ \ \isacommand{by}\isamarkupfalse%
\ {\isacharparenleft}{\kern0pt}simp\ add{\isacharcolon}{\kern0pt}\ mult{\isachardot}{\kern0pt}commute{\isacharparenright}{\kern0pt}\isanewline
\ \ \isanewline
\ \ \isacommand{have}\isamarkupfalse%
\ f{\isadigit{2}}{\isacharunderscore}{\kern0pt}var{\isacharcolon}{\kern0pt}\ {\isachardoublequoteopen}{\isasymAnd}i\isactrlsub {\isadigit{1}}\ i\isactrlsub {\isadigit{2}}{\isachardot}{\kern0pt}\ i\isactrlsub {\isadigit{1}}\ {\isacharless}{\kern0pt}\ s\isactrlsub {\isadigit{1}}\ {\isasymLongrightarrow}\ i\isactrlsub {\isadigit{2}}\ {\isacharless}{\kern0pt}\ s\isactrlsub {\isadigit{2}}\ {\isasymLongrightarrow}\isanewline
\ \ \ \ \ \ \ prob{\isacharunderscore}{\kern0pt}space{\isachardot}{\kern0pt}variance\ {\isacharparenleft}{\kern0pt}measure{\isacharunderscore}{\kern0pt}pmf\ {\isacharparenleft}{\kern0pt}prod{\isacharunderscore}{\kern0pt}pmf\ {\isacharparenleft}{\kern0pt}{\isacharbraceleft}{\kern0pt}{\isadigit{0}}{\isachardot}{\kern0pt}{\isachardot}{\kern0pt}{\isacharless}{\kern0pt}s\isactrlsub {\isadigit{1}}{\isacharbraceright}{\kern0pt}\ {\isasymtimes}\ {\isacharbraceleft}{\kern0pt}{\isadigit{0}}{\isachardot}{\kern0pt}{\isachardot}{\kern0pt}{\isacharless}{\kern0pt}s\isactrlsub {\isadigit{2}}{\isacharbraceright}{\kern0pt}{\isacharparenright}{\kern0pt}\ {\isacharparenleft}{\kern0pt}{\isasymlambda}{\isacharunderscore}{\kern0pt}{\isachardot}{\kern0pt}\ {\isasymOmega}{\isacharparenright}{\kern0pt}{\isacharparenright}{\kern0pt}{\isacharparenright}{\kern0pt}\ {\isacharparenleft}{\kern0pt}{\isasymlambda}{\isasymomega}{\isachardot}{\kern0pt}\ f{\isadigit{2}}\ {\isasymomega}\ i\isactrlsub {\isadigit{1}}\ i\isactrlsub {\isadigit{2}}{\isacharparenright}{\kern0pt}\isanewline
\ \ \ \ \ \ \ \ \ \ \ {\isasymle}\ {\isacharparenleft}{\kern0pt}real{\isacharunderscore}{\kern0pt}of{\isacharunderscore}{\kern0pt}rat\ {\isacharparenleft}{\kern0pt}{\isasymdelta}\ {\isacharasterisk}{\kern0pt}\ F\ k\ as{\isacharparenright}{\kern0pt}{\isacharparenright}{\kern0pt}\isactrlsup {\isadigit{2}}\ {\isacharasterisk}{\kern0pt}\ real\ s\isactrlsub {\isadigit{1}}\ {\isacharslash}{\kern0pt}\ {\isadigit{3}}{\isachardoublequoteclose}\ \isanewline
\ \ \ \ \isacommand{apply}\isamarkupfalse%
\ {\isacharparenleft}{\kern0pt}simp\ only{\isacharcolon}{\kern0pt}\ f{\isadigit{2}}{\isacharunderscore}{\kern0pt}def{\isacharparenright}{\kern0pt}\isanewline
\ \ \ \ \isacommand{apply}\isamarkupfalse%
\ {\isacharparenleft}{\kern0pt}subst\ variance{\isacharunderscore}{\kern0pt}prod{\isacharunderscore}{\kern0pt}pmf{\isacharunderscore}{\kern0pt}slice{\isacharcomma}{\kern0pt}\ simp{\isacharcomma}{\kern0pt}\ simp{\isacharcomma}{\kern0pt}\ rule\ integrable{\isacharunderscore}{\kern0pt}measure{\isacharunderscore}{\kern0pt}pmf{\isacharunderscore}{\kern0pt}finite{\isacharbrackleft}{\kern0pt}OF\ fin{\isacharunderscore}{\kern0pt}omega{\isacharbrackright}{\kern0pt}{\isacharparenright}{\kern0pt}\isanewline
\ \ \ \ \isacommand{apply}\isamarkupfalse%
\ {\isacharparenleft}{\kern0pt}rule\ order{\isacharunderscore}{\kern0pt}trans\ {\isacharbrackleft}{\kern0pt}\isakeyword{where}\ y{\isacharequal}{\kern0pt}{\isachardoublequoteopen}{\isacharparenleft}{\kern0pt}real{\isacharunderscore}{\kern0pt}of{\isacharunderscore}{\kern0pt}rat\ {\isacharparenleft}{\kern0pt}F\ k\ as{\isacharparenright}{\kern0pt}{\isacharparenright}{\kern0pt}\isactrlsup {\isadigit{2}}\ {\isacharasterisk}{\kern0pt}\isanewline
\ \ \ \ \ \ \ \ \ \ \ \ \ \ \ \ \ real\ k\ {\isacharasterisk}{\kern0pt}\ real\ n\ powr\ {\isacharparenleft}{\kern0pt}{\isadigit{1}}\ {\isacharminus}{\kern0pt}\ {\isadigit{1}}\ {\isacharslash}{\kern0pt}\ real\ k{\isacharparenright}{\kern0pt}{\isachardoublequoteclose}{\isacharbrackright}{\kern0pt}{\isacharparenright}{\kern0pt}\isanewline
\ \ \ \ \isacommand{apply}\isamarkupfalse%
\ {\isacharparenleft}{\kern0pt}simp\ add{\isacharcolon}{\kern0pt}\ {\isasymOmega}{\isacharunderscore}{\kern0pt}def{\isacharparenright}{\kern0pt}\isanewline
\ \ \ \ \isacommand{using}\isamarkupfalse%
\ assms\ False\ fk{\isacharunderscore}{\kern0pt}alg{\isacharunderscore}{\kern0pt}core{\isacharunderscore}{\kern0pt}var{\isacharbrackleft}{\kern0pt}\isakeyword{where}\ k{\isacharequal}{\kern0pt}{\isachardoublequoteopen}k{\isachardoublequoteclose}{\isacharbrackright}{\kern0pt}\ \isacommand{apply}\isamarkupfalse%
\ simp\isanewline
\ \ \ \ \isacommand{using}\isamarkupfalse%
\ f{\isadigit{2}}{\isacharunderscore}{\kern0pt}var{\isacharunderscore}{\kern0pt}{\isadigit{1}}\ \isacommand{by}\isamarkupfalse%
\ blast\isanewline
\isanewline
\ \ \isacommand{have}\isamarkupfalse%
\ f{\isadigit{1}}{\isacharunderscore}{\kern0pt}exp{\isacharunderscore}{\kern0pt}{\isadigit{1}}{\isacharcolon}{\kern0pt}\ {\isachardoublequoteopen}{\isacharparenleft}{\kern0pt}real{\isacharunderscore}{\kern0pt}of{\isacharunderscore}{\kern0pt}rat\ {\isacharparenleft}{\kern0pt}F\ k\ as{\isacharparenright}{\kern0pt}{\isacharparenright}{\kern0pt}\ {\isacharequal}{\kern0pt}\ {\isacharparenleft}{\kern0pt}{\isasymSum}i\ {\isasymin}\ {\isacharbraceleft}{\kern0pt}{\isadigit{0}}{\isachardot}{\kern0pt}{\isachardot}{\kern0pt}{\isacharless}{\kern0pt}s\isactrlsub {\isadigit{1}}{\isacharbraceright}{\kern0pt}{\isachardot}{\kern0pt}\ {\isacharparenleft}{\kern0pt}real{\isacharunderscore}{\kern0pt}of{\isacharunderscore}{\kern0pt}rat\ {\isacharparenleft}{\kern0pt}F\ k\ as{\isacharparenright}{\kern0pt}{\isacharparenright}{\kern0pt}{\isacharslash}{\kern0pt}real\ s\isactrlsub {\isadigit{1}}{\isacharparenright}{\kern0pt}{\isachardoublequoteclose}\isanewline
\ \ \ \ \isacommand{by}\isamarkupfalse%
\ {\isacharparenleft}{\kern0pt}simp\ add{\isacharcolon}{\kern0pt}s{\isadigit{1}}{\isacharunderscore}{\kern0pt}nonzero{\isacharparenright}{\kern0pt}\isanewline
\isanewline
\ \ \isacommand{have}\isamarkupfalse%
\ f{\isadigit{1}}{\isacharunderscore}{\kern0pt}exp{\isacharcolon}{\kern0pt}\ {\isachardoublequoteopen}{\isasymAnd}i{\isachardot}{\kern0pt}\ i\ {\isacharless}{\kern0pt}\ s\isactrlsub {\isadigit{2}}\ {\isasymLongrightarrow}\ \isanewline
\ \ \ \ \ \ has{\isacharunderscore}{\kern0pt}bochner{\isacharunderscore}{\kern0pt}integral\ {\isacharparenleft}{\kern0pt}prod{\isacharunderscore}{\kern0pt}pmf\ {\isacharparenleft}{\kern0pt}{\isacharbraceleft}{\kern0pt}{\isadigit{0}}{\isachardot}{\kern0pt}{\isachardot}{\kern0pt}{\isacharless}{\kern0pt}s\isactrlsub {\isadigit{1}}{\isacharbraceright}{\kern0pt}\ {\isasymtimes}\ {\isacharbraceleft}{\kern0pt}{\isadigit{0}}{\isachardot}{\kern0pt}{\isachardot}{\kern0pt}{\isacharless}{\kern0pt}s\isactrlsub {\isadigit{2}}{\isacharbraceright}{\kern0pt}{\isacharparenright}{\kern0pt}\ {\isacharparenleft}{\kern0pt}{\isasymlambda}{\isacharunderscore}{\kern0pt}{\isachardot}{\kern0pt}\ {\isasymOmega}{\isacharparenright}{\kern0pt}{\isacharparenright}{\kern0pt}\ {\isacharparenleft}{\kern0pt}{\isasymlambda}{\isasymomega}{\isachardot}{\kern0pt}\ f{\isadigit{1}}\ {\isasymomega}\ i{\isacharparenright}{\kern0pt}\ \isanewline
\ \ \ \ {\isacharparenleft}{\kern0pt}real{\isacharunderscore}{\kern0pt}of{\isacharunderscore}{\kern0pt}rat\ {\isacharparenleft}{\kern0pt}F\ k\ as{\isacharparenright}{\kern0pt}{\isacharparenright}{\kern0pt}{\isachardoublequoteclose}\isanewline
\ \ \ \ \isacommand{apply}\isamarkupfalse%
\ {\isacharparenleft}{\kern0pt}simp\ add{\isacharcolon}{\kern0pt}f{\isadigit{1}}{\isacharunderscore}{\kern0pt}def\ sum{\isacharunderscore}{\kern0pt}divide{\isacharunderscore}{\kern0pt}distrib{\isacharparenright}{\kern0pt}\isanewline
\ \ \ \ \isacommand{apply}\isamarkupfalse%
\ {\isacharparenleft}{\kern0pt}subst\ f{\isadigit{1}}{\isacharunderscore}{\kern0pt}exp{\isacharunderscore}{\kern0pt}{\isadigit{1}}{\isacharparenright}{\kern0pt}\isanewline
\ \ \ \ \isacommand{apply}\isamarkupfalse%
\ {\isacharparenleft}{\kern0pt}rule\ has{\isacharunderscore}{\kern0pt}bochner{\isacharunderscore}{\kern0pt}integral{\isacharunderscore}{\kern0pt}sum{\isacharparenright}{\kern0pt}\isanewline
\ \ \ \ \isacommand{apply}\isamarkupfalse%
\ {\isacharparenleft}{\kern0pt}rule\ has{\isacharunderscore}{\kern0pt}bochner{\isacharunderscore}{\kern0pt}integral{\isacharunderscore}{\kern0pt}divide{\isacharunderscore}{\kern0pt}zero{\isacharparenright}{\kern0pt}\isanewline
\ \ \ \ \isacommand{by}\isamarkupfalse%
\ {\isacharparenleft}{\kern0pt}simp\ add{\isacharcolon}{\kern0pt}\ f{\isadigit{2}}{\isacharunderscore}{\kern0pt}exp{\isacharparenright}{\kern0pt}\isanewline
\isanewline
\ \ \isacommand{have}\isamarkupfalse%
\ f{\isadigit{1}}{\isacharunderscore}{\kern0pt}var{\isacharcolon}{\kern0pt}\ {\isachardoublequoteopen}{\isasymAnd}i{\isachardot}{\kern0pt}\ i\ {\isacharless}{\kern0pt}\ s\isactrlsub {\isadigit{2}}\ {\isasymLongrightarrow}\ \isanewline
\ \ \ \ \ \ prob{\isacharunderscore}{\kern0pt}space{\isachardot}{\kern0pt}variance\ {\isacharparenleft}{\kern0pt}prod{\isacharunderscore}{\kern0pt}pmf\ {\isacharparenleft}{\kern0pt}{\isacharbraceleft}{\kern0pt}{\isadigit{0}}{\isachardot}{\kern0pt}{\isachardot}{\kern0pt}{\isacharless}{\kern0pt}s\isactrlsub {\isadigit{1}}{\isacharbraceright}{\kern0pt}\ {\isasymtimes}\ {\isacharbraceleft}{\kern0pt}{\isadigit{0}}{\isachardot}{\kern0pt}{\isachardot}{\kern0pt}{\isacharless}{\kern0pt}s\isactrlsub {\isadigit{2}}{\isacharbraceright}{\kern0pt}{\isacharparenright}{\kern0pt}\ {\isacharparenleft}{\kern0pt}{\isasymlambda}{\isacharunderscore}{\kern0pt}{\isachardot}{\kern0pt}\ {\isasymOmega}{\isacharparenright}{\kern0pt}{\isacharparenright}{\kern0pt}\ {\isacharparenleft}{\kern0pt}{\isasymlambda}{\isasymomega}{\isachardot}{\kern0pt}\ f{\isadigit{1}}\ {\isasymomega}\ i{\isacharparenright}{\kern0pt}\ \isanewline
\ \ \ \ \ \ {\isasymle}\ real{\isacharunderscore}{\kern0pt}of{\isacharunderscore}{\kern0pt}rat\ {\isacharparenleft}{\kern0pt}{\isasymdelta}\ {\isacharasterisk}{\kern0pt}\ F\ k\ as{\isacharparenright}{\kern0pt}{\isacharcircum}{\kern0pt}{\isadigit{2}}{\isacharslash}{\kern0pt}{\isadigit{3}}{\isachardoublequoteclose}\ {\isacharparenleft}{\kern0pt}\isakeyword{is}\ {\isachardoublequoteopen}{\isasymAnd}i{\isachardot}{\kern0pt}\ {\isacharunderscore}{\kern0pt}\ {\isasymLongrightarrow}\ {\isacharquery}{\kern0pt}rhs\ i{\isachardoublequoteclose}{\isacharparenright}{\kern0pt}\isanewline
\ \ \isacommand{proof}\isamarkupfalse%
\ {\isacharminus}{\kern0pt}\isanewline
\ \ \ \ \isacommand{fix}\isamarkupfalse%
\ i\isanewline
\ \ \ \ \isacommand{assume}\isamarkupfalse%
\ f{\isadigit{1}}{\isacharunderscore}{\kern0pt}var{\isacharunderscore}{\kern0pt}{\isadigit{1}}{\isacharcolon}{\kern0pt}{\isachardoublequoteopen}i\ {\isacharless}{\kern0pt}\ s\isactrlsub {\isadigit{2}}{\isachardoublequoteclose}\ \isanewline
\ \ \ \ \isacommand{have}\isamarkupfalse%
\ {\isachardoublequoteopen}prob{\isacharunderscore}{\kern0pt}space{\isachardot}{\kern0pt}variance\ {\isacharparenleft}{\kern0pt}prod{\isacharunderscore}{\kern0pt}pmf\ {\isacharparenleft}{\kern0pt}{\isacharbraceleft}{\kern0pt}{\isadigit{0}}{\isachardot}{\kern0pt}{\isachardot}{\kern0pt}{\isacharless}{\kern0pt}s\isactrlsub {\isadigit{1}}{\isacharbraceright}{\kern0pt}\ {\isasymtimes}\ {\isacharbraceleft}{\kern0pt}{\isadigit{0}}{\isachardot}{\kern0pt}{\isachardot}{\kern0pt}{\isacharless}{\kern0pt}s\isactrlsub {\isadigit{2}}{\isacharbraceright}{\kern0pt}{\isacharparenright}{\kern0pt}\ {\isacharparenleft}{\kern0pt}{\isasymlambda}{\isacharunderscore}{\kern0pt}{\isachardot}{\kern0pt}\ {\isasymOmega}{\isacharparenright}{\kern0pt}{\isacharparenright}{\kern0pt}\ {\isacharparenleft}{\kern0pt}{\isasymlambda}{\isasymomega}{\isachardot}{\kern0pt}\ f{\isadigit{1}}\ {\isasymomega}\ i{\isacharparenright}{\kern0pt}\ {\isacharequal}{\kern0pt}\ \isanewline
\ \ \ \ \ \ \ {\isacharparenleft}{\kern0pt}{\isasymSum}j\ {\isacharequal}{\kern0pt}\ {\isadigit{0}}{\isachardot}{\kern0pt}{\isachardot}{\kern0pt}{\isacharless}{\kern0pt}s\isactrlsub {\isadigit{1}}{\isachardot}{\kern0pt}\ prob{\isacharunderscore}{\kern0pt}space{\isachardot}{\kern0pt}variance\ \ {\isacharparenleft}{\kern0pt}prod{\isacharunderscore}{\kern0pt}pmf\ {\isacharparenleft}{\kern0pt}{\isacharbraceleft}{\kern0pt}{\isadigit{0}}{\isachardot}{\kern0pt}{\isachardot}{\kern0pt}{\isacharless}{\kern0pt}s\isactrlsub {\isadigit{1}}{\isacharbraceright}{\kern0pt}\ {\isasymtimes}\ {\isacharbraceleft}{\kern0pt}{\isadigit{0}}{\isachardot}{\kern0pt}{\isachardot}{\kern0pt}{\isacharless}{\kern0pt}s\isactrlsub {\isadigit{2}}{\isacharbraceright}{\kern0pt}{\isacharparenright}{\kern0pt}\ {\isacharparenleft}{\kern0pt}{\isasymlambda}{\isacharunderscore}{\kern0pt}{\isachardot}{\kern0pt}\ {\isasymOmega}{\isacharparenright}{\kern0pt}{\isacharparenright}{\kern0pt}\ \ {\isacharparenleft}{\kern0pt}{\isasymlambda}{\isasymomega}{\isachardot}{\kern0pt}\ f{\isadigit{2}}\ {\isasymomega}\ j\ i\ {\isacharslash}{\kern0pt}\ real\ s\isactrlsub {\isadigit{1}}{\isacharparenright}{\kern0pt}{\isacharparenright}{\kern0pt}{\isachardoublequoteclose}\isanewline
\ \ \ \ \ \ \isacommand{apply}\isamarkupfalse%
\ {\isacharparenleft}{\kern0pt}simp\ add{\isacharcolon}{\kern0pt}f{\isadigit{1}}{\isacharunderscore}{\kern0pt}def\ sum{\isacharunderscore}{\kern0pt}divide{\isacharunderscore}{\kern0pt}distrib{\isacharparenright}{\kern0pt}\isanewline
\ \ \ \ \ \ \isacommand{apply}\isamarkupfalse%
\ {\isacharparenleft}{\kern0pt}subst\ measure{\isacharunderscore}{\kern0pt}pmf{\isachardot}{\kern0pt}var{\isacharunderscore}{\kern0pt}sum{\isacharunderscore}{\kern0pt}all{\isacharunderscore}{\kern0pt}indep{\isacharcomma}{\kern0pt}\ simp{\isacharcomma}{\kern0pt}\ simp{\isacharparenright}{\kern0pt}\isanewline
\ \ \ \ \ \ \ \ \isacommand{apply}\isamarkupfalse%
\ {\isacharparenleft}{\kern0pt}rule\ integrable{\isacharunderscore}{\kern0pt}measure{\isacharunderscore}{\kern0pt}pmf{\isacharunderscore}{\kern0pt}finite{\isacharbrackleft}{\kern0pt}OF\ fin{\isacharunderscore}{\kern0pt}omega{\isacharunderscore}{\kern0pt}{\isadigit{2}}{\isacharbrackright}{\kern0pt}{\isacharparenright}{\kern0pt}\isanewline
\ \ \ \ \ \ \ \isacommand{apply}\isamarkupfalse%
\ {\isacharparenleft}{\kern0pt}rule\ indep{\isacharunderscore}{\kern0pt}vars{\isacharunderscore}{\kern0pt}restrict{\isacharunderscore}{\kern0pt}intro{\isacharbrackleft}{\kern0pt}\isakeyword{where}\ f{\isacharequal}{\kern0pt}{\isachardoublequoteopen}{\isasymlambda}j{\isachardot}{\kern0pt}\ {\isacharbraceleft}{\kern0pt}j{\isacharbraceright}{\kern0pt}\ {\isasymtimes}\ {\isacharbraceleft}{\kern0pt}i{\isacharbraceright}{\kern0pt}{\isachardoublequoteclose}{\isacharbrackright}{\kern0pt}{\isacharparenright}{\kern0pt}\isanewline
\ \ \ \ \ \ \ \ \ \ \ \ \isacommand{apply}\isamarkupfalse%
\ {\isacharparenleft}{\kern0pt}simp\ add{\isacharcolon}{\kern0pt}f{\isadigit{2}}{\isacharunderscore}{\kern0pt}def{\isacharparenright}{\kern0pt}\isanewline
\ \ \ \ \ \ \ \ \ \ \ \isacommand{apply}\isamarkupfalse%
\ {\isacharparenleft}{\kern0pt}simp\ add{\isacharcolon}{\kern0pt}disjoint{\isacharunderscore}{\kern0pt}family{\isacharunderscore}{\kern0pt}on{\isacharunderscore}{\kern0pt}def{\isacharparenright}{\kern0pt}\isanewline
\ \ \ \ \ \ \ \ \ \ \isacommand{apply}\isamarkupfalse%
\ {\isacharparenleft}{\kern0pt}simp\ add{\isacharcolon}{\kern0pt}s{\isadigit{1}}{\isacharunderscore}{\kern0pt}nonzero{\isacharparenright}{\kern0pt}\isanewline
\ \ \ \ \ \ \ \ \ \isacommand{apply}\isamarkupfalse%
\ {\isacharparenleft}{\kern0pt}simp\ add{\isacharcolon}{\kern0pt}f{\isadigit{1}}{\isacharunderscore}{\kern0pt}var{\isacharunderscore}{\kern0pt}{\isadigit{1}}{\isacharparenright}{\kern0pt}\isanewline
\ \ \ \ \ \ \ \ \isacommand{apply}\isamarkupfalse%
\ simp\isanewline
\ \ \ \ \ \ \ \isacommand{apply}\isamarkupfalse%
\ simp\isanewline
\ \ \ \ \ \ \isacommand{by}\isamarkupfalse%
\ simp\isanewline
\ \ \ \ \isacommand{also}\isamarkupfalse%
\ \isacommand{have}\isamarkupfalse%
\ {\isachardoublequoteopen}{\isachardot}{\kern0pt}{\isachardot}{\kern0pt}{\isachardot}{\kern0pt}\ {\isacharequal}{\kern0pt}\ {\isacharparenleft}{\kern0pt}{\isasymSum}j\ {\isacharequal}{\kern0pt}\ {\isadigit{0}}{\isachardot}{\kern0pt}{\isachardot}{\kern0pt}{\isacharless}{\kern0pt}s\isactrlsub {\isadigit{1}}{\isachardot}{\kern0pt}\ prob{\isacharunderscore}{\kern0pt}space{\isachardot}{\kern0pt}variance\ \ {\isacharparenleft}{\kern0pt}prod{\isacharunderscore}{\kern0pt}pmf\ {\isacharparenleft}{\kern0pt}{\isacharbraceleft}{\kern0pt}{\isadigit{0}}{\isachardot}{\kern0pt}{\isachardot}{\kern0pt}{\isacharless}{\kern0pt}s\isactrlsub {\isadigit{1}}{\isacharbraceright}{\kern0pt}\ {\isasymtimes}\ {\isacharbraceleft}{\kern0pt}{\isadigit{0}}{\isachardot}{\kern0pt}{\isachardot}{\kern0pt}{\isacharless}{\kern0pt}s\isactrlsub {\isadigit{2}}{\isacharbraceright}{\kern0pt}{\isacharparenright}{\kern0pt}\ {\isacharparenleft}{\kern0pt}{\isasymlambda}{\isacharunderscore}{\kern0pt}{\isachardot}{\kern0pt}\ {\isasymOmega}{\isacharparenright}{\kern0pt}{\isacharparenright}{\kern0pt}\ \ {\isacharparenleft}{\kern0pt}{\isasymlambda}{\isasymomega}{\isachardot}{\kern0pt}\ f{\isadigit{2}}\ {\isasymomega}\ j\ i{\isacharparenright}{\kern0pt}\ {\isacharslash}{\kern0pt}\ real\ s\isactrlsub {\isadigit{1}}{\isacharcircum}{\kern0pt}{\isadigit{2}}{\isacharparenright}{\kern0pt}{\isachardoublequoteclose}\isanewline
\ \ \ \ \ \ \isacommand{apply}\isamarkupfalse%
\ {\isacharparenleft}{\kern0pt}rule\ sum{\isachardot}{\kern0pt}cong{\isacharcomma}{\kern0pt}\ simp{\isacharparenright}{\kern0pt}\isanewline
\ \ \ \ \ \ \isacommand{apply}\isamarkupfalse%
\ {\isacharparenleft}{\kern0pt}rule\ measure{\isacharunderscore}{\kern0pt}pmf{\isachardot}{\kern0pt}variance{\isacharunderscore}{\kern0pt}divide{\isacharparenright}{\kern0pt}\isanewline
\ \ \ \ \ \ \isacommand{by}\isamarkupfalse%
\ {\isacharparenleft}{\kern0pt}rule\ integrable{\isacharunderscore}{\kern0pt}measure{\isacharunderscore}{\kern0pt}pmf{\isacharunderscore}{\kern0pt}finite{\isacharbrackleft}{\kern0pt}OF\ fin{\isacharunderscore}{\kern0pt}omega{\isacharunderscore}{\kern0pt}{\isadigit{2}}{\isacharbrackright}{\kern0pt}{\isacharparenright}{\kern0pt}\isanewline
\ \ \ \ \isacommand{also}\isamarkupfalse%
\ \isacommand{have}\isamarkupfalse%
\ {\isachardoublequoteopen}{\isachardot}{\kern0pt}{\isachardot}{\kern0pt}{\isachardot}{\kern0pt}\ {\isasymle}\ {\isacharparenleft}{\kern0pt}{\isasymSum}j\ {\isacharequal}{\kern0pt}\ {\isadigit{0}}{\isachardot}{\kern0pt}{\isachardot}{\kern0pt}{\isacharless}{\kern0pt}s\isactrlsub {\isadigit{1}}{\isachardot}{\kern0pt}\ {\isacharparenleft}{\kern0pt}{\isacharparenleft}{\kern0pt}real{\isacharunderscore}{\kern0pt}of{\isacharunderscore}{\kern0pt}rat\ {\isacharparenleft}{\kern0pt}{\isasymdelta}\ {\isacharasterisk}{\kern0pt}\ F\ k\ as{\isacharparenright}{\kern0pt}{\isacharparenright}{\kern0pt}\isactrlsup {\isadigit{2}}\ {\isacharasterisk}{\kern0pt}\ real\ s\isactrlsub {\isadigit{1}}\ {\isacharslash}{\kern0pt}\ {\isadigit{3}}{\isacharparenright}{\kern0pt}\ {\isacharslash}{\kern0pt}\ {\isacharparenleft}{\kern0pt}real\ s\isactrlsub {\isadigit{1}}{\isacharcircum}{\kern0pt}{\isadigit{2}}{\isacharparenright}{\kern0pt}{\isacharparenright}{\kern0pt}{\isachardoublequoteclose}\isanewline
\ \ \ \ \ \ \isacommand{apply}\isamarkupfalse%
\ {\isacharparenleft}{\kern0pt}rule\ sum{\isacharunderscore}{\kern0pt}mono{\isacharparenright}{\kern0pt}\isanewline
\ \ \ \ \ \ \isacommand{apply}\isamarkupfalse%
\ {\isacharparenleft}{\kern0pt}rule\ divide{\isacharunderscore}{\kern0pt}right{\isacharunderscore}{\kern0pt}mono{\isacharparenright}{\kern0pt}\ \isanewline
\ \ \ \ \ \ \ \isacommand{apply}\isamarkupfalse%
\ {\isacharparenleft}{\kern0pt}rule\ f{\isadigit{2}}{\isacharunderscore}{\kern0pt}var{\isacharbrackleft}{\kern0pt}OF\ {\isacharunderscore}{\kern0pt}\ f{\isadigit{1}}{\isacharunderscore}{\kern0pt}var{\isacharunderscore}{\kern0pt}{\isadigit{1}}{\isacharbrackright}{\kern0pt}{\isacharcomma}{\kern0pt}\ simp{\isacharparenright}{\kern0pt}\isanewline
\ \ \ \ \ \ \isacommand{by}\isamarkupfalse%
\ simp\isanewline
\ \ \ \ \isacommand{also}\isamarkupfalse%
\ \isacommand{have}\isamarkupfalse%
\ {\isachardoublequoteopen}{\isachardot}{\kern0pt}{\isachardot}{\kern0pt}{\isachardot}{\kern0pt}\ {\isacharequal}{\kern0pt}\ real{\isacharunderscore}{\kern0pt}of{\isacharunderscore}{\kern0pt}rat\ {\isacharparenleft}{\kern0pt}{\isasymdelta}\ {\isacharasterisk}{\kern0pt}\ F\ k\ as{\isacharparenright}{\kern0pt}{\isacharcircum}{\kern0pt}{\isadigit{2}}{\isacharslash}{\kern0pt}{\isadigit{3}}{\isachardoublequoteclose}\isanewline
\ \ \ \ \ \ \isacommand{apply}\isamarkupfalse%
\ simp\isanewline
\ \ \ \ \ \ \isacommand{apply}\isamarkupfalse%
\ {\isacharparenleft}{\kern0pt}subst\ nonzero{\isacharunderscore}{\kern0pt}divide{\isacharunderscore}{\kern0pt}eq{\isacharunderscore}{\kern0pt}eq{\isacharcomma}{\kern0pt}\ simp\ add{\isacharcolon}{\kern0pt}s{\isadigit{1}}{\isacharunderscore}{\kern0pt}nonzero{\isacharparenright}{\kern0pt}\isanewline
\ \ \ \ \ \ \isacommand{by}\isamarkupfalse%
\ {\isacharparenleft}{\kern0pt}simp\ add{\isacharcolon}{\kern0pt}power{\isadigit{2}}{\isacharunderscore}{\kern0pt}eq{\isacharunderscore}{\kern0pt}square{\isacharparenright}{\kern0pt}\ \isanewline
\ \ \ \ \isacommand{finally}\isamarkupfalse%
\ \isacommand{show}\isamarkupfalse%
\ {\isachardoublequoteopen}{\isacharquery}{\kern0pt}rhs\ i{\isachardoublequoteclose}\ \isacommand{by}\isamarkupfalse%
\ simp\isanewline
\ \ \isacommand{qed}\isamarkupfalse%
\isanewline
\isanewline
\ \ \isacommand{have}\isamarkupfalse%
\ d{\isacharcolon}{\kern0pt}\ {\isachardoublequoteopen}\ {\isasymAnd}i{\isachardot}{\kern0pt}\ i\ {\isacharless}{\kern0pt}\ s\isactrlsub {\isadigit{2}}\ {\isasymLongrightarrow}\ measure{\isacharunderscore}{\kern0pt}pmf{\isachardot}{\kern0pt}prob\ {\isacharparenleft}{\kern0pt}prod{\isacharunderscore}{\kern0pt}pmf\ {\isacharparenleft}{\kern0pt}{\isacharbraceleft}{\kern0pt}{\isadigit{0}}{\isachardot}{\kern0pt}{\isachardot}{\kern0pt}{\isacharless}{\kern0pt}s\isactrlsub {\isadigit{1}}{\isacharbraceright}{\kern0pt}\ {\isasymtimes}\ {\isacharbraceleft}{\kern0pt}{\isadigit{0}}{\isachardot}{\kern0pt}{\isachardot}{\kern0pt}{\isacharless}{\kern0pt}s\isactrlsub {\isadigit{2}}{\isacharbraceright}{\kern0pt}{\isacharparenright}{\kern0pt}\ {\isacharparenleft}{\kern0pt}{\isasymlambda}{\isacharunderscore}{\kern0pt}{\isachardot}{\kern0pt}\ {\isasymOmega}{\isacharparenright}{\kern0pt}{\isacharparenright}{\kern0pt}\ \isanewline
\ \ {\isacharbraceleft}{\kern0pt}y{\isachardot}{\kern0pt}\ real{\isacharunderscore}{\kern0pt}of{\isacharunderscore}{\kern0pt}rat\ {\isacharparenleft}{\kern0pt}{\isasymdelta}\ {\isacharasterisk}{\kern0pt}\ F\ k\ as{\isacharparenright}{\kern0pt}\ {\isacharless}{\kern0pt}\ {\isasymbar}f{\isadigit{1}}\ y\ i\ {\isacharminus}{\kern0pt}\ real{\isacharunderscore}{\kern0pt}of{\isacharunderscore}{\kern0pt}rat\ {\isacharparenleft}{\kern0pt}F\ k\ as{\isacharparenright}{\kern0pt}{\isasymbar}{\isacharbraceright}{\kern0pt}\ {\isasymle}\ {\isadigit{1}}{\isacharslash}{\kern0pt}{\isadigit{3}}{\isachardoublequoteclose}\ {\isacharparenleft}{\kern0pt}\isakeyword{is}\ {\isachardoublequoteopen}{\isasymAnd}i{\isachardot}{\kern0pt}\ {\isacharunderscore}{\kern0pt}\ {\isasymLongrightarrow}\ {\isacharquery}{\kern0pt}lhs\ i\ {\isasymle}\ {\isacharunderscore}{\kern0pt}{\isachardoublequoteclose}{\isacharparenright}{\kern0pt}\isanewline
\ \ \isacommand{proof}\isamarkupfalse%
\ {\isacharminus}{\kern0pt}\isanewline
\ \ \ \ \isacommand{fix}\isamarkupfalse%
\ i\isanewline
\ \ \ \ \isacommand{assume}\isamarkupfalse%
\ d{\isacharunderscore}{\kern0pt}{\isadigit{1}}{\isacharcolon}{\kern0pt}{\isachardoublequoteopen}i\ {\isacharless}{\kern0pt}\ s\isactrlsub {\isadigit{2}}{\isachardoublequoteclose}\isanewline
\ \ \ \ \isacommand{define}\isamarkupfalse%
\ a\ \isakeyword{where}\ {\isachardoublequoteopen}a\ {\isacharequal}{\kern0pt}\ real{\isacharunderscore}{\kern0pt}of{\isacharunderscore}{\kern0pt}rat\ {\isacharparenleft}{\kern0pt}{\isasymdelta}\ {\isacharasterisk}{\kern0pt}\ F\ k\ as{\isacharparenright}{\kern0pt}{\isachardoublequoteclose}\isanewline
\ \ \ \ \isacommand{have}\isamarkupfalse%
\ d{\isacharunderscore}{\kern0pt}{\isadigit{2}}{\isacharcolon}{\kern0pt}\ {\isachardoublequoteopen}{\isadigit{0}}\ {\isacharless}{\kern0pt}\ a{\isachardoublequoteclose}\ \isacommand{apply}\isamarkupfalse%
\ {\isacharparenleft}{\kern0pt}simp\ add{\isacharcolon}{\kern0pt}a{\isacharunderscore}{\kern0pt}def{\isacharparenright}{\kern0pt}\isanewline
\ \ \ \ \ \ \isacommand{using}\isamarkupfalse%
\ assms\ fk{\isacharunderscore}{\kern0pt}nonzero\ mult{\isacharunderscore}{\kern0pt}pos{\isacharunderscore}{\kern0pt}pos\ \isacommand{by}\isamarkupfalse%
\ blast\isanewline
\ \ \ \ \isacommand{have}\isamarkupfalse%
\ d{\isacharunderscore}{\kern0pt}{\isadigit{3}}{\isacharcolon}{\kern0pt}\ {\isachardoublequoteopen}integrable\ {\isacharparenleft}{\kern0pt}measure{\isacharunderscore}{\kern0pt}pmf\ {\isacharparenleft}{\kern0pt}prod{\isacharunderscore}{\kern0pt}pmf\ {\isacharparenleft}{\kern0pt}{\isacharbraceleft}{\kern0pt}{\isadigit{0}}{\isachardot}{\kern0pt}{\isachardot}{\kern0pt}{\isacharless}{\kern0pt}s\isactrlsub {\isadigit{1}}{\isacharbraceright}{\kern0pt}\ {\isasymtimes}\ {\isacharbraceleft}{\kern0pt}{\isadigit{0}}{\isachardot}{\kern0pt}{\isachardot}{\kern0pt}{\isacharless}{\kern0pt}s\isactrlsub {\isadigit{2}}{\isacharbraceright}{\kern0pt}{\isacharparenright}{\kern0pt}\ {\isacharparenleft}{\kern0pt}{\isasymlambda}{\isacharunderscore}{\kern0pt}{\isachardot}{\kern0pt}\ {\isasymOmega}{\isacharparenright}{\kern0pt}{\isacharparenright}{\kern0pt}{\isacharparenright}{\kern0pt}\ {\isacharparenleft}{\kern0pt}{\isasymlambda}x{\isachardot}{\kern0pt}\ {\isacharparenleft}{\kern0pt}f{\isadigit{1}}\ x\ i{\isacharparenright}{\kern0pt}\isactrlsup {\isadigit{2}}{\isacharparenright}{\kern0pt}{\isachardoublequoteclose}\isanewline
\ \ \ \ \ \ \isacommand{by}\isamarkupfalse%
\ {\isacharparenleft}{\kern0pt}rule\ integrable{\isacharunderscore}{\kern0pt}measure{\isacharunderscore}{\kern0pt}pmf{\isacharunderscore}{\kern0pt}finite{\isacharbrackleft}{\kern0pt}OF\ fin{\isacharunderscore}{\kern0pt}omega{\isacharunderscore}{\kern0pt}{\isadigit{2}}{\isacharbrackright}{\kern0pt}{\isacharparenright}{\kern0pt}\isanewline
\ \ \ \ \isacommand{have}\isamarkupfalse%
\ {\isachardoublequoteopen}{\isacharquery}{\kern0pt}lhs\ i\ {\isasymle}\ measure{\isacharunderscore}{\kern0pt}pmf{\isachardot}{\kern0pt}prob\ {\isacharparenleft}{\kern0pt}prod{\isacharunderscore}{\kern0pt}pmf\ {\isacharparenleft}{\kern0pt}{\isacharbraceleft}{\kern0pt}{\isadigit{0}}{\isachardot}{\kern0pt}{\isachardot}{\kern0pt}{\isacharless}{\kern0pt}s\isactrlsub {\isadigit{1}}{\isacharbraceright}{\kern0pt}\ {\isasymtimes}\ {\isacharbraceleft}{\kern0pt}{\isadigit{0}}{\isachardot}{\kern0pt}{\isachardot}{\kern0pt}{\isacharless}{\kern0pt}s\isactrlsub {\isadigit{2}}{\isacharbraceright}{\kern0pt}{\isacharparenright}{\kern0pt}\ {\isacharparenleft}{\kern0pt}{\isasymlambda}{\isacharunderscore}{\kern0pt}{\isachardot}{\kern0pt}\ {\isasymOmega}{\isacharparenright}{\kern0pt}{\isacharparenright}{\kern0pt}\isanewline
\ \ \ \ \ \ {\isacharbraceleft}{\kern0pt}y{\isachardot}{\kern0pt}\ real{\isacharunderscore}{\kern0pt}of{\isacharunderscore}{\kern0pt}rat\ {\isacharparenleft}{\kern0pt}{\isasymdelta}\ {\isacharasterisk}{\kern0pt}\ F\ k\ as{\isacharparenright}{\kern0pt}\ {\isasymle}\ {\isasymbar}f{\isadigit{1}}\ y\ i\ {\isacharminus}{\kern0pt}\ real{\isacharunderscore}{\kern0pt}of{\isacharunderscore}{\kern0pt}rat\ {\isacharparenleft}{\kern0pt}F\ k\ as{\isacharparenright}{\kern0pt}{\isasymbar}{\isacharbraceright}{\kern0pt}{\isachardoublequoteclose}\isanewline
\ \ \ \ \ \ \isacommand{by}\isamarkupfalse%
\ {\isacharparenleft}{\kern0pt}rule\ pmf{\isacharunderscore}{\kern0pt}mono{\isacharunderscore}{\kern0pt}{\isadigit{1}}{\isacharcomma}{\kern0pt}\ simp{\isacharparenright}{\kern0pt}\isanewline
\ \ \ \ \isacommand{also}\isamarkupfalse%
\ \isacommand{have}\isamarkupfalse%
\ {\isachardoublequoteopen}{\isachardot}{\kern0pt}{\isachardot}{\kern0pt}{\isachardot}{\kern0pt}\ {\isasymle}\ prob{\isacharunderscore}{\kern0pt}space{\isachardot}{\kern0pt}variance\ {\isacharparenleft}{\kern0pt}prod{\isacharunderscore}{\kern0pt}pmf\ {\isacharparenleft}{\kern0pt}{\isacharbraceleft}{\kern0pt}{\isadigit{0}}{\isachardot}{\kern0pt}{\isachardot}{\kern0pt}{\isacharless}{\kern0pt}s\isactrlsub {\isadigit{1}}{\isacharbraceright}{\kern0pt}\ {\isasymtimes}\ {\isacharbraceleft}{\kern0pt}{\isadigit{0}}{\isachardot}{\kern0pt}{\isachardot}{\kern0pt}{\isacharless}{\kern0pt}s\isactrlsub {\isadigit{2}}{\isacharbraceright}{\kern0pt}{\isacharparenright}{\kern0pt}\ {\isacharparenleft}{\kern0pt}{\isasymlambda}{\isacharunderscore}{\kern0pt}{\isachardot}{\kern0pt}\ {\isasymOmega}{\isacharparenright}{\kern0pt}{\isacharparenright}{\kern0pt}\ {\isacharparenleft}{\kern0pt}{\isasymlambda}{\isasymomega}{\isachardot}{\kern0pt}\ f{\isadigit{1}}\ {\isasymomega}\ i{\isacharparenright}{\kern0pt}{\isacharslash}{\kern0pt}a{\isacharcircum}{\kern0pt}{\isadigit{2}}{\isachardoublequoteclose}\isanewline
\ \ \ \ \ \ \isacommand{using}\isamarkupfalse%
\ f{\isadigit{1}}{\isacharunderscore}{\kern0pt}exp{\isacharbrackleft}{\kern0pt}OF\ d{\isacharunderscore}{\kern0pt}{\isadigit{1}}{\isacharbrackright}{\kern0pt}\isanewline
\ \ \ \ \ \ \isacommand{using}\isamarkupfalse%
\ prob{\isacharunderscore}{\kern0pt}space{\isachardot}{\kern0pt}Chebyshev{\isacharunderscore}{\kern0pt}inequality{\isacharbrackleft}{\kern0pt}OF\ prob{\isacharunderscore}{\kern0pt}space{\isacharunderscore}{\kern0pt}measure{\isacharunderscore}{\kern0pt}pmf\ {\isacharunderscore}{\kern0pt}\ d{\isacharunderscore}{\kern0pt}{\isadigit{3}}\ d{\isacharunderscore}{\kern0pt}{\isadigit{2}}{\isacharcomma}{\kern0pt}\ simplified{\isacharbrackright}{\kern0pt}\isanewline
\ \ \ \ \ \ \isacommand{by}\isamarkupfalse%
\ {\isacharparenleft}{\kern0pt}simp\ add{\isacharcolon}{\kern0pt}a{\isacharunderscore}{\kern0pt}def{\isacharbrackleft}{\kern0pt}symmetric{\isacharbrackright}{\kern0pt}\ has{\isacharunderscore}{\kern0pt}bochner{\isacharunderscore}{\kern0pt}integral{\isacharunderscore}{\kern0pt}iff{\isacharparenright}{\kern0pt}\isanewline
\ \ \ \ \isacommand{also}\isamarkupfalse%
\ \isacommand{have}\isamarkupfalse%
\ {\isachardoublequoteopen}{\isachardot}{\kern0pt}{\isachardot}{\kern0pt}{\isachardot}{\kern0pt}\ {\isasymle}\ {\isadigit{1}}{\isacharslash}{\kern0pt}{\isadigit{3}}{\isachardoublequoteclose}\ \isacommand{using}\isamarkupfalse%
\ d{\isacharunderscore}{\kern0pt}{\isadigit{2}}\isanewline
\ \ \ \ \ \ \isacommand{using}\isamarkupfalse%
\ f{\isadigit{1}}{\isacharunderscore}{\kern0pt}var{\isacharbrackleft}{\kern0pt}OF\ d{\isacharunderscore}{\kern0pt}{\isadigit{1}}{\isacharbrackright}{\kern0pt}\ \isanewline
\ \ \ \ \ \ \isacommand{by}\isamarkupfalse%
\ {\isacharparenleft}{\kern0pt}simp\ add{\isacharcolon}{\kern0pt}algebra{\isacharunderscore}{\kern0pt}simps{\isacharcomma}{\kern0pt}\ simp\ add{\isacharcolon}{\kern0pt}a{\isacharunderscore}{\kern0pt}def{\isacharparenright}{\kern0pt}\isanewline
\ \ \ \ \isacommand{finally}\isamarkupfalse%
\ \isacommand{show}\isamarkupfalse%
\ {\isachardoublequoteopen}{\isacharquery}{\kern0pt}lhs\ i\ {\isasymle}\ {\isadigit{1}}{\isacharslash}{\kern0pt}{\isadigit{3}}{\isachardoublequoteclose}\isanewline
\ \ \ \ \ \ \isacommand{by}\isamarkupfalse%
\ blast\isanewline
\ \ \isacommand{qed}\isamarkupfalse%
\isanewline
\isanewline
\ \ \isacommand{show}\isamarkupfalse%
\ {\isacharquery}{\kern0pt}thesis\isanewline
\ \ \ \ \isacommand{apply}\isamarkupfalse%
\ {\isacharparenleft}{\kern0pt}simp\ add{\isacharcolon}{\kern0pt}\ b\ comp{\isacharunderscore}{\kern0pt}def\ map{\isacharunderscore}{\kern0pt}pmf{\isacharunderscore}{\kern0pt}def{\isacharbrackleft}{\kern0pt}symmetric{\isacharbrackright}{\kern0pt}{\isacharparenright}{\kern0pt}\isanewline
\ \ \ \ \isacommand{apply}\isamarkupfalse%
\ {\isacharparenleft}{\kern0pt}subst\ c{\isacharbrackleft}{\kern0pt}symmetric{\isacharbrackright}{\kern0pt}{\isacharparenright}{\kern0pt}\isanewline
\ \ \ \ \isacommand{apply}\isamarkupfalse%
\ {\isacharparenleft}{\kern0pt}simp\ add{\isacharcolon}{\kern0pt}f{\isacharprime}{\kern0pt}{\isacharunderscore}{\kern0pt}def{\isacharparenright}{\kern0pt}\isanewline
\ \ \ \ \isacommand{apply}\isamarkupfalse%
\ {\isacharparenleft}{\kern0pt}rule\ prob{\isacharunderscore}{\kern0pt}space{\isachardot}{\kern0pt}median{\isacharunderscore}{\kern0pt}bound{\isacharunderscore}{\kern0pt}{\isadigit{2}}{\isacharbrackleft}{\kern0pt}\isakeyword{where}\ X{\isacharequal}{\kern0pt}{\isachardoublequoteopen}{\isasymlambda}i\ {\isasymomega}{\isachardot}{\kern0pt}\ f{\isadigit{1}}\ {\isasymomega}\ i{\isachardoublequoteclose}\ \isakeyword{and}\ M{\isacharequal}{\kern0pt}{\isachardoublequoteopen}{\isacharparenleft}{\kern0pt}prod{\isacharunderscore}{\kern0pt}pmf\ {\isacharparenleft}{\kern0pt}{\isacharbraceleft}{\kern0pt}{\isadigit{0}}{\isachardot}{\kern0pt}{\isachardot}{\kern0pt}{\isacharless}{\kern0pt}s\isactrlsub {\isadigit{1}}{\isacharbraceright}{\kern0pt}\ {\isasymtimes}\ {\isacharbraceleft}{\kern0pt}{\isadigit{0}}{\isachardot}{\kern0pt}{\isachardot}{\kern0pt}{\isacharless}{\kern0pt}s\isactrlsub {\isadigit{2}}{\isacharbraceright}{\kern0pt}{\isacharparenright}{\kern0pt}\ {\isacharparenleft}{\kern0pt}{\isasymlambda}{\isacharunderscore}{\kern0pt}{\isachardot}{\kern0pt}\ {\isasymOmega}{\isacharparenright}{\kern0pt}{\isacharparenright}{\kern0pt}{\isachardoublequoteclose}{\isacharcomma}{\kern0pt}\ simplified{\isacharbrackright}{\kern0pt}{\isacharparenright}{\kern0pt}\isanewline
\ \ \ \ \ \ \ \ \ \isacommand{apply}\isamarkupfalse%
\ {\isacharparenleft}{\kern0pt}simp\ add{\isacharcolon}{\kern0pt}prob{\isacharunderscore}{\kern0pt}space{\isacharunderscore}{\kern0pt}measure{\isacharunderscore}{\kern0pt}pmf{\isacharparenright}{\kern0pt}\isanewline
\ \ \ \ \ \ \ \ \isacommand{using}\isamarkupfalse%
\ assms{\isacharparenleft}{\kern0pt}{\isadigit{2}}{\isacharparenright}{\kern0pt}\ \isacommand{apply}\isamarkupfalse%
\ simp\isanewline
\ \ \ \ \ \ \ \isacommand{using}\isamarkupfalse%
\ assms{\isacharparenleft}{\kern0pt}{\isadigit{2}}{\isacharparenright}{\kern0pt}\ \isacommand{apply}\isamarkupfalse%
\ simp\isanewline
\ \ \ \ \ \ \ \isacommand{apply}\isamarkupfalse%
\ {\isacharparenleft}{\kern0pt}simp\ add{\isacharcolon}{\kern0pt}f{\isadigit{1}}{\isacharunderscore}{\kern0pt}def\ f{\isadigit{2}}{\isacharunderscore}{\kern0pt}def{\isacharparenright}{\kern0pt}\isanewline
\ \ \ \ \ \ \ \isacommand{apply}\isamarkupfalse%
\ {\isacharparenleft}{\kern0pt}rule\ indep{\isacharunderscore}{\kern0pt}vars{\isacharunderscore}{\kern0pt}restrict{\isacharunderscore}{\kern0pt}intro{\isacharbrackleft}{\kern0pt}\isakeyword{where}\ f{\isacharequal}{\kern0pt}{\isachardoublequoteopen}{\isasymlambda}i{\isachardot}{\kern0pt}\ {\isacharparenleft}{\kern0pt}{\isacharbraceleft}{\kern0pt}{\isadigit{0}}{\isachardot}{\kern0pt}{\isachardot}{\kern0pt}{\isacharless}{\kern0pt}s\isactrlsub {\isadigit{1}}{\isacharbraceright}{\kern0pt}{\isasymtimes}{\isacharbraceleft}{\kern0pt}i{\isacharbraceright}{\kern0pt}{\isacharparenright}{\kern0pt}{\isachardoublequoteclose}{\isacharbrackright}{\kern0pt}{\isacharparenright}{\kern0pt}\isanewline
\ \ \ \ \ \ \ \ \ \ \ \isacommand{apply}\isamarkupfalse%
\ {\isacharparenleft}{\kern0pt}simp{\isacharparenright}{\kern0pt}\isanewline
\ \ \ \ \ \ \ \ \ \ \isacommand{apply}\isamarkupfalse%
\ {\isacharparenleft}{\kern0pt}simp\ add{\isacharcolon}{\kern0pt}disjoint{\isacharunderscore}{\kern0pt}family{\isacharunderscore}{\kern0pt}on{\isacharunderscore}{\kern0pt}def{\isacharcomma}{\kern0pt}\ blast{\isacharparenright}{\kern0pt}\isanewline
\ \ \ \ \ \ \ \ \ \isacommand{apply}\isamarkupfalse%
\ {\isacharparenleft}{\kern0pt}simp\ add{\isacharcolon}{\kern0pt}s{\isadigit{2}}{\isacharunderscore}{\kern0pt}nonzero{\isacharparenright}{\kern0pt}\isanewline
\ \ \ \ \ \ \ \ \isacommand{apply}\isamarkupfalse%
\ {\isacharparenleft}{\kern0pt}rule\ subsetI{\isacharcomma}{\kern0pt}\ simp{\isacharcomma}{\kern0pt}\ force{\isacharparenright}{\kern0pt}\isanewline
\ \ \ \ \ \ \ \isacommand{apply}\isamarkupfalse%
{\isacharparenleft}{\kern0pt}simp{\isacharparenright}{\kern0pt}\isanewline
\ \ \ \ \ \ \isacommand{apply}\isamarkupfalse%
\ {\isacharparenleft}{\kern0pt}simp{\isacharparenright}{\kern0pt}\isanewline
\ \ \ \ \ \isacommand{apply}\isamarkupfalse%
\ {\isacharparenleft}{\kern0pt}simp\ add{\isacharcolon}{\kern0pt}\ s\isactrlsub {\isadigit{2}}{\isacharunderscore}{\kern0pt}def{\isacharparenright}{\kern0pt}\ \isanewline
\ \ \ \ \ \ \ \isacommand{using}\isamarkupfalse%
\ of{\isacharunderscore}{\kern0pt}nat{\isacharunderscore}{\kern0pt}ceiling\ \isacommand{apply}\isamarkupfalse%
\ blast\ \isanewline
\ \ \ \ \ \isacommand{using}\isamarkupfalse%
\ d\ \isacommand{by}\isamarkupfalse%
\ simp\isanewline
\isacommand{qed}\isamarkupfalse%
%
\endisatagproof
{\isafoldproof}%
%
\isadelimproof
\isanewline
%
\endisadelimproof
\isanewline
\isacommand{fun}\isamarkupfalse%
\ fk{\isacharunderscore}{\kern0pt}space{\isacharunderscore}{\kern0pt}usage\ {\isacharcolon}{\kern0pt}{\isacharcolon}{\kern0pt}\ {\isachardoublequoteopen}{\isacharparenleft}{\kern0pt}nat\ {\isasymtimes}\ nat\ {\isasymtimes}\ nat\ {\isasymtimes}\ rat\ {\isasymtimes}\ rat{\isacharparenright}{\kern0pt}\ {\isasymRightarrow}\ real{\isachardoublequoteclose}\ \isakeyword{where}\isanewline
\ \ {\isachardoublequoteopen}fk{\isacharunderscore}{\kern0pt}space{\isacharunderscore}{\kern0pt}usage\ {\isacharparenleft}{\kern0pt}k{\isacharcomma}{\kern0pt}\ n{\isacharcomma}{\kern0pt}\ m{\isacharcomma}{\kern0pt}\ {\isasymepsilon}{\isacharcomma}{\kern0pt}\ {\isasymdelta}{\isacharparenright}{\kern0pt}\ {\isacharequal}{\kern0pt}\ {\isacharparenleft}{\kern0pt}\isanewline
\ \ \ \ let\ s\isactrlsub {\isadigit{1}}\ {\isacharequal}{\kern0pt}\ nat\ {\isasymlceil}{\isadigit{3}}{\isacharasterisk}{\kern0pt}real\ k{\isacharasterisk}{\kern0pt}{\isacharparenleft}{\kern0pt}real\ n{\isacharparenright}{\kern0pt}\ powr\ {\isacharparenleft}{\kern0pt}{\isadigit{1}}{\isacharminus}{\kern0pt}{\isadigit{1}}{\isacharslash}{\kern0pt}\ real\ k{\isacharparenright}{\kern0pt}\ {\isacharslash}{\kern0pt}\ {\isacharparenleft}{\kern0pt}real{\isacharunderscore}{\kern0pt}of{\isacharunderscore}{\kern0pt}rat\ {\isasymdelta}{\isacharparenright}{\kern0pt}\isactrlsup {\isadigit{2}}\ {\isasymrceil}\ in\isanewline
\ \ \ \ let\ s\isactrlsub {\isadigit{2}}\ {\isacharequal}{\kern0pt}\ nat\ {\isasymlceil}{\isacharminus}{\kern0pt}{\isacharparenleft}{\kern0pt}{\isadigit{1}}{\isadigit{8}}\ {\isacharasterisk}{\kern0pt}\ ln\ {\isacharparenleft}{\kern0pt}real{\isacharunderscore}{\kern0pt}of{\isacharunderscore}{\kern0pt}rat\ {\isasymepsilon}{\isacharparenright}{\kern0pt}{\isacharparenright}{\kern0pt}{\isasymrceil}\ in\ \isanewline
\ \ \ \ {\isadigit{5}}\ {\isacharplus}{\kern0pt}\isanewline
\ \ \ \ {\isadigit{2}}\ {\isacharasterisk}{\kern0pt}\ log\ {\isadigit{2}}\ {\isacharparenleft}{\kern0pt}s\isactrlsub {\isadigit{1}}\ {\isacharplus}{\kern0pt}\ {\isadigit{1}}{\isacharparenright}{\kern0pt}\ {\isacharplus}{\kern0pt}\isanewline
\ \ \ \ {\isadigit{2}}\ {\isacharasterisk}{\kern0pt}\ log\ {\isadigit{2}}\ {\isacharparenleft}{\kern0pt}s\isactrlsub {\isadigit{2}}\ {\isacharplus}{\kern0pt}\ {\isadigit{1}}{\isacharparenright}{\kern0pt}\ {\isacharplus}{\kern0pt}\isanewline
\ \ \ \ {\isadigit{2}}\ {\isacharasterisk}{\kern0pt}\ log\ {\isadigit{2}}\ {\isacharparenleft}{\kern0pt}real\ k\ {\isacharplus}{\kern0pt}\ {\isadigit{1}}{\isacharparenright}{\kern0pt}\ {\isacharplus}{\kern0pt}\isanewline
\ \ \ \ {\isadigit{2}}\ {\isacharasterisk}{\kern0pt}\ log\ {\isadigit{2}}\ {\isacharparenleft}{\kern0pt}real\ m\ {\isacharplus}{\kern0pt}\ {\isadigit{1}}{\isacharparenright}{\kern0pt}\ {\isacharplus}{\kern0pt}\isanewline
\ \ \ \ s\isactrlsub {\isadigit{1}}\ {\isacharasterisk}{\kern0pt}\ s\isactrlsub {\isadigit{2}}\ {\isacharasterisk}{\kern0pt}\ {\isacharparenleft}{\kern0pt}{\isadigit{3}}\ {\isacharplus}{\kern0pt}\ {\isadigit{2}}\ {\isacharasterisk}{\kern0pt}\ log\ {\isadigit{2}}\ {\isacharparenleft}{\kern0pt}real\ n{\isacharplus}{\kern0pt}{\isadigit{1}}{\isacharparenright}{\kern0pt}\ {\isacharplus}{\kern0pt}\ {\isadigit{2}}\ {\isacharasterisk}{\kern0pt}\ log\ {\isadigit{2}}\ {\isacharparenleft}{\kern0pt}real\ m{\isacharplus}{\kern0pt}{\isadigit{1}}{\isacharparenright}{\kern0pt}{\isacharparenright}{\kern0pt}{\isacharparenright}{\kern0pt}{\isachardoublequoteclose}\isanewline
\isanewline
\isacommand{definition}\isamarkupfalse%
\ encode{\isacharunderscore}{\kern0pt}fk{\isacharunderscore}{\kern0pt}state\ {\isacharcolon}{\kern0pt}{\isacharcolon}{\kern0pt}\ {\isachardoublequoteopen}fk{\isacharunderscore}{\kern0pt}state\ {\isasymRightarrow}\ bool\ list\ option{\isachardoublequoteclose}\ \isakeyword{where}\isanewline
\ \ {\isachardoublequoteopen}encode{\isacharunderscore}{\kern0pt}fk{\isacharunderscore}{\kern0pt}state\ {\isacharequal}{\kern0pt}\ \isanewline
\ \ \ \ N\isactrlsub S\ {\isasymtimes}\isactrlsub D\ {\isacharparenleft}{\kern0pt}{\isasymlambda}s\isactrlsub {\isadigit{1}}{\isachardot}{\kern0pt}\ \isanewline
\ \ \ \ N\isactrlsub S\ {\isasymtimes}\isactrlsub D\ {\isacharparenleft}{\kern0pt}{\isasymlambda}s\isactrlsub {\isadigit{2}}{\isachardot}{\kern0pt}\ \isanewline
\ \ \ \ N\isactrlsub S\ {\isasymtimes}\isactrlsub S\ \ \isanewline
\ \ \ \ N\isactrlsub S\ {\isasymtimes}\isactrlsub S\ \ \isanewline
\ \ \ \ {\isacharparenleft}{\kern0pt}List{\isachardot}{\kern0pt}product\ {\isacharbrackleft}{\kern0pt}{\isadigit{0}}{\isachardot}{\kern0pt}{\isachardot}{\kern0pt}{\isacharless}{\kern0pt}s\isactrlsub {\isadigit{1}}{\isacharbrackright}{\kern0pt}\ {\isacharbrackleft}{\kern0pt}{\isadigit{0}}{\isachardot}{\kern0pt}{\isachardot}{\kern0pt}{\isacharless}{\kern0pt}s\isactrlsub {\isadigit{2}}{\isacharbrackright}{\kern0pt}\ {\isasymrightarrow}\isactrlsub S\ {\isacharparenleft}{\kern0pt}N\isactrlsub S\ {\isasymtimes}\isactrlsub S\ N\isactrlsub S{\isacharparenright}{\kern0pt}{\isacharparenright}{\kern0pt}{\isacharparenright}{\kern0pt}{\isacharparenright}{\kern0pt}{\isachardoublequoteclose}\isanewline
\isanewline
\isacommand{lemma}\isamarkupfalse%
\ {\isachardoublequoteopen}inj{\isacharunderscore}{\kern0pt}on\ encode{\isacharunderscore}{\kern0pt}fk{\isacharunderscore}{\kern0pt}state\ {\isacharparenleft}{\kern0pt}dom\ encode{\isacharunderscore}{\kern0pt}fk{\isacharunderscore}{\kern0pt}state{\isacharparenright}{\kern0pt}{\isachardoublequoteclose}\isanewline
%
\isadelimproof
\ \ %
\endisadelimproof
%
\isatagproof
\isacommand{apply}\isamarkupfalse%
\ {\isacharparenleft}{\kern0pt}rule\ encoding{\isacharunderscore}{\kern0pt}imp{\isacharunderscore}{\kern0pt}inj{\isacharparenright}{\kern0pt}\isanewline
\ \ \isacommand{apply}\isamarkupfalse%
\ {\isacharparenleft}{\kern0pt}simp\ add{\isacharcolon}{\kern0pt}encode{\isacharunderscore}{\kern0pt}fk{\isacharunderscore}{\kern0pt}state{\isacharunderscore}{\kern0pt}def{\isacharparenright}{\kern0pt}\isanewline
\ \ \isacommand{apply}\isamarkupfalse%
\ {\isacharparenleft}{\kern0pt}rule\ dependent{\isacharunderscore}{\kern0pt}encoding{\isacharcomma}{\kern0pt}\ metis\ nat{\isacharunderscore}{\kern0pt}encoding{\isacharparenright}{\kern0pt}\isanewline
\ \ \isacommand{apply}\isamarkupfalse%
\ {\isacharparenleft}{\kern0pt}rule\ dependent{\isacharunderscore}{\kern0pt}encoding{\isacharcomma}{\kern0pt}\ metis\ nat{\isacharunderscore}{\kern0pt}encoding{\isacharparenright}{\kern0pt}\isanewline
\ \ \isacommand{apply}\isamarkupfalse%
\ {\isacharparenleft}{\kern0pt}rule\ prod{\isacharunderscore}{\kern0pt}encoding{\isacharcomma}{\kern0pt}\ metis\ nat{\isacharunderscore}{\kern0pt}encoding{\isacharparenright}{\kern0pt}\isanewline
\ \ \isacommand{apply}\isamarkupfalse%
\ {\isacharparenleft}{\kern0pt}rule\ prod{\isacharunderscore}{\kern0pt}encoding{\isacharcomma}{\kern0pt}\ metis\ nat{\isacharunderscore}{\kern0pt}encoding{\isacharparenright}{\kern0pt}\isanewline
\ \ \isacommand{by}\isamarkupfalse%
\ {\isacharparenleft}{\kern0pt}metis\ encode{\isacharunderscore}{\kern0pt}extensional\ prod{\isacharunderscore}{\kern0pt}encoding\ nat{\isacharunderscore}{\kern0pt}encoding{\isacharparenright}{\kern0pt}%
\endisatagproof
{\isafoldproof}%
%
\isadelimproof
\isanewline
%
\endisadelimproof
\isanewline
\isacommand{theorem}\isamarkupfalse%
\ fk{\isacharunderscore}{\kern0pt}exact{\isacharunderscore}{\kern0pt}space{\isacharunderscore}{\kern0pt}usage{\isacharcolon}{\kern0pt}\isanewline
\ \ \isakeyword{assumes}\ {\isachardoublequoteopen}k\ {\isasymge}\ {\isadigit{1}}{\isachardoublequoteclose}\isanewline
\ \ \isakeyword{assumes}\ {\isachardoublequoteopen}{\isasymepsilon}\ {\isasymin}\ {\isacharbraceleft}{\kern0pt}{\isadigit{0}}{\isacharless}{\kern0pt}{\isachardot}{\kern0pt}{\isachardot}{\kern0pt}{\isacharless}{\kern0pt}{\isadigit{1}}{\isacharbraceright}{\kern0pt}{\isachardoublequoteclose}\isanewline
\ \ \isakeyword{assumes}\ {\isachardoublequoteopen}{\isasymdelta}\ {\isachargreater}{\kern0pt}\ {\isadigit{0}}{\isachardoublequoteclose}\isanewline
\ \ \isakeyword{assumes}\ {\isachardoublequoteopen}set\ as\ {\isasymsubseteq}\ {\isacharbraceleft}{\kern0pt}{\isadigit{0}}{\isachardot}{\kern0pt}{\isachardot}{\kern0pt}{\isacharless}{\kern0pt}n{\isacharbraceright}{\kern0pt}{\isachardoublequoteclose}\isanewline
\ \ \isakeyword{defines}\ {\isachardoublequoteopen}M\ {\isasymequiv}\ fold\ {\isacharparenleft}{\kern0pt}{\isasymlambda}a\ state{\isachardot}{\kern0pt}\ state\ {\isasymbind}\ fk{\isacharunderscore}{\kern0pt}update\ a{\isacharparenright}{\kern0pt}\ as\ {\isacharparenleft}{\kern0pt}fk{\isacharunderscore}{\kern0pt}init\ k\ {\isasymdelta}\ {\isasymepsilon}\ n{\isacharparenright}{\kern0pt}{\isachardoublequoteclose}\isanewline
\ \ \isakeyword{shows}\ {\isachardoublequoteopen}AE\ {\isasymomega}\ in\ M{\isachardot}{\kern0pt}\ bit{\isacharunderscore}{\kern0pt}count\ {\isacharparenleft}{\kern0pt}encode{\isacharunderscore}{\kern0pt}fk{\isacharunderscore}{\kern0pt}state\ {\isasymomega}{\isacharparenright}{\kern0pt}\ {\isasymle}\ fk{\isacharunderscore}{\kern0pt}space{\isacharunderscore}{\kern0pt}usage\ {\isacharparenleft}{\kern0pt}k{\isacharcomma}{\kern0pt}\ n{\isacharcomma}{\kern0pt}\ length\ as{\isacharcomma}{\kern0pt}\ {\isasymepsilon}{\isacharcomma}{\kern0pt}\ {\isasymdelta}{\isacharparenright}{\kern0pt}{\isachardoublequoteclose}\ {\isacharparenleft}{\kern0pt}\isakeyword{is}\ {\isachardoublequoteopen}AE\ {\isasymomega}\ in\ M{\isachardot}{\kern0pt}\ {\isacharparenleft}{\kern0pt}{\isacharunderscore}{\kern0pt}\ \ {\isasymle}\ {\isacharquery}{\kern0pt}rhs{\isacharparenright}{\kern0pt}{\isachardoublequoteclose}{\isacharparenright}{\kern0pt}\isanewline
%
\isadelimproof
%
\endisadelimproof
%
\isatagproof
\isacommand{proof}\isamarkupfalse%
\ {\isacharparenleft}{\kern0pt}cases\ {\isachardoublequoteopen}as\ {\isacharequal}{\kern0pt}\ {\isacharbrackleft}{\kern0pt}{\isacharbrackright}{\kern0pt}{\isachardoublequoteclose}{\isacharparenright}{\kern0pt}\isanewline
\ \ \isacommand{case}\isamarkupfalse%
\ True\isanewline
\ \ \isacommand{have}\isamarkupfalse%
\ a{\isacharcolon}{\kern0pt}{\isachardoublequoteopen}M\ {\isacharequal}{\kern0pt}\ fk{\isacharunderscore}{\kern0pt}init\ k\ {\isasymdelta}\ {\isasymepsilon}\ n{\isachardoublequoteclose}\isanewline
\ \ \ \ \isacommand{using}\isamarkupfalse%
\ True\ \isacommand{by}\isamarkupfalse%
\ {\isacharparenleft}{\kern0pt}simp\ add{\isacharcolon}{\kern0pt}M{\isacharunderscore}{\kern0pt}def{\isacharparenright}{\kern0pt}\isanewline
\ \ \isacommand{define}\isamarkupfalse%
\ s\isactrlsub {\isadigit{1}}\ \isakeyword{where}\ {\isachardoublequoteopen}s\isactrlsub {\isadigit{1}}\ {\isacharequal}{\kern0pt}\ nat\ {\isasymlceil}{\isadigit{3}}{\isacharasterisk}{\kern0pt}real\ k{\isacharasterisk}{\kern0pt}{\isacharparenleft}{\kern0pt}real\ n{\isacharparenright}{\kern0pt}\ powr\ {\isacharparenleft}{\kern0pt}{\isadigit{1}}{\isacharminus}{\kern0pt}{\isadigit{1}}{\isacharslash}{\kern0pt}\ real\ k{\isacharparenright}{\kern0pt}{\isacharslash}{\kern0pt}\ {\isacharparenleft}{\kern0pt}real{\isacharunderscore}{\kern0pt}of{\isacharunderscore}{\kern0pt}rat\ {\isasymdelta}{\isacharparenright}{\kern0pt}\isactrlsup {\isadigit{2}}{\isasymrceil}{\isachardoublequoteclose}\isanewline
\ \ \isacommand{define}\isamarkupfalse%
\ s\isactrlsub {\isadigit{2}}\ \isakeyword{where}\ {\isachardoublequoteopen}s\isactrlsub {\isadigit{2}}\ {\isacharequal}{\kern0pt}\ nat\ {\isasymlceil}{\isacharminus}{\kern0pt}{\isacharparenleft}{\kern0pt}{\isadigit{1}}{\isadigit{8}}\ {\isacharasterisk}{\kern0pt}\ ln\ {\isacharparenleft}{\kern0pt}real{\isacharunderscore}{\kern0pt}of{\isacharunderscore}{\kern0pt}rat\ {\isasymepsilon}{\isacharparenright}{\kern0pt}{\isacharparenright}{\kern0pt}{\isasymrceil}{\isachardoublequoteclose}\isanewline
\ \ \isacommand{define}\isamarkupfalse%
\ w\ \isakeyword{where}\ {\isachardoublequoteopen}w\ {\isacharequal}{\kern0pt}\ {\isacharparenleft}{\kern0pt}{\isadigit{2}}{\isacharcolon}{\kern0pt}{\isacharcolon}{\kern0pt}ereal{\isacharparenright}{\kern0pt}{\isachardoublequoteclose}\isanewline
\isanewline
\ \ \isacommand{have}\isamarkupfalse%
\ h{\isacharcolon}{\kern0pt}\ {\isachardoublequoteopen}{\isasymAnd}x{\isachardot}{\kern0pt}\ x\ {\isasymin}\ {\isacharparenleft}{\kern0pt}{\isasymlambda}x{\isachardot}{\kern0pt}\ {\isacharparenleft}{\kern0pt}{\isadigit{0}}{\isacharcomma}{\kern0pt}\ {\isadigit{0}}{\isacharparenright}{\kern0pt}{\isacharparenright}{\kern0pt}\ {\isacharbackquote}{\kern0pt}\ {\isacharparenleft}{\kern0pt}{\isacharbraceleft}{\kern0pt}{\isadigit{0}}{\isachardot}{\kern0pt}{\isachardot}{\kern0pt}{\isacharless}{\kern0pt}s\isactrlsub {\isadigit{1}}{\isacharbraceright}{\kern0pt}\ {\isasymtimes}\ {\isacharbraceleft}{\kern0pt}{\isadigit{0}}{\isachardot}{\kern0pt}{\isachardot}{\kern0pt}{\isacharless}{\kern0pt}s\isactrlsub {\isadigit{2}}{\isacharbraceright}{\kern0pt}{\isacharparenright}{\kern0pt}\ {\isasymLongrightarrow}\ bit{\isacharunderscore}{\kern0pt}count\ {\isacharparenleft}{\kern0pt}{\isacharparenleft}{\kern0pt}N\isactrlsub S\ {\isasymtimes}\isactrlsub S\ N\isactrlsub S{\isacharparenright}{\kern0pt}\ x{\isacharparenright}{\kern0pt}\ {\isasymle}\ {\isadigit{2}}{\isachardoublequoteclose}\isanewline
\ \ \isacommand{proof}\isamarkupfalse%
\ {\isacharminus}{\kern0pt}\isanewline
\ \ \ \ \isacommand{fix}\isamarkupfalse%
\ x\isanewline
\ \ \ \ \isacommand{assume}\isamarkupfalse%
\ h{\isacharunderscore}{\kern0pt}a{\isacharcolon}{\kern0pt}\ {\isachardoublequoteopen}x\ {\isasymin}\ {\isacharparenleft}{\kern0pt}{\isasymlambda}x{\isachardot}{\kern0pt}\ {\isacharparenleft}{\kern0pt}{\isadigit{0}}\ {\isacharcolon}{\kern0pt}{\isacharcolon}{\kern0pt}\ nat{\isacharcomma}{\kern0pt}\ {\isadigit{0}}\ {\isacharcolon}{\kern0pt}{\isacharcolon}{\kern0pt}\ nat{\isacharparenright}{\kern0pt}{\isacharparenright}{\kern0pt}\ {\isacharbackquote}{\kern0pt}\ {\isacharparenleft}{\kern0pt}{\isacharbraceleft}{\kern0pt}{\isadigit{0}}{\isachardot}{\kern0pt}{\isachardot}{\kern0pt}{\isacharless}{\kern0pt}s\isactrlsub {\isadigit{1}}{\isacharbraceright}{\kern0pt}\ {\isasymtimes}\ {\isacharbraceleft}{\kern0pt}{\isadigit{0}}{\isachardot}{\kern0pt}{\isachardot}{\kern0pt}{\isacharless}{\kern0pt}s\isactrlsub {\isadigit{2}}{\isacharbraceright}{\kern0pt}{\isacharparenright}{\kern0pt}{\isachardoublequoteclose}\isanewline
\ \ \ \ \isacommand{have}\isamarkupfalse%
\ h{\isacharunderscore}{\kern0pt}{\isadigit{1}}{\isacharcolon}{\kern0pt}\ {\isachardoublequoteopen}fst\ x\ {\isasymle}\ {\isadigit{0}}{\isachardoublequoteclose}\ \isacommand{using}\isamarkupfalse%
\ h{\isacharunderscore}{\kern0pt}a\ \isacommand{by}\isamarkupfalse%
\ force\isanewline
\ \ \ \ \isacommand{have}\isamarkupfalse%
\ h{\isacharunderscore}{\kern0pt}{\isadigit{2}}{\isacharcolon}{\kern0pt}\ {\isachardoublequoteopen}snd\ x\ {\isasymle}\ {\isadigit{0}}{\isachardoublequoteclose}\ \isacommand{using}\isamarkupfalse%
\ h{\isacharunderscore}{\kern0pt}a\ \isacommand{by}\isamarkupfalse%
\ force\isanewline
\ \ \ \ \isanewline
\ \ \ \ \isacommand{have}\isamarkupfalse%
\ {\isachardoublequoteopen}bit{\isacharunderscore}{\kern0pt}count\ \ {\isacharparenleft}{\kern0pt}{\isacharparenleft}{\kern0pt}N\isactrlsub S\ {\isasymtimes}\isactrlsub S\ N\isactrlsub S{\isacharparenright}{\kern0pt}\ x{\isacharparenright}{\kern0pt}\ {\isasymle}\ \ ereal\ {\isacharparenleft}{\kern0pt}{\isadigit{2}}\ {\isacharasterisk}{\kern0pt}\ log\ {\isadigit{2}}\ {\isacharparenleft}{\kern0pt}{\isadigit{1}}\ {\isacharplus}{\kern0pt}\ real\ {\isadigit{0}}{\isacharparenright}{\kern0pt}\ {\isacharplus}{\kern0pt}\ {\isadigit{1}}{\isacharparenright}{\kern0pt}\ {\isacharplus}{\kern0pt}\ \ ereal\ {\isacharparenleft}{\kern0pt}{\isadigit{2}}\ {\isacharasterisk}{\kern0pt}\ log\ {\isadigit{2}}\ {\isacharparenleft}{\kern0pt}{\isadigit{1}}\ {\isacharplus}{\kern0pt}\ real\ {\isadigit{0}}{\isacharparenright}{\kern0pt}\ {\isacharplus}{\kern0pt}\ {\isadigit{1}}{\isacharparenright}{\kern0pt}{\isachardoublequoteclose}\isanewline
\ \ \ \ \ \ \isacommand{apply}\isamarkupfalse%
\ {\isacharparenleft}{\kern0pt}subst\ prod{\isacharunderscore}{\kern0pt}bit{\isacharunderscore}{\kern0pt}count{\isacharunderscore}{\kern0pt}{\isadigit{2}}{\isacharparenright}{\kern0pt}\isanewline
\ \ \ \ \ \ \isacommand{apply}\isamarkupfalse%
\ {\isacharparenleft}{\kern0pt}rule\ add{\isacharunderscore}{\kern0pt}mono{\isacharparenright}{\kern0pt}\isanewline
\ \ \ \ \ \ \ \isacommand{apply}\isamarkupfalse%
\ {\isacharparenleft}{\kern0pt}rule\ nat{\isacharunderscore}{\kern0pt}bit{\isacharunderscore}{\kern0pt}count{\isacharunderscore}{\kern0pt}est{\isacharcomma}{\kern0pt}\ rule\ h{\isacharunderscore}{\kern0pt}{\isadigit{1}}{\isacharparenright}{\kern0pt}\isanewline
\ \ \ \ \ \ \isacommand{by}\isamarkupfalse%
\ {\isacharparenleft}{\kern0pt}rule\ nat{\isacharunderscore}{\kern0pt}bit{\isacharunderscore}{\kern0pt}count{\isacharunderscore}{\kern0pt}est{\isacharcomma}{\kern0pt}\ rule\ h{\isacharunderscore}{\kern0pt}{\isadigit{2}}{\isacharparenright}{\kern0pt}\isanewline
\ \ \ \ \isacommand{also}\isamarkupfalse%
\ \isacommand{have}\isamarkupfalse%
\ {\isachardoublequoteopen}{\isachardot}{\kern0pt}{\isachardot}{\kern0pt}{\isachardot}{\kern0pt}\ {\isacharequal}{\kern0pt}\ {\isadigit{2}}{\isachardoublequoteclose}\isanewline
\ \ \ \ \ \ \isacommand{by}\isamarkupfalse%
\ simp\isanewline
\ \ \ \ \isacommand{finally}\isamarkupfalse%
\ \isacommand{show}\isamarkupfalse%
\ {\isachardoublequoteopen}bit{\isacharunderscore}{\kern0pt}count\ \ {\isacharparenleft}{\kern0pt}{\isacharparenleft}{\kern0pt}N\isactrlsub S\ {\isasymtimes}\isactrlsub S\ N\isactrlsub S{\isacharparenright}{\kern0pt}\ x{\isacharparenright}{\kern0pt}\ {\isasymle}\ {\isadigit{2}}{\isachardoublequoteclose}\ \isacommand{by}\isamarkupfalse%
\ simp\isanewline
\ \ \isacommand{qed}\isamarkupfalse%
\isanewline
\isanewline
\ \ \isacommand{have}\isamarkupfalse%
\ {\isachardoublequoteopen}bit{\isacharunderscore}{\kern0pt}count\ {\isacharparenleft}{\kern0pt}N\isactrlsub S\ s\isactrlsub {\isadigit{1}}{\isacharparenright}{\kern0pt}\ {\isacharplus}{\kern0pt}\ bit{\isacharunderscore}{\kern0pt}count\ {\isacharparenleft}{\kern0pt}N\isactrlsub S\ s\isactrlsub {\isadigit{2}}{\isacharparenright}{\kern0pt}\ {\isacharplus}{\kern0pt}\ bit{\isacharunderscore}{\kern0pt}count\ {\isacharparenleft}{\kern0pt}N\isactrlsub S\ k{\isacharparenright}{\kern0pt}\ {\isacharplus}{\kern0pt}\ bit{\isacharunderscore}{\kern0pt}count\ {\isacharparenleft}{\kern0pt}N\isactrlsub S\ {\isadigit{0}}{\isacharparenright}{\kern0pt}\ {\isacharplus}{\kern0pt}\ \isanewline
\ \ \ \ bit{\isacharunderscore}{\kern0pt}count\ {\isacharparenleft}{\kern0pt}{\isacharparenleft}{\kern0pt}List{\isachardot}{\kern0pt}product\ {\isacharbrackleft}{\kern0pt}{\isadigit{0}}{\isachardot}{\kern0pt}{\isachardot}{\kern0pt}{\isacharless}{\kern0pt}s\isactrlsub {\isadigit{1}}{\isacharbrackright}{\kern0pt}\ {\isacharbrackleft}{\kern0pt}{\isadigit{0}}{\isachardot}{\kern0pt}{\isachardot}{\kern0pt}{\isacharless}{\kern0pt}s\isactrlsub {\isadigit{2}}{\isacharbrackright}{\kern0pt}\ {\isasymrightarrow}\isactrlsub S\ N\isactrlsub S\ {\isasymtimes}\isactrlsub S\ N\isactrlsub S{\isacharparenright}{\kern0pt}\ {\isacharparenleft}{\kern0pt}{\isasymlambda}{\isacharunderscore}{\kern0pt}{\isasymin}{\isacharbraceleft}{\kern0pt}{\isadigit{0}}{\isachardot}{\kern0pt}{\isachardot}{\kern0pt}{\isacharless}{\kern0pt}s\isactrlsub {\isadigit{1}}{\isacharbraceright}{\kern0pt}\ {\isasymtimes}\ {\isacharbraceleft}{\kern0pt}{\isadigit{0}}{\isachardot}{\kern0pt}{\isachardot}{\kern0pt}{\isacharless}{\kern0pt}s\isactrlsub {\isadigit{2}}{\isacharbraceright}{\kern0pt}{\isachardot}{\kern0pt}\ {\isacharparenleft}{\kern0pt}{\isadigit{0}}{\isacharcomma}{\kern0pt}\ {\isadigit{0}}{\isacharparenright}{\kern0pt}{\isacharparenright}{\kern0pt}{\isacharparenright}{\kern0pt}\isanewline
\ \ \ \ {\isasymle}\ ereal\ {\isacharparenleft}{\kern0pt}{\isadigit{2}}\ {\isacharasterisk}{\kern0pt}\ log\ {\isadigit{2}}\ {\isacharparenleft}{\kern0pt}real\ s\isactrlsub {\isadigit{1}}\ {\isacharplus}{\kern0pt}\ {\isadigit{1}}{\isacharparenright}{\kern0pt}\ {\isacharplus}{\kern0pt}\ {\isadigit{1}}{\isacharparenright}{\kern0pt}\ {\isacharplus}{\kern0pt}\ ereal\ {\isacharparenleft}{\kern0pt}{\isadigit{2}}\ {\isacharasterisk}{\kern0pt}\ log\ {\isadigit{2}}\ {\isacharparenleft}{\kern0pt}real\ s\isactrlsub {\isadigit{2}}\ {\isacharplus}{\kern0pt}\ {\isadigit{1}}{\isacharparenright}{\kern0pt}\ {\isacharplus}{\kern0pt}\ {\isadigit{1}}{\isacharparenright}{\kern0pt}\ {\isacharplus}{\kern0pt}\ \isanewline
\ \ \ \ ereal\ {\isacharparenleft}{\kern0pt}{\isadigit{2}}\ {\isacharasterisk}{\kern0pt}\ log\ {\isadigit{2}}\ {\isacharparenleft}{\kern0pt}real\ k\ {\isacharplus}{\kern0pt}\ {\isadigit{1}}{\isacharparenright}{\kern0pt}\ {\isacharplus}{\kern0pt}\ {\isadigit{1}}{\isacharparenright}{\kern0pt}\ {\isacharplus}{\kern0pt}\ ereal\ {\isacharparenleft}{\kern0pt}{\isadigit{2}}\ {\isacharasterisk}{\kern0pt}\ log\ {\isadigit{2}}\ {\isacharparenleft}{\kern0pt}real\ {\isadigit{0}}\ {\isacharplus}{\kern0pt}\ {\isadigit{1}}{\isacharparenright}{\kern0pt}\ {\isacharplus}{\kern0pt}\ {\isadigit{1}}{\isacharparenright}{\kern0pt}\ {\isacharplus}{\kern0pt}\ \isanewline
\ \ \ {\isacharparenleft}{\kern0pt}ereal\ {\isacharparenleft}{\kern0pt}real\ s\isactrlsub {\isadigit{1}}\ {\isacharasterisk}{\kern0pt}\ real\ s\isactrlsub {\isadigit{2}}{\isacharparenright}{\kern0pt}\ {\isacharasterisk}{\kern0pt}\ {\isacharparenleft}{\kern0pt}w\ {\isacharplus}{\kern0pt}\ {\isadigit{1}}{\isacharparenright}{\kern0pt}\ {\isacharplus}{\kern0pt}\ {\isadigit{1}}{\isacharparenright}{\kern0pt}{\isachardoublequoteclose}\isanewline
\ \ \ \ \isacommand{apply}\isamarkupfalse%
\ {\isacharparenleft}{\kern0pt}rule\ add{\isacharunderscore}{\kern0pt}mono{\isacharparenright}{\kern0pt}\isanewline
\ \ \ \ \isacommand{apply}\isamarkupfalse%
\ {\isacharparenleft}{\kern0pt}rule\ add{\isacharunderscore}{\kern0pt}mono{\isacharparenright}{\kern0pt}\isanewline
\ \ \ \ \isacommand{apply}\isamarkupfalse%
\ {\isacharparenleft}{\kern0pt}rule\ add{\isacharunderscore}{\kern0pt}mono{\isacharparenright}{\kern0pt}\isanewline
\ \ \ \ \ \ \ \isacommand{apply}\isamarkupfalse%
\ {\isacharparenleft}{\kern0pt}rule\ add{\isacharunderscore}{\kern0pt}mono{\isacharcomma}{\kern0pt}\ rule\ nat{\isacharunderscore}{\kern0pt}bit{\isacharunderscore}{\kern0pt}count{\isacharparenright}{\kern0pt}\isanewline
\ \ \ \ \ \ \isacommand{apply}\isamarkupfalse%
\ {\isacharparenleft}{\kern0pt}rule\ nat{\isacharunderscore}{\kern0pt}bit{\isacharunderscore}{\kern0pt}count{\isacharparenright}{\kern0pt}\isanewline
\ \ \ \ \ \isacommand{apply}\isamarkupfalse%
\ {\isacharparenleft}{\kern0pt}rule\ nat{\isacharunderscore}{\kern0pt}bit{\isacharunderscore}{\kern0pt}count{\isacharparenright}{\kern0pt}\isanewline
\ \ \ \ \ \isacommand{apply}\isamarkupfalse%
\ {\isacharparenleft}{\kern0pt}rule\ nat{\isacharunderscore}{\kern0pt}bit{\isacharunderscore}{\kern0pt}count{\isacharparenright}{\kern0pt}\isanewline
\ \ \ \ \isacommand{apply}\isamarkupfalse%
\ {\isacharparenleft}{\kern0pt}simp\ add{\isacharcolon}{\kern0pt}fun\isactrlsub S{\isacharunderscore}{\kern0pt}def{\isacharparenright}{\kern0pt}\isanewline
\ \ \ \ \isacommand{apply}\isamarkupfalse%
\ {\isacharparenleft}{\kern0pt}rule\ list{\isacharunderscore}{\kern0pt}bit{\isacharunderscore}{\kern0pt}count{\isacharunderscore}{\kern0pt}est{\isacharbrackleft}{\kern0pt}\isakeyword{where}\ xs{\isacharequal}{\kern0pt}{\isachardoublequoteopen}map\ {\isacharparenleft}{\kern0pt}{\isasymlambda}{\isacharunderscore}{\kern0pt}{\isasymin}{\isacharbraceleft}{\kern0pt}{\isadigit{0}}{\isachardot}{\kern0pt}{\isachardot}{\kern0pt}{\isacharless}{\kern0pt}s\isactrlsub {\isadigit{1}}{\isacharbraceright}{\kern0pt}\ {\isasymtimes}\ {\isacharbraceleft}{\kern0pt}{\isadigit{0}}{\isachardot}{\kern0pt}{\isachardot}{\kern0pt}{\isacharless}{\kern0pt}s\isactrlsub {\isadigit{2}}{\isacharbraceright}{\kern0pt}{\isachardot}{\kern0pt}\ {\isacharparenleft}{\kern0pt}{\isadigit{0}}{\isacharcomma}{\kern0pt}\ {\isadigit{0}}{\isacharparenright}{\kern0pt}{\isacharparenright}{\kern0pt}\ {\isacharparenleft}{\kern0pt}List{\isachardot}{\kern0pt}product\ {\isacharbrackleft}{\kern0pt}{\isadigit{0}}{\isachardot}{\kern0pt}{\isachardot}{\kern0pt}{\isacharless}{\kern0pt}s\isactrlsub {\isadigit{1}}{\isacharbrackright}{\kern0pt}\ {\isacharbrackleft}{\kern0pt}{\isadigit{0}}{\isachardot}{\kern0pt}{\isachardot}{\kern0pt}{\isacharless}{\kern0pt}s\isactrlsub {\isadigit{2}}{\isacharbrackright}{\kern0pt}{\isacharparenright}{\kern0pt}{\isachardoublequoteclose}{\isacharcomma}{\kern0pt}\ simplified{\isacharbrackright}{\kern0pt}{\isacharparenright}{\kern0pt}\isanewline
\ \ \ \ \isacommand{by}\isamarkupfalse%
\ {\isacharparenleft}{\kern0pt}subst\ w{\isacharunderscore}{\kern0pt}def{\isacharcomma}{\kern0pt}\ metis\ h{\isacharparenright}{\kern0pt}\isanewline
\ \ \isacommand{also}\isamarkupfalse%
\ \isacommand{have}\isamarkupfalse%
\ {\isachardoublequoteopen}{\isachardot}{\kern0pt}{\isachardot}{\kern0pt}{\isachardot}{\kern0pt}\ {\isasymle}\ ereal\ {\isacharparenleft}{\kern0pt}fk{\isacharunderscore}{\kern0pt}space{\isacharunderscore}{\kern0pt}usage\ {\isacharparenleft}{\kern0pt}k{\isacharcomma}{\kern0pt}\ n{\isacharcomma}{\kern0pt}\ length\ as{\isacharcomma}{\kern0pt}\ {\isasymepsilon}{\isacharcomma}{\kern0pt}\ {\isasymdelta}{\isacharparenright}{\kern0pt}{\isacharparenright}{\kern0pt}{\isachardoublequoteclose}\ \isanewline
\ \ \ \ \isacommand{apply}\isamarkupfalse%
\ {\isacharparenleft}{\kern0pt}simp\ add{\isacharcolon}{\kern0pt}s\isactrlsub {\isadigit{1}}{\isacharunderscore}{\kern0pt}def{\isacharbrackleft}{\kern0pt}symmetric{\isacharbrackright}{\kern0pt}\ s\isactrlsub {\isadigit{2}}{\isacharunderscore}{\kern0pt}def{\isacharbrackleft}{\kern0pt}symmetric{\isacharbrackright}{\kern0pt}\ w{\isacharunderscore}{\kern0pt}def\ True{\isacharparenright}{\kern0pt}\isanewline
\ \ \ \ \isacommand{apply}\isamarkupfalse%
\ {\isacharparenleft}{\kern0pt}rule\ mult{\isacharunderscore}{\kern0pt}left{\isacharunderscore}{\kern0pt}mono{\isacharparenright}{\kern0pt}\ \isanewline
\ \ \ \ \isacommand{by}\isamarkupfalse%
\ simp{\isacharplus}{\kern0pt}\isanewline
\ \ \isacommand{finally}\isamarkupfalse%
\ \isacommand{have}\isamarkupfalse%
\ {\isachardoublequoteopen}bit{\isacharunderscore}{\kern0pt}count\ {\isacharparenleft}{\kern0pt}N\isactrlsub S\ s\isactrlsub {\isadigit{1}}{\isacharparenright}{\kern0pt}\ {\isacharplus}{\kern0pt}\ {\isacharparenleft}{\kern0pt}bit{\isacharunderscore}{\kern0pt}count\ {\isacharparenleft}{\kern0pt}N\isactrlsub S\ s\isactrlsub {\isadigit{2}}{\isacharparenright}{\kern0pt}\ {\isacharplus}{\kern0pt}\ {\isacharparenleft}{\kern0pt}bit{\isacharunderscore}{\kern0pt}count\ {\isacharparenleft}{\kern0pt}N\isactrlsub S\ k{\isacharparenright}{\kern0pt}\ {\isacharplus}{\kern0pt}\ {\isacharparenleft}{\kern0pt}bit{\isacharunderscore}{\kern0pt}count\ {\isacharparenleft}{\kern0pt}N\isactrlsub S\ {\isadigit{0}}{\isacharparenright}{\kern0pt}\ {\isacharplus}{\kern0pt}\isanewline
\ \ \ \ bit{\isacharunderscore}{\kern0pt}count\ {\isacharparenleft}{\kern0pt}{\isacharparenleft}{\kern0pt}List{\isachardot}{\kern0pt}product\ {\isacharbrackleft}{\kern0pt}{\isadigit{0}}{\isachardot}{\kern0pt}{\isachardot}{\kern0pt}{\isacharless}{\kern0pt}s\isactrlsub {\isadigit{1}}{\isacharbrackright}{\kern0pt}\ {\isacharbrackleft}{\kern0pt}{\isadigit{0}}{\isachardot}{\kern0pt}{\isachardot}{\kern0pt}{\isacharless}{\kern0pt}s\isactrlsub {\isadigit{2}}{\isacharbrackright}{\kern0pt}\ {\isasymrightarrow}\isactrlsub S\ N\isactrlsub S\ {\isasymtimes}\isactrlsub S\ N\isactrlsub S{\isacharparenright}{\kern0pt}\ {\isacharparenleft}{\kern0pt}{\isasymlambda}{\isacharunderscore}{\kern0pt}{\isasymin}{\isacharbraceleft}{\kern0pt}{\isadigit{0}}{\isachardot}{\kern0pt}{\isachardot}{\kern0pt}{\isacharless}{\kern0pt}s\isactrlsub {\isadigit{1}}{\isacharbraceright}{\kern0pt}\ {\isasymtimes}\ {\isacharbraceleft}{\kern0pt}{\isadigit{0}}{\isachardot}{\kern0pt}{\isachardot}{\kern0pt}{\isacharless}{\kern0pt}s\isactrlsub {\isadigit{2}}{\isacharbraceright}{\kern0pt}{\isachardot}{\kern0pt}\ {\isacharparenleft}{\kern0pt}{\isadigit{0}}{\isacharcomma}{\kern0pt}\ {\isadigit{0}}{\isacharparenright}{\kern0pt}{\isacharparenright}{\kern0pt}{\isacharparenright}{\kern0pt}{\isacharparenright}{\kern0pt}{\isacharparenright}{\kern0pt}{\isacharparenright}{\kern0pt}\isanewline
\ \ \ \ {\isasymle}\ ereal\ {\isacharparenleft}{\kern0pt}fk{\isacharunderscore}{\kern0pt}space{\isacharunderscore}{\kern0pt}usage\ {\isacharparenleft}{\kern0pt}k{\isacharcomma}{\kern0pt}\ n{\isacharcomma}{\kern0pt}\ length\ as{\isacharcomma}{\kern0pt}\ {\isasymepsilon}{\isacharcomma}{\kern0pt}\ {\isasymdelta}{\isacharparenright}{\kern0pt}{\isacharparenright}{\kern0pt}{\isachardoublequoteclose}\isanewline
\ \ \ \ \isacommand{by}\isamarkupfalse%
\ {\isacharparenleft}{\kern0pt}simp\ add{\isacharcolon}{\kern0pt}add{\isachardot}{\kern0pt}assoc\ del{\isacharcolon}{\kern0pt}fk{\isacharunderscore}{\kern0pt}space{\isacharunderscore}{\kern0pt}usage{\isachardot}{\kern0pt}simps\ N\isactrlsub S{\isachardot}{\kern0pt}simps{\isacharparenright}{\kern0pt}\isanewline
\ \ \isacommand{thus}\isamarkupfalse%
\ {\isacharquery}{\kern0pt}thesis\ \isanewline
\ \ \ \ \isacommand{by}\isamarkupfalse%
\ {\isacharparenleft}{\kern0pt}simp\ add{\isacharcolon}{\kern0pt}\ a\ Let{\isacharunderscore}{\kern0pt}def\ s\isactrlsub {\isadigit{1}}{\isacharunderscore}{\kern0pt}def\ s\isactrlsub {\isadigit{2}}{\isacharunderscore}{\kern0pt}def\ encode{\isacharunderscore}{\kern0pt}fk{\isacharunderscore}{\kern0pt}state{\isacharunderscore}{\kern0pt}def\ \ AE{\isacharunderscore}{\kern0pt}measure{\isacharunderscore}{\kern0pt}pmf{\isacharunderscore}{\kern0pt}iff\ dependent{\isacharunderscore}{\kern0pt}bit{\isacharunderscore}{\kern0pt}count\ prod{\isacharunderscore}{\kern0pt}bit{\isacharunderscore}{\kern0pt}count\isanewline
\ \ \ \ \ \ \ \ del{\isacharcolon}{\kern0pt}fk{\isacharunderscore}{\kern0pt}space{\isacharunderscore}{\kern0pt}usage{\isachardot}{\kern0pt}simps\ N\isactrlsub S{\isachardot}{\kern0pt}simps\ encode{\isacharunderscore}{\kern0pt}prod{\isachardot}{\kern0pt}simps\ encode{\isacharunderscore}{\kern0pt}dependent{\isacharunderscore}{\kern0pt}sum{\isachardot}{\kern0pt}simps{\isacharparenright}{\kern0pt}\ \isanewline
\isacommand{next}\isamarkupfalse%
\isanewline
\ \ \isacommand{case}\isamarkupfalse%
\ False\isanewline
\ \ \isacommand{define}\isamarkupfalse%
\ s\isactrlsub {\isadigit{1}}\ \isakeyword{where}\ {\isachardoublequoteopen}s\isactrlsub {\isadigit{1}}\ {\isacharequal}{\kern0pt}\ nat\ {\isasymlceil}{\isadigit{3}}{\isacharasterisk}{\kern0pt}real\ k{\isacharasterisk}{\kern0pt}{\isacharparenleft}{\kern0pt}real\ n{\isacharparenright}{\kern0pt}\ powr\ {\isacharparenleft}{\kern0pt}{\isadigit{1}}{\isacharminus}{\kern0pt}{\isadigit{1}}{\isacharslash}{\kern0pt}\ real\ k{\isacharparenright}{\kern0pt}{\isacharslash}{\kern0pt}\ {\isacharparenleft}{\kern0pt}real{\isacharunderscore}{\kern0pt}of{\isacharunderscore}{\kern0pt}rat\ {\isasymdelta}{\isacharparenright}{\kern0pt}\isactrlsup {\isadigit{2}}{\isasymrceil}{\isachardoublequoteclose}\isanewline
\ \ \isacommand{define}\isamarkupfalse%
\ s\isactrlsub {\isadigit{2}}\ \isakeyword{where}\ {\isachardoublequoteopen}s\isactrlsub {\isadigit{2}}\ {\isacharequal}{\kern0pt}\ nat\ {\isasymlceil}{\isacharminus}{\kern0pt}{\isacharparenleft}{\kern0pt}{\isadigit{1}}{\isadigit{8}}\ {\isacharasterisk}{\kern0pt}\ ln\ {\isacharparenleft}{\kern0pt}real{\isacharunderscore}{\kern0pt}of{\isacharunderscore}{\kern0pt}rat\ {\isasymepsilon}{\isacharparenright}{\kern0pt}{\isacharparenright}{\kern0pt}{\isasymrceil}{\isachardoublequoteclose}\isanewline
\isanewline
\ \ \isacommand{have}\isamarkupfalse%
\ a{\isacharcolon}{\kern0pt}{\isachardoublequoteopen}M\ {\isacharequal}{\kern0pt}\ map{\isacharunderscore}{\kern0pt}pmf\ {\isacharparenleft}{\kern0pt}{\isasymlambda}x{\isachardot}{\kern0pt}\ {\isacharparenleft}{\kern0pt}s\isactrlsub {\isadigit{1}}{\isacharcomma}{\kern0pt}s\isactrlsub {\isadigit{2}}{\isacharcomma}{\kern0pt}k{\isacharcomma}{\kern0pt}length\ as{\isacharcomma}{\kern0pt}\ x{\isacharparenright}{\kern0pt}{\isacharparenright}{\kern0pt}\isanewline
\ \ \ \ {\isacharparenleft}{\kern0pt}prod{\isacharunderscore}{\kern0pt}pmf\ {\isacharparenleft}{\kern0pt}{\isacharbraceleft}{\kern0pt}{\isadigit{0}}{\isachardot}{\kern0pt}{\isachardot}{\kern0pt}{\isacharless}{\kern0pt}s\isactrlsub {\isadigit{1}}{\isacharbraceright}{\kern0pt}\ {\isasymtimes}\ {\isacharbraceleft}{\kern0pt}{\isadigit{0}}{\isachardot}{\kern0pt}{\isachardot}{\kern0pt}{\isacharless}{\kern0pt}s\isactrlsub {\isadigit{2}}{\isacharbraceright}{\kern0pt}{\isacharparenright}{\kern0pt}\ {\isacharparenleft}{\kern0pt}{\isasymlambda}{\isacharunderscore}{\kern0pt}{\isachardot}{\kern0pt}\ pmf{\isacharunderscore}{\kern0pt}of{\isacharunderscore}{\kern0pt}set\ {\isacharbraceleft}{\kern0pt}{\isacharparenleft}{\kern0pt}u{\isacharcomma}{\kern0pt}v{\isacharparenright}{\kern0pt}{\isachardot}{\kern0pt}\ v\ {\isacharless}{\kern0pt}\ count{\isacharunderscore}{\kern0pt}list\ as\ u{\isacharbraceright}{\kern0pt}{\isacharparenright}{\kern0pt}{\isacharparenright}{\kern0pt}{\isachardoublequoteclose}\isanewline
\ \ \ \ \isacommand{apply}\isamarkupfalse%
\ {\isacharparenleft}{\kern0pt}subst\ M{\isacharunderscore}{\kern0pt}def{\isacharparenright}{\kern0pt}\isanewline
\ \ \ \ \isacommand{apply}\isamarkupfalse%
\ {\isacharparenleft}{\kern0pt}subst\ fk{\isacharunderscore}{\kern0pt}alg{\isacharunderscore}{\kern0pt}sketch{\isacharbrackleft}{\kern0pt}OF\ assms{\isacharparenleft}{\kern0pt}{\isadigit{1}}{\isacharparenright}{\kern0pt}\ assms{\isacharparenleft}{\kern0pt}{\isadigit{3}}{\isacharparenright}{\kern0pt}\ assms{\isacharparenleft}{\kern0pt}{\isadigit{4}}{\isacharparenright}{\kern0pt}\ False{\isacharbrackright}{\kern0pt}{\isacharparenright}{\kern0pt}\isanewline
\ \ \ \ \isacommand{by}\isamarkupfalse%
\ {\isacharparenleft}{\kern0pt}simp\ add{\isacharcolon}{\kern0pt}s\isactrlsub {\isadigit{1}}{\isacharunderscore}{\kern0pt}def{\isacharbrackleft}{\kern0pt}symmetric{\isacharbrackright}{\kern0pt}\ s\isactrlsub {\isadigit{2}}{\isacharunderscore}{\kern0pt}def{\isacharbrackleft}{\kern0pt}symmetric{\isacharbrackright}{\kern0pt}{\isacharparenright}{\kern0pt}\isanewline
\isanewline
\ \ \isacommand{have}\isamarkupfalse%
\ {\isachardoublequoteopen}set\ as\ {\isasymnoteq}\ {\isacharbraceleft}{\kern0pt}{\isacharbraceright}{\kern0pt}{\isachardoublequoteclose}\ \isacommand{using}\isamarkupfalse%
\ assms\ False\ \isacommand{by}\isamarkupfalse%
\ blast\isanewline
\ \ \isacommand{hence}\isamarkupfalse%
\ n{\isacharunderscore}{\kern0pt}nonzero{\isacharcolon}{\kern0pt}\ {\isachardoublequoteopen}n\ {\isachargreater}{\kern0pt}\ {\isadigit{0}}{\isachardoublequoteclose}\ \isacommand{using}\isamarkupfalse%
\ assms{\isacharparenleft}{\kern0pt}{\isadigit{4}}{\isacharparenright}{\kern0pt}\ \isacommand{by}\isamarkupfalse%
\ fastforce\isanewline
\ \ \isacommand{have}\isamarkupfalse%
\ length{\isacharunderscore}{\kern0pt}xs{\isacharunderscore}{\kern0pt}gr{\isacharunderscore}{\kern0pt}{\isadigit{0}}{\isacharcolon}{\kern0pt}\ {\isachardoublequoteopen}length\ as\ {\isachargreater}{\kern0pt}\ {\isadigit{0}}{\isachardoublequoteclose}\ \isacommand{using}\isamarkupfalse%
\ False\ \isacommand{by}\isamarkupfalse%
\ blast\isanewline
\isanewline
\ \ \isacommand{have}\isamarkupfalse%
\ b{\isacharcolon}{\kern0pt}{\isachardoublequoteopen}{\isasymAnd}y{\isachardot}{\kern0pt}\ y{\isasymin}{\isacharbraceleft}{\kern0pt}{\isadigit{0}}{\isachardot}{\kern0pt}{\isachardot}{\kern0pt}{\isacharless}{\kern0pt}s\isactrlsub {\isadigit{1}}{\isacharbraceright}{\kern0pt}\ {\isasymtimes}\ {\isacharbraceleft}{\kern0pt}{\isadigit{0}}{\isachardot}{\kern0pt}{\isachardot}{\kern0pt}{\isacharless}{\kern0pt}s\isactrlsub {\isadigit{2}}{\isacharbraceright}{\kern0pt}\ {\isasymrightarrow}\isactrlsub E\ {\isacharbraceleft}{\kern0pt}{\isacharparenleft}{\kern0pt}u{\isacharcomma}{\kern0pt}\ v{\isacharparenright}{\kern0pt}{\isachardot}{\kern0pt}\ v\ {\isacharless}{\kern0pt}\ count{\isacharunderscore}{\kern0pt}list\ as\ u{\isacharbraceright}{\kern0pt}\ {\isasymLongrightarrow}\isanewline
\ \ \ \ \ \ \ bit{\isacharunderscore}{\kern0pt}count\ {\isacharparenleft}{\kern0pt}encode{\isacharunderscore}{\kern0pt}fk{\isacharunderscore}{\kern0pt}state\ {\isacharparenleft}{\kern0pt}s\isactrlsub {\isadigit{1}}{\isacharcomma}{\kern0pt}\ s\isactrlsub {\isadigit{2}}{\isacharcomma}{\kern0pt}\ k{\isacharcomma}{\kern0pt}\ length\ as{\isacharcomma}{\kern0pt}\ y{\isacharparenright}{\kern0pt}{\isacharparenright}{\kern0pt}\ {\isasymle}\ {\isacharquery}{\kern0pt}rhs{\isachardoublequoteclose}\isanewline
\ \ \isacommand{proof}\isamarkupfalse%
\ {\isacharminus}{\kern0pt}\isanewline
\ \ \ \ \isacommand{fix}\isamarkupfalse%
\ y\isanewline
\ \ \ \ \isacommand{assume}\isamarkupfalse%
\ b{\isadigit{0}}{\isacharcolon}{\kern0pt}{\isachardoublequoteopen}y\ {\isasymin}\ {\isacharbraceleft}{\kern0pt}{\isadigit{0}}{\isachardot}{\kern0pt}{\isachardot}{\kern0pt}{\isacharless}{\kern0pt}s\isactrlsub {\isadigit{1}}{\isacharbraceright}{\kern0pt}\ {\isasymtimes}\ {\isacharbraceleft}{\kern0pt}{\isadigit{0}}{\isachardot}{\kern0pt}{\isachardot}{\kern0pt}{\isacharless}{\kern0pt}s\isactrlsub {\isadigit{2}}{\isacharbraceright}{\kern0pt}\ {\isasymrightarrow}\isactrlsub E\ {\isacharbraceleft}{\kern0pt}{\isacharparenleft}{\kern0pt}u{\isacharcomma}{\kern0pt}\ v{\isacharparenright}{\kern0pt}{\isachardot}{\kern0pt}\ v\ {\isacharless}{\kern0pt}\ count{\isacharunderscore}{\kern0pt}list\ as\ u{\isacharbraceright}{\kern0pt}{\isachardoublequoteclose}\isanewline
\ \ \ \ \isacommand{have}\isamarkupfalse%
\ {\isachardoublequoteopen}{\isasymAnd}x{\isachardot}{\kern0pt}\ x\ {\isasymin}\ y\ {\isacharbackquote}{\kern0pt}\ {\isacharparenleft}{\kern0pt}{\isacharbraceleft}{\kern0pt}{\isadigit{0}}{\isachardot}{\kern0pt}{\isachardot}{\kern0pt}{\isacharless}{\kern0pt}s\isactrlsub {\isadigit{1}}{\isacharbraceright}{\kern0pt}\ {\isasymtimes}\ {\isacharbraceleft}{\kern0pt}{\isadigit{0}}{\isachardot}{\kern0pt}{\isachardot}{\kern0pt}{\isacharless}{\kern0pt}s\isactrlsub {\isadigit{2}}{\isacharbraceright}{\kern0pt}{\isacharparenright}{\kern0pt}\ {\isasymLongrightarrow}\ {\isadigit{1}}\ {\isasymle}\ count{\isacharunderscore}{\kern0pt}list\ as\ {\isacharparenleft}{\kern0pt}fst\ x{\isacharparenright}{\kern0pt}{\isachardoublequoteclose}\isanewline
\ \ \ \ \ \ \isacommand{using}\isamarkupfalse%
\ b{\isadigit{0}}\ \isacommand{by}\isamarkupfalse%
\ {\isacharparenleft}{\kern0pt}simp\ add{\isacharcolon}{\kern0pt}PiE{\isacharunderscore}{\kern0pt}iff\ case{\isacharunderscore}{\kern0pt}prod{\isacharunderscore}{\kern0pt}beta{\isacharcomma}{\kern0pt}\ fastforce{\isacharparenright}{\kern0pt}\isanewline
\ \ \ \ \isacommand{hence}\isamarkupfalse%
\ b{\isadigit{1}}{\isacharcolon}{\kern0pt}{\isachardoublequoteopen}{\isasymAnd}x{\isachardot}{\kern0pt}\ x\ {\isasymin}\ y\ {\isacharbackquote}{\kern0pt}\ {\isacharparenleft}{\kern0pt}{\isacharbraceleft}{\kern0pt}{\isadigit{0}}{\isachardot}{\kern0pt}{\isachardot}{\kern0pt}{\isacharless}{\kern0pt}s\isactrlsub {\isadigit{1}}{\isacharbraceright}{\kern0pt}\ {\isasymtimes}\ {\isacharbraceleft}{\kern0pt}{\isadigit{0}}{\isachardot}{\kern0pt}{\isachardot}{\kern0pt}{\isacharless}{\kern0pt}s\isactrlsub {\isadigit{2}}{\isacharbraceright}{\kern0pt}{\isacharparenright}{\kern0pt}\ {\isasymLongrightarrow}\ fst\ x\ {\isasymle}\ n{\isachardoublequoteclose}\isanewline
\ \ \ \ \ \ \isacommand{by}\isamarkupfalse%
\ {\isacharparenleft}{\kern0pt}metis\ assms{\isacharparenleft}{\kern0pt}{\isadigit{4}}{\isacharparenright}{\kern0pt}\ atLeastLessThan{\isacharunderscore}{\kern0pt}iff\ count{\isacharunderscore}{\kern0pt}notin\ in{\isacharunderscore}{\kern0pt}mono\ less{\isacharunderscore}{\kern0pt}or{\isacharunderscore}{\kern0pt}eq{\isacharunderscore}{\kern0pt}imp{\isacharunderscore}{\kern0pt}le\ not{\isacharunderscore}{\kern0pt}one{\isacharunderscore}{\kern0pt}le{\isacharunderscore}{\kern0pt}zero{\isacharparenright}{\kern0pt}\isanewline
\ \ \ \ \isacommand{have}\isamarkupfalse%
\ b{\isadigit{2}}{\isacharcolon}{\kern0pt}\ {\isachardoublequoteopen}{\isasymAnd}x{\isachardot}{\kern0pt}\ x\ {\isasymin}\ y\ {\isacharbackquote}{\kern0pt}\ {\isacharparenleft}{\kern0pt}{\isacharbraceleft}{\kern0pt}{\isadigit{0}}{\isachardot}{\kern0pt}{\isachardot}{\kern0pt}{\isacharless}{\kern0pt}s\isactrlsub {\isadigit{1}}{\isacharbraceright}{\kern0pt}\ {\isasymtimes}\ {\isacharbraceleft}{\kern0pt}{\isadigit{0}}{\isachardot}{\kern0pt}{\isachardot}{\kern0pt}{\isacharless}{\kern0pt}s\isactrlsub {\isadigit{2}}{\isacharbraceright}{\kern0pt}{\isacharparenright}{\kern0pt}\ {\isasymLongrightarrow}\ snd\ x\ {\isasymle}\ length\ as{\isachardoublequoteclose}\isanewline
\ \ \ \ \ \ \isacommand{using}\isamarkupfalse%
\ count{\isacharunderscore}{\kern0pt}le{\isacharunderscore}{\kern0pt}length\ b{\isadigit{0}}\ \isacommand{apply}\isamarkupfalse%
\ {\isacharparenleft}{\kern0pt}simp\ add{\isacharcolon}{\kern0pt}PiE{\isacharunderscore}{\kern0pt}iff\ case{\isacharunderscore}{\kern0pt}prod{\isacharunderscore}{\kern0pt}beta{\isacharparenright}{\kern0pt}\ \isanewline
\ \ \ \ \ \ \isacommand{using}\isamarkupfalse%
\ dual{\isacharunderscore}{\kern0pt}order{\isachardot}{\kern0pt}strict{\isacharunderscore}{\kern0pt}trans{\isadigit{1}}\ \isacommand{by}\isamarkupfalse%
\ fastforce\isanewline
\ \ \ \ \isacommand{have}\isamarkupfalse%
\ b{\isadigit{3}}{\isacharcolon}{\kern0pt}\ {\isachardoublequoteopen}y\ {\isasymin}\ extensional\ {\isacharparenleft}{\kern0pt}{\isacharbraceleft}{\kern0pt}{\isadigit{0}}{\isachardot}{\kern0pt}{\isachardot}{\kern0pt}{\isacharless}{\kern0pt}s\isactrlsub {\isadigit{1}}{\isacharbraceright}{\kern0pt}\ {\isasymtimes}\ {\isacharbraceleft}{\kern0pt}{\isadigit{0}}{\isachardot}{\kern0pt}{\isachardot}{\kern0pt}{\isacharless}{\kern0pt}s\isactrlsub {\isadigit{2}}{\isacharbraceright}{\kern0pt}{\isacharparenright}{\kern0pt}{\isachardoublequoteclose}\ \isacommand{using}\isamarkupfalse%
\ b{\isadigit{0}}\ PiE{\isacharunderscore}{\kern0pt}iff\ \isacommand{by}\isamarkupfalse%
\ blast\isanewline
\ \ \ \ \isacommand{hence}\isamarkupfalse%
\ {\isachardoublequoteopen}bit{\isacharunderscore}{\kern0pt}count\ {\isacharparenleft}{\kern0pt}encode{\isacharunderscore}{\kern0pt}fk{\isacharunderscore}{\kern0pt}state\ {\isacharparenleft}{\kern0pt}s\isactrlsub {\isadigit{1}}{\isacharcomma}{\kern0pt}\ s\isactrlsub {\isadigit{2}}{\isacharcomma}{\kern0pt}\ k{\isacharcomma}{\kern0pt}\ length\ as{\isacharcomma}{\kern0pt}\ y{\isacharparenright}{\kern0pt}{\isacharparenright}{\kern0pt}\ {\isasymle}\ \isanewline
\ \ \ \ \ \ ereal\ {\isacharparenleft}{\kern0pt}{\isadigit{2}}\ {\isacharasterisk}{\kern0pt}\ log\ {\isadigit{2}}\ {\isacharparenleft}{\kern0pt}real\ s\isactrlsub {\isadigit{1}}\ {\isacharplus}{\kern0pt}\ {\isadigit{1}}{\isacharparenright}{\kern0pt}\ {\isacharplus}{\kern0pt}\ {\isadigit{1}}{\isacharparenright}{\kern0pt}\ {\isacharplus}{\kern0pt}\ {\isacharparenleft}{\kern0pt}\isanewline
\ \ \ \ \ \ ereal\ {\isacharparenleft}{\kern0pt}{\isadigit{2}}\ {\isacharasterisk}{\kern0pt}\ log\ {\isadigit{2}}\ {\isacharparenleft}{\kern0pt}real\ s\isactrlsub {\isadigit{2}}\ {\isacharplus}{\kern0pt}\ {\isadigit{1}}{\isacharparenright}{\kern0pt}\ {\isacharplus}{\kern0pt}\ {\isadigit{1}}{\isacharparenright}{\kern0pt}\ {\isacharplus}{\kern0pt}\ {\isacharparenleft}{\kern0pt}\ \isanewline
\ \ \ \ \ \ ereal\ {\isacharparenleft}{\kern0pt}{\isadigit{2}}\ {\isacharasterisk}{\kern0pt}\ log\ {\isadigit{2}}\ {\isacharparenleft}{\kern0pt}real\ k\ {\isacharplus}{\kern0pt}\ {\isadigit{1}}{\isacharparenright}{\kern0pt}\ {\isacharplus}{\kern0pt}\ {\isadigit{1}}{\isacharparenright}{\kern0pt}\ {\isacharplus}{\kern0pt}\ {\isacharparenleft}{\kern0pt}\isanewline
\ \ \ \ \ \ ereal\ {\isacharparenleft}{\kern0pt}{\isadigit{2}}\ {\isacharasterisk}{\kern0pt}\ log\ {\isadigit{2}}\ {\isacharparenleft}{\kern0pt}real\ {\isacharparenleft}{\kern0pt}length\ as{\isacharparenright}{\kern0pt}\ {\isacharplus}{\kern0pt}\ {\isadigit{1}}{\isacharparenright}{\kern0pt}\ {\isacharplus}{\kern0pt}\ {\isadigit{1}}{\isacharparenright}{\kern0pt}\ {\isacharplus}{\kern0pt}\ {\isacharparenleft}{\kern0pt}\isanewline
\ \ \ \ \ \ \ {\isacharparenleft}{\kern0pt}ereal\ {\isacharparenleft}{\kern0pt}real\ s\isactrlsub {\isadigit{1}}\ {\isacharasterisk}{\kern0pt}\ real\ s\isactrlsub {\isadigit{2}}{\isacharparenright}{\kern0pt}\ {\isacharasterisk}{\kern0pt}\ {\isacharparenleft}{\kern0pt}{\isacharparenleft}{\kern0pt}ereal\ {\isacharparenleft}{\kern0pt}{\isadigit{2}}\ {\isacharasterisk}{\kern0pt}\ log\ {\isadigit{2}}\ {\isacharparenleft}{\kern0pt}{\isacharparenleft}{\kern0pt}n{\isacharparenright}{\kern0pt}{\isacharplus}{\kern0pt}{\isadigit{1}}{\isacharparenright}{\kern0pt}\ {\isacharplus}{\kern0pt}\ {\isadigit{1}}{\isacharparenright}{\kern0pt}\ {\isacharplus}{\kern0pt}\ ereal\ {\isacharparenleft}{\kern0pt}{\isadigit{2}}\ {\isacharasterisk}{\kern0pt}\ log\ {\isadigit{2}}\ {\isacharparenleft}{\kern0pt}{\isacharparenleft}{\kern0pt}length\ as{\isacharparenright}{\kern0pt}{\isacharplus}{\kern0pt}{\isadigit{1}}{\isacharparenright}{\kern0pt}\ {\isacharplus}{\kern0pt}\ {\isadigit{1}}{\isacharparenright}{\kern0pt}{\isacharparenright}{\kern0pt}\ {\isacharplus}{\kern0pt}\ {\isadigit{1}}{\isacharparenright}{\kern0pt}{\isacharparenright}{\kern0pt}{\isacharplus}{\kern0pt}\ {\isadigit{1}}{\isacharparenright}{\kern0pt}{\isacharparenright}{\kern0pt}{\isacharparenright}{\kern0pt}{\isacharparenright}{\kern0pt}{\isachardoublequoteclose}\isanewline
\ \ \ \ \ \ \isacommand{apply}\isamarkupfalse%
\ {\isacharparenleft}{\kern0pt}simp\ add{\isacharcolon}{\kern0pt}encode{\isacharunderscore}{\kern0pt}fk{\isacharunderscore}{\kern0pt}state{\isacharunderscore}{\kern0pt}def\ dependent{\isacharunderscore}{\kern0pt}bit{\isacharunderscore}{\kern0pt}count\ prod{\isacharunderscore}{\kern0pt}bit{\isacharunderscore}{\kern0pt}count\ PiE{\isacharunderscore}{\kern0pt}iff\ comp{\isacharunderscore}{\kern0pt}def\ fun\isactrlsub S{\isacharunderscore}{\kern0pt}def\isanewline
\ \ \ \ \ \ \ \ \ \ del{\isacharcolon}{\kern0pt}N\isactrlsub S{\isachardot}{\kern0pt}simps\ encode{\isacharunderscore}{\kern0pt}prod{\isachardot}{\kern0pt}simps\ encode{\isacharunderscore}{\kern0pt}dependent{\isacharunderscore}{\kern0pt}sum{\isachardot}{\kern0pt}simps\ plus{\isacharunderscore}{\kern0pt}ereal{\isachardot}{\kern0pt}simps\ sum{\isacharunderscore}{\kern0pt}list{\isacharunderscore}{\kern0pt}ereal\ times{\isacharunderscore}{\kern0pt}ereal{\isachardot}{\kern0pt}simps{\isacharparenright}{\kern0pt}\isanewline
\ \ \ \ \ \ \isacommand{apply}\isamarkupfalse%
\ {\isacharparenleft}{\kern0pt}rule\ add{\isacharunderscore}{\kern0pt}mono{\isacharcomma}{\kern0pt}\ simp\ add{\isacharcolon}{\kern0pt}\ nat{\isacharunderscore}{\kern0pt}bit{\isacharunderscore}{\kern0pt}count{\isacharbrackleft}{\kern0pt}simplified{\isacharbrackright}{\kern0pt}{\isacharparenright}{\kern0pt}\isanewline
\ \ \ \ \ \ \isacommand{apply}\isamarkupfalse%
\ {\isacharparenleft}{\kern0pt}rule\ add{\isacharunderscore}{\kern0pt}mono{\isacharcomma}{\kern0pt}\ simp\ add{\isacharcolon}{\kern0pt}\ nat{\isacharunderscore}{\kern0pt}bit{\isacharunderscore}{\kern0pt}count{\isacharbrackleft}{\kern0pt}simplified{\isacharbrackright}{\kern0pt}{\isacharparenright}{\kern0pt}\isanewline
\ \ \ \ \ \ \isacommand{apply}\isamarkupfalse%
\ {\isacharparenleft}{\kern0pt}rule\ add{\isacharunderscore}{\kern0pt}mono{\isacharcomma}{\kern0pt}\ simp\ add{\isacharcolon}{\kern0pt}\ nat{\isacharunderscore}{\kern0pt}bit{\isacharunderscore}{\kern0pt}count{\isacharbrackleft}{\kern0pt}simplified{\isacharbrackright}{\kern0pt}{\isacharparenright}{\kern0pt}\isanewline
\ \ \ \ \ \ \isacommand{apply}\isamarkupfalse%
\ {\isacharparenleft}{\kern0pt}rule\ add{\isacharunderscore}{\kern0pt}mono{\isacharcomma}{\kern0pt}\ simp\ add{\isacharcolon}{\kern0pt}\ nat{\isacharunderscore}{\kern0pt}bit{\isacharunderscore}{\kern0pt}count{\isacharbrackleft}{\kern0pt}simplified{\isacharbrackright}{\kern0pt}{\isacharparenright}{\kern0pt}\isanewline
\ \ \ \ \ \ \isacommand{apply}\isamarkupfalse%
\ {\isacharparenleft}{\kern0pt}rule\ list{\isacharunderscore}{\kern0pt}bit{\isacharunderscore}{\kern0pt}count{\isacharunderscore}{\kern0pt}est{\isacharbrackleft}{\kern0pt}\isakeyword{where}\ xs{\isacharequal}{\kern0pt}{\isachardoublequoteopen}map\ y\ {\isacharparenleft}{\kern0pt}List{\isachardot}{\kern0pt}product\ {\isacharbrackleft}{\kern0pt}{\isadigit{0}}{\isachardot}{\kern0pt}{\isachardot}{\kern0pt}{\isacharless}{\kern0pt}s\isactrlsub {\isadigit{1}}{\isacharbrackright}{\kern0pt}\ {\isacharbrackleft}{\kern0pt}{\isadigit{0}}{\isachardot}{\kern0pt}{\isachardot}{\kern0pt}{\isacharless}{\kern0pt}s\isactrlsub {\isadigit{2}}{\isacharbrackright}{\kern0pt}{\isacharparenright}{\kern0pt}{\isachardoublequoteclose}{\isacharcomma}{\kern0pt}\ simplified{\isacharbrackright}{\kern0pt}{\isacharparenright}{\kern0pt}\isanewline
\ \ \ \ \ \ \isacommand{apply}\isamarkupfalse%
\ {\isacharparenleft}{\kern0pt}subst\ prod{\isacharunderscore}{\kern0pt}bit{\isacharunderscore}{\kern0pt}count{\isacharunderscore}{\kern0pt}{\isadigit{2}}{\isacharparenright}{\kern0pt}\isanewline
\ \ \ \ \ \ \isacommand{apply}\isamarkupfalse%
\ {\isacharparenleft}{\kern0pt}rule\ add{\isacharunderscore}{\kern0pt}mono{\isacharparenright}{\kern0pt}\isanewline
\ \ \ \ \ \ \isacommand{apply}\isamarkupfalse%
\ {\isacharparenleft}{\kern0pt}rule\ nat{\isacharunderscore}{\kern0pt}bit{\isacharunderscore}{\kern0pt}count{\isacharunderscore}{\kern0pt}est{\isacharcomma}{\kern0pt}\ metis\ b{\isadigit{1}}{\isacharparenright}{\kern0pt}\isanewline
\ \ \ \ \ \ \isacommand{by}\isamarkupfalse%
\ {\isacharparenleft}{\kern0pt}rule\ nat{\isacharunderscore}{\kern0pt}bit{\isacharunderscore}{\kern0pt}count{\isacharunderscore}{\kern0pt}est{\isacharcomma}{\kern0pt}\ metis\ b{\isadigit{2}}{\isacharparenright}{\kern0pt}\isanewline
\ \ \ \ \isacommand{also}\isamarkupfalse%
\ \isacommand{have}\isamarkupfalse%
\ {\isachardoublequoteopen}{\isachardot}{\kern0pt}{\isachardot}{\kern0pt}{\isachardot}{\kern0pt}\ {\isasymle}\ {\isacharquery}{\kern0pt}rhs{\isachardoublequoteclose}\isanewline
\ \ \ \ \ \ \isacommand{using}\isamarkupfalse%
\ n{\isacharunderscore}{\kern0pt}nonzero\ length{\isacharunderscore}{\kern0pt}xs{\isacharunderscore}{\kern0pt}gr{\isacharunderscore}{\kern0pt}{\isadigit{0}}\ \isacommand{apply}\isamarkupfalse%
\ {\isacharparenleft}{\kern0pt}simp\ add{\isacharcolon}{\kern0pt}\ s\isactrlsub {\isadigit{1}}{\isacharunderscore}{\kern0pt}def{\isacharbrackleft}{\kern0pt}symmetric{\isacharbrackright}{\kern0pt}\ s\isactrlsub {\isadigit{2}}{\isacharunderscore}{\kern0pt}def{\isacharbrackleft}{\kern0pt}symmetric{\isacharcomma}{\kern0pt}simplified{\isacharbrackright}{\kern0pt}{\isacharparenright}{\kern0pt}\isanewline
\ \ \ \ \ \ \isacommand{by}\isamarkupfalse%
\ {\isacharparenleft}{\kern0pt}simp\ add{\isacharcolon}{\kern0pt}algebra{\isacharunderscore}{\kern0pt}simps{\isacharparenright}{\kern0pt}\isanewline
\ \ \ \ \isacommand{finally}\isamarkupfalse%
\ \isacommand{show}\isamarkupfalse%
\ {\isachardoublequoteopen}bit{\isacharunderscore}{\kern0pt}count\ {\isacharparenleft}{\kern0pt}encode{\isacharunderscore}{\kern0pt}fk{\isacharunderscore}{\kern0pt}state\ {\isacharparenleft}{\kern0pt}s\isactrlsub {\isadigit{1}}{\isacharcomma}{\kern0pt}\ s\isactrlsub {\isadigit{2}}{\isacharcomma}{\kern0pt}\ k{\isacharcomma}{\kern0pt}\ length\ as{\isacharcomma}{\kern0pt}\ y{\isacharparenright}{\kern0pt}{\isacharparenright}{\kern0pt}\ {\isasymle}\ {\isacharquery}{\kern0pt}rhs{\isachardoublequoteclose}\isanewline
\ \ \ \ \ \ \isacommand{by}\isamarkupfalse%
\ blast\isanewline
\ \ \isacommand{qed}\isamarkupfalse%
\isanewline
\ \ \ \ \isanewline
\ \ \isacommand{show}\isamarkupfalse%
\ {\isacharquery}{\kern0pt}thesis\isanewline
\ \ \ \ \isacommand{apply}\isamarkupfalse%
\ {\isacharparenleft}{\kern0pt}simp\ add{\isacharcolon}{\kern0pt}\ a\ AE{\isacharunderscore}{\kern0pt}measure{\isacharunderscore}{\kern0pt}pmf{\isacharunderscore}{\kern0pt}iff\ del{\isacharcolon}{\kern0pt}fk{\isacharunderscore}{\kern0pt}space{\isacharunderscore}{\kern0pt}usage{\isachardot}{\kern0pt}simps{\isacharparenright}{\kern0pt}\isanewline
\ \ \ \ \isacommand{apply}\isamarkupfalse%
\ {\isacharparenleft}{\kern0pt}subst\ set{\isacharunderscore}{\kern0pt}prod{\isacharunderscore}{\kern0pt}pmf{\isacharcomma}{\kern0pt}\ simp{\isacharcomma}{\kern0pt}\ simp\ add{\isacharcolon}{\kern0pt}PiE{\isacharunderscore}{\kern0pt}def\ del{\isacharcolon}{\kern0pt}fk{\isacharunderscore}{\kern0pt}space{\isacharunderscore}{\kern0pt}usage{\isachardot}{\kern0pt}simps{\isacharparenright}{\kern0pt}\isanewline
\ \ \ \ \isacommand{apply}\isamarkupfalse%
\ {\isacharparenleft}{\kern0pt}subst\ set{\isacharunderscore}{\kern0pt}pmf{\isacharunderscore}{\kern0pt}of{\isacharunderscore}{\kern0pt}set\ {\isacharbrackleft}{\kern0pt}OF\ non{\isacharunderscore}{\kern0pt}empty{\isacharunderscore}{\kern0pt}space{\isacharbrackleft}{\kern0pt}OF\ False{\isacharbrackright}{\kern0pt}\ fin{\isacharunderscore}{\kern0pt}space{\isacharbrackleft}{\kern0pt}OF\ False{\isacharbrackright}{\kern0pt}{\isacharbrackright}{\kern0pt}{\isacharparenright}{\kern0pt}\isanewline
\ \ \ \ \isacommand{apply}\isamarkupfalse%
\ {\isacharparenleft}{\kern0pt}subst\ PiE{\isacharunderscore}{\kern0pt}def{\isacharbrackleft}{\kern0pt}symmetric{\isacharbrackright}{\kern0pt}{\isacharparenright}{\kern0pt}\isanewline
\ \ \ \ \isacommand{by}\isamarkupfalse%
\ {\isacharparenleft}{\kern0pt}metis\ b{\isacharparenright}{\kern0pt}\isanewline
\isacommand{qed}\isamarkupfalse%
%
\endisatagproof
{\isafoldproof}%
%
\isadelimproof
\isanewline
%
\endisadelimproof
\isanewline
\isacommand{lemma}\isamarkupfalse%
\ fk{\isacharunderscore}{\kern0pt}asympotic{\isacharunderscore}{\kern0pt}space{\isacharunderscore}{\kern0pt}complexity{\isacharcolon}{\kern0pt}\isanewline
\ \ {\isachardoublequoteopen}fk{\isacharunderscore}{\kern0pt}space{\isacharunderscore}{\kern0pt}usage\ {\isasymin}\ \isanewline
\ \ O{\isacharbrackleft}{\kern0pt}at{\isacharunderscore}{\kern0pt}top\ {\isasymtimes}\isactrlsub F\ at{\isacharunderscore}{\kern0pt}top\ {\isasymtimes}\isactrlsub F\ at{\isacharunderscore}{\kern0pt}top\ {\isasymtimes}\isactrlsub F\ at{\isacharunderscore}{\kern0pt}right\ {\isacharparenleft}{\kern0pt}{\isadigit{0}}{\isacharcolon}{\kern0pt}{\isacharcolon}{\kern0pt}rat{\isacharparenright}{\kern0pt}\ {\isasymtimes}\isactrlsub F\ at{\isacharunderscore}{\kern0pt}right\ {\isacharparenleft}{\kern0pt}{\isadigit{0}}{\isacharcolon}{\kern0pt}{\isacharcolon}{\kern0pt}rat{\isacharparenright}{\kern0pt}{\isacharbrackright}{\kern0pt}{\isacharparenleft}{\kern0pt}{\isasymlambda}\ {\isacharparenleft}{\kern0pt}k{\isacharcomma}{\kern0pt}\ n{\isacharcomma}{\kern0pt}\ m{\isacharcomma}{\kern0pt}\ {\isasymepsilon}{\isacharcomma}{\kern0pt}\ {\isasymdelta}{\isacharparenright}{\kern0pt}{\isachardot}{\kern0pt}\isanewline
\ \ real\ k{\isacharasterisk}{\kern0pt}{\isacharparenleft}{\kern0pt}real\ n{\isacharparenright}{\kern0pt}\ powr\ {\isacharparenleft}{\kern0pt}{\isadigit{1}}{\isacharminus}{\kern0pt}{\isadigit{1}}{\isacharslash}{\kern0pt}\ real\ k{\isacharparenright}{\kern0pt}\ {\isacharslash}{\kern0pt}\ {\isacharparenleft}{\kern0pt}of{\isacharunderscore}{\kern0pt}rat\ {\isasymdelta}{\isacharparenright}{\kern0pt}\isactrlsup {\isadigit{2}}\ {\isacharasterisk}{\kern0pt}\ {\isacharparenleft}{\kern0pt}ln\ {\isacharparenleft}{\kern0pt}{\isadigit{1}}\ {\isacharslash}{\kern0pt}\ of{\isacharunderscore}{\kern0pt}rat\ {\isasymepsilon}{\isacharparenright}{\kern0pt}{\isacharparenright}{\kern0pt}\ {\isacharasterisk}{\kern0pt}\ {\isacharparenleft}{\kern0pt}ln\ {\isacharparenleft}{\kern0pt}real\ n{\isacharparenright}{\kern0pt}\ {\isacharplus}{\kern0pt}\ ln\ {\isacharparenleft}{\kern0pt}real\ m{\isacharparenright}{\kern0pt}{\isacharparenright}{\kern0pt}{\isacharparenright}{\kern0pt}{\isachardoublequoteclose}\isanewline
\ \ {\isacharparenleft}{\kern0pt}\isakeyword{is}\ {\isachardoublequoteopen}{\isacharunderscore}{\kern0pt}\ {\isasymin}\ O{\isacharbrackleft}{\kern0pt}{\isacharquery}{\kern0pt}F{\isacharbrackright}{\kern0pt}{\isacharparenleft}{\kern0pt}{\isacharquery}{\kern0pt}rhs{\isacharparenright}{\kern0pt}{\isachardoublequoteclose}{\isacharparenright}{\kern0pt}\isanewline
%
\isadelimproof
%
\endisadelimproof
%
\isatagproof
\isacommand{proof}\isamarkupfalse%
\ {\isacharminus}{\kern0pt}\isanewline
\ \ \isacommand{define}\isamarkupfalse%
\ k{\isacharunderscore}{\kern0pt}of\ {\isacharcolon}{\kern0pt}{\isacharcolon}{\kern0pt}\ {\isachardoublequoteopen}nat\ {\isasymtimes}\ nat\ {\isasymtimes}\ nat\ {\isasymtimes}\ rat\ {\isasymtimes}\ rat\ {\isasymRightarrow}\ nat{\isachardoublequoteclose}\ \isakeyword{where}\ {\isachardoublequoteopen}k{\isacharunderscore}{\kern0pt}of\ {\isacharequal}{\kern0pt}\ {\isacharparenleft}{\kern0pt}{\isasymlambda}{\isacharparenleft}{\kern0pt}k{\isacharcomma}{\kern0pt}\ n{\isacharcomma}{\kern0pt}\ m{\isacharcomma}{\kern0pt}\ {\isasymepsilon}{\isacharcomma}{\kern0pt}\ {\isasymdelta}{\isacharparenright}{\kern0pt}{\isachardot}{\kern0pt}\ k{\isacharparenright}{\kern0pt}{\isachardoublequoteclose}\isanewline
\ \ \isacommand{define}\isamarkupfalse%
\ n{\isacharunderscore}{\kern0pt}of\ {\isacharcolon}{\kern0pt}{\isacharcolon}{\kern0pt}\ {\isachardoublequoteopen}nat\ {\isasymtimes}\ nat\ {\isasymtimes}\ nat\ {\isasymtimes}\ rat\ {\isasymtimes}\ rat\ {\isasymRightarrow}\ nat{\isachardoublequoteclose}\ \isakeyword{where}\ {\isachardoublequoteopen}n{\isacharunderscore}{\kern0pt}of\ {\isacharequal}{\kern0pt}\ {\isacharparenleft}{\kern0pt}{\isasymlambda}{\isacharparenleft}{\kern0pt}k{\isacharcomma}{\kern0pt}\ n{\isacharcomma}{\kern0pt}\ m{\isacharcomma}{\kern0pt}\ {\isasymepsilon}{\isacharcomma}{\kern0pt}\ {\isasymdelta}{\isacharparenright}{\kern0pt}{\isachardot}{\kern0pt}\ n{\isacharparenright}{\kern0pt}{\isachardoublequoteclose}\isanewline
\ \ \isacommand{define}\isamarkupfalse%
\ m{\isacharunderscore}{\kern0pt}of\ {\isacharcolon}{\kern0pt}{\isacharcolon}{\kern0pt}\ {\isachardoublequoteopen}nat\ {\isasymtimes}\ nat\ {\isasymtimes}\ nat\ {\isasymtimes}\ rat\ {\isasymtimes}\ rat\ {\isasymRightarrow}\ nat{\isachardoublequoteclose}\ \isakeyword{where}\ {\isachardoublequoteopen}m{\isacharunderscore}{\kern0pt}of\ {\isacharequal}{\kern0pt}\ {\isacharparenleft}{\kern0pt}{\isasymlambda}{\isacharparenleft}{\kern0pt}k{\isacharcomma}{\kern0pt}\ n{\isacharcomma}{\kern0pt}\ m{\isacharcomma}{\kern0pt}\ {\isasymepsilon}{\isacharcomma}{\kern0pt}\ {\isasymdelta}{\isacharparenright}{\kern0pt}{\isachardot}{\kern0pt}\ m{\isacharparenright}{\kern0pt}{\isachardoublequoteclose}\isanewline
\ \ \isacommand{define}\isamarkupfalse%
\ {\isasymepsilon}{\isacharunderscore}{\kern0pt}of\ {\isacharcolon}{\kern0pt}{\isacharcolon}{\kern0pt}\ {\isachardoublequoteopen}nat\ {\isasymtimes}\ nat\ {\isasymtimes}\ nat\ {\isasymtimes}\ rat\ {\isasymtimes}\ rat\ {\isasymRightarrow}\ rat{\isachardoublequoteclose}\ \isakeyword{where}\ {\isachardoublequoteopen}{\isasymepsilon}{\isacharunderscore}{\kern0pt}of\ {\isacharequal}{\kern0pt}\ {\isacharparenleft}{\kern0pt}{\isasymlambda}{\isacharparenleft}{\kern0pt}k{\isacharcomma}{\kern0pt}\ n{\isacharcomma}{\kern0pt}\ m{\isacharcomma}{\kern0pt}\ {\isasymepsilon}{\isacharcomma}{\kern0pt}\ {\isasymdelta}{\isacharparenright}{\kern0pt}{\isachardot}{\kern0pt}\ {\isasymepsilon}{\isacharparenright}{\kern0pt}{\isachardoublequoteclose}\isanewline
\ \ \isacommand{define}\isamarkupfalse%
\ {\isasymdelta}{\isacharunderscore}{\kern0pt}of\ {\isacharcolon}{\kern0pt}{\isacharcolon}{\kern0pt}\ {\isachardoublequoteopen}nat\ {\isasymtimes}\ nat\ {\isasymtimes}\ nat\ {\isasymtimes}\ rat\ {\isasymtimes}\ rat\ {\isasymRightarrow}\ rat{\isachardoublequoteclose}\ \isakeyword{where}\ {\isachardoublequoteopen}{\isasymdelta}{\isacharunderscore}{\kern0pt}of\ {\isacharequal}{\kern0pt}\ {\isacharparenleft}{\kern0pt}{\isasymlambda}{\isacharparenleft}{\kern0pt}k{\isacharcomma}{\kern0pt}\ n{\isacharcomma}{\kern0pt}\ m{\isacharcomma}{\kern0pt}\ {\isasymepsilon}{\isacharcomma}{\kern0pt}\ {\isasymdelta}{\isacharparenright}{\kern0pt}{\isachardot}{\kern0pt}\ {\isasymdelta}{\isacharparenright}{\kern0pt}{\isachardoublequoteclose}\isanewline
\isanewline
\ \ \isacommand{define}\isamarkupfalse%
\ g{\isadigit{1}}\ \isakeyword{where}\ {\isachardoublequoteopen}g{\isadigit{1}}\ {\isacharequal}{\kern0pt}\ {\isacharparenleft}{\kern0pt}{\isasymlambda}x{\isachardot}{\kern0pt}\ real\ {\isacharparenleft}{\kern0pt}k{\isacharunderscore}{\kern0pt}of\ x{\isacharparenright}{\kern0pt}{\isacharasterisk}{\kern0pt}{\isacharparenleft}{\kern0pt}real\ {\isacharparenleft}{\kern0pt}n{\isacharunderscore}{\kern0pt}of\ x{\isacharparenright}{\kern0pt}{\isacharparenright}{\kern0pt}\ powr\ {\isacharparenleft}{\kern0pt}{\isadigit{1}}{\isacharminus}{\kern0pt}{\isadigit{1}}{\isacharslash}{\kern0pt}\ real\ {\isacharparenleft}{\kern0pt}k{\isacharunderscore}{\kern0pt}of\ x{\isacharparenright}{\kern0pt}{\isacharparenright}{\kern0pt}\ {\isacharslash}{\kern0pt}\ \isanewline
\ \ \ \ {\isacharparenleft}{\kern0pt}of{\isacharunderscore}{\kern0pt}rat\ {\isacharparenleft}{\kern0pt}{\isasymdelta}{\isacharunderscore}{\kern0pt}of\ x{\isacharparenright}{\kern0pt}{\isacharparenright}{\kern0pt}\isactrlsup {\isadigit{2}}{\isacharparenright}{\kern0pt}{\isachardoublequoteclose}\isanewline
\isanewline
\ \ \isacommand{define}\isamarkupfalse%
\ g\ \isakeyword{where}\ {\isachardoublequoteopen}g\ {\isacharequal}{\kern0pt}\ {\isacharparenleft}{\kern0pt}{\isasymlambda}x{\isachardot}{\kern0pt}\ g{\isadigit{1}}\ x\ {\isacharasterisk}{\kern0pt}\ {\isacharparenleft}{\kern0pt}ln\ {\isacharparenleft}{\kern0pt}{\isadigit{1}}\ {\isacharslash}{\kern0pt}\ of{\isacharunderscore}{\kern0pt}rat\ {\isacharparenleft}{\kern0pt}{\isasymepsilon}{\isacharunderscore}{\kern0pt}of\ x{\isacharparenright}{\kern0pt}{\isacharparenright}{\kern0pt}{\isacharparenright}{\kern0pt}\ {\isacharasterisk}{\kern0pt}\ {\isacharparenleft}{\kern0pt}ln\ {\isacharparenleft}{\kern0pt}real\ {\isacharparenleft}{\kern0pt}n{\isacharunderscore}{\kern0pt}of\ x{\isacharparenright}{\kern0pt}{\isacharparenright}{\kern0pt}\ {\isacharplus}{\kern0pt}\ ln\ {\isacharparenleft}{\kern0pt}real\ {\isacharparenleft}{\kern0pt}m{\isacharunderscore}{\kern0pt}of\ x{\isacharparenright}{\kern0pt}{\isacharparenright}{\kern0pt}{\isacharparenright}{\kern0pt}{\isacharparenright}{\kern0pt}{\isachardoublequoteclose}\isanewline
\isanewline
\ \ \isacommand{have}\isamarkupfalse%
\ k{\isacharunderscore}{\kern0pt}inf{\isacharcolon}{\kern0pt}\ {\isachardoublequoteopen}{\isasymAnd}c{\isachardot}{\kern0pt}\ eventually\ {\isacharparenleft}{\kern0pt}{\isasymlambda}x{\isachardot}{\kern0pt}\ c\ {\isasymle}\ {\isacharparenleft}{\kern0pt}real\ {\isacharparenleft}{\kern0pt}k{\isacharunderscore}{\kern0pt}of\ x{\isacharparenright}{\kern0pt}{\isacharparenright}{\kern0pt}{\isacharparenright}{\kern0pt}\ {\isacharquery}{\kern0pt}F{\isachardoublequoteclose}\isanewline
\ \ \ \ \isacommand{apply}\isamarkupfalse%
\ {\isacharparenleft}{\kern0pt}simp\ add{\isacharcolon}{\kern0pt}k{\isacharunderscore}{\kern0pt}of{\isacharunderscore}{\kern0pt}def\ case{\isacharunderscore}{\kern0pt}prod{\isacharunderscore}{\kern0pt}beta{\isacharprime}{\kern0pt}{\isacharparenright}{\kern0pt}\isanewline
\ \ \ \ \isacommand{apply}\isamarkupfalse%
\ {\isacharparenleft}{\kern0pt}subst\ eventually{\isacharunderscore}{\kern0pt}prod{\isadigit{1}}{\isacharprime}{\kern0pt}{\isacharcomma}{\kern0pt}\ simp\ add{\isacharcolon}{\kern0pt}prod{\isacharunderscore}{\kern0pt}filter{\isacharunderscore}{\kern0pt}eq{\isacharunderscore}{\kern0pt}bot{\isacharparenright}{\kern0pt}\isanewline
\ \ \ \ \isacommand{by}\isamarkupfalse%
\ {\isacharparenleft}{\kern0pt}meson\ eventually{\isacharunderscore}{\kern0pt}at{\isacharunderscore}{\kern0pt}top{\isacharunderscore}{\kern0pt}linorder\ nat{\isacharunderscore}{\kern0pt}ceiling{\isacharunderscore}{\kern0pt}le{\isacharunderscore}{\kern0pt}eq{\isacharparenright}{\kern0pt}\isanewline
\isanewline
\ \ \isacommand{have}\isamarkupfalse%
\ n{\isacharunderscore}{\kern0pt}inf{\isacharcolon}{\kern0pt}\ {\isachardoublequoteopen}{\isasymAnd}c{\isachardot}{\kern0pt}\ eventually\ {\isacharparenleft}{\kern0pt}{\isasymlambda}x{\isachardot}{\kern0pt}\ c\ {\isasymle}\ {\isacharparenleft}{\kern0pt}real\ {\isacharparenleft}{\kern0pt}n{\isacharunderscore}{\kern0pt}of\ x{\isacharparenright}{\kern0pt}{\isacharparenright}{\kern0pt}{\isacharparenright}{\kern0pt}\ {\isacharquery}{\kern0pt}F{\isachardoublequoteclose}\ \isanewline
\ \ \ \ \isacommand{apply}\isamarkupfalse%
\ {\isacharparenleft}{\kern0pt}simp\ add{\isacharcolon}{\kern0pt}n{\isacharunderscore}{\kern0pt}of{\isacharunderscore}{\kern0pt}def\ case{\isacharunderscore}{\kern0pt}prod{\isacharunderscore}{\kern0pt}beta{\isacharprime}{\kern0pt}{\isacharparenright}{\kern0pt}\isanewline
\ \ \ \ \isacommand{apply}\isamarkupfalse%
\ {\isacharparenleft}{\kern0pt}subst\ eventually{\isacharunderscore}{\kern0pt}prod{\isadigit{2}}{\isacharprime}{\kern0pt}{\isacharcomma}{\kern0pt}\ simp\ add{\isacharcolon}{\kern0pt}prod{\isacharunderscore}{\kern0pt}filter{\isacharunderscore}{\kern0pt}eq{\isacharunderscore}{\kern0pt}bot{\isacharparenright}{\kern0pt}\isanewline
\ \ \ \ \isacommand{apply}\isamarkupfalse%
\ {\isacharparenleft}{\kern0pt}subst\ eventually{\isacharunderscore}{\kern0pt}prod{\isadigit{1}}{\isacharprime}{\kern0pt}{\isacharcomma}{\kern0pt}\ simp\ add{\isacharcolon}{\kern0pt}prod{\isacharunderscore}{\kern0pt}filter{\isacharunderscore}{\kern0pt}eq{\isacharunderscore}{\kern0pt}bot{\isacharparenright}{\kern0pt}\isanewline
\ \ \ \ \isacommand{by}\isamarkupfalse%
\ {\isacharparenleft}{\kern0pt}meson\ eventually{\isacharunderscore}{\kern0pt}at{\isacharunderscore}{\kern0pt}top{\isacharunderscore}{\kern0pt}linorder\ nat{\isacharunderscore}{\kern0pt}ceiling{\isacharunderscore}{\kern0pt}le{\isacharunderscore}{\kern0pt}eq{\isacharparenright}{\kern0pt}\isanewline
\isanewline
\ \ \isacommand{have}\isamarkupfalse%
\ m{\isacharunderscore}{\kern0pt}inf{\isacharcolon}{\kern0pt}\ {\isachardoublequoteopen}{\isasymAnd}c{\isachardot}{\kern0pt}\ eventually\ {\isacharparenleft}{\kern0pt}{\isasymlambda}x{\isachardot}{\kern0pt}\ c\ {\isasymle}\ {\isacharparenleft}{\kern0pt}real\ {\isacharparenleft}{\kern0pt}m{\isacharunderscore}{\kern0pt}of\ x{\isacharparenright}{\kern0pt}{\isacharparenright}{\kern0pt}{\isacharparenright}{\kern0pt}\ {\isacharquery}{\kern0pt}F{\isachardoublequoteclose}\ \isanewline
\ \ \ \ \isacommand{apply}\isamarkupfalse%
\ {\isacharparenleft}{\kern0pt}simp\ add{\isacharcolon}{\kern0pt}m{\isacharunderscore}{\kern0pt}of{\isacharunderscore}{\kern0pt}def\ case{\isacharunderscore}{\kern0pt}prod{\isacharunderscore}{\kern0pt}beta{\isacharprime}{\kern0pt}{\isacharparenright}{\kern0pt}\isanewline
\ \ \ \ \isacommand{apply}\isamarkupfalse%
\ {\isacharparenleft}{\kern0pt}subst\ eventually{\isacharunderscore}{\kern0pt}prod{\isadigit{2}}{\isacharprime}{\kern0pt}{\isacharcomma}{\kern0pt}\ simp\ add{\isacharcolon}{\kern0pt}prod{\isacharunderscore}{\kern0pt}filter{\isacharunderscore}{\kern0pt}eq{\isacharunderscore}{\kern0pt}bot{\isacharparenright}{\kern0pt}\isanewline
\ \ \ \ \isacommand{apply}\isamarkupfalse%
\ {\isacharparenleft}{\kern0pt}subst\ eventually{\isacharunderscore}{\kern0pt}prod{\isadigit{2}}{\isacharprime}{\kern0pt}{\isacharcomma}{\kern0pt}\ simp\ add{\isacharcolon}{\kern0pt}prod{\isacharunderscore}{\kern0pt}filter{\isacharunderscore}{\kern0pt}eq{\isacharunderscore}{\kern0pt}bot{\isacharparenright}{\kern0pt}\isanewline
\ \ \ \ \isacommand{apply}\isamarkupfalse%
\ {\isacharparenleft}{\kern0pt}subst\ eventually{\isacharunderscore}{\kern0pt}prod{\isadigit{1}}{\isacharprime}{\kern0pt}{\isacharcomma}{\kern0pt}\ simp\ add{\isacharcolon}{\kern0pt}prod{\isacharunderscore}{\kern0pt}filter{\isacharunderscore}{\kern0pt}eq{\isacharunderscore}{\kern0pt}bot{\isacharparenright}{\kern0pt}\isanewline
\ \ \ \ \isacommand{by}\isamarkupfalse%
\ {\isacharparenleft}{\kern0pt}meson\ eventually{\isacharunderscore}{\kern0pt}at{\isacharunderscore}{\kern0pt}top{\isacharunderscore}{\kern0pt}linorder\ nat{\isacharunderscore}{\kern0pt}ceiling{\isacharunderscore}{\kern0pt}le{\isacharunderscore}{\kern0pt}eq{\isacharparenright}{\kern0pt}\isanewline
\isanewline
\ \ \isacommand{have}\isamarkupfalse%
\ eps{\isacharunderscore}{\kern0pt}inf{\isacharcolon}{\kern0pt}\ {\isachardoublequoteopen}{\isasymAnd}c{\isachardot}{\kern0pt}\ eventually\ {\isacharparenleft}{\kern0pt}{\isasymlambda}x{\isachardot}{\kern0pt}\ c\ {\isasymle}\ {\isadigit{1}}\ {\isacharslash}{\kern0pt}\ {\isacharparenleft}{\kern0pt}real{\isacharunderscore}{\kern0pt}of{\isacharunderscore}{\kern0pt}rat\ {\isacharparenleft}{\kern0pt}{\isasymepsilon}{\isacharunderscore}{\kern0pt}of\ x{\isacharparenright}{\kern0pt}{\isacharparenright}{\kern0pt}{\isacharparenright}{\kern0pt}\ {\isacharquery}{\kern0pt}F{\isachardoublequoteclose}\isanewline
\ \ \ \ \isacommand{apply}\isamarkupfalse%
\ {\isacharparenleft}{\kern0pt}simp\ add{\isacharcolon}{\kern0pt}{\isasymepsilon}{\isacharunderscore}{\kern0pt}of{\isacharunderscore}{\kern0pt}def\ case{\isacharunderscore}{\kern0pt}prod{\isacharunderscore}{\kern0pt}beta{\isacharprime}{\kern0pt}{\isacharparenright}{\kern0pt}\isanewline
\ \ \ \ \isacommand{apply}\isamarkupfalse%
\ {\isacharparenleft}{\kern0pt}subst\ eventually{\isacharunderscore}{\kern0pt}prod{\isadigit{2}}{\isacharprime}{\kern0pt}{\isacharcomma}{\kern0pt}\ simp\ add{\isacharcolon}{\kern0pt}prod{\isacharunderscore}{\kern0pt}filter{\isacharunderscore}{\kern0pt}eq{\isacharunderscore}{\kern0pt}bot{\isacharparenright}{\kern0pt}\isanewline
\ \ \ \ \isacommand{apply}\isamarkupfalse%
\ {\isacharparenleft}{\kern0pt}subst\ eventually{\isacharunderscore}{\kern0pt}prod{\isadigit{2}}{\isacharprime}{\kern0pt}{\isacharcomma}{\kern0pt}\ simp{\isacharparenright}{\kern0pt}\isanewline
\ \ \ \ \isacommand{apply}\isamarkupfalse%
\ {\isacharparenleft}{\kern0pt}subst\ eventually{\isacharunderscore}{\kern0pt}prod{\isadigit{2}}{\isacharprime}{\kern0pt}{\isacharcomma}{\kern0pt}\ simp{\isacharparenright}{\kern0pt}\isanewline
\ \ \ \ \isacommand{apply}\isamarkupfalse%
\ {\isacharparenleft}{\kern0pt}subst\ eventually{\isacharunderscore}{\kern0pt}prod{\isadigit{1}}{\isacharprime}{\kern0pt}{\isacharcomma}{\kern0pt}\ simp{\isacharparenright}{\kern0pt}\isanewline
\ \ \ \ \isacommand{by}\isamarkupfalse%
\ {\isacharparenleft}{\kern0pt}rule\ inv{\isacharunderscore}{\kern0pt}at{\isacharunderscore}{\kern0pt}right{\isacharunderscore}{\kern0pt}{\isadigit{0}}{\isacharunderscore}{\kern0pt}inf{\isacharparenright}{\kern0pt}\isanewline
\isanewline
\ \ \isacommand{have}\isamarkupfalse%
\ delta{\isacharunderscore}{\kern0pt}inf{\isacharcolon}{\kern0pt}\ {\isachardoublequoteopen}{\isasymAnd}c{\isachardot}{\kern0pt}\ eventually\ {\isacharparenleft}{\kern0pt}{\isasymlambda}x{\isachardot}{\kern0pt}\ c\ {\isasymle}\ {\isadigit{1}}\ {\isacharslash}{\kern0pt}\ {\isacharparenleft}{\kern0pt}real{\isacharunderscore}{\kern0pt}of{\isacharunderscore}{\kern0pt}rat\ {\isacharparenleft}{\kern0pt}{\isasymdelta}{\isacharunderscore}{\kern0pt}of\ x{\isacharparenright}{\kern0pt}{\isacharparenright}{\kern0pt}{\isacharparenright}{\kern0pt}\ {\isacharquery}{\kern0pt}F{\isachardoublequoteclose}\isanewline
\ \ \ \ \isacommand{apply}\isamarkupfalse%
\ {\isacharparenleft}{\kern0pt}simp\ add{\isacharcolon}{\kern0pt}{\isasymdelta}{\isacharunderscore}{\kern0pt}of{\isacharunderscore}{\kern0pt}def\ case{\isacharunderscore}{\kern0pt}prod{\isacharunderscore}{\kern0pt}beta{\isacharprime}{\kern0pt}{\isacharparenright}{\kern0pt}\isanewline
\ \ \ \ \isacommand{apply}\isamarkupfalse%
\ {\isacharparenleft}{\kern0pt}subst\ eventually{\isacharunderscore}{\kern0pt}prod{\isadigit{2}}{\isacharprime}{\kern0pt}{\isacharcomma}{\kern0pt}\ simp\ add{\isacharcolon}{\kern0pt}prod{\isacharunderscore}{\kern0pt}filter{\isacharunderscore}{\kern0pt}eq{\isacharunderscore}{\kern0pt}bot{\isacharparenright}{\kern0pt}\isanewline
\ \ \ \ \isacommand{apply}\isamarkupfalse%
\ {\isacharparenleft}{\kern0pt}subst\ eventually{\isacharunderscore}{\kern0pt}prod{\isadigit{2}}{\isacharprime}{\kern0pt}{\isacharcomma}{\kern0pt}\ simp{\isacharparenright}{\kern0pt}\isanewline
\ \ \ \ \isacommand{apply}\isamarkupfalse%
\ {\isacharparenleft}{\kern0pt}subst\ eventually{\isacharunderscore}{\kern0pt}prod{\isadigit{2}}{\isacharprime}{\kern0pt}{\isacharcomma}{\kern0pt}\ simp{\isacharparenright}{\kern0pt}\isanewline
\ \ \ \ \isacommand{apply}\isamarkupfalse%
\ {\isacharparenleft}{\kern0pt}subst\ eventually{\isacharunderscore}{\kern0pt}prod{\isadigit{2}}{\isacharprime}{\kern0pt}{\isacharcomma}{\kern0pt}\ simp{\isacharparenright}{\kern0pt}\isanewline
\ \ \ \ \isacommand{by}\isamarkupfalse%
\ {\isacharparenleft}{\kern0pt}rule\ inv{\isacharunderscore}{\kern0pt}at{\isacharunderscore}{\kern0pt}right{\isacharunderscore}{\kern0pt}{\isadigit{0}}{\isacharunderscore}{\kern0pt}inf{\isacharparenright}{\kern0pt}\isanewline
\isanewline
\ \ \isacommand{have}\isamarkupfalse%
\ zero{\isacharunderscore}{\kern0pt}less{\isacharunderscore}{\kern0pt}eps{\isacharcolon}{\kern0pt}\ {\isachardoublequoteopen}eventually\ {\isacharparenleft}{\kern0pt}{\isasymlambda}x{\isachardot}{\kern0pt}\ {\isadigit{0}}\ {\isacharless}{\kern0pt}\ {\isacharparenleft}{\kern0pt}real{\isacharunderscore}{\kern0pt}of{\isacharunderscore}{\kern0pt}rat\ {\isacharparenleft}{\kern0pt}{\isasymepsilon}{\isacharunderscore}{\kern0pt}of\ x{\isacharparenright}{\kern0pt}{\isacharparenright}{\kern0pt}{\isacharparenright}{\kern0pt}\ {\isacharquery}{\kern0pt}F{\isachardoublequoteclose}\isanewline
\ \ \ \ \isacommand{apply}\isamarkupfalse%
\ {\isacharparenleft}{\kern0pt}simp\ add{\isacharcolon}{\kern0pt}{\isasymepsilon}{\isacharunderscore}{\kern0pt}of{\isacharunderscore}{\kern0pt}def\ case{\isacharunderscore}{\kern0pt}prod{\isacharunderscore}{\kern0pt}beta{\isacharprime}{\kern0pt}{\isacharparenright}{\kern0pt}\isanewline
\ \ \ \ \isacommand{apply}\isamarkupfalse%
\ {\isacharparenleft}{\kern0pt}subst\ eventually{\isacharunderscore}{\kern0pt}prod{\isadigit{2}}{\isacharprime}{\kern0pt}{\isacharcomma}{\kern0pt}\ simp{\isacharparenright}{\kern0pt}\isanewline
\ \ \ \ \isacommand{apply}\isamarkupfalse%
\ {\isacharparenleft}{\kern0pt}subst\ eventually{\isacharunderscore}{\kern0pt}prod{\isadigit{2}}{\isacharprime}{\kern0pt}{\isacharcomma}{\kern0pt}\ simp{\isacharparenright}{\kern0pt}\isanewline
\ \ \ \ \isacommand{apply}\isamarkupfalse%
\ {\isacharparenleft}{\kern0pt}subst\ eventually{\isacharunderscore}{\kern0pt}prod{\isadigit{2}}{\isacharprime}{\kern0pt}{\isacharcomma}{\kern0pt}\ simp{\isacharparenright}{\kern0pt}\isanewline
\ \ \ \ \isacommand{apply}\isamarkupfalse%
\ {\isacharparenleft}{\kern0pt}subst\ eventually{\isacharunderscore}{\kern0pt}prod{\isadigit{1}}{\isacharprime}{\kern0pt}{\isacharcomma}{\kern0pt}\ simp{\isacharparenright}{\kern0pt}\isanewline
\ \ \ \ \isacommand{by}\isamarkupfalse%
\ {\isacharparenleft}{\kern0pt}rule\ eventually{\isacharunderscore}{\kern0pt}at{\isacharunderscore}{\kern0pt}rightI{\isacharbrackleft}{\kern0pt}\isakeyword{where}\ b{\isacharequal}{\kern0pt}{\isachardoublequoteopen}{\isadigit{1}}{\isachardoublequoteclose}{\isacharbrackright}{\kern0pt}{\isacharcomma}{\kern0pt}\ simp{\isacharcomma}{\kern0pt}\ simp{\isacharparenright}{\kern0pt}\isanewline
\isanewline
\ \ \isacommand{have}\isamarkupfalse%
\ zero{\isacharunderscore}{\kern0pt}less{\isacharunderscore}{\kern0pt}delta{\isacharcolon}{\kern0pt}\ {\isachardoublequoteopen}eventually\ {\isacharparenleft}{\kern0pt}{\isasymlambda}x{\isachardot}{\kern0pt}\ {\isadigit{0}}\ {\isacharless}{\kern0pt}\ {\isacharparenleft}{\kern0pt}real{\isacharunderscore}{\kern0pt}of{\isacharunderscore}{\kern0pt}rat\ {\isacharparenleft}{\kern0pt}{\isasymdelta}{\isacharunderscore}{\kern0pt}of\ x{\isacharparenright}{\kern0pt}{\isacharparenright}{\kern0pt}{\isacharparenright}{\kern0pt}\ {\isacharquery}{\kern0pt}F{\isachardoublequoteclose}\isanewline
\ \ \ \ \isacommand{apply}\isamarkupfalse%
\ {\isacharparenleft}{\kern0pt}simp\ add{\isacharcolon}{\kern0pt}{\isasymdelta}{\isacharunderscore}{\kern0pt}of{\isacharunderscore}{\kern0pt}def\ case{\isacharunderscore}{\kern0pt}prod{\isacharunderscore}{\kern0pt}beta{\isacharprime}{\kern0pt}{\isacharparenright}{\kern0pt}\isanewline
\ \ \ \ \isacommand{apply}\isamarkupfalse%
\ {\isacharparenleft}{\kern0pt}subst\ eventually{\isacharunderscore}{\kern0pt}prod{\isadigit{2}}{\isacharprime}{\kern0pt}{\isacharcomma}{\kern0pt}\ simp{\isacharparenright}{\kern0pt}\isanewline
\ \ \ \ \isacommand{apply}\isamarkupfalse%
\ {\isacharparenleft}{\kern0pt}subst\ eventually{\isacharunderscore}{\kern0pt}prod{\isadigit{2}}{\isacharprime}{\kern0pt}{\isacharcomma}{\kern0pt}\ simp{\isacharparenright}{\kern0pt}\isanewline
\ \ \ \ \isacommand{apply}\isamarkupfalse%
\ {\isacharparenleft}{\kern0pt}subst\ eventually{\isacharunderscore}{\kern0pt}prod{\isadigit{2}}{\isacharprime}{\kern0pt}{\isacharcomma}{\kern0pt}\ simp{\isacharparenright}{\kern0pt}\isanewline
\ \ \ \ \isacommand{apply}\isamarkupfalse%
\ {\isacharparenleft}{\kern0pt}subst\ eventually{\isacharunderscore}{\kern0pt}prod{\isadigit{2}}{\isacharprime}{\kern0pt}{\isacharcomma}{\kern0pt}\ simp{\isacharparenright}{\kern0pt}\isanewline
\ \ \ \ \isacommand{by}\isamarkupfalse%
\ {\isacharparenleft}{\kern0pt}rule\ eventually{\isacharunderscore}{\kern0pt}at{\isacharunderscore}{\kern0pt}rightI{\isacharbrackleft}{\kern0pt}\isakeyword{where}\ b{\isacharequal}{\kern0pt}{\isachardoublequoteopen}{\isadigit{1}}{\isachardoublequoteclose}{\isacharbrackright}{\kern0pt}{\isacharcomma}{\kern0pt}\ simp{\isacharcomma}{\kern0pt}\ simp{\isacharparenright}{\kern0pt}\isanewline
\isanewline
\ \ \isacommand{have}\isamarkupfalse%
\ unit{\isacharunderscore}{\kern0pt}{\isadigit{9}}{\isacharcolon}{\kern0pt}\ {\isachardoublequoteopen}{\isacharparenleft}{\kern0pt}{\isasymlambda}{\isacharunderscore}{\kern0pt}{\isachardot}{\kern0pt}\ {\isadigit{1}}{\isacharparenright}{\kern0pt}\ {\isasymin}\ O{\isacharbrackleft}{\kern0pt}{\isacharquery}{\kern0pt}F{\isacharbrackright}{\kern0pt}{\isacharparenleft}{\kern0pt}{\isasymlambda}x{\isachardot}{\kern0pt}\ real\ {\isacharparenleft}{\kern0pt}n{\isacharunderscore}{\kern0pt}of\ x{\isacharparenright}{\kern0pt}\ powr\ {\isacharparenleft}{\kern0pt}{\isadigit{1}}\ {\isacharminus}{\kern0pt}\ {\isadigit{1}}\ {\isacharslash}{\kern0pt}\ real\ {\isacharparenleft}{\kern0pt}k{\isacharunderscore}{\kern0pt}of\ x{\isacharparenright}{\kern0pt}{\isacharparenright}{\kern0pt}{\isacharparenright}{\kern0pt}{\isachardoublequoteclose}\isanewline
\ \ \ \ \isacommand{apply}\isamarkupfalse%
\ {\isacharparenleft}{\kern0pt}rule\ landau{\isacharunderscore}{\kern0pt}o{\isachardot}{\kern0pt}big{\isacharunderscore}{\kern0pt}mono{\isacharcomma}{\kern0pt}\ simp{\isacharparenright}{\kern0pt}\isanewline
\ \ \ \ \isacommand{apply}\isamarkupfalse%
\ {\isacharparenleft}{\kern0pt}rule\ eventually{\isacharunderscore}{\kern0pt}mono{\isacharbrackleft}{\kern0pt}OF\ eventually{\isacharunderscore}{\kern0pt}conj{\isacharbrackleft}{\kern0pt}OF\ n{\isacharunderscore}{\kern0pt}inf{\isacharbrackleft}{\kern0pt}\isakeyword{where}\ c{\isacharequal}{\kern0pt}{\isachardoublequoteopen}{\isadigit{1}}{\isachardoublequoteclose}{\isacharbrackright}{\kern0pt}\ k{\isacharunderscore}{\kern0pt}inf{\isacharbrackleft}{\kern0pt}\isakeyword{where}\ c{\isacharequal}{\kern0pt}{\isachardoublequoteopen}{\isadigit{1}}{\isachardoublequoteclose}{\isacharbrackright}{\kern0pt}{\isacharbrackright}{\kern0pt}{\isacharbrackright}{\kern0pt}{\isacharparenright}{\kern0pt}\isanewline
\ \ \ \ \isacommand{by}\isamarkupfalse%
\ {\isacharparenleft}{\kern0pt}simp\ add{\isacharcolon}{\kern0pt}\ ge{\isacharunderscore}{\kern0pt}one{\isacharunderscore}{\kern0pt}powr{\isacharunderscore}{\kern0pt}ge{\isacharunderscore}{\kern0pt}zero{\isacharparenright}{\kern0pt}\isanewline
\isanewline
\ \ \isacommand{have}\isamarkupfalse%
\ unit{\isacharunderscore}{\kern0pt}{\isadigit{8}}{\isacharcolon}{\kern0pt}\ {\isachardoublequoteopen}{\isacharparenleft}{\kern0pt}{\isasymlambda}{\isacharunderscore}{\kern0pt}{\isachardot}{\kern0pt}\ {\isadigit{1}}{\isacharparenright}{\kern0pt}\ {\isasymin}\ O{\isacharbrackleft}{\kern0pt}{\isacharquery}{\kern0pt}F{\isacharbrackright}{\kern0pt}{\isacharparenleft}{\kern0pt}{\isasymlambda}x{\isachardot}{\kern0pt}\ real\ {\isacharparenleft}{\kern0pt}k{\isacharunderscore}{\kern0pt}of\ x{\isacharparenright}{\kern0pt}{\isacharparenright}{\kern0pt}{\isachardoublequoteclose}\isanewline
\ \ \ \ \isacommand{by}\isamarkupfalse%
\ {\isacharparenleft}{\kern0pt}rule\ landau{\isacharunderscore}{\kern0pt}o{\isachardot}{\kern0pt}big{\isacharunderscore}{\kern0pt}mono{\isacharcomma}{\kern0pt}\ simp{\isacharcomma}{\kern0pt}\ rule\ k{\isacharunderscore}{\kern0pt}inf{\isacharparenright}{\kern0pt}\isanewline
\ \ \isacommand{have}\isamarkupfalse%
\ unit{\isacharunderscore}{\kern0pt}{\isadigit{6}}{\isacharcolon}{\kern0pt}\ {\isachardoublequoteopen}{\isacharparenleft}{\kern0pt}{\isasymlambda}{\isacharunderscore}{\kern0pt}{\isachardot}{\kern0pt}\ {\isadigit{1}}{\isacharparenright}{\kern0pt}\ {\isasymin}\ O{\isacharbrackleft}{\kern0pt}{\isacharquery}{\kern0pt}F{\isacharbrackright}{\kern0pt}{\isacharparenleft}{\kern0pt}{\isasymlambda}x{\isachardot}{\kern0pt}\ real\ {\isacharparenleft}{\kern0pt}m{\isacharunderscore}{\kern0pt}of\ x{\isacharparenright}{\kern0pt}{\isacharparenright}{\kern0pt}{\isachardoublequoteclose}\ \isanewline
\ \ \ \ \isacommand{by}\isamarkupfalse%
\ {\isacharparenleft}{\kern0pt}rule\ landau{\isacharunderscore}{\kern0pt}o{\isachardot}{\kern0pt}big{\isacharunderscore}{\kern0pt}mono{\isacharcomma}{\kern0pt}\ simp{\isacharcomma}{\kern0pt}\ rule\ m{\isacharunderscore}{\kern0pt}inf{\isacharparenright}{\kern0pt}\isanewline
\ \ \isacommand{have}\isamarkupfalse%
\ unit{\isacharunderscore}{\kern0pt}n{\isacharcolon}{\kern0pt}\ {\isachardoublequoteopen}{\isacharparenleft}{\kern0pt}{\isasymlambda}{\isacharunderscore}{\kern0pt}{\isachardot}{\kern0pt}\ {\isadigit{1}}{\isacharparenright}{\kern0pt}\ {\isasymin}\ O{\isacharbrackleft}{\kern0pt}{\isacharquery}{\kern0pt}F{\isacharbrackright}{\kern0pt}{\isacharparenleft}{\kern0pt}{\isasymlambda}x{\isachardot}{\kern0pt}\ real\ {\isacharparenleft}{\kern0pt}n{\isacharunderscore}{\kern0pt}of\ x{\isacharparenright}{\kern0pt}{\isacharparenright}{\kern0pt}{\isachardoublequoteclose}\ \isanewline
\ \ \ \ \isacommand{by}\isamarkupfalse%
\ {\isacharparenleft}{\kern0pt}rule\ landau{\isacharunderscore}{\kern0pt}o{\isachardot}{\kern0pt}big{\isacharunderscore}{\kern0pt}mono{\isacharcomma}{\kern0pt}\ simp{\isacharcomma}{\kern0pt}\ rule\ n{\isacharunderscore}{\kern0pt}inf{\isacharparenright}{\kern0pt}\isanewline
\isanewline
\ \ \isacommand{have}\isamarkupfalse%
\ unit{\isacharunderscore}{\kern0pt}{\isadigit{2}}{\isacharcolon}{\kern0pt}\ {\isachardoublequoteopen}{\isacharparenleft}{\kern0pt}{\isasymlambda}{\isacharunderscore}{\kern0pt}{\isachardot}{\kern0pt}\ {\isadigit{1}}{\isacharparenright}{\kern0pt}\ {\isasymin}\ O{\isacharbrackleft}{\kern0pt}{\isacharquery}{\kern0pt}F{\isacharbrackright}{\kern0pt}{\isacharparenleft}{\kern0pt}{\isasymlambda}x{\isachardot}{\kern0pt}\ ln\ {\isacharparenleft}{\kern0pt}{\isadigit{1}}\ {\isacharslash}{\kern0pt}\ real{\isacharunderscore}{\kern0pt}of{\isacharunderscore}{\kern0pt}rat\ {\isacharparenleft}{\kern0pt}{\isasymepsilon}{\isacharunderscore}{\kern0pt}of\ x{\isacharparenright}{\kern0pt}{\isacharparenright}{\kern0pt}{\isacharparenright}{\kern0pt}{\isachardoublequoteclose}\isanewline
\ \ \ \ \isacommand{apply}\isamarkupfalse%
\ {\isacharparenleft}{\kern0pt}rule\ landau{\isacharunderscore}{\kern0pt}o{\isachardot}{\kern0pt}big{\isacharunderscore}{\kern0pt}mono{\isacharcomma}{\kern0pt}\ simp{\isacharparenright}{\kern0pt}\isanewline
\ \ \ \ \isacommand{apply}\isamarkupfalse%
\ {\isacharparenleft}{\kern0pt}rule\ eventually{\isacharunderscore}{\kern0pt}mono{\isacharbrackleft}{\kern0pt}OF\ eventually{\isacharunderscore}{\kern0pt}conj{\isacharbrackleft}{\kern0pt}OF\ zero{\isacharunderscore}{\kern0pt}less{\isacharunderscore}{\kern0pt}eps\ eps{\isacharunderscore}{\kern0pt}inf{\isacharbrackleft}{\kern0pt}\isakeyword{where}\ c{\isacharequal}{\kern0pt}{\isachardoublequoteopen}exp\ {\isadigit{1}}{\isachardoublequoteclose}{\isacharbrackright}{\kern0pt}{\isacharbrackright}{\kern0pt}{\isacharbrackright}{\kern0pt}{\isacharparenright}{\kern0pt}\isanewline
\ \ \ \ \isacommand{by}\isamarkupfalse%
\ {\isacharparenleft}{\kern0pt}meson\ abs{\isacharunderscore}{\kern0pt}ge{\isacharunderscore}{\kern0pt}self\ dual{\isacharunderscore}{\kern0pt}order{\isachardot}{\kern0pt}trans\ exp{\isacharunderscore}{\kern0pt}gt{\isacharunderscore}{\kern0pt}zero\ ln{\isacharunderscore}{\kern0pt}ge{\isacharunderscore}{\kern0pt}iff\ order{\isacharunderscore}{\kern0pt}trans{\isacharunderscore}{\kern0pt}rules{\isacharparenleft}{\kern0pt}{\isadigit{2}}{\isadigit{2}}{\isacharparenright}{\kern0pt}{\isacharparenright}{\kern0pt}\isanewline
\isanewline
\ \ \isacommand{have}\isamarkupfalse%
\ unit{\isacharunderscore}{\kern0pt}{\isadigit{1}}{\isadigit{0}}{\isacharcolon}{\kern0pt}\ {\isachardoublequoteopen}{\isacharparenleft}{\kern0pt}{\isasymlambda}{\isacharunderscore}{\kern0pt}{\isachardot}{\kern0pt}\ {\isadigit{1}}{\isacharparenright}{\kern0pt}\ {\isasymin}\ O{\isacharbrackleft}{\kern0pt}{\isacharquery}{\kern0pt}F{\isacharbrackright}{\kern0pt}{\isacharparenleft}{\kern0pt}{\isasymlambda}x{\isachardot}{\kern0pt}\ ln\ {\isacharparenleft}{\kern0pt}real\ {\isacharparenleft}{\kern0pt}n{\isacharunderscore}{\kern0pt}of\ x{\isacharparenright}{\kern0pt}{\isacharparenright}{\kern0pt}{\isacharparenright}{\kern0pt}{\isachardoublequoteclose}\isanewline
\ \ \ \ \isacommand{apply}\isamarkupfalse%
\ {\isacharparenleft}{\kern0pt}rule\ landau{\isacharunderscore}{\kern0pt}o{\isachardot}{\kern0pt}big{\isacharunderscore}{\kern0pt}mono{\isacharcomma}{\kern0pt}\ simp{\isacharparenright}{\kern0pt}\isanewline
\ \ \ \ \isacommand{apply}\isamarkupfalse%
\ {\isacharparenleft}{\kern0pt}rule\ eventually{\isacharunderscore}{\kern0pt}mono\ {\isacharbrackleft}{\kern0pt}OF\ n{\isacharunderscore}{\kern0pt}inf{\isacharbrackleft}{\kern0pt}\isakeyword{where}\ c{\isacharequal}{\kern0pt}{\isachardoublequoteopen}exp\ {\isadigit{1}}{\isachardoublequoteclose}{\isacharbrackright}{\kern0pt}{\isacharbrackright}{\kern0pt}{\isacharparenright}{\kern0pt}\ \isanewline
\ \ \ \ \isacommand{by}\isamarkupfalse%
\ {\isacharparenleft}{\kern0pt}metis\ abs{\isacharunderscore}{\kern0pt}ge{\isacharunderscore}{\kern0pt}self\ linorder{\isacharunderscore}{\kern0pt}not{\isacharunderscore}{\kern0pt}le\ ln{\isacharunderscore}{\kern0pt}ge{\isacharunderscore}{\kern0pt}iff\ not{\isacharunderscore}{\kern0pt}exp{\isacharunderscore}{\kern0pt}le{\isacharunderscore}{\kern0pt}zero\ order{\isachardot}{\kern0pt}trans{\isacharparenright}{\kern0pt}\isanewline
\isanewline
\ \ \isacommand{have}\isamarkupfalse%
\ unit{\isacharunderscore}{\kern0pt}{\isadigit{3}}{\isacharcolon}{\kern0pt}\ {\isachardoublequoteopen}{\isacharparenleft}{\kern0pt}{\isasymlambda}x{\isachardot}{\kern0pt}\ {\isadigit{1}}{\isacharparenright}{\kern0pt}\ {\isasymin}\ O{\isacharbrackleft}{\kern0pt}{\isacharquery}{\kern0pt}F{\isacharbrackright}{\kern0pt}{\isacharparenleft}{\kern0pt}{\isasymlambda}x{\isachardot}{\kern0pt}\ ln\ {\isacharparenleft}{\kern0pt}real\ {\isacharparenleft}{\kern0pt}n{\isacharunderscore}{\kern0pt}of\ x{\isacharparenright}{\kern0pt}{\isacharparenright}{\kern0pt}\ {\isacharplus}{\kern0pt}\ ln\ {\isacharparenleft}{\kern0pt}real\ {\isacharparenleft}{\kern0pt}m{\isacharunderscore}{\kern0pt}of\ x{\isacharparenright}{\kern0pt}{\isacharparenright}{\kern0pt}{\isacharparenright}{\kern0pt}{\isachardoublequoteclose}\ \isanewline
\ \ \ \ \isacommand{apply}\isamarkupfalse%
\ {\isacharparenleft}{\kern0pt}rule\ landau{\isacharunderscore}{\kern0pt}sum{\isacharunderscore}{\kern0pt}{\isadigit{1}}{\isacharparenright}{\kern0pt}\isanewline
\ \ \ \ \ \ \isacommand{apply}\isamarkupfalse%
\ {\isacharparenleft}{\kern0pt}rule\ eventually{\isacharunderscore}{\kern0pt}ln{\isacharunderscore}{\kern0pt}ge{\isacharunderscore}{\kern0pt}iff{\isacharbrackleft}{\kern0pt}OF\ n{\isacharunderscore}{\kern0pt}inf{\isacharbrackright}{\kern0pt}{\isacharparenright}{\kern0pt}\isanewline
\ \ \ \ \ \isacommand{apply}\isamarkupfalse%
\ {\isacharparenleft}{\kern0pt}rule\ eventually{\isacharunderscore}{\kern0pt}ln{\isacharunderscore}{\kern0pt}ge{\isacharunderscore}{\kern0pt}iff{\isacharbrackleft}{\kern0pt}OF\ m{\isacharunderscore}{\kern0pt}inf{\isacharbrackright}{\kern0pt}{\isacharparenright}{\kern0pt}\isanewline
\ \ \ \ \isacommand{by}\isamarkupfalse%
\ {\isacharparenleft}{\kern0pt}rule\ unit{\isacharunderscore}{\kern0pt}{\isadigit{1}}{\isadigit{0}}{\isacharparenright}{\kern0pt}\isanewline
\isanewline
\ \ \isacommand{have}\isamarkupfalse%
\ unit{\isacharunderscore}{\kern0pt}{\isadigit{7}}{\isacharcolon}{\kern0pt}\ {\isachardoublequoteopen}{\isacharparenleft}{\kern0pt}{\isasymlambda}{\isacharunderscore}{\kern0pt}{\isachardot}{\kern0pt}\ {\isadigit{1}}{\isacharparenright}{\kern0pt}\ {\isasymin}\ O{\isacharbrackleft}{\kern0pt}{\isacharquery}{\kern0pt}F{\isacharbrackright}{\kern0pt}{\isacharparenleft}{\kern0pt}{\isasymlambda}x{\isachardot}{\kern0pt}\ {\isadigit{1}}\ {\isacharslash}{\kern0pt}\ {\isacharparenleft}{\kern0pt}real{\isacharunderscore}{\kern0pt}of{\isacharunderscore}{\kern0pt}rat\ {\isacharparenleft}{\kern0pt}{\isasymdelta}{\isacharunderscore}{\kern0pt}of\ x{\isacharparenright}{\kern0pt}{\isacharparenright}{\kern0pt}\isactrlsup {\isadigit{2}}{\isacharparenright}{\kern0pt}{\isachardoublequoteclose}\isanewline
\ \ \ \ \isacommand{apply}\isamarkupfalse%
\ {\isacharparenleft}{\kern0pt}rule\ landau{\isacharunderscore}{\kern0pt}o{\isachardot}{\kern0pt}big{\isacharunderscore}{\kern0pt}mono{\isacharcomma}{\kern0pt}\ simp{\isacharparenright}{\kern0pt}\isanewline
\ \ \ \ \isacommand{apply}\isamarkupfalse%
\ {\isacharparenleft}{\kern0pt}rule\ eventually{\isacharunderscore}{\kern0pt}mono{\isacharbrackleft}{\kern0pt}OF\ eventually{\isacharunderscore}{\kern0pt}conj{\isacharbrackleft}{\kern0pt}OF\ zero{\isacharunderscore}{\kern0pt}less{\isacharunderscore}{\kern0pt}delta\ delta{\isacharunderscore}{\kern0pt}inf{\isacharbrackleft}{\kern0pt}\isakeyword{where}\ c{\isacharequal}{\kern0pt}{\isachardoublequoteopen}{\isadigit{1}}{\isachardoublequoteclose}{\isacharbrackright}{\kern0pt}{\isacharbrackright}{\kern0pt}{\isacharbrackright}{\kern0pt}{\isacharparenright}{\kern0pt}\isanewline
\ \ \ \ \isacommand{by}\isamarkupfalse%
\ {\isacharparenleft}{\kern0pt}metis\ one{\isacharunderscore}{\kern0pt}le{\isacharunderscore}{\kern0pt}power\ power{\isacharunderscore}{\kern0pt}one{\isacharunderscore}{\kern0pt}over{\isacharparenright}{\kern0pt}\isanewline
\isanewline
\ \ \isacommand{have}\isamarkupfalse%
\ unit{\isacharunderscore}{\kern0pt}{\isadigit{4}}{\isacharcolon}{\kern0pt}\ {\isachardoublequoteopen}{\isacharparenleft}{\kern0pt}{\isasymlambda}{\isacharunderscore}{\kern0pt}{\isachardot}{\kern0pt}\ {\isadigit{1}}{\isacharparenright}{\kern0pt}\ {\isasymin}\ O{\isacharbrackleft}{\kern0pt}{\isacharquery}{\kern0pt}F{\isacharbrackright}{\kern0pt}{\isacharparenleft}{\kern0pt}g{\isadigit{1}}{\isacharparenright}{\kern0pt}{\isachardoublequoteclose}\isanewline
\ \ \ \ \isacommand{apply}\isamarkupfalse%
\ {\isacharparenleft}{\kern0pt}simp\ add{\isacharcolon}{\kern0pt}g{\isadigit{1}}{\isacharunderscore}{\kern0pt}def{\isacharparenright}{\kern0pt}\isanewline
\ \ \ \ \isacommand{apply}\isamarkupfalse%
\ {\isacharparenleft}{\kern0pt}subst\ {\isacharparenleft}{\kern0pt}{\isadigit{2}}{\isacharparenright}{\kern0pt}\ div{\isacharunderscore}{\kern0pt}commute{\isacharparenright}{\kern0pt}\isanewline
\ \ \ \ \isacommand{apply}\isamarkupfalse%
\ {\isacharparenleft}{\kern0pt}rule\ landau{\isacharunderscore}{\kern0pt}o{\isachardot}{\kern0pt}big{\isacharunderscore}{\kern0pt}mult{\isacharunderscore}{\kern0pt}{\isadigit{1}}{\isacharbrackleft}{\kern0pt}OF\ unit{\isacharunderscore}{\kern0pt}{\isadigit{7}}{\isacharbrackright}{\kern0pt}{\isacharparenright}{\kern0pt}\isanewline
\ \ \ \ \isacommand{by}\isamarkupfalse%
\ {\isacharparenleft}{\kern0pt}rule\ landau{\isacharunderscore}{\kern0pt}o{\isachardot}{\kern0pt}big{\isacharunderscore}{\kern0pt}mult{\isacharunderscore}{\kern0pt}{\isadigit{1}}{\isacharbrackleft}{\kern0pt}OF\ unit{\isacharunderscore}{\kern0pt}{\isadigit{8}}\ unit{\isacharunderscore}{\kern0pt}{\isadigit{9}}{\isacharbrackright}{\kern0pt}{\isacharparenright}{\kern0pt}\isanewline
\isanewline
\ \ \isacommand{have}\isamarkupfalse%
\ unit{\isacharunderscore}{\kern0pt}{\isadigit{5}}{\isacharcolon}{\kern0pt}\ {\isachardoublequoteopen}{\isacharparenleft}{\kern0pt}{\isasymlambda}{\isacharunderscore}{\kern0pt}{\isachardot}{\kern0pt}\ {\isadigit{1}}{\isacharparenright}{\kern0pt}\ {\isasymin}\ O{\isacharbrackleft}{\kern0pt}{\isacharquery}{\kern0pt}F{\isacharbrackright}{\kern0pt}{\isacharparenleft}{\kern0pt}{\isasymlambda}x{\isachardot}{\kern0pt}\ g{\isadigit{1}}\ x\ {\isacharasterisk}{\kern0pt}\ ln\ {\isacharparenleft}{\kern0pt}{\isadigit{1}}\ {\isacharslash}{\kern0pt}\ real{\isacharunderscore}{\kern0pt}of{\isacharunderscore}{\kern0pt}rat\ {\isacharparenleft}{\kern0pt}{\isasymepsilon}{\isacharunderscore}{\kern0pt}of\ x{\isacharparenright}{\kern0pt}{\isacharparenright}{\kern0pt}{\isacharparenright}{\kern0pt}{\isachardoublequoteclose}\isanewline
\ \ \ \ \isacommand{by}\isamarkupfalse%
\ {\isacharparenleft}{\kern0pt}rule\ landau{\isacharunderscore}{\kern0pt}o{\isachardot}{\kern0pt}big{\isacharunderscore}{\kern0pt}mult{\isacharunderscore}{\kern0pt}{\isadigit{1}}{\isacharbrackleft}{\kern0pt}OF\ unit{\isacharunderscore}{\kern0pt}{\isadigit{4}}\ unit{\isacharunderscore}{\kern0pt}{\isadigit{2}}{\isacharbrackright}{\kern0pt}{\isacharparenright}{\kern0pt}\isanewline
\isanewline
\ \ \isacommand{have}\isamarkupfalse%
\ unit{\isacharunderscore}{\kern0pt}{\isadigit{1}}{\isacharcolon}{\kern0pt}\ {\isachardoublequoteopen}{\isacharparenleft}{\kern0pt}{\isasymlambda}{\isacharunderscore}{\kern0pt}{\isachardot}{\kern0pt}\ {\isadigit{1}}{\isacharparenright}{\kern0pt}\ {\isasymin}\ O{\isacharbrackleft}{\kern0pt}{\isacharquery}{\kern0pt}F{\isacharbrackright}{\kern0pt}{\isacharparenleft}{\kern0pt}g{\isacharparenright}{\kern0pt}{\isachardoublequoteclose}\isanewline
\ \ \ \ \isacommand{apply}\isamarkupfalse%
\ {\isacharparenleft}{\kern0pt}simp\ add{\isacharcolon}{\kern0pt}g{\isacharunderscore}{\kern0pt}def{\isacharparenright}{\kern0pt}\isanewline
\ \ \ \ \isacommand{by}\isamarkupfalse%
\ {\isacharparenleft}{\kern0pt}rule\ landau{\isacharunderscore}{\kern0pt}o{\isachardot}{\kern0pt}big{\isacharunderscore}{\kern0pt}mult{\isacharunderscore}{\kern0pt}{\isadigit{1}}{\isacharbrackleft}{\kern0pt}OF\ unit{\isacharunderscore}{\kern0pt}{\isadigit{5}}\ unit{\isacharunderscore}{\kern0pt}{\isadigit{3}}{\isacharbrackright}{\kern0pt}{\isacharparenright}{\kern0pt}\isanewline
\isanewline
\ \ \isacommand{have}\isamarkupfalse%
\ l{\isadigit{6}}{\isacharcolon}{\kern0pt}\ {\isachardoublequoteopen}{\isacharparenleft}{\kern0pt}{\isasymlambda}x{\isachardot}{\kern0pt}\ real\ {\isacharparenleft}{\kern0pt}nat\ {\isasymlceil}{\isadigit{3}}\ {\isacharasterisk}{\kern0pt}\ real\ {\isacharparenleft}{\kern0pt}k{\isacharunderscore}{\kern0pt}of\ x{\isacharparenright}{\kern0pt}\ {\isacharasterisk}{\kern0pt}\ real\ {\isacharparenleft}{\kern0pt}n{\isacharunderscore}{\kern0pt}of\ x{\isacharparenright}{\kern0pt}\ powr\ {\isacharparenleft}{\kern0pt}{\isadigit{1}}\ {\isacharminus}{\kern0pt}\ {\isadigit{1}}\ {\isacharslash}{\kern0pt}\ real\ {\isacharparenleft}{\kern0pt}k{\isacharunderscore}{\kern0pt}of\ x{\isacharparenright}{\kern0pt}{\isacharparenright}{\kern0pt}\ {\isacharslash}{\kern0pt}\ {\isacharparenleft}{\kern0pt}real{\isacharunderscore}{\kern0pt}of{\isacharunderscore}{\kern0pt}rat\ {\isacharparenleft}{\kern0pt}{\isasymdelta}{\isacharunderscore}{\kern0pt}of\ x{\isacharparenright}{\kern0pt}{\isacharparenright}{\kern0pt}\isactrlsup {\isadigit{2}}{\isasymrceil}{\isacharparenright}{\kern0pt}{\isacharparenright}{\kern0pt}\isanewline
\ \ \ \ {\isasymin}\ O{\isacharbrackleft}{\kern0pt}{\isacharquery}{\kern0pt}F{\isacharbrackright}{\kern0pt}{\isacharparenleft}{\kern0pt}g{\isadigit{1}}{\isacharparenright}{\kern0pt}{\isachardoublequoteclose}\ \isanewline
\ \ \ \ \isacommand{apply}\isamarkupfalse%
\ {\isacharparenleft}{\kern0pt}rule\ landau{\isacharunderscore}{\kern0pt}nat{\isacharunderscore}{\kern0pt}ceil{\isacharbrackleft}{\kern0pt}OF\ unit{\isacharunderscore}{\kern0pt}{\isadigit{4}}{\isacharbrackright}{\kern0pt}{\isacharparenright}{\kern0pt}\isanewline
\ \ \ \ \isacommand{apply}\isamarkupfalse%
\ {\isacharparenleft}{\kern0pt}simp\ add{\isacharcolon}{\kern0pt}g{\isadigit{1}}{\isacharunderscore}{\kern0pt}def{\isacharparenright}{\kern0pt}\isanewline
\ \ \ \ \isacommand{apply}\isamarkupfalse%
\ {\isacharparenleft}{\kern0pt}subst\ {\isacharparenleft}{\kern0pt}{\isadigit{2}}{\isacharparenright}{\kern0pt}\ div{\isacharunderscore}{\kern0pt}commute{\isacharcomma}{\kern0pt}\ subst\ {\isacharparenleft}{\kern0pt}{\isadigit{4}}{\isacharparenright}{\kern0pt}\ div{\isacharunderscore}{\kern0pt}commute{\isacharparenright}{\kern0pt}\isanewline
\ \ \ \ \isacommand{apply}\isamarkupfalse%
\ {\isacharparenleft}{\kern0pt}rule\ landau{\isacharunderscore}{\kern0pt}o{\isachardot}{\kern0pt}mult{\isacharcomma}{\kern0pt}\ simp{\isacharparenright}{\kern0pt}\isanewline
\ \ \ \ \isacommand{by}\isamarkupfalse%
\ simp\isanewline
\isanewline
\ \ \isacommand{have}\isamarkupfalse%
\ l{\isadigit{9}}{\isacharcolon}{\kern0pt}\ {\isachardoublequoteopen}{\isacharparenleft}{\kern0pt}{\isasymlambda}x{\isachardot}{\kern0pt}\ real\ {\isacharparenleft}{\kern0pt}nat\ {\isasymlceil}{\isacharminus}{\kern0pt}\ {\isacharparenleft}{\kern0pt}{\isadigit{1}}{\isadigit{8}}\ {\isacharasterisk}{\kern0pt}\ ln\ {\isacharparenleft}{\kern0pt}real{\isacharunderscore}{\kern0pt}of{\isacharunderscore}{\kern0pt}rat\ {\isacharparenleft}{\kern0pt}{\isasymepsilon}{\isacharunderscore}{\kern0pt}of\ x{\isacharparenright}{\kern0pt}{\isacharparenright}{\kern0pt}{\isacharparenright}{\kern0pt}{\isasymrceil}{\isacharparenright}{\kern0pt}{\isacharparenright}{\kern0pt}\isanewline
\ \ \ \ {\isasymin}\ O{\isacharbrackleft}{\kern0pt}{\isacharquery}{\kern0pt}F{\isacharbrackright}{\kern0pt}{\isacharparenleft}{\kern0pt}{\isasymlambda}x{\isachardot}{\kern0pt}\ ln\ {\isacharparenleft}{\kern0pt}{\isadigit{1}}\ {\isacharslash}{\kern0pt}\ real{\isacharunderscore}{\kern0pt}of{\isacharunderscore}{\kern0pt}rat\ {\isacharparenleft}{\kern0pt}{\isasymepsilon}{\isacharunderscore}{\kern0pt}of\ x{\isacharparenright}{\kern0pt}{\isacharparenright}{\kern0pt}{\isacharparenright}{\kern0pt}{\isachardoublequoteclose}\ \isanewline
\ \ \ \ \ \isacommand{apply}\isamarkupfalse%
\ {\isacharparenleft}{\kern0pt}rule\ landau{\isacharunderscore}{\kern0pt}nat{\isacharunderscore}{\kern0pt}ceil{\isacharbrackleft}{\kern0pt}OF\ unit{\isacharunderscore}{\kern0pt}{\isadigit{2}}{\isacharbrackright}{\kern0pt}{\isacharparenright}{\kern0pt}\isanewline
\ \ \ \ \isacommand{apply}\isamarkupfalse%
\ {\isacharparenleft}{\kern0pt}subst\ minus{\isacharunderscore}{\kern0pt}mult{\isacharunderscore}{\kern0pt}right{\isacharparenright}{\kern0pt}\isanewline
\ \ \ \ \ \ \isacommand{apply}\isamarkupfalse%
\ {\isacharparenleft}{\kern0pt}subst\ cmult{\isacharunderscore}{\kern0pt}in{\isacharunderscore}{\kern0pt}bigo{\isacharunderscore}{\kern0pt}iff{\isacharcomma}{\kern0pt}\ rule\ disjI{\isadigit{2}}{\isacharparenright}{\kern0pt}\isanewline
\ \ \ \ \ \ \isacommand{apply}\isamarkupfalse%
\ {\isacharparenleft}{\kern0pt}subst\ landau{\isacharunderscore}{\kern0pt}o{\isachardot}{\kern0pt}big{\isachardot}{\kern0pt}in{\isacharunderscore}{\kern0pt}cong{\isacharbrackleft}{\kern0pt}\isakeyword{where}\ g{\isacharequal}{\kern0pt}{\isachardoublequoteopen}{\isasymlambda}x{\isachardot}{\kern0pt}\ ln{\isacharparenleft}{\kern0pt}\ {\isadigit{1}}\ {\isacharslash}{\kern0pt}\ {\isacharparenleft}{\kern0pt}real{\isacharunderscore}{\kern0pt}of{\isacharunderscore}{\kern0pt}rat\ {\isacharparenleft}{\kern0pt}{\isasymepsilon}{\isacharunderscore}{\kern0pt}of\ x{\isacharparenright}{\kern0pt}{\isacharparenright}{\kern0pt}{\isacharparenright}{\kern0pt}{\isachardoublequoteclose}{\isacharbrackright}{\kern0pt}{\isacharparenright}{\kern0pt}\isanewline
\ \ \ \ \ \ \ \isacommand{apply}\isamarkupfalse%
\ {\isacharparenleft}{\kern0pt}rule\ eventually{\isacharunderscore}{\kern0pt}mono{\isacharbrackleft}{\kern0pt}OF\ zero{\isacharunderscore}{\kern0pt}less{\isacharunderscore}{\kern0pt}eps{\isacharbrackright}{\kern0pt}{\isacharparenright}{\kern0pt}\isanewline
\ \ \ \ \isacommand{by}\isamarkupfalse%
\ {\isacharparenleft}{\kern0pt}subst\ ln{\isacharunderscore}{\kern0pt}div{\isacharcomma}{\kern0pt}\ simp{\isacharcomma}{\kern0pt}\ simp{\isacharcomma}{\kern0pt}\ simp{\isacharcomma}{\kern0pt}\ simp{\isacharparenright}{\kern0pt}\isanewline
\isanewline
\ \ \isacommand{have}\isamarkupfalse%
\ l{\isadigit{1}}{\isacharcolon}{\kern0pt}\ {\isachardoublequoteopen}{\isacharparenleft}{\kern0pt}{\isasymlambda}x{\isachardot}{\kern0pt}\ real\ {\isacharparenleft}{\kern0pt}nat\ {\isasymlceil}{\isadigit{3}}\ {\isacharasterisk}{\kern0pt}\ real\ {\isacharparenleft}{\kern0pt}k{\isacharunderscore}{\kern0pt}of\ x{\isacharparenright}{\kern0pt}\ {\isacharasterisk}{\kern0pt}\ real\ {\isacharparenleft}{\kern0pt}n{\isacharunderscore}{\kern0pt}of\ x{\isacharparenright}{\kern0pt}\ powr\ {\isacharparenleft}{\kern0pt}{\isadigit{1}}\ {\isacharminus}{\kern0pt}\ {\isadigit{1}}\ {\isacharslash}{\kern0pt}\ real\ {\isacharparenleft}{\kern0pt}k{\isacharunderscore}{\kern0pt}of\ x{\isacharparenright}{\kern0pt}{\isacharparenright}{\kern0pt}\ {\isacharslash}{\kern0pt}\ {\isacharparenleft}{\kern0pt}real{\isacharunderscore}{\kern0pt}of{\isacharunderscore}{\kern0pt}rat\ {\isacharparenleft}{\kern0pt}{\isasymdelta}{\isacharunderscore}{\kern0pt}of\ x{\isacharparenright}{\kern0pt}{\isacharparenright}{\kern0pt}\isactrlsup {\isadigit{2}}{\isasymrceil}{\isacharparenright}{\kern0pt}\ {\isacharasterisk}{\kern0pt}\isanewline
\ \ \ \ \ \ \ \ \ \ real\ {\isacharparenleft}{\kern0pt}nat\ {\isasymlceil}{\isacharminus}{\kern0pt}\ {\isacharparenleft}{\kern0pt}{\isadigit{1}}{\isadigit{8}}\ {\isacharasterisk}{\kern0pt}\ ln\ {\isacharparenleft}{\kern0pt}real{\isacharunderscore}{\kern0pt}of{\isacharunderscore}{\kern0pt}rat\ {\isacharparenleft}{\kern0pt}{\isasymepsilon}{\isacharunderscore}{\kern0pt}of\ x{\isacharparenright}{\kern0pt}{\isacharparenright}{\kern0pt}{\isacharparenright}{\kern0pt}{\isasymrceil}{\isacharparenright}{\kern0pt}\ {\isacharasterisk}{\kern0pt}\isanewline
\ \ \ \ \ \ \ \ \ \ {\isacharparenleft}{\kern0pt}{\isadigit{3}}\ {\isacharplus}{\kern0pt}\ {\isadigit{2}}\ {\isacharasterisk}{\kern0pt}\ log\ {\isadigit{2}}\ {\isacharparenleft}{\kern0pt}real\ {\isacharparenleft}{\kern0pt}n{\isacharunderscore}{\kern0pt}of\ x{\isacharparenright}{\kern0pt}\ {\isacharplus}{\kern0pt}\ {\isadigit{1}}{\isacharparenright}{\kern0pt}\ {\isacharplus}{\kern0pt}\ {\isadigit{2}}\ {\isacharasterisk}{\kern0pt}\ log\ {\isadigit{2}}\ {\isacharparenleft}{\kern0pt}real\ {\isacharparenleft}{\kern0pt}m{\isacharunderscore}{\kern0pt}of\ x{\isacharparenright}{\kern0pt}\ {\isacharplus}{\kern0pt}\ {\isadigit{1}}{\isacharparenright}{\kern0pt}{\isacharparenright}{\kern0pt}{\isacharparenright}{\kern0pt}\ {\isasymin}\ O{\isacharbrackleft}{\kern0pt}{\isacharquery}{\kern0pt}F{\isacharbrackright}{\kern0pt}{\isacharparenleft}{\kern0pt}g{\isacharparenright}{\kern0pt}{\isachardoublequoteclose}\isanewline
\ \ \ \ \isacommand{apply}\isamarkupfalse%
\ {\isacharparenleft}{\kern0pt}simp\ add{\isacharcolon}{\kern0pt}g{\isacharunderscore}{\kern0pt}def{\isacharparenright}{\kern0pt}\isanewline
\ \ \ \ \isacommand{apply}\isamarkupfalse%
\ {\isacharparenleft}{\kern0pt}rule\ landau{\isacharunderscore}{\kern0pt}o{\isachardot}{\kern0pt}mult{\isacharparenright}{\kern0pt}\isanewline
\ \ \ \ \ \isacommand{apply}\isamarkupfalse%
\ {\isacharparenleft}{\kern0pt}rule\ landau{\isacharunderscore}{\kern0pt}o{\isachardot}{\kern0pt}mult{\isacharcomma}{\kern0pt}\ simp\ add{\isacharcolon}{\kern0pt}l{\isadigit{6}}{\isacharcomma}{\kern0pt}\ simp\ add{\isacharcolon}{\kern0pt}l{\isadigit{9}}{\isacharparenright}{\kern0pt}\isanewline
\ \ \ \ \isacommand{apply}\isamarkupfalse%
\ {\isacharparenleft}{\kern0pt}rule\ sum{\isacharunderscore}{\kern0pt}in{\isacharunderscore}{\kern0pt}bigo{\isacharparenright}{\kern0pt}\isanewline
\ \ \ \ \ \isacommand{apply}\isamarkupfalse%
\ {\isacharparenleft}{\kern0pt}rule\ sum{\isacharunderscore}{\kern0pt}in{\isacharunderscore}{\kern0pt}bigo{\isacharcomma}{\kern0pt}\ simp\ add{\isacharcolon}{\kern0pt}unit{\isacharunderscore}{\kern0pt}{\isadigit{3}}{\isacharparenright}{\kern0pt}\isanewline
\ \ \ \ \ \isacommand{apply}\isamarkupfalse%
\ {\isacharparenleft}{\kern0pt}simp\ add{\isacharcolon}{\kern0pt}log{\isacharunderscore}{\kern0pt}def{\isacharparenright}{\kern0pt}\isanewline
\ \ \ \ \ \isacommand{apply}\isamarkupfalse%
\ {\isacharparenleft}{\kern0pt}rule\ landau{\isacharunderscore}{\kern0pt}sum{\isacharunderscore}{\kern0pt}{\isadigit{1}}\ {\isacharbrackleft}{\kern0pt}OF\ eventually{\isacharunderscore}{\kern0pt}ln{\isacharunderscore}{\kern0pt}ge{\isacharunderscore}{\kern0pt}iff{\isacharbrackleft}{\kern0pt}OF\ n{\isacharunderscore}{\kern0pt}inf{\isacharbrackright}{\kern0pt}\ eventually{\isacharunderscore}{\kern0pt}ln{\isacharunderscore}{\kern0pt}ge{\isacharunderscore}{\kern0pt}iff{\isacharbrackleft}{\kern0pt}OF\ m{\isacharunderscore}{\kern0pt}inf{\isacharbrackright}{\kern0pt}{\isacharbrackright}{\kern0pt}{\isacharparenright}{\kern0pt}\isanewline
\ \ \ \ \ \isacommand{apply}\isamarkupfalse%
\ {\isacharparenleft}{\kern0pt}rule\ landau{\isacharunderscore}{\kern0pt}ln{\isacharunderscore}{\kern0pt}{\isadigit{2}}{\isacharbrackleft}{\kern0pt}\isakeyword{where}\ a{\isacharequal}{\kern0pt}{\isachardoublequoteopen}{\isadigit{2}}{\isachardoublequoteclose}{\isacharbrackright}{\kern0pt}{\isacharcomma}{\kern0pt}\ simp{\isacharcomma}{\kern0pt}\ simp{\isacharcomma}{\kern0pt}\ rule\ n{\isacharunderscore}{\kern0pt}inf{\isacharparenright}{\kern0pt}\isanewline
\ \ \ \ \isacommand{apply}\isamarkupfalse%
\ {\isacharparenleft}{\kern0pt}rule\ sum{\isacharunderscore}{\kern0pt}in{\isacharunderscore}{\kern0pt}bigo{\isacharcomma}{\kern0pt}\ simp{\isacharcomma}{\kern0pt}\ simp\ add{\isacharcolon}{\kern0pt}unit{\isacharunderscore}{\kern0pt}n{\isacharparenright}{\kern0pt}\isanewline
\ \ \ \ \isacommand{apply}\isamarkupfalse%
\ {\isacharparenleft}{\kern0pt}simp\ add{\isacharcolon}{\kern0pt}log{\isacharunderscore}{\kern0pt}def{\isacharparenright}{\kern0pt}\isanewline
\ \ \ \ \ \isacommand{apply}\isamarkupfalse%
\ {\isacharparenleft}{\kern0pt}rule\ landau{\isacharunderscore}{\kern0pt}sum{\isacharunderscore}{\kern0pt}{\isadigit{2}}\ {\isacharbrackleft}{\kern0pt}OF\ eventually{\isacharunderscore}{\kern0pt}ln{\isacharunderscore}{\kern0pt}ge{\isacharunderscore}{\kern0pt}iff{\isacharbrackleft}{\kern0pt}OF\ n{\isacharunderscore}{\kern0pt}inf{\isacharbrackright}{\kern0pt}\ eventually{\isacharunderscore}{\kern0pt}ln{\isacharunderscore}{\kern0pt}ge{\isacharunderscore}{\kern0pt}iff{\isacharbrackleft}{\kern0pt}OF\ m{\isacharunderscore}{\kern0pt}inf{\isacharbrackright}{\kern0pt}{\isacharbrackright}{\kern0pt}{\isacharparenright}{\kern0pt}\isanewline
\ \ \ \ \isacommand{apply}\isamarkupfalse%
\ {\isacharparenleft}{\kern0pt}rule\ landau{\isacharunderscore}{\kern0pt}ln{\isacharunderscore}{\kern0pt}{\isadigit{2}}{\isacharbrackleft}{\kern0pt}\isakeyword{where}\ a{\isacharequal}{\kern0pt}{\isachardoublequoteopen}{\isadigit{2}}{\isachardoublequoteclose}{\isacharbrackright}{\kern0pt}{\isacharcomma}{\kern0pt}\ simp{\isacharcomma}{\kern0pt}\ simp{\isacharcomma}{\kern0pt}\ rule\ m{\isacharunderscore}{\kern0pt}inf{\isacharparenright}{\kern0pt}\isanewline
\ \ \ \ \isacommand{by}\isamarkupfalse%
\ {\isacharparenleft}{\kern0pt}rule\ sum{\isacharunderscore}{\kern0pt}in{\isacharunderscore}{\kern0pt}bigo{\isacharcomma}{\kern0pt}\ simp{\isacharcomma}{\kern0pt}\ simp\ add{\isacharcolon}{\kern0pt}unit{\isacharunderscore}{\kern0pt}{\isadigit{6}}{\isacharparenright}{\kern0pt}\isanewline
\isanewline
\ \ \isacommand{have}\isamarkupfalse%
\ l{\isadigit{2}}{\isacharcolon}{\kern0pt}\ {\isachardoublequoteopen}{\isacharparenleft}{\kern0pt}{\isasymlambda}x{\isachardot}{\kern0pt}\ ln\ {\isacharparenleft}{\kern0pt}real\ {\isacharparenleft}{\kern0pt}m{\isacharunderscore}{\kern0pt}of\ x{\isacharparenright}{\kern0pt}\ {\isacharplus}{\kern0pt}\ {\isadigit{1}}{\isacharparenright}{\kern0pt}{\isacharparenright}{\kern0pt}\ {\isasymin}\ O{\isacharbrackleft}{\kern0pt}{\isacharquery}{\kern0pt}F{\isacharbrackright}{\kern0pt}{\isacharparenleft}{\kern0pt}g{\isacharparenright}{\kern0pt}{\isachardoublequoteclose}\isanewline
\ \ \ \ \isacommand{apply}\isamarkupfalse%
\ {\isacharparenleft}{\kern0pt}simp\ add{\isacharcolon}{\kern0pt}g{\isacharunderscore}{\kern0pt}def{\isacharparenright}{\kern0pt}\isanewline
\ \ \ \ \isacommand{apply}\isamarkupfalse%
\ {\isacharparenleft}{\kern0pt}rule\ landau{\isacharunderscore}{\kern0pt}o{\isachardot}{\kern0pt}big{\isacharunderscore}{\kern0pt}mult{\isacharunderscore}{\kern0pt}{\isadigit{1}}{\isacharprime}{\kern0pt}{\isacharbrackleft}{\kern0pt}OF\ unit{\isacharunderscore}{\kern0pt}{\isadigit{5}}{\isacharbrackright}{\kern0pt}{\isacharparenright}{\kern0pt}\isanewline
\ \ \ \ \isacommand{apply}\isamarkupfalse%
\ {\isacharparenleft}{\kern0pt}rule\ landau{\isacharunderscore}{\kern0pt}sum{\isacharunderscore}{\kern0pt}{\isadigit{2}}\ {\isacharbrackleft}{\kern0pt}OF\ eventually{\isacharunderscore}{\kern0pt}ln{\isacharunderscore}{\kern0pt}ge{\isacharunderscore}{\kern0pt}iff{\isacharbrackleft}{\kern0pt}OF\ n{\isacharunderscore}{\kern0pt}inf{\isacharbrackright}{\kern0pt}\ eventually{\isacharunderscore}{\kern0pt}ln{\isacharunderscore}{\kern0pt}ge{\isacharunderscore}{\kern0pt}iff{\isacharbrackleft}{\kern0pt}OF\ m{\isacharunderscore}{\kern0pt}inf{\isacharbrackright}{\kern0pt}{\isacharbrackright}{\kern0pt}{\isacharparenright}{\kern0pt}\isanewline
\ \ \ \ \isacommand{apply}\isamarkupfalse%
\ {\isacharparenleft}{\kern0pt}rule\ landau{\isacharunderscore}{\kern0pt}ln{\isacharunderscore}{\kern0pt}{\isadigit{2}}{\isacharbrackleft}{\kern0pt}\isakeyword{where}\ a{\isacharequal}{\kern0pt}{\isachardoublequoteopen}{\isadigit{2}}{\isachardoublequoteclose}{\isacharbrackright}{\kern0pt}{\isacharcomma}{\kern0pt}\ simp{\isacharcomma}{\kern0pt}\ simp{\isacharcomma}{\kern0pt}\ rule\ m{\isacharunderscore}{\kern0pt}inf{\isacharparenright}{\kern0pt}\isanewline
\ \ \ \ \isacommand{by}\isamarkupfalse%
\ {\isacharparenleft}{\kern0pt}rule\ sum{\isacharunderscore}{\kern0pt}in{\isacharunderscore}{\kern0pt}bigo{\isacharcomma}{\kern0pt}\ simp{\isacharcomma}{\kern0pt}\ rule\ unit{\isacharunderscore}{\kern0pt}{\isadigit{6}}{\isacharparenright}{\kern0pt}\isanewline
\isanewline
\ \ \isacommand{have}\isamarkupfalse%
\ l{\isadigit{7}}{\isacharcolon}{\kern0pt}\ {\isachardoublequoteopen}{\isacharparenleft}{\kern0pt}{\isasymlambda}x{\isachardot}{\kern0pt}\ ln\ {\isacharparenleft}{\kern0pt}real\ {\isacharparenleft}{\kern0pt}k{\isacharunderscore}{\kern0pt}of\ x{\isacharparenright}{\kern0pt}\ {\isacharplus}{\kern0pt}\ {\isadigit{1}}{\isacharparenright}{\kern0pt}{\isacharparenright}{\kern0pt}\ {\isasymin}\ O{\isacharbrackleft}{\kern0pt}{\isacharquery}{\kern0pt}F{\isacharbrackright}{\kern0pt}{\isacharparenleft}{\kern0pt}g{\isadigit{1}}{\isacharparenright}{\kern0pt}{\isachardoublequoteclose}\isanewline
\ \ \ \ \isacommand{apply}\isamarkupfalse%
\ {\isacharparenleft}{\kern0pt}simp\ add{\isacharcolon}{\kern0pt}g{\isadigit{1}}{\isacharunderscore}{\kern0pt}def{\isacharparenright}{\kern0pt}\isanewline
\ \ \ \ \isacommand{apply}\isamarkupfalse%
\ {\isacharparenleft}{\kern0pt}subst\ {\isacharparenleft}{\kern0pt}{\isadigit{2}}{\isacharparenright}{\kern0pt}\ div{\isacharunderscore}{\kern0pt}commute{\isacharparenright}{\kern0pt}\isanewline
\ \ \ \ \isacommand{apply}\isamarkupfalse%
\ {\isacharparenleft}{\kern0pt}rule\ landau{\isacharunderscore}{\kern0pt}o{\isachardot}{\kern0pt}big{\isacharunderscore}{\kern0pt}mult{\isacharunderscore}{\kern0pt}{\isadigit{1}}{\isacharprime}{\kern0pt}{\isacharbrackleft}{\kern0pt}OF\ unit{\isacharunderscore}{\kern0pt}{\isadigit{7}}{\isacharbrackright}{\kern0pt}{\isacharparenright}{\kern0pt}\isanewline
\ \ \ \ \isacommand{apply}\isamarkupfalse%
\ {\isacharparenleft}{\kern0pt}rule\ landau{\isacharunderscore}{\kern0pt}o{\isachardot}{\kern0pt}big{\isacharunderscore}{\kern0pt}mult{\isacharunderscore}{\kern0pt}{\isadigit{1}}{\isacharparenright}{\kern0pt}\isanewline
\ \ \ \ \ \isacommand{apply}\isamarkupfalse%
\ {\isacharparenleft}{\kern0pt}rule\ landau{\isacharunderscore}{\kern0pt}ln{\isacharunderscore}{\kern0pt}{\isadigit{3}}{\isacharcomma}{\kern0pt}\ simp{\isacharparenright}{\kern0pt}\isanewline
\ \ \ \ \isacommand{by}\isamarkupfalse%
\ {\isacharparenleft}{\kern0pt}rule\ sum{\isacharunderscore}{\kern0pt}in{\isacharunderscore}{\kern0pt}bigo{\isacharcomma}{\kern0pt}\ simp{\isacharcomma}{\kern0pt}\ simp\ add{\isacharcolon}{\kern0pt}unit{\isacharunderscore}{\kern0pt}{\isadigit{8}}{\isacharcomma}{\kern0pt}\ simp\ add{\isacharcolon}{\kern0pt}\ unit{\isacharunderscore}{\kern0pt}{\isadigit{9}}{\isacharparenright}{\kern0pt}\isanewline
\isanewline
\ \ \isacommand{have}\isamarkupfalse%
\ l{\isadigit{3}}{\isacharcolon}{\kern0pt}\ {\isachardoublequoteopen}{\isacharparenleft}{\kern0pt}{\isasymlambda}x{\isachardot}{\kern0pt}\ ln\ {\isacharparenleft}{\kern0pt}real\ {\isacharparenleft}{\kern0pt}k{\isacharunderscore}{\kern0pt}of\ x{\isacharparenright}{\kern0pt}\ {\isacharplus}{\kern0pt}\ {\isadigit{1}}{\isacharparenright}{\kern0pt}{\isacharparenright}{\kern0pt}\ {\isasymin}\ O{\isacharbrackleft}{\kern0pt}{\isacharquery}{\kern0pt}F{\isacharbrackright}{\kern0pt}{\isacharparenleft}{\kern0pt}g{\isacharparenright}{\kern0pt}{\isachardoublequoteclose}\isanewline
\ \ \ \ \isacommand{apply}\isamarkupfalse%
\ {\isacharparenleft}{\kern0pt}simp\ add{\isacharcolon}{\kern0pt}g{\isacharunderscore}{\kern0pt}def{\isacharparenright}{\kern0pt}\isanewline
\ \ \ \ \isacommand{apply}\isamarkupfalse%
\ {\isacharparenleft}{\kern0pt}rule\ landau{\isacharunderscore}{\kern0pt}o{\isachardot}{\kern0pt}big{\isacharunderscore}{\kern0pt}mult{\isacharunderscore}{\kern0pt}{\isadigit{1}}{\isacharparenright}{\kern0pt}\isanewline
\ \ \ \ \isacommand{apply}\isamarkupfalse%
\ {\isacharparenleft}{\kern0pt}rule\ landau{\isacharunderscore}{\kern0pt}o{\isachardot}{\kern0pt}big{\isacharunderscore}{\kern0pt}mult{\isacharunderscore}{\kern0pt}{\isadigit{1}}{\isacharparenright}{\kern0pt}\isanewline
\ \ \ \ \ \ \isacommand{apply}\isamarkupfalse%
\ {\isacharparenleft}{\kern0pt}simp\ add{\isacharcolon}{\kern0pt}l{\isadigit{7}}{\isacharparenright}{\kern0pt}\isanewline
\ \ \ \ \isacommand{by}\isamarkupfalse%
\ {\isacharparenleft}{\kern0pt}rule\ unit{\isacharunderscore}{\kern0pt}{\isadigit{2}}{\isacharcomma}{\kern0pt}\ rule\ unit{\isacharunderscore}{\kern0pt}{\isadigit{3}}{\isacharparenright}{\kern0pt}\isanewline
\isanewline
\ \ \isacommand{have}\isamarkupfalse%
\ l{\isadigit{4}}{\isacharcolon}{\kern0pt}\ {\isachardoublequoteopen}\ {\isacharparenleft}{\kern0pt}{\isasymlambda}x{\isachardot}{\kern0pt}\ ln\ {\isacharparenleft}{\kern0pt}real\ {\isacharparenleft}{\kern0pt}nat\ {\isasymlceil}{\isacharminus}{\kern0pt}\ {\isacharparenleft}{\kern0pt}{\isadigit{1}}{\isadigit{8}}\ {\isacharasterisk}{\kern0pt}\ ln\ {\isacharparenleft}{\kern0pt}real{\isacharunderscore}{\kern0pt}of{\isacharunderscore}{\kern0pt}rat\ {\isacharparenleft}{\kern0pt}{\isasymepsilon}{\isacharunderscore}{\kern0pt}of\ x{\isacharparenright}{\kern0pt}{\isacharparenright}{\kern0pt}{\isacharparenright}{\kern0pt}{\isasymrceil}{\isacharparenright}{\kern0pt}\ {\isacharplus}{\kern0pt}\ {\isadigit{1}}{\isacharparenright}{\kern0pt}{\isacharparenright}{\kern0pt}\ {\isasymin}\ O{\isacharbrackleft}{\kern0pt}{\isacharquery}{\kern0pt}F{\isacharbrackright}{\kern0pt}{\isacharparenleft}{\kern0pt}g{\isacharparenright}{\kern0pt}{\isachardoublequoteclose}\isanewline
\ \ \ \ \isacommand{apply}\isamarkupfalse%
\ {\isacharparenleft}{\kern0pt}simp\ add{\isacharcolon}{\kern0pt}g{\isacharunderscore}{\kern0pt}def{\isacharparenright}{\kern0pt}\isanewline
\ \ \ \ \isacommand{apply}\isamarkupfalse%
\ {\isacharparenleft}{\kern0pt}rule\ landau{\isacharunderscore}{\kern0pt}o{\isachardot}{\kern0pt}big{\isacharunderscore}{\kern0pt}mult{\isacharunderscore}{\kern0pt}{\isadigit{1}}{\isacharparenright}{\kern0pt}\isanewline
\ \ \ \ \ \isacommand{apply}\isamarkupfalse%
\ {\isacharparenleft}{\kern0pt}rule\ landau{\isacharunderscore}{\kern0pt}o{\isachardot}{\kern0pt}big{\isacharunderscore}{\kern0pt}mult{\isacharunderscore}{\kern0pt}{\isadigit{1}}{\isacharprime}{\kern0pt}{\isacharbrackleft}{\kern0pt}OF\ unit{\isacharunderscore}{\kern0pt}{\isadigit{4}}{\isacharbrackright}{\kern0pt}{\isacharparenright}{\kern0pt}\isanewline
\ \ \ \ \ \isacommand{apply}\isamarkupfalse%
\ {\isacharparenleft}{\kern0pt}rule\ landau{\isacharunderscore}{\kern0pt}ln{\isacharunderscore}{\kern0pt}{\isadigit{3}}{\isacharcomma}{\kern0pt}\ simp{\isacharparenright}{\kern0pt}\isanewline
\ \ \ \ \isacommand{by}\isamarkupfalse%
\ {\isacharparenleft}{\kern0pt}rule\ sum{\isacharunderscore}{\kern0pt}in{\isacharunderscore}{\kern0pt}bigo{\isacharcomma}{\kern0pt}\ simp\ add{\isacharcolon}{\kern0pt}l{\isadigit{9}}{\isacharcomma}{\kern0pt}\ rule\ unit{\isacharunderscore}{\kern0pt}{\isadigit{2}}{\isacharcomma}{\kern0pt}\ rule\ unit{\isacharunderscore}{\kern0pt}{\isadigit{3}}{\isacharparenright}{\kern0pt}\isanewline
\isanewline
\ \ \isacommand{have}\isamarkupfalse%
\ l{\isadigit{5}}{\isacharcolon}{\kern0pt}\ {\isachardoublequoteopen}{\isacharparenleft}{\kern0pt}{\isasymlambda}x{\isachardot}{\kern0pt}\ ln\ {\isacharparenleft}{\kern0pt}real\ {\isacharparenleft}{\kern0pt}nat\ {\isasymlceil}{\isadigit{3}}\ {\isacharasterisk}{\kern0pt}\ real\ {\isacharparenleft}{\kern0pt}k{\isacharunderscore}{\kern0pt}of\ x{\isacharparenright}{\kern0pt}\ {\isacharasterisk}{\kern0pt}\ real\ {\isacharparenleft}{\kern0pt}n{\isacharunderscore}{\kern0pt}of\ x{\isacharparenright}{\kern0pt}\ powr\ {\isacharparenleft}{\kern0pt}{\isadigit{1}}\ {\isacharminus}{\kern0pt}\ {\isadigit{1}}\ {\isacharslash}{\kern0pt}\ real\ {\isacharparenleft}{\kern0pt}k{\isacharunderscore}{\kern0pt}of\ x{\isacharparenright}{\kern0pt}{\isacharparenright}{\kern0pt}\ {\isacharslash}{\kern0pt}\ {\isacharparenleft}{\kern0pt}real{\isacharunderscore}{\kern0pt}of{\isacharunderscore}{\kern0pt}rat\ {\isacharparenleft}{\kern0pt}{\isasymdelta}{\isacharunderscore}{\kern0pt}of\ x{\isacharparenright}{\kern0pt}{\isacharparenright}{\kern0pt}\isactrlsup {\isadigit{2}}{\isasymrceil}{\isacharparenright}{\kern0pt}\ {\isacharplus}{\kern0pt}\ {\isadigit{1}}{\isacharparenright}{\kern0pt}{\isacharparenright}{\kern0pt}\ \isanewline
\ \ \ \ {\isasymin}\ O{\isacharbrackleft}{\kern0pt}{\isacharquery}{\kern0pt}F{\isacharbrackright}{\kern0pt}{\isacharparenleft}{\kern0pt}g{\isacharparenright}{\kern0pt}{\isachardoublequoteclose}\isanewline
\ \ \ \ \isacommand{apply}\isamarkupfalse%
\ {\isacharparenleft}{\kern0pt}rule\ landau{\isacharunderscore}{\kern0pt}ln{\isacharunderscore}{\kern0pt}{\isadigit{3}}{\isacharcomma}{\kern0pt}\ simp{\isacharparenright}{\kern0pt}\isanewline
\ \ \ \ \isacommand{apply}\isamarkupfalse%
\ {\isacharparenleft}{\kern0pt}rule\ sum{\isacharunderscore}{\kern0pt}in{\isacharunderscore}{\kern0pt}bigo{\isacharparenright}{\kern0pt}\isanewline
\ \ \ \ \ \isacommand{apply}\isamarkupfalse%
\ {\isacharparenleft}{\kern0pt}simp\ add{\isacharcolon}{\kern0pt}g{\isacharunderscore}{\kern0pt}def{\isacharparenright}{\kern0pt}\isanewline
\ \ \ \ \ \isacommand{apply}\isamarkupfalse%
\ {\isacharparenleft}{\kern0pt}rule\ landau{\isacharunderscore}{\kern0pt}o{\isachardot}{\kern0pt}big{\isacharunderscore}{\kern0pt}mult{\isacharunderscore}{\kern0pt}{\isadigit{1}}{\isacharparenright}{\kern0pt}\isanewline
\ \ \ \ \ \isacommand{apply}\isamarkupfalse%
\ {\isacharparenleft}{\kern0pt}rule\ landau{\isacharunderscore}{\kern0pt}o{\isachardot}{\kern0pt}big{\isacharunderscore}{\kern0pt}mult{\isacharunderscore}{\kern0pt}{\isadigit{1}}{\isacharparenright}{\kern0pt}\isanewline
\ \ \ \ \ \ \ \isacommand{apply}\isamarkupfalse%
\ {\isacharparenleft}{\kern0pt}simp\ add{\isacharcolon}{\kern0pt}l{\isadigit{6}}{\isacharparenright}{\kern0pt}\isanewline
\ \ \ \ \isacommand{by}\isamarkupfalse%
\ {\isacharparenleft}{\kern0pt}rule\ unit{\isacharunderscore}{\kern0pt}{\isadigit{2}}{\isacharcomma}{\kern0pt}\ rule\ unit{\isacharunderscore}{\kern0pt}{\isadigit{3}}{\isacharcomma}{\kern0pt}\ rule\ unit{\isacharunderscore}{\kern0pt}{\isadigit{1}}{\isacharparenright}{\kern0pt}\isanewline
\isanewline
\ \ \isacommand{have}\isamarkupfalse%
\ {\isachardoublequoteopen}fk{\isacharunderscore}{\kern0pt}space{\isacharunderscore}{\kern0pt}usage\ {\isacharequal}{\kern0pt}\ {\isacharparenleft}{\kern0pt}{\isasymlambda}x{\isachardot}{\kern0pt}\ fk{\isacharunderscore}{\kern0pt}space{\isacharunderscore}{\kern0pt}usage\ {\isacharparenleft}{\kern0pt}k{\isacharunderscore}{\kern0pt}of\ x{\isacharcomma}{\kern0pt}\ n{\isacharunderscore}{\kern0pt}of\ x{\isacharcomma}{\kern0pt}\ m{\isacharunderscore}{\kern0pt}of\ x{\isacharcomma}{\kern0pt}\ {\isasymepsilon}{\isacharunderscore}{\kern0pt}of\ x{\isacharcomma}{\kern0pt}\ {\isasymdelta}{\isacharunderscore}{\kern0pt}of\ x{\isacharparenright}{\kern0pt}{\isacharparenright}{\kern0pt}{\isachardoublequoteclose}\isanewline
\ \ \ \ \isacommand{apply}\isamarkupfalse%
\ {\isacharparenleft}{\kern0pt}rule\ ext{\isacharparenright}{\kern0pt}\isanewline
\ \ \ \ \isacommand{by}\isamarkupfalse%
\ {\isacharparenleft}{\kern0pt}simp\ add{\isacharcolon}{\kern0pt}case{\isacharunderscore}{\kern0pt}prod{\isacharunderscore}{\kern0pt}beta{\isacharprime}{\kern0pt}\ k{\isacharunderscore}{\kern0pt}of{\isacharunderscore}{\kern0pt}def\ n{\isacharunderscore}{\kern0pt}of{\isacharunderscore}{\kern0pt}def\ {\isasymepsilon}{\isacharunderscore}{\kern0pt}of{\isacharunderscore}{\kern0pt}def\ {\isasymdelta}{\isacharunderscore}{\kern0pt}of{\isacharunderscore}{\kern0pt}def\ m{\isacharunderscore}{\kern0pt}of{\isacharunderscore}{\kern0pt}def{\isacharparenright}{\kern0pt}\isanewline
\ \ \isacommand{also}\isamarkupfalse%
\ \isacommand{have}\isamarkupfalse%
\ {\isachardoublequoteopen}{\isachardot}{\kern0pt}{\isachardot}{\kern0pt}{\isachardot}{\kern0pt}\ {\isasymin}\ O{\isacharbrackleft}{\kern0pt}{\isacharquery}{\kern0pt}F{\isacharbrackright}{\kern0pt}{\isacharparenleft}{\kern0pt}g{\isacharparenright}{\kern0pt}{\isachardoublequoteclose}\isanewline
\ \ \ \ \isacommand{apply}\isamarkupfalse%
\ {\isacharparenleft}{\kern0pt}simp\ add{\isacharcolon}{\kern0pt}\ Let{\isacharunderscore}{\kern0pt}def{\isacharparenright}{\kern0pt}\isanewline
\ \ \ \ \isacommand{apply}\isamarkupfalse%
\ {\isacharparenleft}{\kern0pt}rule\ sum{\isacharunderscore}{\kern0pt}in{\isacharunderscore}{\kern0pt}bigo{\isacharunderscore}{\kern0pt}r{\isacharcomma}{\kern0pt}\ simp\ add{\isacharcolon}{\kern0pt}l{\isadigit{1}}{\isacharparenright}{\kern0pt}\isanewline
\ \ \ \ \isacommand{apply}\isamarkupfalse%
\ {\isacharparenleft}{\kern0pt}rule\ sum{\isacharunderscore}{\kern0pt}in{\isacharunderscore}{\kern0pt}bigo{\isacharunderscore}{\kern0pt}r{\isacharcomma}{\kern0pt}\ simp\ add{\isacharcolon}{\kern0pt}l{\isadigit{2}}\ log{\isacharunderscore}{\kern0pt}def{\isacharparenright}{\kern0pt}\isanewline
\ \ \ \ \isacommand{apply}\isamarkupfalse%
\ {\isacharparenleft}{\kern0pt}rule\ sum{\isacharunderscore}{\kern0pt}in{\isacharunderscore}{\kern0pt}bigo{\isacharunderscore}{\kern0pt}r{\isacharcomma}{\kern0pt}\ simp\ add{\isacharcolon}{\kern0pt}l{\isadigit{3}}\ log{\isacharunderscore}{\kern0pt}def{\isacharparenright}{\kern0pt}\isanewline
\ \ \ \ \isacommand{apply}\isamarkupfalse%
\ {\isacharparenleft}{\kern0pt}rule\ sum{\isacharunderscore}{\kern0pt}in{\isacharunderscore}{\kern0pt}bigo{\isacharunderscore}{\kern0pt}r{\isacharcomma}{\kern0pt}\ simp\ add{\isacharcolon}{\kern0pt}l{\isadigit{4}}\ log{\isacharunderscore}{\kern0pt}def{\isacharparenright}{\kern0pt}\isanewline
\ \ \ \ \isacommand{apply}\isamarkupfalse%
\ {\isacharparenleft}{\kern0pt}rule\ sum{\isacharunderscore}{\kern0pt}in{\isacharunderscore}{\kern0pt}bigo{\isacharunderscore}{\kern0pt}r{\isacharcomma}{\kern0pt}\ simp\ add{\isacharcolon}{\kern0pt}l{\isadigit{4}}\ log{\isacharunderscore}{\kern0pt}def{\isacharparenright}{\kern0pt}\isanewline
\ \ \ \ \isacommand{by}\isamarkupfalse%
\ {\isacharparenleft}{\kern0pt}simp\ add{\isacharcolon}{\kern0pt}l{\isadigit{5}}{\isacharcomma}{\kern0pt}\ simp\ add{\isacharcolon}{\kern0pt}unit{\isacharunderscore}{\kern0pt}{\isadigit{1}}{\isacharparenright}{\kern0pt}\isanewline
\ \ \isacommand{also}\isamarkupfalse%
\ \isacommand{have}\isamarkupfalse%
\ {\isachardoublequoteopen}{\isachardot}{\kern0pt}{\isachardot}{\kern0pt}{\isachardot}{\kern0pt}\ {\isacharequal}{\kern0pt}\ O{\isacharbrackleft}{\kern0pt}{\isacharquery}{\kern0pt}F{\isacharbrackright}{\kern0pt}{\isacharparenleft}{\kern0pt}{\isacharquery}{\kern0pt}rhs{\isacharparenright}{\kern0pt}{\isachardoublequoteclose}\isanewline
\ \ \ \ \isacommand{apply}\isamarkupfalse%
\ {\isacharparenleft}{\kern0pt}rule\ arg{\isacharunderscore}{\kern0pt}cong{\isadigit{2}}{\isacharbrackleft}{\kern0pt}\isakeyword{where}\ f{\isacharequal}{\kern0pt}{\isachardoublequoteopen}bigo{\isachardoublequoteclose}{\isacharbrackright}{\kern0pt}{\isacharcomma}{\kern0pt}\ simp{\isacharparenright}{\kern0pt}\isanewline
\ \ \ \ \isacommand{apply}\isamarkupfalse%
\ {\isacharparenleft}{\kern0pt}rule\ ext{\isacharparenright}{\kern0pt}\isanewline
\ \ \ \ \isacommand{by}\isamarkupfalse%
\ {\isacharparenleft}{\kern0pt}simp\ add{\isacharcolon}{\kern0pt}case{\isacharunderscore}{\kern0pt}prod{\isacharunderscore}{\kern0pt}beta{\isacharprime}{\kern0pt}\ g{\isadigit{1}}{\isacharunderscore}{\kern0pt}def\ g{\isacharunderscore}{\kern0pt}def\ n{\isacharunderscore}{\kern0pt}of{\isacharunderscore}{\kern0pt}def\ {\isasymepsilon}{\isacharunderscore}{\kern0pt}of{\isacharunderscore}{\kern0pt}def\ {\isasymdelta}{\isacharunderscore}{\kern0pt}of{\isacharunderscore}{\kern0pt}def\ m{\isacharunderscore}{\kern0pt}of{\isacharunderscore}{\kern0pt}def\ k{\isacharunderscore}{\kern0pt}of{\isacharunderscore}{\kern0pt}def{\isacharparenright}{\kern0pt}\isanewline
\ \ \isacommand{finally}\isamarkupfalse%
\ \isacommand{show}\isamarkupfalse%
\ {\isacharquery}{\kern0pt}thesis\ \isacommand{by}\isamarkupfalse%
\ simp\isanewline
\isacommand{qed}\isamarkupfalse%
%
\endisatagproof
{\isafoldproof}%
%
\isadelimproof
\isanewline
%
\endisadelimproof
%
\isadelimtheory
\isanewline
%
\endisadelimtheory
%
\isatagtheory
\isacommand{end}\isamarkupfalse%
%
\endisatagtheory
{\isafoldtheory}%
%
\isadelimtheory
%
\endisadelimtheory
%
\end{isabellebody}%
\endinput
%:%file=Frequency_Moment_k.tex%:%
%:%11=1%:%
%:%27=3%:%
%:%28=3%:%
%:%29=4%:%
%:%30=5%:%
%:%39=7%:%
%:%40=8%:%
%:%42=10%:%
%:%43=10%:%
%:%44=11%:%
%:%45=12%:%
%:%46=12%:%
%:%47=13%:%
%:%52=18%:%
%:%53=19%:%
%:%54=20%:%
%:%55=20%:%
%:%56=21%:%
%:%66=31%:%
%:%67=32%:%
%:%68=33%:%
%:%69=33%:%
%:%70=34%:%
%:%73=37%:%
%:%74=38%:%
%:%75=39%:%
%:%76=39%:%
%:%77=40%:%
%:%87=50%:%
%:%88=51%:%
%:%89=52%:%
%:%90=52%:%
%:%91=53%:%
%:%101=63%:%
%:%102=64%:%
%:%103=65%:%
%:%104=65%:%
%:%107=66%:%
%:%111=66%:%
%:%112=66%:%
%:%117=66%:%
%:%120=67%:%
%:%121=68%:%
%:%122=68%:%
%:%123=69%:%
%:%124=70%:%
%:%131=71%:%
%:%132=71%:%
%:%133=72%:%
%:%134=72%:%
%:%135=73%:%
%:%136=74%:%
%:%137=74%:%
%:%138=75%:%
%:%139=75%:%
%:%140=76%:%
%:%141=77%:%
%:%142=77%:%
%:%143=78%:%
%:%144=78%:%
%:%145=79%:%
%:%146=79%:%
%:%147=80%:%
%:%148=80%:%
%:%149=81%:%
%:%150=81%:%
%:%151=82%:%
%:%152=82%:%
%:%153=82%:%
%:%154=83%:%
%:%155=83%:%
%:%156=84%:%
%:%157=84%:%
%:%158=84%:%
%:%159=85%:%
%:%160=85%:%
%:%161=86%:%
%:%162=86%:%
%:%163=87%:%
%:%164=87%:%
%:%165=88%:%
%:%166=88%:%
%:%167=89%:%
%:%168=89%:%
%:%169=90%:%
%:%170=90%:%
%:%171=90%:%
%:%172=91%:%
%:%173=91%:%
%:%174=92%:%
%:%175=92%:%
%:%176=92%:%
%:%177=92%:%
%:%178=93%:%
%:%184=93%:%
%:%187=94%:%
%:%188=95%:%
%:%189=95%:%
%:%190=96%:%
%:%191=97%:%
%:%192=98%:%
%:%193=99%:%
%:%200=100%:%
%:%201=100%:%
%:%202=101%:%
%:%203=101%:%
%:%204=102%:%
%:%205=102%:%
%:%206=103%:%
%:%207=103%:%
%:%208=104%:%
%:%209=105%:%
%:%210=105%:%
%:%211=106%:%
%:%212=106%:%
%:%213=106%:%
%:%214=107%:%
%:%215=108%:%
%:%216=108%:%
%:%217=109%:%
%:%218=109%:%
%:%219=110%:%
%:%220=110%:%
%:%221=111%:%
%:%222=111%:%
%:%223=112%:%
%:%224=112%:%
%:%225=112%:%
%:%226=113%:%
%:%227=114%:%
%:%228=114%:%
%:%229=115%:%
%:%230=115%:%
%:%231=116%:%
%:%232=116%:%
%:%233=117%:%
%:%239=117%:%
%:%242=118%:%
%:%243=119%:%
%:%244=119%:%
%:%245=120%:%
%:%246=121%:%
%:%247=122%:%
%:%254=123%:%
%:%255=123%:%
%:%256=124%:%
%:%257=124%:%
%:%258=125%:%
%:%259=126%:%
%:%260=126%:%
%:%261=127%:%
%:%262=128%:%
%:%263=129%:%
%:%264=129%:%
%:%265=130%:%
%:%266=130%:%
%:%267=131%:%
%:%268=131%:%
%:%269=132%:%
%:%270=132%:%
%:%271=133%:%
%:%272=133%:%
%:%273=134%:%
%:%274=134%:%
%:%275=135%:%
%:%276=135%:%
%:%277=136%:%
%:%278=136%:%
%:%279=136%:%
%:%280=137%:%
%:%281=137%:%
%:%282=138%:%
%:%283=139%:%
%:%284=139%:%
%:%285=139%:%
%:%286=140%:%
%:%287=140%:%
%:%288=140%:%
%:%289=141%:%
%:%290=141%:%
%:%291=142%:%
%:%292=142%:%
%:%293=143%:%
%:%294=143%:%
%:%295=144%:%
%:%296=144%:%
%:%297=145%:%
%:%298=145%:%
%:%299=145%:%
%:%300=145%:%
%:%301=146%:%
%:%302=146%:%
%:%303=147%:%
%:%304=147%:%
%:%305=147%:%
%:%306=148%:%
%:%307=148%:%
%:%308=149%:%
%:%309=149%:%
%:%310=149%:%
%:%311=150%:%
%:%312=150%:%
%:%313=151%:%
%:%314=151%:%
%:%315=151%:%
%:%316=151%:%
%:%317=152%:%
%:%318=152%:%
%:%319=153%:%
%:%320=154%:%
%:%321=154%:%
%:%322=155%:%
%:%323=155%:%
%:%324=156%:%
%:%325=156%:%
%:%326=157%:%
%:%327=157%:%
%:%328=158%:%
%:%329=158%:%
%:%330=158%:%
%:%331=159%:%
%:%332=159%:%
%:%333=159%:%
%:%334=160%:%
%:%335=160%:%
%:%336=160%:%
%:%337=161%:%
%:%338=161%:%
%:%339=162%:%
%:%340=162%:%
%:%341=163%:%
%:%342=163%:%
%:%343=164%:%
%:%344=164%:%
%:%345=165%:%
%:%346=165%:%
%:%347=166%:%
%:%348=166%:%
%:%349=167%:%
%:%350=167%:%
%:%351=168%:%
%:%352=168%:%
%:%353=169%:%
%:%354=169%:%
%:%355=170%:%
%:%356=170%:%
%:%357=171%:%
%:%358=171%:%
%:%359=171%:%
%:%360=172%:%
%:%361=172%:%
%:%362=173%:%
%:%363=173%:%
%:%364=174%:%
%:%365=174%:%
%:%366=174%:%
%:%367=175%:%
%:%368=175%:%
%:%369=176%:%
%:%370=176%:%
%:%371=176%:%
%:%372=177%:%
%:%373=177%:%
%:%374=178%:%
%:%375=178%:%
%:%376=179%:%
%:%377=179%:%
%:%378=180%:%
%:%379=180%:%
%:%380=181%:%
%:%381=181%:%
%:%382=182%:%
%:%383=182%:%
%:%384=183%:%
%:%385=183%:%
%:%386=184%:%
%:%387=184%:%
%:%388=185%:%
%:%389=185%:%
%:%390=186%:%
%:%391=187%:%
%:%392=187%:%
%:%393=188%:%
%:%394=188%:%
%:%395=188%:%
%:%396=189%:%
%:%397=189%:%
%:%398=190%:%
%:%399=190%:%
%:%400=190%:%
%:%401=191%:%
%:%402=191%:%
%:%403=192%:%
%:%409=192%:%
%:%412=193%:%
%:%413=194%:%
%:%414=194%:%
%:%415=195%:%
%:%416=196%:%
%:%417=197%:%
%:%420=198%:%
%:%424=198%:%
%:%425=198%:%
%:%426=199%:%
%:%427=199%:%
%:%428=200%:%
%:%429=200%:%
%:%430=201%:%
%:%431=201%:%
%:%432=202%:%
%:%433=202%:%
%:%434=203%:%
%:%435=203%:%
%:%436=203%:%
%:%437=204%:%
%:%438=204%:%
%:%439=205%:%
%:%440=205%:%
%:%441=206%:%
%:%442=206%:%
%:%443=207%:%
%:%444=207%:%
%:%445=208%:%
%:%446=208%:%
%:%447=209%:%
%:%448=209%:%
%:%450=211%:%
%:%451=212%:%
%:%452=212%:%
%:%453=213%:%
%:%454=213%:%
%:%455=214%:%
%:%456=214%:%
%:%457=215%:%
%:%458=215%:%
%:%459=215%:%
%:%460=215%:%
%:%461=216%:%
%:%462=216%:%
%:%463=217%:%
%:%464=218%:%
%:%465=218%:%
%:%466=218%:%
%:%467=219%:%
%:%468=219%:%
%:%469=220%:%
%:%470=220%:%
%:%471=221%:%
%:%472=221%:%
%:%473=222%:%
%:%474=222%:%
%:%475=223%:%
%:%476=223%:%
%:%477=224%:%
%:%478=224%:%
%:%479=225%:%
%:%480=225%:%
%:%481=226%:%
%:%482=226%:%
%:%483=227%:%
%:%484=227%:%
%:%485=228%:%
%:%486=228%:%
%:%487=229%:%
%:%488=229%:%
%:%489=230%:%
%:%490=230%:%
%:%491=231%:%
%:%492=231%:%
%:%493=232%:%
%:%494=232%:%
%:%495=233%:%
%:%496=233%:%
%:%497=234%:%
%:%498=234%:%
%:%499=235%:%
%:%500=235%:%
%:%501=236%:%
%:%507=236%:%
%:%510=237%:%
%:%511=238%:%
%:%512=238%:%
%:%514=240%:%
%:%515=241%:%
%:%517=243%:%
%:%518=243%:%
%:%519=244%:%
%:%520=245%:%
%:%521=246%:%
%:%528=247%:%
%:%529=247%:%
%:%530=248%:%
%:%531=248%:%
%:%532=249%:%
%:%533=249%:%
%:%534=250%:%
%:%535=250%:%
%:%536=251%:%
%:%537=251%:%
%:%538=252%:%
%:%539=252%:%
%:%540=253%:%
%:%541=253%:%
%:%542=254%:%
%:%543=254%:%
%:%544=255%:%
%:%545=255%:%
%:%546=255%:%
%:%547=256%:%
%:%548=256%:%
%:%549=257%:%
%:%550=257%:%
%:%551=258%:%
%:%552=258%:%
%:%553=259%:%
%:%554=259%:%
%:%555=260%:%
%:%556=260%:%
%:%557=261%:%
%:%558=262%:%
%:%559=262%:%
%:%560=263%:%
%:%561=263%:%
%:%562=264%:%
%:%563=264%:%
%:%564=265%:%
%:%565=265%:%
%:%566=266%:%
%:%567=266%:%
%:%568=267%:%
%:%569=267%:%
%:%570=268%:%
%:%571=268%:%
%:%572=269%:%
%:%573=269%:%
%:%574=270%:%
%:%575=270%:%
%:%576=271%:%
%:%577=272%:%
%:%578=272%:%
%:%579=272%:%
%:%580=273%:%
%:%581=273%:%
%:%582=274%:%
%:%583=274%:%
%:%584=275%:%
%:%585=275%:%
%:%586=276%:%
%:%587=276%:%
%:%588=277%:%
%:%589=277%:%
%:%590=278%:%
%:%591=278%:%
%:%592=279%:%
%:%593=279%:%
%:%594=280%:%
%:%595=280%:%
%:%596=281%:%
%:%597=281%:%
%:%598=282%:%
%:%599=282%:%
%:%600=283%:%
%:%601=283%:%
%:%602=284%:%
%:%603=284%:%
%:%604=285%:%
%:%605=285%:%
%:%606=286%:%
%:%607=286%:%
%:%608=287%:%
%:%609=287%:%
%:%610=288%:%
%:%611=288%:%
%:%612=289%:%
%:%613=289%:%
%:%614=290%:%
%:%615=290%:%
%:%616=291%:%
%:%622=291%:%
%:%625=292%:%
%:%626=293%:%
%:%627=293%:%
%:%628=294%:%
%:%629=295%:%
%:%630=296%:%
%:%631=297%:%
%:%632=298%:%
%:%633=299%:%
%:%634=300%:%
%:%635=301%:%
%:%636=302%:%
%:%638=304%:%
%:%641=305%:%
%:%645=305%:%
%:%646=305%:%
%:%647=306%:%
%:%648=306%:%
%:%649=307%:%
%:%650=307%:%
%:%651=308%:%
%:%652=308%:%
%:%653=308%:%
%:%654=309%:%
%:%655=309%:%
%:%656=310%:%
%:%657=310%:%
%:%658=311%:%
%:%659=311%:%
%:%660=312%:%
%:%661=312%:%
%:%662=313%:%
%:%663=313%:%
%:%664=314%:%
%:%665=314%:%
%:%666=315%:%
%:%667=316%:%
%:%668=317%:%
%:%669=317%:%
%:%670=318%:%
%:%671=319%:%
%:%672=319%:%
%:%673=320%:%
%:%674=321%:%
%:%675=321%:%
%:%676=322%:%
%:%677=322%:%
%:%678=322%:%
%:%679=322%:%
%:%680=323%:%
%:%681=324%:%
%:%682=324%:%
%:%683=324%:%
%:%684=325%:%
%:%685=325%:%
%:%686=326%:%
%:%687=326%:%
%:%688=327%:%
%:%689=327%:%
%:%690=328%:%
%:%691=328%:%
%:%692=329%:%
%:%693=329%:%
%:%694=330%:%
%:%700=330%:%
%:%703=331%:%
%:%704=332%:%
%:%705=332%:%
%:%706=333%:%
%:%707=334%:%
%:%714=335%:%
%:%715=335%:%
%:%716=336%:%
%:%717=336%:%
%:%718=337%:%
%:%719=337%:%
%:%720=338%:%
%:%721=338%:%
%:%722=338%:%
%:%723=339%:%
%:%724=339%:%
%:%725=340%:%
%:%726=340%:%
%:%727=341%:%
%:%728=341%:%
%:%729=342%:%
%:%730=342%:%
%:%731=343%:%
%:%732=343%:%
%:%733=344%:%
%:%734=344%:%
%:%735=344%:%
%:%736=345%:%
%:%737=345%:%
%:%738=346%:%
%:%739=346%:%
%:%740=347%:%
%:%741=347%:%
%:%742=347%:%
%:%743=347%:%
%:%744=348%:%
%:%745=348%:%
%:%746=348%:%
%:%747=349%:%
%:%748=349%:%
%:%749=350%:%
%:%750=350%:%
%:%751=351%:%
%:%752=351%:%
%:%753=351%:%
%:%754=352%:%
%:%755=352%:%
%:%756=353%:%
%:%757=353%:%
%:%758=353%:%
%:%759=353%:%
%:%760=354%:%
%:%766=354%:%
%:%769=355%:%
%:%770=356%:%
%:%771=356%:%
%:%772=357%:%
%:%773=358%:%
%:%774=359%:%
%:%775=360%:%
%:%782=361%:%
%:%783=361%:%
%:%784=362%:%
%:%785=362%:%
%:%786=363%:%
%:%787=363%:%
%:%788=364%:%
%:%789=364%:%
%:%790=364%:%
%:%791=365%:%
%:%792=365%:%
%:%793=365%:%
%:%794=366%:%
%:%795=366%:%
%:%796=367%:%
%:%797=367%:%
%:%798=368%:%
%:%799=368%:%
%:%800=368%:%
%:%801=369%:%
%:%802=369%:%
%:%803=369%:%
%:%804=370%:%
%:%805=370%:%
%:%806=371%:%
%:%807=371%:%
%:%808=372%:%
%:%809=372%:%
%:%810=373%:%
%:%811=373%:%
%:%812=373%:%
%:%813=374%:%
%:%814=374%:%
%:%815=375%:%
%:%816=375%:%
%:%817=376%:%
%:%818=376%:%
%:%819=377%:%
%:%820=377%:%
%:%821=378%:%
%:%822=378%:%
%:%823=378%:%
%:%824=379%:%
%:%825=379%:%
%:%826=379%:%
%:%827=379%:%
%:%828=380%:%
%:%838=382%:%
%:%840=383%:%
%:%841=383%:%
%:%842=384%:%
%:%843=385%:%
%:%844=386%:%
%:%847=387%:%
%:%851=387%:%
%:%852=387%:%
%:%853=387%:%
%:%854=388%:%
%:%855=388%:%
%:%856=389%:%
%:%857=389%:%
%:%862=389%:%
%:%865=390%:%
%:%866=391%:%
%:%867=391%:%
%:%868=392%:%
%:%869=393%:%
%:%870=394%:%
%:%871=395%:%
%:%872=396%:%
%:%879=397%:%
%:%880=397%:%
%:%881=398%:%
%:%882=398%:%
%:%883=399%:%
%:%884=399%:%
%:%885=400%:%
%:%886=400%:%
%:%887=400%:%
%:%888=401%:%
%:%889=401%:%
%:%890=402%:%
%:%891=403%:%
%:%892=403%:%
%:%893=403%:%
%:%894=404%:%
%:%895=404%:%
%:%896=405%:%
%:%897=405%:%
%:%898=405%:%
%:%899=406%:%
%:%900=406%:%
%:%901=406%:%
%:%902=407%:%
%:%903=408%:%
%:%904=408%:%
%:%905=409%:%
%:%906=409%:%
%:%907=410%:%
%:%908=410%:%
%:%909=411%:%
%:%910=411%:%
%:%911=411%:%
%:%912=412%:%
%:%913=412%:%
%:%914=413%:%
%:%915=414%:%
%:%916=414%:%
%:%917=415%:%
%:%918=415%:%
%:%919=416%:%
%:%920=416%:%
%:%921=417%:%
%:%922=417%:%
%:%923=417%:%
%:%924=418%:%
%:%925=419%:%
%:%926=419%:%
%:%927=420%:%
%:%928=420%:%
%:%929=420%:%
%:%930=421%:%
%:%931=421%:%
%:%932=421%:%
%:%933=422%:%
%:%934=422%:%
%:%935=423%:%
%:%936=423%:%
%:%937=424%:%
%:%938=424%:%
%:%939=424%:%
%:%940=424%:%
%:%941=425%:%
%:%942=425%:%
%:%943=425%:%
%:%944=425%:%
%:%945=426%:%
%:%946=426%:%
%:%947=427%:%
%:%948=427%:%
%:%949=427%:%
%:%950=428%:%
%:%951=428%:%
%:%952=429%:%
%:%953=430%:%
%:%954=430%:%
%:%955=431%:%
%:%956=431%:%
%:%957=431%:%
%:%958=432%:%
%:%959=432%:%
%:%960=433%:%
%:%961=433%:%
%:%962=434%:%
%:%963=434%:%
%:%964=435%:%
%:%965=435%:%
%:%966=436%:%
%:%967=436%:%
%:%968=437%:%
%:%969=437%:%
%:%970=438%:%
%:%971=438%:%
%:%972=438%:%
%:%973=439%:%
%:%974=439%:%
%:%975=440%:%
%:%976=440%:%
%:%977=441%:%
%:%978=441%:%
%:%979=442%:%
%:%980=442%:%
%:%981=443%:%
%:%982=443%:%
%:%983=443%:%
%:%984=444%:%
%:%985=444%:%
%:%986=445%:%
%:%987=445%:%
%:%988=446%:%
%:%989=446%:%
%:%990=447%:%
%:%991=447%:%
%:%992=447%:%
%:%993=448%:%
%:%994=448%:%
%:%995=449%:%
%:%996=450%:%
%:%997=450%:%
%:%998=451%:%
%:%999=451%:%
%:%1000=452%:%
%:%1001=453%:%
%:%1002=453%:%
%:%1003=454%:%
%:%1004=454%:%
%:%1005=455%:%
%:%1006=455%:%
%:%1007=456%:%
%:%1008=456%:%
%:%1009=456%:%
%:%1010=457%:%
%:%1011=457%:%
%:%1012=458%:%
%:%1013=458%:%
%:%1014=458%:%
%:%1015=459%:%
%:%1016=459%:%
%:%1017=459%:%
%:%1018=460%:%
%:%1019=460%:%
%:%1020=461%:%
%:%1021=461%:%
%:%1022=461%:%
%:%1023=462%:%
%:%1024=462%:%
%:%1025=462%:%
%:%1026=463%:%
%:%1027=463%:%
%:%1028=464%:%
%:%1029=464%:%
%:%1030=464%:%
%:%1031=464%:%
%:%1032=465%:%
%:%1033=465%:%
%:%1034=465%:%
%:%1035=465%:%
%:%1036=466%:%
%:%1037=466%:%
%:%1038=467%:%
%:%1039=467%:%
%:%1040=467%:%
%:%1041=468%:%
%:%1042=468%:%
%:%1043=469%:%
%:%1044=469%:%
%:%1045=470%:%
%:%1046=470%:%
%:%1047=471%:%
%:%1048=471%:%
%:%1049=472%:%
%:%1050=472%:%
%:%1051=473%:%
%:%1052=473%:%
%:%1053=474%:%
%:%1054=474%:%
%:%1055=475%:%
%:%1056=475%:%
%:%1057=476%:%
%:%1058=476%:%
%:%1059=476%:%
%:%1060=477%:%
%:%1061=477%:%
%:%1062=478%:%
%:%1063=479%:%
%:%1064=479%:%
%:%1065=480%:%
%:%1066=480%:%
%:%1067=481%:%
%:%1068=481%:%
%:%1069=481%:%
%:%1070=481%:%
%:%1071=482%:%
%:%1072=482%:%
%:%1073=482%:%
%:%1074=483%:%
%:%1075=484%:%
%:%1076=484%:%
%:%1077=485%:%
%:%1078=485%:%
%:%1079=486%:%
%:%1080=486%:%
%:%1081=487%:%
%:%1082=487%:%
%:%1083=488%:%
%:%1084=488%:%
%:%1085=488%:%
%:%1086=489%:%
%:%1087=489%:%
%:%1088=490%:%
%:%1089=490%:%
%:%1090=491%:%
%:%1091=491%:%
%:%1092=491%:%
%:%1093=492%:%
%:%1094=492%:%
%:%1095=492%:%
%:%1096=493%:%
%:%1097=493%:%
%:%1098=494%:%
%:%1099=494%:%
%:%1100=495%:%
%:%1101=495%:%
%:%1102=496%:%
%:%1103=496%:%
%:%1104=497%:%
%:%1105=497%:%
%:%1106=498%:%
%:%1107=498%:%
%:%1108=498%:%
%:%1109=499%:%
%:%1110=499%:%
%:%1111=499%:%
%:%1112=499%:%
%:%1113=499%:%
%:%1114=500%:%
%:%1115=500%:%
%:%1116=500%:%
%:%1117=501%:%
%:%1118=501%:%
%:%1119=502%:%
%:%1120=502%:%
%:%1121=503%:%
%:%1122=503%:%
%:%1123=504%:%
%:%1124=504%:%
%:%1125=504%:%
%:%1126=505%:%
%:%1132=505%:%
%:%1135=506%:%
%:%1136=507%:%
%:%1137=507%:%
%:%1138=508%:%
%:%1139=509%:%
%:%1140=510%:%
%:%1141=511%:%
%:%1148=512%:%
%:%1149=512%:%
%:%1150=513%:%
%:%1151=513%:%
%:%1152=514%:%
%:%1153=514%:%
%:%1154=515%:%
%:%1155=515%:%
%:%1156=516%:%
%:%1157=516%:%
%:%1158=517%:%
%:%1159=517%:%
%:%1160=518%:%
%:%1161=518%:%
%:%1162=519%:%
%:%1163=519%:%
%:%1164=520%:%
%:%1165=520%:%
%:%1166=521%:%
%:%1167=521%:%
%:%1168=521%:%
%:%1169=522%:%
%:%1170=522%:%
%:%1171=523%:%
%:%1172=523%:%
%:%1173=524%:%
%:%1174=524%:%
%:%1175=525%:%
%:%1176=525%:%
%:%1177=526%:%
%:%1178=526%:%
%:%1179=527%:%
%:%1180=527%:%
%:%1181=527%:%
%:%1182=528%:%
%:%1188=528%:%
%:%1191=529%:%
%:%1192=530%:%
%:%1193=530%:%
%:%1194=531%:%
%:%1195=532%:%
%:%1196=533%:%
%:%1197=534%:%
%:%1199=536%:%
%:%1206=537%:%
%:%1207=537%:%
%:%1208=538%:%
%:%1209=538%:%
%:%1210=539%:%
%:%1211=540%:%
%:%1212=540%:%
%:%1213=541%:%
%:%1214=542%:%
%:%1215=542%:%
%:%1216=543%:%
%:%1217=543%:%
%:%1218=544%:%
%:%1219=544%:%
%:%1220=545%:%
%:%1221=545%:%
%:%1222=546%:%
%:%1223=546%:%
%:%1224=546%:%
%:%1225=547%:%
%:%1226=548%:%
%:%1227=548%:%
%:%1228=548%:%
%:%1229=548%:%
%:%1230=549%:%
%:%1231=550%:%
%:%1232=550%:%
%:%1233=551%:%
%:%1234=551%:%
%:%1235=552%:%
%:%1236=552%:%
%:%1237=553%:%
%:%1238=553%:%
%:%1239=554%:%
%:%1240=554%:%
%:%1241=555%:%
%:%1242=555%:%
%:%1243=556%:%
%:%1244=556%:%
%:%1245=557%:%
%:%1246=557%:%
%:%1247=557%:%
%:%1248=557%:%
%:%1249=558%:%
%:%1250=558%:%
%:%1251=558%:%
%:%1252=559%:%
%:%1253=559%:%
%:%1254=560%:%
%:%1255=560%:%
%:%1256=560%:%
%:%1257=561%:%
%:%1258=561%:%
%:%1259=562%:%
%:%1260=562%:%
%:%1261=562%:%
%:%1262=563%:%
%:%1263=563%:%
%:%1264=564%:%
%:%1265=564%:%
%:%1266=564%:%
%:%1267=565%:%
%:%1268=565%:%
%:%1269=566%:%
%:%1270=566%:%
%:%1271=567%:%
%:%1272=568%:%
%:%1273=568%:%
%:%1274=569%:%
%:%1275=570%:%
%:%1276=570%:%
%:%1277=571%:%
%:%1278=571%:%
%:%1279=572%:%
%:%1280=572%:%
%:%1281=573%:%
%:%1282=573%:%
%:%1283=574%:%
%:%1284=574%:%
%:%1285=575%:%
%:%1286=575%:%
%:%1287=576%:%
%:%1288=576%:%
%:%1289=576%:%
%:%1290=577%:%
%:%1291=577%:%
%:%1292=578%:%
%:%1293=578%:%
%:%1294=579%:%
%:%1295=579%:%
%:%1296=580%:%
%:%1297=580%:%
%:%1298=581%:%
%:%1299=581%:%
%:%1300=582%:%
%:%1301=582%:%
%:%1302=583%:%
%:%1303=583%:%
%:%1304=583%:%
%:%1305=584%:%
%:%1306=584%:%
%:%1307=584%:%
%:%1308=585%:%
%:%1309=585%:%
%:%1310=586%:%
%:%1311=586%:%
%:%1312=587%:%
%:%1313=587%:%
%:%1314=588%:%
%:%1315=588%:%
%:%1316=588%:%
%:%1317=589%:%
%:%1318=589%:%
%:%1319=590%:%
%:%1320=590%:%
%:%1321=591%:%
%:%1322=591%:%
%:%1323=592%:%
%:%1324=592%:%
%:%1325=593%:%
%:%1326=593%:%
%:%1327=594%:%
%:%1328=594%:%
%:%1329=594%:%
%:%1330=595%:%
%:%1331=596%:%
%:%1332=596%:%
%:%1333=597%:%
%:%1334=598%:%
%:%1335=598%:%
%:%1336=599%:%
%:%1337=600%:%
%:%1338=600%:%
%:%1339=601%:%
%:%1340=601%:%
%:%1341=601%:%
%:%1342=602%:%
%:%1343=603%:%
%:%1344=603%:%
%:%1345=604%:%
%:%1346=604%:%
%:%1347=605%:%
%:%1348=605%:%
%:%1349=606%:%
%:%1350=606%:%
%:%1351=607%:%
%:%1352=607%:%
%:%1353=608%:%
%:%1354=608%:%
%:%1355=609%:%
%:%1356=609%:%
%:%1357=609%:%
%:%1358=610%:%
%:%1359=610%:%
%:%1360=610%:%
%:%1361=611%:%
%:%1362=612%:%
%:%1363=612%:%
%:%1364=613%:%
%:%1365=614%:%
%:%1366=614%:%
%:%1367=615%:%
%:%1368=615%:%
%:%1369=616%:%
%:%1370=616%:%
%:%1371=617%:%
%:%1372=617%:%
%:%1373=618%:%
%:%1374=618%:%
%:%1375=619%:%
%:%1376=619%:%
%:%1377=619%:%
%:%1378=620%:%
%:%1384=620%:%
%:%1387=621%:%
%:%1388=622%:%
%:%1389=622%:%
%:%1390=623%:%
%:%1391=624%:%
%:%1392=625%:%
%:%1393=626%:%
%:%1394=627%:%
%:%1395=628%:%
%:%1396=629%:%
%:%1397=630%:%
%:%1398=631%:%
%:%1399=632%:%
%:%1402=633%:%
%:%1406=633%:%
%:%1407=633%:%
%:%1408=634%:%
%:%1409=634%:%
%:%1410=635%:%
%:%1411=635%:%
%:%1412=636%:%
%:%1413=636%:%
%:%1414=637%:%
%:%1415=637%:%
%:%1416=638%:%
%:%1417=638%:%
%:%1418=639%:%
%:%1419=639%:%
%:%1420=640%:%
%:%1421=640%:%
%:%1426=640%:%
%:%1429=641%:%
%:%1430=642%:%
%:%1431=642%:%
%:%1432=643%:%
%:%1433=644%:%
%:%1434=645%:%
%:%1435=646%:%
%:%1436=647%:%
%:%1437=648%:%
%:%1444=649%:%
%:%1445=649%:%
%:%1446=650%:%
%:%1447=650%:%
%:%1448=651%:%
%:%1449=651%:%
%:%1450=651%:%
%:%1451=651%:%
%:%1452=652%:%
%:%1453=652%:%
%:%1454=652%:%
%:%1455=652%:%
%:%1456=653%:%
%:%1457=653%:%
%:%1458=653%:%
%:%1459=654%:%
%:%1460=654%:%
%:%1461=655%:%
%:%1462=655%:%
%:%1463=656%:%
%:%1464=656%:%
%:%1465=657%:%
%:%1466=657%:%
%:%1467=658%:%
%:%1468=659%:%
%:%1469=659%:%
%:%1470=660%:%
%:%1472=662%:%
%:%1473=663%:%
%:%1474=664%:%
%:%1475=664%:%
%:%1476=665%:%
%:%1477=666%:%
%:%1478=666%:%
%:%1479=667%:%
%:%1480=668%:%
%:%1481=668%:%
%:%1482=669%:%
%:%1483=670%:%
%:%1484=671%:%
%:%1485=671%:%
%:%1486=671%:%
%:%1487=671%:%
%:%1488=672%:%
%:%1489=672%:%
%:%1490=672%:%
%:%1491=672%:%
%:%1492=673%:%
%:%1493=674%:%
%:%1494=674%:%
%:%1495=674%:%
%:%1496=674%:%
%:%1497=675%:%
%:%1498=676%:%
%:%1499=676%:%
%:%1500=677%:%
%:%1501=677%:%
%:%1502=678%:%
%:%1503=678%:%
%:%1504=679%:%
%:%1505=679%:%
%:%1506=680%:%
%:%1507=680%:%
%:%1508=680%:%
%:%1509=681%:%
%:%1510=681%:%
%:%1511=682%:%
%:%1512=682%:%
%:%1513=683%:%
%:%1514=683%:%
%:%1515=683%:%
%:%1516=683%:%
%:%1517=684%:%
%:%1518=684%:%
%:%1519=685%:%
%:%1520=685%:%
%:%1521=685%:%
%:%1522=686%:%
%:%1523=686%:%
%:%1524=687%:%
%:%1525=687%:%
%:%1526=688%:%
%:%1527=689%:%
%:%1528=689%:%
%:%1529=690%:%
%:%1530=690%:%
%:%1531=691%:%
%:%1532=691%:%
%:%1533=692%:%
%:%1534=692%:%
%:%1535=692%:%
%:%1536=693%:%
%:%1537=693%:%
%:%1538=694%:%
%:%1539=694%:%
%:%1540=695%:%
%:%1541=695%:%
%:%1542=696%:%
%:%1543=696%:%
%:%1544=697%:%
%:%1545=698%:%
%:%1546=698%:%
%:%1547=699%:%
%:%1548=700%:%
%:%1549=700%:%
%:%1550=701%:%
%:%1551=701%:%
%:%1552=702%:%
%:%1553=703%:%
%:%1554=703%:%
%:%1555=704%:%
%:%1556=704%:%
%:%1557=705%:%
%:%1558=706%:%
%:%1559=706%:%
%:%1560=707%:%
%:%1561=707%:%
%:%1562=708%:%
%:%1563=708%:%
%:%1564=709%:%
%:%1565=709%:%
%:%1566=710%:%
%:%1567=710%:%
%:%1568=711%:%
%:%1569=711%:%
%:%1570=712%:%
%:%1571=712%:%
%:%1572=713%:%
%:%1573=713%:%
%:%1574=714%:%
%:%1575=715%:%
%:%1576=715%:%
%:%1577=716%:%
%:%1578=717%:%
%:%1579=717%:%
%:%1580=718%:%
%:%1581=718%:%
%:%1582=719%:%
%:%1583=720%:%
%:%1584=720%:%
%:%1586=722%:%
%:%1587=723%:%
%:%1588=723%:%
%:%1589=724%:%
%:%1590=724%:%
%:%1591=725%:%
%:%1592=725%:%
%:%1593=726%:%
%:%1594=727%:%
%:%1595=727%:%
%:%1596=728%:%
%:%1597=728%:%
%:%1598=728%:%
%:%1599=729%:%
%:%1600=729%:%
%:%1601=729%:%
%:%1602=730%:%
%:%1603=730%:%
%:%1604=731%:%
%:%1605=731%:%
%:%1606=732%:%
%:%1607=732%:%
%:%1608=733%:%
%:%1609=733%:%
%:%1610=733%:%
%:%1611=734%:%
%:%1612=734%:%
%:%1613=735%:%
%:%1614=735%:%
%:%1615=735%:%
%:%1616=736%:%
%:%1617=736%:%
%:%1618=737%:%
%:%1619=737%:%
%:%1620=738%:%
%:%1621=739%:%
%:%1622=739%:%
%:%1623=740%:%
%:%1624=740%:%
%:%1625=740%:%
%:%1626=741%:%
%:%1627=741%:%
%:%1628=742%:%
%:%1629=742%:%
%:%1630=743%:%
%:%1631=743%:%
%:%1632=743%:%
%:%1633=744%:%
%:%1634=744%:%
%:%1635=745%:%
%:%1636=745%:%
%:%1637=745%:%
%:%1638=746%:%
%:%1639=746%:%
%:%1640=747%:%
%:%1641=748%:%
%:%1642=748%:%
%:%1644=750%:%
%:%1645=751%:%
%:%1646=751%:%
%:%1647=752%:%
%:%1648=752%:%
%:%1649=753%:%
%:%1650=753%:%
%:%1651=754%:%
%:%1652=755%:%
%:%1653=755%:%
%:%1654=756%:%
%:%1655=756%:%
%:%1656=756%:%
%:%1657=757%:%
%:%1658=757%:%
%:%1659=757%:%
%:%1660=758%:%
%:%1661=759%:%
%:%1662=759%:%
%:%1663=760%:%
%:%1664=760%:%
%:%1665=761%:%
%:%1666=762%:%
%:%1667=762%:%
%:%1669=764%:%
%:%1670=765%:%
%:%1671=765%:%
%:%1672=766%:%
%:%1673=766%:%
%:%1674=767%:%
%:%1675=767%:%
%:%1676=768%:%
%:%1677=768%:%
%:%1678=769%:%
%:%1679=769%:%
%:%1680=770%:%
%:%1681=771%:%
%:%1682=771%:%
%:%1684=773%:%
%:%1685=774%:%
%:%1686=774%:%
%:%1687=775%:%
%:%1688=775%:%
%:%1689=776%:%
%:%1690=776%:%
%:%1691=777%:%
%:%1692=777%:%
%:%1693=778%:%
%:%1694=779%:%
%:%1695=779%:%
%:%1696=780%:%
%:%1697=780%:%
%:%1698=781%:%
%:%1699=781%:%
%:%1700=782%:%
%:%1701=782%:%
%:%1702=783%:%
%:%1703=783%:%
%:%1704=784%:%
%:%1705=784%:%
%:%1706=785%:%
%:%1707=785%:%
%:%1708=786%:%
%:%1709=786%:%
%:%1710=787%:%
%:%1711=787%:%
%:%1712=788%:%
%:%1713=788%:%
%:%1714=789%:%
%:%1715=789%:%
%:%1716=790%:%
%:%1717=790%:%
%:%1718=790%:%
%:%1719=791%:%
%:%1720=791%:%
%:%1721=792%:%
%:%1722=792%:%
%:%1723=793%:%
%:%1724=793%:%
%:%1725=794%:%
%:%1726=794%:%
%:%1727=794%:%
%:%1728=795%:%
%:%1729=795%:%
%:%1730=796%:%
%:%1731=796%:%
%:%1732=797%:%
%:%1733=797%:%
%:%1734=798%:%
%:%1735=798%:%
%:%1736=799%:%
%:%1737=799%:%
%:%1738=799%:%
%:%1739=800%:%
%:%1740=800%:%
%:%1741=801%:%
%:%1742=801%:%
%:%1743=802%:%
%:%1744=802%:%
%:%1745=803%:%
%:%1746=803%:%
%:%1747=803%:%
%:%1748=803%:%
%:%1749=804%:%
%:%1750=804%:%
%:%1751=805%:%
%:%1752=806%:%
%:%1753=806%:%
%:%1754=807%:%
%:%1755=808%:%
%:%1756=808%:%
%:%1757=809%:%
%:%1758=809%:%
%:%1759=810%:%
%:%1760=810%:%
%:%1761=811%:%
%:%1762=811%:%
%:%1763=812%:%
%:%1764=812%:%
%:%1765=812%:%
%:%1766=813%:%
%:%1767=813%:%
%:%1768=813%:%
%:%1769=814%:%
%:%1770=814%:%
%:%1771=815%:%
%:%1772=815%:%
%:%1773=816%:%
%:%1774=816%:%
%:%1775=817%:%
%:%1776=818%:%
%:%1777=818%:%
%:%1778=819%:%
%:%1779=819%:%
%:%1780=819%:%
%:%1781=820%:%
%:%1782=820%:%
%:%1783=821%:%
%:%1784=821%:%
%:%1785=822%:%
%:%1786=822%:%
%:%1787=823%:%
%:%1788=823%:%
%:%1789=823%:%
%:%1790=823%:%
%:%1791=824%:%
%:%1792=824%:%
%:%1793=825%:%
%:%1794=825%:%
%:%1795=826%:%
%:%1796=826%:%
%:%1797=826%:%
%:%1798=827%:%
%:%1799=827%:%
%:%1800=828%:%
%:%1801=828%:%
%:%1802=829%:%
%:%1803=830%:%
%:%1804=830%:%
%:%1805=831%:%
%:%1806=831%:%
%:%1807=832%:%
%:%1808=832%:%
%:%1809=833%:%
%:%1810=833%:%
%:%1811=834%:%
%:%1812=834%:%
%:%1813=835%:%
%:%1814=835%:%
%:%1815=836%:%
%:%1816=836%:%
%:%1817=836%:%
%:%1818=837%:%
%:%1819=837%:%
%:%1820=837%:%
%:%1821=838%:%
%:%1822=838%:%
%:%1823=839%:%
%:%1824=839%:%
%:%1825=840%:%
%:%1826=840%:%
%:%1827=841%:%
%:%1828=841%:%
%:%1829=842%:%
%:%1830=842%:%
%:%1831=843%:%
%:%1832=843%:%
%:%1833=844%:%
%:%1834=844%:%
%:%1835=845%:%
%:%1836=845%:%
%:%1837=846%:%
%:%1838=846%:%
%:%1839=847%:%
%:%1840=847%:%
%:%1841=847%:%
%:%1842=848%:%
%:%1843=848%:%
%:%1844=848%:%
%:%1845=849%:%
%:%1851=849%:%
%:%1854=850%:%
%:%1855=851%:%
%:%1856=851%:%
%:%1857=852%:%
%:%1865=860%:%
%:%1866=861%:%
%:%1867=862%:%
%:%1868=862%:%
%:%1869=863%:%
%:%1874=868%:%
%:%1875=869%:%
%:%1876=870%:%
%:%1877=870%:%
%:%1880=871%:%
%:%1884=871%:%
%:%1885=871%:%
%:%1886=872%:%
%:%1887=872%:%
%:%1888=873%:%
%:%1889=873%:%
%:%1890=874%:%
%:%1891=874%:%
%:%1892=875%:%
%:%1893=875%:%
%:%1894=876%:%
%:%1895=876%:%
%:%1896=877%:%
%:%1897=877%:%
%:%1902=877%:%
%:%1905=878%:%
%:%1906=879%:%
%:%1907=879%:%
%:%1908=880%:%
%:%1909=881%:%
%:%1910=882%:%
%:%1911=883%:%
%:%1912=884%:%
%:%1913=885%:%
%:%1920=886%:%
%:%1921=886%:%
%:%1922=887%:%
%:%1923=887%:%
%:%1924=888%:%
%:%1925=888%:%
%:%1926=889%:%
%:%1927=889%:%
%:%1928=889%:%
%:%1929=890%:%
%:%1930=890%:%
%:%1931=891%:%
%:%1932=891%:%
%:%1933=892%:%
%:%1934=892%:%
%:%1935=893%:%
%:%1936=894%:%
%:%1937=894%:%
%:%1938=895%:%
%:%1939=895%:%
%:%1940=896%:%
%:%1941=896%:%
%:%1942=897%:%
%:%1943=897%:%
%:%1944=898%:%
%:%1945=898%:%
%:%1946=898%:%
%:%1947=898%:%
%:%1948=899%:%
%:%1949=899%:%
%:%1950=899%:%
%:%1951=899%:%
%:%1952=900%:%
%:%1953=901%:%
%:%1954=901%:%
%:%1955=902%:%
%:%1956=902%:%
%:%1957=903%:%
%:%1958=903%:%
%:%1959=904%:%
%:%1960=904%:%
%:%1961=905%:%
%:%1962=905%:%
%:%1963=906%:%
%:%1964=906%:%
%:%1965=906%:%
%:%1966=907%:%
%:%1967=907%:%
%:%1968=908%:%
%:%1969=908%:%
%:%1970=908%:%
%:%1971=908%:%
%:%1972=909%:%
%:%1973=909%:%
%:%1974=910%:%
%:%1975=911%:%
%:%1976=911%:%
%:%1980=915%:%
%:%1981=916%:%
%:%1982=916%:%
%:%1983=917%:%
%:%1984=917%:%
%:%1985=918%:%
%:%1986=918%:%
%:%1987=919%:%
%:%1988=919%:%
%:%1989=920%:%
%:%1990=920%:%
%:%1991=921%:%
%:%1992=921%:%
%:%1993=922%:%
%:%1994=922%:%
%:%1995=923%:%
%:%1996=923%:%
%:%1997=924%:%
%:%1998=924%:%
%:%1999=925%:%
%:%2000=925%:%
%:%2001=926%:%
%:%2002=926%:%
%:%2003=926%:%
%:%2004=927%:%
%:%2005=927%:%
%:%2006=928%:%
%:%2007=928%:%
%:%2008=929%:%
%:%2009=929%:%
%:%2010=930%:%
%:%2011=930%:%
%:%2012=930%:%
%:%2014=932%:%
%:%2015=933%:%
%:%2016=933%:%
%:%2017=934%:%
%:%2018=934%:%
%:%2019=935%:%
%:%2020=935%:%
%:%2021=936%:%
%:%2022=937%:%
%:%2023=937%:%
%:%2024=938%:%
%:%2025=938%:%
%:%2026=939%:%
%:%2027=939%:%
%:%2028=940%:%
%:%2029=940%:%
%:%2030=941%:%
%:%2031=942%:%
%:%2032=942%:%
%:%2033=943%:%
%:%2034=944%:%
%:%2035=944%:%
%:%2036=945%:%
%:%2037=945%:%
%:%2038=946%:%
%:%2039=946%:%
%:%2040=947%:%
%:%2041=948%:%
%:%2042=948%:%
%:%2043=948%:%
%:%2044=948%:%
%:%2045=949%:%
%:%2046=949%:%
%:%2047=949%:%
%:%2048=949%:%
%:%2049=950%:%
%:%2050=950%:%
%:%2051=950%:%
%:%2052=950%:%
%:%2053=951%:%
%:%2054=952%:%
%:%2055=952%:%
%:%2056=953%:%
%:%2057=954%:%
%:%2058=954%:%
%:%2059=955%:%
%:%2060=955%:%
%:%2061=956%:%
%:%2062=956%:%
%:%2063=957%:%
%:%2064=957%:%
%:%2065=958%:%
%:%2066=958%:%
%:%2067=958%:%
%:%2068=959%:%
%:%2069=959%:%
%:%2070=960%:%
%:%2071=960%:%
%:%2072=961%:%
%:%2073=961%:%
%:%2074=962%:%
%:%2075=962%:%
%:%2076=962%:%
%:%2077=963%:%
%:%2078=963%:%
%:%2079=963%:%
%:%2080=964%:%
%:%2081=964%:%
%:%2082=964%:%
%:%2083=964%:%
%:%2084=965%:%
%:%2085=965%:%
%:%2090=970%:%
%:%2091=971%:%
%:%2092=971%:%
%:%2093=972%:%
%:%2094=973%:%
%:%2095=973%:%
%:%2096=974%:%
%:%2097=974%:%
%:%2098=975%:%
%:%2099=975%:%
%:%2100=976%:%
%:%2101=976%:%
%:%2102=977%:%
%:%2103=977%:%
%:%2104=978%:%
%:%2105=978%:%
%:%2106=979%:%
%:%2107=979%:%
%:%2108=980%:%
%:%2109=980%:%
%:%2110=981%:%
%:%2111=981%:%
%:%2112=982%:%
%:%2113=982%:%
%:%2114=982%:%
%:%2115=983%:%
%:%2116=983%:%
%:%2117=983%:%
%:%2118=984%:%
%:%2119=984%:%
%:%2120=985%:%
%:%2121=985%:%
%:%2122=985%:%
%:%2123=986%:%
%:%2124=986%:%
%:%2125=987%:%
%:%2126=987%:%
%:%2127=988%:%
%:%2128=989%:%
%:%2129=989%:%
%:%2130=990%:%
%:%2131=990%:%
%:%2132=991%:%
%:%2133=991%:%
%:%2134=992%:%
%:%2135=992%:%
%:%2136=993%:%
%:%2137=993%:%
%:%2138=994%:%
%:%2139=994%:%
%:%2140=995%:%
%:%2146=995%:%
%:%2149=996%:%
%:%2150=997%:%
%:%2151=997%:%
%:%2152=998%:%
%:%2154=1000%:%
%:%2155=1001%:%
%:%2162=1002%:%
%:%2163=1002%:%
%:%2164=1003%:%
%:%2165=1003%:%
%:%2166=1004%:%
%:%2167=1004%:%
%:%2168=1005%:%
%:%2169=1005%:%
%:%2170=1006%:%
%:%2171=1006%:%
%:%2172=1007%:%
%:%2173=1007%:%
%:%2174=1008%:%
%:%2175=1009%:%
%:%2176=1009%:%
%:%2177=1010%:%
%:%2178=1011%:%
%:%2179=1012%:%
%:%2180=1012%:%
%:%2181=1013%:%
%:%2182=1014%:%
%:%2183=1014%:%
%:%2184=1015%:%
%:%2185=1015%:%
%:%2186=1016%:%
%:%2187=1016%:%
%:%2188=1017%:%
%:%2189=1017%:%
%:%2190=1018%:%
%:%2191=1019%:%
%:%2192=1019%:%
%:%2193=1020%:%
%:%2194=1020%:%
%:%2195=1021%:%
%:%2196=1021%:%
%:%2197=1022%:%
%:%2198=1022%:%
%:%2199=1023%:%
%:%2200=1023%:%
%:%2201=1024%:%
%:%2202=1025%:%
%:%2203=1025%:%
%:%2204=1026%:%
%:%2205=1026%:%
%:%2206=1027%:%
%:%2207=1027%:%
%:%2208=1028%:%
%:%2209=1028%:%
%:%2210=1029%:%
%:%2211=1029%:%
%:%2212=1030%:%
%:%2213=1030%:%
%:%2214=1031%:%
%:%2215=1032%:%
%:%2216=1032%:%
%:%2217=1033%:%
%:%2218=1033%:%
%:%2219=1034%:%
%:%2220=1034%:%
%:%2221=1035%:%
%:%2222=1035%:%
%:%2223=1036%:%
%:%2224=1036%:%
%:%2225=1037%:%
%:%2226=1037%:%
%:%2227=1038%:%
%:%2228=1038%:%
%:%2229=1039%:%
%:%2230=1040%:%
%:%2231=1040%:%
%:%2232=1041%:%
%:%2233=1041%:%
%:%2234=1042%:%
%:%2235=1042%:%
%:%2236=1043%:%
%:%2237=1043%:%
%:%2238=1044%:%
%:%2239=1044%:%
%:%2240=1045%:%
%:%2241=1045%:%
%:%2242=1046%:%
%:%2243=1046%:%
%:%2244=1047%:%
%:%2245=1048%:%
%:%2246=1048%:%
%:%2247=1049%:%
%:%2248=1049%:%
%:%2249=1050%:%
%:%2250=1050%:%
%:%2251=1051%:%
%:%2252=1051%:%
%:%2253=1052%:%
%:%2254=1052%:%
%:%2255=1053%:%
%:%2256=1053%:%
%:%2257=1054%:%
%:%2258=1054%:%
%:%2259=1055%:%
%:%2260=1056%:%
%:%2261=1056%:%
%:%2262=1057%:%
%:%2263=1057%:%
%:%2264=1058%:%
%:%2265=1058%:%
%:%2266=1059%:%
%:%2267=1059%:%
%:%2268=1060%:%
%:%2269=1060%:%
%:%2270=1061%:%
%:%2271=1061%:%
%:%2272=1062%:%
%:%2273=1062%:%
%:%2274=1063%:%
%:%2275=1064%:%
%:%2276=1064%:%
%:%2277=1065%:%
%:%2278=1065%:%
%:%2279=1066%:%
%:%2280=1066%:%
%:%2281=1067%:%
%:%2282=1067%:%
%:%2283=1068%:%
%:%2284=1069%:%
%:%2285=1069%:%
%:%2286=1070%:%
%:%2287=1070%:%
%:%2288=1071%:%
%:%2289=1071%:%
%:%2290=1072%:%
%:%2291=1072%:%
%:%2292=1073%:%
%:%2293=1073%:%
%:%2294=1074%:%
%:%2295=1074%:%
%:%2296=1075%:%
%:%2297=1076%:%
%:%2298=1076%:%
%:%2299=1077%:%
%:%2300=1077%:%
%:%2301=1078%:%
%:%2302=1078%:%
%:%2303=1079%:%
%:%2304=1079%:%
%:%2305=1080%:%
%:%2306=1081%:%
%:%2307=1081%:%
%:%2308=1082%:%
%:%2309=1082%:%
%:%2310=1083%:%
%:%2311=1083%:%
%:%2312=1084%:%
%:%2313=1084%:%
%:%2314=1085%:%
%:%2315=1086%:%
%:%2316=1086%:%
%:%2317=1087%:%
%:%2318=1087%:%
%:%2319=1088%:%
%:%2320=1088%:%
%:%2321=1089%:%
%:%2322=1089%:%
%:%2323=1090%:%
%:%2324=1090%:%
%:%2325=1091%:%
%:%2326=1092%:%
%:%2327=1092%:%
%:%2328=1093%:%
%:%2329=1093%:%
%:%2330=1094%:%
%:%2331=1094%:%
%:%2332=1095%:%
%:%2333=1095%:%
%:%2334=1096%:%
%:%2335=1097%:%
%:%2336=1097%:%
%:%2337=1098%:%
%:%2338=1098%:%
%:%2339=1099%:%
%:%2340=1099%:%
%:%2341=1100%:%
%:%2342=1100%:%
%:%2343=1101%:%
%:%2344=1101%:%
%:%2345=1102%:%
%:%2346=1103%:%
%:%2347=1103%:%
%:%2348=1104%:%
%:%2349=1104%:%
%:%2350=1105%:%
%:%2351=1106%:%
%:%2352=1106%:%
%:%2353=1107%:%
%:%2354=1107%:%
%:%2355=1108%:%
%:%2356=1108%:%
%:%2357=1109%:%
%:%2358=1110%:%
%:%2359=1110%:%
%:%2360=1111%:%
%:%2361=1112%:%
%:%2362=1112%:%
%:%2363=1113%:%
%:%2364=1113%:%
%:%2365=1114%:%
%:%2366=1114%:%
%:%2367=1115%:%
%:%2368=1115%:%
%:%2369=1116%:%
%:%2370=1116%:%
%:%2371=1117%:%
%:%2372=1118%:%
%:%2373=1118%:%
%:%2374=1119%:%
%:%2375=1120%:%
%:%2376=1120%:%
%:%2377=1121%:%
%:%2378=1121%:%
%:%2379=1122%:%
%:%2380=1122%:%
%:%2381=1123%:%
%:%2382=1123%:%
%:%2383=1124%:%
%:%2384=1124%:%
%:%2385=1125%:%
%:%2386=1125%:%
%:%2387=1126%:%
%:%2388=1127%:%
%:%2389=1127%:%
%:%2391=1129%:%
%:%2392=1130%:%
%:%2393=1130%:%
%:%2394=1131%:%
%:%2395=1131%:%
%:%2396=1132%:%
%:%2397=1132%:%
%:%2398=1133%:%
%:%2399=1133%:%
%:%2400=1134%:%
%:%2401=1134%:%
%:%2402=1135%:%
%:%2403=1135%:%
%:%2404=1136%:%
%:%2405=1136%:%
%:%2406=1137%:%
%:%2407=1137%:%
%:%2408=1138%:%
%:%2409=1138%:%
%:%2410=1139%:%
%:%2411=1139%:%
%:%2412=1140%:%
%:%2413=1140%:%
%:%2414=1141%:%
%:%2415=1141%:%
%:%2416=1142%:%
%:%2417=1142%:%
%:%2418=1143%:%
%:%2419=1144%:%
%:%2420=1144%:%
%:%2421=1145%:%
%:%2422=1145%:%
%:%2423=1146%:%
%:%2424=1146%:%
%:%2425=1147%:%
%:%2426=1147%:%
%:%2427=1148%:%
%:%2428=1148%:%
%:%2429=1149%:%
%:%2430=1149%:%
%:%2431=1150%:%
%:%2432=1151%:%
%:%2433=1151%:%
%:%2434=1152%:%
%:%2435=1152%:%
%:%2436=1153%:%
%:%2437=1153%:%
%:%2438=1154%:%
%:%2439=1154%:%
%:%2440=1155%:%
%:%2441=1155%:%
%:%2442=1156%:%
%:%2443=1156%:%
%:%2444=1157%:%
%:%2445=1157%:%
%:%2446=1158%:%
%:%2447=1159%:%
%:%2448=1159%:%
%:%2449=1160%:%
%:%2450=1160%:%
%:%2451=1161%:%
%:%2452=1161%:%
%:%2453=1162%:%
%:%2454=1162%:%
%:%2455=1163%:%
%:%2456=1163%:%
%:%2457=1164%:%
%:%2458=1164%:%
%:%2459=1165%:%
%:%2460=1166%:%
%:%2461=1166%:%
%:%2462=1167%:%
%:%2463=1167%:%
%:%2464=1168%:%
%:%2465=1168%:%
%:%2466=1169%:%
%:%2467=1169%:%
%:%2468=1170%:%
%:%2469=1170%:%
%:%2470=1171%:%
%:%2471=1171%:%
%:%2472=1172%:%
%:%2473=1173%:%
%:%2474=1173%:%
%:%2475=1174%:%
%:%2476=1175%:%
%:%2477=1175%:%
%:%2478=1176%:%
%:%2479=1176%:%
%:%2480=1177%:%
%:%2481=1177%:%
%:%2482=1178%:%
%:%2483=1178%:%
%:%2484=1179%:%
%:%2485=1179%:%
%:%2486=1180%:%
%:%2487=1180%:%
%:%2488=1181%:%
%:%2489=1181%:%
%:%2490=1182%:%
%:%2491=1183%:%
%:%2492=1183%:%
%:%2493=1184%:%
%:%2494=1184%:%
%:%2495=1185%:%
%:%2496=1185%:%
%:%2497=1186%:%
%:%2498=1186%:%
%:%2499=1186%:%
%:%2500=1187%:%
%:%2501=1187%:%
%:%2502=1188%:%
%:%2503=1188%:%
%:%2504=1189%:%
%:%2505=1189%:%
%:%2506=1190%:%
%:%2507=1190%:%
%:%2508=1191%:%
%:%2509=1191%:%
%:%2510=1192%:%
%:%2511=1192%:%
%:%2512=1193%:%
%:%2513=1193%:%
%:%2514=1194%:%
%:%2515=1194%:%
%:%2516=1194%:%
%:%2517=1195%:%
%:%2518=1195%:%
%:%2519=1196%:%
%:%2520=1196%:%
%:%2521=1197%:%
%:%2522=1197%:%
%:%2523=1198%:%
%:%2524=1198%:%
%:%2525=1198%:%
%:%2526=1198%:%
%:%2527=1199%:%
%:%2533=1199%:%
%:%2538=1200%:%
%:%2543=1201%:%


\appendix
\section{Informal proof of correctness for the $F_0$ algorithm}
This section contains a detailed informal proof for the correctness of the $F_0$-algorithm.
Because of the standard argument about medians we only want to show that each of the estimates the median is taken from is within the desired interval with success probability $\frac{2}{3}$.

To verify the latter, let $a_1, \ldots, a_m$ be the stream elements, where we assume that the elements are a subset of $\{0,\ldots,n-1\}$ and $0 < \delta < 1$ be the desired relative accuracy.
Let $p$ be the smallest prime such that $p \geq \max (n,19)$ and let $h$ be a random polynomial over $GF(p)$ with degree strictly less than $2$.
The algoritm also introduces the internal parameters $t, r$ defined by:
\begin{eqnarray*}
    t & := & \lceil 80\delta^{-2} \rceil \\
    r & := & 4 \log_2 \lceil \delta^{-1} \rceil + 24
\end{eqnarray*}
Now we can describe the estimate the algorithm obtains:
\begin{align*}
    A := & \left\{ a_1, \ldots, a_m \right\} &
    H := & \left\{ \lfloor h(a) \rfloor_r \middle \vert a \in A \right\} \\
    R := & \begin{cases} t p \left(\mathrm{rank}_t (H) \right)^{-1} & \textrm{ if } \size{H} \geq t \\
    \size{H} & \textrm{ othewise,} \end{cases} &
\end{align*}
We want to show that
\[
    P(\size{R - F_0} \leq \delta \size{F_0}) \geq \frac{2}{3} \textrm{.}  
\]
We show the result by investigating the two cases $F_0 \geq t$ and $F_0 < t$ seperately.
\subsection{Case $F_0 \geq t$}
Let us introduce:
\begin{eqnarray*}
    H^* & := & \left\{ h(a) \middle \vert a \in A \right\}^{\#} \\
    R^* & := & tp \left( \mathrm{rank}^{\#}_t(H^*) \right)^{-1}
\end{eqnarray*}
These definitions correspond to the $H$, $R$ but with a few minor modifications.
We compute $H^*$ as a multiset, this means we are keeping track of the multiplicities of its elements.
Note that by definition: $\size{H^*}=\size{A}$.
Similarly the operation $rank^{\#}_t$ obtains the rank-$t$ element of the multiset (taking multiplicities into account).
We also avoid the rounding operation $\lfloor \cdot \rfloor_r$ in the definition of $H^*$.
The key reason for the introduction of these alternative versions of $H, R$ is that it is easier to show probabilistic bounds on the distances $\size{R^* - F_0}$, 
$\size{R^* - R}$ as opposed to $\size{R - F_0}$ directly.
In particular, what we plan to show is that:
\begin{eqnarray}
 \delta' & := & \frac{3}{4}\delta \\
 P\left(\size{R^*-F_0} > \delta' F_0\right) & \leq & \frac{2}{9} \textrm{, and} \label{eq:r_star_dev} \\
 P\left(\size{R^*-F_0} \leq \delta' F_0 \wedge \size{R-R^*} > \frac{\delta}{4} F_0\right) & \leq & \frac{1}{9} \label{eq:r_star_r}
\end{eqnarray}
I.e. the probability that $R^*$ has not the relative accuracy of $\frac{3}{4}\delta$ is less that $\frac{2}{9}$ and the probability 
that assuming $R^*$ has the relative accuracy of $\frac{3}{4}\delta$ but that $R$ deviates by more that $\frac{1}{4}\delta F_0$ is at most $\frac{1}{9}$.
Hence, the probability that neither of these events happen is at least $\frac{2}{3}$ but in that case:
\begin{equation}
    \label{eq:concl}
    \size{R-F_0} \leq \size{R - R^*} + \size{R^*-F_0} \leq \frac{\delta}{4} F_0 + \frac{3 \delta}{4} F_0 = \delta F_0 \textrm{.}
\end{equation}

For the verification of \autoref{eq:r_star_dev} let us introduce:
\[
    Q(u) = \size{\left\{h(a) < u \mid a \in A \right\}}
\]
then we observe that $\mathrm{rank}_t^{\#}(H^*) < u$ if $Q(u) \geq t$ and $\mathrm{rank}_t^{\#}(H^*) \geq v$ if $Q(v) \leq t-1$.
To see why this is true note that, if at least $t$ elements of $A$ are mapped by $h$ below a certain value, then the rank $t$ element must also be within them, and thus also be below that value.
And that the opposite direction of this conclusion is also true.
Note that this relies on the fact that $H^*$ is a multiset and we are taking multiplicities into account, when computing the rank $t$ element. 

Alternatively, we could also write $Q(u) = \sum_{a \in A} 1_{\{h(a) < u\}}$\footnote{The notation $1_A$ is shorthand for the indicator function of $A$, i.e., $1_A(x)=1$ if $x \in A$ and $0$ otherwise.}, i.e., $Q$ is a sum of pairwise independent $\{0,1\}$-valued random variables, with expectation $\frac{u}{p}$ and variance $\frac{u}{p} - \frac{u^2}{p^2}$.
\footnote{A consequence of $h$ being choosen uniformly from a $2$-independent hash family.}
Using lineariy of expectation and Bienaym\'e's identity, we can conclude that $\var \, Q(u) \leq \expectation \, Q(u) = |A|u p^{-1} = F_0 u p^{-1}$ for $u \in \{0,\ldots,p\}$.

For $v = \left\lfloor \frac{tp}{(1-\delta') F_0} \right\rfloor$ we have
\begin{eqnarray*}
    t-1 & \leq\footnotemark & \frac{t}{(1-\delta')} - 3\sqrt{\frac{t}{(1-\delta')}} - 1 \\
     &\leq&  \frac{F_0 v}{p} - 3 \sqrt{\frac{F_0 v}{p}} \leq \expectation Q(v) - 3 \sqrt{\var Q(v)}
\end{eqnarray*}
\footnotetext{The verification of this inequality is a lengthy but straightforward calculcation using the definition of $\delta'$ and $t$.}
and thus we can conclude using Tchebyshev's inequality:
\begin{align}
    P\left(R^* < \left(1-\delta'\right) F_0\right) & = P\left(\mathrm{rank}_t^{\#}(H^*) > \frac{tp}{(1-\delta')F_0}\right) \nonumber \\ 
    & \leq P(\mathrm{rank}_t^{\#}(H^*) \geq v) = P(Q(v) \leq t-1) \label{eq:r_star_upper_bound} \\
    & \leq P\left(Q(v) \leq \expectation Q(v) - 3 \sqrt{\var Q(v)}\right) \leq \frac{1}{9} \textrm{.} \nonumber
\end{align}
Similarly for $u = \left\lceil \frac{tp}{(1+\delta') F_0} \right\rceil$ we have
\begin{eqnarray*}
    t & \geq & \frac{t}{(1+\delta')} + 3\sqrt{\frac{t}{(1+\delta')}+1} + 1 \\
     &\geq&  \frac{F_0 u}{p} + 3 \sqrt{\frac{F_0 u}{p}} \geq \expectation Q(u) + 3 \sqrt{\var Q(v)}
\end{eqnarray*}
and thus we can conclude using Tchebyshev's inequality:
\begin{align}
    P\left(R^* > \left(1+\delta'\right) F_0\right) & = P\left(\mathrm{rank}_t^{\#}(H^*) < \frac{tp}{(1+\delta')F_0}\right) \nonumber \\ 
    & \leq P(\mathrm{rank}_t^{\#}(H^*) < u) = P(Q(u) \geq t) \label{eq:r_star_lower_bound} \\
    & \leq P\left(Q(u) \geq \expectation Q(u) + 3 \sqrt{\var Q(u)}\right) \leq \frac{1}{9} \textrm{.} \nonumber
\end{align}
To verfiy \autoref{eq:r_star_r} we note that
\begin{equation}
    \label{eq:rank_eq}
    \mathrm{rank}_t(H) = \lfloor \mathrm{rank}_t^{\#}(H^*) \rfloor_r
\end{equation}
if there are no collisions, induced by the application of $\lfloor h(\cdot) \rfloor_r$ on the elements of $A$.
If we are even more careful, we note that the equation would remain true, as long as there are no collision within the smallest $t$ elements of $H^*$.
Because we need to show \autoref{eq:r_star_r} only in the case where $R^* \geq (1-\delta') F_0$, i.e., when $\mathrm{rank}_t^{\#}(H^*) \leq v$,
it is enough to bound the probability of a collision in the range $[0; v]$.
Moreover \autoref{eq:rank_eq} implies $\size{\mathrm{rank}_t(H) - \mathrm{rank}_t^{\#}(H^*)} \leq \max(\mathrm{rank}_t^{\#}(H^*), \mathrm{rank}_t(H)) 2^{-r}$ from
which it is possible to derive $\size{R^*-R} \leq \frac{\delta}{4} F_0$.
% R* = tp/rank_t#, R = tp/rank_t => |R-R*| = | tp [ rank_t - rank_t# / rank_t rank_t# ] | <= 
% tp \max{rank_t, rank_t#} 2^-r / rank_t rank_t# <= t 2^-r <= t 2^-24 d'^4 <= F_0 d'^4 2^-24 <= 1/4 d' F_0
%
%Let's summarize: If
%\begin{eqnarray*}
%    R^* & \geq & (1-\delta') F_0 \\
%    \lfloor h(a) \rfloor_r & \neq & \lfloor h(b) \rfloor_r \textrm{ for } a \neq b \in A \wedge h(a) \leq v \wedge h(b) \leq v
%\end{eqnarray*}
%then $\mathrm{rank}_t(H) = \lfloor \mathrm{rank}_t^{\#}(H^*) \rfloor_r$.
Another observation we want to make is that $h$ is injective with probability $1-\frac{1}{p}$, this is because $h$ is choosen uniformly from the polynomials of degree less than $2$.
If it is a degree $1$ polynomial, it is a linear function on $GF(p)$ and thus injective.
Because we have choosen $p \geq 18$, we can bound the probability that $h$ is not injective by $1/18$.
However, even if $h$ is injective, there is still a possibility of collision, because of the application of the rounding operation $\lfloor \cdot \rfloor_r$. 
The plan is to bound that probability by $1/18$ as well to be able to show \autoref{eq:r_star_r}.

\begin{eqnarray*}
    P\left( \size{R^*-F_0} \leq \delta' F_0 \wedge \size{R-R^*} > \frac{\delta}{4} F_0 \right) & \leq & \\
    P\left( R^* \geq (1-\delta') F_0 \wedge \mathrm{rank}_t^{\#}(H^*) \neq \mathrm{rank}_t(H) \wedge h \textrm{ injective}\right) + P(\neg h \textrm{ injective}) & \leq & \\
    P\left( \exists a \neq b \in A. \lfloor h(a) \rfloor_r = \lfloor h(b) \rfloor_r \leq v \wedge h(a) \neq h(b) \right) + \frac{1}{18} & \leq & \\
    \frac{1}{18} + \sum_{a \neq b \in A} P\left(\lfloor h(a) \rfloor_r = \lfloor h(b) \rfloor_r \leq v \wedge h(a) \neq h(b) \right) & \leq & \\
    \frac{1}{18} + \sum_{a \neq b \in A} P\left(\size{h(a) - h(b)} \leq v 2^{-r} \wedge h(a) \leq v (1+2^{-r}) \wedge h(a) \neq h(b) \right) & \leq & \\
    \frac{1}{18} + \sum_{a \neq b \in A} \sum_{\substack{a', b' \in \{0,\ldots, p-1\} \wedge a' \neq b' \\ \size{a'-b'} \leq v 2^{-r} \wedge a' \leq v (1+2^{-r})}} P(h(a) = a') P(h(b)= b') & \leq & \\
    \frac{1}{18} + 6 \frac{F_0^2 v^2}{p^2} 2^{-r} & \leq & \frac{1}{9} \textrm{.}
%    96 t^2 2^{-r} + \frac{1}{18} & \leq & \frac{1}{9}
\end{eqnarray*}
Which shows that \autoref{eq:r_star_r} is true and using \autoref{eq:r_star_upper_bound} and~\ref{eq:r_star_lower_bound} we can verify 
\autoref{eq:r_star_dev}, which means with the reasoning in \autoref{eq:concl} we can confirm:
\begin{equation}
    P(\size{R - F_0} \leq \delta \size{F_0}) \geq \frac{2}{3}
\end{equation}

In the following subsection, we will confirm that this is also true for the remaining case, if $F_0 < t$, concluding the proof.
\subsection{Case $F_0 < t$}
Note that in this case $\size{H} \leq F_0 < t$ and thus $R = \size{H}$. We want to show that
$P(\size{H} \neq F_0) \leq \frac{1}{3}$.

The latter can only happen, if there is a collision induced by the application of $\lfloor h(\cdot)\rfloor_r$. As before, we rely on the fact that $h$ is not injective with probability at $\frac{1}{18}$.
\begin{eqnarray*}
    P\left( \size{R - F_0} > \delta F_0\right) & \leq & \\
    P\left( R \neq F_0 \right) & \leq & \\
    \frac{1}{18} + P\left( R \neq F_0 \wedge h \textrm{ injective} \right) & \leq & \\
    \frac{1}{18} + P\left( \exists a \neq b \in A. \lfloor h(a) \rfloor_r = \lfloor h(b) \rfloor_r  \right) & \leq & \\
    \frac{1}{18} + \sum_{a \neq b \in A} P\left(\lfloor h(a) \rfloor_r = \lfloor h(b) \rfloor_r \wedge h(a) \neq h(b) \right) & \leq & \\
    \frac{1}{18} + \sum_{a \neq b \in A} P\left(\size{h(a) - h(b)} \leq p 2^{-r} \wedge h(a) \neq h(b) \right) & \leq & \\
    \frac{1}{18} + \sum_{a \neq b \in A} \sum_{\substack{a', b' \in \{0,\ldots, p-1\} \\  a' \neq b' \wedge \size{a'-b'} \leq p 2^{-r}}} P(h(a) = a') P(h(b)= b') & \leq & \\
    \frac{1}{18} + F_0^2 2^{-r+1} \leq \frac{1}{9} \textrm{.}
\end{eqnarray*}
Which concludes the proof. \qed

% optional bibliography
%\bibliographystyle{abbrv}
%\bibliography{root}

\end{document}

%%% Local Variables:
%%% mode: latex
%%% TeX-master: t
%%% End:
